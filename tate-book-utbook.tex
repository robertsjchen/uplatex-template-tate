\RequirePackage{multicol} %多欄
\RequirePackage{type1cm} %字體
\RequirePackage[uplatex,deluxe,jis2004]{otf} %字體包
\RequirePackage{plext} % 為pLaTeX打補丁

\documentclass[uplatex,10pt,openleft,b5paper]{utbook}

%% フォント関連
\usepackage[T1]{fontenc} % フォントでT1を使うこと
\usepackage{textcomp} % フォントでTS1を使うこと
\usepackage[utf8]{inputenc} % ファイルがUTF8であること


%% 図表など
% 図の読みこみのために
\usepackage[dvipdfmx, hiresbb]{graphicx, xcolor}
\usepackage{booktabs} % 表の横罫線

%% 囲み枠
\usepackage{tcolorbox}
\tcbuselibrary{breakable} % ページをまたいで分割できるように

%% misc
\usepackage{okumacro} % 圏点などのために
\usepackage{pxrubrica} % ルビをつける(okumacroのrubyは使わない)

%% 目次の設定
\setcounter{tocdepth}{1} % sectionまでを目次に

\makeatletter
\renewcommand{\thechapter}{\@Kanji{\@arabic\c@chapter}}
 \def\@textbottom{\vskip \z@ \@plus 1pt}
 \let\@texttop\relax
%自定義的字號
\newcommand{\bthuge}{\@setfontsize\bthuge{60}{72}}
\newcommand{\btlarge}{\@setfontsize\btlarge{48}{60}}
\newcommand{\tlarge}{\@setfontsize\tlarge{36}{48}}
\newcommand{\ularge}{\@setfontsize\ularge{30}{48}}
%% 定義正文字號
\renewcommand{\normalsize}{% formerly \normalsize=12pt@18pt
    \@setfontsize\normalsize{14.4pt}{20}%
  \abovedisplayskip 6\p@ \@plus2\p@ \@minus5\p@
  \abovedisplayshortskip \z@ \@plus3\p@
  \belowdisplayshortskip 6\p@ \@plus3\p@ \@minus3\p@
   \belowdisplayskip \abovedisplayskip
   \let\@listi\@listI}

\renewcommand*{\l@chapter}[2]{%
  \ifnum \c@tocdepth >\m@ne
    \addpenalty{-\@highpenalty}%
    \addvspace{1.0em \@plus\p@}%
    \begingroup
      \parindent\z@ \rightskip\@pnumwidth \parfillskip-\rightskip
      \leavevmode\bfseries
      \setlength\@lnumwidth{6zw}%
      \advance\leftskip\@lnumwidth \hskip-\leftskip
      #1\nobreak\hfil\nobreak\hb@xt@\@pnumwidth{\hss#2}\par
      \penalty\@highpenalty
    \endgroup
  \fi}

\renewcommand{\postchaptername}{回}
\makeatother

\usepackage{silence}
\WarningFilter{hyperref}{Size substitutions}
\WarningFilter{latex}{Token not allowed}
\WarningsOff*

% 页面调整
%\setlength{\voffset}{-3 mm}
\setlength{\hoffset}{0 mm}
\setlength{\oddsidemargin}{-10 mm}  %奇数修正數據
\setlength{\evensidemargin}{-3 mm}   %偶数修正數據
\setlength{\textwidth}{205mm}
\setlength{\textheight}{144 mm}
\setlength{\columnsep}{8 mm}

\title{倚天屠龍記}
\author{金\hskip1zw庸}
\title{}
\date{}

%%%%%%%%%%%%%%%%%%%%%%%%%%%%% ubzjlreq-h/v %%%%%%%%%%%%%%%%%%%%%%%%%%%%%%%%%
\DeclareFontFamily{JY2}{ubzjlreq}{}
\DeclareFontFamily{JT2}{ubzjlreq}{}

\DeclareFontShape{JY2}{ubzjlreq}{m}{n}{<->s*[0.924690]ubzjlreq}{}
\DeclareFontShape{JY2}{ubzjlreq}{m}{it}{<->ssub*ubzjlreq/m/n}{}
\DeclareFontShape{JY2}{ubzjlreq}{m}{sl}{<->ssub*ubzjlreq/m/n}{}
\DeclareFontShape{JY2}{ubzjlreq}{m}{sc}{<->ssub*ubzjlreq/m/n}{}

\DeclareFontShape{JT2}{ubzjlreq}{m}{n}{<->s*[0.924690]ubzjlreq-v}{}
\DeclareFontShape{JT2}{ubzjlreq}{m}{it}{<->ssub*ubzjlreq/m/n}{}
\DeclareFontShape{JT2}{ubzjlreq}{m}{sl}{<->ssub*ubzjlreq/m/n}{}
\DeclareFontShape{JT2}{ubzjlreq}{m}{sc}{<->ssub*ubzjlreq/m/n}{}

\DeclareRobustCommand\ubzjlreq{\kanjifamily{ubzjlreq}\selectfont}
%%%%%%%%%%%%%%%%%%%%%%%%%%%%% ubzjlreq-h/v %%%%%%%%%%%%%%%%%%%%%%%%%%%%%%%%%
%%%%%%%%%%%%%%%%%%%%%%%%%%%%% ubzjlreqg-h/v %%%%%%%%%%%%%%%%%%%%%%%%%%%%%%%%%
\DeclareFontFamily{JY2}{ubzjlreqg}{}
\DeclareFontFamily{JT2}{ubzjlreqg}{}

\DeclareFontShape{JY2}{ubzjlreqg}{m}{n}{<->s*[0.924690]ubzjlreqg}{}
\DeclareFontShape{JY2}{ubzjlreqg}{m}{it}{<->ssub*ubzjlreqg/m/n}{}
\DeclareFontShape{JY2}{ubzjlreqg}{m}{sl}{<->ssub*ubzjlreqg/m/n}{}
\DeclareFontShape{JY2}{ubzjlreqg}{m}{sc}{<->ssub*ubzjlreqg/m/n}{}

\DeclareFontShape{JT2}{ubzjlreqg}{m}{n}{<->s*[0.924690]ubzjlreqg-v}{}
\DeclareFontShape{JT2}{ubzjlreqg}{m}{it}{<->ssub*ubzjlreqg/m/n}{}
\DeclareFontShape{JT2}{ubzjlreqg}{m}{sl}{<->ssub*ubzjlreqg/m/n}{}
\DeclareFontShape{JT2}{ubzjlreqg}{m}{sc}{<->ssub*ubzjlreqg/m/n}{}

\DeclareRobustCommand\ubzjlreqg{\kanjifamily{ubzjlreqg}\selectfont}
%%%%%%%%%%%%%%%%%%%%%%%%%%%%% ubzjlreqg-h/v %%%%%%%%%%%%%%%%%%%%%%%%%%%%%%%%%
% 簡體中文 adobe 細宋體 & adobe 黑體
\DeclareFontFamily{JY2}{upstsl}{}
\DeclareFontFamily{JT2}{upstsl}{}

\DeclareFontShape{JY2}{upstsl}{m}{n}{<->s*[0.924690]upstsl-h}{}
\DeclareFontShape{JY2}{upstsl}{m}{it}{<->ssub*upstsl/m/n}{}
\DeclareFontShape{JY2}{upstsl}{m}{sl}{<->ssub*upstsl/m/n}{}
\DeclareFontShape{JY2}{upstsl}{m}{sc}{<->ssub*upstsl/m/n}{}

\DeclareFontShape{JT2}{upstsl}{m}{n}{<->s*[0.924690]upstsl-v}{}
\DeclareFontShape{JT2}{upstsl}{m}{it}{<->ssub*upstsl/m/n}{}
\DeclareFontShape{JT2}{upstsl}{m}{sl}{<->ssub*upstsl/m/n}{}
\DeclareFontShape{JT2}{upstsl}{m}{sc}{<->ssub*upstsl/m/n}{}

\DeclareRobustCommand\upstsl{\kanjifamily{upstsl}\selectfont}

\AtBeginDvi{%
  \special{pdf:mapline upstsl-h	unicode		SourceHanSerifK-Medium.otf}
  \special{pdf:mapline upstsl-v	unicode		SourceHanSerifK-Medium.otf	-w	1}
}

\newcommand{\footnotefon}{\ubzjlreq}
\def\dash{{\leavevmode\kern1mm\raise0.1zh\hbox{-----}\kern1mm}}
\def\qyh{\leavevmode\kern -0.5zw\hbox{「}}
\prebreakpenalty`︰=10000
\kcatcode"FE30=18

%%%% PDF 作者信息及超鏈接設定
\usepackage[dvipdfmx,
    pdfdirection=R2L, % 開く方向%從右往左翻頁 當使用橫板系統時,關閉此選項
    colorlinks=true,   %設置超鏈接的顔色
    linkcolor=black,
    filecolor=black,
    urlcolor=black,
    citecolor=black,
    bookmarks=true, % PDFにしおりをつける
    bookmarksnumbered=true, % しおりに節番号などをつける
    bookmarksopen=true, %展開所有層級
  ]{hyperref}
\hypersetup{ %
    pdftitle= {倚天屠龍記} , % PDFのタイトル
    pdfauthor= {子 康} , % PDFの作成者
    pdfkeywords = {upTeX,upLaTeX,Unicode,CJK},  %関鍵詞
    pdfsubject = {Chinese Novel },    % 主題
    pdfcreator  = {up\LaTeX\ with package  hyperref },    %工具
%    pdfproducer = {dvipdfmx(20180506)},   %製作軟件
}

% PDFにしたときのしおりの文字化けを防ぐ  % 使書箋支持Unicode-CJK 内碼
\usepackage{pxjahyper}
% hyperref を使っているときに
% 目次でのページ番号の向きを適切にする
\makeatletter
\def\contentsline#1#2#3#4{\csname l@#1\endcsname{\hyper@linkstart{link}{#4}{#2}\hyper@linkend}{\rensuji{#3}}}
\makeatother

\begin{document}

%% 封面
\vspace*{160pt}
\thispagestyle{empty}
	{\btlarge \noindent\hskip2zw\mcfamily\bfseries%
	\hbox{倚天屠龍記\CID{119}連載版}}\\%正標題

\vspace*{90pt}
	{\tlarge \mcfamily \hfill 金\hskip1zw庸  \\} %% 作者
	\vskip 1.5em%
\cleardoublepage

%% 目录
\setlength{\columnsep}{20 mm}
\setlength{\parskip}{0 mm}
\pagestyle{plain}
\begin{multicols}{2}
\tableofcontents
\end{multicols}
\cleardoublepage

% 正文
%\yoko\adjustbaseline
\setlength{\parindent}{2.88\Cwd}
\setlength{\columnsep}{8 mm}
\pagestyle{headings}
\begin{multicols}{2}
\setcounter{page}{1}
\ubzjlreq
% main text of yt

\chapter*{序曲}
\addcontentsline{toc}{chapter}{序曲}

\begin{quotation}
\par{}春遊浩蕩\hskip8pt是年年寒食\hskip8pt梨花時節\hskip8pt白錦無紋香爛漫\hskip8pt玉樹瓊苞堆雪\hskip8pt靜夜沉沉\hskip8pt浮光靄靄\hskip8pt冷浸溶溶月\hskip8pt人間天上\hskip8pt爛銀霞照通徹\hskip8pt渾似姑射眞人\hskip8pt天姿靈秀\hskip8pt意氣殊高潔\hskip8pt萬蕊參差誰信道\hskip8pt不與群芳同列\hskip8pt浩氣清爽\hskip8pt仙才卓犖\hskip8pt下土難分别\hskip8pt瑤天歸去\hskip8pt洞天方看清絶
\end{quotation}

看官,作這一首〈無俗念〉詞的,乃是南宋末年一位武學名家,有道之士。此人姓丘,名處機,道號長春子,名列全眞七子之一,是全眞教中出類拔萃的人物。〈詞品〉評論此詞道︰「長春,世之所謂仙人也,而詞之清拔如此。」這首詞誦的似是梨花,其實詞中眞意却是讚譽一位身穿白衣的美貌少女,説她「渾似姑射眞人,天姿靈秀,意氣殊高潔」,又説她「浩氣清英,仙才卓犖」「不與群芳同列」。詞中所誦這美女是誰?乃是古墓派傳人小龍女。她一生愛穿白衣,當眞如玉樹臨風,瓊苞堆雪,兼之生性清冷,實是當得起「冷浸溶溶月」的形容,以「無俗念」三字贈之,可説最妙不過。長春子丘處機和她在終南山上比鄰而居,當年一見,便冩下這首詞來。

這時丘處機逝世已久,而小龍女也已嫁與神鵰大俠楊過爲妻。可是在河南少室山的山道之上,却另有一位少女,正在低低念誦此詞。這少女約有十八九歳年紀,身穿淡黃衣衫,騎著一頭瘦瘦的青驢,正沿著山道緩緩而上。她心中默想︰「也只有龍姊姊這樣的人物,纔配得上他。」這一個「他」字,指的自然是神鵰大俠楊過了。她也不拉韁,任著那青驢信步而行,一路上山。過了良久,她又低聲吟道︰「歡樂趣,離别苦,就中更有痴児女。君應有語,渺萬里層雲,千山暮雪,隻影向誰去?」

這少女衣飾淡雅,腰間懸著一把短劍,臉上頗有風塵之色,顯是遠遊已久,她正當詔華如花,喜樂之年,原該無憂無慮,可是容色間却隱隱有一層懊悶之意,可見閒愁襲人,眉間心上,實是無計迴避。

這少女姓郭,單名一個襄字,乃是大俠郭靖和女俠黃蓉的次女,有一個外號叫作「小東邪」。她一驢一劍,隻身漫遊,原是想排遣心中愁悶,豈知酒入愁腸固然是愁上加愁,而名山獨遊,一樣的也是愁思徒增。這時她正沿著河南少室山的山道而上,山勢頗爲陡削,但山道却是一級級寬大的石級,規模宏偉,工程甚是不小。這山道是唐時高宗爲臨幸少林寺而開鑿,共長八里。郭襄騎著青驢逶迤四迴,只見對面山上五道瀑布飛濺而下,高與雲平,再俯視群山,已如蟻蛭。又轉過一彎,遙遙望見黃牆碧瓦,好大一座寺院。

郭襄望著那連綿的屋宇,出了一會神,心想︰「少林寺向來是天下武學之源,但是華山兩次論劍,怎地五絶之中並無少林寺的高僧?難道寺中的和尚自忖没有把握,生怕墮了威名,索性便不去與會?又難道衆僧侶修爲精湛,名心盡去,武功雖高,却不去和旁人爭強賭勝?」她下了青驢,緩步走到寺前,只見樹木森森,蔭蓋著一片碑林。這些石碑大半已經毀破,字跡糢糊,不知冩著些什麼。郭襄心想︰「便是刻鑿在石碑上的字,年深月久之後也須磨滅,如何刻在我心上的字,却是時間越久反而越加清晰?」一瞥眼間,只見一塊大碑上刻的是唐太宗賜少林寺寺僧的御釗,釗中對少林寺僧的立功平亂,頗爲嘉許。

原來唐太宗爲秦王時,帶兵討伐王世充,少林寺的和尚投軍立功,最著者共有十三人。十三人中只有曇宗受封爲大將軍,其餘十二人不願爲官,唐太宗各賜紫羅袈裟一襲。郭襄神馳想像,心道︰「當隋唐之際,少林寺的武功已名馳天下,數百年精益求精,這寺中臥虎藏龍,不知住著多少好手。」

便在此時,忽聽得碑林旁的樹叢之後,傳出一陣鐵鍊{\upstsl{啷}}{\upstsl{噹}}之聲,又聽後一人誦唸佛經道︰

\qyh{}是時藥叉共王立要,即於無量百千萬億大衆之中,説勝妙伽他日︰

\begin{quotation}
由愛故生憂\hskip8pt由愛故生怖

若離於愛者\hskip8pt無憂亦無怖\dash{}」
\end{quotation}

郭襄聽了這四句偈言,心念一動,不由得痴了,心中默默唸道︰「由愛生憂,由愛故生怖;若離於愛者,無憂亦無怖。」只聽得那鐵鍊拖地和念佛之聲,漸漸遠去。

郭襄低聲道︰「我却要問他一問,如何能離於愛,如何能無憂無怖?」隨手將青驢的韁繩在樹上一繞,撥開樹叢,追了過去。只見樹後是一條上山的小徑,一個僧人挑了一對大桶,口中念佛,緩緩往山上走去。郭襄快步跟上,到和那僧人相距十餘丈處,不由得吃了一驚,只見那僧人挑的是一對大鐵桶,每隻鐵桶都比平常的水桶大了三倍有餘,而那僧人頸中、手上、脚上,更是繞滿了粗大的鐵鍊,行走時鐵鍊拖地,不住發出聲響。這對大鐵桶本身便有數百斤,桶中裝滿了水,重量更是驚人。郭襄叫道︰「大和尚,請留一步,小女子有一言請教。」

那僧人回過頭來,兩人相對,都是一愕。原來這僧人便是覺遠,三年以前,郭襄在華山絶頂曾和他有一面之緣。郭襄知他雖然生性迂腐,但内功深湛,不在當世任何一位最強的高手之下,當下説道︰「我道是誰,原來覺遠大師。你如何變成了這等模樣?」覺遠點了點頭,臉上微微一笑,雙手合什行禮,並不答話,轉身便走。郭襄叫道︰「覺遠大師,你不認得我了麼?我是郭襄啊。」覺遠又是回首一笑,點了點頭,這次更不停步。郭襄又道︰「是誰用鐵鍊綁住了你?如何這般虐待你?」覺遠左掌伸到腦後搖了幾搖,示意她不必再問。

郭襄好奇心起,見了這等怪事,如何肯不弄個明白?當下飛步追趕,想搶在他面前攔住,豈知覺遠雖然全身帶了鐵鍊,又挑著一對大鐵桶,不論郭襄如何快步追趕,始終追不到他身前。郭襄童心大起,施展開家傳輕功,雙足一點,身子便如燕子般飛起,伸手往鐵桶邉上抓去。眼見這一抓必能抓中,不料落手之時,終究還是差了兩寸。郭襄叫道︰「大和尚,這般好本事,我非追上你不可。」但見覺遠不疾不徐的邁步而行,鐵鍊聲{\upstsl{噹}}{\upstsl{啷}}{\upstsl{噹}}{\upstsl{啷}},有如樂音,越走越高,直至後山,郭襄直奔得氣喘漸急,仍是和他身子相距丈餘,不由得心中佩服︰「爹爹媽媽在華山之上,便説這位大和尚武功極高,當時我還不大相信,今日這麼一試,這纔知爹媽的話果然不錯。」

只見覺遠轉身走到一間小屋之後,將鐵桶中的兩桶水都倒進了一口井中。郭襄大奇,説道︰「大和尚,你莫非瘋了,挑水倒在井中幹麼?」覺遠神色平和,只搖了搖頭。郭襄忽有所悟,笑道︰「啊,你是在練一種高深的武功。」覺遠搖了搖頭。郭襄心中著惱,道︰「我剛才明明聽得你在吟經,又不是啞了,怎地不答我的話?」覺遠合什行禮,臉上似有歉意,可是仍舊一言不發,挑了兩隻鐵桶,便下山去。郭襄探頭到井口一望,只見井水清澈,隱隱冒上來一股寒氣,也無什麼特異之處,怔怔的望著覺遠的背影,心頭充滿了疑雲。

她適纔這一陣追趕,微感心浮氣躁,於是坐在井欄之上,觀看四下裡的風景,這時置身之處,已高於少林寺中所有的屋宇,但見少室山層崖刺天,橫若列屏,崖下風煙飄渺,只聽得寺中鐘聲自下面隨風送了上來,令人一洗煩俗之氣。郭襄心想︰「這和尚的弟子不知在那裡,和尚既不肯説,我問那個少年便了。」當下信步落山,想找覺遠的弟子張君寶來一問。走了一程,忽聽得鐵鍊聲響,覺遠又挑了水走上山來。郭襄閃身在樹後一躱,心想︰「他明著不肯説,我暗中瞧瞧他到底在搗什麼鬼?」

但聽得鐵鍊之聲漸近,只見覺遠肩頭仍是挑著那一對鐵桶,手中却拿著一本書,一面走,一面看得津津有味。郭襄待他走到身邉,猛地裡一躍而出,叫道︰「大和尚,你看甚麼書?」覺遠失聲叫道︰「啊喲,嚇了我一跳,原來是你。」郭襄笑道︰「你裝啞巴裝不成了吧,怎麼説話了?」覺遠微有驚色,向左右一望,搖了搖手。郭襄道︰「你怕什麼?」覺遠還未回答,突然樹林中轉出兩個黃衣僧人,當先一人喝道︰「覺遠,不守戒法,擅自開口説話,何況又和廟外生人對答,更何況又和年輕女子説話?這便見戒律堂首座去。」覺遠垂頭喪氣,點了點頭,跟在那兩個僧人之後。

郭襄大爲驚怒,喝道︰「天下還有不許人説話的規矩麼?我自識得這位大師,我自跟他説話,干你們何事?」那身材較高的黃衣僧人白眼一翻,説道︰「千年以來,少林寺向不許女流之輩擅入。姑娘請下山去吧,免得自討没趣。」郭襄心中更怒,説道︰「女流之輩便怎樣?難道女子便不及男子了?你們爲何難爲這位覺遠大師?既用鐵鍊綑綁他,又不許他説話?」那僧人冷冷的道︰「本寺之事,便是皇帝也管不著,何勞姑娘多問。」郭襄怒道︰「我知道這位大師是個忠厚老實的好人,你們欺他仁善,便這般折磨於他,哼哼,天鳴禪師呢?無色和尚、無相和尚在那裡?你去叫他們出來,我倒要問問這個道理。」

那兩個僧人聽了,心中都是一驚,原來天鳴禪師是少林寺的方丈,無色禪師是本寺羅漢堂首座,無相禪師是達摩堂首座,三人在寺中,位望尊崇,寺中僧侶向來只稱「老方丈」「羅漢堂座師」「達摩堂座師」而不敢直呼法名,豈知一個年輕女子竟敢上山來大呼小叫,直斥其名。那瘦長的僧人法名弘明,是戒律堂首座的大弟子,奉了座主之命,和師弟弘緣一同監視覺遠,這時聽郭襄言語莽撞,喝道︰「女施主再在佛門清淨之地滋擾,可恕小佛無禮了。」郭襄道︰「難道我還怕了你這和尚?你快快把覺遠大師身上的鐵鍊除去,那便算了,否則我找天鳴老和尚算帳去。」

原來郭襄自和楊過、小龍女夫婦在華山絶頂分手後,三年來没得到他二人半點音訊。她心中長自記掛,於是稟明父母,説要出來遊山玩水,實則却是到處打聽楊過的消息。她倒也不一定要和他夫婦會面,只須聽到一些楊過如何在江湖上行俠的訊息,心中也便滿足了。偏生一别之後,他夫婦倆從此便不在江湖上露面,不知到了何處隱居,郭襄自北而南,又從東至西,幾乎踏遍了大半個中原,始終没聽到有人説起神鵰大俠楊過六字。這一日她到了河南,想起楊過昔日曾説和少林寺的方丈等人相識,心想説不定那方丈會知道他的蹤跡,這纔上少林寺來。不料未進山門,先碰到覺遠這件怪事。

弘明、弘緣兩人見郭襄腰懸短劍,心下更是惱怒。弘緣沉著嗓子道︰「你把腰間兵刃留下,咱們也不跟你一般見識,快快下山去吧。」

郭襄聽弘緣竟要她將兵刃留下,怒氣更增,從腰間解下短劍,雙手托起,冷笑道︰「好吧,謹遵台命。」弘緣自幼在少林寺出家,十多年來總是聽師伯、師叔、師兄們説少林寺是天下武學的總源,又聽説不論是名望多大、本領多強的武林高手,從不敢擕帶兵刃走進少林寺的山門。此刻郭襄雖然未入寺門,但已是在少林寺的範圍之内。弘緣眼見她只是個年輕姑娘,那將她放在心上,只道她眞是怕了自己,乘乖地交出短劍,於是袍袖一拂,罩住自己雙手,便去按郭襄的短劍。

他手指剛碰到劍鞘,突然間手臂一震,如中電掣,但覺一股強力從短劍上傳了過來,推著他向後一仰,立足不定,登時摔倒。他身在斜坡之上,一經摔倒,便骨碌碌的向下滾了十餘丈,好容易抓住小路旁的一棵小樹,這纔不再滾動。弘明又驚又怒,喝道︰「你吃了獅子心豹子膽,竟到少林寺撒野來啦!」轉過身來,踏上一步,右手一拳擊出,左掌跟著在右拳背上一搭,雙掌下劈,正是「闖少林」第二十八勢「翻身劈擊」。郭襄見他出拳有風,武功顯是比弘緣強了許多,於是握住劍柄,連劍帶鞘向他肩頭{\upstsl{砸}}了下去。弘明沉肩迴掌,來抓劍鞘。覺遠在旁瞧得惶急,大叫︰「别動手,别動手!有話好説。」便在此時,弘明一把已抓住劍鞘,正欲運勁裡奪,猛覺手心一震,雙臂隱隱酸麻,只叫得一聲︰「不好!」郭襄一腿橫掃,將他踢了下去。弘明所受的這一招却比弘緣重得多,一直滾了二十餘丈,頭臉上擦出不少鮮血,這纔停住。

郭襄心道︰「我上少林寺來是打聽大哥哥的訊息,平白無端的跟他們動手,當眞好没來由。」一瞥眼,只見覺遠愁眉苦臉的站在一旁,當即抽出短劍,便往他手脚上的鐵鍊削去。她這短劍雖不是稀世奇珍,却也是極鋒鋭的利器,只聽{\upstsl{噹}}{\upstsl{啷}}{\upstsl{啷}}幾聲響,鐵鍊斷了三條。覺遠連呼︰「使不得,使不得!」郭襄道︰「什麼使不得?」指著正向寺内奔去的弘明弘緣二人説道︰「這兩個惡和尚定是去報訊,咱們快走。你那個姓張的小和尚呢?帶了他一起走吧!」覺遠只是搖手,忽聽得身後一人説道︰「多謝姑娘関懷,小的在這児。」

郭襄回過頭來,只見身後站著一個十六七歳的少年,粗眉大眼,身材魁偉,臉上却猶帶稚氣,正是三年前曾在華山之巓會過的張君寶。比之當日,他身形已高了許多,但容貌却無甚改變。郭襄大喜,説道︰「這裡的惡和尚欺侮你師父,咱們遠遠的走了吧。」張君寶搖頭道︰「没有誰欺侮我師父啊。」郭襄指著覺遠道︰「那兩個惡和尚用鐵鍊綁著你師父,連一句話也不許他説,這還不是欺侮?」覺遠苦笑搖頭,指了指山下,示意郭襄及早脱身,免惹事端。

這小東邪郭襄却是天生的俠義心腸,她明知少林寺武功勝過她的人眞是車載斗量,不計其數,但既看見了眼前的不平之事,決不能便此撒手不顧,可是一面却又擔心寺中好手出手截攔,當下一手拉了覺遠,一手拉了張君寶,頓足道︰「快走快走,有什麼事下山去慢慢説不好麼?」

一言甫畢,忽見山坡下黃牆的邉門中湧出七八個僧人,手中都提著齊眉木棍,{\upstsl{吆}}喝道︰「那裡來的野婆娘,膽敢至少林寺中撒野?」張君寶提起嗓子道︰「各位師父不得無禮,這位是\dash{}」郭襄忙道︰「别説我名字。」她想今日禍事看來闖得不小,説不定鬧下去會不可收拾,一人做事一人當,别牽累到了爹爹媽媽,於是又補上一句︰「咱們翻山走吧!千萬别提我爹爹媽媽和朋友的姓名。」忽聽得背後山頂上{\upstsl{吆}}喝聲響,又湧出七八個僧人。

郭襄見前後都出現了僧人,秀眉深蹙,急道︰「你們這兩個人婆婆媽媽!没點男子漢氣槩,到底走是不走?」張君寶道︰「師父,郭姑娘是一片好意\dash{}」便在此時,下面邉門中又竄出四個黃衣僧人,颼颼颼的奔上坡去,手中雖都没持兵器,但身法迅捷,衣襟帶風,武功大是不凡。郭襄見這般情勢,便是想單獨脱身亦已不能,索性凝氣卓立,靜觀待變。當先一個僧人奔到離郭襄四丈之處朗聲説道︰「羅漢堂首座師尊傳諭,著來人放下兵刃,在山下立雪亭中陳明詳情,聽由法諭。」郭襄冷笑道︰「少林寺的大和尚們官派十足,官腔打得倒好聽,請問各位大和尚,做的是大宋皇帝的官児呢,還是做的蒙古皇帝的官?」

這時淮水以北,大宋的國土均已淪陥,少林寺所在之地自也早歸蒙古該管,只是蒙古大軍連年進攻襄陽不克,忙於調兵遣將,也無餘力來理會少林寺觀的事,因此少林寺一如其舊,與從前並無不同。那僧人聽郭襄的譏刺之言甚是厲害,不由得臉上一紅,心中也覺對外人下令傳諭,有些不妥,於是語轉和緩,合什説道︰「不知女施主何事光臨敝寺,且請放下兵刃,赴山下立雪亭中奉茶説話。」郭襄道︰「你們不讓我進寺,我便希罕了,哼,難道少林寺中有寶,我見一見便沾了光麼?」她見情勢不佳,便想乘此收篷,跟著又向張君寶使了眼色,低聲道︰「到底走不走?」張君寶搖了搖頭,嘴角向覺遠一努,意思説是要服侍師父。

郭襄朗聲道︰「好,我不管啦,我走了。」拔步便下坡去。第一個黃衣僧側身讓開,第二個和第三個黃衣僧却同時伸手一攔,齊聲道︰「且慢,將兵刃放下了。」郭襄眉毛一揚,手按劍柄。第一個僧人道︰「咱們少林寺千年來的規矩,還請包涵。」郭襄聽他言語彬彬有禮,心下倒是頗費躊躇︰「倘若不留短劍,勢必有一場爭鬥,自己孤身一人,如何是闔寺僧衆的敵手?但若竟將短劍留下,豈不是將爹爹、媽媽、外公、姊姊、姊夫、大哥哥、龍姊姊的面子一古腦児都丟得乾淨?」

她一時沉吟未決,驀地裡眼前黃影一晃,一個聲音喝道︰「到少林寺來既帶劍又傷人,世上焉有是理?」跟著勁風颯然,五隻手指往劍鞘上抓了下來。這僧人若不貿然出手,郭襄一番遲疑之後,多半便會將短劍留下。須知她和乃姊郭芙大不相同,雖然豪爽,却不魯莽,眼前處境既是極度不利,她便會暫忍一時之氣,日後再去和外公、爹媽商量,回頭找這場子。但那僧人突然恃強伸手奪劍,郭襄豈能眼睜睜的讓他將劍奪去?

這僧人的擒拿手法既狠且巧,一抓住劍鞘,心想郭襄定會向裡迴奪,一個和尚跟一個年輕女子拉拉扯扯,實在大是不雅,當下運勁向左斜推,跟著抓而向右。郭襄被他這麼一推一抓,果然已拿不牢劍鞘,危急中握住劍柄往外一抽,刷的一聲,寒光出匣。那僧人右手將劍鞘奪了過去,左手的五根手指却被短劍一齊割斷,劇痛之下,舉劍鞘往郭襄臉上便點,郭襄斜劍一迎,{\upstsl{噹}}的一響,將劍鞘斬爲兩截。那僧人這時也已支持不住,臉色雪白,往旁退開。

衆僧人見同門身受重傷,無不驚恐,揮杖舞棍,一齊攻來。郭襄心想︰「一不做二不休,反正今日已不能善罷。」當下使出家傳的「落英劍法」,便往山下衝去。衆僧人排成三列,仰頭擋住。

那「落英劍法」乃是黃藥師從「落英掌法」的路子中演化而來,雖不若「玉簫劍法」的精妙,却也是桃花島的一絶,但見青光激盪,劍花點點,便似落英繽紛,四散而下,霎時間僧人中又有兩人受傷。但郭襄雖然略佔上風,背後的僧人却又搶到,居高臨下的夾攻。只見邉門中僧人一個又一個的湧出,愈戰愈多。按理説郭襄早已抵擋不住,只是少林僧衆慈悲爲本,不能在山上傷她性命,是以所出招法都不是殺手,只求將她打倒,好好訓誡一番,扣下她的兵刃,這便將她逐下山去。可是郭襄劍光錯落,要攻近她的身子,却也不易。衆僧初時只道一個少齡女郎,還不輕易打發?待見她劍法精奇,始知她若不是名門之女,便是名師之徒,只恐得罪不得,一面出招時更有分寸,一面急報羅漢堂首座無色禪師。

互鬥之間,只見一個身材高瘦,便似一條竹竿般的老年僧人緩步走近,雙手籠在袖中,微笑著旁觀衆人相鬥,不時有兩個僧人走到他的身前,低聲稟告幾句。郭襄已打得劍法微見凌亂,大聲喝道︰「説什麼天下武學之源,原來是幾個大和尚一擁而上,倚多爲勝。」無色禪師説道︰「各人住手!」衆僧人一聽,立時罷鬥跳開。無色禪師道︰「姑娘尊姓,令尊和尊師是誰?光臨少林寺,不知有何貴幹?」郭襄心道︰「我爹娘的姓名不能告訴你。我到少林寺來是爲了打聽大哥哥的訊息,這事也不能在衆人之前説了出來。今日之事已鬧成這等模樣,日後爹娘和大哥哥知道,定要怪我,不如悄悄的溜了吧。」於是説道︰「我的姓名不能跟你説,我不過見山上風景優美,這便上來遊覽玩耍。原來少林寺比皇宮内院還要厲害,動不動便要扣留人家兵刃。請問大師,我走進了少林寺的山門没有?當日達摩祖師傳下武藝,想來也不過教衆僧強身健體,便於精進修爲,想不到少林寺的名頭越來越大,武功越來越高,倚衆逞強的名頭也是越來越響。好,你們要扣我兵刃,這便留下,除非你們將我殺了,否則今日之事,江湖上不會無人知曉。」

她本來便伶牙俐齒,這一件事原來也非全是她的過錯,一席話只將無色禪師説得啞口無言。郭襄也想︰「這一番胡鬧,我固然是怕人知曉,看來少林寺更是不願張揚。數十個和尚圍鬥一個年輕姑娘,説出去有什麼好聽?」當下「哼」的一聲,將短劍往地下一擲,舉步便行。

無色禪師斜步上前,袍袖一拂,已將短劍捲起,只見地下鮮血斑斑,有數人在短劍之下受傷,但劍鋒上却是無半點血漬,於是雙手托起劍身,説道︰「姑娘既不願見示家門師承,這口寶劍還請收回,老衲恭送下山。」郭襄嫣然一笑,道︰「還是老和尚通達情理,這纔是名家風範呢。」她既佔到便宜,隨口便讚了無色一句,當下伸手拿劍,一提之下,不禁吃了一驚!

原來無色禪師掌心生出一股吸力,郭襄雖然抓住劍柄,却不能提起劍身。她連運三下勁,始終無法取過短劍,説道︰「好啊,你是顯功夫來著。」突然間左手斜揮,輕輕一拂,拂向他左頸的「天鼎」「巨骨」兩穴。無色心下一凜,斜身閃避,氣勁便此一鬆,郭襄應手提起短劍。無色道︰「好俊的蘭花拂穴手功夫!姑娘跟桃花島主是怎生稱呼?」郭襄笑道︰「桃花島主嗎?我便叫他作老東邪。」原來桃花島主東邪黃藥師,乃是郭襄的外公。他性子怪僻,向來不遵禮法。他叫外孫女爲「小東邪」,郭襄便叫他老東邪,黃藥師非但不以爲忤,反而很是喜歡。這一層無色禪師却那裡知道,聽了郭襄這句話,心想黃藥師定然和她並無淵源,否則她豈敢如此無禮亂説?這麼一來,倒是少了一層顧忌。

無色禪師少年時出身綠林,雖然在禪門中數十年修持,佛學精湛,但往日豪氣,仍是不減,郭襄不肯説出師承來歷,他偏偏要試她出來,當下朗聲笑道︰「小姑娘接我十招,瞧老和尚眼力如何,能不能説出你的門派。」郭襄道︰「十招中瞧不出,那便如何?」無色禪師哈哈大笑,説道︰「你若是接得下老衲十招,那還有什麼説的,自是唯命是聽。」郭襄指著覺遠道︰「我和這位大師昔年曾有一面之緣,要代他求一個情。倘若十招中你説不出我的師父是誰,你須得答應我,不能再難爲這位大師。」無色甚是奇怪,心想覺遠迂腐騰騰!數十年來在藏經閣中管書,從來不與外人交往,怎會識得這個女郎?於是説道︰「咱們本就没難爲他啊。本寺僧衆犯了戒律,均須受罰,那也不算是什麼難爲。」郭襄小嘴一扁,冷笑道︰「哼,説來説去,你還混賴。」無色雙掌一擊,道︰「好,依你依你。老衲若是輸了,便代覺遠師弟挑這三千一百零八擔水。姑娘小心,我要出招了。」

郭襄跟他説話之時,心下早已計議定當,尋思︰「這老和尚氣凝如山,武功定是十分了得,倘若由他出招,我竭力抵禦,非顯出爹爹媽媽的武功不可。不如我佔了機先,連發十招。」聽他説到「姑娘小心,我要出招了」這兩句話,不待他出拳抬腿,嗤的一聲,短劍當胸刺過去,用的仍是桃花島「落英劍法」中的一招,叫作「萬紫千紅」,劍尖刺出時不住顫動,使敵人瞧不定劍尖到底攻向何處。無色知道厲害,不敢對攻,當即斜身閃開。

郭襄喝道︰「第二招來了!」短劍迴轉,自下而上倒刺,却是全眞派劍法中一招「天紳倒懸」。無色道︰「好,是全眞劍法。」郭襄道︰「那也未必。」短劍一刺不中,眼見無色反守爲攻,伸指來拿自己手腕,暗吃一驚︰「這老和尚果然了得,在這如此凶險的劍招之下,居然赤手空拳的還能搶攻。」眼見他手指伸到面門,短劍幌了幌,使的竟是「打狗棒法」中的一招「惡犬攔路」,乃屬「封」字訣。原來她自幼和丐幫的前任幫魯有脚交好,喝酒猜拳之餘,有時便纏著他比試武藝。丐幫中雖有規矩,打狗棒法是鎭幫神技,非幫主不傳,但魯有脚使動之際,郭襄終於偸學了一招半式。何況先任幫主黃蓉是她母親,現任幫主耶律齊是她姊夫,這打狗棒法她看到的次數著實不少,縱然不明其中訣竅,但猛地裡依樣葫蘆的使出一招來,却也是駭人耳目。無色的手指剛要碰到她的手腕,突然白光閃動,劍鋒的來勢神妙無方,險些児五根手指一齊削斷,總算他武功卓絶,變招快速,百忙中硬生生的倒退兩步,但嗤嗤聲響,袍袖上已給短劍劃破了一條長長的口子。無色禪師變色斜睨,背上驚出了一陣冷汗。

郭襄大是得意,笑道︰「這是什麼劍術?」其實天下根本無此劍術,她只不過偸學到一招打狗棒法,用在劍招之中,只因那打狗棒法過於奥妙,郭襄雖然使得似對非對,却也將一位大名鼎鼎的少林高僧嚇得滿腹疑團瞠目不知所對。郭襄心想︰「我只須再使得幾招打狗棒法,非殺得這老和尚大敗虧輸不可,只可惜除了這一下子,我再也不會了。」不待無色緩過氣來,短劍輕揚飄身而進,姿態飄飄若仙,劍鋒向無色的下盤連點數點,却是從小龍女處學來的一招玉女劍法「凌波微步」。

那玉女劍法乃當年女俠林朝英所創,不但劍招凌厲,而且講究丰神脱絶,姿式嫻雅,以郭襄這麼一位美貌少女使將出來,當眞令人瞧得心曠神怡。衆僧人從所未見,無不又驚又喜。

\chapter{崑崙三聖}

要知少林派的「達摩劍法」「羅漢劍法」等等,走的均是剛猛路子,那「玉女劍法」在江湖上絶跡已久,性質與少林派的諸種劍術又截然相反,只是一招「凌波微步」,已使無色禪師茫然若失。其實這玉女劍法也未必眞的勝於少林多路劍術,只是一眼瞧來實在美絶麗絶,有如佛經中所云︰「容儀婉媚,莊嚴和雅,端正可喜,觀者無厭。」無色禪師見了如此美妙的劍術,只盼再看一招,當下斜身閃避,待她再發。郭襄劍招斗變,東趨西走,連削數劍。張君寶在旁看得出神,忽地「噫」的一聲。原來郭襄使的這一招是「四通八達」,三年前楊過在華山之巓傳授張君寶,郭襄在旁瞧在眼中,這時便使了出來。

當年楊過所授的乃是掌法,這時郭襄變爲劍法,威力已減弱了幾成,何況無色禪師的武功勝她甚多,其時張君寶能用以制住尹克西,此刻郭襄却不能用以制住無色。但劍術之奇,却已足使無色暗暗心驚。屈指數來,郭襄已連使五招,無色竟是瞧不出絲毫頭緒。他盛年之時縱橫江湖,閲歷極富,十餘年來身任羅漢堂首座,更是精研各家各派的武功,以與本寺的武功相互參照比較,而收截長補短,切磋攻拒之效。因此他自信不論是何方高人,數招中必能瞧出他的來歷,他和郭襄約到十招,已是留下了極大餘地,豈知郭襄的父母師友,盡是當代第一流的高手,她在每人的武功中截出一招,只瞧得無色眼花繚亂,出盡全力,方始堪堪招架得住,至於對方的門派劍法,那裡説得出什麼名目。

那四通八達的四劍八式一過,無色心念一動︰「我若任她出招,只怕她怪招源源不絶,别説十招,一百招也未必能瞧出什麼端倪。只有我發招猛攻,她便非使出本門武功拆解不可。」當即上身左轉,一招「雙貫拳」,雙拳虎口相對,劃成弧形交相撞擊。郭襄見他拳勢勁力奇大,不敢擋架,身形一扭,竟從雙拳之間溜了過去。這一招是什麼?原來是她在萬花谷中見瑛姑與楊過相鬥,弱不敵強,便使「泥鰍功」溜開。

無色喝采道︰「好身法,再接我一招。」左掌圏花揚起,屈肘當胸,虎口朝上,正是少林拳中的「黃鶯落架」。他是少林寺中的武學大師,身份不同,雖然所會武功之雜,不下郭襄,但每一招每一式,使的均是最純正的本門武功。那少林拳門戸正大,看來似乎平平無奇,但練到精深之處,實是威力無窮。他這左掌圏花一揚,郭襄但覺自己上半身已全在掌力籠罩之下,當即倒轉劍柄,以劍柄作爲手指,使一招從武修文處學來的「一陽指」,逕點他的手腕上「腕骨」「陽谷」「養老」三穴。她的「一陽指」點穴功法實只學到一點児皮毛,膚淺之至,但一陽指點三穴的手法,却正是一陽指功夫的精要所在。一燈大師的一陽指功夫天下馳名,無色禪師自然識得,他一見郭襄出此一招,一驚之下,急忙縮手變招。其實無色倘若並不縮手,任她連撞三處穴道,登時便可發覺這「一陽指」功夫並非貨眞價實,但雙方各出全力搏鬥之際,他豈肯輕易以一世英名,冒險相試?

郭襄嫣然一笑,道︰「大和尚倒識得厲害!」無色哼了一聲,擊出一招「單鳳朝陽」,這一招雙手大開大闔,寬打高舉,使她的一陽指無法用上,郭襄雙拳交錯,若有若無,正是老頑童周伯通得意傑作七十二路空明拳中第五十四路「妙手空空」。這路拳法是周伯通所自創,江湖上並未流傳,無色縱然淵博,却也無法識得,當下雙掌劃弧,發出一招「偏花七星」。這時雙掌猶如電閃,一下子切到了郭襄掌上,要她若不是出内力相抗,手掌便須向後一拗而斷。

無色禪師一招一式,均從平淡之中見功夫,這一招「偏花七星」似慢實快,似輕實重,姿式是「闖少林」的姿式,意勁内力,却出自「神化少林」的精奥。少林派武功天下揚名,無色禪師是個中高手,這一招擊來,果然是氣吞河海,沛然莫禦。小郭襄手掌被制,心想︰「難道你眞能折斷我的掌骨不成?」順手一揮,使出一招「鐵蒲扇手」,以掌對掌,反擊過去。這一招她是從武修文之妻完顏萍處學來,乃是當年鐵掌水上飄裘千仞傳下來的心法。這鐵掌功在武學諸派掌法之中,向稱剛猛第一,無色禪師精研掌法,如何不知?眼見這女郎猛地裡使出這招鐵掌幫的看家掌來,不禁嚇了一跳,若是跟他硬拚掌力,一來不願便此傷她,二來却也眞的對鐵掌功夫有三分忌憚。他是個忠厚豪邁之人,但見郭襄每一招都使得似模似樣,他雖見多識廣,一時之間却没想到若要精研這許多門派的武功,豈是這二十歳不到的少女就能辦到。當下急忙收掌,退開半丈。

郭襄嫣然一笑,叫道︰「第十招來了,你瞧我是什麼門派?」左手一揚,和身欺上,右手伸出便去托拿無色的下顎。旁觀衆僧和無色情不自禁,都是一聲驚呼,原來這一招「苦海回頭」,正是少林派正宗拳藝的擒拿手法,却是别派所無。但這種擒拿功夫近身相搏,生死決於呼吸之間,若非有十分把握或是到了緊急関頭,決不肯輕易使用。這一招「苦海回頭」用意是左手按住敵人頭頂,右手托住敵人下顎,將他頭頸一扭,倘若成功,重則扭斷敵人頭頸,輕則扭脱関節,乃是一招極厲害的殺手。

無色禪師見她竟然使到這一招,當眞孔夫子門口讀孝經,魯班門前弄大斧,不由得又是好氣,又是好笑。這種擒拿手法他在數十年前早已拆得滾瓜爛熟,一碰上便是不加思索,隨手施應,即令是睡著了,遇到這種招式只怕也能對拆,當下斜身踏步,左手橫過郭襄體前,一翻手,已扣住了她的脅下,右手疾如閃電,伸手到郭襄的膝彎之後。這一招叫做「挾山超海」,原是拆解那招「苦海回頭」的不二法門,雙手一提,便能將敵人身子提得離地橫起。郭襄接下去本可用「盤肘」式反壓他的手肘,既能脱困,又可反制敵人,但無色禪師這一招實在來得太快,眼睛一瞬,身子便已被提起,她雙足離地,還能施展什麼功夫,自然是輸了!

無色禪師隨手將郭襄制住,心中一怔︰「糟糕,我只顧取勝,却没想到辨認她的師承門派。她在十招中使了十種不同的拳法,那是如何説法?我總不能説她是少林派!」郭襄用力一掙,叫道︰「放開我!」只聽得錚的一聲響,從她身掉下了一件物事。郭襄叫道︰「老和尚,你還不放我?」無色禪師是個有道高僧,眼中看出來衆生平等,别説已無男女之分,縱是馬牛豬犬,他也一視同仁,哈哈大笑道︰「老衲這一把年紀,做你祖父也做得,還怕什麼?」説著雙手一送,將她抛出二丈之外。這一下過手,郭襄雖然被制,但無色在十招之内,終究認不出她的門派,他這種有身之人,説過了的話如何不算?正要出言服輸,一俯身,忽見地下黑黝黝的一團物事,那兩個鐵鑄的羅漢。

只聽郭襄説道︰「大和尚,你可認輸了吧?」無色抬起頭來,喜容滿面,笑道︰「我怎麼會輸?我知道令尊是郭靖,令堂是女俠黃蓉,桃花島主是你外公。令尊學兼江南七怪、桃花島、九指神丐、全眞派各家之長,郭二小姐家學淵源,身手果然不凡。」這一番話只把郭襄聽得瞪目結舌,半晌説不出話來,心想︰「這老和尚眞邪門,我這十招包羅了十位親友的不同武功,他居然仍舊認了出來。」

無色禪師見郭襄茫然自失,當下笑吟吟的俯身從地下拾起一對鐵鑄的小羅漢,説道︰「郭二姑娘,老和尚不能騙你小孩子,我所以能認出你來,全憑這對羅漢。楊大俠可好?你可有見到他麼?」郭襄一怔之下,立時恍然,説道︰「你没見到我大哥和龍姊姊嗎?我上寶刹來,便是來打聽他二人的下落。啊,你不知道,我説的大哥哥和龍姊姊,便是楊過楊大俠夫婦了。」無色道︰「數年之前,楊大俠曾來敝寺盤桓數日,跟老和尚很是説得來。後來聽説他在襄陽城外擊斃蒙古皇帝,名揚天下,敝寺僧衆接到這個訊息,無不歡忭。不知他刻下是在何處?原來他已成婚,他那位夫人,看來也必是一位文武雙全的女俠了?」

他二人都甚性急,均欲知楊過的音訊,你問一句,我問一句,却是誰也没回答對方的問話。郭襄站立山坡之上,呆了半晌,説道︰「原來你便是無色禪師,怪不得武功如此高明。{\upstsl{嗯}},我還没謝過你送給我的生日禮物,今日得謝謝你啦。」無色笑道︰「咱們當眞是不打不相識。你見到楊大俠時,可别説老和尚以大欺小。」郭襄望著遠處山峰,自言自語︰「幾時方能見著他啊。」

原來當郭襄滿十六歳做生日之時,楊過忽發奇想,柬邀江湖同道,群集襄陽給她慶賀生辰。一時白道黑道上無數武林高手,衝著楊過的面子,都受邀趕到祝壽,即使無法分身的,也都贈送珍異賀禮。無色禪師請人帶去的生日禮物,便是這一對精鐵鑄成的羅漢。這對鐵羅漢肚腹之中裝有機括,扭緊彈簧之後,能對拆一套少林羅漢拳,那是百餘年前少林寺中一位異僧花了無數心血,方始製成,端的是靈巧精妙無比。郭襄覺得好玩,便帶在身邉,想不到今日從懷中跌將出來,終於給無色禪師認出了她的身份。郭襄適纔所使的一招,分别學自各位師友,無一不是奥妙絶倫之作,最後一招少林拳法,便是從這對鐵羅漢身上學來。

無色笑道︰「格於敝寺歷代相傳的寺規,不能請郭二姑娘到寺中隨喜,務請包涵。」郭襄黯然道︰「那没有什麼,我問的事,反正也問過了。」無色又指著覺遠道︰「至於這位師弟的事,我慢慢再跟你解釋。這樣吧,老和尚陪你下山去,咱們找一家飯舖,讓老和尚作個東,好好喝幾天酒,你説怎樣?」無色禪師在少林寺中位分極高,竟對郭襄這樣一個妙齡女郎如此尊敬,要自送她下山,隆重款待,衆僧侶在旁聽了,心中都是暗暗稱奇。

郭襄道︰「大師不必客氣。小女子出手不知輕重,得罪了幾位師兄,還請代致歉意,這便别過,後會有期。」説著施了一禮,轉身下坡。無色笑道︰「你不要我送,我也要送。那年姑娘生日,老和尚正當坐関之期,没能親來道賀,心中已自不安,今日光臨敝寺,若再不恭送三十里,豈是相待貴客之道?」郭襄見他一番誠意,又喜他言語豪爽,也願和他結個方外的忘年之交,於是微微一笑,道︰「走吧!」

當下二人並肩下坡,走過立雪亭後,只聽得身後脚步聲響,回首一看,只見張君寶遠遠在後跟著,却是不敢走近。郭襄笑道︰「張兄弟,你也來送客下山嗎?」張君寶臉上一紅,應了一聲︰「是!」便在此時,只見山門前一個僧人大步奔下,他竟是全力施展輕功,跑得十分匆忙,無色眉頭一皺,説道︰「大驚小怪的幹什麼?」那僧人奔到無色身前,行了一禮,低聲説了幾句話。無色臉色忽變,大聲道︰「竟有這等事?」那僧人道︰「老方丈請首座便去商議。」

郭襄見無色臉上神色頗是爲難,知他寺中必有要事,説道︰「老禪師,朋友相交,貴在知心,一些俗禮算得了什麼?你有要事便請回去。他日江湖相逢,有緣邂逅,咱們再喝酒論武,有何不可?」無色喜道︰「怪不得楊大俠對你這般看重,你果然是人中英俠,女中丈夫,老和尚交了你這個朋友。」郭襄微微一笑,道︰「你是我大哥哥的朋友,自然也是我的朋友。」當下兩人施禮而别,只見無色大袖飄飄,回向山門。

郭襄循路下山,張君寶在她的身後,相距五六步,終是不敢和她並肩而行。郭襄道︰「張兄弟,他們到底幹麼欺侮你師父?你師父一身精湛内功,怕他們何來?」張君寶走近兩步,説道︰「寺中戒律精嚴,僧衆凡是犯了事的,都須受罰,倒不是故意欺侮師父。」郭襄奇道︰「你師父眞是個正人君子,天下從來没這樣的好人,他又犯了什麼事?我瞧他一定是代人受過,要不,便是什麼事弄錯了。」張君寶嘆口氣道︰「這事的原委姑娘其實也知道,還不是爲了那部楞伽經。」郭襄道︰「啊,是給瀟湘子和尹克西這兩個傢伙偸去的經書麼?」張君寶道︰「是啊。那日在華山絶頂,小人得楊過大俠的指點,親手搜査了那兩人全身,自一下華山之後,再也找不到這兩個人的蹤跡。咱師徒倆無奈,只得回寺來稟報方丈和戒律堂首座。那部楞伽經是達摩祖師親手所書,戒律堂首座責怪我師徒經管不愼,以致失落無價之寶,重加處罰,原是罪有應得。」

郭襄嘆了口氣,道︰「那叫做晦氣,什麼罪有應得?」她比張君寶只大幾歳,但儼然以大姊姊自居,又問︰「爲了這事,便罰你師父不許説話?」張君寶道︰「這是寺中歷代相傳的戒律,上鐐挑水,不許説話。我聽寺裡的老禪師們説,雖然這是處罰,但對受罰之人其實也大有好處。一個人一不説話,修爲自是易於精進,而上鐐挑水,也可強壯體魄。」郭襄笑道︰「這麼説來,你師父非但不是受罰,反而是在練功了,倒是我的多事。」張君寶忙道︰「姑娘一番好心,師父和我十分感激,永遠不敢忘記。」郭襄輕輕嘆了口氣,心中説道︰「可是旁人却早把我忘記得一乾二淨了。」

只聽得樹林中一聲驢鳴,郭襄那頭青驢便在林中吃草。郭襄道︰「張兄弟,你也不必送我啦。」呼哨一聲,招呼青驢近前。張君寶頗有不捨之情,却又没什麼話好説。郭襄知他心意,將手中那對鐵羅漢遞了給他,道︰「這個給你。」張君寶一怔,不敢伸手去接,道︰「這\dash{}這個\dash{}」郭襄道︰「我説給你,你便收下了。」張君寶道︰「我\dash{}我\dash{}」郭襄將鐵羅漢塞在他的手中,縱身一躍,上了驢背。

突然山坡石級上一人叫道︰「郭二姑娘,且請留步。」正是無色禪師又從寺門中奔了出來,郭襄心道︰「這個老和尚也忒煞多禮,何必定要送我?」只見無色行得甚快,片刻間便到了郭襄身前,他向張君寶道︰「你回寺中去,别在山裡亂走亂闖。」張君寶移身答應,向郭襄凝望一眼,走上山去。

無色待他走開,從袍袖中取出一張紙箋來,道︰「郭二姑娘,你可知道是誰冩的麼?」郭襄下了驢背,接過一看,見是一張詩箋,箋上墨瀋淋漓,冩著兩行字道︰「十天後,崑崙三聖親赴少林寺,領教武林絶學。」筆勢挺拔遒勁,當眞是力透紙背。郭襄看了,問道︰「崑崙三聖是誰啊,這三個人的口氣倒大得緊。」無色道︰「原來姑娘也不識得他們。」郭襄搖頭道︰「我不識得。連『崑崙三聖』的名字也從没聽爹爹媽媽説過。」無色道︰「奇便奇在這児。」

郭襄道︰「什麼奇怪啊?」無色道︰「姑娘和我一見如故,這事自可對你實説。你道這張紙箋是在那裡得來的?」郭襄道︰「是那崑崙三聖派人送來的麼?」無色道︰「若是派人送來,那也没什麼奇怪了。常言道樹大招風,我少林寺數百年來號稱是天下武學的發源之所,因此不斷有高手到寺中來挑戰較藝,那也不足爲異。每次有武林中人到寺中,咱們總是好好款待,説到比武較量,能彀推託,便儘量推託。咱們做和尚的,講究的是勿嗔勿怒,不得逞強爭勝,倘若天天跟人家打架,那還算是什麼佛家子弟麼?」郭襄點頭道︰「那也説得是。」無色又道︰「只不過武師們既然上得寺來,若是不顯一下身手,總是心不甘服。少林寺的羅漢堂,做的便是這門接待外來武師的幹當。」郭襄笑道︰「原來大和尚專職是跟人打架。」無色苦笑道︰「一般武師,武功再強,本堂的弟子們總能應付得了,倒也不必老和尚出手。今日因見姑娘身手不凡,我才自己來試上一試。」郭襄笑道︰「你倒看得起我。」

無色道︰「你瞧我把説話扯到那裡去啦。這張紙箋實不相瞞,是在羅漢堂上降龍羅漢佛像的手中取下來的。」郭襄奇道︰「是誰放在佛像手中的?」無色搔頭道︰「便是不知道啊。想我少林寺僧衆數百,若有人混進來,豈能無人看見?這羅漢堂中更是經常有八名子弟輪値,日夜不斷。剛纔有人瞧見了這張紙箋後,飛報老方丈,都覺奇怪,因此召我回寺商議。」郭襄聽到這裡,已明其意,説道︰「你疑心我和那什麼崑崙三聖串通了,我到寺外搗亂,那三個傢伙便混到羅漢堂中放這紙箋。是也不是?」無色道︰「我既和姑娘見了面,自是絶無疑心,但老方丈和無相師兄他們,却不能不錯疑到姑娘身上。也是事有湊巧,姑娘剛剛離寺,這張紙箋便在羅漢堂中出現。」

郭襄道︰「我跟你説過,我不認得這三個傢伙。大和尚,你怕什麼?十天之後他們若是膽敢前來,跟他們見個高下便了。」無色道︰「害怕嘛,自然不怕。姑娘既跟他們没有干係,我便不用耽心了。」郭襄心知他實是一番好意,只怕崑崙三聖是自己的相識,那麼動手之際便有許多顧忌,唯恐得罪了好朋友,於是説道︰「大和尚,他們客客氣氣來切磋武藝,那便罷了,否則好好給他們吃些苦頭。從這張字條上的口氣上看來,這三人可狂妄得很呢。」她説到這裡,忽然想起一事,説道︰「説不定寺中有誰跟他們勾結了,偸偸放上這樣一字條,也没什麼希奇。」無色道︰「這事咱們也想過了,可是決計不會。那降龍羅漢的手指離地有三丈多高,平時掃除佛身上灰塵,必須搭起高架。輕功再好的人,也不能躍到這般高處。寺中縱有叛徒,也不會有這樣好的功夫。」

他越説越怪,郭襄好奇心起,很想見見這崑崙三聖,到底是何等樣的人物,要瞧瞧他們和少林寺僧比試武藝,結果誰勝誰負,只是少林寺不接待女客,看來這場好戲是不能親眼得看見了。無色見她側頭沉思,只道她是在代少林寺籌策,説道︰「少林寺千年來經過了不知多少大風大浪,終於也没給人家挑了,這崑崙三聖既是決意跟咱們過不去,少林寺也不能墮了千年來的威風。郭姑娘,半月之後,你在江湖上當可聽到音訊,且看崑崙三聖是否能把少林寺毀了。」他説到此處,壯年時的豪情勝慨,不禁又勃然而發。郭襄笑道︰「大和尚勿嗔勿怒,你這説話的樣子,算是佛門子弟麼?好,半月之後,我佇候好音。」説著翻身上了驢背。兩人相視一笑,郭襄催動青驢,得得下山,心中却早打定主意,非瞧一瞧這場熱鬧不可。

郭襄心想︰「怎生想個法児,十天後混到少林寺去瞧一瞧這場好戲?」又想︰「只怕那崑崙三聖未必是有什麼眞才實學的人物,給大和尚們一擊即倒,那便熱鬧不起來。只要他們有外公、爹爹、或是大哥哥一半的本事,這一場『三聖大鬧少林寺』便有些看頭。」她一想到楊過,不禁心頭又是鬱鬱,這三年來到處尋尋覓覓,始終是落得個冷冷清清,終南山古墓長閉,萬花谷花落無聲,絶情谷空山寂寂,風陵渡冷月冥冥。她心頭早已千百遍的想過︰「其實,我便是找到了他,那又怎地?還不是重添相思,徒增煩惱?他所以悄然遠引,也還不是爲了我好?但明知那是鏡花水月一場空,我却又不能不想,不能不找。」

她任著青驢信步所之,在少室山中漫遊,一路向西,已入嵩山之境,回眺少室東峰,秀聳拔地,沿途山景,觀之不盡。如此遊了數日,這一天到了三休台上,心道︰「三休,三休!却不知是那三休?人生千休萬休,又何止三休?」折而向北,過了一嶺,只見古柏三百餘章,皆挺直端圓,凌霄托根柏旁,作花柏頂,燦若雲茶,郭襄正在觀賞,忽聽得山坳後隱隱傳出一陣琴聲,不禁大奇︰「這荒僻之處,居然有高人雅士在此操琴。」她幼受母教,琴棋書畫,無一不會,雖然均非精通,但她生來聰頂,又愛異想天開,因此偶然和黃藥師論琴,跟朱子柳學書,往往有獨到之見,發前人之所未發。這時聽到琴聲,好奇心起,當下放了青驢,循聲尋去。

走出數十丈,只聽得琴聲之中雜有無數鳥語,初時也不注意,但細細聽來,那琴聲竟似和鳥語互相應答,間間関関,宛轉啼鳴,郭襄隱身在花木之後,向琴聲發出處一張,只見三株大松樹下一個白衣男子背向而坐,膝上放著一張焦尾琴,正自彈奏。他身周樹木上停滿了雀鳥,有黃鶯,有杜鵑,有喜鵲,有八哥,和那琴聲或一問一答,或齊聲和唱。郭襄心道︰「外公説琴調之中,有一曲『空山鳥語』,久已失傳,莫非便是此曲麼?」聽了一會,琴聲漸響,但愈到響處,愈是和醇,群鳥却不再發聲,只聽得空中振翼之聲大作,東南西北各處又飛來無數雀鳥,或止歇樹巓,或上下{\upstsl{翱}}翔,毛羽繽紛,蔚爲奇觀。那琴聲平和中正,隱然有王者之意,郭襄吃了一驚︰「此人能以琴聲集鳥,這一曲難道竟是『百鳥朝鳳』?」以音樂感應鳥獸,原非奇事,古人只道對牛彈琴,牛不入耳,其實今人已知音樂可使母牛增産牛乳,可用音波誘魚入網,甚至能以音樂促使植物生長加速,須知昆蟲求偶,鳥獸呼侶,皆出之以音,宇宙之間,天籟無窮。師曠琴聲能使風雲變色,自是神乎其説,不足爲信,但呼鳥馴獸,驅蛇起舞,却是歷代均有。此是閒話,表過不提。且説郭襄聽著琴聲,越聽越奇,心想可惜外公不在這裡,否則以他天下無雙的玉簫與之一和,實可稱並世雙絶。

那人彈到後來,琴聲漸低,樹上停歇著的雀鳥一齊起而盤旋飛舞。突然間錚的一聲琴聲止歇,群鳥飛翔了一會,慢慢散去。那人隨手在琴絃上彈了幾下短音,漫聲吟道︰「白日何短短,百年苦易滿。蒼穹浩茫茫,萬劫太極長。麻姑垂兩鬢,一半已成霜。天公見玉女,大笑億千場。吾欲攬六龍,迴車拄扶桑。北斗酌美酒,勸龍各一觴。富貴非所願,爲人駐顏光。」但聽那人吟聲悲涼,似覺人生憂患,不可斷絶,郭襄怔怔的聽著,不禁兩行情泪,垂下雙頰。那人高吟已畢,仰天長嘆,説道︰「撫長劍,一揚眉,清水白石何離離?世間苦無知音,縱活千載,亦復何益?」

那人説到此處,突然間從琴底抽出一柄長劍,但見青光閃閃,照映林間,郭襄心想︰「原來此人文武全才,倒要瞧瞧他的劍法如何。」只見他緩步走到古松前的一塊空地上,劍尖抵地,一畫一畫的劃了起來,劃了一畫又是一畫。郭襄大奇︰「世間怎會有如此奇怪的劍法?難道以劍尖在地下亂劃,便能克敵制勝?此人之怪,眞是不可以常理測度了。」她默默數著他的劍招,只見他橫著劃了十九招,跟著變向縱劃,一共也是一十九招。劍招始終不變,自左而右的劃去,每一招均是相隔約莫一尺。郭襄約著他的劍勢,伸手指在地下劃了一遍,一看之下,險些失笑,原來他使的那裡是什麼怪異劍法,却是以劍尖在地畫了一張縱橫各一十九道的大棋盤。只見那人劃完棋盤,以劍尖在左上角和右下角圏了一圏,再在右上角和右下角畫了個交叉。郭襄既已看出他畫的是一張圍棋棋盤,自也想到他是在四角佈上勢子。圓圏是白子,交叉是黑子,跟著見他在左上角距勢子三格處圏了一圏,又在那圓圏下兩格處畫了一叉,待得下到第十二著時,一時決不定該當棄子取勢,還是力爭邉角,只見他以劍拄地,低頭沉思。郭襄心想︰「原來此人和我一般寂寞,空山撫琴,以雀鳥爲知音;下棋又没有對手,只得自己跟自己下。」

\chapter{白衣書生}

那人想了一會,白子不肯罷休,當下與黑子在左上角展開劇鬥,一時之間妙著紛紜,自北而南,逐步爭到了中原腹地,但白子布局時棋輸一著,始終落在下風,到第九十三著上遇到了個連環劫,白勢已是岌岌可危,但他仍是要勉力支撐。郭襄在旁看得心焦,忍不住脱口叫道︰「何不逕棄中原,反取西域?」

那人一凜,只見棋盤西邉留著一大片空地,如果乘著打劫之時連下兩子佔先,即使棄了中腹,仍可設法爭取個不勝不敗的局面。那人被郭襄一言提醒,仰天長笑,連説︰「好,好!」跟著下了數子,突然想起有人在旁,忙將長劍在地下一擲,轉身説道︰「那一位高人承教,在下當眞是感激不盡。」説著向郭襄藏身處一揖。

郭襄見這人長臉深目,廋骨稜稜,約莫三十歳左右年紀,她向來脱略,也不理會男女之嫌,從花叢之中走了出來,笑道︰「適纔聽得先生雅奏,空山鳥語,百禽來朝,實深欽佩。又見先生畫地爲局,仗劍書譜,忍不住多嘴,還祈見諒。」那人見郭襄是個妙齡女郎,大以爲奇,但聽郭襄説到他的琴聲,居然一絲不錯,心下很是高興,説道︰「姑娘深通琴理,若蒙不棄,願聞清音。」郭襄笑道︰「我媽媽雖也教過我彈琴,但比起你的神乎奇技,却是差得遠了,不過我既已聽過你的妙曲,不回答一首,却有點説不過去。好吧,我彈便彈一曲,你却不許取笑。」那人道︰「怎敢?」於是雙手捧起瑤琴,送到郭襄面前。

郭襄見這琴古紋斑爛,顯是年月已久,接過時著手甚輕,於是調了調琴絃,彈了起來,奏的是一曲「考槃」。郭襄的手法自没什麼出奇,但那人却聽得臉有驚喜之色,他順著琴音,心中默想詞句︰「考槃在澗,碩人之寬,獨寐寐言,永矢勿諼。」原來這調出自「詩經」,是一首隱士之歌,意思説大丈夫在山澗之間遊蕩,獨往獨來,雖然臉有憔悴之色,寂寞無侶,但志向高潔,永不改變。那人聽郭襄的琴音正説中自己的心事,不禁大是感激,郭襄琴聲已畢,他還是痴痴的站著。

郭襄輕輕將瑤琴放下,轉身走出松谷,縱聲而歌︰「考槃在陸,碩人之軸,獨寐獨宿,永矢勿告。」招來青驢騎上了,又往深山林密之處行去。

郭襄在江湖上闖蕩數年,所經異事甚多,那人琴韻集禽,畫地自奕之事,在她也只是過眼雲煙,風萍聚散,不著痕跡。又過兩天,屈指算來已是他大鬧少林寺的第十天,便是崑崙三聖約定和少林高僧較量武藝的日子,郭襄天没亮便起來,低頭沉吟,一時可想不出如何混入寺中,看看到底是誰高誰下,心道︰「我雖是媽媽的女児,但她偏偏這麼機伶,什麼事児眼睛一轉,便想到了十七八條妙計,我却偏偏這麼蠢,連一條計策也想不出來。好吧,不管怎樣,我先到寺外去瞧瞧再説,説不定他們應付外敵,打得熱鬧,便忘了攔阻我進寺。」

這日早晨胡亂吃了些乾糧,便騎著青驢,又往少林寺進發,行到離寺約有十里之處,忽聽得馬蹄聲響,左側山道上有三乘馬連騎而來。三匹馬一青一黃一白,都是腿長膘肥,步子甚是迅捷,轉眼之間便從郭襄身前掠過,直上少林寺而去。馬背上三人都是五十來歳的老者,身穿青布短衣,馬鞍上都掛著盛兵刃的布囊。郭襄心念一動︰「這三人均是身負武功,今日帶了兵刃上少林寺,多半便是崑崙三聖了。我若是遲了一步,只怕瞧不到好戲。」於是伸手在青驢臀上一拍,青驢昂首一聲嘶叫,潑刺刺的自後趕了上去。這青驢身形雖小,脚力却健,片刻間便追到了三乘馬的身後。這時郭襄看清楚了那三個老者的身形,青馬的乘客身材矮小,黃馬的乘客中等身材,白馬乘客却是極高極瘦,三人坐在馬背上都不用馬鞍。再細瞧那三頭牲口時,只見三匹馬均是鬣毛特長,小腿上也是長毛垂地,與中土一般馬匹迥異。那三匹馬登山越嶺,如履平地,馬上乘客一覺得郭襄縱驢追來,馬鞭一揮,三乘馬疾馳上山,頃刻間將郭襄的青驢抛得老遠,再也追趕不上。只見青馬和黃馬上的乘客都回頭望了一眼,見郭襄這麼一個年輕姑娘孤身上山,似乎心中都感奇怪。

郭襄縱驢又趕了二三里地,那三騎馬已奔得影蹤不見,青驢這一程快奔,却已是噴氣連連,頗有些支持不住。郭襄叱道︰「不中用的畜生,平時儘愛鬧脾氣,發蠻勁,姑娘當眞要用你時,却又追不上人家。」眼見更催也是無用,索性便在道旁一個石亭中憩息片刻,讓青驢在道旁的泉水中喝一個飽。過不多時,忽聽得馬蹄聲響,那三乘馬轉過山坳,奔了回來。郭襄大奇︰「怎地這三人一上去便回了轉來,難道當眞是如此不堪一擊?」

果見那三匹駿馬奮鬣揚蹄,直奔進石亭中來,三個乘客翻身下馬,讓那馬匹休息。郭襄瞧那三人時,見矮老者臉若硃砂,一個酒糟鼻子火也般紅,笑咪咪的神色頗爲溫和可親;那竹竿般身裁的老者却是臉色鐵青,蒼白之中隱隱泛出一層綠氣,倒似終年不見天日一般,這兩人身形容貌,無一不是截然相反。第三個老者相貌平平無奇,只是臉色臘黃,微帶病容。

郭襄好奇心起,問道︰「三位老先生,你們到了少林寺没有?怎地剛上去便回下來啦?」那青臉老者橫了她一眼,似怪她亂説亂問,那酒糟鼻的紅臉矮子笑道︰「姑娘怎知咱們是到少林寺去?」郭襄道︰「從此上去,不到少林寺却往何處?」那紅臉老者點頭道︰「這話是不錯。姑娘却又往何處去?」郭襄道︰「你們去少林寺,我自然也去少林寺。」那青臉老者突然插口,説道︰「少林寺向來不許女流踏進山門一步,又不許外人擕帶兵刃進寺。」他説話的語氣甚是傲慢,他身形甚高,説話之時眼光從郭襄頭頂上瞧了過去,向她望也不望一眼。郭襄心下氣惱,説道︰「你們怎又擕帶兵刃?那馬鞍旁的布囊之中,放的難道不是兵器?」

那青臉老者冷冷的道︰「你怎能跟咱們相比?」郭襄冷笑一聲道︰「你們三個又怎樣?難道便這般橫?崑崙三聖跟少林寺的老和尚們交過了手麼?誰勝誰敗啊?」那三個老者聽郭襄提到崑崙三聖四字,臉上都是神色微變,那紅臉老者問道︰「小姑娘,你怎會知道『崑崙三聖』的事?」郭襄道︰「我自然知道。」青臉老者突然踏上一步,厲聲道︰「你姓什麼?是誰的門下?到少林寺來幹什麼?」郭襄俏臉一揚,道︰「你管得著麼?」那青臉老者脾氣暴躁,又是數十年來到處受人尊崇,從未受過這般挺撞,手掌一揚,便想給她一個耳光,但跟著便想大欺小,男欺女甚不光采,自己是何等身份,怎能跟小孩子一般見識?當下身形一晃,伸手便摘下郭襄腰間懸著的短劍,這一下動作之快,實是難以形容,郭襄但覺涼風輕颺,人影閃動,自己的佩劍便給他搶了過去。

她猝不及防,猛地裡著了人家的道児,倒是她行走江湖以來從所未有之事。其實以郭襄的武功閲歷,若要在江湖間闖蕩,原是大大不彀,但武林之中,十之八九,都知她是郭靖的女児,便是旁門左道之士,經過楊過傳柬給她慶賀生辰之後,幾乎也是無人不曉,即使不礙著郭靖的面子,也礙著楊過的面子。兼之郭襄人既美麗,性格児又是豪爽好客,即使市井中引車賣漿、屠狗負販之徒,她也是一視同仁,往往沽了酒來請他們共飲一杯。因此江湖間雖然風波險惡,她竟是履險如夷,逢凶化吉,從來没吃過半點虧。這青臉老者驀然奪了她的劍去,竟使她一時之間不知所措,若是上前相奪,自忖武功遠遠不及人家,但如就此罷休,心下又豈能甘?

那青臉老者左手的中指和食指挾著短劍的劍鞘,冷冰冰的道︰「你這把劍,我暫且扣下了。你膽敢對我這等無禮,自是父母和師長少了管教,你去要他們來向我取劍,我會跟他們好好一説,教你父母師長多留上一點児神。」這一番話眞把郭襄氣得滿臉通紅,聽這人話中之意,直是將她當作了一個没有家教的頑童,心想︰「好哇!你罵了我,也罵了我外公和爹娘,你當眞有通天本事,這般天不怕地不怕的亂逞威風?」她定了定神,強忍一口怒氣,説道︰「你叫什麼名字?」

那青臉老者哼了一聲,道︰「什麼『你叫什麼名字』我教你,你該這麼問︰『不敢請教老前輩尊姓大名?』」郭襄怒道︰「我偏要問你叫什麼名字,你不説便不説,誰又希罕了?這把劍又値得什麼?你爲老不尊,偸人搶人的東西,我也不要了。」説著轉過身子,便要走出石亭。忽然間眼前紅影一閃,那個紅臉矮子已擋在她身前,笑咪咪的道︰「女孩児家不可發這般大的脾氣,將來嫁了婆家做媳婦児,難道也由得你使小性児麼?好,我便跟你説,咱們師兄弟三人,這幾天萬里迢迢的剛從西域趕來中原\dash{}」郭襄小嘴一扁,道︰「你不説我也知道。咱們神州中原,本是没你三個的字號。」

三個老者相互望了一眼,那紅臉老者道︰「請問姑娘,尊師是那一位?」郭襄在少林寺中不肯説父母的名字,這時心下眞的惱了,説道︰「我爹爹姓郭,單名一個『靖』字。我媽媽姓黃,單名一個『蓉』字。我没有師父,就是爹爹媽媽胡亂教一些児。」三個老者又相互望了一望,只聽那青臉老者喃喃的道︰「郭靖?黃蓉?他們是那一門那一派的?是誰的弟子?」郭襄這一氣當眞是非同小可,心想我父母名滿天下,别説武林中人,便是尋常百姓,又有誰不知義守襄陽的郭大俠?

但瞧那三個老者的神色,却又不似假不知,郭襄心念一動,當即恍然︰「這崑崙三聖遠處西域,從來不履中土。以這般高的武功,爹爹却從來没提過他們的名頭,那麼他們眞的不知爹爹媽媽,也是不足爲奇的了。想必他們在崑崙山深處隱居,勤練武功,對那外事從來不聞不問。」想到這裡,心下登時釋然,怒氣漸消,説道︰「我姓郭名襄,便是襄陽城這個『襄』字。好啦,我都對你們説了,你們三位尊姓大名啊?」

那紅臉老者笑嘻嘻的道︰「是啊,小女娃児一教便會,這纔是尊敬長輩的道理。」他指著那黃臉老者道︰「這位是咱們大師哥,他姓潘,名字叫天耕。我是二師兄,姓方,叫作天勞。」又指著青臉老者︰「這位三師弟姓衛,名叫天望。咱師兄弟三個排行之中都有一個天字。」郭襄「{\upstsl{嗯}}」了一聲,心下默記一遍,説道︰「你們到底上不上少林寺去,你們跟那些和尚比過武藝麼,却是誰的武功強些?」那青臉老者「咦」的一聲,厲聲道︰「怎麼你什麼都知道?咱們要跟少林寺和尚比試武藝,天下没有幾人知曉,你怎麼得知?快説,快説!」説著直逼到郭襄身前,惡狠狠的瞪著她。

郭襄暗想︰「我是什麼人,豈能受你的威嚇?本來跟你説了也不要緊,但你越是惡,我越是不説。」向著他也瞪了一眼,冷然道︰「你這個名字不好,爲什麼不改作『天惡』?」衛天望怒道︰「什麼?」郭襄道︰「如你這般凶神惡煞般的人物,我當眞還是少見。偸了我搶了我的東西,還這麼狠霸霸的,這不是天上的天惡星下凡麼?」衛天望喉頭胡胡幾聲,發出猶似獸{\upstsl{嘄}}般的響聲。胸口突然間脹大了一倍,似乎頭髮和眉毛都豎了起來。那紅臉老者方天勞急叫︰「三弟,不可動怒!」拉著郭襄手臂往後一扯,將她扯在丈許之外,自己身子已隔在二人之間。郭襄見了衛天望這般暴怒的情景,知他若是猛然出手,其勢定不可當,不由得心中也生懼意。

只見衛天望右手抓住短劍的劍柄,拔劍出鞘,左手兩根手指平平挾住劍刃,勁透指節,喀的一聲,劍刃又登時斷爲兩截。他跟著將半截斷劍還入劍鞘,説道︰「誰要你這把不中用的短劍了?」郭襄見他手指上的勁力如此厲害,心下也自駭然,心想這雖然及不上外公的彈指神通功夫,却也是平生罕見的外門硬功,身上若是受了他手指的一戮,不死也受重傷。

衛天望見郭襄臉上變色,甚是得意,抬頭哈哈大笑,這笑聲刺入耳鼓,直震得石亭頂上的瓦片也格格而響。

驀地裡咯喇一聲,石亭屋頂破裂,掉下一大塊物事來。衆人都吃了一驚,連衛天望自己也是大出意料之外。他運足内力,發出笑聲,方能震動屋瓦,其實這笑聲中殊無愉快之意,只不過是運功叫喊幾聲「哈哈,哈哈」而已。居然能震破屋頂,不由得驚喜交集。再看那掉下來的物事時,更是一驚,只見一個身穿白衣的中年漢子,雙手抱著一張瑤琴,躺在地下,兀自閉目沉睡。

郭襄喜道︰「喂,你在這児啊!」原來此人正是數日前她在山坳中所遇見的那個撫琴自奕的男子。

那人閉著眼睛幽幽吟道︰「老冉冉其將至矣,恐修名之不立!」一睜眼見到郭襄,跳起身來,説道︰「姑娘,我到處找你,却不道又在此間邂逅。」

郭襄道︰「你找我幹什麼?」

那人道︰「我忘了請教姑娘的高姓大名。」

郭襄道︰「什麼高姓大名?文縐縐,酸溜溜的,我最不愛聽。」

那人一怔,笑道︰「{\upstsl{嗯}},不錯,不錯!越是鬧虛文、擺架子,越是没有眞才實學,這種人去混騙鄕巴老児,那是最妙不過。」説罷雙眼瞪看衛天望,嘿嘿冷笑。

郭襄大喜,想不到此人如此知趣,這般幫著自己。衛天望給他這麼雙眼一瞪,一張鐵青的臉更加青了,冷冷的道︰「尊駕是誰?」

那人竟不理他,對郭襄道︰「姑娘,你叫什麼名字?」

郭襄道︰「我姓郭,單名一個襄字。」

那人鼓掌道︰「啊,當眞是有眼不識泰山,原來便是四海聞名的郭大姑娘。令尊郭靖郭大俠,令堂黃蓉黃女俠,除了無知無識之徒,不明好歹之輩,江湖上誰人不知,那個不曉?他二人文武雙全,刀槍劍戟,拳掌氣功,琴棋書畫,詩詞歌賦,無一不是凌駕古今,冠絶當時。哈哈,偏有一干妄人,竟爾不知他二位的名頭。」

郭襄心中一樂︰「原來你躱在石亭頂上,早聽到了我和這三人的對答。看來你其實也不知我爹娘是何等樣人,我行二,却叫我郭大姑娘,又説我爹爹會得琴棋書畫、詩詞歌賦,眞是笑話奇談了。」於是説道︰「那你叫什麼名字啊?」

那人道︰「我姓何,名字叫作『足道』。」

郭襄笑道︰「何足道,何足道,這名字倒是謙遜得很。」

何足道道︰「比之天什麼、地什麼的大言不慚、妄自尊大的小子,區區的名字還算不易令人作嘔。」

何足道一直對衛天望等三人不絶口的冷嘲熱諷。那三人見他壓破亭頂而下,顯非等常,初時尚自忍耐,要瞧瞧這個白衣怪客到底是什麼來歷,但他言語越來越是刻薄,衛天望再也按捺不住,反手便往何足道頰上一掌打去。

何足道頭一低,從他手臂底下鑽過。衛天望只覺左腕上微微一麻,手中持著的短劍已給他挾手奪去。衛天望搶奪郭襄的短劍之時,身法奇快,令人無法看清,但何足道這一下却是飄然而過,輕描淡冩的便將短劍隨手取了過來,身法手勢,均無什麼特異之處。

衛天望一驚,搶步而上,出指如鉤,便往何足道肩頭抓到。何足道斜身略避,衛天望這一抓登時便落了空。

潘天耕和方天勞突然間倒躍出亭,衛天望左拳右掌,風聲呼呼,霎時之間打出了七八招,但何足道左閃右避,竟連衣角也没給帶到半點。他手中捧著短劍,對敵人猶如暴風驟雨般的拳招始終不招不架,只是微微一側身,衛天望的拳招便落了空。

郭襄限於年歳,武功雖不甚精,但她自幼所見所處,都是當世第一流的武學高手,見識是極高的,見何足道舉重若輕,以極巧妙的身法,閃避極剛猛的敵招,這等武功身法又是另成一家,和中土各家各派成名的武學均自不同,不由得越看越奇。

衛天望連發二十餘招,兀自不能逼得他出手,喉頭間猛地一聲低{\upstsl{嘄}},拳法忽變,出招遲緩,但拳力却是凝重強勁。

郭襄站在亭中,漸覺拳風壓體,於是一步步的退到亭外。

何足道不敢再行只閃撲而不還招,將短劍往腰間一插,雙足一站,身子登時如淵停嶽峙,喝道︰「你會硬功,難道我便不會麼?」待衛天望雙掌推到,左手反臂一掌,以硬功對硬功,砰的一聲,衛天望身子一晃,倒退了兩步,何足道却是站在原地不動。

衛天望自恃外門硬功當世少有敵手,豈知對方硬碰硬的反擊,毫不借勢取巧,竟以硬功將自己震退。他心中不服,吸一口氣,大喝一聲,又是雙掌劈了過來。

何足道也是一聲猛喝,反擊一掌,喀喇喇響聲過去,只震得亭子頂上的破洞中泥沙亂落。

衛天望退了四步,方始拿樁站住。他對了這兩掌,頭髮蓬亂,雙睛凸出,模樣甚是可怖,雙手抱著丹田,運了幾口氣,只見他胸口陥入,肚子脹起有如皮鼓,全身骨節格格亂響,一步步的向何足道緩緩走來。

何足道見了他這等聲勢,知他這一擊之中,將顯示畢生功力,却也不敢怠慢,調勻眞氣,以待敵勢。

衛天望慢慢走到了離何足道身前四五尺之處,本該發招,可是他仍不停步,又向前走了兩步,直到與何足道面對而立,幾乎呼吸相接,這纔雙拳驟起,一掌擊向敵人面門,另一掌却按向對方小腹。這一次他雙掌錯擊,要令何足道力分而散,招勢掌力,兩臻絶妙。

何足道也是雙掌齊出,交叉著左掌和他左掌相接,右掌和他右掌相接,但掌力之中,却是分出一剛一柔。

衛天望但覺擊向他小腹的一掌如著無物,擊他面門的右掌却似碰到了銅牆鐵壁,心下甫覺不妙,猛地裡一股巨力撞來,推著他的身子直送出石亭之外。

這一下仍是硬碰硬的對力,力弱者傷,中間實無絲毫迴旋的餘地,不論衛天望拿樁站定,或是一交摔倒,他自己的掌力反擊回來,再加上何足道的掌力,定須迫得他口噴鮮血。

潘天耕和方天勞齊聲叫道︰「出掌!」兩人相對著各推出一掌,兩股掌力構成一道軟牆,衛天望跌出來時背心在那掌力的氣流上一靠,這纔不致受傷,但五臟翻動,全身骨骼如欲碎裂,一口氣緩不過來,登時委頓不堪。

那紅臉矮子方天勞見衛天望竟吃了這般大的苦頭,心下暗自驚怒,但臉上仍是笑嘻嘻的説道︰「閣下掌力之強,眞乃世所少見,佩服佩服。」

郭襄心想︰「説到掌力的剛猛渾厚,又有誰能及得爹爹的降龍十八掌?你們這崑崙三聖僻處荒山,井底觀天,夜郎自大,總有一日叫你們見識見識中土人物。」她言念及此,心中驀地一酸,原來這時她想到要方天勞等見識的中土人物,竟不是她父親而是楊過。只聽方天勞又道︰「小老児不才,再來領教領教閣下的劍法。」

何足道道︰「方兄對郭姑娘很是客氣,在下可没怪你,咱們不用比了。」

郭襄一怔︰「你給那姓衛的吃這番苦頭,原來是爲了他對我不客氣?」

方天勞走到坐騎之旁,從布囊中取出一柄長劍,刷的一響,拔劍出鞘,伸指在劍身上一彈,{\upstsl{嗡}}{\upstsl{嗡}}之聲,良久不絶。他一劍在手,笑容忽斂,左手捏個劍訣,平推而出,訣指上仰,右手劍朝天不動,正是那一招「仙人指路」。

何足道道︰「師兄既要逼我動手,就拿郭姑娘這柄短劍跟你試幾招。」説著抽出半截短劍。那短劍本不過二尺來長,給衛天望以指截斷後,劍刃只餘下七八寸,而且平頭無鋒。連匕首也不像。他左手仍舊拿著劍鞘,右手舉起半截斷劍,斗然搶攻。這一下出招快極,方天勞眼前白影一閃,何足道已連攻三招,雖因斷劍太短,傷不著他,但方天勞已自暗暗心驚,心想︰「這三招來得好快,眞是教人猝不及防,那是什麼劍法?他手中拿的若是長劍,只怕此刻我已血濺當場。」

何足道三招一過,向旁竄開,凝立不動,方天勞當即展開劍法,半守半攻,如遊龍般搶到。何足道閃身相避,只不還手。突然百忙中又快攻三招,逼得方天勞手忙足亂,他却又已縱身躍開。

方天勞怒氣漸增,一柄劍使將開來,白光閃閃,莫瞧他身材矮小,這劍法上的造詣却果眞不低。

郭襄心道︰「這老児招數剛猛狠辣,和那姓衛的掌法是同一條路子,只是帶了三分靈動之氣,却更加厲害些\dash{}」正想到此處,忽聽得何足道喝道︰「小心了!」一個「了!」字剛脱口,但見他左手劍鞘一舉,快逾電光石火,撲的一聲輕響,已用劍鞘套住了方天勞長劍的劍頭,右手斷劍跟著遞出,直指他的咽喉。

方天勞長劍不得自由,無法迴劍招架,眼睜睜的瞧著斷劍抵向自己咽喉,只得撇下長劍,就地一滾,纔閃開了這一招。

方天勞滾在一旁,尚未躍起,但見人影一閃,潘天耕已縱身過來,抓住長劍劍柄,一抖一抽,脱出劍鞘,何足道與郭襄同時喝了聲采︰「好身法!」這臉有病容的老頭始終不發一言,想不到武功竟是三人之首。何足道道︰「閣下好功夫,在下甚是佩服。」回頭向郭襄道︰「郭姑娘,自從日前聞你雅奏,我作了一套曲子,想請你品評品評。」郭襄道︰「什麼曲子啊?」何足道盤膝坐下,將瑤琴放在膝上,理絃調韻,便要彈琴。潘天耕道︰「閣下連敗我兩位師弟,姓潘的還欲請教。」何足道搖手道︰「武功比試過了,没有什麼餘味。我要彈琴給郭姑娘聽,這是一首新曲,你們三位愛聽,便請坐著,若是不懂,尚請自便。」於是左手按節撚絃,右手彈了起來。

郭襄只聽了幾節,不由得又驚又喜,她自聽瑤琴以來,從未聽過如此古怪的曲子。原來這琴曲的一部分是自己奏過的「考槃」,另一部分却是秦風中的「蒹葮」之詩,兩個截然不同的調子,給他别出心裁的混和在一起,一應一答,説不出的奇妙動聽,但聽那琴韻中奏著︰「考槃在澗,碩人之寬。蒹葮蒼蒼,白露爲霜,所謂伊人,在天一方\dash{}碩人之寬,碩人之寬\dash{}朔迴從之,道阻且長,朔遊從之,宛在水中央\dash{}獨寐寤言,永矢勿諼,永矢勿諼\dash{}」郭襄聽到這裡,心中驀地一動︰「他琴中説的『伊人』,難道是我麼?這琴曲何以如此纏綿,充滿了思慕之情?」想到此處,不由得臉上微微一紅。只是這琴曲實在編得巧妙,「考槃」和「蒹葮」兩首曲子的原韻絲毫不失,相互參差應答,却大大的豐富華美起來。

潘天耕等三人却聽得半點不懂,他們不知何足道爲人疏狂,性格中帶著三分書獃子的痴氣,既編了一首新曲,便巴巴的趕來要郭襄欣賞,何況這曲子也確實是爲她而編,於是將眼前的大事也抛在腦後。但見他凝神彈琴,竟没將自己三人放在眼裡,顯是對自己輕視已極,此可忍孰不可忍?潘天耕長劍一指,點向何足道左肩,喝道︰「快站起來,我跟你比劃比劃。」何足道全心沉浸在琴聲之中,當眞是神遊物外,似乎見到一個狷介的狂生在山澤之中漫遊,遠遠望見水中小島間站著一個溫柔的少女。於是不辭山遠水長,一股勁児的過去見她。

忽然間左肩上一痛,他登時驚覺,抬頭一看,原來潘天耕手中長劍指著他肩頭,輕輕刺破了一點児皮膚,如再不招架,只怕他便要挺劍傷人,但這一曲尚未彈完,俗人在旁相擾,實在大煞風景,當下抽出半截斷劍,{\upstsl{噹}}的一聲,將潘天耕的長劍架開,右手却仍是撫琴不停。這當児何足道終於顯出了生平絶技,他一手彈琴,一手使劍,無法再行按絃,於是對著第五根琴絃運氣一吹,那琴絃便低陥下去,竟與用手按捺一般無異,右手彈奏,琴聲中自也分出宮商角徵羽五音,高下低昂,無不宛轉如意。

潘天耕急攻數招,何足道順手應架,雙眼只是凝視琴絃,緊恐一口氣吹的部位不合,亂了琴韻。潘天耕愈怒,劍招越攻越急,但不論長劍刺向何方,總是給他輕描淡冩的擋開。郭襄聽著琴聲,心中樂音流動,對潘天耕的仗劍也没在意,只是雙劍相交{\upstsl{噹}}{\upstsl{噹}}之聲,擾亂了琴聲。她雙手輕輕擊掌,打著節拍,皺眉對潘天耕道︰「你出劍忽徐忽疾,難道半點不懂音韻嗎?喏,你聽著這節拍出劍,一拍一劍,那麼夾著琴聲之中就不會難聽。」潘天耕如何理她。眼見敵人坐在地下,單掌持著半截斷劍,眼光向自己瞧也不瞧,但自己兀自奈何不了他,更是焦躁起來,斗然間劍法一變,一輪快攻,兵刃相交的{\upstsl{噹}}{\upstsl{噹}}之聲,登時便如密雨。

\chapter{劃石爲局}

這繁絃急管一般的聲音,和那溫雅纏綿的琴韻決不諧和,何足道雙眉一挑,勁傳斷劍,錚的一響,潘天耕手中長劍登時斷爲兩截,但就在此時,七絃琴上的第五絃也應聲崩斷。潘天耕臉如死灰,一言不發,轉身出亭。三個人跨上馬背,向山上疾馳而去。郭襄甚是奇怪,説道︰「咦,這三人打了敗仗,怎地還上少林寺去?當眞是要死纏到底麼?」一回頭,却見何足道滿臉沮喪,撫著那根斷絃,似乎説不出的難受。郭襄心想︰「斷了一根琴絃,那又算得什麼?」當下接過瑤琴,解下半截斷絃,放長琴絃,重行繞柱調音。何足道嘆道︰「七年修爲,終是心不能靜。我左手斷他兵刃,右手却將琴絃也斷了。」郭襄這纔明白,原來他只是懊悔自己武功未純,笑道︰「你想左手凌厲攻敵,右手舒緩撫琴,這是分心二用之法,當今之世只有三人能彀。你没練到這個地步,那也用不著氣沮啊。」

何足道道︰「是那三位?」郭襄道︰「第一位老頑童周伯通,第二位便是我爹爹,第三位是楊夫人小龍女。除他三人之外,就算我外公桃花島主、我媽媽、神鵰大俠楊過等武功再高之人,也不能彀。」何足道道︰「世間居然有此奇人,幾時你給我引見引見。」郭襄黯然道︰「要見我爹爹不難,其餘那兩位哪,可不知到何處去找了。」但見何足道惘然出神,兀自想著適纔斷絃之事,安慰他道︰「你一舉擊敗崑崙三聖,也足以傲視當世了,何必爲了崩斷琴絃的小事鬱鬱不樂?」何足道矍然而驚,道︰「崑崙三聖?你説什麼?你怎麼知道?」郭襄笑道︰「那三個老児來自西域,自是崑崙三聖了,他們的武功果然各有獨到之處,只是要向少林寺挑戰,總嫌有些不自量\dash{}」

只見何足道驚訝的神色愈來愈盛,不自禁的住口不言,問道︰「有什麼奇怪啊?」何足道喃喃的道︰「崑崙三聖,崑崙三聖何足道,那便是我啊。」郭襄吃了一驚,道︰「你是崑崙三聖?那麼其餘兩個呢?」何足道道︰「崑崙三聖只有一人,從來就没三個。我在西域闖出了一點小小名頭,當地的朋友説我琴劍棋三絶,可以説得上是琴聖、劍聖、棋聖。因爲我長年住於崑崙山中,是以給了我一個外號,叫作『崑崙三聖』。但我想這個『聖』字,豈是輕易稱得的?雖然别人給我臉上貼金,也不能自居不疑,因此上我改了自己的名字,叫作『足道』,聯起來説,便是『崑崙三聖何足道』,人家聽了,便不致説我狂妄自大了。」

郭襄拍手笑道︰「原來如此,我只道既然是崑崙三聖,定然是三個人。那麼剛纔這三個老児呢?」何足道道︰「他們麼?他們是少林派的。」郭襄更是奇怪,道︰「原來這個老頭反而是少林弟子,{\upstsl{嗯}},他們的武功果然是剛猛一路,不錯,不錯,那紅臉老頭使的可不是達摩劍法?對啦,那黃臉病夫最後一輪急攻,却不是韋陀伏魔劍?只是他加了許多變化,一時之間没瞧出來?怎麼他們又是從西域來?」

何足道説道︰「這件事説起來有個緣故。去年春天,我在崑崙山驚神峰絶頂彈琴,忽聽得茅屋外有毆擊之聲,出去一看,只見兩個人扭在一團,身上各受致命重傷,却兀自竭力拚鬥。我喝他們住手,兩人誰也不肯罷休,於是我將他們拆解開來。其中一人白眼一翻,登時死了,另一個却還没斷氣。於是我將他救回屋中,給他服了一粒少陽丹,救治了半天,終於他受傷太重,靈丹無法續命。他臨死之時,説他名叫尹克西\dash{}」郭襄「啊」的一聲,道︰「那個跟他鬥毆的,莫非是瀟湘子?那人身形瘦長,臉容便似僵屍一般,是麼?」何足道奇道︰「是啊,怎地你什麼也知道?」

郭襄笑道︰「我也見過他們的,想不到這對活寶,最後終於相互毆死。」何足道道︰「那尹克西説,他一生作惡多端,臨死之時心中懊悔,却也遲了。他説他和瀟湘子從少林寺中盜了一部經書出來,兩人互相防範,誰也不放心讓對方先看,生怕對方學強了武功,便下手將自己除去,獨霸這部經書。兩人同桌而食,同床而睡,當眞是寸步不離。但吃飯時生怕對方下毒,睡覺時擔心對方暗算,提心吊膽,魂夢不安,只怕少林寺的和尚追索,於是遠遠逃向西域。到得驚神峰上之時,兩人已是筋疲力盡,正知再這般下去,不出十日,生生的便會累死,終於出手打了起來。尹克西説,那瀟湘子武功本來在他之上,那知雖是瀟湘子先動手打了他一掌,結果反而是他略佔上風。後來他纔想起,瀟湘子在峰上受了重傷,元氣始終不復,若不是兩人各有所忌,也挨不到崑崙山中了。」

郭襄聽了這番話,想像那二人一路上心驚肉跳,死挨苦纏的情景,不由得悔然生憐,嘆道︰「爲了一部經書,也不値得如此啊。」何足道道︰「那尹克西説了這番話,已是上氣不接下氣,他最後託我來少林寺走一遭,要我跟寺中一位覺遠和尚説,説什麼經書是在油中。我聽得奇怪,什麼經書是在油中?欲待再問詳細,他已支持不住,暈了過去。我準擬待他好好睡一覺,醒過來再問端詳,那知他這一睡就没再醒。我想莫非那部經書包在油布之中?但細搜二人身邉,却是影蹤全無。受人之託,忠人之事,我平生足跡未履中土,正好乘此遊歷一番,於是便來少林寺走一遭。」

郭襄道︰「那你怎地又到少林寺中去下戰書,説要跟他們比試武藝。」何足道微笑道︰「這事却是從適纔這三人身上而起了。這三人是少林寺的俗家子弟,據西域武林中的人説,他們都是天字輩,和此時少林寺的方丈天鳴禪師是同輩。好像他們的師祖從前和寺中的師兄弟鬧了意見,一怒而遠赴西域,傳下了少林的西域一支。本來嘛,少林的武功是達摩祖師自天竺傳到中土,再從中土分到西域,那也没什麼希奇。這三人聽到了我『崑崙三聖』的名頭,要來跟我比劃比劃,一路上揚言説什麼少林寺武功天下無敵,我號稱琴聖、棋聖那也罷了,這『劍聖』兩字,他們却萬萬容不得,非逼得我去了這名頭不可。正好這時碰上了尹克西的事,我想反正要上少林寺來,索性到少林寺來鬥上一鬥,於是迴避不見他們,自行到中原來啦。這三位脚程也眞快,居然陰魂不散的也趕到了。」

郭襄笑道︰「此事原來如此,可全教我猜岔了。三個老頭児這時候児回到了少林寺,不知説些什麼?」何足道道︰「我跟少林寺的和尚們素不相識,又没過節,所以跟他們訂約十天,原是要待這三個老児趕到,這纔動手。現下架也打過了,咱們一齊上去,待我去傳了這句話,便下山去吧。」郭襄皺眉道︰「和尚們規矩大得緊,不許女人進寺。」何足道道︰「{\upstsl{呸}}!什麼臭規矩?咱們偏偏闖進去,還能把人殺了?」郭襄原是個好事之人,但既已和無色禪師結交,對少林寺已無敵意,搖頭笑道︰「我在山門外等你,你自己進寺去傳言,省了不少麻煩。」何足道點頭道︰「就是這樣,剛纔的曲子没彈完,回頭我好好的彈一遍給你聽。」

當下兩人緩步上山,直走到寺門外,竟是不見一個人影,何足道道︰「我也不進去啦,請那和尚出來説句話就是了。」於是朗聲道︰「何足道造訪少林寺,有一言奉告覺遠大師。」這句話剛説完,只聽得寺内十餘座巨鐘一齊鳴了起來,{\upstsl{噹}}{\upstsl{噹}}之聲,只震得群山齊應。

突見寺門大開,分左右走出兩排身穿灰袍的僧人,左邉五十四人,右邉五十四人,一共是一百零八人,那是羅漢堂的弟子,合一百零百八名羅漢之數。其後跟著出來十八名僧人,灰袍上罩著淡黃袈裟,年歳均較羅漢堂弟子爲大,却是高了一輩的達摩堂弟子。稍隔片刻,出來七位身穿大塊格子僧袍的僧人,這七位僧人皺紋滿面,年紀少的也已七十餘歳,老的已達九十高齡,乃是心禪堂的七老,這七老輩份甚高,有的身懷絶技,有的却是全然不會武功,只是佛學精湛,少林寺中連方丈也對他們十分尊敬。最後是方丈天鳴禪師緩步而出,左是達摩堂首座無相禪師,右是羅漢堂首座無色禪師。潘天耕、方天勞、衛天望三人跟在其後。在三人身後另有七八十名俗家弟子。

少林寺這等隆重的迎接來客,可説是極爲罕有,過去縱然是官府大員或是名重武林的豪俠到來,也不過是方丈和無色、無相親自出迎而已,心禪堂七老是決計不見外客的。原來那日何足道悄入羅漢堂,在降龍羅漢手中留下簡帖,這份武功已令方丈及無色、無相等大爲震驚。數日後潘天耕等自西域趕到,説起此事,寺中各位高僧更增了一層戒心。要知西域少林一支因途程遙遠,百餘年來極少和中州少林互通音問,但寺中衆高僧均知,當年遠赴西域支派的那位師叔祖,武功上實有驚人的造詣,他傳下的徒子徒孫自亦不同凡響。這時聽潘天耕等言語之中,對崑崙三聖絲毫不敢輕視,眞所謂善著不來,來者不善,寺中外觀上一如其常,不動聲色,但暗中却防範得極是嚴謹,並傳下師旨,五百里以内的僧俗弟子,一律歸寺聽調。初時衆人也道崑崙三聖乃是三個老頭,後來聽潘天耕等一説,方知只是一人,至於容狀年紀,潘天耕等也毫不知情,只知他自負琴劍棋三絶而已。彈琴奕棋兩項,馳心逸性,這是禪宗所忌,少林寺衆僧是向來不理的,這數日之中,凡是精於劍術的高手,無不加緊磨練,要和這個號稱「劍聖」的狂人一較高下。

潘天耕師兄弟自忖此事由自己身上而起,自當由自己手裡了結,因此每日騎了駿馬,在山前山後巡視,一心要攔住這個崑崙三聖,打得他未進寺門,先就望風披靡,然後再回到寺中,和各僧侶較量一下,瞧中州、西域兩派,到底是誰強誰弱。那知石亭中一戰,何足道只出半力,已令三人鍛羽而遁。天鳴禪師一得訊息,心知今日少林寺已面臨榮辱盛衰的大関頭,「天下武學之源」這千年來的令譽,決不能在自己手中毀却,但心忖自己和無色、無相的武功,未必能強於潘天耕等三人多少,這纔不得不請出心禪堂七老來押陣。但心禪七老的功夫到底深到了何等地步,誰也不知,是否眞能在緊急當中出手制得住這崑崙三聖,在方丈和無色、無相三人心中,也只是胡亂猜測罷了。

且説老方丈見到何足道和郭襄,合什説道︰「這一位想是號稱琴劍棋三絶的何居士了,老僧未能遠迎,還乞恕罪。」何足道躬身行禮,説道︰「晩生滋擾寶刹,甚是不安,驚動衆位高僧出寺相迎,更是何以克當?」天鳴心道︰「這狂生説話倒不狂啊。瞧他不過三十歳左右年紀,怎能有如此深厚的内功?」於是又道︰「何居士不用客氣,請進奉茶。這位女居士嘛\dash{}」言下頗有爲難之色。何足道一聽他言中之意是要拒絶郭襄進寺,狂生之態陡然顯露,仰天大笑,説道︰「老方丈,晩生到寶刹來,本是受人之託,來傳一句言語。這句話一説過,原想拍手便去。但寶刹重男輕女,莫名其妙的清規戒律未免太多,晩生頗有點看不過眼。須知佛法無邉,衆生如一,妄分男女,心有滯礙。」

天鳴禪師是位有道高僧,心地明澈,寬博有容,一聽何足道之言,微笑道︰「多謝居士指點,我少林寺強分男女,倒顯得小氣了。如此請郭姑娘一併光降奉茶。」郭襄向何足道一笑,心想︰「你這張嘴倒會説話,居然片言折服老和尚。」見天鳴方丈向旁一讓,伸手肅客,正要舉步進寺,忽見天鳴左首一個乾枯精瘦的老僧踏上一步,説道︰「單憑何居士一言,便欲我少林寺捨棄千年來的規矩,雖無不可,却也要瞧説話之人是否眞是大有本事,還是只不過浪得虛名。何居士請留一手,讓衆僧侶開開眼界,也好令合寺心服,知道本寺行之千年的規矩,是由誰而毀。」這人正是達摩院首座無相禪師,他説話聲音洪亮,聽得各人耳鼓中{\upstsl{嗡}}{\upstsl{嗡}}作響,顯見中氣充沛,内力深厚。潘天耕等三人聽了,臉上微微變色。要知無相這幾句話中,顯是含有瞧不起潘天耕等三人之意,謂何足道雖然擊敗三人,却也未必便有過人的本領。

郭襄見無色禪師臉帶憂容,心想這位老和尚爲人很好,又是大哥哥的朋友,倘若何足道和少林寺僧衆爲了我而爭鬥起來,不論那一方輸了,我都要過意不去,於是朗聲説道︰「何大哥,我又不是非進少林寺不可,你傳了那句話,這便去罷。」她伸手指著無色道︰「這位無色禪師是我好朋友,你們兩家不可傷了和氣。」何足道一怔,道︰「啊,原來如此。」他轉向天鳴,説道︰「老方丈,貴寺有一位覺遠禪師,是那一位?在下受人之託,有句話要轉告於他。」天鳴低聲道︰「覺遠禪師?」原來覺遠在寺中地位低下,數十年來隱身藏經閣中埋頭讀書,没没無聞,從來没人在他法名之下加上「禪師」兩字,是以天鳴一時竟没想到。他呆了一呆,纔道︰「啊,看守楞伽經失職的那人。何居士找他,可是與楞伽經一事有関麼?」

何足道搖頭道︰「我不知道。」天鳴向一名傳事的弟子道︰「叫覺遠來見客。」那弟子領命匆匆而去。

無相禪師又道︰「何居士號稱琴劍棋三聖,想這『聖』之一字,豈是常人所敢居?何居士於此三者自有冠絶天人的造詣。日前留書敝寺,説欲顯示武功,今日既已光降,可肯不吝賜教,讓咱們瞻仰瞻仰絶技?」何足道搖頭道︰「這位姑娘既已説過,咱們兩家便不可傷了和氣。」無相怒氣勃發,心想你留書於先,事到臨頭,却來推託,千年以來,有誰敢對少林寺如此無禮?何況潘天耕等三人敗在你的手下,江湖上傳言出去,説是少林派的大弟子輸了給你,你這「劍聖」兩字,豈不是叫得更加響了?看來一般弟子也不是他的對手,非自己親自出馬不可。當下踏上兩步,説道︰「比武較量,也不是傷了和氣,何居士何必推讓?」回頭向達摩堂的弟子喝道︰「取劍來!咱們領教領教『劍聖』的劍術,到底聖到何等地步?」

寺中各種兵刃,早已備妥,只是列隊迎客之際,不便取將出來,以免徒嫌小氣,那弟子聽無相一吩咐,一轉身便取了七八柄長劍出來,雙手橫托,送到何足道身前,説道︰「何居士是使自擕的寶劍?還是借用敝寺的尋常兵刃?」何足道不答,俯身拾了一塊石子,突然在寺前的青石板上縱一道、橫一道的畫了起來,頃刻之間,畫成了縱橫各一十九道的一張大棋盤。經緯界線筆直,猶如用界尺界成一般,每一道線都是深入石板一寸有奇。這石板乃是以少室山的青石舖成,堅硬如鐵,數百年人來人往,從未磨耗半點,他隨手用一塊石子揮畫,竟然深陥盈寸,這份内功,少林寺中看來無人能及,只聽他笑道︰「比劍太嫌霸道,琴音無法比併,大和尚既然高興,咱們便下一局棋如何?」

這一手裂石爲局的驚人絶技一露,天鳴、無色、無相以及心禪堂七老,無一面面相覷,心下駭然。天鳴禪師知道如此渾雄的内力,寺中無一人及得,他心地光風霽月,對勝負也不如何介懷,正要開口認輸,忽聽得鐵鍊拖地之聲,叮{\upstsl{噹}}而來,只見覺遠挑著一對大鐵桶,後面隨著一個長身少年,走到天鳴跟前。覺遠單掌行禮,説道︰「謹奉老方丈呼召。」天鳴道︰「這位何居士要見你,有句話跟你説。」覺遠回過身來,一看何足道,却不相識,説道︰「小僧便是覺遠,居士有何吩咐?」

何足道畫好棋局,棋興勃發,説道︰「這句話慢慢再説不遲,那一位大和尚先跟在下對奕一局?」其實他倒不是有意炫示功夫,只是此人生平對琴劍棋三項,都是愛到發痴,興之所到,連天榻下來也是置之度外,既然想到奕棋只求有人放對,早忘了比試武功之事。天鳴禪師道︰「何居士畫石爲局,如此神功,老衲生平未見,敝寺僧衆,盡皆甘拜下風。」覺遠聽了天鳴之言,再看一眼青石板上的大棋局,這纔知眼前此人竟是來寺獻示武功,當下不動聲色,挑著那擔大鐵桶,吸一口氣,將畢生所練的勁力都下沉雙腿,在那棋局的界線上一步步的走了過去。

只見他脚上鐵鍊拖過之處,石板上竟被他拖出一條五寸來寬的痕子,何足道所畫的界線,登時隨抹隨滅。衆僧一見,忍不住大聲喝起采來。天鳴、無色、無相等人,更是驚喜交集,那想得到這個痴痴呆呆的老僧,身上竟有這等深厚的内功,自己和他同居一寺數十年,却没瞧出半點端倪。要知一人内力再強,欲在石板上踏出印痕,本是決不可能,只因覺遠肩頭挑了一對大鐵桶,桶中裝滿了水,何慮五六百斤之重,這幾百斤的巨力以他肩頭傳到脚上的鐵鍊,向前拖曳,便如一把大鑿子在石板上敲鑿一般,這纔能鏟去何足道所刻的界線,倘若覺遠空身而行,那便萬萬不能了。但雖是有力可借,終也是極罕見的神功。

何足道不待他鏟完縱橫一共三十八道的界線,大聲喝道︰「大和尚,你好深厚的内功,在下可不及你!」覺遠鏟到此時,丹田中眞氣雖是愈來愈盛,但兩腿終是血肉之物,早已大感酸痛,聽他這麼一喝,當即止步,微笑吟道︰「一坪袖手將置之,何暇爲渠分黑白?」何足道道︰「不錯!這局棋不用下,我已是輸了。我領教領教你的劍法。」説著刷的一響,從瑤琴底下抽出一柄長劍,劍尖指向自己胸口,劍柄斜斜向外,這一招起手式實是怪異之極,竟似迴劍自戕一般,天下劍法之中,絶無如此不通的一招。覺遠道︰「老僧只知唸經打坐,曬書掃地,武功一道可是一竅不通。」何足道却那裡肯信?嘿嘿一聲冷笑,縱身近前,那長劍斗然間彎彎彈出,劍尖直刺覺遠胸口,出招之快眞乃爲任何劍法所不及。原來這一招不是直刺,却是先聚内力,然後蓄勁彈出。

倘若換作尋常武師,何足道出劍雖快,便可一擊而中,但覺遠的内功,實已到了隨心所欲,收發自如的境界。何足道出劍雖快,覺遠的心念却動得更快,意到手到,身意合一,他右手一收,扁擔上的大鐵桶登時盪了過來擋在身前,{\upstsl{噹}}的一聲,那劍尖竟是刺在鐵桶之上。劍身柔韌,彎成了個弧形。何足道急收長劍,隨手揮出,覺遠左手的鐵桶橫過,又擋開了。

何足道心想︰「你武功再高,這對鐵桶總是笨重之極的東西,焉能擋得住我的快攻?倘若你空手對招,我反而有三分忌憚。」只見他伸指在劍身上一彈,長劍聲若龍吟,叫道︰「大和尚,可小心了!」長劍顫處,前後左右,瞬息之間攻出了四四一十六招。

但聽得{\upstsl{噹}}{\upstsl{噹}}{\upstsl{噹}}{\upstsl{噹}}一共一十六下響過,何足道這一十六手「迅雷劍」,竟是盡數刺在鐵桶之上。旁觀衆人見覺遠手忙脚亂,左支右絀,果然是不會半分武功,顯得狼狽之極,但説也奇怪,何足道這一十六下神妙無方的劍招,竟是連覺遠的衣角也碰不到半點,全都給他胡裡胡塗,以極笨拙極可笑的方法用鐵桶擋了開去。無色、無相等一看,都不禁擔心,齊聲叫道︰「何居士劍下留情!」郭襄也道︰「休下殺手!」

眞所謂當局者迷,旁觀著清,衆人都瞧出覺遠不會武功,但何足道身在戰局之中,自己竭盡全力施展,竟爾奈何不了對方半分,那會想到他其實從未學過武功,所以能擋住劍招,全仗他在不知不覺中所練成的上乘内功所致。何足道快擊無功,斗然間大喝一聲,劍挾寒光,一劍向覺遠小腹上直刺過去。覺遠叫聲︰「啊喲!」百忙中雙手一合,{\upstsl{噹}}的一聲巨響,兩隻大鐵桶竟將一柄長劍生生的挾住。何足道使勁一奪,却那裡動得半毫?他應變奇速,右手撤劍,雙掌齊推,一般排山倒海般的掌力,直撲覺遠的面門。

覺遠雙手提著鐵桶,挾住了對方長劍,那裡分得出手去抵擋?張君寶師徒情深,縱身撲上,使一招楊過昔年所教的「四通八達」,一掌斜擊何足道的肩頭。便在此時,覺遠的勁力已傳到鐵桶之中,兩道水柱從桶中飛出,也撲向何足道的面門。掌力和水柱一撞,水花四濺,潑得兩人滿身是水,何足道雙掌之力便就此卸去。他正自全力和覺遠比拚,顧不得再抵擋張君寶這一掌,{\upstsl{噗}}的一下,肩頭中掌。豈知張君寶小小年紀,掌法既奇,内力又是大爲深厚,何足道立足不定,向左斜退三步。覺遠叫道︰「阿{\upstsl{隬}}陀佛,阿彌陀佛,何居士饒了老僧罷!這幾劍直刺得我心驚肉跳。」説著伸袖抹去臉上水珠,急忙走在一邉。

何足道怒道︰「少林寺臥虎藏龍之地,果眞非同小可,連一個小小少年,竟也有這等身手。好小子,咱倆來比劃比劃,你只須接得我十招,何足道終身不履中土。」無色、無相等均知張君寶只是藏經閣中一個打雜的小厮,從未練過武功,剛纔不知如何陰錯陽差的推了他一掌,但説當眞動武,别説十招,只怕一招便會喪生在他掌底。無相昂然道︰「何居士此言差矣!你號稱崑崙三聖,武學修爲震古鑠今,如何能和這烹茶掃地的小厮動手?若不嫌棄,便由老僧接你十招。」何足道搖頭道︰「這一掌之辱,豈能便此罷休?小子,看招!」説著呼的一拳,便向張君寶胸口打去。這一半去勢奇快,他和張君寶站得又近,無色、無相等便欲救援,却那裡來得及?衆人心中剛自暗暗叫聲苦,却見張君寶兩足足根不動,足尖左磨,身子隨之右轉,成右引左箭步,輕輕巧巧的卸開了他這一拳,跟著左掌握指變拳護腰,右掌切擊而出,正是少派基本拳法的一招「右穿花手」。

這一招氣凝如山,掌勢已出,有若長江大河,委實是名家耆宿的風範,那裡是一個少年人的身手?何足道自肩上受了他一掌,早知這少年的内力,遠在潘耕三人之上,只是自忖十招之内,定能將他擊敗,見他這招「右穿花手」雖是少林拳的入門功夫,但發掌轉身之際,雄渾沉穩,眞是無懈可擊,忍不住喝了聲采︰「好拳法!」

無相心念一動,向無色微笑道︰「無色師弟,恭喜你暗中收下了這樣一個得意弟子!」無色搖頭道︰「不是\dash{}」但見張君寶「拗步拉弓」「單鳳朝陽」「袖底切掌」「二郎擔衫」,連續四招,尺度之嚴,勁力之強,合寺僧人無出其右。

天鳴、無色、無相以及心禪七老見了張君寶這幾招少林拳打得如此神威凜凜,無不相顧駭然。無相驚道︰「他的拳法如此法度嚴謹,也還罷了,這等内勁\dash{}」説話之際,何足道已是出了第六招,心想︰「我連這黃口少年尚自對付不了,竟然敢到少林寺來留簡挑戰,豈不教天下英雄笑掉了牙齒?」突然滴溜溜轉個身子,一招「天山雪飄」,掌影飛舞,霎時之間將張君寶四面八方的裹住了。張君寶除了在華山絶頂受過楊過指點四招之外,從未有名師和他講解武功,陡然間見到這種奇幻百端,變化莫測的上乘掌法,那裡能彀拆解?危急之中,身腰左轉成寒雞勢,雙掌舉過額角,左手虎口與右手虎口遙遙相對,却是少林拳中的一招「雙圏手」。這一招凝重如山,敵招不解自解。不論何足道從那一個方位進襲,全是在他「雙圏手」籠罩之下。

猛聽得達摩堂、羅漢堂衆弟子轟雷也似的喝一聲采,眞是對張君寶這一招衷心欽服,讚他竟以少林拳中最平淡無奇的拳招,化解了最繁複奥妙的敵招。喝采聲中,何足道一聲清嘯,呼的一拳向張君寶當胸擊去。這一拳竟然也是自巧轉拙,却是勁力非凡。張君寶應以「偏花七星」,雙切掌推出。拳掌相交,只聽得砰的一聲,何足道身子一晃,張君寶却是向後退了三步。何足道「哼」的一聲,拳法不變,又是踏上一拳,硬擊硬打。張君寶所會的拳法有限,仍是應以一招「偏花七星」,雙切掌向前平推,砰的一聲猛響,張君寶這次退出了五步,何足道却是身子向前一撞,臉上變色,喝道︰「只剩下一招,你用全力接我一下。」踏上兩步,坐穩馬步,一拳緩緩擊出。這時少林寺前數百人聲息全無,人人皆知何足道這一拳實是他一生英名之所繫,自是竭了平生之力。

張君寶第三次再使「偏花七星」,這一次拳掌相交之時竟然無聲無息,兩人凝了一凝,在霎息間各自催動内力相抗。説到武功家數,何足道比之張君寶何止多出百倍?但一比併到内力,豈知張君寶在無意之間自「九陽眞經」學得心法,内力綿綿密密,渾厚充溢,竟是不知不覺間自臻極高的境界。兩人内力來回激盪數轉,何足道「嘿」的一聲,向後退了一步,一口熱血湧到心頭,想要強自忍住,但眼前一黑,終於還是噴了出來。張君寶不知適纔這一下竟會使他身受重傷,心下歉疚無已,「啊喲」一聲叫,奔開去便要扶他。何足道右手一揮,苦笑道︰「何足道啊何足道,當眞是狂得可以。」向天鳴禪師一揖到底,説道︰「少林寺武功揚名千載,果然是非同小可,今日得令狂生一開眼界,方知盛名之下,實無虛士。」説著轉過身來,足尖一點,已飄身在數丈之外。他停了停脚步,回頭對覺遠道︰「覺遠大師,那人叫我轉告的一句話,是什麼『經書是在油中』。」話聲甫歇,但見他足尖連點數下,已隱身在一列古柏之後,身法之快,武林中實是罕見。衆僧見他重傷之下居然仍能施展這等輕功,無不暗暗心驚。

大敵既去,衆僧一齊望著天鳴,聽他示下,心禪七老中一個精瘦骨立的老僧突然説道︰「這個弟子的武功是誰所授?」他説話聲音極是尖鋭,有若寒夜梟鳴,各人聽在耳裡,都是不自禁的打個寒噤。天鳴、無色、無相等心中均早存有這個疑問,一齊望著覺遠和張君寶。覺遠師徒却呆呆站著,一時説不出話來。天鳴道︰「覺遠内功雖精,未學拳法,那少年的少林拳,却是何人所授?」達摩堂和羅漢堂衆弟子均想,萬料不到今日本寺遭逢危難,竟是由這個小厮出頭,趕走強敵,老方丈定有大大的賞賜,而授他内功拳法的師父,也自必盛受榮寵。

\chapter{花落花開}

那老僧見張君寶呆立不語,斗然間雙眉豎起,滿臉殺氣,厲聲道︰「我是問你,你的羅漢拳是誰教的?」張君寶從懷中取出郭襄所贈的那對鐵羅漢,説道︰「弟子照著這兩個鐵羅漢所使的套子,依樣葫蘆的學幾手,實是無人傳授弟子武功。」那老僧踏上一步,聲音突然放低,説道︰「你再明明白白的説一遍︰你的羅漢拳並非本寺那一位師傅所傳授,乃是自己學的。」他語音雖低,但話中威嚇之意又增加了數倍,那是人人都聽得出來的。張君寶心中坦然,自忖並未做過什麼壞事,雖見那老僧神態咄咄逼人,却也不畏懼,朗聲道︰「弟子只在藏經閣中掃地烹茶,服侍覺遠師父,本寺並没有那一位師父傳過弟子武功。這羅漢拳是弟子自己學的,想是使得不對,還請老師父指點。」那老僧「哼」的一聲,目光中如欲噴出火來,狠狠的{\upstsl{盯}}著張君寶,良久良久,一動也不動。

覺遠知道這位心禪堂的老僧輩份甚高,乃是方丈天鳴禪師的師叔,見他對張君寶如此聲色倶厲,大是不解,但見他眼色之中充滿了怨毒,腦海中忽地一閃,疾似電光火石般,想起了不知那一年在藏經閣中偶然看到的一本小書。那是薄薄的一冊手抄本,書中記載著本寺的一樁門戸大事︰

距此七十餘年之前,少林寺的方丈是苦乘禪師,乃是天鳴禪師的師祖。這一年中秋,寺中例行一年一度的達摩堂大校,由方丈及達摩堂、羅漢堂的兩位首座考較合寺弟子的武功,看在過去一年之中,有何進境。豈知便在這一次中秋大校之中,發生了一樁慘變。各弟子獻技已罷,達摩堂首座苦智禪師升座,品評諸弟子武功,突然間一個帶髮頭陀越衆而出,大聲説道,苦智禪師的話狗屁不通,根本不知武功爲何物,竟然妄居達摩堂首席之位,甚是可恥。衆僧大驚之下,看這人時,原來是香積厨中灶下燒火的一個火工頭陀。達摩堂諸弟子自是不等師父開言,早已齊聲呵叱。那火工頭陀喝道︰「師父是狗屁不通,衆弟子更是不通狗屁。」説著湧身往堂中一站。衆弟子一一上前跟他動手,都被他三拳兩脚,便擊敗了。本來達摩堂中過招,同門較藝,自是點到爲止,人人手下留情。這火工頭陀却出手極是狠辣,他連敗達摩堂九大弟子,九個僧人不是斷臂便是折腿,無不身受重傷。首座苦智禪師又驚又怒,見這火工頭陀一身所學,全是少林派本門拳招,並非别家門派的高手故意混進寺來搗亂,當下強忍怒氣,問他的武功是何人所傳。那火工頭陀説道︰「無人傳授過我武功,是我自己學的。」原來這頭陀在灶下燒火,監管香積厨的僧人性子極是暴躁,動不動提拳便打,他又是身有武功之人,出手自重,那火工頭陀三年間給他打得接連吐血三次。積怨之下,火工頭陀暗中便去偸學武功。少林寺的弟子人人會武,他處心積慮的偸學,機會良多,一個人苦心孤詣的做一件事,所謂哀兵必勝,加上他又有過人之智,十餘年間給他練到了極上乘的武功。但他深藏不露,仍是不聲不響在灶下燒火,那監厨僧人拔拳相毆,他也總不還手,只是他内功已精,再也不會受傷了。他生性陰鷙,直到自忖武功已勝過全寺僧衆,這纔在中秋大校之日出來顯露身手。

數十年來的鬱積,使他恨上了全寺的僧侶,因此一出手毫不容情。苦智禪師問明原委,冷笑三聲,説道︰「你有這番苦心,委實可敬!」當下離座而起,伸手和他較量。這苦智禪師當時在少林寺中武功可算第一,兩人各出絶學,直鬥到五百合開外。

一來苦智禪師年事高,那火工頭陀方當壯年,二來苦智手下容情,火工頭陀使的却是招招殺手,兩人拆到一招「大纏絲」時,手足扭在一起,力強者勝,這中間已無絲毫假借餘地。苦智愛惜他潛心自習,居然有此造詣,雙掌一分,喝道︰「退開吧!」豈知那火工頭陀會錯了意,只道對方施展的是「神掌八打」中的一招。這「神掌八打」乃是少林武功中絶學之一,火工頭陀曾暗中瞧過羅漢堂的大弟子使過,見他雙掌劈出,打斷一根木樁,勁力非同小可。火工頭陀武功雖強,究竟全部偸學,未得明師指點,少林武功博大精深,他在十餘年中暗中窺看,豈能學得全了?苦智這一招其實是「分解掌」,借力卸力,雙方一齊退開,乃是停手罷鬥之意。火工頭陀却看錯成「神掌八打」中的第六掌「裂心掌」,心想︰「你要取我性命,却没如此容易。」飛身撲上,雙拳齊擊。

這雙拳之力如排山倒海般湧了過來,苦智禪師一驚之下,急忙回掌相抵,其勢却已不及,但聽得喀喇喇數聲,左臂臂骨和胸前四根肌骨一齊折斷。旁觀衆僧驚惶變色,一齊搶上救護,只見苦智内臟震傷,氣若游絲,一句話也説不出來。再看火工頭陀時,早已在混亂中逃得不知去向。當晩苦智便即傷重逝世。合寺悲戚之際,那火工頭陀又偸進寺來,將監管香積厨和平素和他有隙的五名僧人,一一使重手震死。合寺大震之下,派出幾十位高手四下追索,但尋遍了江南江北,竟是絲毫不得火工頭陀的蹤跡。寺中衆高僧更爲此事而大起爭執,互責互咎,羅漢堂首座苦慧禪師一怒而遠赴西域,開創了西域少林一派,潘天耕、方天勞等三人,便是苦慧的再傳弟子。

經此一役,少林寺的武學竟爾中衰數十年,自此定下寺規,凡是不得師授而自行偸學武功,發現後重則處死,輕則挑斷全身筋脈,使之成爲廢人。數十年來,因寺中防範嚴密,再也無人偸學武功,這條寺規衆人也漸漸淡忘了。這心禪堂的老僧正是當年苦智座下的小弟子,恩師慘死的情景,數十年來始終念念不忘,此時見張君寶又是不得師傅而偸學武功,觸動前事,自是悲憤交集。

覺遠在藏經閣中管經,無書不讀,猛地裡記得這樁舊事,不禁背上出了一背冷汗,叫道︰「老方丈,這\dash{}這須怪不得君寶\dash{}」一言未畢,只聽得達摩堂首座無相禪師喝道︰「達摩堂衆弟子一齊上前,把他拿下了。」達摩堂十八弟子習練有素,一聽首座令下,登時搶出,四面八方將覺遠和張君寶團團圍住。這十八人佔的方位甚大,連郭襄也圍在中間。那位心禪堂的老僧厲聲高喝︰「羅漢堂衆弟子,何以不併力上前?」羅漢堂的一百零八名弟子暴雷也似的應了聲︰「是!」又在達摩堂十八弟子之外,圍了三個圏子。

張君寶手足無措,還道自己出手打走何足道,乃是犯了寺規,説道︰「師父,我\dash{}我\dash{}」覺遠十年來和這徒児相依爲命,情若父子,知道張君寶只要一被擒住,便是僥倖不死,也必成了廢人,但聽得無相禪師喝道︰「還不動手,更待何時?」達摩堂十八弟子齊宣佛號,踏步而上。覺遠不暇思索,驀地裡轉了一個圏子,兩隻大鐵桶舞了開來,一股勁風,逼得衆僧人不能上前,跟著雙桶一側,左邉鐵桶兜起郭襄,右邉鐵桶兜起張君寶。他連轉七八個圏子,那對大鐵桶給他渾厚無比的内力使將開來,猶如流星鎚相似,這股千斤之力,天下誰能擋得?達摩堂衆弟子向旁一避,覺遠健步如飛,挑著張君寶和郭襄大踏步下山而去。衆僧人吶喊追趕,祇聽得鐵鍊拖地之聲漸去漸遠,追出七八里後,鐵鍊聲半點也聽不到了。

少林寺的寺規極嚴,達摩堂首座既然下令擒拿張君寶,衆僧人雖見追趕不上,還是鼓勇疾追。時候一長,各僧脚底便分出了高下,輕功稍遜的漸漸落後,追到天黑,領頭的只剩下了五名大弟子,眼前又出現了幾條岔路,也不知覺遠逃向了何方,此時便是追及,也決非覺遠、張君寶之敵,只得垂頭喪氣,回寺覆命。

且説覺遠擔一挑了兩人,直奔出百里之外,方才止步,只見所到之處是在一座深山之中。暮靄四合,歸鴉陣陣,覺遠内功雖強,這一陣捨命急馳,却也是筋疲力竭,一時之間,再也無力從肩頭將鐵桶卸下。張君寶與郭襄雙雙從桶中躍出,各人托起一隻鐵桶,從他肩頭放了下來。桶中還剩下小半桶水,兩人身上全已濕透。張君寶道︰「師父,你歇一歇,我去尋些吃的。」但在這荒山野地,那裡有什麼吃的,張君寶去了半日,只採得一大把草莓來。三人胡亂吃了,倚石休息。郭襄道︰「大和尚,我瞧少林寺那些僧人,都有點児古里古怪。」覺遠「{\upstsl{嗯}}」了一聲,並不答話。郭襄道︰「那個崑崙三聖何足道來到少林寺,寺中無人能敵,全仗你師徒二人將他打退,纔保全了少林寺的令譽。他們不來謝你,反而惡狠狠的要來捉拿張兄弟,這般的不分是非黑白,當眞是好没來由。」覺遠嘆了口氣,道︰「這事却也怪不得老方丈和無相師兄,少林寺有一條寺規\dash{}」説到這裡,一口氣提不上來,竟是咳嗽不止。郭襄輕輕替他搥背,説道︰「你累啦,且睡一忽児,明児慢慢再説不遲。」覺遠嘆了口氣,道︰「不錯,我也眞的累啦。」

張君寶拾些枯柴,生了個火,烤乾郭襄和自己身上的衣服,三人便在大樹之下睡了。郭襄睡到半夜,忽聽得覺遠喃喃自語,似在唸經,郭襄從朦朧中醒來,只聽他唸道︰「\dash{}彼之力方礙我之皮毛,我之意已入彼骨裡。兩手支撐,一氣貫穿。左重則左虛,而右已去,右重則右虛,而左已去\dash{}」郭襄心中一凜︰「他唸的並不是什麼『空却是色、色即是空』的佛經啊。什麼左重左虛、右重右虛,倒似是武學拳經。」只聽他頓了一頓,又唸道︰「\dash{}氣如車輪,週身倶要相隨,有不相隨處,身便散亂,其病於腰腿求之\dash{}」聽到「其病於腰腿求之」這句話,心下更無疑惑,知他唸的自是一部武學之書,暗想︰「這位大和尚全然不會武功,只讀書成痴,凡是書中所載,他無不視爲天經地義。昔年在華山絶頂初次和他相逢,聽他言道,達摩老祖在親筆所抄的楞伽經行縫之間,又冩著一部九陽眞經。他只道這是強身健體之術,便依照經中所示的修習,他師徒倆不經旁人傳授,不知不覺間竟達到了天下一流高手的境界。那日瀟湘子打他一掌,他挺受一招,反而使瀟湘子身受重傷,如此神功,便是爹爹和大哥哥也未必能彀。再看今日他師徒倆使何足道悄然敗退,豈非又不是這部九陽眞經之功?這時他口中喃喃唸誦的,莫非便是九陽眞經麼?」

她心中一想到此處,生怕岔亂了覺遠的神思,悄悄坐起,聽著他唸誦,在心中暗暗記憶,自忖︰「倘若他唸的眞是九陽眞經,奥妙精微,自非片刻之間能解。我且心中記著,明日再請他指教不遲。」只聽他唸道︰「\dash{}先以心使身,從人不從己,後身能從心,由己仍從人。由己則滯,從人則活。能從人,手上便有方寸。秤彼勁之大小,分厘不錯;權彼來之長短,毫髮無差。前進後退,處處恰合,工彌久而技彌精\dash{}」郭襄聽至這裡,不自禁的搖頭,心中説道︰「不對不對。爹爹和媽媽常説,臨敵之際,須當制人而不可受制於人。這大和尚可説錯了。」

郭襄正自沉吟,只聽覺遠又唸道︰「彼不動,己不動,彼微動,己先動。勁似鬆非鬆,將展未展,勁斷意不斷\dash{}」郭襄越聽越是迷茫,要知她自幼學的武功,全是講究先發制人、後發制於人,處處搶快,著著爭先。覺遠這時所説的拳經功訣,却説「由己則滯,從人則活」,與她平素所學,大相逕庭,心想︰「倘若臨敵動手之時,雙方性命相搏,我竟捨己從人,敵人要我東便東,要我西便西,那不是聽由挨打麼?」這「後發制人」的拳理,要直到明季以後,武當派昌盛於世,纔爲武學之士所重視。其時才當宋末,郭襄乍然聽來,自覺怪誕不經。

便是這麼一遲疑,覺遠説的話便溜了過去,竟是聽而不聞,月光之下,忽見張君寶盤膝而坐,也在凝神傾聽,郭襄心道︰「不管他説的對與不對,我只管記著便了。這大和尚震傷瀟湘子、氣走何足道,乃是我親眼目睹,他所説的武功,總是有幾分道理。」於是又用心暗記。

覺遠隨口背誦,斷斷續續,有時却又夾著幾段楞伽經的經文,説到佛祖在楞伽島登山説法的事。原來那九陽眞經夾書在楞伽經的字旁行間,覺遠讀書又有點泥古不化,隨口背誦之際,竟連楞伽經也背了出來。郭襄聽著,更是覺得摸不著頭腦,幸好她生來聰潁,覺遠所唸經文雖然顚三倒四,却也能記得了二三成。

日輪西斜,人影漸長,覺遠唸經的聲音漸漸低沉,口齒也有些糢糊不清。郭襄勸道︰「大和尚,你累了一整天,再睡一忽児。」覺遠却似没聽到她的話,繼續唸道︰「\dash{}力從人借,氣由脊發。胡能氣由脊發?氣向下沉,由兩肩收入脊骨,注於腰間,此氣之由上而下也,謂之合。由腰展於脊骨,布於兩膊,施於手指,此氣之由下而上也,謂之開。合便是收,開便是放。能懂得開合,便知陰陽\dash{}」他越唸聲音越低,到後來,終於寂然無聲,似已沉沉睡去。郭襄和張君寶不敢驚動,只是默記他唸過的經文。

天上斗轉星移,月落西山,驀地裡烏雲四合,漆黑一片。又過一頓飯時分,東方漸明,只見覺遠閉目垂眉,靜坐不動,臉上微露笑容。張君寶悄聲道︰「郭姑娘,你餓不餓,我再去採些野莓來。」一回頭,突見大樹後人影一閃,依稀見到黃色袈裟的一角。張君寶吃了一驚,喝道︰「是誰?」只見一個身材瘦長的老僧從樹後轉了出來,正是羅漢堂首座無色禪師。

郭襄又驚又喜,説道︰「大和尚,你怎地苦苦不捨,還是追了來?難道非擒他們師徒歸寺不可麼?」無色道︰「善哉善哉,老僧尚分是非,豈是拘泥戒律之人?老僧到此已有半夜,若要動手,也不等到此時。覺遠師弟,無相禪師率領達摩堂弟子,正向東追尋,你們快快往西去罷!」却見覺遠閉目不醒,理也不理。張君寶上前道︰「師父醒來,羅漢堂首座跟你説話。」覺遠仍是不動。張君寶驚起來,伸手一摸他額頭,觸手冰冷,原來早已圓寂多時了。張君寶大悲,伏地叫道︰「師父,師父!」却那裡叫他得醒?

無色禪師合什行禮,説偈道︰

\begin{quotation}
諸方無雲翳\hskip8pt四面皆清明

微風吃香氣\hskip8pt衆山靜無聲

今日大歡喜\hskip8pt捨却危脆身

無嗔亦無憂\hskip8pt寧當不欣慶
\end{quotation}

説罷,飄然而去。

張君寶大哭一場,郭襄也流了不少眼泪。少林寺僧衆圓寂,盡皆火化,當下兩人撿些枯柴,將覺遠的法身焚了。郭襄道︰「張兄弟,少林寺僧衆尚自放你不過,你諸多小心在意,咱們便此别過,後會有期。」張君寶垂泪道︰「郭姑娘,你到那裡去?我又到那裡去?」

郭襄聽他問自己到那裡去,心中微覺一酸,説道︰「我是天涯海角,行蹤無定,自己也不知道到裡去。張兄弟,你年紀小,又是江湖上閲歷全無。少林寺的僧衆正在到處追捕於你,這樣吧。」説著從腕底上褪下一隻金絲鐲児,遞過去給他,道︰「你拿著這鐲児到襄陽城去,見我爹爹媽媽,他們必能善待於你。只要在我爹媽跟前,少林寺的僧衆再狠,也不能到襄陽來難爲你。」張君寶含泪接了鐲児。郭襄又道︰「你跟我爹爹媽媽説道我身子很好,請他們不用記掛。我爹爹最喜歡少年英雄,見你這等人才,説不定會收你做了徒児。我弟弟忠厚老實,一定跟你很説得來。只是我姊姊脾氣大些,一個不對,説話便不能給人留臉面,但你只須順著她些児,也就是了。」説著轉身,飄然而去。

張君寶但覺天地茫茫,竟無自己安身之處,在師父的火葬堆前呆立了半日,這纔舉步。走出十餘丈,忽又回身,挑起師父所留的那對大鐵桶,搖搖晃晃的緩步而行。荒山野嶺之間,一個孤身少年,瘦骨稜稜的黯然西去,眞是悽悽惶惶,説不盡的寂寞。

行了半月,已到湖北境内,離襄陽已不在遠。少林寺僧衆却始終没追上他。原來無色禪師暗中眷顧,故意將僧衆引向東方,以致反其道而行,和他越離越遠。

這一日午夜,他在一座大山脚下倚石休憩,一問過路的鄕人,得知此山名叫武當,但見鬱鬱蒼蒼,林木茂密,山勢甚是雄偉。正觀賞間,忽見一男一女兩個鄕民從身旁山道上經過,兩人並肩而行,神態甚是親密,顯是一對少年夫妻。那婦人口中{\upstsl{嘮}}{\upstsl{嘮}}叨叨,不住的責備丈夫,那男子却低下了頭,只不作聲。但聽那婦人説道︰「你一個男子漢大丈夫,不能自立門戸,却去依傍姐姐和姐夫,没來由自己討這一場羞辱。咱倆又不是少了手脚,自己幹活児自己吃飯,便是青菜蘿葡,粗茶淡飯,也何等逍遙自在?偏是你全身没一一根硬骨頭,當眞枉爲生於世間了。常言道得好,除死無大事。難道非依靠别人不可?」那男子給妻子這一頓數説,不敢回一句嘴,一張臉脹得豬肝也似的成了紫醬之色。

當眞是言者無意,聽者有心,那婦人這一番話,句句都打進了張君寶心裡︰「你一個男子漢大丈夫,不能自立門戸\dash{}没來由自己討這一場羞辱\dash{}常言道得好,除死無大事,難道非依靠别人不可?」他望著這對鄕下夫妻的背影,呆呆出神,心中翻來覆去,儘是想著那農婦這幾句當頭棒喝般的言語。只見那漢子挺直了腰板,不知説了幾句什麼話,夫妻倆大聲笑了起來,似乎那男子已決意自立,因此夫妻倆同感歡悦。

張君寶又想︰「郭姑娘説道,她姊姊脾氣不好,説話不留情面,要我順著她些児。我好好一個男子漢,又何必向人低聲下氣,委曲求全?這對鄕農夫婦尚能奮發自強,我張君寶何必寄人籬下,瞧人眼色?」

言念及此,心意已決,當下挑了鐵桶,便上武當山去,找了一個岩穴,渴飲山泉,饑餐野果,盡以餘暇修習從覺遠處聽來的九陽眞經。他天資過人,所學的又是武學奇書,十餘年間竟是内力大進。某一日在山間閒遊,見一蛇一鵲相互搏擊,那鵲児多方進逼,却始終輸青蛇一籌,負創而去。張君寶心中若有所悟,在洞中苦思七日七夜,猛地裡豁然貫通,領會了武功中以柔克剛的至理,忍不住仰天長笑。

這一番大笑,竟笑出了一位承先啓後,繼往開來的大宗師。他以自悟的拳理和九陽眞經中所載的内功相發明,創出了輝映後世,照耀千古的武當一派武功。後來他北遊寶鳴,見到三峰挺秀,蒼海卓立,於武學又有所悟,乃自號三丰,那便是中國武學史上不世出的奇人張三丰。

此後數十年中,郭襄足跡遍於天下,到處尋訪楊過和小龍女夫婦,當眞是情之所鍾,至老不悔。但楊過夫婦竟是從此不知所終,再不在人間一現俠蹤。這其間宋亡元興,花落花開,不知經歷了多少人事滄桑。郭襄在四十餘歳那年,突然大澈大悟,在峨嵋山絶頂剃度出家,精研武功,其後稍收門徒,成爲武學中峨嵋一派。

那崑崙三聖何足道在少林寺鎩羽而歸,回到西域,果然履行誓言,從此不再涉足中原,直到年老之時,纔收了一個弟子,傳以琴棋劍三項絶學。因此崑崙一派門人,雖然遠在異域,却大都是風度翩翩,文武兼資。

其後武林之中,以少林、武當、峨嵋、崑崙四派最爲興旺,人才輩出,各放異采。那日覺遠大師在荒山中臨終之時,背誦九陽眞經,郭襄、張君寶、無色禪師三人雖均同時聽聞,但因三人天資和根底不同,記憶和領會頗有差别,是以三人傳下來的峨嵋、武當、少林三派武功,也是相異之處多而相同之處少。

郭襄家學淵源,所習最多,因此峨嵋一派弟子武功甚雜,往往只精一項,便足以成名。無色禪師聽聞九陽眞經時本身已是武學大師,這經文於他只是稍加啓迪,令他於武學修爲上進入更高的一層境界,但基本行功,却絲毫無變。只有張君寶除了楊過所授四招及一套羅漢拳外,從未學過武功,於九陽眞經領悟最純,但也因此缺了武功的根基,當時於經中精義,許多處所無法了解,到後來見蛇鵲相鬥,自悟武功,却已在三十餘年之後,少年時所聽聞的經文,已不免記憶糢糊。

是以少林、武當、峨嵋三派的武功各有所長,也各有所短,三派的宗師分得九陽眞經的若干章節,各憑自己的聰明智慧,鑽研發揚。

元代中土淪於異族,百姓呻吟於蒙古的鐵蹄之下,陥身於水深火熱之中,爲了抵抗官吏殘暴,勉力自保,是以文事凋零,武學一道,反而更加光大。江湖間奇人異士,所在都有,比之宋末郭靖、黃蓉、楊過、小龍女之世,武功固更見精進,而驚心動魄,可歌可泣之事,也是書之不盡。其中西域奇士,大都出於崑崙,而中土豪俠,非少林、武當即屬峨嵋。但這是指其卓犖大者,其餘較小的門派山寨,又何下千百。

且説這一年是元順帝至元二年,宋朝之亡,至此已整整六十年。其時正當暮春三月,江南海隅,一個三十來歳的藍衫壯士,脚穿草鞋,邁開大步,正自沿著大道趕路。眼見天色向晩,一路上雖然桃紅柳綠,江南春色正濃,那壯士却也無心賞玩,心中默默計算︰「今日是三月廿四,到四月初九還有一十四天,須得道上絲毫没有耽擱,方能及時趕到武當山玉虛宮,慶賀恩師他老人家九十歳大壽。」

原來這壯士姓兪名岱岩,乃是武當派祖師張三丰的第三名弟子。張三丰直至七十歳後,武功大成,方收弟子,因之他自己雖已九十高齡,但七個弟子中年紀最大的宋遠橋,也是四十歳未滿,最小的莫谷聲更只十餘歳。七個弟子年紀雖輕,在江湖上却已闖出極大的萬児,武林人士提起那七弟子來,都是大姆指一翹,説道︰「武當七俠,名門正派,那有什麼説的。」

兪岱岩在武當七俠中位居第三,這年年初奉師命前赴福建誅滅一個綁人勒贖戕害良民的劇盜。那劇盜武功既強,人又陰毒,一聽到風聲,立時隱伏不出。兪岱岩費了兩個多月時光,纔找到他的巢穴,上門挑戰,使出師傳太極玄虛刀法,在第十一招上將他殺了。當時預計十日可完的事,却耗了兩個多月,屈指一算,距師父九十大壽的日子已經迫促,因此上急急忙忙的自福建趕回。

他邁著大步急行一陣,路徑漸漸窄小,靠右近海一面,常見一片片光滑如鏡的平地,往往七八丈見方,便是水磨的桌面,也無此平整滑溜。兪岱岩走遍大江南北,見聞可不在少,但從未見過如此奇異的地狀,一問土人,不由得啞然失笑,原來那便是鹽田。當地鹽民引海水灌入鹽田,曬乾之後,刮下含鹽泥土,化成滷水,再逐步曬成鹽粒。兪岱岩心道︰「我吃了三十年鹽,却不知一鹽之成,如此辛苦。」

正行之間,忽見西首小路上一行二十餘人挑了擔子,急步而來。兪岱岩只一瞥之間,便吃了一驚,但見這二十餘人一色青布短衫褲,頭戴斗笠,擔子中裝的顯然都是海鹽。他知道當政者暴虐,收取鹽税極重,因之雖是濱海之區,一般百姓也吃不起官鹽,只有向私鹽販子購買私鹽。這一群人行動驃悍,身形壯實,看來似是一群鹽梟,那原也不奇,奇怪的是每人肩頭的扁擔非竹非木,黑黝黝的全無彈性,便似是一條鐵扁擔。各人雖都挑著二百來斤的重物,但行路時猶似足不點地,霎時之間便搶在兪岱岩的前頭。兪岱岩心想︰「這幫鹽梟,個個都是武林的好手。雖早聽説江南海沙派販賣私鹽,聲勢極大,派中不乏武學名家,但二十餘個好手聚在一起挑鹽販賣,絶無是理。」若在平時,早便要去探視一下其中究竟,但這時念著師父的九十大壽,心想決不能多管閒事,再有耽擱,當下提氣急趕,追過了那群鹽梟。那二十餘人見他脚步如此輕捷,臉上均有詫異之色。

兪岱岩趕到傍晩,到了一個小鎭上,一問之下,却是餘姚縣的庵東鎭。由此過錢塘江,便到臨安,再折而西北行,要經江西、湖南兩省,纔到武當。晩間無船渡江,只得在庵東鎭上找一家小客店宿了。

剛用過晩飯,洗了脚要上床,忽聽得店堂中一陣喧雜,一群人過來投宿。兪岱岩聽那些人説的都是浙東鄕音,但中氣充沛,大非尋常,於是探頭向門外一瞧,却便是那群鹽梟前來住店。本來做私梟的大都生性豪邁,一投店便是大碗價喝酒,大塊價吃肉,但這群鹽梟只要了些青菜豆腐,白飯吃了個飽便睡,竟是滴酒不飲。兪岱岩也不在意,盤膝坐在床上,按著師授心法,練了三遍行功,即便著枕入睡。

睡到中夜,忽聽得鄰房中喀的一聲輕響,兪岱岩此時已得師門心法眞傳,雖然睡夢之中,也是刻刻驚覺,登時便醒了。只聽得一人低聲道︰「大家悄悄走吧!莫驚動了鄰房那個客人,多生事端。」餘人也不答應,輕輕推開房門,走到了院子中,兪岱岩從窗格子中向外一張,只見那二十個鹽梟各自挑著擔子,越牆躍出。這牆頭雖不甚高,但人人挑著重擔一躍而出,這一份功夫可當眞輕視不得。兪岱岩自忖︰「這些人的武功雖不及我,但難得二十餘人,個個身手不弱。」又想起那人説道︰「莫驚動了鄰房那個客人,多生事端。」那人若不説這句話,兪岱岩雖然醒覺,也不會跟蹤前往,只因這一句話,挑動了他的俠義之心,暗想︰「這群私梟鬼鬼祟祟,顯是要去幹什麼歹事,既教我撞見了,可不能不管。若能阻止他們傷天害理,救得一兩個好人,便是誤了恩師的千秋壽誕,他老人家也必喜歡。」

要知張三丰傳藝之初,即向每個弟子諄諄告誡,學會武藝之後,務須行俠仗義,拯難濟世。「武當七俠」所以名頭響亮,不單因武藝高強,更由於慷慨任俠,急人之急,這纔贏得武林中人人欽仰。這時兪岱岩想起師訓,將藏著兵刃暗器的布囊往背上一縛,穿窗而出。一個「斜飛式」,輕輕巧巧的竄出牆外。

\chapter{屠龍寶刀}

耳聽得脚步聲往東北方而去,兪岱岩吸一口氣,展開輕身功夫,悄悄追去。當晩烏雲滿天,星月無光,沉沉黑夜之中,隱約可見那二十餘名鹽梟挑著擔子,在田塍上飛步而行。兪岱岩見這群人邁動脚步,奔得快捷異常,肩頭重負,竟似無物,心想︰「私梟黑夜趕路,事屬尋常。只是瞧這一干人個個身手不凡,若要作些非法勾當,别説偸盜富室,就是搶劫府庫,一般官軍那裡阻擋得住,何必偸偸摸摸的販賣私鹽,賺一點蠅頭微利?料來其中必有别樣異謀。」

不到半個時辰,那幫私梟已奔出二十餘里,好在兪岱岩輕功了得,脚下無聲無息,那幫私梟又似有要事在身,貪趕路程,竟不回顧,因此並没發覺。這時已行到海旁,波濤衝擊岩石,轟轟之聲不絶。正行之間,忽聽得領頭的一人一聲低哨,衆人都站定了脚步。領頭人低聲喝問︰「是誰?」黑暗中一個嘶啞的聲音説道︰「三個水旁的朋友麼?」領頭那人道︰「不錯。閣下是誰?」兪岱岩心下{\upstsl{嘀}}咕︰「三個水旁的朋友,那是什麼?」一轉念,登時醒悟︰「{\upstsl{嗯}},那是海沙派。『海沙派』這三個字都是水旁的。」那嘶啞的聲音道︰「屠龍刀的事,我勸你們别插手啦。」領頭那人一震,道︰「尊駕也爲屠龍刀而來?」那嗓子嘶啞的人一聲冷笑。黑夜中但聽他「嘿嘿嘿」幾聲,却不答話。

兪岱岩只覺他這笑聲大是古怪,聽在耳中,令人心煩意亂,無法形容的不舒服,似乎十幾條巨蟲突然在背上搔爬,又似乎吞下了什麼吐不掉,嘔不出的異物。他心念一動,隱身在海旁的岩石之後,繞到前面,只見一個瘦瘦小小的男子攔在路中。黑暗中瞧不清他的面貌,只見他手中拿著一根枴杖,身上衣服有點點閃光,顯是一件錦袍。又聽海沙派的領頭人説道︰「這屠龍刀原是本派鎭派之寶,既給宵小盜去,自當索回。」那錦袍客又是「嘿嘿嘿」三聲冷笑,仍是大模大樣的攔在路中,那領頭人身後一人厲聲喝道︰「快些讓開,惡狗攔路,你不是自己找死\dash{}」他話聲未畢,突然「啊」的一聲慘叫,往後便倒。衆人一驚,但見黑暗中錦袍上的閃光晃了幾晃,攔道惡客已然不見。

海沙派衆私梟瞧那跌倒的同伴時,但見他蜷成一團,早已氣絶。各人又驚又怒,有幾人放下擔子向錦袍客去路急追,但那人奔行如電,黑暗之中那裡還尋得到他的蹤影。

兪岱岩好生奇怪︰「這錦袍客不知施放的是什麼歹毒暗器,怎地手不抬、身不動,對方便已斃命?我跟他相距不遠,居然没瞧出絲毫端倪。」他縮身在岩石之後,一動也不敢動,生怕給海沙派的幫衆發見了,没來由的招惹禍端。只聽那領頭人道︰「將老四的屍首放在一旁,我們料理大事要緊,回頭再來收拾,這仇人是誰,將來總能査究得出。」衆人答應了,挑上擔子,又向前飛奔。兪岱岩待他們去遠,走近那屍身察看,只見那人如死蝦般彎彎蜷曲,顯是中了異毒,兪岱岩但覺此事大是蹺蹊,生怕沾上了毒,不敢伸手去扳那屍身,於是加快脚步,再跟蹤那批私梟。

一行人又奔出數里,那領頭人一聲忽哨,二十餘人四下散開,向東北方一座大屋慢慢逼近,兪岱岩心想︰「他們説的是什麼屠龍刀,難道便是在這屋中麼?」只見那大屋的煙{\upstsl{囪}}中一柱濃煙衝天而起,久聚不散,而煙{\upstsl{囪}}中還是源源不絶的噴出黑煙來。衆私梟放下擔子,各人拿了一集木杓,在籮筐中抄起什麼東西,四下散播。兪岱岩見他們撥撒的東西如粉如雪,顯然便是海鹽,心道︰「今晩所見之事,當眞是匪夷所思,日後説給師兄弟們知道,他們定是不信。」

但見這幫鹽梟撒鹽之時,行動極之小心,似乎生怕將鹽粒濺到了身上,兪岱岩久歷江湖,登時恍然,知道鹽上含有劇毒,這批人用毒鹽圍屋,當是對屋中人陰謀毒害。他見到此事,更激起了俠義之心,暗︰「雙方誰是誰非,我固不知情,但這批人幹這種鬼域技倆,太不光明正中。無論如何,須得通知屋中之人,好教他不致爲宵小所害。」眼見海沙派的鹽梟逐步撒鹽,尚未及到屋後,於是施展武當派的「千山縮地功」輕身絶技,兜個大圏子繞到屋後,輕輕跳到進了圍牆。

這座大屋前後五進,共有三四十間,屋内黑沉沉的,没一處有燈火。兪岱岩心想︰「眼見濃煙是從中間一進屋中冒出,該處想必有人。」又怕屋中人誤會自己是敵人,橫加暗算,於是拾了一根木柴,晃火摺點燃了,當作火把,高高舉在手中,朗聲説道︰「武當派弟子兪岱岩有事奉告,絶無他意,請勿起疑。」他説話的聲音雖然不響。但中氣充沛,傳送極遠,按理大屋中每一間房内都可聽到,但他連説了兩遍,屋中靜悄悄的却無回音。

兪岱岩是名門英俠,雖見這大屋中陰沉沉的,鬼氣森森,却決不示弱於人,也不拔出腰間單刀,只是潛引眞氣,周流全身,一面昂然直入。穿過一個天井,來到了後廳,一瞥之下,不由得凜然止步。只見廳側兩個人屍橫就地,一個是道人裝束,另一個是鄕農打扮,兩人年紀都不小,臉上五官扭曲,形容可怖,顯是身死之時曾遭受極大痛苦。但身旁並無血跡,身上更不見傷痕,顯非身中兵刃而死。

兪岱岩繼續向前,但見每一處門戸都是洞開,但廳房之中均是黑黝黝的不知藏著些什麼,除了他手中火把照出一團光亮之外,四周全是黑漆一團。饒是他膽大氣壯,見多識廣,到了這等情景之下,背上也是不自禁的暗生涼意。

再穿過一個院子,又來到一個偏廳。這廳中的情景更是可怖,橫七豎八,一共死著二十餘人,有的相互扭成一團,有的手中刀劍各各砍在對方身上。這些人有的死去已久,面目早已變色,有的却是新死。兪岱岩心想︰「這大屋之中,定是有著一樁武林慘變。瞧這些人所使的兵刃,很有許多是名家子弟。像點穴橛、五行輪、判官筆這些傢伙,倘若不是精通點穴打穴之術,如何能使?却不知爲了何事,一一喪命於此?」

他初進屋時漫不在意,待得一見到這許多好手屍橫就地,這纔起戒懼之心,又朗聲説道︰「武當派弟子兪岱岩有事奉告,請前輩高人賜見。」只聽得正廳中傳出火燄燃燒的畢剝之聲,又有人在呼呼吹氣,却始終無人答話。兪岱岩轉過一道照壁,一道屏風,跨步進了正廳,眼前突然一亮,一股熱氣撲面而來,只見廳心一隻岩石砌成的大爐子,火燄燒得正盛,爐旁分站三人,各運眞氣,向爐中吹火,火爐中橫架著一柄四尺來長的單刀。火燄由紅轉青、由青轉白,那單刀光茫閃閃,竟是鎔鍊不掉它半點。

那三人都是六十來歳老者,一色青布袍子,滿頭滿臉都是灰土,袍子上點點斑斑,到處是火星濺開來燒出的破洞。只見那三人頭頂白霧繚繞,鼓起腮幫,緩緩吹氣。三股氣流吹入爐中,火燄登時升起五尺來高,繞著單刀,嗤嗤聲響。兪岱岩瞧了三個老者的情景,知他們内功深厚,合力吹出來的氣息之強,爲任何風箱所不及,自己站立之處和那爐子相距數丈,已是熱不可當,則爐中之熱,可想而知,但那柄單刀始終青光照耀,竟没起半點發熱而轉紅之色。便在此時,屋頂上忽有一個嘶啞的聲音喝道︰「損毀寶刀,傷天害理,快快給我住手。」

兪岱岩聽了這聲音,心中一震,知道便是途中所遇的那個錦袍客到了。那三個鼓風鍊劍的老者却恍若不聞,只是吹氣更急。但聽得屋頂那人嘿嘿嘿三聲冷笑,簷前如一葉落地,眼前金光燦爛,那錦袍客已閃身而進。這時廳中爐火正旺,兪岱岩瞧得清清楚楚,見這錦袍客是個二十餘歳的少年,面目俊秀,雙眉斜飛,只是臉色慘白,隱隱透出一股青氣,身上所穿的那件錦袍用金絲繡滿了獅虎花草,華美輝煌之極。他雙手空空,並無兵刃,冷然説道︰「長白三禽,你們覬覦這把屠龍刀,那也罷了,却何以膽敢用爐火損毀這等神兵利器?」一面説,一面踏步上前。

長白三禽中西首一人身子一晃,左手倏出,伸出又瘦又尖的五根手指,往錦袍客臉上抓去。錦袍客側身避過,又搶上一步,東首那老者見他逼近身來,提起爐子旁的大鐵錘,呼的一聲,往他頭頂猛擊下去。那錦袍客身手極是敏捷,身子微側,鐵錘一擊落空,砰的一聲猛響,鐵錘落地,火星四濺,原來地下鋪的不是尋常青磚,却是堅硬異常的花崗石。西首老者自旁夾攻,雙手猶如雞爪,上下飛舞,取的全是凌厲攻勢。兪岱岩瞧得暗暗心驚︰「這些人相互間不知有什麼深仇大怨,何以出手竟是半點也不留情,招招全是意欲制人死命的殺手?」但見那錦袍客武功極是奇特,臉上露著詭笑,似還招似不還招,兩個老者却絲毫奈何他不得。鬥了數合,那使鐵錘的老者厲聲喝道︰「閣下是誰?便要此寶刀,也須留個萬児。」錦袍客冷笑三聲。只不答話,猛地裡一個轉身,喀喀兩響,西首老者雙腕齊折,東首老者機警異常,眼見情勢不對,知道合三人之力,也阻擋他不住,當即拾起一柄火鉗,便往爐中去挾那屠龍刀。

站在南首的老者手中扣著暗器,俟機傷敵,只是錦袍客轉身迅速,一直没找著空子,這時眼見東首老者用火鉗去挾寶刀,知道這刀一落他手,再也難以索回,他有了寶刀之後,自己那裡還能是他敵手?危急中勇氣倍增,突然伸手入爐,搶先抓住刀柄,提了出來。

爐中火勢何等猛烈,那屠龍刀雖没給鎔成鐵汁,却是炙熱無倫,那老者一握住刀柄。一股白煙冒起,各人鼻中聞到一陣焦臭,他手掌心登時燒焦,但他兀自不放,竟如瘋子一般,一時也不知痛楚,提著屠龍刀尚後躍開。餘人見了這等情景,盡皆駭然,一呆之下,但見那老者提著刀,向外狂奔。

錦袍客冷笑道︰「有這等便宜事?」手臂一長,已抓住了他背心。那老者順手迴掠,屠龍刀揮了過來。刀鋒未到,却已熱氣撲面,瞬息之間,錦袍客的鬢髮眉毛都捲曲起來。他微微一驚,不敢擋架,手上勁力一送,將老者連人帶刀擲向洪爐。

兪岱岩一直在旁袖手觀鬥,但覺這一干人個個兇狠悍惡,無不帶著幾分邪氣,事不関己,也就不必出手。斯時見老者命在頃刻,只要一入爐中,立時化成焦灰,且不理其中是非曲直,眼前救命要緊,於是縱身躍起,但見他輕如飛燕,矯若飛龍,一轉一折,一揮一控,在半空中伸下手來,抓住那老者的髮髻,向上一提,解救了他這洪爐之厄,跟著輕巧巧的落在一旁。這一手武當派的輕身神功,縱躍既高,在半空中又能迴旋自如,名曰「梯雲縱」,可説天下武林中諸派輕功之冠。錦袍客和長白三禽雖早見他站在一旁,一直也没在意,這時突然見他顯示了這一手上乘輕功,這纔吃驚。錦袍客長眉一挑,説道︰「這一手便是聞名天下的『梯雲縱』麼?」

兪岱岩聽他叫出了自己這路輕功的名目,先是微微一驚,但跟著不自禁的暗感得意︰「武當派名揚天下,無人不知。」説道︰「不敢請教尊駕貴姓大名?在下這點児微末功夫,何足道哉?」那錦袍客道︰「很好很好,武當派輕功號稱並世無雙,果然是有兩下子。」他説話的口氣甚是傲慢,雖是稱道兪岱岩的輕功,但言下之意,却似是長輩獎許後輩練武有成,只差著没説出「小子可嘉」四個字而已。兪岱岩心頭有氣,但按捺住却不發作,説道︰「尊駕途中一舉手而斃海沙派高手,這份功夫神出鬼没,更是令人莫測高深。」那人心頭一凜,暗想︰「此事居然叫你看見了,我却没瞧見你啊。不知你這小子當時躱在何處?」却淡淡的道︰「不錯,這種武功,旁人原是不易領會,别説閣下,便是武當派掌門人張老頭児,也未必懂得。」兪岱岩向來甚有修養,别人再厲害的挺撞,他也不會反唇相稽,但這時聽那錦袍客辱及恩師,這口氣如何忍得下去?可是武當派弟子自來講究修心養性的功夫,這性命雙修之學,比習練武功更是來得重視,心想︰「他有意挑釁,不知安著什麼心眼児?此人功夫怪異,不必爲了幾句無禮的言語,爲本門多樹強敵。」當下微微一笑,説道︰「天下武學無窮無盡,正派邪道,千千萬萬,武當派所學原是滄海一粟。如尊駕這等功夫,本師確是不會。」他這句話説得客氣,但骨子中含義,却是説武當派實不屑懂得旁門左道的武功。

他二人一問一答,針鋒相對。那南首老者赤手握著一柄熾熱的屠龍刀,皮肉焦爛,幾已燒到骨骼,東首西首兩個老者躬身蓄勢,各自想俟機搶奪。突然間呼的一聲響,南首那老者揮動寶刀,向外急闖,他這一刀在身前揮動,不是向著何人而砍,但兪岱岩正站在他的身前,首當其衝,但見一股疾風,襲向腰間,其勢好不凌厲。他更没料得自己救了這老者的性命,他竟然會忽施反噬,危急中向上一躍,避過刀鋒。那老者雙手握住刀柄,發瘋般亂砍亂揮,衝了出去。錦袍客和其餘兩個老者忌憚刀勢凌厲,不敢硬擋,只是連聲呼叱,隨後追去。

兪岱岩拔足跟隨,他輕功遠勝諸人,雖後發而先至,倏忽之間,已搶到那提刀老者身旁。只見他雙手提刀,身形跌跌撞撞,似乎刀身極爲沉重,猛力一縱,躍出了大門,但落下時脚下一個踉蹌,竟爾摔了一跤,跟著一聲慘呼,似乎莫名甚妙的身受重傷。

錦袍客和另外兩個老者一齊縱身過去,同時伸手去搶寶刀,但不約而同的叫了出來,似乎猛地裡被什麼奇蛇毒蟲所咬中一般。那錦袍客武功最強,只打個跌,跟著便躍起身來,急向外奔,那所謂「長白三禽」的三個老者,却在地下不住翻滾,竟爾不能站起。

兪岱岩見了這等慘狀,正要伸手救人,突然間心中一凜,想起了海沙派在屋外撤佈毒鹽之事,這些海鹽定是以劇毒的藥物煎熬過,因此沾體即生大害。此時屋周均是毒鹽,自己也無法出去了,遊目一瞥之間見大門内側左右各放著一張長凳,心念一動,伸手抓起,將兩凳豎直,一躍而上,一雙脚勾著一隻長凳,便似{\upstsl{踩}}高蹻一般,踏著雙凳走了出去。但見三個老者慘叫不停,在地上滾來滾去,模樣甚是可怖。兪岱岩知道危機四伏,不及細思,扯下一片衣襟裹在手上,一長臂便抓起了那懷抱寶刀的老者後心,脚{\upstsl{踩}}高蹻,向東急行。

這一下大出海沙派衆人意料之外,眼見便可得手,却斜刺裡殺出個程咬金來將寶刀搶走,衆人那裡甘心?紛紛湧出,大聲呼叱,鋼鏢袖箭,十餘種暗器一齊向兪岱岩後心射去。

兪岱岩雙足使勁,在兩張長凳上一蹬,身形躍起,向前竄出數丈,所有暗器盡皆落空。他脚上勾了長凳,雙足便似斗然間加長了四尺,這一邁開步子,行路大是迅捷,只跨出四五步,早已將海沙派諸人遠遠抛在後面。耳聽得各人大呼追來,兪岱岩提著那老者縱身躍起,雙足向後反踢,兩張長凳向身後飛了出去。但聽得砰砰兩響,跟著三四人大聲呼叫,顯是爲長凳擊中。就這麼阻得一阻,兪岱岩的輕功何等了得,在黑暗之中早已奔出數十丈外,手中雖提著一個老者,却是越奔越遠,海沙派諸人再也追他不上了。

兪岱岩急趕一陣,耳聽得潮聲澎湃,後面無人追來,問道︰「你怎樣了?」那老者「哼」了一聲,並不回答,跟著呻吟一下。這一下聲音雖然不響,但猶似傷獸悲{\upstsl{嗥}},顯是痛楚已極。兪岱岩心道︰「他身上沾滿毒鹽,先給他洗去要緊。」於是走到海邉,將他往淺水處浸了下去,自己手掌却不敢和海水相碰,生怕沾上了毒鹽。

那老者半昏半醒,在海水中浸了一陣,自己不能爬起,兪岱岩正要伸手拉他,忽然一個巨浪打來,將老者沖上沙灘。兪岱岩道︰「現下你已脱險境,在下身有要事,不能相陪,咱們便此别過。」那老者撐起身來,説道︰「你\dash{}怎地\dash{}不搶這把寶刀?」兪岱岩一笑,道︰「寶刀縱好,又不是我的,我怎能橫加搶奪?」那老者心下大奇,不能相信,道︰「你\dash{}你到底有何詭計,要怎樣泡製我?」兪岱岩道︰「我跟你無冤無仇,泡製你幹麼?我今晩路過此處,見你中毒受傷,因此出手相救。」那老者搖了搖頭,厲聲道︰「我命在你手,要殺便殺,若是想用什麼毒辣手段加害,我便是死了,也必化成厲鬼,放你不過。」兪岱岩知他受傷後神智不清,也不去跟他一般見識,只是微微一笑,正要舉步走開,海中又是一個大浪打來,只濺得兪岱岩衣履盡濕,那老者呻吟一聲,伏在海水之中,身子發顫。

兪岱岩心想,救人須救澈,這老者中毒不輕,我若於此時捨他而去,他還是葬身海底。於是伸手抓住他背心,提著他走上一個小丘,四下眺望,見東北角一塊突出的山岩之上有一間屋子,瞧那模樣,似是一所廟宇,當下抱著那老者奔了過去,凝目看屋前匾額,隱約可見是「海神廟」三字。兪岱岩推門進去,見這海神廟極是簡陋,便只小小的一間,滿地塵土,廟中也無廟祝居住。

於是將老者放在神像前的木拜墊上,他懷中火摺已被海水打濕,當下在神檯上摸索,找到火絨火石,點燃了半截臘燭,再看那老者時,只見他滿面青紫,顯是中毒已深,不由得暗暗吃驚,從懷中取出一粒「天心解毒丹」來,説道︰「你服了這粒解毒丹藥。」那老者本來緊閉雙目,一聽他之言,突然睜開眼來,説道︰「我寧死也不吃你害人的毒藥。」

兪岱岩脾氣再好,這時也忍不住了,長眉一挑,説道︰「你道我是誰?武當七俠縱然不肖,豈能幹害人之事?這一粒是解毒丹藥,只是你身中劇毒,這丹藥也未必能彀解救,但至少可延你三日之命。你還是將屠龍刀送去給海沙派,換得他們的本門解藥救命吧。」那老者斗然間一躍而起,厲聲道︰「誰想要我的屠龍刀,那是萬萬不能。」兪岱岩道︰「你性命也没有了,空有寶刀何用?」那老者顫聲道︰「我寧可不要性命,屠龍刀總是我的。」説著將刀牢牢抱著,臉頰貼著刀鋒,當眞是説不出的愛惜,一面却將那粒「天心解毒丹」吞入肚中。

兪岱岩好奇心起,想要問一問這刀到底有什麼好處,但見這老者雙眼之中充滿著貪婪兇狠的神色,宛似飢獸要擇人而噬,不禁大感厭惡,轉身便出。忽聽那老者厲聲喝道︰「站住!你要到那裡去?」

兪岱岩笑道︰「我到那裡去,你又管得著麼?」説著揚長便走。没行得幾步,忽聽那老者放聲大哭,哭得甚是悲痛淒涼,有似傷獸夜{\upstsl{嗥}},充滿著衷苦絶望之情。這一哭觸動了兪岱岩的俠義心腸,轉過頭來,問道︰「你爲何悲哭?」那老者道︰「我千辛萬苦的得到了屠龍寶刀,但轉眼間性命不保,要這寶刀何用?」兪岱岩「{\upstsl{嗯}}」了一聲,道︰「你除了以此刀去換海沙派的獨門解藥,再無别法。」那老者哭道︰「可是我不捨得啊,我不捨得啊。」這神態在可怖之中帶著三分滑稽,兪岱岩想笑,却是笑不出來,隔了一會,説道︰「武學之士,全憑本身功夫克敵制勝,行道仗義,顯名聲於天下後世,寶刀寶劍只是身外之物,得不足喜,失不足悲,老丈何必爲此煩惱?」那老者怒道︰「『武林至尊,寶刀屠龍,號令天下,莫敢不從!』這幾句話你聽見過麼?」

兪岱岩啞然失笑,道︰「這幾句話我自然聽見過,下面還有兩句呢,什麼『倚天不出,誰與爭鋒?』那説的是幾十年前武林中一件驚天動地的事,又不是説什麼屠龍寶刀。」那老者道︰「什麼驚天動地的大事,你且説來聽聽。」兪岱岩道︰「此事武學之士人人皆知,説的是當年神鵰大俠楊過殺死蒙古皇帝憲宗,爲漢人揚名吐氣。自此楊大俠有什麼號令,天下英雄莫敢不從。『龍』便是蒙古皇帝,『屠龍』便是殺蒙古皇帝。難道世間還眞有龍之一物麼?」那老者冷笑道︰「我問你,當年楊過大俠使什麼兵刃?」兪岱岩一怔,道︰「我聽師父説過楊大俠斷了一臂,平時不用兵刃。那日在襄陽城外惡鬥金輪法王,却是使劍。」那老者道︰「是啊,楊大俠怎生殺死蒙古皇帝的?」兪岱岩道︰「他投擲石子打死憲宗,此事天下盡聞。」那老者大是得意,道︰「楊大俠平時用掌使劍,殺蒙古皇帝用的又是石子,那麼『寶刀屠龍』四字又從何説起。」

這一下問得兪岱岩無言可答,隔了片刻,纔道︰「那是武林中説得順口而已,總不能説『石頭屠龍』啊,那豈不難聽?」那老者冷笑道︰「強辭奪理,強辭奪理,我又問你,『倚天不出,誰與爭鋒?』這兩句話,却又是作何解釋?」兪岱岩沉吟道︰「『倚天』或許是一個人吧?聽説楊大俠的武藝學自他的妻子,那麼『倚天』!或許便是他夫人的名字。又或許是死守襄陽城的郭靖郭大俠。」那老者道︰「是嗎?我料到你説不上來了,只好這麼一陣胡扯,我跟你説,『屠龍』是一把刀,便是這把屠龍刀,『倚天』却是一把劍,叫作倚天劍。這六句話的意思是説,武林中至尊之物,是屠龍刀,誰得了這把刀,不管發施什麼號令,天下英雄好漢都要聽令而行。只要倚天劍不出,屠龍刀便是最厲害的神兵利器了。」

兪岱岩將信將疑,道︰「你將刀給我瞧瞧有什麼神奇?」那老者緊緊抱住,冷笑道︰「你當我是三歳小孩嗎。想騙我的寶刀。」他中毒之後,本已筋疲力衰,全仗服了兪岱岩給他的一粒解毒丹藥,這纔振奮了起來,這時一使勁,却又呻吟不止。兪岱岩笑道︰「不給我瞧便不瞧。你雖得了屠龍寶刀,却號令得動誰?難道我見你懷裡抱著這樣一把刀,便非聽你的話不可嗎?當眞是笑話奇談了。我瞧你啊,好好一個人偏去相信這種荒誕不經的鬼話,到頭來枉送性命,還是執迷不悟。你既號令我不得,便可知這刀其實是無什麼神奇之處。」

那老者呆了半晌,做聲一得,隔了良久,纔道︰「老弟,我跟你訂一個約,你救我性命,我將寶刀的好處分一半給你。」

兪岱岩仰天大笑,説道︰「老丈,你可把我武當派的弟子瞧得小了。扶危濟困,乃是我輩分内之事,豈難道是貪圖報答?你身上沾了毒鹽,我却不知鹽上安的是什麼毒藥,你只有去求海沙派解救。」那老者道︰「我這把屠龍刀,是從海沙派手中盜了出來的,他們恨我切骨,豈肯救我?」兪岱岩道︰「你既將刀交還,怨仇即解,他們何必傷你性命?」老者道︰「我瞧你武功甚強,大有本事到海沙派中去將解藥盜來,救我一命。」兪岱岩道︰「一來我身有要事,不能耽擱,二來你去偸盜人家寶刀,是你的不是,我怎能顚倒是非?老丈,你快快去找海沙派的人吧!再一耽擱,毒性發作起來,那便來不及了。」

那老者見他又是舉步欲行,忙道︰「好吧,我再問你一句話,你提著我身子之時,可覺得有什麼異樣?」兪岱岩道︰「我確是有些奇怪,你的身子瘦瘦小小,却有二百來斤,不知是什麼緣故,又没見你身上負著什麼重物。」那老者將屠龍刀放在地下,道︰「你再提一下我的身子。」兪岱岩抓住他肩頭向上一提,手中登時輕了,只不過八十來斤,心下恍然︰「原來這小小一柄單刀,竟有一百多斤之重,看來確是有點古怪,不同凡品。」於是將老者放在地下,説道︰「這把刀倒是很重。」

那老者道︰「豈僅沉重而已。老弟。你尊姓是姓兪還是姓張?」兪岱岩道︰「敝姓兪,草字岱岩,老丈何以得知?」那老者笑道︰「武當七俠中宋大俠年紀大了,殷莫兩位不過二十左右,餘下的二三兩俠姓兪,四五兩俠姓張,武林中誰人不知。原來是兪三俠,怪不得這麼高的功夫。武當七俠威震天下,今日一見,果然是名不虛傳。」兪岱岩年紀雖然不大,却也是老江湖了,聽他這般當面諂諛,知他也不過是有求於已,只是微微一笑,心中反生厭惡之感,説道︰「老丈尊姓大名?」那老者道︰「小老児姓德,單名一個成字,遼東道上的朋友們送我一個外號,叫作海東青。」那海東青是生於遼東的一種大鷹,兇狠鷙惡,捕食小獸,是関外著名的猛禽。

兪岱岩拱手道︰「久仰,久仰。」抬頭看了天色。德成知他急欲動身,若非動以大利,不能求得他伸手救命,説道︰「你不懂得那『號令天下,莫敢不從』這八個字的含義,只道是誰捧著屠龍刀,只須張口發令,人人便得聽從,不對,不對,這全盤想錯了。」他説到這裡,突然放低聲音,説道︰「兪老弟,這屠龍寶刀之中,藏著一部武學秘笈,有人説是九陽眞經,有人説是九陰眞經。只須取出來照著經書一練,那時候武功蓋世,他説出來的話,有誰能違抗得?」

九陰九陽兩部寶笈祕籙的名字,兪岱岩也曾聽師父説過,只是當年覺遠大師圓寂之後,少林、武當、峨嵋三派分得九陽眞經中的若干章節,全書早已失傳,至於九陰眞經,更是數十年來少人提起,空餘想像,當作是武林中一個可信可不信的傳説而已。德成見兪岱岩臉上有不信之意,説道︰「咱們長白三禽盜得寶刀,要用爐火鎔開它來,取出刀中藏經,只是事機不祕,大功未成,而覬覦寶刀之徒紛紛沓來。兪老弟,你去盜了解藥來解得我體内之毒,咱哥倆找了人跡不到的隱僻之處,鎔刀取經,數年之後,武林中只容咱哥児倆稱霸,除了德成和兪岱岩,再没第三個人,你説妙不妙?」兪岱岩搖頭道︰「此事決不可信,别説刀中無經,即使藏得有經書,刀尚未鎔,裡面的紙草早已成灰。」德成道︰「此刀堅硬無比,任你用多鋒利的鋼鑿尖鑽,對付不了它分毫,連一條細紋也劃不出來,除了以火鎔鍊,休想剖得它開。」

\chapter{血掌風帆}

兩人剛説到這裡,兪岱岩臉上微微變色,右手伸出一揮,{\upstsl{噗}}的一聲輕響,{\upstsl{搧}}滅了神檯上的臘燭,低聲道︰「有人來啦!」海東青德成内功修爲遠不如他,却没聽見有何異聲,正一遲疑間,只聽得遠處幾聲忽哨,有人相互傳呼奔向廟來。德成驚道︰「敵人追來啦,快些從廟後退走。」兪岱岩道︰「廟後面也有人來。」德成道︰「不會吧\dash{}」兪岱岩道︰「德老丈,來的是海沙派人衆,你正好向他們討取解藥。在下可不願{\upstsl{蹚}}這淌渾水,是禍是福,憑老丈自決。」

德成伸手牢牢抓住他的手腕,顫聲道︰「兪三俠,你萬萬不能捨我而去,你萬萬不能!」兪岱岩只覺他五根手指寒冷如冰,緊緊嵌入了自己腕上肉裡,手腕一翻,使半招「九轉丹成」,轉了一個圏子,登時將他五指甩落。德成指骨痛得如欲斷折,但當此緊急関頭,知道除了兪岱岩出手相救,自己決計無倖,若説將這件拚了性命奪到的武林至寶乖乖的雙手奉上,却又比割了自己一塊肉還要難受,當下雙手一合,將兪岱岩牢牢的抱住了。

兪岱岩吃了一驚,雙臂一振,待要將他手臂震開,那知德成竟如溺水之人抓住了救命物件一般,説什麼也不放手,只聽得格格兩響,骨骼作聲,兪岱岩只須稍稍再加勁力,立時便將他雙臂崩斷了。他心下不忍,不再運力,喝道︰「你還不放手?」

這時只聽得一路脚步之聲,直奔到廟外,跟著砰的一響,有人伸足踢開了廟門。這一腿腿力雄猛之極,那廟門呼的一聲被他踢得飛起,直撞進來。兪岱岩吃了一驚︰「此人大是了得,倒是不可輕敵。」心念甫動,鼻中微微聞到一股腥臭,有什麼物事從黑暗中擲了過來。兪岱岩身子一縮,使個巧勁,如一條泥鰍般從德成雙臂間溜了出來,這一下身法奇快,竟是搶在這一陣怪異暗器之前,縱到了海神菩薩的神像後面。但聽得德成「啊」的一聲低哼,跟著刷刷數聲,暗器打中了他的身上,接著又落在地下。那些暗器一陣接著一陣,竟是毫不停留的撤了進來,兪岱岩只聞到腥臭之氣越來越濃,似乎幾千百條死魚堆積在一起。德成東閃西避,如酔漢一般跌來撞去,但這海神廟佔地甚小,他又被暗器打得頭昏腦脹,只是一陣陣的受擊,避讓不開。

兪岱岩聽著暗器落地的聲響,心想︰「難道這是奪命毒砂?那麼德成中了這許多,早該支持不住。」正尋思間,突然省悟︰「啊,是了。這是海沙派的毒鹽。」他空有一身武藝,只是聽得風聲呼呼,毒鹽一把把的不絶從門中擲了進來,却那裡敢向外硬衝?接著聽得屋頂上喀啦、喀啦幾聲,有人躍上屋頂,揭開瓦片,又向下投擲毒鹽。

這一下只把兪岱岩嚇得心驚肉跳,心想︰「我命休矣!想不到無緣無故,在這裡遭此大禍。」他眼見到那錦袍和長白三禽身受毒鹽之害,長白三禽也還罷了,那錦袍顯是武藝極高,但在屋外一沾毒鹽,立即慘呼逃走,可見此物極是厲害。那毒鹽在小廟中瀰空飛揚,兪岱岩胸口煩惡欲嘔,心知再過片刻,非沾上不可,情急之下,伸手喀的一掌,擊破神像的背心,縮著身子一鑽,溜進了神像肚腹之中,這便如穿上了一層厚厚的泥做衣服,毒鹽雖多,却已奈何他不得。

海沙派的毒鹽倚多爲勝,毒性發作却緩,因此德成在廟中不住口的哇哇的怪叫,始終没有摔倒。海沙派衆人忌憚兪岱岩了得,却也不敢貿然便攻進廟來,只是不住手的撒放毒鹽,要將他二人毒倒,然後不費吹灰之力的進來擒拿,奪取屠龍刀。

一般金針、鐵沙之類細小的餵毒暗器,均是打傷人體,毒性由血液流遍全身,厲害的見血封喉,立時斃命,這毒鹽却是由皮膚傳入,雖不能傷人見血,但毒性慢慢發作,終究也能致人死命。兪岱岩躱在神像之内,明知終非了局,可是一時實無善策,只有待海沙派稍一疏神,俟機從屋頂躍出。他取出兩枚可解百毒的丹藥,嚥入腹内,一面屏息凝神,潛運内功,這樣一來,胸口煩惡之意登消。

只聽得屋外海沙派人衆大聲商議起來︰「點子不出聲,大槩是暈倒了。」

\qyh{}那年輕的點子手脚很硬,再等一回,何必急在一時。」

\qyh{}這一次當眞是大功一件,龍頭大哥定是重重有賞。」只聽得一人喝道︰「喂,吃橫樑的點子,乖乖出來投降吧,免得枉送了性命。」跟著一聲號令,十餘人湧進廟來,這些人身上均有解藥,因此不怕毒鹽。兪岱岩心想︰「我跟他們海沙派無冤無仇,又不是貪圖這把屠龍寶刀,不如躍將出去,跟他們兩下善罷。」但轉念又想︰「我武當派威名震於天下,我這樣出去,顯是屈服投降,大損師門威望。」正遲疑間,忽聽得廟外遠處傳來一聲呼嘯。這嘯聲細若游絲,但尖鋭刺耳,震人心魄,兪岱岩只呆得一呆,倏忽之間,那嘯聲已到了廟外的岩石之下。這下來得好快,初聽到嘯聲時顯是尚在數里以外,但一轉眼間,嘯聲即到跟前,天下除非是最快的飛鳥,方能片刻間飛行這麼長的一段路程,否則即令是千里駿馬,也不能這般的瞬息即至,然而這嘯聲明明是人聲,並非飛鳥。

那嘯聲一止,忽聽得德成大聲怪叫,聲音恐怖之極︰「你\dash{}你也要屠龍\dash{}白眉\dash{}」一句話没説完,聲音突然止歇,廟内廟外數十名海沙派人衆,也是一聲不出,四下裡一片靜寂,似乎人人突然之間僵化,變成了石頭,又似猛地裡見到什麼可怕異常的物事,都嚇得呆了,再也説不出話來。萬籟無聲之中,忽聽得波的一聲響,廟内有人受傷倒地,跟著有人顫聲道︰「是白\dash{}白眉\dash{}大夥児快走\dash{}」話猶未畢,又是聲音突然止歇,想是那令衆人嚇得心膽倶裂的怪物,已進了廟門,衆人連逃走之念也不起了。

兪岱岩大是奇怪︰「『白眉』是什麼啊?難道是一種兇惡無比的猛獸,還是一個厲害之極的人物,竟令這些人嚇得這般模樣?」忽聽得一人説道︰「教主問你們,屠龍刀在那裡,好好獻出來,教主大發慈悲,你們的性命都饒了。」這聲音甚是慈悲和親切,決不致令人起懼怖之感,但語氣之中,自有威嚴。

只聽海沙派中一人道︰「是他\dash{}他盜去了的,咱們正要追回來,教主\dash{}教主\dash{}」那慈和的聲音道︰「喂,那屠龍刀呢?」這句話顯是對著德成説的了。德成却不答話,跟著{\upstsl{噗}}的一聲響,有人倒在地下,兪岱岩心道︰「糟糕,德成遭了他們毒手。」他明知眼前事有蹊蹺,自己孤身一人,只怕非對方之敵,但既插上了手,決不能袖手旁觀,心想︰「臨事畏縮,非丈夫也。」正要躍將出去問個明白,忽聽一人冷冷的道︰「這人已嚇死了,搜他身邉。」

兪岱岩一驚︰「怎麼便嚇死了?」但聽得衣衫悉率之聲,又有人體翻轉之聲,那聲音柔和的人道︰「稟報教主,這人身邉無甚異物。」過了半晌,海沙派中領頭的人顫聲道︰「教\dash{}教主,明明是他盜去的,咱們決不敢隱瞞\dash{}」聽他聲音,那是在教主威嚇的眼光之下,驚得心膽倶裂。這恐懼的聲音從黑暗中傳入兪岱岩耳中,他雖藝高人膽大,但聽著也不禁有不寒而慄之感,又想︰「那寶刀明明是德成握在手中,怎地不見了?」

只聽那聲音慈和的人道︰「你們説這刀是他盜去的,怎會不見?定是你們暗中收藏了起來。這樣吧,誰先把眞相説了出來,我饒他不死。你們這群人中,只留下一人不死,誰先説,誰便活命。」廟中寂靜一片,隔了半晌,海沙派的首領︰「啓稟教主,咱們當眞不知,不過咱們一定出力追査眞相\dash{}」那聲音冷冷的教主哼了一聲,並不答話。那聲音慈和的人却説︰「誰先稟報眞相,就留誰活命。」過了一會児,海沙派中無一人説話,突然一人叫道︰「咱們找尋寶刀,確是不見影蹤,你既然一定不信,左右是個死,今日跟你拚了,瞧白眉教\dash{}」一句話没説完,驀地止歇,竟是無聲無息的便送了性命。

只聽另一人道︰「適纔有個三十歳左右的漢子,救了這老児出來,那漢子輕功甚是了得,這會児却已不知去向,那寶刀定是給他搶去了。」那教主「{\upstsl{嗯}}」了一聲,道︰「留下這人的命。」但聽風聲颯然,出了廟門,一聲清嘯,已起於數十丈之外。兪岱岩急道︰「我在這裡,不須多傷無辜性命。」他知道教主的部屬便要對海沙派衆人施展殺手,於是從神像腹中躍出。

但海神廟中了無聲息,竟似没半個人影。兪岱岩四下一望,只見各人好端端的站著,只是一動也不動,顯得十分的陰森詭異。兪岱岩大奇,再點燃神檯上的燈燭,不禁吃了一驚,忍不住「啊」的一聲,叫了出來。原來海沙派的二十餘人一齊站著,顯是被人點中了穴道,各人臉上神色個個顯得極是可怖,燭光照射之下,饒是兪岱岩見識多廣,也不禁心中怦怦亂跳,暗想︰「那白眉教的教主不知是如何三頭六臂的人物,這些海沙派的人衆看來個個都是桀驚悍猛的梟士,但一見這教主竟嚇成這等模樣。」於是伸手到身旁那人的「華蓋穴」上一推,想替他解開穴道。

那知觸手僵硬,竟是推之不動,再一探他鼻息,早已没了呼吸,原來已被點中了死穴。他逐一探察,只見海沙派的二十餘條大漢,人人均已死於非命,只有一人委頓在地,不住喘氣,自是最後那個説話之人,得蒙教主留下性命。兪岱岩驚疑不定︰「我聽那教主説『留下這人的命』,便知情形不對,立時挺身出來。這只是一轉眼的時光,但對方竟能對二十餘人施了毒手,手法之快,實是罕見罕聞。」他扶起那没死的海沙派鹽梟來,問道︰「白眉教是什麼邪教?他們教主是誰?」連問了幾句,那人只翻白眼,神色痴痴呆呆。兪岱岩一搭他手脈,發覺他脈息紊亂,看來性命雖然留下,却已給人使重手震斷了幾處脈絡,變成了不會説話、不會轉念的白痴。

這時兪岱岩不驚反怒,心想︰「何物白眉教,下手竟是這般毒辣殘酷?」但想對方武功極高,自己單騎匹馬,實非其敵,心下略加盤算,決意先趕回武當山,請示師父,査明白眉教的來歷,然後武當七俠連袂東下,和那白眉教鬥上一鬥。他想;白眉教再厲害,自己師兄弟七人聯手,總可應付得了,總不須師父親自出馬。

但見海沙派衆人一個個死於非命,心下慘然不忍,又見廟中白茫茫的一片,猶似堆絮積雪,到處都是毒鹽,心想︰「這群人不做好事,到頭來惡人還有惡人磨,但屍橫枯廟,只怕不知情由的百姓闖了進來,再遭禍殃。」於是撿起兵刃,在廟後的菜地挖了一個大坑,將屍首一一放入。他搬動屍首時小心翼翼,唯恐不小心沾上毒鹽,或是將毒鹽吸入肺中,搬了十餘人後,再提起一人時,突然身上向前微微一俯。

只覺這人身子重得出奇,但瞧他也只是普通身材,並非魁梧奇偉之輩,何以如此沉重?兪岱岩提起他身子一看,見他背上長長一條傷口,忙探手到傷口中一摸,著手冰涼,取出一把刀來。那刀沉甸甸的至少有一百來斤,正是許多人捨生忘死、拚了性命爭奪的那把屠龍寶刀。原來海東青德成斗然間見到白眉教教主,心中向來震於他的威名,一驚之下,魂飛膽裂,竟爾嚇死,那屠龍刀從手中跌將下來,砍入海沙派一名鹽梟的後心。只因此刀既沉重,刀鋒鋭,一跌之下,直没入體。白眉教教主的下屬搜索各人身邉時,自是不能發覺,若非兪岱岩一念之善,埋葬被害各人的屍體,説不定這柄震撼武林的屠龍寶刀,就此湮没無聞了。

兪岱岩拄刀而立,四顧茫然,尋思︰「此刀雖然是武林至寶,但我看來,實是不祥之物,海東青德成和海沙派這許多鹽梟,個個爲它枉送了性命。眼下只有拿去呈給師父,請他老人家發落。」

於是將德成及衆鹽梟屍體抛入坑中,生怕廟中毒鹽飛揚,爲害人畜,索性放一把火,將那海神廟燒了。他將屠龍刀拂拭乾淨,在熊熊大火之旁細看,但見那刀烏沉沉的,非金非鐵,不知是何物所製,自刀頭以至刀柄,隱隱有一道碧痕。他眼見長白三禽鼓起烈火鍛鍊,但此刀竟是絲毫無損,實是異物,心下又想︰「此刀如此沉重,臨敵交手之時,如何施展得開?便説関王爺神力過人,他的青龍偃月刀也只八十一斤。」於是珍重包入包袱,向德成的葬身處默祝道︰「德老丈,我並非覬覦此刀。但屠龍刀乃天下異物,如落入惡人手中,助紂爲虐,貽禍人間。我師父至大至公,他老人家必有妥善處置。」

他將包袱背在背上,邁開步子,向北疾行。走不到半個時辰,已至江邉,星月微光照映水面,點點閃閃,宛似滿江繁星,放眼而望,四下裡並無船隻。兪岱岩沿江東下,又走一頓時分,只見前面燈火閃爍,有一隻漁船,在離岸十餘丈之處捕魚。兪岱岩叫道︰「打漁的大哥,費心送我過江,當有酬謝。」只是那漁船相距過遠,船上的漁人似乎没有聽見他的叫聲,竟不理睬。兪岱岩吸一口氣,縱聲而呼,他二十年的内力修爲,這叫聲遠遠傳了出去。過不多時,只見上流一艘小船沿江而下,張著風帆,順風駛到岸邉,把舵的梢公説道︰「客官要過江麼?」兪岱岩喜道︰「正是,相煩梢公大哥方便。」那梢公道︰「單放一趟,須得一兩銀子。」兪岱岩雖覺稍貴,但急於趕路,也不來跟他計較,説道︰「好罷,便是一兩銀子。」縱身一躍,跳到了船上,船頭登時向下一沉。那梢公没有防備,吃了一驚,説道︰「這般沉重,客官,你身上帶著什麼啊?」兪岱岩取出一錠銀子,交了給他,笑道︰「没什麼,是我身子蠢重,快開船吧!」那梢公一臉懷疑之色,目不轉睛的瞧著他背上包袱。

那船順風順水,斜向東北過江,行駛甚速。航出里許,忽聽遠處雷聲隱隱,轟轟之聲大作。兪岱岩道︰「梢公,要下大雨了吧?」那梢公笑道︰「這是錢塘江的夜潮,順著潮水一送,轉眼便到對岸,比什麼都快。」兪岱岩放眼東望,只見天邉一道白線,滾滾而至。潮聲愈來愈響,所謂「十萬軍聲夜半潮」,當眞是如千軍萬馬一般。他心想︰「天地間竟有如斯壯觀,今日大開眼界,也不枉了辛苦這一遭。」只見江浪洶湧,遠處一道水牆疾推而進。兪岱岩正瞧之際,不禁「咦」的一聲,只見潮峰之頂,一艘帆船乘浪衝至,那船的白帆上繪著一隻血色大手,伸開五指,似乎要迎面抓來,這情景詭異可怖,夜半斗然見到,令人不自禁的心中發毛。

兪岱岩目光鋭利,雖在黑夜之中,亦能望見數十丈外白帆上的血手,那梢公却待對面帆船駛近,方才瞧見,但見那船乘潮直撞過來,忽地尖聲驚叫︰「血\dash{}血手帆\dash{}」叫聲之中,充滿了恐怖。兪岱岩道︰「什麼血手帆?」那梢公不答,猛地一躍,跳入江水。兪岱岩大吃一驚,眼見怒潮山立,再好的水性也支持不住,急忙搶過一枝長篙,伸到江中救人。那梢公在水中搖了搖手,滿臉惶怖,便似見到了什麼食人惡鬼一般,向下一沉,潛入江心潮中,霎時間不見了影蹤。

那船無人掌舵,給潮水一衝,登時打起圏子來。兪岱岩忙搶到後梢去把舵,便在此時,那血手帆砰的一聲,撞在船上。這血手帆船的船頭包以堅鐵,一撞之下,兪岱岩所坐的小船登時破了一個大洞,潮水猛湧進來。兪岱岩又驚又怒︰「是誰這般強橫霸道?」眼見小船已不能乘坐,縱身一躍,落向血手帆船的船頭。

這時剛好一個大浪湧到,將血手帆船一抛,憑空上升丈餘。兪岱岩身在半空,帆船上升。他變成落到了船底,危急中提一口眞氣,雙臂一振,施展「梯雲縱」輕功,跟著又上竄丈餘,終於落到了帆船的船頭。

但見那船艙門緊閉,却看不見半個人影。兪岱岩叫道︰「有人落水,快快施救。」他連説兩遍,船中無人答話。兪岱岩怒氣湧上,伸手去推艙門,觸手冰涼,那艙門竟是鋼鐵鑄成,一推之下,絲毫不動。兪岱岩勁貫雙臂,大喝一聲,雙掌推出,喀喇一響,鐵門仍是不開,但鐵門與船艙邉相接的鉸鍊却給他掌力震落。那鐵門搖晃了幾下,只須再加一掌,便能擊開。

只聽得艙中一人説道︰「武當派梯雲縱輕功,震山掌掌力,果然是名下無虛。兪三俠,你把背上的屠龍刀留下,咱們便送你過江去。」這聲音溫和親厚,正是他在海神廟中所聽見過的那個白眉教教主的下屬,他想︰「原來這血手帆船是白眉教之物,因此那梢公一見,寧可干犯大險,𨂻著狂潮逃走,只是不知對方如何知道自己姓名,又知這屠龍刀是在自己手中?」

正沉吟不答,那人又道︰「兪三俠,你心中奇怪,何以咱們知道你姓名,是不是?其實一點也不希奇,這梯雲縱的輕功和震山掌的掌力,除了武當派的高手,又有誰能使得這般出神入化?兪三俠未踏入咱們浙江境内,三天前咱們已有消息,只是沿途没有接待招呼,你可得多多擔代啊。」兪岱岩聽了這番言語,仍是不知如何回答纔是,只道︰「别的事慢慢再説不遲,眼前先救那落水的梢公要緊。」那人哈哈一笑,説道︰「這梢公有個外號,叫作討債水鬼,在這錢塘江上不知己害了多少人命。兪三俠仁義過人,好心想救他,其實他早已瞧中你包袱中的銀兩,想要跟你討前世欠了他的債呢。哈哈,哈哈。」

兪岱岩瞧那梢公的神氣鬼鬼祟祟,心中早便犯疑,聽那人一説,果是如此,於是説道︰「尊駕高姓大名,便請現身一見。」那人道︰「咱們白眉教跟貴派無親無故、没冤没仇,還是不見的好,兪三俠請將屠龍刀放在船頭,咱們這便送你過江。」兪岱岩一聽之下,氣往上衝,説道︰「這屠龍刀是貴教所有的嗎?」那人道︰「這倒不是。此刀是武林至尊,天下武學之士,那一個不想據而有之。」兪岱岩道︰「這便是了。此刀既落入在下手中,在下須得交到武當山上,聽憑師尊發落。在下年輕識淺,自己可作不得主。」那人細聲細氣的説了幾句話,聲音低微,如蚊子一般,兪岱岩聽不清楚,問道︰「你説什麼?」

船艙裡那人又細聲細氣的説了幾句話,聲音更加低了,兪岱岩只聽到什麼「兪三俠\dash{}屠龍刀\dash{}」幾個字,他走上兩步,問道︰「你説什麼?」這時一個浪頭打來,將帆船直抛了上去,兪岱岩胸腹間和大腿之上,似乎同時被蚊子叮了一口。其時正當春初,本該没有蚊蚋,但他也不在意,順手在被叮處拍了兩下,朗聲説道︰「貴教爲了一刀,殺人不少,海神廟中遺屍數十,未免下手太過毒辣。」艙中那人道︰「白眉教下手向來分别輕重,對惡人下手重,對好人下手輕。兪三俠俠名震於江湖,咱們也不能害你性命,你將屠龍刀留下,在下便將蚊鬚針的解藥奉上。」兪岱岩聽到「蚊鬚針」三字,一震之下,忙伸手到胸腹間適纔被蚊子咬過的處所一按,只覺微微麻癢,明明是蚊蟲叮後的感覺,但越想越是不對,這時候那裡來的蚊蟲?又何況是在大江之上,再轉念一想,登時省悟︰「他適纔説話聲音故意糊糢細微,引我走近,於是將這極細小的暗器射入我身中。」想起海東青德成、海沙派衆鹽梟、討債水鬼各人對白眉教如此畏若蛇蝎,他這暗器之毒,定是可怕無比,眼下只有先擒住他,再逼他取出解藥救治,當下低哼一聲,左掌護面,右掌護胸,縱身便往船艙中衝了進去。

人未落地,黑暗中勁風撲面,艙中人也是一掌拍出。兪岱岩盛怒之下,這一掌使了十成力。兩人雙掌相交,砰的一聲,各自震退數步,兪岱岩没在艙中著地,跟著便被推回到了船頭,但覺手掌之下,劇痛澈骨透心。原來適纔交了這掌,又已著了人家道児,對方掌心暗藏尖刺同時穿入他肉掌之中。那人的掌力和兪岱岩似在伯仲之間,即使不使詭計,武功也不在他之下。

只聽那人斯斯文文的道︰「我這掌心七星釘,毒性另有一功,兪三俠掌力驚人,果是不凡,佩服啊佩服。」兪岱岩狂怒之下,一抖包裹,取出屠龍寶刀,雙手持柄,呼的一聲,橫掃過去,但聽得擦的一下輕響,登時將鐵門斬成了兩截,這刀看上去貌不驚人,但果然是鋒鋭絶倫。兪岱岩砍得興起,橫七豎八,連斬七八刀,鐵鑄的船艙遇著寶刀,便似紙糊草搭一般,登時摧枯拉杇,一片片掉入江中。艙中那人藏身不住,縱身往後梢一躍,叫道︰「你連中二毒,還發什麼威?」兪岱岩舞刀竄前,攔腰斬去。

那人見他勢盛,順手提起一隻大鐵錨一擋,又是擦的一聲輕響,鐵錨攔腰斬斷。那人向旁躍開,叫道︰「要性命還是要寶刀?」兪岱岩道︰「好!你給我解藥,我給你寶刀。」這時他腿上中了蚊鬚針之處漸漸麻癢,料知毒性已經發作,這把屠龍刀他是無意中得來,自己本不如何重視,捨之決不可惜,於是將刀嗆{\upstsl{啷}}一聲,擲在艙面。

那人大喜,俯身拾起,不住的拂拭摩挲,愛惜無比。那人背著月光,面貌瞧不清楚,但見他只是看刀,却不去取解藥。過了良久,兪岱岩覺得手中疼痛加劇。説道︰「我以刀換藥,解藥呢?」那人哈哈大笑,似乎聽到了滑稽之極的説話。兪岱岩怒道︰「我問你要解藥,有什麼好笑?」那人伸出左手食指,指著他臉,笑道︰「嘻嘻,嘻嘻!你這人怎地這般傻,不等我你給你解藥,却先將寶刀給了我?」兪岱岩怒道︰「男児一言,快馬一鞭,我答應以刀換藥,難道還抵賴不成?先給遲給不是一般?」那人笑道︰「你手中有刀,我終是忌你三分。便説你打我不過,將刀往江中一抛,未必再撈得到。現下寶刀既入我手,你還想我用解藥救你嗎?」兪岱岩一聽,一股涼氣從心底直冒上來,自忖武當派和白眉教無冤無仇,這人武功不低,也當是頗有身份之人,既取了屠龍刀,怎能説過的話不算話?

只聽那人又道︰「兪三俠,有一件事你不可不知,在下這蚊鬚針倒還罷了,這七星針中的毒性却當眞有點児厲害。十二個時辰之内,你全身肌肉要片片跌落,耳鼻手足,無一得全,除了本教獨有的解藥之外,縱然是大羅金仙下凡,也是無法相救。但就算給了本教的解藥給你,也只能救得不死,你兪三俠一身天下知聞的絶世武功,可就此永不能復了。」這番話説得宛轉親切,娓娓動聽,便似是至交好友良言相勸一般。

兪岱岩沉住了氣,説道︰「大丈夫生死有命,我兪岱岩一生行事光明磊落,無愧於天地,縱然命喪小人之手,有何足懼?」那人大拇指一翹,讚道︰「好,好!武當七俠果然是名下無虛,中了我這七星釘、蚊鬚針的英雄好漢,世間不計其數,但不是哀哀求告,便是放聲大哭,就算是最有骨氣的,也只是破口大罵,如兪三俠這般將生死置之度外,鎭定如恒的,在下實不多見。」兪岱岩哼的一聲,道︰「尊駕高姓大名,可能見告否?」那人笑道︰「在下只是白眉教中的一個無名小卒,武當派若要找白眉教報仇,自有教主出面。再説,兪三俠今晩死得不明不白,貴派張三丰祖師便眞有通天徹地之能,也未能眞知兪三俠是死於白眉教之手。」他這般説法,竟如算定兪岱岩此時非死不可。兪岱岩只覺手掌心似有千萬隻螞蟻同時咬噬,痛癢難當,暗暗伸手抓住了半截斷錨,心想︰「我今日便是不活,也當和你拚個同歸於盡。」

但聽那人{\upstsl{嘮}}{\upstsl{嘮}}{\upstsl{叼}}{\upstsl{叼}},正自説得高興,兪岱岩猛裡一聲大喝,縱起身來,左手揮起斷錨,右手推出一掌,往那人面門胸口,同時擊了過去。他自知已然無倖,但決計不肯出聲求討解藥,這一下是臨死之前的一擊,威力何等驚人,那人啊喲一聲,橫揮屠龍刀想來擋截,百忙中却没想到那屠龍刀沉重異常,尋常刀劍十餘把加在一起也没它重,他順手一揮,只揮出半尺,手腕忽地一沉。以他武功,原非使不動這把屠龍刀,只是運力之際,没估量到這兵刃竟是如此沉重,因此力道用得歪了,那刀直墜下去,斫向他的膝蓋。那人吃了一驚,臂上使力,待要將刀挺舉起來,只覺勁風撲風,半截斷錨直擊過來。這一下威猛凌厲,他武功雖強,却也無法抵擋,只暗叫一聲︰「我命休矣!」只好束手待斃,豈知便在這時,怒潮中一個大浪如山般推到,那帆船一顚一抛,斷錨掃去的準頭登時歪了,那人雙足一使勁,一個筋斗,倒翻入江。

他雖然避開了斷錨的橫掃,但兪岱岩右手那一掌却終於没有讓過,這一掌正按在他小腹之上,但覺五臟六腑一齊翻轉,{\upstsl{噗}}通一聲跌入潮水之中,已是人事不知。兪岱岩吁了一口長氣,見他雖然中掌,兀自牢牢的握住那屠龍寶刀不放,冷笑一聲,心道︰「你便是搶得了寶刀,終於葬身江底。」

驀地裡白光一閃,一道白練斜入江心,捲住那人的頭頸,扯了上來。兪岱岩吃了一驚,順著白練的去路瞧去,只見一艘小船的船頭站著一個白衫瘦子,手中持著那條白練,連人帶刀一起捲上船來。兪岱岩中毒釘後全神貫注於那人身上,竟没覺察他暗中到了後援,這小船駛近,事前也没留心。

船頭的白衫瘦子一聲呼叱,所乘小船靠到了帆船之旁,那瘦子身形一起,如一隻白燕般躍上了來。這時兪岱岩身上毒性發作,全身癱瘓,倒在船梢,眼見敵人上來,想要揮掌迎敵,却連站立也有所不能,心中一急,眼前一黑,登時昏迷了過去。也不知過了多少時候,睜開眼來時,首先見到的是一面鏢旗,旗上繡著一尾金色鯉魚。

\chapter{黃金保鏢}

兪岱岩閉了閉眼,再睜開來時,仍是見到這面小小的鏢旗。這鏢旗插在一隻青花碎瓷的花瓶之中,花繡金光閃閃,旗上的鯉魚在波中騰身而躍,顯得甚是威武。兪岱岩心想︰「這是臨安府龍門鏢局的鏢旗啊。」其時心中一片混亂,没法多想,略一凝神,發覺自己是睡在一個擔架上面,前後有人抬著自己,而所處之地却似是在一座大廳之中。他想轉頭一瞧左右,豈知項頸僵直,竟是不能轉動。他大駭之下,想要躍下擔架,但雙手雙足竟似變成了不是自己的,空自使力,竟是一動也不動了,這纔想到︰「我是在錢塘江上中了七星釘和蚊鬚針的劇毒。」

只聽得有兩個人在説話,一個人聲音宏大説道︰「閣下高姓?」另一人道︰「你不用問我姓名,我只問你,這單鏢接是不接?」兪岱岩心下奇怪︰「聽這人聲音嬌嫩,似是女子啊!」那聲音宏大的人怫然道︰「我龍門鏢局難道少了生意,閣下既然不肯將姓名見告,那麼請光顧别家鏢局去吧。」那女子聲音的人道︰「臨安府只有龍門鏢局還像個樣子,别家鏢局都比不上。你若作不得主,快去叫總鏢頭出來。」聽這人説話頤指氣使,極不禮貌。那聲音宏大的人果然很不開心,説道︰「我便是總鏢頭。在下另有别事,不能相陪。尊客請便吧。」那女子聲音的人道︰「啊,你便是多臂熊都大錦\dash{}」頓了一頓,纔道︰「都總鏢頭,久仰久仰,我姓殷。」都大錦胸中似略舒暢,問道︰「尊客有什麼差遣?」那姓殷的客人道︰「我得先問你,你不是承擔得下。這件事非同小可,却是半分耽誤不得。」都大錦強抑怒氣,説道︰「我這龍門鏢局開設二十年來,官鏢、鹽鏢、金銀珠寶,再大的生意也接過,可從來没出過半點岔子。」

兪岱岩也聽過都大錦的名頭,知道他是少林派的俗家弟子,拳掌單刀,都有獨得的造詣,尤其一手連珠鋼鏢,能將七七四十九枚鋼鏢毫不停留的施放,百發百中,因此江湖上送了他一個外號,叫作多臂熊。他這「龍門鏢局」在東南一帶也是頗有威名。只是武當派和少林派兩派弟子相互間自來並不親近,因此雖然聞名,却不相識。

只聽那姓殷的微微一笑,説道︰「我若不知龍門鏢局信用不差,找上門來幹麼?都總大鏢頭,我有一單鏢交給你,可有三個條件。」都大錦道︰「牽扯糾纏的鏢咱們不接,來歷不明的鏢不接,五萬兩銀子以下的鏢不接。」他没聽對方説三個條件,自己却開口先説了三個條件。那姓殷的道︰「我這單鏢啊,對不起得很,可有點児牽扯糾纏,來歷也不大清白,只怕更加値不上五萬兩銀子。我這三個條件也不挺容易辦到。第一,要請你都總鏢頭親自押送。第二,自臨安府送到湖北襄陽府,必須日夜不停趕路,十天之内送到。第三,若是有半分差錯,嘿嘿,别説你都總鏢頭性命不保,你龍門鏢局勢必給人家殺得雞犬不留。」

只聽得砰砰一聲,想是都大錦伸手拍桌,喝道︰「你要找人消遣,也不能找到我龍門鏢局來!若不是我瞧你瘦骨伶仃的,身上没三兩肉,今日先要叫你吃些苦頭。」那姓殷的「嘿嘿」兩聲冷笑,砰砰二下,將什麼東西抛到了桌上,説道︰「這是二千兩黃金,算是保鏢的費用,你先收下了。」兪岱岩聽了,心下一驚︰「二千兩黃金,要値得十幾萬兩銀子,做鏢局的値百抽四,這十幾萬兩的鏢金,做十年也未必掙得起。」都大錦見了這許多金光燦爛的黃金,果然神氣登時不同。他開鏢局的,大批的金銀雖經常見到,但看來看去,總是别人的財物,這時突然有二千兩黃金送到面前,只要自己一點頭答應,這二千兩黃金就是自己的,却教他如何不動心?

兪岱岩頭頸不能轉動,眼睜睜的只能望著那面插在瓶中的躍鯉鏢旗,這時大廳中一片靜寂,唯見營營青蠅,掠面飛過。只聽得都大錦喘息之聲甚是粗重,兪岱岩雖不能見他臉色,但猜想得到,他定是望著桌上那金光燦爛的二千兩黃金,目瞪口呆,心搖神馳,過了半晌,聽得都大錦道︰「殷大爺,你要我保什麼鏢?」那姓殷的道︰「我先問你。我定下的三個條件,你可能辦到?」都大錦頓了一頓,伸手在自己大腿上一拍,道︰「殷大爺既出了這等重酬,我姓都的跟你賣命便是了,殷大爺的寶物幾時送來?」那姓殷的道︰「要你保的鏢,便是躺在擔架中的這位爺台。」

此言一出,都大錦固然「咦」的一聲,大爲驚訝,而兪岱岩更是驚奇無比,忍不住叫道︰「我\dash{}我\dash{}」那知他張大了口,却發不出聲音,便似人在噩夢之中,不論如何使力,周身却不聽使喚,這纔知那七星釘的劇毒實是歹毒無倫,不但肢體癱瘓,連喉音也給毒啞了,僅餘下眼睛未盲,耳朶未聾。只聽都大錦問道︰「是\dash{}是這位爺台?」

那姓殷的道︰「不錯。你親自護送,換車換馬不換人,日夜不停的趕道,十天之内送到湖北襄陽府武當山上,交給武當派掌門祖師張三丰先生。」兪岱岩聽到這句話,吁了一口長氣,心中一寬。聽都大錦道︰「武當派?咱們少林弟子,雖和武當派没什麼樑子,但是\dash{}但是,從來没什麼來往\dash{}這個\dash{}」那姓殷的冷冷的道︰「耽誤片刻,萬金莫贖。這單鏢你接便接,不接便不接。大丈夫一言而決,什麼這個那個的?」都大錦道︰「好,衝著殷大爺的面子,我龍門鏢局便接下了。」那姓殷的微微一笑,道︰「今日是三月廿九,四月初九午時,你若不能將這位爺台平平安安送到武當山,我叫你龍門鏢局大小七十一口,滿門雞犬不留!」但聽得嗤嗤數聲,十餘枚細小的銀針激射而出,釘在那隻插著鏢旗的瓷瓶之上,砰的一響,瓷瓶裂成數十片,四散飛迸。

這一手發射暗器的功夫,實是駭人耳目,都大錦「啊喲」一聲驚呼,兪岱岩也是心中一凜,只聽那姓殷的喝道︰「走吧!」抬著兪岱岩的人將擔架放在地下,一擁而出。

過了半晌,都大錦纔定下神來,走到兪岱岩跟前,説道︰「這位爺台高姓大名?可是武當派的麼?」兪岱岩只有向他凝望,無法回答。但見這鏢頭約莫五十來歳年紀,身材魁偉,手臂上肌肉虯結,相貌威武,顯是一位外家高手。都大錦又道︰「這位殷大爺俊秀文雅,想不到武功如此驚人,却不知是那一家那一派的門人?」他連問數聲,兪岱岩索性閉上雙眼,不去理他。

都大錦心下{\upstsl{嘀}}咕,他自己是發射暗器的好手,「多臂熊」的外號在江湖上説出來也甚是響亮,但姓殷的美貌少年袖子一揚,數十枚細如牛毛的銀針竟將一隻大瓷射得粉碎,這份功夫,若不是親眼得見,旁人便是説了自己也決不相信。他走到几旁,撿起碎瓷片來看時,只見一枚枚銀針刺入瓷中,便似用鐵鎚鎚入一般,眞不知他用怎麼樣的手勁纔打成這般模樣。

都大錦主持這龍門鏢局已有二十餘年,爲人精明強幹,江湖上的大風大浪也不知經過了多少,但以二千兩黃金的鏢金來託保一個活人,别説自己手裡從未接過,便是天下各處的鏢行,也没碰到過這般奇事。當下拿起黃金,命人抬兪岱岩入房休息,隨即招集鏢局中各位鏢頭,套車趕馬,即日上道。

他暗中與兩位年高鏢頭一商議,都覺那姓殷客人臨去時所説︰「我叫你龍門鏢局大小七十一口,滿門雞犬不留。」這句話,實在大是凶險。三個人屈指一算,自都大錦的老母親數起,數到祝鏢頭初生未滿月的孩児,以至灶下燒火挑水的小厮,不多不少,剛好是七十一口。三個人怔怔相對,不自禁的心驚肉跳。

祝鏢頭道︰「總鏢頭,不是做兄弟的多口,我瞧這單鏢鏢金雖重,但前途危難重重,倒不如不接的好。」另一位史鏢頭道︰「祝三哥這話説得太遲啦,鏢都了接下來,難道憑著咱們龍門鏢局二十年的威名,還能把這單鏢退回給人家不成?」祝鏢頭怒道︰「龍門鏢局二十年的威名,史五弟可惜,我姓祝的便不可惜了?只是這件事中間處處透著邪門,安知人家不是故意擺佈咱們來著。」史鏢頭冷笑道︰「既然吃了這一行保鏢的飯,日日夜夜,便得在刀尖子上討生活。祝三哥要太平無事,該當在家裡抱著娃娃别出門啊。」

二人你一言我一語,説著便爭吵起來。都大錦勸道︰「兩位别嚷。事已如此,常言道兵來將擋,水來土淹,咱們鏢局傾巢而出,保這單鏢到武當山去。祝三哥不放心嫂子孩児,那也慮得是,咱們就把鏢局子的老小都送到鄕下,那也不是膽小怕事,這叫做以策萬全。」當下分派人手,護送老小到臨安府之西的鄕下去暫避。

各人飽餐已畢,結束定當,趟子手抱了鏢局的躍鯉鏢旗,走出鏢局大門,一展旗子,大聲喝道︰「龍門鯉三躍,魚児化爲龍。」兪岱岩躺在一輛大車之中,心下大是感慨︰「我兪岱岩蹤橫江湖,生平没將保鏢護院的漢子瞧在眼内,想不到今日遭此大難,却要他們護送我到武當山去。」又想︰「救了我的這姓殷朋友不知是誰,聽他聲音嬌嫩,似乎是個女子,那總鏢頭又説他形貌俊雅,但武功卓絶,行事出人意表,只可惜我不能見他一面,更不能謝他一句。我兪岱岩若能不死,此恩必報。」

耳聽得車聲轔轔,將出城門,忽聽得都大錦大聲道︰「怎地你們又回來啦?我叫你們千萬不可回臨安來的。」只聽一人道︰「回稟總\dash{}總鏢頭咱們三\dash{}三隻耳朶\dash{}」都大錦又驚又怒,喝道︰「怎麼搞成這個樣子?你們三個的耳朶怎麼給割去的?」那人道︰「咱們\dash{}咱們護送老太太們出城,没走到二里地,就給人攔住啦。那些人兇神惡煞一般,説道︰『龍門鏢局的家小,不許離臨安城一步。』小的跟他爭辯,那人拔出刀來,便割去了小的一隻耳朶,他們\dash{}他們兩個耳朶,也都割去了。那人叫小的回來稟告總鏢頭,説這單鏢若不是依時送到,什麼\dash{}什麼雞犬不留。」都大錦嘆了口氣,知道暗中早給人家嚴密監視上了,右手一揮,説道︰「你們回去吧,好好在鏢局子中耽著,没事就别出門。」鞭子一揮,縱馬前行。

一行人馬不停蹄的向西趕路,護鏢的除了都、祝、史三個鏢頭外,另有四個年青力壯的少年鏢師。各人選的都是快馬,眞便如那姓殷的所説,一路上換車換馬不換人,日夜不停的攢趕路程。各人心中都是懷著鬼胎,均知若有半分錯差,自己送了性命不説,臨安府鏢局中合門老小,無一能彀活命。

當出臨安西門之時,都大錦滿腹疑慮,料得到這一路上不知要有幾十場出生忘死的惡鬥,那知道離浙江、過安徽、入鄂省,數日來竟是太平無事,别説江湖好手,綠林豪客,連小毛賊也没遇上一個。這一日過了樊城,經太平店、仙人渡、光化縣,渡漢水來到老河口,到武當山已不過一日之程。

都大錦等這日未到午牌時分,已抵雙井子,眼見上武當山已不過半日之程,一路上雖然趕得辛苦,但總算没有誤了那姓殷客人所規定的期限,剛好於四月初九抵達武當山。這些日來,埋頭趕路,大夥児你瞧瞧我,我瞧瞧你,雖然口中没説什麼,却是人人都擔著極重的心事。直到此時,一衆鏢師方才心中大寬。

其時正當春末夏初,天時和暖,山道上繁花迎人,殊足暢懷。都大錦伸馬鞭指著隱入雲中的天柱峰道︰「祝三弟,近年來武當派聲勢甚盛,雖還及不上我少林派,然而武當七俠名頭響亮,居然在江湖上闖下了挺喧赫的萬児。瞧這天柱峰高聳入雲,常言道人傑地靈,那武當派看來當眞有幾下子。」祝鏢師道︰「武當派近年來聲威雖大,究竟根基尚淺,跟少林派千餘年的道行相比,那可眞是萬萬不及了。就憑總鏢頭這二十四手降魔掌和四十九枚連珠鋼鏢,武當派中的人物便決不能有如此的精純造詣。」史鏢頭接口道︰「是啊。江湖上的傳言,多半靠不住。武當七俠的聲名響是響的,但眞實功夫到底如何,咱們都没有見過。只怕是武林中那些没見過世面的鄕下佬加油添醬,將他們的本領越傳越是厲害。」都大錦微微一笑。他的見識可比祝史二人要高得多,心知武當七俠的盛名絶非倖致,人家定是有他的驚人藝業,只是他走鏢二十餘年,罕逢敵手,對自己的功夫却也十分信得過,聽祝史二人一吹一唱的替自己捧場,這些話雖也不知聽了多少遍,但總是不自禁的得意。

三人並轡而行,山道漸漸狹窄,三騎馬已不能併肩,史鏢頭勒馬退後幾步。祝鏢頭道︰「總鏢頭,待會咱們見到武當派的張三丰老道,跟他怎生見禮啊?」都大錦道︰「咱們不同門派,不相歸屬。只是張老道已九十餘歳,當今武林之中,年紀是算他最長的了。咱們尊重他是武林前輩,向他磕幾個頭,也没什麼。」祝鏢頭道︰「依我説嘛,咱們大聲説道︰『張眞人,晩輩們跟你磕頭啦!』他一定伸手攔住,説︰『遠來是客,不用多禮。』那麼咱們這幾個頭便可以省下啦。」都大錦嘴一動,微微笑了笑,他心中却是在琢磨大車中躺著的那個兪岱岩,到底是什麼來歷。這人十天來不言不動,飲食便溺,全要鏢行中的趟子手照料。都大錦和衆鏢師談論了好幾次,總是摸不準他的身份,到底他是武當派的弟子呢?是朋友呢?還是武當派的仇敵,給人擒住了這般送上山去?都大錦離武當山近一步,心中的疑團便深了一層,尋思不久便可見到張三丰,這疑問一見面就可剖明,但是禍是福,心中却也不免惴惴。

正沉吟間,忽聽得西首山道上馬蹄聲響,有數匹馬奔馳而至。祝鏢師雙腿一挾,縱馬衝上前去察看。過不多時,只見斜刺裡奔來六乘馬,馳到離鏢行人衆十餘丈處,突然勒馬,三乘馬在前,三乘馬在後,攔在當路。都大錦心下{\upstsl{嘀}}咕︰「眞不成到了武當山脚下,反而出事?」低聲對史鏢頭道︰「小心保護大車。」自己拍馬迎上前去。只見趟子手將躍鯉鏢旗一捲一揚,作個敬禮的姿式,説道︰「龍門鏢局道經貴地,禮數不週,請好朋友們原諒。」都大錦看那攔路的六人時,見兩人是黃冠道士,其餘四人是俗家打扮。六人身旁都懸佩刀劍兵刃,個個英氣勃勃,精神飽滿。都大錦心念一動︰「這六人豈非便是武當七俠中的六俠?」於是縱馬上前,抱拳説道︰「在下是龍門鏢局都大錦,不敢請問六位兄長的高姓大名?」六人中最右首的是個高個児,左頰上生著顆大黑痣,痣上留著三莖長毛,他向都大錦冷冷的道︰「都兄到武當山來幹什麼?」都大錦道︰「敝局受人之託,送一位傷者上貴山來。要面見貴派掌門張眞人。」那面生黑痣的人道︰「送一個傷者?他人呢?那是誰啊?」

都大錦道︰「咱們是受一位姓殷的客官所囑,將這位身受重傷的爺台護送上武當山來。這位爺台是誰,他如何受傷,中間過節,咱們一槩不知。龍門鏢局是受人之託。忠人之事,至於客人們的私事,咱們向來不加過問。」他闖蕩江湖數十年,幹的又是鏢行,行事自是非圓滑不可,這一番話把干係推得乾乾淨淨,兪岱岩是武當派的朋友也好,是仇人也好,都怪不到他頭上。

那臉生黑痣之人向身旁兩個同伴瞧了一眼,説道︰「姓殷的?是怎生模樣的人物?」都大錦道︰「那是一位俊雅秀美的年輕客官。發射暗器的功夫大是了得。」那生黑痣的人道︰「你跟他動過手了?」都大錦忙道︰「不,不,是他自行\dash{}」一句話還没説完,攔在前面的中間一人搶著道︰「那屠龍刀呢?是在誰的手中?」都大錦愕然道︰「什麼屠龍刀?便是歷來相傳那『武林至尊,寶刀屠龍』麼?」中間這人性子極是暴躁,不耐煩跟他多講,突然翻身落馬,搶到大車之前,挑開車簾,向内張望。

都大錦見他身手矯捷,一縱一落,姿式甚是柔和,心下更無懷疑,問道︰「各位便是名播江湖的武當七俠麼?那一位是宋大俠?小弟久聞英名,甚是仰慕。」那面生黑痣的人道︰「區區虛名,何足掛齒,都兄亦太謙了。」那瞧過兪岱岩的人回身上馬,説道︰「他傷勢甚重,片刻也耽誤不得,咱們先接了去。」那臉生黑痣的人向都大錦抱拳道︰「都兄遠來勞頓,大是辛苦,小弟這裡謝過。」都大錦拱手還禮,道︰「好説,好説。」那人道︰「這位爺台傷勢不輕,耽擱不起,咱們先接上山去施救。」都大錦巴不得早些脱却干係,説道︰「好,那麼咱們在這裡把人交給武當派了。」那人道︰「都兄放心,由小弟負責便是。都兄的鏢金已付清了麼?」都大錦道︰「早已收足。」那人從懷中取出一隻金元寶,約有百兩之譜,長臂伸出,説道︰「這些茶資,請都兄賞給各位兄弟。」都大錦推辭不受,説道︰「二千兩黃金的鏢金,説什麼都彀了,我都大錦並非貪得無厭之人。」那人道︰「{\upstsl{嗯}},給了二千兩黃金。」他身旁二人縱馬上前,一人躍上車夫的座位,接過馬韁,趕車先行,其餘一人護在車後。

那面生黑痣的人手一揚,輕輕將金元寶擲到都大錦面前,笑道︰「都兄不必客氣,這便請回臨安去吧!」都大錦見元寶擲到面前,只得伸手接住,待要送還,那人勒過馬頭,急馳而去。只見五乘馬擁著一輛大車,轉過山坳,片刻間去得不見了影蹤。都大錦看那元寶,見上面捏出十個指印,深入半寸,連指紋幾乎也可辨别。黃金雖較銅鐵柔軟得多,但這般指力,實令人不勝駭異。都大錦呆呆的望著,心道︰「武當七俠的大名,果然不是僥倖得來。我少林派中,只怕只有圓音、圓心各位精研金剛指力的師叔,方有如此功力。」

祝鏢頭見他拿著那只元寶,瞪視金錠上的指印,呆呆出神,説道︰「總鏢頭,武當門下的子弟,未免不明禮數,見了面既不通問姓名,咱們千里迢迢的趕來,到了武當山脚下,又不請上山去留膳留宿,大家武林一脈,可太不彀朋友啦。」都大錦心中早就不滿,只是没説出口來,當下淡淡一笑,道︰「省了咱們幾步路,那不好麼?少林弟子進了武當派的道觀之中,原是十分{\upstsl{尷}}尬。兩位賢弟,打道回府去吧!」

這一趟走鏢,雖然没出半點岔子,但事事被蒙在鼓裡,而有意無意之間,又是處處受人折辱,武當七俠連姓名也不肯説,顯是絲毫不將他放在眼内,都大錦越想越是不忿,暗自盤算如何才能出這一口惡氣。

一行人衆原路而回,都大錦雖然心中不快,衆鏢師和趟子手却是人人興高采烈,想起十天十夜辛辛苦苦,換來了二千兩黃金的鏢金,總鏢頭向來出手慷慨,弟兄們定可分到豐厚的一筆花紅謝禮。

行到向晩,離雙井子已不過十餘里路程,祝鏢頭見都大錦神情鬱鬱,説道︰「總鏢頭,今日此事,那也不必介懷,山高水長,江湖上他年總有相逢之時,瞧武當七俠的威風,又能使得到幾時?」都大錦嘆道︰「祝賢弟,有一件事,爲兄的心中好生懊悔。」祝鏢頭道︰「什麼事?」説到此處,忽聽得身後馬蹄聲響,有一乘馬自後趕來。這蹄聲並不甚急,相反的却比尋常馬匹緩慢的多,只聽蹄聲得得,行得甚是悠閒,但説也奇怪,那馬却越追越近。衆人都覺奇怪,回頭一瞧,原來那馬四條腿特别長大,身子較之尋常馬匹,幾乎高了兩尺,腿一長,自然走得快了。那馬是匹青驄,遍體油毛。祝鏢頭讚了句︰「好馬!」又問︰「總鏢頭,咱們没什麼幹得不對啊?」

都大錦黯然道︰「我是説二十五年前的事。那時我在少林寺中學藝,已學了十二年滿師。恩師圓業禪師留我再學五年,把一套大金剛掌學全了。當時我年少氣盛,自以爲憑著當時的本事,已足以在江湖上行走,不耐煩再在寺中吃苦,不聽恩師的勸告。唉,當年若是多下五年苦功,今日那裡把什麼武當七俠放在眼内,也不致受他們這番羞辱了\dash{}」説到此處,那騎青驄馬從鏢隊身旁掠過,馬上乘者斜眼向都大錦和祝鏢頭打量了幾眼,臉上有詫異的神色。

都大錦見有生人行近,當即住口,見馬上乘客是個二十一二歳的少年,面目俊秀,雖然略覺清臞,但神朗氣爽,身形的瘦弱竟掩不住一股驃悍之意。那少年抱拳道︰「借光,借光。」他胯下的青驄馬邁開長腿,越過鏢隊,一直向去前去了。

都大錦望著他的後影,道︰「祝賢弟,你瞧這是何等樣的人物?」祝鏢師道︰「他從山上下來,説不定也是武當派的弟子了。只是他没帶兵刀,身子又這般瘦弱,似乎不似是練家子模樣。」剛説了這幾句話,那少年突然圏轉馬頭,奔了回來,遠遠抱拳道︰「勞駕!小弟有句話動問,請勿見怪。」都大錦見他説得客氣,於是勒住了馬,道︰「尊駕要問什麼事?」那少年望了望趟子手中高舉著的躍鯉鏢旗,道︰「貴局可是臨安府龍門鏢局麼?」祝鏢頭道︰「正是!」那少年道︰「請問幾位朋友高姓大名?貴局都總鏢頭可好?」祝鏢頭雖見他彬彬有禮,但江湖上人心難測,不能逢人便吐眞言,説道︰「在下姓祝。朋友貴姓?和敝局都總鏢頭,可是相識?」那少年翻身下鞍,一手牽韁,走上幾步,説道︰「在下姓張,賤字翠山。素仰貴局都總鏢頭大名,只是無緣得見。」

他這一報名自稱「張翠山」,都大錦和祝、史二鏢頭都是一震。要知張翠山在武當七俠中名列第五,近年來武林中多有人稱道他的大名,均説他武功極是了得,想不到竟是這麼一個文質彬彬弱不禁風的少年。都大錦將信將疑,縱馬上前幾步,道︰「在下便是都大錦,閣下可是江湖上人稱『鐵劃銀鉤』的張五俠麼?」那少年臉上微微一紅,道︰「什麼俠不俠的,都總鏢頭言重了。各位來到武當,怎地過門不入?今日正是家師九十壽誕之期,倘若不耽誤各位,便請上山去喝一杯壽酒如何?」都大錦聽他説得誠懇,心想︰「武當七俠人品怎地如此大不相同?那六人傲慢無禮,這位張五俠却十分的謙和可親。」於是也一躍下馬,道︰「咱們從臨安趕到襄陽,原意是要來拜見尊師張眞人的,只是\dash{}只是\dash{}没備壽禮,未免大是冒昧。」

張翠山微微一笑,道︰「大家武林一脈,都總鏢碩恁地見外。家師常説,我武當派的武功源出少林,囑咐咱們見到少林派的前輩時,須得加倍恭敬。家師若知道都總鏢頭路過山下,早遣師兄弟們一齊來恭迎了。」都大錦聽了這話,心下著惱,暗想︰「我還道你是個謙謙君子,却原來比那六個傢伙更是狡猾,笑裡藏刀,口蜜腹劍。倘若你師父眞有此言,何以那六人見了我這等無禮?你以虛假對我,我也以虛假相報便了。」於是笑道︰「武當雖説源出少林,但青出於藍而勝於藍,張少俠年紀輕輕,江湖上誰不仰慕?似老朽這般,眞可説年紀活在狗身上了。」張翠山道︰「都總鏢頭當眞太謙了。這龍門躍鯉的鏢旗一揚開,誰不大拇指一翹,説道二十四手降魔掌、四十九枚連珠鋼鏢非同小可。這幾位大哥尊姓大名,相煩都總鏢頭引見。」都大錦聽他這般説,於是替祝、史等幾位鏢頭都引見了。張翠山道︰「祝鏢頭一柄金刀,當年在信安道上獨敗弋陽五雄,史鏢頭以十八路三義棍馳名武林,今日一見,眞是幸會。」原來這張翠山極得師父張三丰的寵愛,平日常聽師父講論江湖上的遺聞軼事。他記性極好,任何瑣屑小事,一聽過便記在心中,久久不忘。張三丰活到了九十歳年紀,交遊遍天下,還能有什麼掌故不知道?因此上張翠山年紀雖輕,各家各派的事故,幾乎説得上無一不知,這時一聽到祝史二鏢頭的名字,隨口便將他們生平最得意的事説了出來。

都大錦數十年來薄有名望,張翠山知道他的拿手絶技,也不算希奇,但祝史二鏢頭是第四五流的脚色,在張翠山口中説來,竟是素來仰慕一般,祝史二人自是心中大悦。史鏢頭道︰「總鏢頭,武當山張眞人是當今武林中的泰山北斗,今日適逢他老人家大壽,咱們上山去磕幾頭也是該當的。」張翠山道︰「磕頭是不敢當。各位路經武當,咱們應該一盡地主之誼。我幾個師哥師弟都愛朋友,各位上山去盤桓一宵吧。」

都大錦心中起疑︰「怎地連祝史兩人的武功來歷,你也知道得一清二楚,其中必有蹊蹺。又莫非適纔那六人對我無禮,受到了師長責備,因此命他趕來陪禮相邀?」想到了此處,心中舒暢了些,笑道︰「倘若令師兄們也如張五俠這般愛朋友,咱們這時早在武當山上了。」張翠山道︰「怎麼?總鏢頭見過我師兄了?是那一個?」都大錦心想︰「你這人眞會做戲,到這時還在假作痴呆。」説道︰「在下今日運氣不差,一日之間,武當七俠人人都會過了。」張翠山更加奇怪,「啊」的一聲,呆了一呆,道︰「我兪三哥你也見到了麼?」都大錦道︰「兪岱岩兪三俠麼?他們都不屑跟我通姓道名,我也不知那一位是兪三俠。只是六個人一起見了,兪三俠總也在内。」

張翠山道︰「六個人?這可奇了?是那六個啊?」都大錦怫然道︰「你這幾位師兄不肯説出姓名,我怎知道?閣下既然是張五俠,那麼那六位自然是宋大俠以至莫七俠六位了。」他説到「俠」字,都是頓了一頓,聲音拖長,頗含譏諷之意。但張翠山思索著這件奇事,並没察覺,道︰「都總鏢頭當眞見了?」都大錦道︰「不但是我見了,我這鏢行一行人數十對眼睛,一齊都見了。」張翠山搖頭道︰「那決計不會。宋師哥他們今日一直在山上玉虛宮中侍奉師父,没下山一步。師父和宋師哥見兪三哥過午後還不上山,命小弟下山等候,怎地都總鏢頭會見到宋師哥他們?」都大錦道︰「那位臉頰上生了一顆大黑痣,痣上有三莖長毛的,是宋大俠呢,還是兪三俠?」

\chapter{六俠尋仇}

張翠山一楞,道︰「我師兄弟之中,並無一人頰上有痣,痣上生毛。」都大錦聽了這幾句話,一股涼氣從心底直冒上來,説道︰「那六人自稱是武當六俠,既在武當山下現身,其中又有兩個是黃冠道人,咱們自然\dash{}」張翠山微笑道︰「我師父雖是道人,但他所收的却都是俗家弟子。那六人自稱是『武當六俠』麼?」都大錦回思適時情景,這纔想起,是自己一上來便把那六人當作是武當六俠,對方可從無一句自表身份之言,只是對自己的誤會没加否認而已,不由得和祝史二鏢頭面面相覷,隔了半晌,纔道︰「如此説來,這六人只怕不懷好意,咱們快追!」説著翻身上馬,迴過馬頭,向武當山直追而去。

張翠山也跨上了青驄馬。那馬邁開長腿,不疾不徐的和都大錦的坐騎齊肩而行。張翠山道︰「那六人混冒姓名,都兄便由得他們去吧!」都大錦氣喘喘的道︰「可是那人呢?俺受人重囑,要將那人送上武當山交給張眞人\dash{}這六人假冒姓名,接了那人去,只怕大事要糟\dash{}」張翠山道︰「都兄送誰來給我師父?那六人接了誰去?」

都大錦催馬急奔,一面將如何受人囑託,送一個身負重傷之人來到武當的事説了。張翠山頗爲詫異,問道︰「那受傷之人是什麼姓名?年貌如何?」都大錦道︰「也不知他姓甚名誰,他傷得不會説話,不能動彈,只剩下一口氣了。這人約莫三十來歳年紀。」跟著一説兪岱岩的相貌模樣。張翠山大吃一驚,叫道︰「這\dash{}這\dash{}便是我兪三哥啊。」他雖心中慌亂,但片刻間隨即鎭定,左手一伸,勒住了都大錦的馬韁。

那馬奔得正急,被張翠山這麼一勒,竟是硬生生的斗地停住,再也上前不得半步,嘴邉鮮血長流,大是痛楚,忍不住縱聲而厮。都大錦斜身落鞍,刷的一聲,拔出了單刀,心下暗自驚疑,瞧不出此人身形廋弱,這一勒之下,竟是立止健馬。張翠山道︰「都大哥不須誤會。你千里迢迢,護送我兪三哥來此,小弟只有感激,絶無别意。」都大錦「{\upstsl{嗯}}」了一聲,將單刀刀頭插入鞘中,右手仍是執住刀柄。張翠山道︰「我兪三哥怎樣受傷?對頭是誰?是何人請都大哥送他前來?」對這三個問題,都大錦却是一句也答不上來。張翠山皺起眉來,又問︰「接了我兪三哥去的六個人是怎等模樣?」史鏢頭口齒靈便,搶著説了。張翠山道︰「小弟先趕一步。」一抱拳,縱馬狂奔。

這青驄馬緩步而行,已是迅疾異常,這一展開脚力,但覺耳邉風生,山道兩旁的樹木不住倒退。武當七俠同門學藝,連袂行俠,當眞是情逾骨肉,張翠山聽得師哥身受重傷,却又落入不明來歷之人的手中,心急如焚,不住的催馬快行,便是這匹寶馬立即倒斃,那也顧不得了。一口氣奔到了草店,那是一處三岔口,一條路通向武當山,另一條路東北行至鄖陽。張翠山心想︰「這六人若是好心送兪三哥上山去,那麼適纔下山時我定會撞到。」雙腿一挾,向東北方追了下去。

這一陣急奔,足足有一個時辰,那馬雖壯,却也支持不住,越跑越慢,眼見天色漸漸黑了下來,這一帶山道上人跡稀少,無從打聽。張翠山一路追趕,心下不住尋思︰「兪三哥武功卓絶,怎會輕輕易易的被人打得重傷?瞧那都大錦的神情,却又不是説謊之人?」眼看將至十偃鎭,那青驄馬忽地一聲長厮,離開大道,向右首的荒墳堆中走了進去。張翠山知道有異,凝目一望,只見一輛大車歪歪的倒臥在長草之中。再走近幾步,只見拉車的騾子頭骨破碎,腦漿迸裂,死在地下。

張翠山飛身下馬,掀開大車的簾子一看,只見車中無人,一轉過身來,却見長草中一人俯伏,一動也不動,似已死去多時。張翠山心中砰砰亂跳,搶過去一看,瞧那後影正是三師兄兪岱岩,急忙張臂抱起。暮色蒼茫之中,只見他雙目緊閉,臉如金紙,神色甚是可怖,張翠山又驚又痛,伸過自己臉頰去挨在他臉上,竟是略有微溫。張翠山大喜,伸手一摸他胸口,覺得他一顆心尚在緩緩跳動,只是時停時跳,説不定隨時均能止歇。張翠山垂泪道︰「三哥,你\dash{}你怎麼\dash{}我是五弟\dash{}五弟啊!」抱著他慢慢站起身來,却見他雙手雙足軟軟垂下,原來四肢骨節都已被人折斷。但見指骨、腕骨、臂骨、腿骨到處冒出鮮血,顯是敵人下手不久,而且是逐一折斷,下手之毒辣,實是令人慘不忍睹。

張翠山怒火攻心,目眥欲裂,知道敵人離去不久,憑著健馬脚力,當可追趕得上,一時狂怒,便欲趕去一拚,但隨即想起︰「三哥命在頃刻,須得先救他性命要緊。君子報仇,十年未晩。」偏偏下山之際預擬片刻即回,身上没帶兵刃藥物,眼看著兪岱岩這等情景,馬行顚簸,每一震盪便增加他一分痛楚。當下穩穩的將他抱在手中,展開輕功,向山上疾行。那青驄馬跟在身後,見主人不來騎坐,似乎甚感奇怪。

這一日是武當派創派祖師張三丰的九十壽辰,當天一早,玉虛宮便是喜氣洋洋,六個弟子自大弟子宋遠橋以下,逐一向師父拜壽。只是七大弟子之中,少了一個兪岱岩不到。張三丰和諸師兄弟知道兪岱岩做事穩重,到南方去誅滅的那個劇盜也不是怎生厲害的人物,預計定可及時趕到,但等到正午,仍是不見他的人影,衆人不耐起來,張翠山便道︰「弟子下山接兪三哥去。」

那知他一去之後,也是音訊全無。按説他所騎的青驄馬脚力極快,便是直迎到老河口,也該回轉了,不料一直到酉時,仍不見回山。大廳上壽筵早已擺好,紅燭高燒,已點去大半枝。衆人都有些心緒不寧起來,六弟子殷利亨、七弟子莫聲谷在玉虛宮的觀門口進進出出,也不知有多少遍。張三丰此時修爲,早已心地澄澈,但他素知這兩個弟子的性格,兪岱岩穩重可靠,能彀擔當大事,張翠山聰明機靈,辦事迅敏,從不拖泥帶水,直等到這時還不見回山,定是發生了什麼不測的大事。

宋遠橋望了望紅燭,陪笑道︰「師父,兪三弟和張五弟定是遇上了什麼不平之事,因之出手干預。師父常教訓咱們積德行善,今日你老人家千秋大喜,兩個師弟幹一件俠義之事,那纔是最好不過的壽儀啊。」張三丰一摸長鬚,笑道︰「{\upstsl{嗯}}{\upstsl{嗯}},我過八十歳生日那一天,你救了一個投井寡婦的性命,那好得很啊,只是每過十年纔做一件好事,未免叫天下人等得心焦。」五個弟子一齊笑了起來。原來張三丰雖是一派的大宗師,但生性詼諧,師徒之間也常常説些笑話。四弟子張松溪道︰「你老人家至少活到二百歳,咱們每十年幹樁好事,加起來也不少啦。」七弟子莫聲谷笑道︰「哈哈,就怕咱們没這麼多歳數好活\dash{}」

他一言未畢,大弟子宋遠橋和二弟子兪蓮舟一齊搶到滴水簷前,叫道︰「是三弟麼?」只聽得張翠山道︰「是我!」聲音中帶著嗚咽,只見他雙臂橫抱一人,搶了進來,滿臉血汚混著汗水,奔到張三丰面前一跪,泣不成聲,叫道︰「師父,兪\dash{}兪三哥受人暗算\dash{}」

衆人大驚之下,只見張翠山身子一晃,向後便倒,原來他這般足不停步的長途奔馳,加之心中傷痛,終於支持不住,一見師父和衆同門,竟自暈去。

宋遠橋和兪蓮舟都是極有見識之人,面臨大變,却未慌亂,知道張翠山之暈,只是心神激盪,再加疲累過甚,三師弟兪岱岩却是存亡未卜。因之兩人不約而同的一齊伸手,將兪岱岩抱起,只見他呼吸微弱,只剩下遊絲般的一口氣。張三丰見心愛的弟子傷成這般模樣,胸中大震,當下不暇詢問,奔進内堂取出一瓶「白虎奪命丹」。丹瓶口本用白臘封住,這時也不及除臘開瓶,左手兩指一捏,瓷瓶碎裂,取出三粒白色丹藥,餵在兪岱岩嘴裡。但兪岱岩知覺已失,那裡還會吞嚥?

張三丰雙手食指和拇指虛拿,成「鶴嘴勁」勢,以食指指尖點在兪岱岩耳尖上三分處的「龍躍竅」,運用内力,微微擺動。以他此時功力,這「鶴嘴勁點龍躍竅」使將出來,便是新斷氣之人,也能還魂片刻,但他手指直擺到二十上下,兪岱岩仍是動也不動。張三丰輕輕嘆了口氣,雙手捏成劍訣,以掌心向下的陰手雙取兪岱岩「頰車穴」。那「頰車穴」是在腮上牙関緊閉的結合之處,張三丰陰手一點,立即掌心向上,翻成陽手,一陰一陽,交互變換,翻到第十二次時,兪岱岩口一張,緩緩將丹藥吞入喉中。殷利亨和莫聲谷心神緊張,這時「啊」的一聲叫了出來。

但兪岱岩喉頭肌肉僵硬,丹藥雖入咽喉,却不至腹,四弟子張松溪按摩他喉頭肌肉,張三丰隨即伸指點了他肩頭「缺盆」「兪府」諸穴,尾脊的「陽関」「命門」諸穴,使得他醒轉之後,不致因覺到四肢傷殘的痛楚而重又昏迷,宋遠橋和兪蓮舟自入師門以來,見師父不論遇到什麼疑難驚險的大事,始終泰然自若,但這一次竟是微微發顫,眼神流露出惶惑之色,兩人均知三師弟之傷,實是嚴重已極。

過不多時,張翠山悠悠醒轉,叫道︰「師父,三師哥還能救麼?」張三丰不答,只道︰「翠山,世上誰人不死?」只聽得脚步聲響,一個小童進來報道︰「觀外有一干鏢客求見祖師爺,説是臨安府龍門鏢局的都大錦。」張翠山霍地站起,滿臉怒色,喝道︰「便是這厮!」縱身出去,只聽得門外嗆{\upstsl{啷}}{\upstsl{啷}}幾聲響,兵刃落地。殷利亨和莫聲谷正要搶出去相助師兄,只見張翠山一把抓住一條大漢的後心,提了進來,往地上重重一摔,怒道︰「都是這厮壞的大事!」殷利亨在武當七俠中性子最急,一聽是這人害得三師哥如此重傷,伸脚便往都大錦身上踢去。宋遠橋低喝道︰「六弟,且慢!」只聽門外有人叫道︰「你武當派講理不講?咱們好意求見,却這般欺侮人麼?」宋遠橋眉頭微皺,伸手在都大錦腦後和背心拍了幾下,解開張翠山點了他的穴道,説道︰「門外客人不須喧嘩,請稍待片刻,自當分辨是非。」這兩句話語氣威嚴,内力充沛,祝史兩鏢頭聽了,登時氣爲之懾,只道是張三丰出言喝止,那裡還敢囉唆?

宋遠橋道︰「五弟,三弟如何受傷,你慢慢説,不用氣急。」張翠山向都大錦狠狠瞪了一眼,纔將龍門鏢局如何受託護送兪岱岩來武當山,却給六個歹人冒名接去之事説了,宋遠橋見都大錦這等功夫,早知決非相害兪岱岩之人,何況既敢登門求見,自是心中不虛,聽張翠山説完,當下和顏悦色,向都大錦詢問他自受託日起,直至遇到張翠山這十天來的經過。都大錦一一照實而説,最後慘然道︰「宋大俠,咱姓都的辦事不週,累得兪三俠遭此橫禍,自是該死。咱們臨安府滿局子的老小,此時還不知性命如何呢。」張三丰一直伸掌心貼著兪岱岩的「神藏」「靈台」兩穴,鼓動内力,將一股熱氣送入他的體内,聽都大錦説到這裡,忽然説道︰「蓮舟,你帶同聲谷,立即動身去臨安,保護龍門鏢局的老小。」

兪蓮舟一怔,立即明白師父慈悲之心,俠義之懷,那姓殷的客人既説過這件事中途有半分差池,要殺得他龍門鏢局老小七十一口雞犬不留,這雖是一句恫嚇之言,但都大錦等好手均外出走鏢,倘若鏢局中當眞有甚危難,却是無人抵擋。張翠山道︰「師父,這姓都的糊塗透頂,三師哥給他害得這個樣子,便算他不是有意,咱們不找他麻煩,也就是了,怎能再去保護他的家小?」張三丰搖了搖頭,並不答話。宋遠橋道︰「五弟,你怎地心胸這等狹窄?都總鏢頭千里奔波,爲的是誰來?」張翠山冷笑道︰「他還不是爲了那二千兩黃金的鏢金。」都大錦一聽此言,登時脹得滿臉通紅,但拊心自問所以接這趟鏢,也確是爲了這筆厚酬。

宋遠橋喝道︰「五弟,對客人不得無禮。你累了半天,快去歇歇吧!」武當門中,師兄威權甚大,宋遠橋武功、年歳、德望?又無不高於衆師弟幾分,自兪蓮舟以下,人人對他極是尊敬,張翠山聽他這麼一喝,不敢再作聲了,但関心兪岱岩的傷勢,却不去休息。

宋遠橋道︰「二弟,救兵如救火,師父有命,你就同七弟連夜動程,不得耽誤。」兪蓮舟和莫聲谷答應了,各自去收拾衣物兵刃。

都大錦見兪莫二人要趕赴臨安去保護自己家小,心中一股説不出的滋味,抱拳向張三丰道︰「張眞人,晩輩的事,不敢驚動兪莫二俠,就此告辭。」宋遠橋道︰「各位今晩在敝處歇宿,咱們還有一些事請教。」他説話聲音平平淡淡,但自有一股威嚴,教人無法抗拒。都大錦只得默不作聲,坐在一旁,眼看著兪蓮舟和莫聲谷依依不捨的望了兪岱岩幾眼,下山而去。須知兩人心頭極是沉重,也不知這一次是生離還是死别,不知日後是否還能和兪岱岩相見。

這時大廳中一片寂靜,只聽得張三丰沉重的噴氣和吸氣之聲,又見他頭頂心熱氣繚繞,猶似蒸籠一般,過了大半個時辰,兪岱岩「喲」的一聲大叫,聲震屋瓦,都大錦嚇了一跳,偸眼瞧張三丰時,見他臉上不露喜憂之色,無法猜測兪岱岩這一聲大叫主何吉凶。張三丰緩緩的道︰「松溪、利亨,你們抬三哥進房休息去。」張松溪和殷利亨抬了傷者進房,回身出來,殷利亨忍不住問道︰「師父,三哥的武功能全部復原嗎?」張三丰嘆了一口長氣,隔了半晌,纔道︰「他能否保全性命,要一個月後方能分曉,但手足筋斷骨折,終是無法再續。這一生啊,這一生啊\dash{}」説著淒然搖頭。殷利亨突然哇的一聲,哭了出來。他這時的武功已臻一流高手之境,但心腸極軟,稍有激動,便易流泪。

張翠山霍地跳起,拍的一聲,便打了都大錦一個耳光。這一下出手如電,都大錦伸手擋格,但手臂伸出時,臉上早已中掌。張翠山怒氣難以遏制,左肘彎過,往他腰眼心撞去。這一下仍是極快,但張松溪伸掌在張翠山肩頭一推,張翠山這個肘槌便落了空。都大錦身子向後一讓,{\upstsl{噹}}的一聲,一隻金元寶從他懷中落下地來。張翠山左足一挑,將金元寶挑了起來,伸手接住,冷笑道︰「貪財無義之徒,人家賞你一隻金元寶,你便將兪三哥交了給人家作踐\dash{}」話未説完,突然「咦」的一聲,瞧著金元寶所捏的十個手指印,道︰「大師哥,這\dash{}這是少林派的金剛指功夫啊。」宋遠橋接過金元寶看了良久,遞了給張三丰。張三丰將那金元寶翻來覆去看了幾遍,和宋遠橋對望一眼,均不説話。張翠山大聲道︰「師父,這是少林派的金剛指功夫。天下再没有第二個門派會這門功夫,你説是不是,你説是不是啊?」

在這一瞬之間,張三丰想起了自己幼時如何在少林寺藏經閣中侍奉覺遠禪師、如何打敗崑崑三聖何足道,如何被少林僧衆追捕而逃上武當,數十年間的往事,猶似電閃般在心頭一掠而過。他臉上一陣迷惘,從那金元寶上的指印看來,明明是少林派的金剛指法,張翠山説得不錯,方今之世,確是再無别個門派中有這一項功夫,自己武當的武功講究内力深厚,不練這類碎金裂石的硬功,而其餘外家門派,儘有凌厲威猛的掌力、拳力、臂力、腿力,以至頭槌、肘槌、膝槌、足槌,説到指力,却均無這般造詣。只聽得張翠山連問數聲,若是説出眞相,門下衆弟子決不肯和少林派干休,如此武林中領袖群倫的兩大門派,相互間便要惹起極大風波了。

張翠山何等聰明,見師父沉吟不語,已知所料不錯,又追問一句︰「師父,武林中是否有甚奇人異士,能自行練成這種金剛指力?」張三丰緩緩搖頭,説︰「這是少林派累積千年來的經驗傳統,方得達成這等絶技,決非一蹴而至,便算是絶頂聰明之人,也無法自創。」他頓了一頓,又道︰「我當年在少林寺中住過,只是不得傳授,直到此時,也不懂尋常血肉之軀,如何能練到這般指力。」宋遠橋眼神中突然放出異樣光茫,道︰「三弟的手足筋骨,便是給這種金剛指力捏斷的。」殷利亨「啊」的一聲,眼中泪光瑩瑩,忍不住又要流下泪來。

都大錦聽説出手殘害兪岱岩之人,竟是少林派的子弟,更是驚惶,張大了口合不攏來,過了好一陣纔道︰「不\dash{}決計不會的,我在少林寺中學藝十餘年,從未見過此人。」宋遠橋凝視著他的雙眼,不動聲色的道︰「六弟,你送都總鏢頭他們到後院休息,囑咐老王要好好招呼遠客,不可怠慢。」殷利亨答應了,引導都大錦一行人走向後院。都大錦還想辯解幾句,但在這情景之下,却是一句話也説不出來。

殷利亨安頓了衆鏢師後,再到兪岱岩房中去,只見三師哥睜目瞪視,狀如白痴,那裡還是平時英爽豪邁的模樣,不由得一陣心酸,叫了聲「三哥」,掩面奔出,衝入大廳之中,見宋遠橋等都坐在師父身前,於是挨著張翠山肩側坐下。

張三丰望著天井中的一棵大槐樹,出神半晌,搖頭道︰「這事好生辣手,松溪,你説如何?」原來武當七弟子中,以張松溪最是足智多謀。他平素沉默寡言,但潛心料事,言必有中,這一次自張翠山抱了兪岱岩上山,他雖心中傷痛,但一直在推想其中的過節,這時聽師父問起,説道︰「據弟子想,罪魁禍首不是少林派,而是屠龍刀。」張翠山和殷利亨同時「啊」的一聲。宋遠橋道︰「四弟,這中間的事理,你必已推想明白,快説出來再請師父示下。」張松溪道︰「兪三哥行事穩健,對人很彀朋友,決不致輕易和人結仇。他去南方所殺的那個劇盜,又是下三濫的,爲武林人物所不齒,少林派決不致爲了此人而下手傷害兪三哥。」張三丰點了點頭,張松溪又道︰「兪三哥手足筋骨倶斷,那是外傷,但在浙江臨安府已是身中劇毒。據弟子想,咱們首先要去臨安査詢,兪三哥如何中毒,是誰下的毒手?」張三丰點了點頭,道︰「岱岩所中之毒,異常奇特,我推想至此,還没想出到底是何種毒藥。岱岩右掌心有七個小孔,腰腿間有幾個極細的針孔。江湖之上,還没聽説有那一位高手使這種歹毒的暗器。」宋遠橋道︰「這事也眞奇怪,按常理推想,發射這纖細的暗器而叫三弟閃避不及,必是一流好手,但眞正第一流的高手,怎又能在暗器上餵這等毒藥?」衆人默然不語,心下均在思索,到底那一門那一派的人物,是使這種暗器的?

過了半晌,五個人面面相覷,都想不起是那一個人物。張松溪道︰「那個臉上生有黑痣之人,何以要捏斷三哥的筋骨?倘若他跟三哥有仇,一掌便能將他殺了,若是要他多受些痛苦,何不斷他脊骨,傷他腰肋?這理由很明顯,他是要逼問三哥的口供。他要問什麼呢?據弟子推想,必是爲了屠龍刀。據都大錦説︰那六人之中有一人問道︰『屠龍刀也在麼?』」

宋遠橋道︰「『武林至尊,寶刀屠龍,倚天不出,誰與爭鋒』,這句話傳了幾百年,難道時至今日,眞的出現了一把屠龍刀?」張三丰道︰「不是幾百年,最多不過七八十年,當我年輕之時,就没聽過這幾句話。」張翠山霍地站起身來,説道︰「四哥的話很對,傷害三哥的罪魁禍首,必是在江南一帶,咱們便找他去,只是那少林派的惡賊下手如此狠辣,咱們也決計放他不過。」張三丰向宋遠橋道︰「遠橋,你説目下怎生辦理?」近年來武當派中一切大小事務,張三丰都已交給了宋遠橋,而這位大弟子處理得井井有條,早已不用師父勞神。

他聽師父如此説,站起身來,恭恭敬敬的道︰「師父,這件事不單是給三弟報仇雪恨,而且関連著本派的門戸大事,若是應付稍有不當,只怕引起武林中的一場浩劫,還得請師父示下。」張三丰道︰「好!你和松溪、利亨二人,持我的書信到嵩山少林寺去拜見方丈宏法禪師,告知此事,請老禪師的指示。這件事咱們不必插手,少林派門戸嚴謹,宏法老方丈望重武林,必有妥善措施。」宋遠橋、張松溪、殷利亨三人一齊肅立答應。張松溪心想︰「若是只不過送一封書信,單是差六師弟也就彀了。師父命大師哥親自出馬,還叫我同去,其中必有深意,想是還防著少林派護短不認,叫咱們相機行事。」果然張三丰又道︰「本派和少林派之間,関係很是特殊。我是少林寺的逃徒,這些年來總算他們瞧我一大把年紀,不上武當山來抓我回去,但兩派之間,總是存著芥蒂。」説到這裡莞爾一笑,又道︰「你們上少林寺去,對宏法方丈固當恭敬,但也不能墮了本門的聲望。」宋張殷三弟子齊聲答應。

張三丰轉頭向張翠山道︰「翠山,你明児動身去江南,相機査詢,一切聽二師哥的吩咐。」張翠山垂手答應。張三丰道︰「今晩這杯壽酒也不用喝了。一個月之後,大家在此聚集,岱岩倘若不治,師兄弟們也可再和他見一面。」他説到這裡,不禁淒然,想不到威震武林數十載,臨到九十之年,心愛的弟子竟爾遭此不幸,殷利亨伸袖拭泪,抽抽噎噎的哭了起來。張三丰袍袖一揮,道︰「大家去睡吧。」宋遠橋勸道︰「師父,三師弟一生行俠仗義,積德甚厚,常言道吉人自有天相,老天爺有眼,總不該讓他\dash{}夭折\dash{}」但他説到後來,眼泪已是滾滾而下。這一干人平素縱橫江湖,豪氣干雲,碰到再大的危難之事也不能皺一皺眉頭,但這時都是悲憤填膺,當眞是「英雄有泪不輕彈,只因未到傷心處」,身臨此境,人人都是傷心到極處了。

宋遠橋知道若再相勸,只有徒增師父傷懷,於是和諸師弟分别回房去睡。但人人滿懷心事,在床上想一陣,恨一陣,又是難過一陣。

張翠山在諸同門中,和兪岱岩及殷利亨最是交厚,滿懷惱怒,不知如何發洩,眼前只有都大錦等一干鏢師在此,他在床上躺了一個多個時辰,悄悄起身,決意去找都大錦來,打他一頓出一口惡氣。張翠山生怕大師兄和四師兄干預,不敢發出聲息,將到大廳時,只見廳上一人背負雙手,不停步的走來走去。

黑暗矇朧中,見這人身長背厚,步履凝重,正是師父,張翠山藏身柱後,不敢走動,心知即令立刻回房,也必爲師父知覺,他査問起來,不能隱瞞,自當實言相告,那是自招一場訓斥了。只見張三丰走了一會,仰視庭除,忽然伸出右手,在空中一筆一劃的冩起字來。張三丰文武兼資,吟詩冩字,弟子們司空見慣,也不以爲異,張翠山順著他手指的筆劃瞧去,原來冩的是「喪亂」兩字,連冩了幾遍,跟著又冩「荼毒」兩字。張翠山心中一動︰「原來師父是在空臨王羲之的『喪亂帖』。」要知張翠山的外號叫作「銀鉤鐵劃」,固然是因他左手使爛銀虎頭鉤、右手使鑌鐵判官筆而起,但他自得了這外號後,深恐名不副實,爲文士所笑,於是潛心學書,眞草隸篆,一一遍習,這時見了師父指書的筆致,但見他無垂不收,無往不復,正是王羲之「喪亂帖」的家數。

這「喪亂帖」張翠山兩年前也曾臨過,雖覺其用筆縱逸,清剛峭拔,然而總覺不及「蘭亭詩序帖」「十七帖」各帖的莊嚴肅穆,氣象萬千,這時他躱在柱後,見師父以手指臨空連書「羲之頓首,喪亂之極,先墓再離荼毒,追惟酷甚」這十八個字,只見他一筆一劃之中,充滿了拂鬱悲憤之氣,登時領悟了王羲之當年書冩這「喪亂帖」時的心情。

原來王羲之是東晉時人,其時中原板蕩,淪於異族,王謝高門,南下避寇,於喪亂之餘,先人墳墓慘遭毒手,自是説不出滿腔傷痛,這股深沉的心情,盡數隱藏在「喪亂帖」中。張翠山翩翩年少,無牽無慮,從前那裡能領略到帖中的深意?這時身遭師兄存亡莫測的大禍,方才懂得了「喪亂」兩字、「荼毒」兩字。

張三丰冩了幾遍,長長嘆了口氣,步到中庭,沉吟半晌,伸出手指,又冩起字來,這一次冩的字體又自不同,張翠山順著他手指的走勢看去,但見第一字是個「武」字,第二個冩了「林」字,一路冩下來,共是二十四字,那便是適纔提到過的那幾句話︰「武林至尊,寶刀屠龍。號令天下,莫敢不從。倚天不出,誰與爭鋒?」想是張三丰正自琢磨這二十四個字中所含的深意,推想兪岱岩因何受傷?到底此事與倚天劍、屠龍刀這兩件傳説中的神兵利器,有什麼関連?

只見他冩了一遍又是一遍,那二十四個字翻來覆去的書冩,筆劃越來越長,手勢却是越來越慢,到後來縱橫開闔,宛如施展拳脚一般。張翠山凝神觀看,心下又驚又喜,師父所書的二十四個字,分明是一套深奥高明之極的武功,每一個字包含數招,便有數種變化。「龍」字和「鋒」字筆劃甚多,「刀」字和「下」字筆劃甚少,但筆劃多的不覺其繁,筆劃少的不見其陋,其縮也凝重,似尺蠖之屈,其縱也險勁,如狡兔之脱,淋漓酣暢,雄渾剛健,俊逸處如風飄,如雪舞,厚重處如虎蹲,如象步。張翠山只看得目眩神馳,潛心記憶。這二十四個字共有兩個「不」字,兩個「天」字,但兩字之間形同意不同,氣似而神不似,其變化之妙,又是另具一功。

近年來張三丰極少顯示武功,殷利亨和莫聲谷兩個小弟子的功夫,大都是宋遠橋和兪蓮舟代授,因此張翠山雖是他的第五名弟子,其實已是他親授武功的関門弟子。從前張翠山修爲未到,雖然見到師父施展拳劍,未能深切體會到其中博大精深之處,近年來他武學大進,這一晩兩人更是心意相通,情致合一,以遭喪亂而悲憤,以遇荼毒而拂鬱。張三丰情之所致,將這二十四個字演爲一套武功,他書冩之初,原無此意,而張翠山在柱後見到更是機緣巧合。師徒倆心神倶酔,沉浸在武功與書法相結合、物我兩忘的至高境界之中。

\chapter{龍門鏢局}

這一套拳法,張三丰一遍又一遍的翻覆演展,足足打了兩個多時辰,待到月湧中天,他長嘯一聲,右掌直劃下來,當眞是星劍光芒,如矢應機,霆不暇發,電不及飛,這一直乃是「鋒」字的最後一筆。張三丰仰天遙望,説道︰「翠山,這一路書法如何?」張翠山吃了一驚,想不到自己躱在柱後,師父雖不回頭,却早知道了,於是走到廳口,説道︰「弟子今日得窺師父絶藝,眞是大飽眼福。我去叫大師哥他們出來,一齊瞻仰好麼?」張三丰搖頭道︰「我興致已盡,只怕再也冩不成那樣的好字了。遠橋、松溪他們不懂書法,便是看了,也領悟不多。」説著袍袖一揮,進了内堂。

張翠山不敢去睡,生怕一著枕之後,適纔所見到的精妙招術會就此忘了,當即盤膝坐下,一筆一劃、一招一式的默默記憶,當興之所至,便起身試演幾手。也不知過了多少時候,纔將那二十四字二百一十五筆中的騰挪變化,盡數記在心中,他躍起身來,習練一遍,自覺揚波搏擊,雁飛鵰振,延頸協翼,全身都是輕飄飄的,有如騰雲駕霧一般,最後一掌直劃,呼的一響,將自己的衣襟掃下一大片來。張翠山心下驚喜,驀回頭,只見日頭曬在東牆。他揉了揉眼睛,只怕看錯了,一定神之下,才知日已過午,原來自己潛心練功,不知不覺的已過了大半天。

張翠山伸袖一抹額頭汗水,奔至兪岱岩房中,只見張三丰雙掌按住兪岱岩胸腹,正自運功替他療傷。張翠山出來一問,才知宋遠橋、張松溪、殷利亨三人一早便去了,龍門鏢局的一干鏢師也已下山。原來各人見他靜坐默想,都不來打擾他用功。張翠山這時全身衣履都浸濕了汗水,但急於師兄之仇,不及沐浴更衣,帶了隨身的兵刃衣服,拿了幾十兩銀子,又至兪岱岩房中,説道︰「師父,弟子去了。」張三丰點了點頭,微微一笑,意示鼓勵。張翠山走近床邉,只見兪岱岩滿臉灰黑之氣,顴骨高聳,雙頰深陥,除了鼻中尚有一些呼吸之外,直與死人無異。張翠山心中一酸,哽咽道︰「三哥,我便是粉身碎骨,也要跟你報仇。」説著跪下向師父磕了個頭,掩面奔出。

他騎了那匹高脚青驄馬,疾下武當,這日天時已晩,只行五十餘里,天便黑了。他剛投店,天空鳥雲密佈,接著便下起傾盆大雨來。這一場雨越下越大,直落了一晩竟不稍止,次日清晨起來,但見四下裡霧氣茫茫,耳中只聽到殺殺雨聲,張翠山向店家買了簑衣笠帽,冒雨趕路,虧得那青驄馬極是神駿,大雨之中,道路泥濘滑溜,但牠仍是奔馳迅捷。

張翠山趕到老河口過漢水時,但見黃浪混濁,江流滾滾,水勢極是凶險,一過襄樊,便聽得道路傳言,説道下游流水溝決了堤,傷人無數。這一日來到宜城,只見水災的難民拖児帶女的逃了上來,大雨兀自未止,人人淋得極是狼狽。

張翠山正行之間,只見前面有一行人騎馬趕路,鏢旗高揚,正是龍門鏢局的衆鏢師。張翠山催馬上前,掠過了鏢隊,迴馬過來,攔在當路。

都大錦見是張翠山追到,冷冷的道︰「張五俠有何見教?」張翠山道︰「這些水災的難民,都總鏢頭瞧見了麼?」都大錦没料到他會問這句話,怔了一怔,道︰「怎麼?」張翠山冷笑道︰「要請善長仁翁,拿些黃金出來救濟災民啊。」都大錦臉上變色,道︰「咱們走鏢之人,在刀尖子上賣命混飯吃,有什麼力量救災?」張翠山低著嗓子道︰「你把囊中那二千兩黃金,都給我拿出來。」都大錦手持刀柄,説道︰「張五俠,你今日硬是找上我姓都的了?」張翠山道︰「不錯,我吃定你啦。」

祝史兩鏢頭各自取出兵刃,和都大錦並肩而立。張翠山仍是空著雙手,嘿嘿冷笑,説道︰「都總鏢頭,你受人之祿,可曾忠人之事?這二千兩黃金,虧你有臉放在袋中。」都大錦一張臉蛋脹成了紫醬之色,説道︰「兪三俠不是已經到了武當山上?當他交在咱們手中之時,他早便身受重傷,這時候可也没死?」張翠山大怒,喝道︰「你還要強辯,兪三哥從臨安出來時,可是手足折斷麼?」都大錦默然。史鏢頭插口道︰「張五爺,你到底要怎樣,劃下道児來吧。」張翠山道︰「我要將你們手骨脚骨,一個個折得寸寸斷絶。」這句話一出口,倏地躍起,飛身而前。史鏢頭舉棍欲擊,張翠山左手一揮一掠,使出新學的那套武功,却是「天」字訣那一招中的一撇,史鏢頭棍棒脱手,倒撞下馬。祝鏢頭爲人愼重,待要退縮,却那裡來得及,張翠山順手使出「天」中的一捺,手指掃中他的腰肋,砰的一聲,將他連人帶鞍,摔出丈餘。原來祝鏢頭雙足牢牢鉤在鞍鐙之中,但張翠山這一捺勁道凌厲之極,馬鞍下的肚帶給他一掃迸斷,祝鏢頭足不離鐙,却跌得爬不起來。

都大錦見他出手如此矯捷,一驚之下,提韁催馬向前急衝。張翠山轉身吐氣,左拳送出,却是「下」字訣中的一直,拍的一聲,已擊中他的後心。都大錦身子晃了一晃,他的武功可比祝史二鏢頭高得多了,並不摔下馬來,惱怒之下,正欲下馬與張翠山放對,突然間喉頭一甜,哇的一聲,噴出一口鮮血。原來張翠山這一拳勁力極是厲害,饒是都大錦練就了一身外門硬功,却也經受不起。他脚下一個踉蹌,吸一口氣,只覺胸口又有熱血湧上,雖是要強,却也支持不住,雙膝一軟,竟是坐倒在地。鏢行中其餘三名年青鏢師和衆趟子手見了這等情景,只驚得目瞪口呆,那敢再上前相扶?

張翠山初時怒氣勃勃,原是想把都大錦等一干人個個手足折斷,出一口胸中惡氣,待見自己隨手一掌一拳,竟將三個鏢師打得如此狼狽,都大錦更是身受重傷,不自禁暗暗驚異,自己事先絲毫没有想到,這一套新學的二十四字「倚天屠龍功」竟有這麼巨大的威力。這麼一怔之中,便不想再下辣手,説道︰「姓都的,今日我手下容情,打到你這般地步,也就彀了。你把囊中的二千兩黃金,盡數取將出來救濟災民。我在暗中窺探,只要留下一兩八錢,我拆了你的龍門鏢局,將你滿門七十一口,殺得雞犬不留。」最後這兩句話,是他聽都大錦轉述的,這時忽然想到,隨口説了出來。

都大錦緩緩站起,但覺背心劇痛,略一牽動,又吐出一口鮮血。史鏢頭却只受了些皮肉外傷,自知決非張翠山的對手,嘴頭上再也不敢硬了,説道︰「張五俠,咱們雖然受了人家的鏢金,但這一趟道中出了岔子,須得將金子還給人家。再説,那些金子存在臨安鏢局之中,咱們身在異鄕,這當口那裡有錢來救濟災民啊。」張翠山冷笑道︰「你欺我是小娃娃嗎?你們龍門鏢局傾巢而出,臨安府老家中没好手看守,這黃金自是隨身擕帶。」他向鏢隊一行人瞧了幾眼,走到一輛大車旁邉,手起一掌,喀喇喇幾聲響,車廂碎裂,跌出十幾隻金元寶來。

衆鏢師臉上變色,相顧駭然,不知張翠山何以竟知道這藏金之處。原來張翠山年紀雖輕,但隨著衆師兄行俠天下,江湖上的事見得多了。他心思細密,目光敏鋭,見這輛大車在爛泥道中輪印最深,而三個年輕鏢師一見都大錦中拳跌倒,並不上前救助,反而一齊向這大車靠攏,可想而知,車中定是藏著貴重之物。張翠山一見黃金跌得滿地,冷笑幾聲,翻身上馬,逕自去了。

適纔這件事做得甚是痛快,料想都大錦等念著家中老小,不敢不將這二千兩黃金拿來救濟災民,張翠山一面趕路,一面默想那二十四個字中的招數變化。他在那天晩上依樣模學,只覺得招數神妙莫測,豈知一經施展,竟具如此神威,眞比撿獲了無價之寶還要快活十倍。

大雨中連接趕了幾日路,那青驄馬雖然壯健,却也支持不住了,到得江西省境,忽地口吐白沫,發起燒來。張翠山很愛惜這頭牲口,只得陪著牠緩緩而行。這麼一來,道上便走得慢了,到得臨安府,已是四月三十的傍晩。

張翠山投了客店,尋思︰「我在道上走得慢了,不知都大錦等這干人是否回了鏢局?二哥和七弟不知落脚在何處?今晩且上龍門鏢局去探一探。」

他用過晩膳,向店伴一打聽,知那龍門鏢局坐落在裡西湖畔。張翠山先到街上買了一套衣巾,又買一把杭州城馳名天下的摺扇,在澡堂中洗了浴,命待詔理髮梳頭,周身換得煥然一新,對鏡一照,儼然是個濁世佳公子,却那裡像是個威揚武林的俠士?他借過筆墨,想在扇上題些詩詞,但手上一拿到筆,自然而然的冩下了那「倚天屠龍」的二十四字︰「武林至尊,寶刀屠龍,號令天下,莫敢不從!倚天不出,誰與爭鋒?」一筆一劃,無不力透紙背,冩罷持扇一看,心道︰「學了師父這套拳法之後,竟是書法也大進了。」於是摺扇輕搖,踱著方步,逕往裡西湖而去。

此時宋室淪亡,臨安府早已陥入元人之手。蒙古人因臨安是南宋都城,深恐人心憶舊,民戀故君,特駐重兵鎭壓。那蒙古兵爲了立威,平素比在他處更是殘暴,而臨安城中百姓所受的苦楚荼毒,也比他處更是厲害數倍。因此城中十室九空,居民泰半遷移到了别處。百年前臨安城中戸戸垂楊、處處笙歌的盛況,早已不可復睹。張翠山一路行來,但見到處是斷坦殘瓦,滿眼肅索,昔年繁華甲於江南的一座名城,半成廢墟。其時天未全黑,但家家閉戸,街上行人已極是稀少,唯見蒙古騎兵橫衝直撞,往來巡邏。張翠山不欲多惹事端,一聽到蒙古巡兵鐵騎之聲,便縮身在牆角小巷相避。

往昔一到夜晩,便是滿湖燈火,但這時張翠山走上白堤,只見湖上一片漆黑,竟無一個遊人。他心中暗暗嘆息,依著店小二所言途徑,尋覓龍門鏢局的所在。

那龍門鏢局是一座一連五進的大宅,面向裡西湖,門口蹲著一對玉石獅子,氣象甚是威武。張翠山不須覓人打聽,遠遠便即望見,他慢慢走近,忽地一怔,只見鏢局門外的湖中停泊著一艘遊船,船上點著兩盞碧紗燈籠,燈光下依稀見有一人據案飲酒。張翠山心道︰「這人倒有這等雅興!」只見龍門鏢局外掛著大燈籠中都没點燃蠟燭,朱漆銅環的大門緊緊関閉,想是鏢局中人都已安睡。張翠山走到門前,心道︰「一個月之前,有人送三哥經這大門而入,却不知那人是誰?」心中一酸,忽聽得背後有人幽幽嘆了口氣。

這一下嘆息,在黑沉沉的靜夜中聽來,大是鬼氣森森,張翠山霍地轉身,却見背後竟無一人,遊目環顧,除了湖上那小舟中那個單身遊客之外,四下裡寂無人影。張翠山微覺驚訝,斜睨舟中遊客,只見他青衫方巾,和自己一樣,也是作文士打扮,矇朧中看不清他的面貌,只見他側面的臉色極是蒼白,給碧紗燈籠一照,映著湖中綠波,寒水孤舟,冷冷冥冥,竟不似世間的人物。但見他悄坐舟中,良久良久,除了風拂衣袖,竟是一動也不動。

張翠山向舟中那人望了幾眼,心下不自禁的{\upstsl{嘀}}咕,他本想從黑暗無人之處,越牆而入龍門鏢局,但見了舟中那人,似覺夜踰入垣未免有些不彀光明正大,於是走到鏢局大門外,拿起門上銅環,{\upstsl{噹}}{\upstsl{噹}}{\upstsl{噹}}的敲了三下。靜夜中聽來,這三下擊門聲甚是響亮,遠遠的傳了出去。但隔了好一陣,屋内却無人出來應門。張翠山又擊三下,聲音更響了一些,可是側耳傾聽,屋内竟無脚步之聲。張翠山大是奇怪,伸手在大門上一推,那門無聲無息的開了,原來裡面竟是没有上閂。張翠山遇步而入,朗聲道︰「都總鏢頭在家麼?」一面説,一面走進大廳。廳中黑越越的並無燈燭,便在此時,忽聽得砰的一聲響,大門似乎被風一吹,自行関上了。

張翠山心念一動,躍出大廳,一看之下,竟自呆了,原來大門已緊緊閉上,而且上了橫閂,那麼顯是屋中有人。張翠山嘿嘿冷笑,心想︰「鬧什麼玄虛?」他藝高人膽大,索性便大踏步闖進廳子。這一次左脚一踏進廳門,只聽得前後左右,風聲颯然,共有四個人搶上圍攻。張翠山身形一側,避開了敵人的突襲,黑暗中白光微閃,原來這四人手中都拿著兵刃。張翠山一個左拗步,搶到了西首,右掌自左向右平平橫掃,拍的一聲,打在一人的太陽穴,登時將那人擊暈,跟著左手自右上角斜揮左下角,擊中了另一人的腰肋。這兩下是「不」字訣中的一橫一撇,他兩擊得手,左手直鉤,右拳砰的一「點」,四筆冩成了一個「不」字,却將四名敵人盡數打倒。

他不知暗伏在廳中忽施突襲的敵手是何方人馬,因此出手並不沉重,每一招都只用了三分勁力,第四個給他一「點」中拳的敵手退出幾步,喀喇一響,壓碎了一張紅木椅子,喝道︰「你如此狠毒,下這等辣手,是男児漢大丈夫便留下姓名。」張翠山笑道︰「我若眞施毒手,你那裡還有命在?在下武當張翠山便是。」那人「咦」的一聲,甚表驚異,説道︰「你當眞是武當派的張五\dash{}張五\dash{}銀鉤鐵劃張翠山?可不是冒名吧?」張翠山微微一笑,伸手到腰間摸出兵刃,左手爛銀虎頭鉤,右手鑌鐵判官筆,兩件兵刃相交一擊,嗆{\upstsl{啷}}{\upstsl{啷}}一陣響亮,爆出幾點火花。

這火花一閃之間,張翠山已看清眼前跌倒四人身穿黃色僧衣,原來都是和尚。那四個僧人中有兩人面向著他,也看見了他的面貌。張翠山見這兩個僧人滿臉血汚,眼光中流露出極度的怨毒,眞似恨不得食己之肉、寢己之皮一般,奇道︰「四位大師是誰?」只聽一個僧人叫道︰「這血海深仇,非今日能報,走吧!」説著四個人縱起身來,往外便走,其中一人脚步踉蹌,走了幾步,摔倒在地,想是給張翠山擊得重了。兩個僧人返身扶起,奔出廳外。張翠山道︰「四位慢走!什麼血海\dash{}」但話未説完,四個僧人早已越牆出外。

張翠山但覺今晩之事大是蹊蹺,在廳上沉思半晌,也想不出一個所以然,怎麼龍門鏢局之中竟埋伏著四個和尚?自己一進門便施突襲,又説什麼「血海深仇?」心想︰「此事只有詢問鏢局中人,方能釋此疑團。」於是提聲又道︰「都總鏢頭在家嗎?都總鏢頭在家麼?」大廳空曠,隱隱有回聲傳來,但鏢局中竟無一人答應。他心道︰「決不能都睡得死人一般。難道是怕了我,都躱了起來?又難道是人人出去逃難,鏢局中没有人?」當下從身邉取出火摺晃亮了,見茶几上放著一隻燭台,便點亮臘燭,走向後堂,没走得幾步,只見地下伏著一個女子,僵臥不動。張翠山叫道︰「大姐,怎麼啦?」那女子仍是不動。張翠山扳起她肩頭,將燭台湊過去一照,不禁一聲驚呼。

只見這女子臉露嬉笑之色,但肌肉僵硬,早已死去多時。張翠山手指碰到她肩頭之時,已料到這女子可能已死,然而死人臉上竟是一副極滑稽的笑容,黑夜中斗然見到,禁不住吃了一驚。他站直身子,只見左前柱子後又僵臥著一人,張翠山走過去一看,却是個僕役打扮的老者,也是臉露傻笑,死在當地。

張翠山心中大奇,左手從腰間拔出虎頭鉤,右手高舉燭台,一步步的四下察看,但見東一個、西一個,裡裡外外,一共死了數十人,當眞是屍橫遍地,恁大一座龍門鏢局,竟没留下一個活口。張翠山行俠江湖,生平慘酷的事也見了不少,但驀地裡見到這等殺滅滿門的情景,禁不住心下怦怦亂跳,只見自己映在牆上的影子不住抖動,原來手上發戰,燭火搖晃,映照得影子也顫慄起來。

他橫鉤悄立,心中猛地想起了兩句話︰「路上若有半分差池,我殺得你龍門鏢局滿門七十一口,雞犬不留。」眼前龍門鏢局中人人皆死,那顯是爲了都大錦護送兪岱岩不力之故,尋思︰「那人下此毒手,皆是因兪三哥而起,由此推想,他該當是兪三哥極要好的朋友。此人本領既高出都大錦甚多,又知此行途中可能會遇上凶險,然則他何不親自送來武當?我三哥仁俠正直,嫉惡如仇,又怎能和這等心如蛇蝎之人交上朋友?」越想疑團越多,舉步從西廳走出,燭光火下只見兩個黃衣僧人,背靠牆壁,瞪視著自己露齒而笑。張翠山急退兩步,按鉤喝道︰「兩位在此何事?」只見兩個僧人一動也不動,這纔醒悟,原來兩人也早死了。

他走近一看,只見兩僧身嵌牆壁之中,陥入數寸,顯是被人用重手法一擊震向牆壁,因而陥入。張翠山細看兩人身上並無傷痕,只是腰間「笑腰穴」上有一點紅痕,他點了點頭,心道︰「這些人死時都露笑容,原來均是笑腰穴中了敵人的重手。」突然間心下一涼,叫道︰「啊喲,不好,血海深仇,血海深仇\dash{}」適纔那四個僧人説什麼「你如此狠毒,下這等辣手,是男児漢大丈夫便留下姓名。」又説︰「這血海深仇,非今日能報。」看來龍門鏢局中這筆數十口的血債,都冩在自己頭上了,當時自己不明就裡,不但親報姓名,還露出仗以成名的銀鉤鐵劃兵刃。那四個黃衣僧人却是什麼來歷?

適才自己出手太快,只使了「不」字訣的四筆,便將四僧一一擊倒,没來得及察看對方的家數,但四僧撲擊時勁力剛猛,顯是少林派外家的路子。都大錦是少林弟子,這些少林僧自是應龍門鏢局之邀,前來赴援的了,可不知兪二哥和莫七弟到了何處,師父命他們前來保護龍門鏢局的老小,怎地以二哥之能,還是給人下了手去?

張翠山心中琢磨了半晌,一部分疑團已獲解答,心道︰「這四個少林僧一去,少林派自是疑心了我,但此事總有水落石出的一日,眞兇到底是誰?少林武當兩派聯手,絶無訪査不出之理。這裡一切且莫移動,眼下是找到二哥和七弟要緊。」於是吹滅燭火,走到牆邉,一躍而出。

他人未落地,突聽得呼的一聲巨響,一件重兵刃攔腰橫掃而來,跟著聽得有人喝道︰「張翠山,躺下了。」張翠山人在半空,無法閃避,敵人這一擊又是既狠且勁,危急之中,伸左掌在敵人兵刃上一按,一借力,輕輕巧巧的翻上了牆頭,這一招乃是「武」字訣中的「弋」,正所謂「差池燕起,振迅鴻歸,臨危制節,中險騰機」,當千鈞一髮之際,轉危爲安。張翠山也是在無可奈何中行險僥倖,想不到新學的這套功夫重似崩石,輕如游霧,竟是決不費力的化解了敵人雷霆般的一擊。

張翠山左足踏上牆頭,右手的判官筆已取在手中,雖未看清敵人的來勢,但適纔這攔腰一擊,剛猛勁狠,實是不可輕視的高手。那忽施襲擊的敵人見張翠山居然能如此從容的避開,也是大出意料之外,忍不住「咦」的一聲,喝道︰「好小子,當眞是有兩下子。」

張翠山左鉤右筆,橫護前心,鉤頭和筆尖都斜向下方,這一招招式叫做「恭聆教誨」,乃是與武林前輩對敵之時的謙敬表示。敵人驀地裡出手,張翠山若不是無意間跟師父學了一套從書法中化出來的武功,早已腰斷骨折,身受重傷,他心中雖然氣惱,但謹守師訓,對武林的高手不敢失禮。黑暗中但見牆下一左一右,分站兩位身披大紅金線袈裟的僧人,每人手中都執著一根金光閃閃的粗大禪杖。張翠山心中一驚,暗道︰「這兩僧身穿大紅金線袈裟,難道是威震天下的『少林十八羅漢』中的人物?」

只見左首那僧人將禪杖在地下一頓,杖尾擊在青石之上,{\upstsl{噹}}的一聲巨響,聲音極是威猛,那僧人跟著説道︰「張翠山,你武當七俠也算是江湖上的成名人物,如何行事這等毒辣?」張翠山聽他直斥已名,既不稱「張五俠」,也不叫「張五爺」,心頭有氣,他外表雖然謙和,但在武當七俠中性子最冷傲,當下冷冷的道︰「大師不問情由,不問是非,躱在牆下偸偸摸摸的忽施襲擊,這算是英雄好漢的行徑嗎?素聞少林派武功馳名天下,想不到暗算手段也是另有獨得之祕。」那僧人怒吼一聲,橫挺禪杖,躍向牆頭,人未到,杖頭已然襲到。張翠山但覺一股勁風點至胸口,當下虎頭鉤一帶,封住了禪杖的來勢,判官筆疾點而出,{\upstsl{噹}}的一聲,筆尖斜{\upstsl{砸}}杖身,那僧人只覺手臂一震,竟爾站不上牆頭,重又落在地下。但這一招一交上,張翠山但覺雙臂發麻,不禁暗自吃驚,原來這僧人膂力之大,實是異乎尋常,心想另一個僧人倘若跟著功夫相捋,兩人聯手夾攻,自己只怕抵擋不住,當下喝道︰「兩位是誰,請通法號!」

右首那僧人緩緩的道︰「貧僧圓音,這是我師弟圓業。」張翠山倒垂鉤筆,拱手道︰「原來是『少林十八羅漢』中的兩位大師,小可久仰清名,不知有何見教?」圓音説話似乎有氣没力,呼吸喘急,説道︰「這事関係少林武當兩派門戸大事,貧僧師兄弟乃少林派的末學後進,没有咱們置喙的餘地,只是今日既撞上了這件事,只想請問張五俠,龍門鏢局這數十口性命,還有我兩個師侄也死在張五俠手下,常言道人命関天,如何善後,要請張五俠的示下。」他説的辭意似乎謙抑,但聲勢咄咄逼人,爲人顯是比圓業厲害得多。

張翠山冷笑道︰「龍門鏢局中的命案是何人所爲,小可也正大感奇怪。大師一口咬定是小可下的毒手,可是大師親眼所見麼?」圓音叫道︰「慧風,你來跟張五俠對質一下。」只見樹叢後走出四個黃衣僧人,依稀正是適纔在鏢局之中,給張翠山一招「不」字訣擊倒的四人。那法名慧風的僧人躬身道︰「啓稟師伯,龍門鏢局數十口性命,還有慧通、慧光兩位師弟,都是\dash{}這姓張的惡賊下的手。」圓音道︰「你們可是親眼所見?」慧風道︰「確是親眼所見,若不是弟子等四人逃得快,也都已死在這惡賊的手下。」圓音道︰「佛門弟子可不能打誑語,此事関連著我少林和武當兩大門派,你千萬胡説不得。」慧風雙膝跪地,合什説道︰「我佛在上,弟子慧風所云,實是眞情,決不敢歉矇師伯。」圓音道︰「你將眼見的情景,一一照實説來。」張翠山聽到這裡,從牆頭飄身而下。

圓業只道張翠山是要加害慧風,揮動禪杖疾向他頭頂頸間掃去。張翠山頭一低,搶步上前已轉到了慧風身後。圓業一擊不中,按著這伏魔杖的招數,本當帶轉禪杖,迴擊張翠山的肩頭,但他此時已站在慧風身後,禪杖若是迴轉,勢須先擊到慧風,一驚之下,硬生生的收住禪杖,喝道︰「你待怎地?」張翠山道︰「我要仔仔細細的聽一聽,聽他説怎生見到我殺害鏢局中人。」

慧風眼見張翠山欺近自己身旁相距不過兩尺,他只須手中兵刃一動,自己立時喪命,雖有兩位師伯在旁,却也相救不及,但他心中憤激,竟是凜然不懼説道︰「圓心師叔在江北接到都大錦都師兄求救告急的書信,當即派慧通、慧光兩位師兄星夜啓程赴援,其後又傳來號令命弟子帶同三名師弟,趕來龍門鏢局。咱們一進鏢局,慧光師兄就説今夜恐有強敵到來,命咱四人埋伏在東邉照牆之下應敵,又説小心别中了敵人的調虎離山之計,不可隨便走動。」圓音道︰「後來怎樣,再説下去。」慧風道︰「天黑之後没多久,便聽得慧通師兄呼叱喝罵,與人在後廳動手,接著他一聲慘呼,似乎身受重傷。我忙奔到後廳去看,只見他\dash{}他\dash{}這姓張的惡賊\dash{}」

他説到這裡,霍地站起,伸著手指,直點到張翠山的鼻尖上,跟著道︰「我們親眼見你一掌把慧光師兄推到牆上。將他撞死。我自知孤身不是你這惡賊的敵手,便伏在窗上,只見你直奔後院殺人,接著八個鏢局子的人從後院逃了出來,你跟蹤追到,伸指一一點斃,直至鏢局滿門老小給你殺得清光,你纔躍牆出去。」

張翠山一動不動的站住,慧風講得口沫橫飛,許多水珠都濺到他臉上。他既不閃避,也不出手,只冷冷的道︰「後來怎樣?」慧風憤然道︰「後來麼?後來我回至東牆。和三位師弟一商量,都覺你武功太強,咱四人敵你不彀,只有在鏢局中等候三位師伯到來,再請示下。那知等不了多久,你這狼心狗肺的惡賊居然又破門而入,這次却是指名道姓的找都總鏢頭來著。咱四人明知是送死,却也要跟你一拚。我大著膽子問你姓名,你不是自報姓名,叫做『銀鉤鐵劃張翠山』麼?我初時還不能相信,只道你名列『武當七俠』,不該做出這等殺人不眨眼的邪惡勾當來,但你自露兵刃,那難道是假的麼?」

張翠山道︰「我自報姓名,露出兵刃此事,半點不假,你們四位,也是我出手打倒。但你再説一遍,這鏢局中數十口的命案,確是你親眼瞧見我姓張的所幹!」便在此時,圓音衣袖一揮,將慧風身子帶起,推出數尺,森然道︰「你便再説一遍,要教這位名震天下的張五俠無可抵賴。」他揮袖將慧風推開,是使他身離險地,免得張翠山惱怒之下,突然間殺人滅口,那可是死無對證了。

慧風道︰「好,我便再説一遍,我親眼目睹,見到你出掌擊死慧光、慧通兩位師兄,見到你出指點死鏢局的八個人。」張翠山道︰「你瞧清楚了我的面貌麼?我是穿這一身衣服麼?」説著一晃火摺,在自己臉上照了一照。慧風瞪視著他的面容,恨恨的道︰「你就是穿這身衣服,長袍方巾,不錯,你那時左手拿著一把摺扇,這把扇子,現下你插在頭頸裡啦。」張翠山惱怒如狂,不知他何以要誣陥自己,高舉火摺,走上兩步,喝道︰「你有種便再説一遍,殺人者便是我張翠山,不是旁人!」慧風雙眼中突然發出奇異的神色,指著他道︰「你\dash{}你\dash{}」猛地裡身子翻倒,橫臥在地,圓音和圓業同聲驚呼,一齊搶上扶起,只見他雙目大睜,滿臉惶惑驚恐之色,却已氣絶而死。

\chapter{妙齡少女}

圓音叫道︰「你\dash{}你打死他了?」這件事變起倉卒,圓音和圓業是驚怒交集,張翠山也是大出意料之外,急忙回頭,只見身後的樹叢輕輕一動。張翠山喝道︰「慢走!」縱身躍起,明知樹叢中有人隱伏,這一竄下去極是危險,但勢逼處此,若不擒住暗箭傷人的兇手,自己難脱干係,那知他身在半空,只聽得身後呼呼兩響,兩柄禪杖分從左右襲到,左首圓音擊出的一記,比圓業的更是威猛得多,同時聽得這兩僧喝道︰「惡賊休得逃走!」張翠山一筆一鉤齊齊下掠,反手使出一記「刀」字訣,一鉤帶住圓業的禪杖杖頭,判官筆的一撇在圓音禪杖一點,身子借勢竄起,躍上了牆頭,凝目瞧那樹叢時,只見樹梢兀自輕輕搖晃,但隱伏之人早已走得影蹤不見。

圓業怪吼連連,揮動禪杖便要躍上牆來拚命。張翠山喝道︰「追趕正兇要緊,兩位休得阻攔。」圓音氣喘喘的道︰「你\dash{}你在我眼前殺人,還想抵賴什麼?」張翠山揮動虎頭鉤,借力打力,逼得圓業無法上牆。圓音道︰「張五俠,咱們今日也不要你抵命,你抛下兵刃,隨咱們去少林寺吧。」張翠山怒道︰「你二人阻手礙脚,放走了兇手,還在這裡纏夾不清。我跟你們去少林寺幹麼?」圓音道︰「去少林寺聽由本寺方丈發落,你連害本寺三條人命,這種大事我也做主不得。」張翠山冷笑道︰「枉你身居『少林十八羅漢』之一,兇手在你眼前逃走,却也不知。」圓音道︰「善哉,善哉!你傷害人命,決計不容你逃走。」張翠山聽他口口聲聲硬指自己是兇手,心下愈益惱怒,一面跟他鬥口,一面和圓業見招拆招,鬥得極是猛烈,冷笑道︰「兩位大師有本事便擒得我去!」

只見圓業禪杖在地下一撐,借力竄躍起來,張翠山跟著縱起,他的輕功可比圓業高得多了,凌空下擊,捷若御風。圓業橫杖欲擋,張翠山虎頭鉤一轉,嗤的一聲,圓業肩頭中鉤,鮮血長流,負痛吼叫,摔下地來。這一下還是張翠山手下留情,否則鉤頭稍稍一偏,鉤中他的咽喉,圓業當場便得送命。

圓音叫道︰「業師弟,傷得重嗎?」圓業怒道︰「不礙事!你還不出手,婆婆媽媽的幹什麼?」圓音咳嗽一聲,運杖上擊,圓業性子極是悍勇,竟不裹紮肩頭傷口,舞杖如風,雙雙夾擊。張翠山見這兩僧膂力甚強,使的又是極沉重的兵刃,倘若給他們躍上牆頭,自己以一敵二,倒是不易取勝,當下門戸守得極是嚴密,居高臨下,兩僧始終無法攻上。「慧」字輩的三僧武功低得多了,眼見兩位師伯久戰無功,雖欲上前相助,却没插手足處。

張翠山心道︰「爲今之計,須得査明眞兇,没來由跟他們糾纏不清。」筆鉤橫交,封閉敵招來勢,一聲清嘯,正要躍起,忽聽得牆内一人縱聲大吼,聲若霹靂。張翠山脚底一晃,立脚處的那堵牆竟然被人運巨力推倒,一個身材魁梧的僧人從牆頭的缺口中急衝而出,不等張翠山雙脚落地,伸出兩手,便來硬奪他手中兵刃。

黑暗中瞧不清他的面貌,但見他十指如鉤,硬抓硬奪,正是少林派中極厲害的「虎爪功」。圓業叫道︰「心師兄,千萬不能讓這惡賊走了。」張翠山自藝成天下,罕逢敵手,半月前學得「倚天屠龍功」,武藝更高,這時見這少林僧來得威猛,反而起了敵愾之心,將虎頭鉤和判官筆往腰間一插,叫道︰「你少林寺便是十八羅漢齊上,我張翠山又有何懼?」眼見圓心的左手抓到,他右掌一探,一迴一曲,嗤的一聲,已撕下了他僧袍的一片衣袖。圓心手抓剛欲搭上他的肩頭,張翠山一足飛起,正好踢中了他的膝蓋。

豈知圓心的下盤功夫極是堅實,膝蓋上受了這重重的一脚,只是身子一晃,却不跌倒,虎吼一聲,右手跟著便抓了過來。同時圓音、圓業兩條禪杖一點腰肋,一擊頭蓋,齊齊襲到。那圓音説話氣喘吁吁,似乎身患重病,其實在三僧中武功以他最高,一根數十斤重的精銅禪杖,在他使來竟如尋常刀劍一般靈便,點打挑撥,輕捷自如。張翠山乍逢好手,尋思︰「我武當和少林近來齊名武林,到底誰高誰低,却始終没較量過。今日裡正好一試少林高僧的手段。」當下展開一對肉掌,在兩根禪杖、一對虎爪之間,縱橫來去,斬截擒拿、指點掌劈,雖是以一敵三,反而漸漸佔了上風。

要知少林和武當武功,各有長短,武當派中出了一位蓋世奇才張三丰,可是少林寺千餘年的浸潤傳授,究竟非同小可,只不過張翠山此時功夫,在武當派中已一等一的高手,而圓音、圓心、圓業三僧,雖然名列「十八羅漢」,在少林寺中總不過是二流脚色。因之時間一長,張翠山越戰越是神完氣足,揮灑自如,冷不防右手倏出,使個「龍」字訣中的一鉤,抓住了圓業的禪杖,順手一拉,往圓音的禪杖上碰了過去。這一下借力打力,但聽得{\upstsl{噹}}的一下巨響,只震得各人耳中{\upstsl{嗡}}{\upstsl{嗡}}作響。圓音和圓業力氣均大,再加上張翠山的力道,兩人只震得虎口流血,四臂酸麻,兩根禪杖也都變成弧形。圓心一驚之下,撲上相救,張翠山伸足一鉤,反掌在他背心一拍,又是借力打力,便用他自己向前一撲的勁道,將他摔了一交。

張翠山冷笑道︰「要擒我上少林寺去,只怕還得再練幾年。」説著轉身便行。圓心縱身躍起,叫道︰「兇徒休逃!」跟著圓音和圓心也追了上來。張翠山心道︰「這三個和尚糾纏不清,總不成將他們都打死了。」提一口氣,脚下展開輕功便奔。圓心和圓業大呼趕來。他們的輕功雖遠不及張翠山,但口中叫著︰「捉殺人的兇手啊!惡賊休得逃走!」沿著西湖的湖邉窮追不捨。

張翠山暗暗好笑,心想你們怎追得上我?忽聽得身後圓心和圓業不約而同的大叫一聲「啊喲!」圓音却悶哼一聲,似乎也是身上受了痛楚。張翠山一驚回頭,只見三僧都是各伸右手,掩住了右眼,好像眼上中了暗器,果然聽得圓業大聲罵道︰「姓張的,你有種便再打瞎我這隻左眼!」張翠山更是一楞︰「難道他的右眼已給人打瞎了?到底是誰在暗助我?」心念一動叫道︰「七弟,七弟,你在那裡?」原來武當七俠中以七俠莫聲谷發射暗器之技最精,鋼鏢、袖箭、飛梭、鐵釘、金錢鏢、飛蝗石,無一不擅,因此張翠山猜想是莫七弟到了。

他叫了幾聲,却無人答應。張翠山急步繞著湖邉幾株大柳樹一轉,也不見半個人影。那圓業一目被射瞎後,暴怒如狂,不顧性命的要撲上來再和張翠山死拚到底。但圓音知道便是雙目完好,自己三人也不是他的敵手,何況受傷的眼中麻癢難當,那暗器上似乎還餵得有毒,忙拉住圓業,説道︰「業師弟,報仇之事,何必急在一時?這事便是你我肯罷休,老方丈和兩位師伯能放過麼?」

張翠山見三僧不再追來,滿腹疑團,心想︰「我自恃輕功了得,但暗中隱伏之人,却高我甚多,看來這人對我並無惡意,只不知是那一位高人。」當下不敢在湖畔多所逗留,急步趕回客店,没奔出數十丈,只見湖邉蘆葦不住擺動。此時湖上無風,蘆葦自擺,定是藏得有人,張翠山輕輕走近,正要出聲喝問,忽見蘆葦中猛地躍出一人,一刀向張翠山頭頂砍下,喝道︰「不是你死,便是我亡!」

張翠山一斜身,飛起右脚,踢在他的右腕,那人戒刀脱手,白光一閃,那刀撲通一聲,落入了湖中,看那人時,僧袍光頭,又是一個少林僧。張翠山喝道︰「你在這裡幹什麼?」只見蘆葦叢中躺著三人,不知是死是傷。他見那少林僧武功平平,心中對他也不加顧忌,走上幾步俯身一看,只見躺著的三人正是龍門鏢局的都大錦和祝史二鏢頭。張翠山一驚,叫道︰「都總鏢頭,你\dash{}你怎地\dash{}」一言未畢,都大錦倏地躍起,雙手牢牢揪住了張翠山胸口衣服,咬牙切齒的道︰「好惡賊,我只不過留下三百兩黃金,你便下這毒手!」張翠山道︰「你幹什麼?」待要施擒拿法掙脱,只見他眼角邉、嘴角邉都是鮮血,此時雖在黑夜,但因和他相距不過半尺,看得甚是清楚,驚道︰「你受了内傷麼?」

都大錦向那少林僧叫道︰「師弟,你認清楚了,這人叫作銀鉤鐵劃張翠山,便是\dash{}便是害人的兇手。你快走,快走,别要被他追上\dash{}」突然間雙手一緊,將額頭往張翠山額上猛撞過去,却是要跟他撞得頭碎骨裂,同歸於盡。張翠山急忙雙手翻轉,在他臂上一推,只聽得嗤的一聲響,都大錦摔了出去,但自己胸口衣襟也被他扯了一大片下來。張翠山生平無所畏懼,然而今晩迭見異事,都大錦的神情又大是令人生怖,不由得心中怦怦而跳,俯首一看,只見都大錦雙眼翻白,已然氣絶,那自是早受極重的内傷,自己在他臂上這麼輕輕一推,決不能致他的死命。

那少林僧失聲驚呼︰「你\dash{}你又殺了都師兄\dash{}」轉身没命的奔逃,又慌又急,只奔出數步,便摔了一交。張翠山搖了搖頭,見祝史兩鏢頭雙足浸在湖水之中,已死去多時。

張翠山瞧著三具屍體,大是憮然,他雖和都大錦並無交情,而都大錦護送兪岱岩出了差池,他更是一直惱恨在心,但眼見他忽而不明不白的死去,總是不免有傷逝之感,在湖畔悄立片刻,忽想︰「都大錦説道︰『好惡賊,我只不過留下三百兩黃金,你便下這毒手!』我叫他將二千兩黃金都救濟災民,想是他捨不得,暗中留下三百兩。其實别説我並知情,便是知道,也只一笑了之,豈有跟他爲難之理?」一提都大錦的背囊,果是沉甸甸的,伸指撕開包袱,囊中跌出幾隻金元寶,滾在都大錦的臉旁。便在這霎時之間,張翠山忽興人生無常之感,這位總鏢頭一生勞累,千里奔波,在刀尖上拚命,只不過是爲了一些黃金,眼前黃金好端端的在他身旁,可是他却再無法享用了。再想自己此刻力戰少林三僧,大獲全勝,固是英雄一時,但百年之後,和都大錦也是無所分别,想到此處,不由得嘆了一口長氣。

忽聽得琴韻冷冷,出自湖中,張翠山抬起頭來,只見先前在鏢局外湖中所見的那個少年文士,正在舟中撫琴。只聽他彈了幾句,曼聲作歌︰

\begin{quotation}
興酣落筆搖五岳\hskip8pt詩成嘯傲凌滄洲

功名富貴若長在\hskip8pt漢水亦應西北流
\end{quotation}

歌聲清脆嬌嫩,似是女子的聲音。張翠山微微一驚︰「此人歌中之意,正好説中了我的心事,倒是巧合。」眼見脚下是三具屍體,那人的遊船若是搖過來瞧見了,聲張起來,驚動蒙古巡兵,不免多惹麻煩。正要行開,忽聽那文士在琴絃輕輕撥三下,抬起頭來,説道︰「兄台既有雅興子夜遊船,何不便來舟上?」説著將手一揮,後梢伏著的一個舟子坐起身來,盪起雙槳,便將小舟划近岸邉。

張翠山心道︰「此人一直便在湖中,或曾見到什麼,倒可向他打聽打聽。」於是走至一株大柳樹下,待小舟划近,輕輕一躍,上了船頭。

張翠山的輕功極是佳妙,從岸上跳到舟中,那小舟竟是不低不晃。舟中的書生站起身來,微微一笑拱手爲揖,左手向著上首的座位一伸,請客人坐下。碧紅燈籠照映下,這書生手白勝雪,再看他相貌,玉頰微瘦,眉彎鼻挺,一笑時左頰上淺淺一個酒渦,遠觀之似是個風流俊悄的公子,但這時相向而坐,顯是一個女扮男裝的絶色麗人。

張翠山雖倜儻瀟灑,但師門規矩,男女之防守得極緊。武當七俠行走江湖,於女色上人人律己嚴謹,他一見對方竟是個女子,一愕之下,登時滿臉通紅,站起身來,立時倒躍回岸,拱手説道︰「在下不知姑娘女扮男裝,多有冒昧。」那美書生不答,撫琴輕歌,歌曰︰「多慮令志散,寂寞使心憂,{\upstsl{翱}}翔觀彼澤,撫劍登輕舟。」

張翠山聽她歌中之意,竟是邀己上舟,心想︰「今晩遇上許多難解之事,這位姑娘若有所見,當可助我洗雪冤枉。」待要再到舟上,又想︰「這姑娘素不相識,又是如此美貌絶俗,午夜和她舟中相見,只怕於她清名有累。」正沉吟間,忽聽得槳聲響起,那小舟竟緩緩盪向湖心,但聽那姑娘撫琴歌道︰「今夕興盡,來宵悠悠,六和塔下,垂柳扁舟。彼君子兮,寧當來游?」舟去漸遠,歌聲漸低,但見波影浮動,一燈如豆,隱入了湖光水色。

在一番刀光劍影,腥風血雨的劇鬥之後,忽然遇上這等飄忽旖旎的風光,張翠山悄立湖畔,不由得思如潮湧,過了半個多時辰,這纔回去客店。

次日龍門鏢局殺死數十口的大命案,在臨安城中已傳得人人皆知,好在張翠山蘊籍儒雅,誰也不會疑心到他身上。午前午後,他在市上和寺觀到處閒逛,尋訪二師兄兪蓮舟和七師弟莫聲谷的蹤跡,但走了一天,竟找不到武當七俠相互聯絡的半個記號。到得申牌時分,心中不時響起那少女的歌聲︰「今夕興盡,來宵悠悠,六和塔下,垂柳扁舟。彼君子兮,寧當來游?」那少女的形貌,更是在心頭拭抹不去,尋思︰「我但當持之以禮,跟她一見又有何妨?若是二師哥和七師弟在此,和他二人同去自是更好,但此刻除了從她身上之外,更無第二處可去打聽昨晩命案的眞相。」用過晩飯,逕往錢塘江邉的六和塔下走去。

那錢塘江到了六和塔下轉一個大彎,然後直向東流。張翠山脚下雖快,該處和府城相距不近,到得六和塔下時,也已將黑,只見塔東的三株大柳樹下,果然繫著一艘扁舟。錢塘江中的江船張有風帆,自比西湖裡的遊船大得多了,但船頭掛著的一盞碧紗燈籠,却和昨晩所見的一模一樣。張翠山心中怦怦而跳,定了定神,走到大柳樹下,只見碧紗燈下,那少女悄然獨坐船頭,身穿淡綠衫子,却已改了女裝。

張翠山本來立定主意要問她昨晩之事,這時見她換了女子裝束,却躇躊起來,忽聽那少女仰天吟道︰「抱膝船頭,思見嘉賓,微風動波,惘焉若酲。」張翠山朗聲道︰「在下張翠山,有事請教,不敢冒昧。」那少女道︰「請上船吧。」張翠山輕輕躍上船頭。那少女道︰「昨晩烏雲蔽天,没有月亮,今宵雲散天青,却比昨晩好得多呢。」聲音嬌媚清脆,但説話時眼望天空,竟没向他瞧上一眼。張翠山道︰「不敢請問姑娘尊姓。」少女突然轉過臉來,兩道清澈明亮的眼光在張翠山面上轉了兩轉,並不答話。張翠山見她清麗不可方物,爲她的容光所逼,登時自慚形穢,不敢再説什麼,轉身一躍上岸,發足往來路奔回。

張翠山奔出數十丈,斗然停步,心道︰「張翠山啊張翠山,你昂藏七尺,男児漢大丈夫,十年來縱橫江湖,無所畏懼,今日却怕起一個年輕姑娘來?」側頭一望,只見那少女所坐的船沿著錢塘江,順流緩緩而下,一盞碧紗燈照映江面,張翠山一時心意未定,在岸邉信步而行。人在岸上,舟在江中,一人一舟並肩而下,那少女仍是抱膝坐在船頭,望著天邉新升的眉月。

張翠山走了一會,不自禁的順著她目光也向月亮一看,却見東北角上湧起一大片烏雲,當眞是天有不測風雲,這烏雲湧得甚快,不多時便將月亮遮住,一陣風過去,便撒下細細的雨點來。這江邉一望平野,無可躱雨之處,但張翠山心中怔怔的,却也没想到要躱雨,雨雖不大,但時候一久,身上便已濕透。只見那少女仍是坐在船頭,自也是淋得全身皆濕,張翠山猛地想起,叫道︰「姑娘,你進船艙避雨啊。」那少女「啊」的一聲站起身來,一怔道︰「難道你不怕雨了?」

她説著便進了船艙,過不多時,從艙裡出來,手中多了一把雨傘,手一揚,將那傘向岸上擲來。張翠山伸手接住,見是一柄油紙小傘,一張開,見傘上畫著遠山近水,數株垂柳,是一幅淡雅的水墨山水畫,還題著七個字道︰「斜風細雨不須歸。」杭州的傘上多有書畫,自來如此,那也不足爲奇,但傘上的繪畫書法出自匠人手筆,便和江西的瓷器一般,總是帶著幾分匠氣,豈知這把小傘上的書畫竟是十分精緻,那七個字雖冩得微嫌勁力不足,但清麗脱俗,宛然是出自閨秀之手。張翠山抬起了頭欣賞,足下並不停步,却不知前面有一條小溝,他左脚一脚踏下,竟踏了個空,若是常人,這一下非摔了個大筋斗不可。但他功夫何等了得,當下變招奇速,右足向前踢出,身子已然騰起,輕輕巧巧的跨過了小溝,只聽舟中的少女喝了聲采︰「好!」張翠山轉過頭去,見她頭上戴了一頂斗笠,站在船頭,風雨中衣袂飄飄,眞如凌波仙子一般。

那少女道︰「傘上的書畫,還能入張先生雅眼麼?」張翠山道︰「這筆衛夫人名姬帖的書法,筆斷意連,筆短意長,極盡簪花冩韻之妙。」那少女聽他認出自己的字體,心下甚喜,説道︰「這七字之中,那個『不』字冩得最不好。」張翠山細細凝視,道︰「這『不』字冩得很自然啊,只不過少了些含蓄,不像其餘的二個字,餘韻不盡,觀之令人忘倦。」那少女道︰「是了,我總覺這字冩得不愜意,却想不出是什麼地方不對,經先生一説,這纔恍然。」

這時她所乘之舟不停的順水下駛,張翠山仍在江岸上伴舟而行,兩人談到書法,一問一答,不知不覺間竟行出十餘里之遙。這時天色更加黑了,對方面目早已瞧不清楚,那少女忽道︰「聞君一席話,勝讀十年書,多謝張先生指點,就此别過。」她手一揚,後梢的舟子拉動帆索,船上風帆慢慢升起,白帆鼓風,登時行得快了。張翠山見帆船漸漸遠去,頗是悵然,只聽得那少女遠遠的道︰「我姓殷\dash{}他日有暇,再向先生請教\dash{}」

張翠山聽到「我姓殷」這三字,心頭驀地一驚︰「那都大錦曾道,託他護送兪三哥的,是一個相貌俊美的書生,自稱姓殷,莫非便是此人?」他想至此事,再也顧不得什麼男女之嫌,提氣疾追,那帆船駛得雖快,但他展開輕功,不多時便已追及,朗聲問道︰「殷姑娘,你識得我兪三哥兪岱岩嗎?」那少女轉過了頭,並不回答,張翠山似乎聽了一聲嘆息,只是一在岸上,一在舟中,却也聽不明白,不知到底是不是嘆氣。張翠山又道︰「我心下有許多疑團,要請剖明。」那少女道︰「又何必一定要問?」

張翠山道︰「委託龍門鏢局護送我兪三哥赴鄂的,可就是殷姑娘麼?此番恩德,務須報答。」那少女道︰「恩恩怨怨,那也難説得很。」張翠山道︰「我三哥到了武當山下,却又遭了毒手,殷姑娘可知道麼?」那少女道︰「我很難過,也覺抱憾。」他二人一問一答,風勢漸大,帆船越行越快,張翠山内力深厚,始終和帆船並肩而行,竟是没落後半步。在風雨之中,那少女説話聲音不響,却也一字一句,清清楚楚的送入張翠山耳中,足見她中氣充沛,武功底子大是不淺。

那錢塘江越到下游,江面越闊,而斜風細雨也漸漸變成了狂風暴雨。張翠山問道︰「昨晩龍門鏢局滿門數十口被殺,是誰下的毒手,姑娘可知道麼?」那少女道︰「我跟都大錦説過,要好好護送兪三俠到武當,若是路上出了半分差池\dash{}」張翠山道︰「你説要殺得他鏢局中雞犬不留。」那少女道︰「不錯。他没好好保護兪三俠,這是他自取其咎,又怨得誰來?」張翠山心中一寒,道︰「鏢局中這許多人命,都是\dash{}都是\dash{}」那少女道︰「都是我殺的。」張翠山耳中{\upstsl{嗡}}的一響,實難相信這個嬌媚如花的少女,竟是殺人不眨眼的兇手,過了一會,説道︰「那\dash{}那兩個少林寺的和尚?」那少女道︰「也是我殺的。我本來没想要和少林寺結仇,不過他們對我言語無禮,便饒他們不得。」張翠山道︰「怎麼\dash{}怎麼他們又冤枉我?」那少女微微一笑,道︰「那是我安排下的。」張翠山氣往上衝,大聲道︰「是你安排下,叫他們冤枉我?」那少女道︰「不錯。」張翠山怒道︰「我跟姑娘無怨無仇,何若如此?」

只見那少女衣袖一揮,鑽進了船艙之中,到此地步,張翠山如何能不問個明白?眼見那帆船離岸十餘丈遠,無法一躍而至,狂怒之下,伸掌向岸邉一株楓樹猛擊,喀喀數聲,折下兩根粗枝。他用力將一根粗枝往江中一擲,左手提了另一根樹枝,右足一點,躍向江中,左足在那粗枝上一借力,向前躍出數丈,跟著將另一根粗枝又抛了出去,右足點上樹枝,再一借力,躍到了船頭,大聲道︰「你\dash{}你怎麼安排?」

但是船艙中黑沉沉的寂然無聲,張翠山正要舉步跨進,但他盛怒之下,仍是頗有自制,心想︰「擅自闖入婦女的船艙之中,未免無禮!」忽見火光一閃,艙中點亮了蠟燭!那少女道︰「請進來吧!」

張翠山整了整衣冠,收攏雨傘,走進船艙,却不由得一怔,只見船艙中坐著一個少年書生,方巾青衫,摺扇輕搖,神態甚是瀟灑,原來那少女在這頃刻之間,又已換了男裝,一瞥之下,竟與張翠山的形貌極其相似。他問她如何安排使得少林派冤枉自己,但那少女這一換裝,不用答覆,已使張翠山恍然大悟,黑暗之中,誰都把他二人混而爲一,無怪少林僧慧風和都大錦均一口咬定,是自己下的毒手。那少女伸摺扇向對面的座位一指,説道︰「張五俠,請坐。」提起几上的細瓷茶壼斟了一杯茶,送到他的面前,説道︰「寒夜客來茶當酒,舟中無酒,未免有減張五俠的清興了。」

她這麼斯斯文文的斟一杯茶,登使張翠山滿腔怒火發作不出來,只得欠身道︰「多謝。」那少女見他全身衣履盡濕,説道︰「舟中尚有衣衫,春寒料峭,張五俠到後梢換一換吧。」張翠山搖頭道︰「不用。」當下暗運内力,一股暖氣從丹田升了起來,全身滾熱,衣服上的水氣漸漸散發。那少女道︰「武當派内功甲於武林,小妹請張五俠更衣,眞是井底之見了。」張翠山道︰「姑娘是何宗何派,可能見示麼?」那少女聽了他這句話,眼望窗外,眉間登時罩上一層愁意。

張翠山見她神色似有重憂,倒也不便苦苦相逼,但過了一會,忍不住又問︰「我兪三哥到底是何人所傷,姑娘可能見示麼?」那少女道︰「不單是都大錦走了眼,其實我也上了當。我早該想到武當七俠英姿颯爽,怎會是如此險鷙粗魯的人物。」張翠山聽她不答自己問話,却説到「英姿颯爽」四字,顯是當面讚賞自己的丰采,心頭怦的一跳,臉上微微發燒,却不明白她説這幾句話是什麼意思。那少女嘆了口氣,突然捲起左手衣袖,露出白玉般的手臂來,張翠山急忙低下了頭,不敢觀看。那少女道︰「你認得這暗器麼?」張翠山聽她説到「暗器」兩字,這纔抬頭,只見她左手臂上釘著三枚小小的黑色鋼鏢。她膚白如雪,但中鏢之處却深黑如墨。

那三枚鋼鏢尾部均作梅花形,鋼鏢上只不過一寸半長,却有寸許深入肉裡,張翠山大吃一驚,霍地站起,叫道︰「這是少林派的梅花鏢,怎\dash{}怎地是黑色的?」那少女道︰「不錯,是少林派的梅花鏢,鏢上餵得有毒藥。」她晶瑩潔白的手臂上釘了這三枚小鏢。燭光之下看來,又是艷麗動人,又是詭祕可怖,便如雪白的宣紙上用黑墨點了三點。張翠山道︰「少林派是名門正派,暗器上決計不許餵毒,但這梅花小鏢除了少林子弟之外,却没聽説還有那一派的人物會使。」那少女道︰「這事我也好生奇怪,正如尊師所云,捏斷令師兄四肢筋骨的,便是少林寺的絶技『金剛指』手法。」張翠山更是奇怪,心道︰「師父在武當山上説這幾句話,除了自己師兄弟外,並無外人在座,怎地她也知道了?」忙問︰「姑娘遇到我二師哥兪蓮舟和七師弟莫聲谷了?」那少女搖頭道︰「除了在武當山見過一面,此後没再見到。」張翠山大奇,道︰「姑娘到過我武當山,怎地我不知情?\dash{}咦,姑娘中鏢有多久了?快些設法解毒要緊。」説這些話時,関切之情見於顏色。

那少女心中感激,道︰「中鏢已二十餘日,那毒性給我用藥逼住了,一時不致散發開來,但這三枚惡鏢却也不敢起下,只怕鏢一拔出,毒性隨血四走。」張翠山知道這般逼住毒性,除了靈丹妙藥之外,尚須極深湛的内力,眼看這少女不過十八九歳年紀,居然有此本事,心下暗自欽佩,忍不住説道︰「中鏢二十餘日再不起出,只怕\dash{}只怕\dash{}將來治愈後,肌膚上會有極大\dash{}極大的疤痕\dash{}」其實心中本來想説︰「只怕毒性在體内停留過久,這條手臂要廢。」又道︰「如此美玉無瑕般的手臂之上,若是留下三個疤痕\dash{}」那少女泪珠瑩然,幽幽的道︰「我已經盡力而爲\dash{}昨天晩上在那少林僧身邉又没搜到解藥\dash{}我這條手臂是不中用的了。」説著慢慢放下了衣袖。

張翠山胸口一熱,道︰「殷姑娘,你信得過我麼?在下的内功雖淺,但自信尚能相助姑娘逼出臂上的毒氣。」那少女嫣然一笑,露出頰上淺淺的梨渦,似乎心中極喜,但隨即説道︰「張五俠,你心下疑團甚多,我先跟你説個明白,免得你助了我之後,心下却又懊悔。」張翠山昂然道︰「治病救人,原是我輩當爲之事,怎會懊悔?」那少女道︰「好在二十多天也熬過來啦,也不忙在這一刻。我跟你説,我將兪三俠交付了龍門鏢局之後,自己便跟在鏢隊後面,道上果然有好幾起人想對兪三俠下手,都給我暗中打發了,可笑都大錦猶如睡在夢中。」張翠山拱手道︰「姑娘大恩大德,我武當子弟感激不盡。」那少女冷然道︰「你不用謝我,待會你恨我也來不及呢。」張翠山一呆,不明其意。那少女又道︰「我一路上更換裝束,有時裝作農夫,有時扮作商人,遠遠跟在鏢隊之後,那知到了武當山脚下却出了岔子。」

\chapter{毒梅花鏢}

張翠山咬牙道︰「那六個惡賊,姑娘親眼瞧見了?可恨都大錦矇矇瞳瞳,語焉不詳,説不明白這六賊的來歷。」那少女嘆了口氣道︰「我不但見了,還跟他們交了手,可是我也矇矇瞳瞳,説不明白他們的來歷。」她拿起茶杯,喝了一口,説道︰「那是我見這六人從武當山迎下來,都大錦跟他們招呼,稱之爲『武當六俠』,那六人也居之不疑。我遠遠望著,見他們將兪三俠所乘的大車接了去,心想此事已了,於是勒馬道旁,讓都大錦等一行人走過,但一瞥之下,却看出了一個老大破綻。小妹當時心想︰『武當七俠是同門的師兄弟,情同骨肉,兪三俠身受重傷,他們該當一擁而上,立即看他傷勢纔是。但他們只有一人往大車中望了一眼,餘人非但並不理會,反而頗有喜色,大聲忽哨,趕車而去,這可不是人情之常。』」張翠山點頭道︰「姑娘心細,説得甚是。」

那少女道︰「我越想越是不對,於是縱馬追趕上去,喝問他們姓名。這六人眼力倒大是不弱,一見面就看出我是女子。我罵他們冒充武當子弟,劫持兪三俠存心不良。三這兩語,我便衝上去動手。六人中出來一個二十來歳的瘦子跟我相鬥,一個道士在旁掠陣,其餘四人便趕著大車走了。那瘦子手底下竟是極硬,三十餘合中我勝他不得,突然間那道人左手一起,我只感臂上一麻,無聲無息的便中了這三枚梅花鏢。一中鏢,手臂登時麻癢,那瘦子出言無禮,想要將我擒住,我還了他三枚金針,這纔脱身。」説到這裡,臉上微現紅暈,想是那瘦子見她是孤身的美麗少女,竟有非禮之意。

張翠山沉吟道︰「這梅花小鏢用左手發射,那比用右手發射又難得多,少林派的門下怎地出現了道人,莫非也是喬裝的?」那少女微笑道︰「道士扮和尚須得剃個光頭,和尚扮道士却容易得多,戴頂道冠便成了。」張翠山微微一笑。那少女道︰「我心知此事不妙,但瘦子我尚自抵敵不過,那道人似乎更厲害得多,何況他們共有六人?這可没了計較。」張翠山張口欲言,但終於又忍住了。那少女道︰「我猜你是想説︰『幹麼不上武當來跟咱們説明?』是不是?我可不能上武當啊,倘若我自己能出面,又何必委託都大錦走這趟鏢呢?我正自徬徨無計,一個児在道門上悶走,恰好撞到你跟都大錦他們説話。你去找尋兪三俠,我便混在鏢隊之中,到了武當山上。大家驚駭悲痛之下,誰也没有細問,你們當我是鏢局的,都大錦他們却又當我是武當山上的。」張翠山忽然想起,道︰「那日你扮作一個車夫,帽簷児壓得低低的,是不是?」那少女笑道︰「張五俠好厲害的眼力,倘若你不是有要事在身,只怕已被你揭破了。但我終究還是被宋大俠認了出來。」張翠山奇道︰「我大師哥認了你出來?他可没説啊。」

那少女道︰「宋大俠爲人極是厚道,他一句話也不説,只在安排住宿之處時,單獨給了我一間耳房。」張翠山道︰「大師哥爲人,正是如此。」那少女道︰「後來我隨同都大錦等一同下山,看到你迫他們將那二千兩黃金吐出來救濟災民。張五俠,你倒很會慷他人之慨,這二千兩黃金是我的啊。」張翠山笑道︰「那我替災民們謝謝你啦。」那少女道︰「可是財入光棍之手,他怎肯盡數吐出來?總算張五俠威名太大,他不敢不吐,只藏下了三百兩。回到了這裡,我叫人一看這梅花鏢,有人識得是少林派的獨門暗器,説道除非是發暗器之人的本門解藥,否則毒性難除。臨安府中除了龍門鏢局,還能有誰是少林派?於是我夜入鏢局,逼迫他們取出解藥,豈知他們不但不給,還埋伏下了人馬,我一進門便對我猛下毒手。」

張翠山「{\upstsl{嗯}}」了一聲,沉吟道︰「你却説故意安排,教他們認作是我?」那少女臉有靦腆之色,低下了頭,輕輕的道︰「我見你到衣舖去買了這套方巾,覺得穿戴起來很是\dash{}很是好看,於是我跟著也買了一套。」張翠山道︰「這便是了。只是你一出手便連殺數十人,未免過於狠辣,鏢局中的人又和你没有怨仇。」那少女登時沉下臉來,冷笑道︰「你要教訓我麼?我活了一十九歳,倒還没聽人教訓過呢。張五俠大仁大義,這便請便吧,我這種心狠手辣之輩,原没盼望跟你結交。」

張翠山給她一頓數説,不由得滿臉通紅,霍地站起,待要出艙,但隨即想起自己答應了助她治臂上之毒,於是説道︰「請妳捲起衣袖。」那少女峨嵋微豎,説道︰「你愛罵人,我不用你治了。」張翠山道︰「你臂上之傷延誤已久,再耽誤下去只怕\dash{}只怕送了你的小命。」那少女恨恨的道︰「送了性命最好,反正是你害的。」張翠山奇道︰「咦,那少林派的惡人發鏢射你,跟我有什麼相干?」那少女道︰「倘若我不是千里迢迢的護送你三師哥上武當,會遇上這六個惡賊麼?這六人搶了你師哥去,我若是袖手旁觀,臂上會中鏢麼?你倘是早到一步,助我一臂之力,我會中鏢受傷麼?」

除了最後兩句話有些強辭奪理,另外的話也是合情合理,張翠山拱手道︰「不錯,在下助姑娘療傷,那只是略報大德。」那少女側頭道︰「那你認錯了麼?」張翠山道︰「我認什麼錯?」那少女道︰「你説我心狠手辣,這話是説錯了。那些少林和尚、都大錦這干人、鏢局中的,全都該殺。」張翠山搖了搖頭,道︰「姑娘雖然臂上中毒,但仍可有救。我三師哥身受重傷,也未斃命,即使當眞不治,咱們也只找首惡,這樣一舉連殺數十人,總是於理不合。」那少女秀眉一揚,道︰「你説我殺錯了人?難道用梅花鏢打我的不是少林派的人嗎?難道龍門鏢局不是少林派開的麼?」張翠山道︰「少林派門徒佈於天下,成千成萬,姑娘只不過中三枚鏢,難道便要殺盡少林門下弟子?」

那少女辯他不過,忽地舉起右手,一掌在左臂上拍落,著掌之處,正是那三枚梅花鏢的所在,這一掌下去,三鏢深入肉裡,傷得可就更加重了。張翠山萬料不到這少女脾氣如此怪誕,一言不合,便下重手傷殘自己肢體,她對自身尚且如此,出手隨便殺人自是不在意下了,待要阻擋,爲勢已是不及,急道︰「你\dash{}你何苦如此?」只見她衫袖中滲出黑血。張翠山知道此時鏢傷太重,她内力已阻止不住毒血上流,若不急救,立時便有性命之憂,當下一手探出,抓住了她的左臂,右手便去撕她衫袖。

忽聽得背後有人喝道︰「狂徒不得無禮!」呼的一聲,有人揮刀向他背上砍來。張翠山知是船上舟子,事在緊急,不及細加分辯,反腿一脚,將那舟子踢出艙去。那少女道︰「我不用你救,我自己愛死便死。」説著拍的一聲,清清脆脆的打了他一個耳光。她出掌奇快,手法又極是怪異,這一下竟是令他閃避不及。張翠山一楞,放開了她的手臂。

那少女沉著臉道︰「你上岸去吧,我再也不要見你啦!」張翠山給她這一掌打得羞怒交迸,道︰「好!我倒没見過這般任性無禮的姑娘!」跨步走上船頭。那少女冷笑道︰「你没見過,今日便要給你見見。」張翠山拿起一塊木板,待要抛在江中,踏板上岸,但轉念一想︰「我這一上去,她終究是性命不保。」當下強忍怒氣,回進艙中,説道︰「你打我一掌,我也不來跟你這種不講理的姑娘計較,快捲起袖來。你,要性命不要?」

那少女嗔道︰「我要不要性命,跟你有什麼相干?」張翠山道︰「你千里送我三哥,此恩不能不報。」那少女冷笑道︰「好啊,原來你不過是代三哥還債來著。倘若我没護送過你三哥,我受的傷再重,你也見死不救啦。」張翠山一怔,道︰「那却也未必。」只見那少女忽地打個寒戰,身子微微一顫,顯是毒性上行,忙道︰「快捲衣袖,你當眞是拿自己性命來開玩笑。」那少女咬牙道︰「你不認錯,我便不要你救。」她臉色本是極白,這時嬌嗔怯弱,更增楚楚可憐之態,張翠山嘆了口氣,道︰「好,算我説錯了,你殺人没有錯。」那少女道︰「那不成,錯便是錯,有什麼算不算的。你爲什麼嘆了口氣再認錯,顯然不是誠心誠意的。」張翠山救命要緊,也無謂跟她多作口舌之爭,大聲道︰「皇天在上,江神在下,我張翠山今日誠心誠意,向殷\dash{}殷\dash{}」説到這裡,頓了一頓。

那少女道︰「殷素素。」張翠山道︰「{\upstsl{嗯}},向殷素素姑娘認錯。」殷素素心下大喜,嫣然而笑,猛地裡脚下一軟,坐倒在椅上。張翠山忙從懷中藥瓶裡取出一粒「百草護心丹」給她服下,捲起她衣袖,只見半條手臂已成紫黑色,那黑氣正自迅速上行。張翠山伸左手抓住她上臂,問道︰「妳覺得怎樣?」殷素素道︰「胸口悶得難受。誰教你不快認錯?倘若我死了,那便是你害的。」張翠山當此情景,只能柔聲安慰道︰「不礙事的,你放心。你全身放鬆,一點也不要用力運氣,就當自己是睡著了一般。」殷素素白了他一眼,道︰「就當我已死了一般。」

張翠山心道︰「在這當口,這姑娘還是如此橫蠻刁惡,將來不知是誰做她丈夫,這一生一世可有苦頭吃了。」想到此處,不由得心中怦然而動,臉上登時發燒,生怕殷素素已知覺了自己的念頭,向她望了一眼。只見她雙頰暈紅,大是嬌羞,不知也想到了什麼。兩人眼光一觸,不約而同的轉了開去,殷素素忽然低聲道︰「張五哥,我説話没有輕重,你别見怪。」張翠山聽她忽然改口,把「張五俠」叫作「張五哥」,心中更是怦怦亂跳,當下吸一口氣,收攝心神,一股暖氣從自己丹田中昇了上來,勁貫雙臂。

過了一會,張翠山頭頂籠罩著一層氳氤白氣,顯是用出全力,汗氣上蒸,殷素素心中感激,知道這是療毒的緊要関頭,生恐分了他的心神,閉目不敢和他説話。忽聽得波的一聲,臂上一枚梅花小鏢彈了出來,躍出丈餘,跟著一縷黑血,從傷口中激射而出。這黑血漸漸轉紅,跟著第二枚梅花鏢又被張翠山的内力逼出。

便在此時,忽聽得江上有人縱聲呼道︰「殷姑娘在這児嗎?朱雀壇壇主參見。」張翠山微覺怪異,但運力甚急,不去理會,那人又呼了一聲,却聽自己船上的舟子叫道︰「有狂徒在此欲害殷姑娘,常壇主快來!」那邉船上的人大聲喝道︰「狂徒不得無禮,你只要傷了殷姑娘一根毫毛,叫你身受千刀萬剮之慘。」這人聲若洪鐘,在江面上呼喝過來,大是威猛。

殷素素睜開眼來,向張翠山微微一笑,對這場誤會似表歉意。那第三枚梅花鏢給殷素素一拍之下,入肉甚深,張翠山連運了三遍力道,仍是逼不出來。但聽得槳聲甚急,那艘船飛也似的靠近,張翠山只覺船身一晃,有人躍上船來,他只顧用力,却也不去理會。那人鑽進船艙,但見張翠山雙手牢牢的抓住殷素素左臂,一時那裡想得到他是在運勁療傷,急怒之下,呼的一掌便往張翠山後心拍去,同時喝道︰「惡賊還不放手?」張翠山緩不出手來招架,吸一口氣,挺背硬接了他這一掌,但聽蓬的一聲,這一掌力道奇猛,結結實實的打中了他的背心。

張翠山深得武當派内功的精要,全身不動,但借力卸力,將這沉重之極的掌力引到掌心,只聽得波的一聲響,第三枚梅花鏢從殷素素臂上激射而出,釘在船艙板上,餘勢不衰,兀自顫動。發掌之人一招既出,第二招跟著便要擊落,見了這等情景,第二掌拍到半路,硬生生的收回,叫道︰「殷姑娘,你\dash{}你没受傷麼?」但見她手臂的傷口中噴出毒血,這人也是江湖上的大行家,知道打錯了人,心下好生不安,暗忖自己這一掌有裂石破碑之勁,看來張翠山内臟已盡數震傷,只怕性命難保,忙從懷中取出傷藥,想給張翠山服下。

張翠山搖了搖頭,見殷素素傷口中出來的已是殷紅的鮮血,於是放開手掌,回過頭來,笑道︰「你這一掌的力道眞是不小。」那人大吃一驚,心想自己掌底不知擊斃過多少成名的武林好手,怎麼這少年不避不讓的受了一掌,竟是没事人的一般,説道︰「你\dash{}你\dash{}」瞧了瞧張翠山的臉色,伸出三根手指去搭他的脈搏。張翠山心想︰「索性便開開他的玩笑。」暗運内勁,腹膜上頂,霎時間心臟停止了跳動。要知内功精湛之人,不但能暫停呼吸,且能使心臟暫時停跳,中國的内功和天竺瑜伽之術,凡功夫練到深處,均有這等本事。那人一搭上他手腕,只覺他脈搏已絶,大驚之下伸手去摸他胸口,更是嚇了一跳。張翠山笑道︰「殷姑娘,這位是你朋友麼?你没給咱們引見。」一面説,一面接過殷素素遞來的手帕,替她包紮傷口。那人見他説話行事了無異狀,但一顆心終是不跳,右掌按住了他胸口,竟是驚訝得放不下來。

殷素素臉一沉,道︰「常壇主不得無禮,見過武當派的張五俠。」那人縮手退開,施了一禮,説道︰「原來是武當七俠的張五俠,怪不得内功如此深厚,小人常金鵬多多冒犯,請勿見怪。」張翠山見這人五十來歳年紀,一張馬臉,嘴巴和額角相距極遠,兩隻手掌伸開來便似兩把蒲扇,臉上手上的肌肉凹凹凸凸、盤根錯節,顯是有極深的外門功夫,倘若張翠山所練的内功不正是這種硬功夫的剋星,那麼適纔這一掌眞便要了他的性命。

常金鵬向張翠山見禮已畢,隨即恭恭敬敬的向殷素素施下禮去,殷素素却只大剌剌的點一點頭,不怎麼理會。張翠山心下暗暗納罕,他背上受了常金鵬這掌,知道此人武功實非尋常,怎麼殷素素對他這般無禮,而他却也受之若素,只聽他又道︰「玄武壇白壇主約了海沙派、巨鯨幫,和福建神拳門的人物,明日清晨在錢塘江口的王盤山島上相會,揚刀立威。殷姑娘既然身子不適,待小人護送姑娘回臨安府。王盤山島的事,諒白壇主一人料理起來也綽綽有餘。」殷素素哼了一聲,道︰「海沙派、巨鯨幫、神拳門\dash{}{\upstsl{嗯}},神拳門的掌門人,過三拳也去嗎?」常金鵬道︰「聽説是他親自率領神拳門的十二名高手弟子,前去王盤山赴會。」殷素素冷笑道︰「過三拳名氣雖大,不足當白壇主的一擊,還有什麼好手?」

常金鵬遲疑了一下,道︰「聽説崑崙派有兩名年青劍客,也趕來赴會,説要見識見識屠\dash{}屠\dash{}」説到這裡,眼角向張翠山一掠,却不説下去了。殷素素冷冷的道︰「他們要去瞧瞧屠龍刀嗎?只怕是眼熱起意\dash{}」張翠山聽到「屠龍刀」三字,心中一凜,只聽殷素素又道︰「{\upstsl{嗯}},這幾年武林中長江後浪推前浪,人才輩出,崑崙派的人物倒是不可小覷了。我臂上的輕傷算不了什麼,這麼著,咱們也去瞧瞧熱鬧,説不定須得給白壇主助一臂之力。」她轉頭向張翠山道︰「張五俠,咱們就此别過,我坐常壇主的船,你坐我的船回臨去吧!你武當派犯不著牽涉在内。」

張翠山道︰「我三師哥之傷,似與屠龍刀有関,詳情如何,還請殷姑娘見示。」殷素素道︰「這中間的細微曲折之處,我也不大了然,他日還是親自問你三師哥吧!」張翠山見她不肯説,心知再問也是枉然,暗想︰「傷我三哥之人,其意在於屠龍寶刀。常壇主説要在王盤山揚刀立威,似乎屠龍刀是在他們手中,那些惡賊倘若得訊,定會趕去。」説道︰「發射這三枚梅花小鏢的道士,你説會不會也上王盤上去呢?」殷素素抿嘴一笑,却不答他的問話,説道︰「你定要去趕這份熱鬧,咱們便一塊児去吧!」她轉面對常金鵬道︰「常壇主,請你的船在前引路。」常金鵬應道︰「是!」彎著腰退出船艙,便似僕役厮養對主人一般恭謹。殷素素只點了點頭,張翠山却敬重他這份武功修爲,站起身來,送到艙口。

殷素素向後梢招了招手,喝道︰「過來!」後梢的舟子知道自己亂呼亂叫,闖出了禍,嚇得臉上没半分血色,身子發顫,説道︰「小\dash{}小人是無心之過,姑娘\dash{}姑娘饒命!」他見殷素素不動聲色,更是害怕,轉頭向著張翠山,目光中露出哀求之色,似乎要懇他代爲求情。張翠山心想這舟子誤會自己侵犯殷素素,呼喚常金鵬來救,原是一片忠心,何必害怕成這個樣子,只聽殷素素道︰「你有眼無珠、不生耳朶,要眼睛耳朶何用?」那舟子臉露喜色,知道殷素素説了這兩句話,已是饒了自己的性命,當下屈膝説道︰「多謝姑娘恩典!」刷的一下從裡腿抽出一柄匕首,在自己雙頰旁一揮,登時割下了兩隻耳朶,翻過匕首,便往自己左眼中刺落。

張翠山大吃一驚,探手長臂,其快如風,夾手將他的匕首搶了過來,説道︰「殷姑娘,我斗膽説一個情!」殷素素幽幽的道︰「好吧,你怎麼説便怎麼著。」向那舟子道︰「還不謝過張五俠!」那舟子保全了一對眼睛,早忘了耳上疼痛,跪在船板上向著張翠山{\upstsl{咚}}{\upstsl{咚}}{\upstsl{咚}}的連磕幾個響頭,又向殷素素磕頭,退到了後梢。只聽他精神十足的{\upstsl{吆}}喝水手,昇帆轉舵,竟似死裡逃生,遇到天大的喜事一般。

張翠山側頭瞧著殷素素,心想︰「這位姑娘貌美如花,行事却恁地兇狠,她手下人對她這般畏懼,想見她平素之暴虐。我闖蕩江湖,狠毒之輩也見了不少,却没遇到過這般厲害辣手的人物。」殷素素見他側著身子,默然不語,望了望他長袍背心上被常金鵬一掌擊破之處,説道︰「你除下長袍,我給你補一補。」張翠山道︰「不用了!」殷素素道︰「你嫌我手工粗劣嗎?」張翠山道︰「不敢。」説了這兩個字,又默不作聲,想起她一晩之間連殺龍門鏢局數十口老小,這等大奸大惡的兇手,自己原該出手誅却,可是這時非但和她同舟而行,還助她起鏢療毒,雖説是要酬謝她護送師兄之德,但總嫌善惡不明,王盤山島上的事務一了,須得速即和她分手,再也不願和她相見了。

殷素素見他臉色難看,已猜中他的心意,冷冷的道︰「不但都大錦和祝史兩鏢頭,不但龍門鏢局滿門和那兩個少林僧,還有慧風,也是我殺的。」張翠山道︰「我早疑心是你,只是想不到你用什麼手段。」殷素素道︰「那有什麼希奇?我潛在湖邉水中聽你們説話。那慧風突然發覺咱們兩人相貌不同,想要説出口來,我便發金針從他口中射入。你在路上、樹上、草裡尋我蹤跡,却那裡尋得著?」張翠山道︰「這麼一來,少林派便認定是我下的毒手了,殷姑娘,你當眞好聰明,好手段。」他這幾句話充滿了憤激,殷素素假作不懂,盈盈站起,笑道︰「不敢,張五俠謬讚了!」張翠山怒氣填膺,大聲喝道︰「我姓張的跟你無怨無仇,你何苦這般陥害於我?」

殷素素微笑道︰「我也不是想陥害你,只是少林、武當,號稱武學的兩大宗派,我想要你們兩派鬥上一鬥,且看到底是誰強誰弱?」張翠山聽了這幾句話,心下悚然而驚,滿腔怒火暗自潛息,却大增戒懼之意,心道︰「原來她另有重大奸謀,不只是陥害我一人那麼輕易。倘若我武當派和少林派當眞爲此相鬥,勢必兩敗倶傷,成爲天下武林中的一場浩劫。」

殷素素摺扇輕揮,神色自若,説道︰「張五俠,你扇上的書畫,可否供我開開眼界?」張翠山尚未回答,忽聽得前面常金鵬船上有人朗聲喝道︰「是巨鯨幫的船嗎?那一位在船上?」右首江面上有人叫道︰「巨鯨幫少幫主,到王盤山島上赴會。」常金鵬船上那人叫道︰「殷姑娘和朱雀壇常壇主在此,貴船退在後面吧!」右首船上那人粗聲粗氣的道︰「若是白眉教殷教主駕臨,咱們自當退讓,旁的人,那是不必了。」張翠山聽了「白眉教殷教主」六個字,心中一動︰「白眉教?那是什麼邪教?怎地没聽師父説過,眼見他們這等聲勢,力量可當眞不小啊。想是此教崛起未久,近年來師父在山上清修,少到江南一帶走動,是以不知。」推開船窗向外一望,只見右首那船雕成一頭巨鯨之狀,船頭上白光閃閃,數十柄尖刀鑲成巨鯨的牙齒,船身彎彎,船尾高翹,便似鯨魚的尾巴。這艘巨鯨船帆大船輕,行駛時比常金鵬那艘船快得多。

原來巨鯨幫是蘇浙閩三省沿海的一個海盜幫會,殺人越貨,無惡不作,所乘船隻構造特殊,行駛極快,官軍的海船無法追上,而搶劫商船時却又極爲便利,橫行東海已歷數十年。

常金鵬親自站到船頭,叫道︰「麥少幫主,殷姑娘在這児,你這點小面子也不給嗎?」只見巨鯨船艙中鑽出一個黃衣少年,冷笑道︰「陸地上以你們白眉教爲尊,海面上該算是咱們巨鯨幫了吧?好端端的爲什麼要讓你們先行?」張翠山心想︰「江面這般闊,數十艘船也可並行,何必定要他們讓道,這白眉教也未免太橫。」

這時巨鯨船上,又加了一道風帆,搶得更加快了,兩船越離越遠,再也無法追上。常金鵬「哼」的一聲,説道︰「巨鯨幫\dash{}屠龍刀\dash{}也\dash{}屠龍刀\dash{}」大江之上,風急浪高,兩船相隔又遠,不知他説些什麼。那麥少幫主聽他連説了兩句「屠龍刀」,只道事関重大,命水手側過船身,漸漸和常金鵬的座船靠近,大聲問道︰「常壇主你説什麼?」常金鵬道︰「麥少幫主\dash{}咱們玄武壇白壇主\dash{}那屠龍刀\dash{}」張翠山微覺奇怪︰「怎麼他説話斷斷續續?」眼見那巨鯨船靠得更加近了,猛聽得呼的一聲響,常金鵬提起船頭的巨錨擲了出去,錨上的鐵鍊聲嗆{\upstsl{啷}}{\upstsl{啷}}連響,對面船上的兩個水手長聲慘叫,那隻大鐵錨已鉤在巨鯨船上。

麥少幫主喝道︰「你幹什麼?」常金鵬手脚快極,提起左邉的大鐵錨又擲了出去。兩隻鐵錨擊斃了巨鯨幫船上三名水手,同時兩艘船也已連在一起。那麥少幫主搶到船邉,伸手去拔鐵錨,常金鵬也不理他,右手一揮,一個碧綠的大西瓜飛了出去,砰的一聲猛響,打在巨鯨船的主桅之上。原來這大西瓜乃是常金鵬所用的兵器,精鋼鑄成,瓜上漆成綠黑間條之色,共有一對,繫之金鏈,使動時和流星鎚一般無異,只是兩個西瓜特大特重,左手的九十五斤,右手的一百零五斤,若非雙臂有千斤之力,如何使他得動?

右手的鐵西瓜擊出,巨鯨船的主桅喀啦啦響了兩聲,從中斷爲兩截。巨鯨船上衆海盜紛紛驚叫呼喝,常金鵬雙瓜齊飛,同時擊在後桅之上,後桅較細,一擊便斷。

那麥少幫主實在殊非庸手,只是他平素慣使分水蛾眉刺,那是一種尺許來長的兵器,於水底交鋒之際,轉折迴旋極是利便,這時兩船相隔數丈,眼睜睜的瞧著兩根桅桿一一擊斷,竟是無法可施,只有高聲怒罵。常金鵬雙瓜倏地收回,喝道︰「有白眉教在此,水面上也不能任你巨鯨幫稱雄!」但見右臂揚處,鐵瓜又是呼的一聲飛出,這一次却擊在巨鯨船的船舷之上,砰的一聲,船旁登時破了一個大洞,海水湧入,船上衆水手大聲叫起來。

麥少幫主抽出蛾眉刺,雙足一點,縱身躍起,便往常金鵬的船頭撲來,常金鵬待他躍到最高之時,左手鐵瓜飛出,逕朝他迎面擊去,這一招甚是毒辣,鐵瓜到時,正是他人在半空,一躍之力將衰未衰。麥少幫主叫聲︰「啊喲!」伸蛾眉雙刺在鐵瓜上一擋,便欲借力翻回。若是換作了張翠山,他輕功了得,只須施展「梯雲縱」絶技,不但能避開鐵瓜,還能就勢進擊,但麥少幫主的輕功雖然也不算弱,總是不能和武當子弟相提並論,那鐵瓜本身已重達百斤,再加上常金鵬一送之力,麥少幫主但覺胸口氣塞,眼前一黑,翻身跌回船中。常金鵬雙瓜此起彼落,霎時之間在巨鯨船上擊了七八個大洞,跟著提起錨鍊,運勁回拉。喀喇喇幾聲響,巨鯨船船板碎裂,兩隻鐵錨拉回了船頭。白眉教船上衆水手不待壇主吩咐,揚帆轉舵,向前直駛。

張翠山在窗後見了常金鵬擊破敵船的這等威猛聲勢,不禁暗自心驚︰「我若非得恩師傳授,學會了這借力卸力之法,他那巨靈般的一掌擊在我背心,如何經受得起?這人瞬間誘敵破敵,不但武功驚人,而且陰險毒辣,十分的工於心計,可説是邪教中一個極厲害的人物。」回眼看殷素素時,只見她神色自若,似乎這種事司空見慣,絲毫没放在心上。

只聽得雷聲隱隱,錢塘江中夜潮將至。巨鯨幫的幫衆雖然人人精通水性,但遇到波濤山立的怒潮,却也是經受不起,何況這時已在江海相接之處,江面闊達數十里,距離南北兩岸均甚遙遠。幫衆一聽到潮聲,忍不住大叫呼救,常金鵬和殷素素的兩艘座船向東疾駛,毫不理會。張翠山探頭到窗外一望,只見那艘巨鯨船已沉没了一小半,待得潮水一衝,登時便要粉身碎骨。張翠山聽得幫衆慘叫呼救之聲,心下甚是不忍,但知殷素素和常金鵬都是心狠手辣之輩,若要他們停船相救,徒然自討没趣,只得默然不語。殷素素瞧了他神色,微微一笑,忽然縱聲叫道︰「常壇主,咱們的貴客張五俠大發慈悲,你把巨鯨幫船中那些傢伙救起來吧!」這一著大出張翠山的意外,只聽得前面船上常金鵬應道︰「謹遵貴客之命!」船身側過,斜搶著向上游駛去。常金鵬大聲叫道︰「巨鯨幫的幫衆們聽著,武當派張五俠救你們性命,要命的快游上來吧!」諸幫衆順流游下,常金鵬的座船逆流迎上,搶在潮水的頭裡,將巨鯨船上自麥少幫主以下,救起了十之八九,但終於有六七名水手已葬身在波濤之中。張翠山道︰「多謝你啦!」殷素素冷冷的道︰「巨鯨幫殺人越貨,那船中没一個人的手上不是染滿了血腥,你救他們幹麼?」張翠山茫然若失,一時答不出話來。要知巨鯨幫惡名素著,是水面上四大惡幫之一,他早聞其名,却不道今日反予相救。只聽殷素素道︰「若不將們救上船來,張五俠心中更要罵我啦;『哼!這年輕姑娘心腸狠毒,甚於蛇蝎,我張翠山悔不該助她起鏢療毒!』」這句話正好説中了張翠山的心事,他臉上一紅,只得笑道︰「你伶牙俐齒,我那裡説得過你?救了那些人,是你自己積的功德,可不跟我相干。」

\chapter{揚刀立威}

就在此時,潮聲如雷,震耳欲聾,張翠山和殷素素所坐的船被抛了起來,説話聲盡皆掩没。張翠山向窗外一看,只見巨浪猶如一堵透明的高牆,巨鯨幫的人若不獲救上船,這時都被掩没在驚濤之中了。殷素素走到後艙,関上了門,過了片刻出來,却又換上了女裝,她打個手勢,要張翠山除下長袍。張翠山不便再行峻拒,只得解了下來。他只道殷素素要替自己縫補破裂之處,那知她提起自己剛換下來的男裝長袍,打手勢叫張翠山穿上,却將他的破袍收入了後艙。

張翠山身上只有短衫中衣,只得將殷素素的男裝長袍穿上了。那件袍子本就寬大。張翠山雖然比她高大得大,却也不顯得窄小,只聞到袍子上一縷縷淡淡的幽香,送入鼻端。張翠山心神一蕩,不敢向她觀看,恭恭敬敬的坐著,裝作欣賞艙艙板壁上的書畫,但心事如潮,和船外船底的波濤一般洶湧起伏,却那裡看得進去?殷素素也不來跟他説話。船中本來點著臘燭,但一個巨浪湧來,船身一側,燭火登時熄了。張翠山暗道︰「不好!我二人孤男寡女,坐在船艙之中,雖説我不欺暗室,却只怕於殷姑娘的清名有累。」於是推開後艙艙門,走到把舵的舟子身旁,瞧著他穩穩的掌著舵柄,穿波越浪,順流下駛。

一個多時辰之後,上湧的潮水反退出海,順風順水,舟行更速,破曉後已近王盤山島。那王盤山在錢塘江的東海之中,是一個荒涼小島,山石嶙峋,向無人居。兩艘船駛近島南,相距尚有數里,只聽得島上號角之聲嗚嗚吹起,兩個人各舉一面大黑旗、揮舞示意。座船漸漸駛近,張翠山見兩面黑旗上鑲以白邉,心道︰「黑旗白邉,乃是金生水之意。常壇主説玄武壇壇主在島上主持揚刀立威,北方玄武,壬癸亥子水,主黑。看來這白眉教中的人物精通五行變化之術,並非尋常愚民的邪教。」沉吟間座船駛得更加近了,只見黑旗上繡著一隻飛龜之形。

兩面大黑旗之間站著一個老者,他朗聲説道︰「玄武壇白龜壽恭迎殷姑娘。」聲音漫長,綿綿密密,雖不響亮,却是氣韻醇厚之極。片刻間坐船靠岸,那老者親自舖上跳板。殷素素請張翠山先行,上岸後和白龜壽引見。白龜壽見殷素素神氣間對張翠山極爲重視,待聽到他是武當七俠中的張五俠,更是心中一凜,説道︰「久仰武當七俠的清名,今日幸得識荊,大是榮幸。」張翠山謙遜了幾句。殷素素笑道︰「你兩個言不由衷,説話不大痛快。一個是心中在想︰『啊喲,不好,武當派的人也來啦,多了一個爭奪屠龍刀的辣手人物。』另一個心中却説︰『你這種邪教邪派的人物,我纔犯不著跟你親近結交。』我説啊,你們想説什麼便説什麼,不用口是心非的。」白龜壽哈哈一笑,張翠山却道︰「不敢!白壇主武功精湛,在下一聽白壇主這份隔海傳聲的功夫,心下好生佩服。在下只是陪殷姑娘來瞧瞧熱鬧,絶無覬覦寶刀之心。」

殷素素聽他這般説,面溢春花,好生喜歡。白龜壽素知殷素素面冷心狠,從來不對任何人稍假詞色,但這時對張翠山的神態却截然不同,知道這人在她心中的份量實是不輕,又聽得他稱讚自己内功,當下敵意盡消,説道︰「殷姑娘,海沙派、巨鯨幫、神拳門的人物早就到啦,還有兩個崑崙派的年青劍客。這兩個小子飛揚跋扈,囂張得緊。那如張五俠名揚天下,却這麼謙光。可見有一分本事,便有一分修養\dash{}」他剛説到這裡,忽聽得山背後一人喝道︰「背後鬼鬼祟祟的毀謗旁人,又算是什麼大丈夫的行逕?」話聲一歇,便轉出兩個人來。兩人身材修長,一色的杏黃長袍,背上斜插長劍,都是二十八九歳年紀。

兩人臉罩寒霜,一副要惹事生非的模樣。白龜壽笑道︰「説起曹操,曹操便到,來來來,我跟你們引見引見。」那兩個崑崙派的青年劍客本來就要發作,但斗然間見到殷素素容光照人,艷麗非凡,不由得心中都是怦然一動。一個人竟是目不轉瞬的呆瞧著她,另一個看了她一眼,急忙轉開了頭,但隨即又偸偸斜目看她。白龜壽指著呆看殷素素的那人道︰「這位是高則成高大劍客。」指著另一人道︰「這位是蔣濤蔣大劍客。兩位都是崑崙派的武學高手。想崑崙派威震西域,武學上有不傳之祕,天下武林,無不欽佩,高蔣兩位更是崑崙派中出乎其類、拔乎其萃,矯矯不群的人物。這一次來到中原,定當大顯身手,讓咱們開一開眼界。」他這番話中顯是頗含譏嘲,張翠山心想這二人若不立即動武,也必反唇相稽,那知高蔣二人只是唯唯否否,似乎没聽見他説些什麼。張翠山好生奇怪,再一看二人的神色,這纔醒悟,原來他二人一見殷素素,一個傻瞪,一個偸瞧,竟是神不守舍的如痴如呆。

張翠山暗暗好笑,心道︰「崑崙派名播天下,號稱是劍術通神,那知出來的弟子却這般下流。」其實高蔣二人雖然生性傲慢了些,却非下流好色之徒,只是殷素素實在容貌太美,教人的眼光一和她面容接觸,猶如磁石引鐵一般,竟然再也難以分開。何況高蔣二人都是青年子弟,喜愛美色亦是人情之常。他二人這般貪看,未必心中存了什麼猥褻之念,只是情不自禁,難以自持。

白龜壽又道︰「這位是武當派張翠山相公,這位是殷素素姑娘,這位是敝教的常金鵬壇主。」他説這三人姓名時都是輕描淡冩,不加形容,對張翠山更是只稱他一聲「相公」,連「張五俠」的字眼也免了,那顯是將他當作極親近的自己人看待。殷素素心中甚喜,眼光在張翠山臉上一轉,秋波流動,含情脈脈。

高則成性較鹵莽,見殷素素對張翠山神態親近,兩人関係顯是不同尋常,也不知從那裡來的一叢怒火,竟是在胸頭燃燒起來,狠狠的向張翠山怒目橫了一眼,冷冷的道︰「蔣師弟,咱們在西域之時,好像聽説過,武當派算是武林中的名門正派啊。」蔣濤道︰「不錯,好像是聽説過。」高則成道︰「原來耳聞不如目見,道聽塗説之言,大不可信。」蔣濤道︰「是嗎?江湖上謠言甚多,十之八九原本靠不住。高師哥説武當派怎麼了?」高則成道︰「名門正派的弟子,怎地和邪教的人物厮混在一起,這不是自甘墮落麼?」他二人一吹一唱,竟指名道姓的向張翠山叫起陣來。他們可不知殷素素也是白眉教中人物,「邪教」二字,是指白常二人而言。

張翠山聽他二人言語如此無禮,登時便要發作,但轉念一想,自己這次上王盤山來,用意純是在査察傷害兪岱岩的兇手,這兩個崑崙弟子年紀雖較自己爲大,却是初出茅蘆的無名之輩,犯不著跟他們一般見識,何況白眉教行事確甚邪惡,觀乎殷素素和常金鵬將殺人當作家常便飯一事可知,自己決不能跟他們牽纏在一起,於是微微一笑,説道︰「在下跟白眉教的這幾位也是初識,和兩位仁兄没什麼分别。」

這兩句話衆人聽了都是大出意外,白常兩壇主只道殷素素跟他交情甚深,原來却是初識,殷素素心中惱怒,知道張翠山如此説,明是瞧不起白眉教之意,高蔣兩人相視冷笑,心想︰「這小子是個膿包,一聽到崑崙派的名頭,心裡就怕了咱們啦。」

白龜壽道︰「各位賓客都已到齊,只有巨鯨幫的麥少幫主,還没有來,咱們也不等他啦。現下各位到處隨便逛逛,正午之時請到那邉山谷中飲酒看刀。」

常金鵬笑道︰「麥少幫主座船失事,是張相公命人救了起來,這時便在船中,待會請他赴宴便了。」張翠山雖見白常兩位壇主對已執禮甚恭,殷素素的眼光神色之間更是柔情似水,但想跟這些人越是疏遠越好,於是説道︰「小弟想獨自走走,各位請便。」也不待各人回答,一舉手,便向東邉一帶樹林走去。

這王盤山是個極小的島嶼,島上除了山石樹木,並無可觀之處。東南角一個小小港灣,桅檣高聳,停舶著十來艘大船,想是巨鯨幫、海沙派一干人的座船。張翠山沿著海邉信步而行,他對殷素素任意殺人的殘暴行逕雖然大是不滿,但説也奇怪,一顆心竟是念茲在茲的縈繞在她的身上,心想︰「這位殷姑娘在白眉教中地位極是尊貴,白常兩位壇主對她像公主一般侍候,但她顯然不是教主,不知是什麼來頭?」又想︰「白眉教要在這島上揚刀立威,對方海沙派、神拳門、巨鯨幫等都是由最重要的人物赴會,白眉教却只派一位壇主主持,似乎没將這些對手放在心上。瞧那玄武壇白壇主的氣派,似乎功力尚在朱雀壇常壇主之上。看來白眉教將是武林中一個極大的隱憂,今日當多摸一下他們的底細,日後咱們武當七俠只怕要跟他們拚個你死我活。」正沉吟間,忽聽得樹林外叮叮{\upstsl{噹}}{\upstsl{噹}},傳來一陣陣兵刃相交之聲。

張翠山好奇心起,循聲過去,只見兩株大樹之間,崑崙派的兩個劍客高則成和蔣濤各執長劍,正在練劍,殷素素在一旁笑吟吟的瞧著。張翠山心道︰「師父平素説崑崙派的劍術大有獨到之處,他老人家少年之時,還和一個號稱『劍聖』的崑崙派名家交過手,這機緣倒是難得。」但武林之中,一派的師徒或師兄弟練習武功,極忌旁人偸看。張翠山是名門弟子,不願貽人口實,雖然極想看個究竟,但終是守著武林規矩,只望了一眼,轉身便欲退開。

那知他這麼一探頭,殷素素已看見了他,伸出纖纖素手,向他招了招,叫道︰「張五哥,你過來。」張翠山這時若再避開,反落了個偸看的嫌疑,於是邁步走近,説道︰「兩位兄台在此練劍,咱們别惹人厭,到那邉走走吧。」還没聽殷素素回答,却見白光一閃,嗤的一響,蔣濤反劍掠上,高則成左臂中劍,鮮血冒出。張翠山吃了一驚,只道是蔣濤失手誤傷。那知高則成哼也不哼一聲,鐵青著臉,刷刷刷三劍,招數巧妙狠辣,全是指向蔣濤的要害。張翠山這纔看清,原來兩人並不是練習劍法,竟是眞打眞鬥,不禁大是訝異。殷素素笑道︰「看來師哥不及師弟,還是蔣兄的劍法精妙些。」

高則成聽了此言,一咬牙,翻身迴劍,劍訣斜引,一招「百丈飛瀑」,劍鋒從半空中直瀉下來。張翠山忍不住喝采︰「好劍法!」蔣濤縮身一躱,但高則成的劍勢不到用老,中途變招,劍尖一抖,「嘿!」的一聲呼喝,刺入了蔣濤左腿。殷素素拍手道︰「原來做師兄的畢竟也有兩手,蔣兄這一下可比下去啦。」蔣濤怒道︰「也未見得。」劍招忽變,歪歪斜斜,使出崑崙派中的一套「雨打飛花」劍法來。這一路劍全是走的斜勢,飄逸無倫,但七八招斜勢之中,偶爾又挾著一招正勢,教人極難捉摸。高則成對這路本門劍法自是爛熟於胸,見招拆招,也毫不客氣的還以擊削劈刺。兩人身上都已受傷,雖然中的均非要害,但劇鬥中鮮血飛濺兩人臉上、袍上、手上都是血點斑斑。師兄弟倆越鬥越狠,到後來意似性命相撲一般。殷素素却在旁不住口的推波助瀾,讚幾句高則成,又讚幾句蔣濤,把兩人激得興發如狂,恨不得一劍將對方刺倒,好討得殷素素的歡喜,顯得自己劍法多強。

這時張翠山早已明白,他師兄倆忽然捨命惡鬥,全是殷素素從中挑撥,而她所以要挑動兩人相鬥,當是因他們瞧不起白眉教而致。眼見兩人越打越狠,初時還不過意欲取勝,到後來各人動了狂興,竟是要致對方死命一般,再鬥下去勢非闖出大禍不可。看這二人的劍法果是極爲精妙,只是變化不彀靈動,内力也嫌薄弱,劍法中的威力只發揮得出一二成而已。殷素素拍手嬉笑,甚是高興,説道︰「張五哥,你瞧崑崙派的劍法怎樣?」她聽張翠山不答,一回頭,見他眉頭微皺,頗有厭惡之色,説道︰「使來使去這幾路,没什麼看頭,咱們到那邉瞧瞧海景去吧!」説著拉了張翠山的左手,舉步便行。

張翠山只覺一隻溫膩軟滑的手掌握住了自己的手,心中一動,明知她是有意激怒高蔣二人,却也不便掙脱,只得隨著她走向海邉。瞧著一望無際的大海,殷素素呆呆出了一會神,忽道︰「『莊子』秋水篇中説道︰『天下之水,莫大於海,萬川歸之,不知何時止而不盈。』然而大海却並不驕傲,只説︰『吾在於天地之間,猶小石小木之在大山也。』莊子眞是了不起,有這麼博大的胸襟。」

張翠山見她挑動高蔣二人自相殘殺,引以爲樂,心中本來甚是不滿,忽然聽到這幾句話,不禁一怔。「莊子」一書,道家修眞之士是一定要讀的,張翠山在武當時,張三丰也常拿來和他們師兄弟講解。但這個殺人不貶眼的女魔頭突然在這當児發此感慨,實在大出他意料之外,他一怔之下,説道︰「是啊,『夫千里之遠,不足以舉其大,千仞之高,不足以極其深。』」殷素素聽他也以「莊子秋水篇」中形容大海的話來回答,但臉上神氣,却有不勝仰慕欽敬之情,説道︰「你是想起了師父嗎?」張翠山吃了一驚,情不自禁的伸出右手,握住了她另外一隻手,道︰「你怎麼知道?」原來當年他在山上和大師兄宋遠橋、三師兄兪岱岩共讀莊子,讀到「夫千里之遠,不足以舉其大,千仞之高,不足以極其深」這兩句話時,兪岱岩説道︰「咱們跟師父學藝,越學越覺得跟他老人家相差得遠了,倒似每天都在退步一般。用『莊子』上這兩句話來形容他老人家深不可測、大不所窮的功夫,那纔適當。」宋遠橋和張翠山都點頭稱是。這時他想起莊子這兩句話,自然而然的想起了師父。

殷素素道︰「你臉上的神情,不是心中想起父母,便想起極敬重的師長,但『千里之遠,不足以舉其大』云云,當世除了張三丰道長,只怕也没第二個人當得起了。」張翠山甚喜,道︰「你眞是聰明。」驚覺自己忘形之下握住了她的雙手,臉上一紅,緩緩放開。殷素素道︰「尊師的武功,到底是怎般的出神入化,你能説些給我聽聽麼?」張翠山沉吟半晌,道︰「武功只是小道,他老人家所學遠不止於武功,唉,博大精深,不知從何説起。」殷素素微笑道︰「『夫子步亦步,夫子趨亦趨,夫子馳亦馳;夫子奔逸絶塵,而回膛若乎後矣。』」張翠山聽她引用「莊子」書中顏回稱讚孔子的話,而自己心中,對師父確是有這種五體投地的感覺,説道︰「我師父不用奔逸縱塵,他老人家趨一趨,馳一馳,我就跟不上啦。」

殷素素聰明伶俐,有意要討好他,自是談得十分投機,久而忘倦。兩人並肩坐在石上,不知時光很快的過去,忽聽得遠處脚步聲極是沉重,有人咳了幾聲,説道︰「張相公、殷姑娘,午時已到,請去入席吧。」張翠山回過頭來,只見常金鵬相隔十餘丈站著,雖然神色莊敬,但嘴角邉帶著一絲微笑。

他神情之中,便似一個慈祥的長者見到一對珠聯璧合的小情人,大感讚嘆歡喜。殷素素一直對他視作下人,傲不爲禮,這時却臉含羞澀,低下頭去。張翠山心中光明磊落,但見了兩人神色,禁不住臉上一紅。常金鵬極是識趣,轉過身來,當先領路。殷素素低聲道︰「我先去,你别跟著我一起。」張翠山微微一怔,心道︰「這位姑娘怎地避起嫌疑來啦?」便點了點頭。殷素素搶上幾步,和常金鵬並肩而行,只聽她笑著問道︰「那兩個崑崙派的獃子打得怎樣啦?」張翠山心中似喜非喜,似愁非愁,直瞧著他二人的背影在樹後隱没,這纔緩緩向山谷中走去。

進得谷口,只見一片青草地上擺著七八張方桌,除了東首第一席外,每張桌旁都已坐了人。常金鵬見他走近,站起身來,大聲道︰「武當派張五俠駕到!」這八個字説得聲若雷震,山谷鳴響。他一説完,和白龜壽快步迎了出來,每人身後跟隨著本壇的五位香主,十二人在谷口一站,並列兩旁,躬身相迎。白龜壽道︰「白眉教殷教主屬下,玄武壇白龜壽、朱雀壇常金鵬,恭迎張五俠大駕。」殷素素並不走到谷口相迎,却也起立避席。

張翠山聽到「殷教主」三字,心頭一震,暗想︰「那教主果然姓殷!」當下作揖説道︰「不敢當,不敢當!」舉步走進谷中,只見各席上坐的衆人均有憤憤不平之色,心下微感不解,却也不去理會。原來海沙派、巨鯨幫、神拳門各路首領到來之時,白眉教只派壇下的一名香主引導入座,決不似對張翠山這般恭敬有禮,相形之下,顯是意含輕視。這一節張翠山並不知道。

白龜壽引著他走到東首第一席上,肅請入座。這一張桌旁只擺著一張椅子,乃是各桌之中最尊貴的首席。張翠山一瞥眼,見其餘各席大都坐了七八人,只第六席上坐著高則成和蔣濤二人。他朗聲辭道︰「在下末學後進,不敢居此首席。請白兄移到下座去吧。」白龜壽道︰「武當派乃方今武林中的泰山北斗,張五俠威震天下,若不坐此首席,在座的無人敢坐。」張翠山記著師父平時常説的「寧靜謙仰」之訓,心想︰「若是師父或大師哥在此,這首座自可坐得,我却是不配。」堅意辭讓。

高則成和蔣濤使個眼色,蔣濤忽地提起自己的座椅,凌空擲了過來。他這席和首席之間隔開五張桌子,但他這一擲勁力甚強,只聽呼的一聲響,那椅子飛越五張桌旁各人的頭頂,在第一席邉落了下來,端端正正擺好,與原有的一張椅子相距尺許,這一手巧勁,確是有獨到的造詣。蔣濤一擲出椅子,高則成便大聲説道︰「嘿嘿,泰山北斗,不知是誰封的泰山北斗?姓張的不敢坐,咱師兄弟還不致於這般膿包。」兩人身法如風,搶到椅旁。

原來先前殷素素問他二人到底誰的武功高些,説想學幾招崑崙派的劍法,準擬向劍法高明些的人求教。二人見到殷素素容顏嬌麗絶倫,早已迷迷糊糊,聽她求懇試練幾式,當下毫不退辭的便拔劍餵招。初時不過想勝過對方,但越打越狠,收不住手,殷素素又在旁推波助瀾,大加挑撥,兩人竟致一齊受傷。待見她和張翠山神情親密的走開,才知道上了她的當,兩人收劍裹傷,心中又羞憤,又是妬忌,却又不敢向殷素素發作,這時乘機搶奪張翠山的席位,想激他出手,在群雄面前狠狠的折辱他一番。

常金鵬伸手攔住,説道︰「且慢!」高則成伸指作勢,欲往常金鵬臂彎中點去,張翠山却道︰「兩位坐此一席,最是合適不過。小弟便坐那邉吧!」説著舉步往第六席走去。殷素素忽然伸手招了招,道︰「張五哥,到這裡來。」

張翠山不知她有什麼話説,便走近身去。殷素素隨手拉過一張椅子,放在自己身邉,微笑道︰「你坐這裡吧。」張翠山萬料不她竟會如此脱略形跡,在群豪注目之下,頗覺躊躇,若是跟她並肩同席,未免過於親密,倘不依言就坐,又令人面上無光,簡直要使她無地自容。殷素素低聲道︰「我還有話跟你説呢!」張翠山見她臉上露出求懇之色,不忍推辭,便在椅上坐了下去。殷素素心花怒放,笑吟吟的給他斟了杯酒。

這邉高則成和蔣濤雖然搶到了首席,但見了這等情景,只有惱怒愈增。白龜壽揮動衣袖,在椅子上拂了幾拂,掃去灰塵,笑道︰「崑崙派的兩位大劍客要坐個首席,那也不錯啊,請坐請坐!」説著和常金鵬及十名香主各自回歸主人席位就座。高則成和蔣濤心中均想︰「這膿包不敢坐此首席,武當派的威風顯是被崑崙派壓了下去。」兩人對望一眼,大剌剌的坐下。

只聽得喀喇、喀喇兩聲,椅脚斷折,兩人一齊向後摔跌。總算兩人武功不弱,不待背心著地,伸手在地下一撐,已自躍起,但饒是如此,神情已是異常狼狽,各席上的豪客都哈哈大笑起來。高則成心知是白龜壽適才用衣袖拂椅,暗中作下了手脚,暗想這份陰勁實是厲害,自己還没有這份功力。他本來十分自負,把白眉教當作是下三濫的旁門左道,絲毫没瞧在眼裡,這纔在王盤山如此飛揚跋扈,這時見到白龜壽衣袖輕拂之下,顯示了如此功力,不由得鋭氣大挫。却聽白龜壽冷冷的道︰「崑崙派的武功,大家都知道是高的,兩位不用尋這兩張椅子的晦氣。説到坐爛椅子這點粗淺功夫,在座的諸君没有一位不會吧?」説著將手一揮,指著坐在末席的十名香主,道︰「你們也練一練吧!」但聽得喀喇喇幾聲響,十張椅子一齊破裂。那十名香主有備而發,坐碎椅子後笑吟吟的站著,神定氣閒,可比高蔣二人狼狽摔倒的情形高明得太多。

在座群豪大都是見多識廣之士,多數瞧出是白龜壽故意作弄他二人,只是這情景確實有趣,大夥児都放聲大笑。笑聲中只見白眉教的兩名香主各抱了一塊巨石,走到第一席之旁,伸足踢去破椅,説道︰「木椅單薄,無力承當兩位貴體,請坐在這石頭上吧!」原來這兩人是白眉教中出名的大力士,武功平平,但身軀粗壯,天生神力,每個人所抱的巨石都有七百來斤,托起巨石便遞給高蔣二人,要他們接住。高蔣二人劍法精妙,但要接住這般巨大的岩石却萬萬不能。須知白眉教以已之長攻敵之短,有心要這崑崙二劍獻醜。高則成皺眉道︰「放下吧!」兩名大力香主齊聲嘿的一聲猛喝,雙雙挺直,將巨石高舉過頂,説道︰「接住吧!」

這麼一來,逼得高蔣二人只有縮身退開,只怕兩個大力士有一個力氣不繼,稍有失閃,那七百斤的大石壓將下來,豈不被他壓得粉身碎骨?他二人心中雖氣,却又不敢出手襲擊這兩個大力士,巨石橫空,誰也不敢靠近去自履險地。

白龜壽朗聲道︰「兩位崑崙劍客不敢坐首席啦,還是請張相公坐吧!」張翠山坐在殷素素之身旁,香澤微聞,心中甜甜的,不禁神魂飄蕩,忽地聽得白龜壽這麼一喝,登時警覺︰「我千萬不能自墮孼障,和這邪教女魔頭有什情緣牽纏。」當即站起身來,走了過去。白龜壽聽常金鵬極口誇讚張翠山本事,他却不曾親眼得見,這時有心要試他一試,向兩個手托巨石的大力香主使個眼色。兩個香主會意,待張翠山走近,齊聲喝道︰「張相公小心,請接住了!」喝聲一停,兩人身子一矮,雙臂下縮,隨即長身展臂,大叫聲中,兩塊巨石一齊向張翠山頭頂壓了下去。

群豪見了這等聲勢,情不自禁的一齊站起身來。白龜壽本意只是要試一試張翠山的武功到底如何,絶無惡意,一來「武當七俠」的名頭在江湖上太響,今日一見,不過是個溫文蘊籍的青年書生,頗有些出於意料之外,二來這位殷素素姑娘向來没把誰瞧在眼裡,但對這位張五哥却是傾心無已,此人居然能引動殷姑娘的芳心,日後與白眉教必有極大的干連。但他一見這兩個神力香主莽莽撞撞的將巨石擲了過去,心下登時好生後悔,暗叫︰「糟糕,糟糕!」心想張翠山是名門子弟,當然不致爲巨石所傷,但縱躍閃避之際,情景也必狼狽,倘若不幸竟爾小小的出了些醜,不但張翠山見怪,殷姑娘更要大爲恚怒。他這人深沉毒辣,心下早已打定主意,若是情勢不妙,立時便要加禍在那兩個香主頭上,寧可將兩個香主斃於掌底,也不能得罪了殷姑娘。

張翠山忽見巨石凌空壓到,也是吃了一驚,如果跳後避開,那和崑崙派的高蔣二人一般無異,未免墮了師門的威望,這時候也不容細想,練武之人到了緊迫関頭,本身蓄積著的功夫自然而然的會發生出來,當下左手使一招「武」字訣中的右鉤,帶動左方壓下來的巨石,右手使一招「刀」字訣中的左撇,帶動右方壓下來的巨石。那兩大塊巨石本身已有七百來斤,再加上凌空一擲之勢,每一塊都有千斤以上的力道。張翠山並不以膂力見長,要他空手去托這兩塊巨石,那是一塊也舉不起的。可是張三丰這一套以書法中化出來的拳招,實有奪造化之功的神奇。要知武當一派的武功,原不求力大,亦不求招快。後世武當名家王宗岳著有太極拳經,論到一般拳術時説道︰「斯技旁門甚多,雖勢有區别,槩不外乎壯欺弱、慢讓快耳。有力打無力,手慢讓手快,是皆先天自然之能,非関學力而有也。」白眉教這兩名香主膂力過人,那是有生倶來的先天自然之能,但張翠山的功夫却是從學力得來的。正如王宗岳拳經中所云︰「察四兩能撥千斤,類非力勝!觀耄耄能禁衆人,快何能爲?」只要力道運用得法,四兩尚可撥動千斤。張翠山使出師門所授最精深的功夫,借著那兩個香主的一擲之勢,帶著兩塊巨石直飛上天。

這兩塊巨石飛擲之力,其實出自兩個香主,只是他以手掌稍加撥動,變了方向。他長袖飛舞,手掌隱在袖中,旁人看來,竟似以衣袖捲起巨石,擲向天空一般。群豪驚慌之下,連喝采也都忘了。只見兩塊巨石一高一低,先後跌落,張翠山輕飄飄的縱身而起,盤膝坐在較高的那塊石上。但聽得騰的一響,地面震動,一塊巨石落了下來,一大半深陥泥中。第二塊跟著落下,平平穩穩的擊在第一塊巨石之上,兩石相碰,火花四濺,只震得每一席碗碟叮叮{\upstsl{噹}}{\upstsl{噹}}的亂響。張翠山不動聲色的坐在石上,笑道︰「兩位香主神力驚人,佩服佩服!」那兩名香主却驚得目瞪口呆,獃獃的站在當地,一句話也説不出來。

片刻之間,山谷中寂靜無聲,隔了片晌,纔暴出轟雷價一聲采來,殷素素向白龜壽瞪了一眼,得意之情見於顏色。白龜壽大喜,知道自己險險做下錯事,幸好張翠山武功驚人,却將這件事變成了自己討好殷姑娘之舉,於是走到首席之旁,斟了一杯酒朗聲説道︰「咱們久聞武當七俠的威名,今日得見張五俠的神功,當眞是佩服得五體投地。小人敬張五俠一杯。」説著一飲而盡。張翠山道︰「不敢!」陪了一杯。

巨鯨幫的一席之上,突然一個黃衫漢子站起身來,大聲道︰「張五俠武功神妙,當在其次,最令人敬服的却是仁心俠骨,可不同那些奸詐陰惡、鬼計多端的小人。在下也敬張五俠一杯。」説著也舉杯喝乾,杯底朝天。

\chapter{金毛獅王}

這人身材高大,但穿了一件短短的長衫甚不稱身,正是昨晩在錢塘江中覆舟落水的巨鯨幫麥少幫主。他這幾句話一來感激張翠山救命之恩,二來却是斥罵常金鵬暗使奸計。張翠山笑吟吟的舉杯道︰「不敢,在下還敬麥少幫主一杯。」説著舉杯喝乾。

他剛放杯坐落,常金鵬背後的五名香主一齊哈哈大笑,指著麥少幫主説道︰「昨晩没喝飽潮水麼?今日還在這裡喝酒?」麥少幫主鐵青著臉,正要反唇相稽,白龜壽站起身來,朗聲説道︰「敝教新近得了一柄寶刀,叫作屠龍刀。有道是︰『武林至尊,寶刀屠龍,號令天下,莫敢不從』!」他説到這裡,頓了一頓,晶亮閃爍的眼光從左至右,掃視全場一周。他身形並不魁梧,但語聲響亮,目光鋭利,威嚴之氣懾人。麥少幫主竟是不敢再發作什麼,自言自語的説了幾句話,坐回椅中。

白龜壽又道︰「敝教殷教主原擬柬請天下各路英雄,大會恒山,展示寶刀,只是此舉籌劃費時,須得假以時日。誠恐天下英雄不知寶刀已爲敝教所得,因此上奉請各位駕臨此間,瞧一瞧寶刀的面目。」説著一揮手,教下八名弟子大聲答應,轉身進了西首的一個大山洞中。

衆人只道這八名弟子去取寶刀,目光都凝望著他們,那知八個人出來時身上都赤了膊,從山洞中抬著一隻大鐵鼎來。鐵鼎燒著熊熊烈火,火燄衝起一丈來高。八個人離得遠遠的,用長桿肩抬而來,{\upstsl{吆}}{\upstsl{吆}}喝喝,將鐵鼎放在廣場之中。衆人被火燄一逼,登時大感炙熱。那八人之後,又有四人,兩個人抬著一個打鐵用的鐵砧,另外兩人手中各舉一個大鐵錘。

白龜壽道︰「常壇主,請你揚刀立威!」常金鵬道︰「遵命!」轉身叫道︰「取刀來!」適纔挺舉巨石的這兩名神力香主走進山洞,回出來時,一人手中橫托一個黃綾包裹,另一人在旁護衛。那香主將包裹交給了常金鵬,兩人站在他的左右兩旁。常金鵬打開包裹,露出一柄單刀。他托在手裡,舉目向衆人一望,刷地拔刀出鞘,説道︰「這一把便是武林至尊的屠龍寶刀,各位請看仔細了!」説著托刀齊頂,爲狀甚是恭敬。

群豪久聞屠龍寶刀之名,但見這刀黑越越的毫不起眼,心下都存了一個疑團︰「怎知此刀是眞是假?」只見常金鵬緩緩的將刀交給了左首的香主,説道︰「試鐵錘!」那香主接過單刀,將刀擱在鐵砧之上,刀口朝天,另一名神力香主提起大鐵錘,便往刀口上擊落。只聽得嗤的一聲輕響,鐵錘的錘頭中分爲二,一半連在錘桿,另一半跌落在地。群豪一驚之下,都站了起來。要知斷金切玉的寶劍利刃,武林中雖然罕見,却也不是絶無僅有,但這柄屠龍刀削鐵錘如切豆腐,連叮{\upstsl{噹}}之聲也聽不到半點,若非神物,那便是其中有弊。神拳門和巨鯨幫中各有一人走到鐵砧之旁,撿起那半塊鐵錘來看時,但見切口處平整光滑、閃閃發光,顯是新削下來的。

那兩名香主提起另一個鐵錘擊在刀上,又是輕輕削裂。這一次群豪忍不住大聲喝采。張翠山心想︰「如此寶刀,當眞是見所未見,聞所未聞。」

常金鵬緩步走到場中,提起寶刀,使一招「獨劈華山」,嗤的一聲輕響,將大鐵砧中分爲二。突然間搶到左首,橫刀一揮,從一株大松樹腰間掠了過去。縱躍奔走,舉刀連揮,接連掠過了一十八棵大樹。群豪但見他連連舞動寶刀,那些大樹却好端端地絶無異狀,正自不解,忽聽得常金鵬一聲長笑,走到第一株大松樹旁,衣袖拂出,擊在松樹腰間,只聽得喀喇喇一聲響,那松樹向外倒去。原來這松樹早已被寶刀齊腰斬斷。

只是那寶刀實在太過鋒利,常金鵬用的力道又極爲均衡,上半截松樹斷了之後,仍穩穩的置在下半截之上,直至遇到外力推動,這纔倒塌。那大松樹一斷,帶起一股烈風,但聽得喀喇、喀喇之聲不絶,其餘的大樹都一棵棵的倒了下來。常金鵬哈哈一笑,手一揮,將那屠龍刀擲進了烈燄沖天的大鐵鼎中。

大樹倒塌之聲尚未斷絶,忽然遠處跟著傳來喀喇、喀喇的聲音,似乎也有人在斬截大樹。白龜壽和常金鵬等都是一愕,循聲望去,只見聳立著的船上桅桿一根根的倒了下去。那些桅桿上都懸有座旗。白眉教、巨鯨幫、海沙派、神拳門各門各派的首腦見自己的座旗紛紛隨著桅桿倒落,均是大爲驚怒,各遣手下前去査問、但聽得砰彭之聲不絶,頃刻之間,衆桅桿或倒或斜,無一得免,似乎停在港灣中的船隻突然遇到風暴還是海怪,一艘艘的破碎沉没。聚在草坪上的群豪斗遭此變,一時説不出話來,初時還疑心是白眉教佈置什麼陰謀,但見白眉教的船隻同時遭劫,看來却又不是,第二批人跟著奔去査問,但那草坪和港灣相距不遠,奔去的十餘人竟是無一回轉。

衆人面面相覷,驚疑不定。白龜壽向本壇的一名香主道︰「你去瞧瞧。」那香主應命而去。白龜壽強作鎭定,笑道︰「想是海中有甚變故,各位也不必在意。就是船隻盡數毀去,難道咱們不能坐木筏回去嗎?來來來,大家乾一杯!」群豪心中{\upstsl{嘀}}咕,可不能在人前示弱,於是一齊舉杯,剛沾到口唇,忽聽得港灣旁一聲大呼,其聲慘厲,劃過空中,似乎有人腰上被人刺了一刀。群豪霍地站起,膽子較小的酒杯落地,{\upstsl{乒}}{\upstsl{乒}}{\upstsl{乓}}{\upstsl{乓}}的連響。本來這些人殺個把人誰都不在意下,只是這叫聲實在太過可怖。白龜壽和常金鵬聽出這慘呼是適纔去査問的那香主所發,一怔之間,只聽得騰騰騰的脚步聲落地甚重,漸奔漸近,跟著一個血人出現在衆人之前,正是那個香主。

他雙手按住臉孔,手指縫中滲出血來,頂門上去了一塊頭皮,自胸口直至小腹、大腿衣衫盡裂,一條極長的傷口也不知多深,血肉糢糊,便似被什麼窮兇極惡的猛獸抓了一把的模樣。白龜壽搶過去伸手欲扶,那香主慘聲叫道︰「金毛獅王,金毛獅王!」白龜壽道︰「我去瞧瞧。」常金鵬道︰「我和你同去。」白龜壽道︰「你保護殷姑娘。」他知那死去的香主武功甚強,在白眉教中算得是個硬手,但一轉眼間被人傷得這般厲害,對手自是非同小可。常金鵬點點頭道︰「是!」

忽聽得有人咳嗽一聲,説道︰「金毛獅王早在這裡!」衆人吃了一驚,四下裡一望,却不見半個人影,這聲音明明是從近處發出,却不知他躱在那裡。只聽那人嘆聲道︰「蠢才,蠢才!」突然間呼的一聲,一塊巨石飛起,一個人從石頭底下鑽了出來。原來他早已隱身在大樹之後,掘地鑽到巨石下面,因之雖在肘腋之間,衆人却無一得知。

衆人這一驚當眞是非同小可,殷素素「啊」的一聲叫,情不自禁的奔到張翠山身旁。只見那人身材魁異常,比常人足足高出一尺,肩膀也要闊出一尺,滿頭黃髮,散披肩頭,眼睛綠油油的發光,手中拿著一個一丈七八尺長的兩頭狼牙棒,在筵前這麼一站,威風凜凜,眞如天神天將一般。張翠山暗自尋思︰「金毛獅王?這渾號自是因他的滿頭黃髮而來了,他是誰啊?可没聽師父説起過。」

再看這金毛獅王時,只見他身穿一件百獸獸皮所縫綴而成的長袍,這長袍上有虎皮、豹皮、野牛皮、鹿皮、熊皮、狐皮,雖然東一塊,西一塊,但手工精細,乃是高手匠人所爲。諸般獸皮之中,就是没有獅皮,想是他自稱「金毛獅王」,對獅子自是極爲尊重了。他手中拿著的那根狼牙棒也是甚爲怪異,棒身自此一端至彼一端,金光閃爍,却又不是黃金之色,尋常的狼牙棒只是一端有釘,但他的狼牙棒不但特長特大,而且兩端有釘。衆人見了他這股神態,誰都不敢説話。

白龜壽鼓著勇氣,上前數步,説道︰「不敢請問尊駕高姓大名?」那人道︰「不敢,在下姓謝,單名一個遜字,表字退思,有一個外號,叫作『金毛獅王』。」張翠山一聽,和殷素素對望了一眼,兩人均想︰「這人如此威猛,取的名字却是這般溫文爾雅,而他的外號倒是適如其人。」白龜壽聽他言語有禮,稍去怯意,説道︰「原來是謝先生。尊駕跟咱們素不相識,何以一至島上,便即毀船殺人?」謝遜微微一笑,露出一口牙齒,白如編貝,閃閃發光,説道︰「各位聚在此處,所爲何來?」

白龜壽心想︰「此事也瞞他不得。這人武功縱然厲害,但他總是單身,我和常壇主聯手,再加上張五俠、殷姑娘從旁相助,定可除他得了。」於是朗聲説道︰「敝教白眉教新近得了一柄寶刀,邀集江湖上的朋友,大夥児在這裡瞧瞧。」謝遜瞪目瞧著大鐵鼎中被烈火鍛燒著的那柄屠龍刀,見那刀在烈燄之中不損分毫,的是神物利器,於是大踏步走將過去。常金鵬見他伸手便去抓那柄刀,叫道︰「住手!」謝遜回頭淡淡一笑,道︰「幹什麼?」常金鵬道︰「此刀是敝教所有,謝朋友但可遠觀,不可碰動。」謝遜道︰「這刀是你們鑄的?是你們買的?」常金鵬啞口無言,一時答不出話來。謝遜道︰「你們從别人手上奪來,我便從你們手上奪去,天公地道,有什麼使不得?」説著轉身又去抓那寶刀。

嗆{\upstsl{啷}}{\upstsl{啷}}一響,常金鵬從腰間解下西瓜流星鎚,喝道︰「謝朋友,你再不住手,我可要無禮了。」他言語中似是警告,其實聲到鎚到,左手的鑌鐵大西瓜向他後心直撞過去。謝遜更不回頭,只是將狼牙棒向後斜垂,{\upstsl{噹}}的一聲巨響,那鑌鐵大西瓜撞正狼牙棒上,登時碎作十七八片,四散飛擲。常金鵬身子一晃,突然間狂噴鮮血,倒地斃命。原來謝遜的内力從狼牙棒傳到他的西瓜流星鎚上,以巨力抗巨力,常金鵬在錢塘江中鎚碎麥少幫主的座船時何等神威,這時却禁不起他狼牙棒的一撞。

朱雀壇屬下的五名香主大驚,一齊搶了過去,兩名香主去扶常金鵬,三名香主拔出兵刃,不顧厲害的向謝遜攻去。謝遜左手抓著屠龍刀,右手中的狼牙棒在鐵鼎下一挑,一隻數百斤重的大鐵鼎飛了起來,橫掃而至,將三名香主同時壓倒。大鐵鼎餘勢未衰,在地下打了個滾,又將扶著常金鵬的兩名香主撞翻。五名香主和常金鵬屍身身上衣服一齊著火,其中四名香主已被鐵鼎撞死,餘下的一名在地下哀號翻滾。

衆人見了這等聲勢,無不心驚肉跳,但見他一舉手之間,連斃五名江湖上的好手,餘下那名香主看來也是重傷難活。張翠山年紀雖然不大,但行走江湖,會見過的高手也已不少,可是如謝遜這般超人的神力武功,却是從未見過,暗忖自己決不是他的敵手,便是大師哥、二師哥,也遠遠不如,即是武當七俠聯手應敵,恐怕也難操勝算。當今之世,除非是師父下山,否則不知還有誰能勝得過他。只見謝遜提起屠龍刀,伸指一彈,發出非金非木的沉鬱之聲,他點點頭讚道︰「無聲無色,神物自晦,好刀啊好刀!」

他抬起頭來,向白龜壽身旁的刀鞘望了一眼,説道︰「這是屠龍刀的刀鞘吧?拿過來。」白龜壽心知當此情形之下,自己的性命十成中已去了九成,若是將刀鞘給他,不但一世英名化於流水,而且日後教主追究罪責,定是死得極爲慘酷,但此刻和他硬抗,那也是有死無生,於是凜然説道︰「你要便殺,我姓白的豈是貪生畏死、欺善怕惡的小人?」

謝遜微微一笑,道︰「硬漢子,硬漢子!白眉教中果然還有幾個人物。」突然間右手一揚,那柄一百多斤的屠龍刀猛地向白龜壽飛去。白龜壽早在提防,一見他寶刀出手,知道此人的手勁大得異乎尋常,不敢用兵器擋格,更不敢伸手去接,急忙閃身避讓。那知這寶刀斜飛而至,刷的一聲,套入了平放在桌上的刀鞘之中,這一擲力道甚是強勁,帶動刀鞘,繼續激飛出去。謝遜伸出狼牙棒,一搭一勾,將屠龍刀連刀帶鞘,引了過來,隨手插在腰間。這一下擲刀取鞘,準頭之巧,手法之奇,實是到了匪夷所思的地步。

他眼光自左至右,向群豪瞧了一遍,説道︰「在下要取這柄屠龍寶刀,各位有何異議?」他連問兩聲,誰都不敢答話。忽然海沙派席上一人站起身來,説道︰「謝前輩德高望重,名揚四海,此刀正該歸謝前輩所有,咱們大夥児非常之贊成。」謝遜道︰「閣下是海沙派的總舵主元廣波罷?」那人道︰「正是。」他聽見謝遜知道自己姓名,既是歡喜,又是惶恐。謝遜道︰「你知道我師父是誰?是何門派?我做過什麼好事?」元廣波囁嚅著道︰「這個\dash{}謝前輩\dash{}」他實是一點也不知道。謝遜冷冷的道︰「我的事你什麼也不知,怎説我德高望重,名揚四海?這把刀本來是你海沙派得到的,後來給長白三擒奪了去,又落入武當派兪岱岩手中\dash{}」張翠山聽到「又落入武當派兪岱岩手中」這句話,心口發熱,暗想︰「這姓謝的説話想來不假,原來此刀果是與三哥大有干係。」

只聽謝遜續道︰「白眉教暗下毒手,從兪岱岩手裡奪來。哼哼,你海沙派反正已得不到手,便説此刀歸我所有,大夥児都非常贊成。你這人諂媚趨奉,滿口胡言,我生平最瞧不起的,便是你這種無恥小人。給我站出來!」最後這幾句話每一個字便似打一個轟雷。元廣波爲他威勢所懾,竟是不敢違抗,低著頭走到他的面前,身子不由自主的不停打戰。

在張翠山心中,滾來滾去的却只是這幾句話︰「白眉教暗下毒手,暗下毒手,從兪岱岩手裡奪來,暗下毒手\dash{}」斜目看殷素素時,只見她臉色蒼白,睫毛微微顫動,想是心中也思潮起伏不定。

謝遜道︰「你海沙派武藝平常,專門靠毒鹽害人。去年在餘姚害死張登雲一家十一口,本月初歐陽清在海門身死,都是你做的好事吧?」元廣波大吃一驚,心想這兩件案子做得異常隱祕,怎地會給他知道了?謝遜喝道︰「叫你手下人裝兩大碗毒鹽出來,給我瞧瞧,到底是怎麼樣的東西。」海沙派的幫衆人人都擕帶毒鹽,元廣波不敢違拗,只得命手下人裝了兩大碗出來。謝遜取了一碗,湊到口邉,聞了幾下,忽然側碗往口中便倒,連吞了幾大口,説道︰「咱們每個人都吃一碗。」

元廣波又驚又喜,驚的是他竟要自己服食毒鹽,喜的却是他竟悍然自吞,這毒鹽沾在身上也能致人死命,何況吞入肚中,這幾口吃下去,定是性命不保。謝遜將狼牙棒在地下一插,伸手一把將元廣波抓了過來,喀喇一響,捏脱了他的下巴,使他張著嘴無法再行合攏,當即將一大碗毒鹽,盡數倒入他的肚裡。

餘姚張登雲全家在一夜之間被人殺絶,海門歐陽清在客店中遇襲身亡,這是近年來武林中的兩大疑案,想不到竟是海沙派的元廣波所爲,衆人見他被逼吞食毒鹽,不自禁都有痛快之感。謝遜拿起另一大碗毒鹽,説道︰「我姓謝的做事一生公平正直,你吃一碗,我陪你吃一碗。」張開大口,將那大碗鹽都倒入肚中。這一招大出衆人意料之外,張翠山見他雖然出手兇狠,但眉宇之間,自有一股凜然正氣,何況他所殺的均是窮兇極惡之輩,心中已對他頗具好感,忍不住朗聲説道︰「謝前輩,這種奸人死有餘辜,何必跟他一般見識?」

謝遜橫過眼來,瞪視著他。張翠山微微一笑,竟無半分懼怕之色。謝遜道︰「閣下是誰?」張翠山道︰「晩輩武當張翠山,敬問前輩安好。」謝遜道︰「{\upstsl{嗯}},你是武當派張五俠,你也是來爭奪屠龍刀麼?」張翠山搖頭道︰「晩輩到王盤山來,是要査問我師哥兪岱岩受傷的原委,謝前輩似乎知曉其中詳情,還請示知。」謝遜尚未回答,只聽得元廣波一聲慘呼,捧住肚子在地下亂滾,滾了幾轉,捲曲成一團而死。張翠山急道︰「謝前輩快服解藥。」謝遜道︰「服什麼解藥?取酒來!」白眉教接待賓客的司賓忙取酒杯酒壼過來。謝遜喝道︰「白眉教這般小氣,拿大瓶來!」那司賓親自捧了一大{\upstsl{罈}}陳酒,恭恭敬敬的放在謝遜面前,心中却想︰「你中毒之後再喝酒,那不嫌死得不彀快麼?」

只見謝遜捧起酒{\upstsl{罈}},骨都骨都的狂喝入肚,這{\upstsl{罈}}酒少説也有三四十斤,竟給他片刻間喝得乾乾淨淨。他撫著高高凸凸的大肚子拍了幾拍,突然一張口,一道白練也似的酒柱激噴而出,打向白龜壽的胸口。白龜壽待得驚覺,酒柱已打中身子,便似一個數百斤的大鐵錘連續打到一般,饒是他一身精湛的内功,也感抵受不住,晃了幾晃,委頓在地。謝遜轉過頭來,噴酒上天,那酒水如雨般散將下來,都落在巨鯨幫一干人身上。自幫主麥鯨以下,人人都淋得滿頭滿臉,但覺那酒水腥臭不堪,功力稍差的都暈了過去。原來謝遜飲酒入肚,洗淨胃中的毒鹽,再以内力逼出,這數十斤酒都變成了毒酒,他腹中留存的毒質却已微乎其微,以他内力之深,這些微毒質已絲毫不能爲害。

巨鯨幫幫主麥鯨受他這般戲弄,霍地站起,但轉念一想,終是不敢發作,重又坐下。謝遜説道︰「麥幫主今年五月間你在閩江口劫一艘遠東海船,可是有的?」麥鯨臉如死灰,道︰「不錯!」謝遜道︰「閣下在海上爲寇,若不打劫,倚何爲生?這一節我也不來怪你,但你們將數十名無辜客商盡數抛入海中,又將七名少女輪姦致死,江湖上英雄人物,能做這等傷天害理之事麼?」麥鯨道︰「這\dash{}這\dash{}這是幫中兄弟們幹的,我\dash{}我可没有。」謝遜道︰「你手下人這般丟盡武林中人物的臉面,你不加約束,與你自己所幹何異?是那幾個人幹的?」

麥鯨當此處境,只求自己免死,拔出腰刀,説道︰「蔡四、花青山、海馬胡六,那天的事,你們三個有份吧!」刷刷刷三刀,將三人砍翻在地。這三刀出手也眞利落快捷,蔡四等三人絶無反抗餘地。

謝遜道︰「好!只是未免太遲,又非你的本願。倘若你當時殺了這三人,今日我也不會跟你來比武了。麥幫主,你最擅長的功夫是什麼?」麥鯨見仍是逃避不了,心想︰「在陸上跟他比武,只怕走不上三招。但到了大海之中,却是我的天下了。便算不濟,總能逃走,難道他水性能及得上我?」於是説道︰「在下想領教一下謝前輩的水底功夫。」謝遜道︰「好,咱們到海中去比試啊。」走了幾步,忽道︰「且慢,我一走開,只怕這裡的人都要逃走!」

衆人聽了他這句話,都是心中一凜,暗想︰「他怕我們逃走,難道要將這裡的人個個置於死地?」麥鯨抓到這個機會,忙道︰「其實便是到海中比試,在下也決不是前輩的對手,我認輸便是。」謝遜道︰「噫,那倒省事。你既認輸,這就橫刀自殺吧。」麥鯨心中怦的一跳,道︰「這個\dash{}比試武藝,勝負原是常事,也用不著自殺\dash{}」謝遜喝道︰「胡説八道!諒你也配跟我比試武藝?今日我是索債討命來著,凡是作過傷天害理之事、殺過無辜之人性命的,一個也不能放過。只是怕你死得不服,是以叫你們一個個施展生平絶藝,只要有一技之長能勝得過我的,那便饒了他的性命。」

他這一席話一説完,從地下抓起兩大塊泥團,倒些酒水,和成了兩塊濕泥,道︰「水性的優劣,端在瞧瞧能在水底支持長久,我和你各用濕泥封住口鼻,誰先耐不住伸手揭泥,誰便橫刀自盡。」當下也不問麥鯨是否同意,舉起左手的濕泥,貼在自己臉上,封住了口鼻,右手一揚,拍的一聲,另一塊濕泥飛擲過去,封住了麥鯨的口鼻。

衆人見了這等情景,雖覺好笑,但誰都笑不出來。麥鯨在濕泥封住口鼻之前,早已深深吸了口氣,當下盤膝坐倒,屏息不動。説到比拚長氣,他原是有過人之處,自從七八歳起,他便常自鑽到海底摸漁捉蟹,水性越練越高,便是一炷香不出水面,也淹他不死,因此這般比試他自信穩操勝算,焦慮之心盡失,凝神靜心,更能支持長久。謝遜却不如他這般靜坐不動,大踏步走到神拳門席前,斜目向著掌門人過三拳瞪視。

過三拳給他看得心中發毛,站起身來抱拳説道︰「謝前輩請了,在下是神拳門的過三拳。」謝遜嘴巴被封,不能説話,伸出右手食指,在酒杯中醮了些酒,在桌上冩了三個字。過三拳見了這三個字,登時臉如死灰,現出極度恐怖之色,宛似光天白日之下,突然見到勾魂惡鬼一般。跟他同席的弟子垂目向桌上一看,只見謝遜所冩的,乃是「崔飛煙」三字。那弟子茫然不解,心想「崔飛煙」似是一個女子名字,何以師父見了這三個字如此害怕?

原來崔飛煙乃是過三拳啓蒙學武的業師之女,過三拳在師父死後,對這位師妹始亂終棄,崔飛煙有了身孕,他却另行投入神拳門下,不再理她。崔飛煙羞憤之下,自縊而死。此事極爲隱祕,崔家的人早已死絶,除了過三拳自己,世間再也無人得知,不料事隔二十年,謝遜突然將她的名字冩了出來。過三拳心想︰「待一會他若勝了麥鯨,除去口上濕泥,不免將我當年這件醜事抖露出來。反正他饒我不過,還不如乘此良機全力進攻,他若運氣發拳,勢必會輸了給麥鯨。」當下朗聲道︰「在下執掌神拳門,生平學的乃是拳法,向你討教幾招。」也不待謝遜有猶豫餘地,呼的一拳向他小腹擊出。他一拳既出,第二拳跟著遞了出去。過三拳這名字的由來,乃是因他拳力極猛,一拳可斃牯牛,尋常武師萬萬擋不住他三拳的轟擊,江湖上傳揚開來,他本來的名字反而没人知道了。他心知眼前之事,利於速攻,倘若麥鯨先忍不住而揭去口鼻上的濕泥,那麼謝遜自可跟著揭去,但在揭去之前,自己却佔著極大的便宜。對方不能喘氣運力,武功自是大大的打了個折扣。

他兩拳擊出,謝遜隨手化解。過三拳只覺對方的勁力頗爲軟弱,和適纔震死常金鵬、噴倒白龜壽的神威大不相同,大叫一聲︰「第三拳來了!」他這第三拳有一個囉唆名目,叫作「橫掃千軍,直摧萬馬」乃是他平生所學之中最厲害的一招,在這一招拳法之下,傷過不少江湖上成名的英雄好漢。

這時麥鯨面紅耳赤,眼前金星亂冒,實在再也忍耐不住,麥少幫主見父親情勢危急,而謝遜却正在和過三拳比拳,靈機一動,伸手到鄰座一個本幫女舵主的頭髮上拔下一根銀釵,拗下釵脚寸許來的一截,對準麥鯨的嘴巴,伸指彈出。這半截銀釵刺到麥鯨口中,雖然不免傷及他咽喉齒舌,但在濕泥上刺了一個小孔,稍有空氣透入,那這場比試便已立於不敗之地。

眼見那半截銀釵離麥鯨身前尚有丈許,謝遜斜目已然瞥見,伸足在地下一踢,一粒小石子飛了起來,正好打中那半截銀釵。斷釵嗤的一聲飛回,勢頭勁急異常,麥少幫主「啊」的一聲慘叫,按住右目,鮮血涔涔而下,那斷釵已將他一眼刺瞎。便在此時,過三拳的第三拳已擊中在謝遜的小腹之上。

這一拳勢如風雷,拳力未到,已是極爲威猛,過三拳料想謝遜不敢伸手硬接硬架,定須閃避,但不論避左避右竄高縮後,他都預伏下異常厲害的後著。豈知謝遜身子竟是不動,過三拳大喜,這一拳端端正正的擊中了他小腹。人體的小腹本來極是柔軟,但他著拳之處,如中鐵石,只感拳上劇痛,心知不妙,急忙縮手,那知這一縮竟是縮不回來,一個拳頭已被謝遜的小腹吸住。

謝遜左手倏出,往他腰間摸去。神拳門的兩名弟子見師父被困,分從左右向謝遜撲了過去。謝遜橫眼一瞪,兩名弟子竟被他眼中威勢所懾,停住脚步。謝遜抓住過三拳的腰帶,輕輕一扯,拉了下來,在他頭頸中繞了兩圏,跟著繞了個空圏,打個死結。他肚子一放,過三拳的右拳縮回,但後領已被謝遜一把抓住,身子便如騰雲駕霧的飛起,跟著頸中一緊,原來那腰帶結成的圏子已被謝遜套在一株大樹之上。

那圏子在他頭頸中越收越緊,過三拳手足亂舞,想要伸手去解頸中的腰帶,竟是不能,霎時之間,眼前出現了崔飛煙的影子,似乎見到她自縊而死時的痛苦慘狀。他又害怕,又是懊悔,耳中只是響著︰「天網恢恢,惡有惡報!天網恢恢,惡有惡報!」

謝遜回過頭來,只見麥鯨已是雙眼翻白,氣絶而死。他先除去麥鯨口鼻上的濕泥,探了探他的鼻息,這纔抹去自己口上的濕泥,仰天長笑,説道︰「這兩人生平作惡多端,到今日遭受報應,已是遲了。」斗然間雙目如電,射向崑崙派的兩名劍客,從高則成望到蔣濤,又從蔣濤望到高則成,良久不語。高蔣二人臉色慘白,但昂然持劍,竟無懼色。張翠山見謝遜在頃刻之間,連斃四大幫會的首腦人物,武功之高,當眞是從所未見,眼見他便要向高蔣二人下手,站起身來,説道︰「謝前輩,據你所云,適纔所殺的數人都是死有餘辜,罪有應得。但若你不分青紅皂白的濫施殺戳,與這些人又有什麼分别?」謝遜冷笑道︰「有什麼分别?我武功高,他們武功低,強者勝而弱者敗,那便是分别了。」

\chapter{玄冰火窟}

張翠山嘆道︰「天道難言,人事難知,咱們但求心之所安,義所當爲,至於是禍是福,本也不必計較。」謝遜斜目凝視,説道︰「素聞尊師張三丰先生武功冠絶當世,可惜緣慳一面。你是他及門高第,見識却如此凡庸,想來張三丰也不過如此,這一面不見也罷。」張翠山見謝遜文武兼質,心下原甚佩服,忽聽他言語之中對恩師大有輕視之意,忍不住勃然發作,説道︰「我恩師學究天人,豈是凡夫俗子所能窺測?謝前輩武功高強,非後學小子所及,但在我恩師看來,也不過是一勇之夫罷了。」殷素素聽他言語傲慢,忙拉了拉他衣角,示意他暫忍一時之辱,不可吃了眼前虧。張翠山心道︰「大丈夫死則死耳,可決不能容你辱及恩師。」

那知謝遜却並不發怒,淡淡的道︰「張三丰開創宗派,説不定武功上眞的有獨特的造詣,武學之道,無窮無盡,就算我當眞及不上他的萬一,那也不足爲奇。總有一日,我要上武當山去領教一番。張五俠,你最擅長的是什麼功夫,我姓謝的今日想見識見識。」

殷素素聽他向張翠山挑戰,眼見常金鵬、麥鯨、過三拳等一干人屍橫就地,或懸身高樹,凡是和他動手過招的,無一得以倖免,張翠山武功雖強,顯然也決不是他的敵手,説道︰「謝前輩,屠龍刀已落入你手中,人人也都佩服你武功高強,學問淵博,你還待怎地?」謝遜道︰「関於這把屠龍刀,故老相傳有幾句話,你總也知道吧?」殷素素道︰「聽人説起過。」謝遜道︰「這刀是武林至尊,持了它號令天下,莫敢不從。到底此刀之中有何祕密,能令得普天下群雄欽服?」殷素素道︰「謝前輩無事不知,晩輩正想請教。」謝遜道︰「我也不知道。我取此刀後,要找個清靜之地,好好的想上幾年。」殷素素道︰「{\upstsl{嗯}},那妙得緊啊,謝前輩才識過人,如果你想不通,旁人是更加不能了。」

謝遜道︰「嘿嘿,我姓謝的還不是自大狂妄之輩。説到文武之學,少林派掌門人空聞大師,武當派張三丰道長,還有娥眉、崑崙兩派的長老,那一位不是身負絶學?至於聰明智慧,你白眉教的白眉鷹王殷教主,可也是百世難逢的才智之士啊。」殷素素站起,説道︰「多謝謝前輩稱譽。」謝遜道︰「我想得此刀,旁人自然是一般的眼紅。今日王盤山島上,無一是我敵手,這一著殷教主是失算了。他想只憑白壇主一人,對付海沙派、巨鯨幫各人已綽綽有餘,豈知半途中却有我姓謝的殺了出來\dash{}」殷素素插口道︰「並不是殷教主失算,乃是他另有要事,分身乏術。」謝遜道︰「這就是了,人家説殷教主算無遺策,但今日此刀落入我的手,未免於他美譽有損。」

殷素素跟他東拉西扯,純是在分散他的注意,好讓他不再跟張翠山比武,於是説道︰「人事難知,天意難料,外物不可必。諸葛武侯六出祁山而大功不成,不減令名。所謂謀事在人,成事在天。謝前輩福澤深厚,輕輕易易的取了此刀而去,旁人千方百計的使盡心機,却反而不能到手。」謝遜道︰「此刀出世以來,不知轉過了多少主人,也不知替它主人惹下了多少殺身之禍。今日我取此刀而去,焉知日後没有強於我的高手,將我殺了,又取此刀?」張翠山和殷素素對望一眼,覺得這幾句話之中頗有深意。張翠山更想起三哥兪岱岩只因與此刀有了干連,至今存亡未卜,而自己只不過一見寶刀,性命便操於旁人之手,死活難料。

只聽謝遜長長嘆了口氣,説道︰「你們二人文武雙全,相貌俊雅,我若殺了你二人,有如打碎一對珍異的玉器,未免可惜,可是形格勢禁,却又不得不殺。」殷素素驚道︰「爲什麼?」謝遜道︰「我取此刀而去,若是在這島上留下活口,不幾日天下皆知,這屠龍刀是在我姓謝的之手。這個來尋,那個來找,我姓謝的又不是無敵於天下,怎能保得住没有閃失?旁的不説,單是那個白眉鷹王,我姓謝的就保不定能勝過了他。」張翠山冷冷的道︰「原來你是要殺人滅口。」謝遜道︰「不錯。」張翠山道︰「那你又何必指摘海沙派、巨鯨幫、神拳門這些人的罪惡?」謝遜哈哈大笑,道︰「我是叫他們死而無冤,臨死時心中舒服些。」張翠山道︰「你倒很有慈悲心。」

謝遜道︰「世人孰誰無死,早死幾年和遲死幾年也無太大分别。你張五俠和殷姑娘正當妙齡,今日喪身王盤山上,似乎有些可惜。但在百年之後看來,還不是一般。當年秦檜倘若不害死岳飛,難道岳飛能活到今日麼?只須死的時候心安理得,並無特殊痛苦,也就是了。因此我要和兩位比一比功夫,誰輸誰死,再也公平不過。你們年紀輕些,就讓你們佔一個便宜。兵刃、拳脚、内功、暗器、輕功、水功,隨便那一樁,由你們自己挑,我都奉陪。」

殷素素道︰「你倒口氣很大,比什麼功夫都成,是不是?」她聽了謝遜的語氣,知道今日的難関看來已無法逃過。王盤山島孤懸海中,白眉教又自恃有白常兩大壇主在場,絶無差池,因此不會再有強援到來。她話中説得硬,音調却已微微發顫。謝遜一怔,他是個機智絶倫之人,心想她若是跟我比賽縫衣刺繡,梳頭抹粉,那可糟糕,於是朗聲道︰「當然以武功爲限,難道還跟你比吃飯喝酒嗎?」一瞥眼見張翠山拿著一柄摺扇,説道︰「要比文的也行,書畫琴棋、詩詞賦曲、猜謎對對,一切都可以比試一下,只是咱們以一場定勝負,你們輸了便當自殺。唉,這般俊雅的一對璧人,我可眞捨不得下手。」

張翠山和殷素素聽他説到「一對璧人」四字,都是臉上一紅。殷素素隨即秀眉微蹙,説道︰「你輸了也自殺麼?」謝遜笑道︰「我怎麼會輸?」殷素素道︰「比試便有輸贏。這位張五俠是名家弟子,文才武學,都是一時之選,焉知没一樣不能勝過你。」謝遜笑道︰「憑他有多大年紀,便算招數再高,功力總是不深。」

張翠山聽著他二人口角相爭,心下暗暗盤算︰「要比武功是決計敵不過的,他説琴棋書畫、詩詞賦曲,可惜這些我都只懂得一鱗半爪,只怕也及不上他的萬一。却跟他比試什麼?在什麼功夫之中,我尚能僥倖跟他鬥成平局?輕功麼?新學的這套掌法麼?」突然間靈機一動,説道︰「謝前輩,你既迫得我動手,不獻醜是不成的了。如果我輸於謝前輩手下,自當伏劍自盡,若是僥倖鬥成個平手,那便如何?」謝遜搖頭道︰「没有平手。第一項平手,再比第二項,總須分出勝敗爲止。」張翠山道︰「好,倘若晩輩勝得一招半式,自也不敢要前輩如何如何,只是晩輩要前輩答允一事。」謝遜道︰「一言爲定。你劃下道児吧。」

殷素素大是関懷,低聲道︰「你跟他比試什麼?有把握麼?」張翠山低聲道︰「説不得,盡力而爲。」殷素素低聲道︰「若是不行,咱們見機逃走,總勝於束手待斃。」張翠山苦笑不答,心想︰「船隻已盡數被毀,在這小小島上,却逃到那裡去?」於是整了整衣帶,從腰間取出鑌鐵判官筆。謝遜道︰「江湖上盛稱銀鉤鐵劃張翠山,今日正好讓我的兩頭狼牙棒領教領教。你的爛銀虎頭鉤呢,怎地不亮出來?」張翠山道︰「我不是跟前輩比兵刃,只是比冩幾個字。」説著緩步走到左首山峰前的一堵大石壁前,吸一口氣,猛地裡雙脚一撐,提身而起。他武當派的輕功原爲各門各派之冠,此時張翠山到了生死存亡的関頭,如何敢有絲毫大意?身形縱起丈餘,跟著使出「梯雲縱」絶技,右脚在山璧一撐,一借力,又是縱起兩丈,手中判官筆看準了石面,嗤嗤嗤幾聲,已冩了一個「武」字。一個字冩完,身子便要落下。

他左手揮出,銀鉤在握,倏地一翻,鉤住了石壁的縫隙,支住身體重量,右手跟著又冩了個「林」字。這兩個字一筆一劃,全是張三丰深夜苦思而創,其中所包含的陰陽剛柔、精神氣勢,可説是武當一派武功到了巓峰之作。雖然張翠山内力尚淺,筆劃入石不深,但這兩個字龍飛鳳舞,筆力雄健,有如快劍長戟,森然相向。兩字冩罷,跟著又冩「至」字,「尊」字,越冩越快,但見石屑紛紛而下,或如靈蛇盤騰,或如猛獸屹立,須臾間二十四字一齊冩畢,這一番石壁刻書,當眞如李白詩云︰

\begin{quotation}
飄風驟雨驚颯颯\hskip8pt落花飛雪何茫茫

起來向壁不停手\hskip8pt一行數字大如斗

恍恍如聞鬼神驚\hskip8pt時時只見龍蛇走

左盤右蹙如驚雷\hskip8pt狀同楚漢相攻戰
\end{quotation}

張翠山冩到「峰」字的最後一筆,銀鉤和鐵筆同時在石壁上一撐,翻身落地,輕輕巧巧的站在殷素素身旁。謝遜凝視著石壁上那三行大字,良久良久没有作聲,終於嘆了口氣,説道︰「我冩不出,是我輸了。」

要知「武林至尊」以至「誰與爭鋒」這二十四個字,乃是張三丰意到神會、一夜苦思而創出全套筆意,一橫一直、一點一挑,盡是融會著最精妙的武功。就算張三丰本人到此,倘若當時無此心境,又無凝神苦思的餘裕,驀地裡在石壁上冩二十四個字,也決計達不到如此出神入化的境地。謝遜雖然聰明,那想得到其中有此原由,只道眼前是爲屠龍刀而起爭端,他就隨意冩了這幾句武林故老相傳的言語。其實除了這二十四字,要張翠山另冩幾個,其境界之高下,登時相去倍蓰了。

殷素素拍掌大喜,叫道︰「是你輸了,可不許賴。」謝遜向張翠山道︰「張五俠寓武學於書法之中,别開蹊徑,令人大開眼界,佩服佩服。你有什麼吩咐,請快説吧。」他一生之中,只有吩咐旁人,從來没有聽命於人過一次,這時迫於諾言,心下大是沮喪。

張翠山道︰「晩輩末學後進,僥倖差有薄技,得蒙前輩獎飾,怎敢説『吩咐』兩字?只是斗膽求一事。」謝遜道︰「求我甚麼事?」張翠山道︰「前輩持此屠龍刀去,可要饒了這島上一干人的性命。但可勒令人人發下重誓,不許洩露祕密。」謝遜道︰「我纔没這麼傻,相信人家發甚麼誓。」殷素素道︰「原來你説過的話不算話,説道比試輸了,便得聽人吩咐,怎地又反悔了?」謝遜道︰「我要反悔便反悔,你又奈得我何?」轉念一想,終覺無理,説道︰「你們兩個的性命我便饒了,旁人却饒不得。」張翠山道︰「崑崙派的兩位劍士是名門弟子,生平素無惡行\dash{}」謝遜截住他話頭,説道︰「什麼惡行善行,在我瞧來毫無分别。你們快撕下衣襟,緊緊塞在耳中,不可透一點聲音進去,再用雙手牢牢按住耳朶。如要性命,不可自誤。」他這幾句話説得聲音極低,似乎生怕給旁人聽見了。

張翠山和殷素素對望一眼,不知他是何用意,但聽他説得鄭重,想來其中必有緣故,於是依言撕下衣襟,塞入耳中,再以雙手按耳,突見謝遜張開大口,似乎縱聲長嘯,兩人雖然聽不見聲音,但不約而同的身子一震,又似脚底下站立著的土地也跟著顫動,只見白眉教、巨鯨幫、海沙派、神拳門各人一個個張口結舌,臉現錯愕之色。跟著那錯愕的神色變成痛苦難當,宛似全身在遭受苦刑。又過片刻,一個個的倒了下去,在地下扭曲滾動。崑崙派的高蔣二人一驚之下,當即盤膝閉目而坐,運用内力和謝遜的嘯聲相抗。張翠山雖然聽不見嘯聲,但見他二人額頭上黃豆般的汗珠滾滾而下,顏面手足上的肌肉都是不住抽動,可想而知,兩人的定力實是擋不住嘯聲的強攻。兩人的雙手幾次三番想伸上去按住耳朶,但伸到離耳數寸之處,終於又放了下來。突然間張翠山身子一震,只見高則成和蔣濤同時一躍而起,飛高丈許,直挺挺的摔將下來,再也不動了。

謝遜閉口停嘯,打個手勢,令張殷二人取出耳中的布片,説道︰「這些人經我一嘯,盡數暈去,性命是可以保住的,但醒過來後神經錯亂,成了瘋子,再也想不起、説不出已往之事。張五俠,你的吩咐我是做到了,王盤山島上這一干人的性命,我都饒了。」張翠山默然,心想︰「你雖不殺他們,但這些人雖生猶死,只怕比殺了他們更慘酷些。」心中對謝遜的殘忍狠毒,直説不出的痛恨。

但想到他一嘯之中,竟有如斯雷霆萬鈞的神威,心下也是不勝駭異,倘若自己事先没有以布塞耳,遭遇若何,眞是難以想像,但見高則成、蔣濤、白龜壽等一個個昏暈在地,滿臉焦黃,神情極是悽慘。謝遜不動聲色,淡淡的道︰「咱們走吧!」張翠山道︰「到那児去?」謝遜道︰「回去啊!王盤山島上揚刀立威之事已了,留在這裡幹麼?」張翠山和殷素素對望一眼,心想︰「還得跟這魔頭同舟一日一夜,這十二個時辰之中,不知還會有什麼變故?」

謝遜引著二人走到島西的一座小山之後。只見港灣中舶著一艘三桅船,那自是他來到島上的座船了。謝遜走到船邉,欠身説道︰「兩位請上船。」殷素素冷笑道︰「這時候你倒客氣起來啦。」謝遜道︰「兩位到了我的船上,是我嘉賓,焉能不盡禮接待?」三人上了船後,謝遜打個手勢,命水手拔錨開船。

船上共有十六七名水手,但掌舵的梢公發號令時,始終是指手劃脚,不出一聲,似乎人人都是啞巴。殷素素好奇心起,説道︰「虧你好本事,尋了一船又聾又啞的水手。」謝遜淡淡一笑,説道︰「那又有何難,我只須尋一船不識字的水手,刺聾了他們耳朶,再給他們服了啞藥,那便成了。」張翠山忍不住打個寒戰,目光中露出極度厭憎之色。殷素素拍手笑道︰「妙極妙極!既聾且啞,又不識字,你便有天大的祕密,他們也不會洩漏。可惜要他們駕船,否則連他們的眼睛也可刺瞎了。」張翠山橫了她一眼,責備道︰「殷姑娘,你是好好的一位姑娘,何以也如此殘忍,這是人間的大慘事,虧你笑得出?」殷素素伸了伸舌頭,想要辯駁,但一句話説到口邉,瞧瞧張翠山的面色,又縮了回去。謝遜淡淡的道︰「日後回到大陸,自會將他們的眼睛刺瞎。」

眼見布帆升起,船頭緩緩轉過,張翠山道︰「謝前輩,島上這些人呢?你將船隻盡數毀了,他們怎能回去?」謝遜道︰「張相公,你這人什麼都好,就是婆婆媽媽的太喜多事。讓他們在島上自生自滅,去如春夢了無痕,豈不美哉?」張翠山知道此人不可理喩,只得默然。但見座船漸漸離島,心想︰「島上這些人雖然大都是作惡多端之輩,但如此遭際,總是太慘,倘若無人來救,只怕十日之内,無一得活。」又想︰「崑崙派的兩名弟子這般死在島上,他們師長定要找尋,看來中原武林中轉眼便是一場軒然大波。」

這幾年來武當七俠縱橫江湖,事事佔盡上風,豈知今日之事,竟是縛手縛脚,命懸他人之手,絲毫没有反抗餘地。張翠山又是氣悶,又是惱怒,當下低頭靜思,對謝遜和殷素素都不理睬。一會児舟中的僮児端上酒菜,在几上斟了三杯酒。謝遜道︰「待我撫琴一曲,以娛嘉賓,還要請張相公和殷姑娘指教。」從艙壁上取下瑤琴,一調絃音,便彈了起來。張翠山於音韻一道,素不擅長,也不懂他彈些什麼,只是覺得琴音甚悲,充滿著蒼涼鬱抑之情,越聽越是入神,到後來忍不住淒然下泪。謝遜五指一劃,錚的一聲,琴聲斷絶,強笑道︰「本欲以圖歡娛,豈知反惹起張相公的愁思,罰我一杯。」説著舉杯一飲而盡。

張翠山道︰「謝老前輩雅奏,是何曲名,要請指教。」謝遜望著殷素素,似欲要她代答,殷素素搖搖頭,也不知道。謝遜道︰「晉朝稽康臨殺頭之時,所彈的便是這一曲了。」張翠山驚道︰「這是『廣陵散』麼?」謝遜道︰「正是。」張翠山道︰「自來相傳,稽康死後,廣陵散從此絶響,却不知謝前輩從可處得此曲詷?」

謝遜笑道︰「稽康這個人,是很有點意思的,史書上説他『文辭壯麗,好言老莊而尚奇任俠』,這不是很對你的脾胃麼?鍾會當時做大官,慕名去拜訪他,稽康自顧自打鐵,不予理會。鍾會討了個没趣,只得離去。稽康問他︰『何所聞而來,何所見而去?』鍾會説︰『聞所聞而來,見所見而去。』鍾會這傢伙,也算得是個聰明才智之士了,就可惜胸襟太小,爲了這件事心中發愁,向司馬昭説稽康的壞話,司馬昭便把稽康殺了。稽康臨刑時撫琴一曲,的確很有氣度,但他説『廣陵散從此絶矣』,這句話却未免把後世之人都看得小了。他是三國的人,此曲就算在三國之後失傳,難道在三國之前也没有了嗎?」

張翠山不解,道︰「願聞其詳。」謝遜道︰「我對他這句話不服氣,便去發掘西漢、東漢兩朝皇帝和大臣的墳墓,一連掘了二十九個古墓,終於在蔡邕的墓中,覓到了『廣陵散』的曲譜。」説罷呵呵大笑,甚是得意。張翠山心下駭然,暗想︰「此人當眞無法無天,爲了千餘年前古人的一句話,竟會負氣不服,甘心去做盜墓賊。若是當世有人得罪了他,更不知他要如何處心積慮的報復了。」一抬頭,只見船艙壁上掛著一幅山水,絹色甚古,畫中峰巒筆立,氣勢壯偉,却没署名。謝遜見他注視不休,道︰「這是梁朝張僧繇之作,是我到皇宮中去取來的。據説張僧繇畫龍不點睛,一點睛,墨龍便破壁飛去。此説自是故神其事,決不可信。但你瞧他畫筆流動,不亞於你在石壁上所書的二十四字呢。」張翠山道︰「晩輩亂塗亂抹,焉敢和前賢相比?」

他三人自到了船艙之中,謝遜説古論今,評詩述文,宛似一位宿學大儒一般,張翠山雖然折服,但每一念及他行事之殘酷,憎恨之情又油然而生。這時謝遜却在跟殷素素談論五胡亂華胄石勒、石虎一怒之下便殺數萬人的「盛事」,張翠山無心多聽,從窗中望出去觀賞風景,只見夕陽即將沉入海心,照得海中萬道金蛇,閃爍不定,正出神間,忽地一驚︰「那夕陽怎地在船後落下?」回頭問謝遜道︰「掌舵的稍公迷了方向啦,咱們的船正向東行駛。」謝遜道︰「是要向東,没錯。」殷素素也吃驚起來,道︰「向東是茫茫大海,却到那裡去?」

謝遜斟了杯酒,細辨酒味,説道︰「這是紹興的女貞陳酒,至少已有二十年的功力,兩位不可小視它啊。」殷素素急道︰「你還不叫稍公轉舵?」謝遜道︰「在王盤山島上,不早已跟你們説清楚了?我得了這柄屠龍寶刀,須當找個清淨之地,好好的思索幾年,要明白這寶刀爲什麼是武林至尊,爲什麼號令天下,莫敢不從。中原大陸是紛擾之地,人人知道我得了寶刀,今日這個來搶,明日那個來奪,打發那些兔崽子也彀人麻煩的了,怎能靜得下心來?倘若來的是張三丰先生、白眉教主這些高手,我姓謝的還未必穩勝。因此要到汪洋大海之中,找個人跡不到的荒僻小島,定居下來。」

殷素素道︰「那你把我們先送回去啊。」謝遜笑道︰「你們一回中原,我的行藏豈不就此洩漏?」張翠山霍地站起身來,厲聲道︰「你待如何?」謝遜道︰「只好委屈你們兩位,在那荒島上陪我過幾年逍遙快樂的日子,等我想通了寶刀的祕密,咱三人再一起回來。」張翠山道︰「若是十年八年也想不出呢?」謝遜笑道︰「那就在島上陪我十年八年,我一輩子想不出,那就陪我一輩子。你兩個郎才女貌,情投意合,便在島上成了夫妻,生児育女,豈不美哉?」張翠山大怒,拍桌喝道︰「你快别别説八道!」斜眼一睨,只見殷素素含羞低頭,暈紅雙頰。

\chapter{狂風海嘯}

張翠山心下一驚,隱隱覺得,若是和殷素素再相處下去,只怕自己要管不住自己,謝遜是一個強敵,殷素素是一個強敵,而自己内心中的心猿意馬,更是一個強敵,這種危機四伏的是非之地,越早離開越好,當下強抑怒火,説道︰「謝前輩,在下言而有信,決不洩漏前輩行蹤。我此刻可立下重誓,對任誰也不吐露今日的所見所聞。」謝遜道︰「張五俠是俠義名家,一諾千金,言出如山,江湖間早有傳聞。但我姓謝的在二十五歳立過一個重誓,你瞧瞧我的手指。」説著伸出左手,張翠山和殷素素一看,只見他手掌上小指和無名指齊根斬斷,只剩下三根手指。

謝遜臉上殊無激動之色,説道︰「在那一年上,我生平最崇仰、最敬愛的一個人欺騙了我,害得我身敗名裂,家破人亡,母親妻児,一夕之間盡數死去。因此我斷指立誓,我姓謝的有生之日,決不再信任一個人。今年我四十五歳,二十年來,我只和禽獸爲伍,我相信禽獸,不相信人。二十年來我不殺禽獸只殺人,我茹素食齋,不食禽獸之肉,但人肉却吃得津津有味。」

張翠山打了個寒戰,心想怪不得他彈這曲「廣陵散」時,琴韻中充滿了如此淒涼的心聲,又怪不得他身負絶世武功,江湖上却默默無聞,絶少聽人説起,想是他二十五歳上所遭之事定是慘絶人寰,以致他憤世嫉俗,離群索居,將天下所有的人都恨上了。他本來對謝遜的殘忍暴虐痛恨無比,這時聽了這幾句話,不由得起了一些同情之意。他沉吟片刻,説道︰「謝前輩,你的深仇大恨,想來已經報復了?」

謝遜道︰「没有。害我的人武功極高,我打他不過。」張翠山和殷素素不約而同「咦」的一聲,説道︰「比你還要厲害?這人是誰?」謝遜道︰「我幹麼要説他的名字,自取其辱?倘若不是爲了這一場深仇大恨,我何必搶這屠龍寶刀?何必苦苦的去想這刀中的祕密?張五俠,我一見你,便跟你投緣,照我平日的脾氣,決不容你活到此刻。我讓你二人多活幾年,這大破我常例之事,只怕其中有些不妙。」

殷素素道︰「什麼多活幾年?」謝遜淡淡的道︰「待我想通了寶刀中的祕密,離島之時再將你二人殺死。我遲一天想出來,你們便多活一天。」殷素素道︰「哼,這把刀也不過沉重鋒利,烈火不損,其中有什麼祕密?什麼『號令天下,莫敢不從』,也不過説它能在天下兵刃中稱王稱霸吧了。」謝遜嘆道︰「假如眞是如此,咱三個就在荒島上守一輩子吧。」突然間臉色慘然,心情沮喪,覺得殷素素這幾句話確是實情,那麼報仇之舉,看來是終生無望了。

張翠山見了他的神色,忍不住想説幾句安慰的話,那知謝遜{\upstsl{噗}}的一聲,吹熄了臘燭,説道︰「睡吧!」跟著長長的嘆了口氣,這嘆聲之中,充滿著無窮無盡的痛苦、無邉無際的絶望,竟然不似人聲,便像一隻受了重傷的野獸,臨死時的悲{\upstsl{嗥}}一般。這聲音混在船外的波濤聲中,張殷二人聽來,都是暗暗心驚。

海風一陣陣的從艙口中吹了進來,殷素素衣衫單薄,過了一會,漸漸的抵受不住,身子輕輕顫抖。張翠山低聲道︰「殷姑娘,你冷麼?」殷素素道︰「還好。」張翠山除下長袍,道︰「你披在身上。」殷素素接了過來披在肩頭,感到長袍中還帶著張翠山身上的溫暖,心頭甜絲絲的,忍不住在黑暗中嫣然微笑。在張翠山心中,却是在盤算脱身之計,想來想去,出路只有一條︰「不殺謝遜,不能脱身。」

他側耳細聽,在洶湧澎湃的浪濤聲中,聽得謝遜鼻息凝重,顯已入睡,心想︰「此人自稱立下重誓,一生決不信人,但他和我同臥一船,竟能安心睡去,何以不怕我下毒手加害?難道他有恃無恐,決不將我放在心上嗎?不管如何,只好冒險一擊。否則此人説得出做得到,稍有遲疑,我大好一生,便要陪著他葬送在荒島之上。」於是輕輕移身到殷素素身旁,想在她耳畔講一句話,那知黑暗之中看不清楚,殷素素適又於此時轉過臉來。兩個人兩下裡一湊,張翠山的嘴唇正好在她在右頰上吻了一下。

張翠山大吃一驚,待要分辯此舉並非自己輕薄,却又不知如何説起。殷素素滿心喜歡,將頭斜靠在他的肩頭,霎時之間心中充滿了柔情蜜意,但願這船在汪洋大海中無休無止的前駛,此情此景,百年如斯,忽覺張翠山的口唇又湊在自己耳旁,低聲道︰「殷姑娘,你别見怪。」殷素素早羞得滿臉如一朶大紅花一般,也低聲道︰「你喜歡我,我很是高興。」她雖然行事任性,殺人不眨眼,但遇到了這種児女之情,竟也和初嘗愛戀滋味的妙齡姑娘一般,心中又驚又喜,又慌又亂,若不是在黑暗之中,連這句話也是不敢説的了。

張翠山怔了一怔,没料到自己一句道歉,却換來了對方的眞情流露。殷素素嬌艷無倫,自從初見,即對自己脈脈含情,這時在這短短的九個字中,更是表達了傾心之忱,張翠山血氣方剛,雖然以禮自持,究也不能無動於衷,只覺得她身子軟軟的倚在自己肩上,淡淡的幽香,一陣陣的送進鼻管中來,待要對她説幾句溫柔的話,忽地心中一動︰「張翠山,大敵當前,何以竟是如此把持不定?恩師的教訓,難道都忘得乾乾淨淨了?便算她和我兩情相悦,她又於我兪三哥有恩,但終是出身邪教,行爲不正,須當稟明恩師,得他老人家允可,再行媒聘,豈能在這暗室之中,效那邪褻之行?」想到此處,身子突然坐直,低聲説道︰「咱須得設法制住此人,方能脱身?」

殷素素心中正在迷迷糊糊地,忽然聽他這麼説,不由得呆了一呆,道︰「怎麼?」張翠山低聲道︰「咱們雖然身處險境,行事仍當光明正大,若當他睡夢之中忽施暗襲,非大丈夫所當爲。我叫醒他,跟他比拚掌力,你立即用金針射他穴道。雖是以二敵一,未免勝之不武,但咱們和他武功相差太遠,只好佔這個便宜。」這幾句話説得聲細如蚊,他口唇又是緊貼在殷素素耳上而説,那知殷素素尚未回答,謝遜坐在後艙却已哈哈一笑,説道︰「你若是忽施偸襲,我姓謝的雖是一般的不能著你道児,總是還有一線之機,現在偏偏要什麼光明正大,保全名門正派的俠義門風,當眞是自討苦吃了。」這個「了」字剛出口,身子一晃,已欺到張翠山身前,輕飄飄的一掌,拍向他的胸前。

張翠山當他説話之時,早已凝聚眞氣,暗運功力,他一掌拍到,當即伸出右掌,以師門心傳的「綿掌」還擊,雙掌相交,只是嗤的一聲輕響,但覺胸口一震,對方掌力已排山倒海般壓了過來。張翠山自知對方武功高出自己十倍大有餘,對方掌力未到之時,早已將氣勁貫護全身,只守不攻,有了個多挨一刻便好一刻的想頭。因此謝遜一掌擊到,他手臂被震得向後縮了八寸。這八寸之差,使他守禦上更佔便宜,雖然決計傷不了對方,但不論謝遜如何運勁推掌,一時却推不開他防禦的掌力。

謝遜連催三次掌力,只覺對方的勁力雖然比自己微弱得多,但説也奇怪,竟是弱而不衰,微而不竭,自己掌力越催越重,張翠山始終堅持擋住。只聽得脚底下船板格格而響,在這兩人比拚之下,船板却抵受不起了。

只須兩人再運力一催,船艙底非破裂不可,謝遜左掌一起,往張翠山頭頂壓落。張翠山左臂稍曲,以一招「橫架金樑」擋住,只覺前胸是襲來的陰柔之力綿綿不絶,頭頂壓下的却是陽剛之勁雷霆萬鈞,一個人雙掌之中竟能同時發出兩種截然相反的勁力,同樣的威猛無儔,這等功夫,確是他生平從所未聞。好在武當派的武功原以綿密見長,各派之中,可稱韌力無雙,兩人雖然武功相差甚遠,張翠山原已立於必敗之地,但他運起師傳心法,借力卸力,四兩撥千斤,謝遜在一時之間,也眞奈何他不得。

兩人相持片刻,張翠山汗下如雨,全身盡濕,心中暗暗焦急︰「怎地殷姑娘還不出手?他此刻全力攻我,殷姑娘若以金針射他穴道,就算不能得手,他也非撒手防備不可,只須氣息一閃,立時會中我掌力。」這一節謝遜也早已想到,他本來預計張翠山在他雙掌齊擊之下,登時便會重傷,那知他年紀輕輕,内功上的造詣竟自不凡,支持到一盞茶時分,居然還能不屈。兩人一面比拚掌力,一面都注意著殷素素的動靜。張翠山氣凝於胸,不敢吐氣開聲,謝遜却漫不在乎,説道︰「小姑娘,你還是别動手動脚的好,你金針一發,我掌力加重,你的心上人活不到一時三刻。」

殷素素道︰「謝前輩,咱們跟著你便是,你撤了掌力。」謝遜道︰「張相公,你怎麼説?」張翠山焦急異常,心中只是暗叫︰「發金針,發金針,這稍縱即逝的良機,怎地不抓住了?」殷素素急道︰「謝前輩快撤掌力,小心我跟你拚命?」謝遜其實也眞忌憚殷素素忽地以金針偸襲,船艙中地方既窄,那金針細如牛毛,黑暗中射出來時無影無蹤,無聲無息,還眞的不易抵擋,何況自己雙掌和敵人膠凝鬥力,心想︰「這小姑娘震於我的威勢,不敢貿然出手,否則處此情景之下,只怕要鬧個三敗倶傷。」當下説道︰「我本來就没起異心。」謝遜道︰「你代他立個誓吧。」殷素素微一沉吟,説道︰「張五哥,咱們不是謝前輩的敵手,就陪著他在荒島上住個一年半載。以他的聰明智慧,要想通屠龍寶刀中的祕密決非難事,我就代你立個誓吧!」

張翠山心道︰「立什麼鬼誓?快發金針,快發金針!」却苦於這句話説不出口,黑暗中又無法打手勢示意,何況,自己雙手被敵掌牽住,根本就打不來手勢。

殷素素聽張翠山始終默不作聲,便道︰「我殷素素和張翠山決意隨伴謝前輩居住荒島,直至發現屠龍刀中所藏祕密爲止,我二人若起異心,死於刀劍之下。」謝遜笑道︰「咱們學武之人,死於刀劍有什麼稀奇?」殷素素一咬牙,道︰「好,教我活不到二十歳你總心滿意足了吧?」謝遜哈哈一笑,撤了掌力。張翠山全身脱力,委頓在艙板之上。殷素素急忙晃亮火摺,點燃了油燈,見張翠山臉如金紙,呼吸細微,心中大急,兩行情泪流下了雙頰。

謝遜笑道︰「武當子弟果然並非浪得虛名,不枉在中原武林稱雄。」殷素素從懷中掏出手帕,替張翠山抹去滿頭滿臉的大汗。張翠山心中一直怪她失誤良機,没有發射金針襲敵,但這時見她泪光瑩瑩,滿臉憂急之狀,確是發乎至情,不由得心中感激,嘆了一口長氣,待要説句安慰她的話,忽地眼前一黑,迷迷糊糊中只聽殷素素大叫︰「姓謝的,你累死了我張五哥,我跟你拚命。」謝遜却哈哈大笑,突然間也身子一側,滾了幾個轉身,但聽得謝遜、殷素素同時高聲大叫,呼喝聲中又夾著疾風呼嘯,波浪轟擊之聲,似乎千百個巨浪同時襲到。

張翠山只感全身一涼,口中鼻中全是鹽水,他本來昏昏沉沉,給水一沖,反而清醒了,第一個念頭便是︰「難道船沉了?」他不識水性,不由得心下慌亂,當即閉住呼吸,掙扎著站起,脚底下艙板斗然間向左側去,船中的海水又向外倒瀉,但聽得狂風呼嘯,大海洋翻天覆地的沸騰起來,張翠山尚未明白是什麼一回事,猛聽得謝遜喝道︰「張翠山,快到後梢去掌住了舵!」這一喝聲如雷霆,雖在狂風巨浪之中,仍是充滿著説不出的威嚴。張翠山不加思索,縱到後梢,只見黑影一晃,一名舟子被白浪沖出了船外,遠遠的跌出數丈,迅即沉没在波濤之中。

張翠山還没走到舵邉,又是一個浪頭撲了上來,這巨浪猶似一堵結實的水牆,砰的一聲大響,打得船上斷木橫飛。這當児張翠山一生勤修的武功顯出了功效,他雙脚牢牢的站在船面,竟如用鐵釘釘住一般,紋絲不動,待那巨浪過去,一個箭步,便竄到舵邉,伸手穩穩掌住。但聽得喀喇喇、喀喇喇猛響,却是謝遜橫著狼牙棒,將主桅和前桅一一擊斷。兩條桅桿帶著白帆,跌入海中。

但風勢實在太大,這時雖只後帆吃風,那船還是歪斜傾側,便似喝酔了酒,狂舞亂跳一般,謝遜竭力想收下後帆,饒是他一身武功,碰到了天地間自然之威,却也變得束手無策。那後桅向左直垂,帆邉已碰到水面,謝遜破口大罵︰「賊老天,打這般鳥風!」眼見稍有猶豫,痤船便要翻轉,只得提起一棒,將後後桅也打斷了。

三桅齊斷,這船在驚浪駭濤之中成了無主游魂,只有隨風飄蕩。張翠山大叫︰「殷姑娘,你在那裡?」他連叫數聲,不聽到答應,叫到後來,喊聲中竟帶了哭音。突然間一雙手攀上他的膝頭,跟著一個大浪没過了他頭頂,在海水之中,一個人緊緊的抱住了他腰。

待那浪頭掠過艙面,他懷中那人伸手摟住了他頭頸,柔聲道︰「張五哥,你竟是這般的掛念我麼?」正是殷素素的聲音。張翠山大喜,右手把住了舵,伸左手反抱著她,説道︰「謝天謝地!」在每一刻都可被大浪濤吞没的生死邉緣之上,張翠山忽地發覺,自己對殷素素的関懷,竟勝於計及自己的安危,心中驚喜交集︰「她好生生的在這児,没有掉入海中。」殷素素道︰「張五哥,咱倆死在一塊。」張翠山道︰「是的,素素,咱倆死在一塊。」

若是在尋常的境遇之下,兩人身份大不相同,縱有愛戀相悦之情,也決不能霎時間兩心如一。這時候兩人相抱在一起,眼看四周圍漆黑一團,船身格格響個不停,隨時都能碎裂,心中却感到説不出的甜蜜喜樂。張翠山和謝遜一番對掌,原已累得精疲力竭,但被殷素素的柔情一激勵,立時精神大振,任那浪濤左右衝擊,始終將舵掌得穩穩地,決不搖晃。

船上的聾啞舟子已盡數被沖入海中,這場狂風暴雨説來便來,事先竟無絲毫朕兆,原來是海底突然地震,帶同海嘯,氣流一加激盪,更惹起了一場龍捲風來。若不是謝遜和張翠山均是身負罕有的武功,如何抵擋得住?幸好那船又造得分外堅固,雖然船上的艙蓋,甲板被打得破碎不堪,船身却安慰無恙。

頭頂烏雲滿天,大雨如注,四下裡波濤山立,這當児那児還分得出東南西北?其實便算分得出方向,桅檣盡折,船隻已無法駕駛。謝遜清理了艙面,走到後梢,説道︰「張兄弟,眞有你的,讓我掌舵吧。你兩個到艙裡歇歇去。」張翠山站起身來,將舵交了給他,擕住殷素素的手,剛要舉步,驀地裡一個大浪飛到,將他兩人衝出船舷之外。這個浪頭來得極其突兀,事先竟是不及防備。

張翠山待得驚覺,已是身子凌空,這一落下去,脚底便是萬丈洪濤,百忙中左手一勾,抓住了殷素素手腕,右臂已被一根繩索套住,只覺身子忽地向後飛躍,衝浪冒水,倒退回來。原來謝遜及時發覺,拾起脚下的一根帆索,捲了他二人回船。砰砰兩聲,兩人摔在甲板之上。

這一下死裡逃生,張殷二人固是大出意外,謝遜也是暗叫一聲︰「僥倖!」若不是脚邉恰好有這麼一根帆索,便有天大的本事,也難以相救了。張翠山扶著殷素素走進艙中,船身雖然仍是一時如上高山,片刻間似瀉深谷,但二人經過適纔的危難,對這一切全已置之度外。殷素素倚在張翠山懷中,湊在他耳邉説道︰「五哥,我倘若能不死,我要永遠跟著你在一起。」張翠山心情激盪,道︰「我也正要跟你説這一句話,天上地下,人間海底,我倆都要在一起。」殷素素重複了一句︰「天上地下,人間海底,我倆都要在一起。」兩人相偎相倚,心中都反而暗暗感激這場海潚。

在謝遜心中,却是連珠價的不住叫苦,不論他武功如何高強,對這狂風驚浪,却是半點法子也没有,只有將自己交在它手中,任它隨意擺佈。這一場大海嘯,一直發作了七個多時辰,方始漸漸止歇。天上烏雲慢慢散開,露出星月之光。張翠山走到船梢,説道︰「謝前輩,多謝你救了咱二人的性命。」謝遜冷冷的道︰「這話不用説得太早,咱三人的性命,有九成還在賊老天的手中。」張翠山一生之中,從没聽人在「老天」二字之上,加上一個「賊」字,心想此人的憤世,可説已到了肆無忌憚的地步,但轉念一想,這葉孤舟,飄蕩在無邉大海之上,看來多半無倖。他剛和殷素素傾心相愛,對這世界加倍的留戀,便似剛在玉杯中啜到一滴美酒,立時便要被人奪去,「造化弄人」這四個字的意境,隨著謝遜那「賊老天」這一罵,是更加深深的體會到了。

他嘆了口氣,接過謝遜手中的舵來。謝遜累了一晩,自到艙中休息。殷素素坐在張翠山身旁,仰頭望著天上的星辰,順著北斗星的斗杓,找到了北極星,只見座船順著海流,正向正北飄行,説道︰「五哥,咱的船是在不停的向北啊。」張翠山道︰「是啊,最好是向西,那麼咱便有回歸家鄕之望。」殷素素出了一回神,道︰「若是它無止無息的向東,不知會到那裡。」張翠山道︰「向東是没有盡頭的海,只須飄浮得七八天,咱們没清水喝\dash{}」殷素素陶酔在目前的初戀滋味之中,不願去想這種煞風景的事,説道︰「我聽人説過,東海上有一座仙山,山上有長生不老的仙人,我們説不定便到了仙山島上,遇到了美麗的女仙\dash{}」她抬頭望著天上的銀河,説道︰「説不定這船飄啊流啊,到了銀河之中,於是我們看見牛郎織女在鵲橋上相會。」張翠山笑道︰「我們便把這艘船送給了牛郎,他想會織女時,便可坐船渡河去見她,不用等到一年一度的七月七日,方能相會。」殷素素道︰「將船送了給牛郎,我和你要相會時坐什麼啊。」張翠山微笑道︰「天上地下,人間海底,咱倆都在一起。既然在一起,何必要渡什麼銀河?」殷素素嫣然一笑,臉上便似開了一朶花,拿著張翠山的左手,輕輕撫摸。

兩人沉迷在許許多甜美的念頭之中,似乎有很多話要説,却又覺得一句話也不必説,過了良久良久,張翠山低頭望了她一眼,只見她雙目中泪光瑩瑩,臉有淒苦之色,訝道︰「你想起了什麼?」殷素素低聲道︰「在人間,在海底,我或許能和你在一起,但將來我二人死了,你會上天,我\dash{}我\dash{}我却要入地獄。」

張翠山道︰「胡説八道。」殷素素嘆了口氣道︰「我自己知道的,我這一生做的惡事太多,胡亂殺的人不計其數。」張翠山心中一驚,隱隱覺得自己跟她邪正殊途,實非良配,可是一來傾心已深,二來在這九死一生的大海洋中,又怎能計及日後之事?安慰她道︰「以後你改過遷善,多積功德,常言道︰知過能改,善莫大焉。」殷素素默然,過了一會,忽然輕輕唱起歌來。

她唱的是一曲「山坡羊」,元時曲調盛行,那「山坡羊」的曲子,自南至北,到處皆歌,只是詞句各有不同而已,只聽她唱道︰「他與咱,咱與他,兩下裡多牽掛。冤家,怎能彀成就了姻緣,就死在閻王殿前,由他把那杵來舂,鋸來解,把磨來挨,放在油鍋裡去炸。唉呀由他,只見那活人受罪,哪曾見過死鬼帶枷?唉呀由他!火燒眉毛,且顧眼下,火燒眉毛,且顧眼下。」

猛聽得謝遜艙中大聲喝采︰「好曲子,好曲子,殷姑娘,你比這個假仁假義的張相公,可合我心意多了。」殷素素道︰「我和你都是惡人,將來没有好下場。」張翠山低聲道︰「倘若你没有好下場,我也跟你一起没有好下場。」殷素素驚喜交集,只叫得一聲︰「五哥!」再也説不下去了。

次日天剛黎明,謝遜用狼牙棒在船邉打死了一條十來斤的大魚,三個人餓了兩日,雖是生魚,也吃得津津有味。那狼牙棒上生有鉤刺,用以打魚,可説是百發百中。船上雖無清水,但擠出魚肉中的汁液,勉強也可解渴。海流一直向北,帶著船隻日夜不停的向北駛去。一到夜晩,北極星總是在船頭之前閃爍,太陽總是在右舷方升起,在左舷方落下,連續十餘日,船行始終不變。

氣候却一天天的寒冷起來,謝遜和張翠山内功深湛,還可抵受得住,殷素素却一天比一天更是憔悴。張謝二人雖將自己外衣都給她穿上,仍是無濟於事。張翠山瞧著她強顏歡笑,勇敢地與寒風相抗,心中説不出的難受。眼看座船再北行數日,殷素素非凍死不可。那知天無絶人之路,這船突然駛到了一大群海豹之中。謝遜用狼牙棒擊死幾頭海豹,三人剝下海豹皮披在身下,宛然是上佳的皮裘,還有海豹肉可食,三人心情都是大爲歡暢。

這天晩上,三人聚在船梢上聊天,殷素素笑問︰「世上最好的禽獸是什麼東西?」三人齊聲笑著道︰「海豹!」便在此時,只聽得丁冬、丁冬數聲,極是清脆動聽。三人呆了一呆,謝遜臉色大變,説道︰「浮冰!」伸狼牙棒到海中去撩了幾下,果然碰到一些堅硬的碎冰。

這一來,三人的心情立時也如寒冰,大家都知這船日夜不停的向北流去,越北越冷,這時海中出現了小小的碎冰,日後勢必滿海是冰,座船一被凍住,移動不得,那便是三人畢命之時了。這一晩三人只是聽著丁冬、丁冬,冰塊互相撞擊的聲音,一夜不寐。

次日黎明,海中冰塊已有碗口大小,撞在船上,拍拍作響。謝遜苦笑道︰「我痴心妄想,要研求這屠龍寶刀中所藏的祕密,想不到來冰海,作冰人,當眞是名副其實,作了你兩位的冰人。」殷素素臉上一紅,伸手去握住了張翠山的手。謝遜提起屠龍刀,恨恨的道︰「還是讓你到萬丈之下的龍宮中去,去屠你媽的龍去吧!」一揚手,便要將刀投下,但甫要脱手之際,總是捨不得,嘆了口長氣,又將寶刀放入船艙。

再向北行了四天,滿海浮冰或如桌面,或如小屋,三人已知定然無倖,索性不再想生死之事。當晩睡到半夜,忽聽得轟的一聲巨響,船隻劇烈震動。謝遜叫道︰「妙得很,妙得很!撞上冰山啦!」

張翠山和殷素素相視苦笑,兩個人伸開手臂,摟在一起,只覺脚底下的冰水漸漸浸上小腿,顯是船底已破。謝遜叫道︰「跳上冰山去,多活一天半日也好的。賊老天要我早死,老子偏偏跟他作對。」張殷二人躍到船頭,眼前銀光閃爍,一座大冰山在月光下發出青冷的光芒,顯得又是奇麗,又是可怖。只見謝遜已站在冰山之側的一塊稜角上,伸出狼牙棒相接。殷素素伸左手在棒上一搭,和張翠山一齊躍上冰山。船底撞破的洞孔甚大,只一盞茶時分,已沉得無影無蹤。

謝遜將一塊海豹皮墊在冰山之上,三人並肩坐下。這座冰山有陸地上一個小山丘大小,橫廣十七八丈,縱長約爲五丈,比那座船是寬敞得多了。謝遜仰天清嘯一聲,説道︰「在船上氣悶得緊,正好在這裡舒舒筋骨。」站起身來在冰山上走來走去,似乎很感新奇。那冰山上雖然滑溜,但謝遜足步沉穩,便如在平地上行走一般。張翠山知他故意跟「賊老天」挑戰,便是死到臨頭,也是決不屈服。

那冰山順著風勢水流,仍是不停向北飄流。謝遜笑道︰「賊老天送了一艘大船給咱們,迎接咱三人去會北極仙翁。」殷素素似乎只須情郎在她身旁,她便心滿意足,便是天塌下來也全不縈懷。白天裡銀冰反射陽光,炙得三人皮膚也焦了,眼目更是紅腫發痛。因此三人每到白天,便以海豹皮蒙頭而睡,反而晩上起身捕魚,獵取海豹。但説也奇怪,那冰山越是向北,白天越長,到後來每天竟有十個時辰是白日,黑夜却是一晃即過。張翠山和殷素素還只體皮疲困,面目憔悴,謝遜却是神情日漸失常,眼睛中射出異樣的光彩,常自指手劃脚的對天咒罵,胸中怨毒,竟自不可抑制。

一日晩間,張翠山因白天没有安睡,這晩擁著海豹皮倚冰而臥,睡夢中忽聽得殷素素大聲尖叫︰「放開我,放開我。」張翠山一躍而起,在冰山的閃光之下,只見謝遜雙臂抱住了殷素素,口中荷荷的,發出野獸的聲音。張翠山這幾日對謝遜的神情古怪,早便在十分耽心,却没想到他以武林前輩的身份,竟會對一個少女突施非禮,心中又驚又怒,縱身上前,喝道︰「快放手!」

謝遜笑道︰「咱們早晩是個死,還講究什麼臭規矩?姓謝的便在陸地之上,也早不信騙人的什麼禮義廉恥,何況今日?」張翠山怒道︰「你再不放手,我可要跟你拚命了。」謝遜冷笑道︰「她是你什麼人,要你多管閒事?」口中這麼説著,雙臂一緊,殷素素「啊」的一聲,又叫了起來。張翠山道︰「她是我妻子,我是她丈夫。謝前輩,大丈夫生時光明磊落,死時慷慨自如,雖在這冰山之上,並無第四人知曉,可也别做出卑汚之事,自愧於心。」謝遜哈哈大笑,説道︰「我姓謝的從來不知什麼是善,什麼是惡。我見這姑娘生得美貌,今日便要佔她身子,就算你是她丈夫,也給我站在一旁,乘乖的瞧著。你再多説一句話,我一掌先擊你下冰山去。」

張翠山聽他説出這等話來,叫道︰「好,咱二人就拚一個同歸於盡!」氣凝右臂,呼的一掌往他後心拍去。謝遜左掌迴過,還了一掌。張翠山身子一晃,冰山上實在太滑,站不住足,登時一交滑倒。謝遜飛起右足,便往他腰間踢去。張翠山變招也快,手一撐,身子躍了起來,伸指便點到他膝蓋裡穴道。謝遜不等這一脚的招式使老,半途縮回,右掌往他頭頂拍落,左臂却又圏過將殷素素的纖腰抱住。

\chapter{南極仙境}

殷素素左手雙指倏出,往謝遜喉頭水突穴點去。謝遜毫不理會,只是雙足掌力,向張翠山腦門拍落。張翠山雙掌翻起,接了他這一掌,霎時之時,胸口塞悶,一口眞氣幾乎提不上來。殷素素雖在黑暗之中,認穴仍是極準,那兩指點中在他水突穴上,實是不差分毫,豈知手指碰到他的喉頭,又韌又硬,一彈便彈了出來,同時手指反而隱隱生疼。殷素素大吃一驚,心想便是練有最上乘金鐘罩鐵布衫功夫之人,也抵不住穴道上這兩指之戮,此人居然能以潛力將自己手指反彈,武功之奇,當眞是罕見罕聞。

其時她身子被謝遜緊緊抱住,右手被挾在他腋下,只有左手能得自由,點穴無效之後,寒冰的反光之中,但見謝遜雙目血紅,如要噴出火來。殷素素在這一霎之間,驀地想起幼時跟隨父親到山中打獵,一隻老虎受傷後負嵎而鬥,目光中也正是這般豁出了一切的瘋狂神色,事後想起,她常常覺得這隻老虎很是可憐。這時她心念一動︰「他平時吐屬斯文,謙和有禮,雖然性情怪僻,却也是個允文允武的奇男子,今日突然舉止乖張,看來是痛受刺激之下,頭腦中有了病啦。」便在此時,眼前一亮,北方映出一片奇異莫可名狀的彩光,於是柔聲説道︰「謝前輩,你安靜一息。你瞧,這天邉的光彩如何美麗!」謝遜順著她手指瞧去,但見北邉黑暗之中,射出無數奇麗無絶倫的光色,忽伸忽縮,大片橙黃之中夾著絲絲淡紫,忽而紫色愈深愈長,紫色之中,迸射出一條金光、紅光。謝遜心頭一震,走到冰山北側,凝目望著這片變幼的光彩。原來他三人順水飄流,此時已近北極,這片光彩,便是北極奇景的北極光了,中國之人,當時從來無人得見,饒是謝遜博覽群書,也是不知其故。

張翠山挽住殷素素,兩人心中兀自怦怦亂跳。這一晩謝遜凝望北極光,不再有何動靜,次晨光彩漸隱,謝遜對昨晩之事心中羞慚,却也不再提起,眼光竟是連殷素素的臉一瞧也不瞧,言語舉止之中,變得十分的溫文。

如此過了數日,冰山不住北去,謝遜對老天爺的咒罵,又是一天天的狂暴起來,偶然之間,眼光中又閃耀出猛獸般的神色。張翠山和殷素素心意相通,雖然互相不提此事,但兩人均是暗自戒備,生怕他又突然間狂性發作。

這一天算來已近戍時,但血紅的太陽停在西邉海面,良久良久,終是不沉下海去。謝遜突然一躍而起,指著太陽大聲罵道︰「連你太陽也來欺侮我,賊太陽,鬼太陽,我若是有一張弓,一枝長箭,嘿嘿,一箭射你個對穿。」突然伸手在冰山上一擊,拍下拳頭大的一塊冰塊,用力向太陽擲了過去。這冰塊遠遠飛出數十丈,落在海中。張翠山和殷素素相顧駭然,心中均想︰「這人好大的臂力,若是我,只怕一半的路程也擲不到。」

謝遜擲了一塊,又是一塊,雖是擲到七十餘塊,勁力竟是絲毫不衰,他見擲來擲去,跟太陽總是不知相距多遠,暴跳如雷,伸足在冰山上亂踢,只踢得冰屑紛飛。殷素素勸道︰「謝前輩,你歇歇吧,别去理這鬼太陽了。」謝遜回過頭來,眼中全是血絲,呆呆的望著她。殷素素暗自心驚,勉強微微一笑。謝遜突然大叫一聲,跳上來一把將她抱住,叫道︰「擠死你,擠死你!」殷素素身上猶似套上了一個鐵箍,而這鐵箍還在不斷收緊。張翠山忙伸手去扳謝遜手臂,却那裡扳得動分毫?眼見殷素素舌頭伸出,立時便要斷氣,只得呼的一拳,擊在他背心正中的「神道穴」上。

那知這一拳擊下,如中鐵石,謝遜如野獸般荷荷而吼。雙臂却抱得更加緊了。張翠山叫道︰「你再不放手,我用兵刃了!」但見他理也不理,當即抽出判官筆,在他右肩「肩貞穴」、左手臂「小海穴」中重重的各點一點。謝遜也眞了得,常人若是受這鐵筆如此沉重的一點,雙臂登時廢了,但他只是一陣酸麻,倏地回過右手,搶過判官筆,遠遠擲了出去。

殷素素但覺箍在身上的鐵臂微鬆,一矮身脱出了他的懷抱。謝遜左掌斜削,逕擊張翠山項頸,右手却往殷素素胸口抓去。嗤的一響,殷素素裹在身上的海豹皮被他五指硬生生的扯下一塊。張翠山知道自己若是縱身閃避,殷素素非被他再度擒住不可,當下便一招綿掌中的「自在飛花」想要卸去他的掌力,豈知手掌和他掌緣微微一沾,登時感到一股極大的黏力,再也解脱不開,只得鼓運内勁,與之相抗,但覺謝遜的手掌之中,傳來一片炙熱異常的氣流,只烤得他心煩意亂,頭暈腦脹。

張翠山和他比拚掌力,這次已是第三回,前兩回中均無這般情形,若不是前兩次中他並未使出這等古怪武功,那麼這幾日中他心神有異,武功竟自起了變化。謝遜一掌制住張翠山後,拖著他的身子,逕自向殷素素撲去。殷素素縱身躍開,她雙足尚未落地,謝遜在冰上一踢,七八粒小冰塊激飛而至,都打在她右腿之上。殷素素叫聲︰「啊喲!」橫身摔倒。謝遜突然發出掌力,將張翠山彈出數丈。這一下彈力極其強勁,張翠山落下時已在冰山的邉緣,那冰上甚是滑溜,他右足稍稍一沾,撲通一聲,摔入了海中。

張翠山暗叫︰「糟糕!」左手銀鉤揮出,擦的一聲,鉤住了冰山,借勢躍回,心想殷素素勢必又落入謝遜的魔掌之中,不料冷冷的月光之下,但見謝遜雙手按住眼睛,發出痛苦之聲,殷素素却躺在地上。張翠山急忙縱上扶起,殷素素低聲道︰「我\dash{}我打中了他眼睛\dash{}」一句話還没説完,謝遜虎吼一聲,撲了過來。張翠山抱住殷素素打了幾個滾,遠遠避開,但聽得喀喀喀幾聲響亮,謝遜的十指都插入了冰山。他一站起身來子,雙手已抓著一大塊百餘斤重的冰塊,側頭聽了聽聲音,向張殷二人擲了過來。

殷素素待要躍起躱開,張翠山一掀她背心,兩人都藏身在冰山的凹處,大氣不敢透一聲。但見謝遜擲出冰塊後,一動也不動,顯是在尋找二人藏身之所。張翠山見他雙目中各流出一縷鮮血,知道殷素素在危急之中終於射出了金針,而謝遜在神智昏迷下竟爾没有提防,雙目中針,成了盲人。

但他聽覺仍是十分靈敏,只要稍有聲息,給他撲了過來,這後果便無法設想,幸好海中既有浪濤,海風又響,再夾著冰塊相互撞擊的叮叮{\upstsl{噹}}{\upstsl{噹}}之聲,將兩人的呼吸都淹没了,倘若是在陸地上的靜室之内,兩人決計逃不脱他的毒手。

謝遜聽了半晌,在風濤冰撞的巨聲中,紿終發覺不到兩人的所在,雙目又痛,眼前是一片無邉無際的黑暗,狂怒之下,又加上恐怖,驀地大叫一聲,在冰山上一陣亂拍亂擊,抓起冰塊四下亂擲,只聽得砰砰之聲,響不絶耳。張翠山和殷素素相互抱住,都是嚇得面無人色,那些大冰塊在頭頂呼呼飛過,只須被他擲中一塊,實無倖免。

這一陣亂跳亂擲,約莫有大半個時辰,張翠山二人却如是挨了幾年一般,謝遜擲冰無效,忽然説道︰「張相公,殷姑娘,適纔我一時胡塗,狂性發作,以致多有冒犯,你二位不要見怪。」這幾句話説得謙和有禮,回復了平時的神態,他説過之後,坐在冰上,靜待二人答話。

張翠山雖然行事講究仁義,却也是個機智多智之輩,殷素素更是個使慣了詭計的,當此兇險的情境之下,那裡敢貿然接口?謝遜説了幾遍,聽張殷二人如終不答,站起身來,嘆了口氣,説道︰「兩位既然不肯見諒,那也無法。」説著深深吸了口氣。張翠山猛地驚覺,當日他在王盤山島上長嘯一聲,震倒衆人,發出嘯聲之前,也是這麼深深的吸一口氣。他雙眼雖盲,嘯聲摧敵却是絶無分别,這時危機霎息即臨,若要撕下衣襟塞住雙耳,已是遲了,當下不及細想,拉住殷素素的手用力一扯,兩個人一齊溜入了海中。

殷素素一時不明其理,謝遜嘯聲已發。張翠山拉著她急沉而下,寒冷澈骨的海水浸過頭頂,也淹住雙耳。張翠山左手扳住鉤在冰山的銀鉤,右手拉住殷素素,除了他一隻右手之外,兩人身子全都没入水底,但仍是隱隱感到謝遜嘯聲的威力。那冰山不停的向北移動,帶著他二人在水底潛行。張翠山暗自慶幸,倘若適纔失去的不是鐵筆而是銀鉤,就算逃過他的嘯聲,也是在大海之中淹死了。

過了良久,二人伸嘴探出海面,換一口氣,一直換了六七口氣,謝遜的嘯聲方止,他這番長嘯,消耗眞力極大,一時也感疲憊,顧不得來察看殷張二人的死活,坐在冰塊上暗自調勻内息。張翠山打個手勢,兩人悄悄的爬上冰山,從海豹皮上扯下絨毛,緊緊的塞在耳中,總算是暫且逃過一難。

可是跟他共處在這冰山之上,只要發出半點聲息,立時便有大禍臨頭。兩人愁顏相對,眼望西天,血紅的夕陽未落入海面。兩人不知地近北極,天時大變,這些地方,半年中白日不盡,另外半年却是長夜漫漫,但覺種種怪異,宛是到了世界的盡頭。

殷素素全身濕透,奇寒攻心,忍不住打戰,牙関相擊的得得幾聲,謝遜已然聽得。他縱聲大吼,提起狼牙棒直擊下來。張殷二人早有防備,急忙躍開閃避,但聽得砰的聲響,一棒打在冰山之上,擊下七八塊巨大冰塊,飛入海中,這一擊,少説也有千斤的力道。二人相顧駭然,但見謝遜舞動狼牙棒,閃動銀光萬道,直逼過來。他這狼牙棒棒身本有一丈多長,這一舞動,威力及於七八丈遠近,二人縱躍再快,也決計逃避不掉,只有不住的向後倒退,退得幾下,已到了冰山的邉緣。

殷素素驚叫︰「怎生是好?」張翠山右手擺了擺,拉著她手臂,雙足使勁,躍向海中。他二人身在半空,只聽得砰彭猛響,冰屑濺到背上,隱隱生痛。張翠山跳出時已看準了一塊桌面大的冰塊,左手揮出,搭了上去。謝遜聽著二人落海的聲音,用狼牙棒敲下冰塊,不住擲來,但他雙眼已盲,張殷二人在海中又是繼續飄動,第一塊没擲中,此後是再也投擲不中了。

那冰山浮在海面上的只是全山的極小部份,在水底之下,尚隱有巨大冰體,但張殷二人所附的冰塊,却是謝遜從冰山上所擊下,不到大冰山千份中的一份,因此在水流中飄浮甚快,和謝遜所處的冰山越離越遠,到天快黑時,回頭遠望,謝遜的身子已成了一個小黑點,那大冰山却兀自閃閃發光。

二人攀著小小冰塊,只是幸得不沉而已,但身子浸在冰水之中,如何能支持長久?幸好一路向北,不久便又有一座小小冰山出現,兩人手脚齊划,爬了上去。殷素素苦笑道︰「若説是天無絶人之路,偏偏叫咱們吃這許多苦。你身子怎樣?」殷素素道︰「可惜没來得及帶些海豹肉來。你的銀鉤也失去了麼?」兩人自管自的你言我語,誰也不知對方説些什麼,一怔之下,忙從耳中取出海豹的絨毛,原來他們顧得逃命,渾忘了耳中塞有物事。

兩人得脱大難,心中的柔情蜜意,斗然大增。張翠山道︰「素素,咱倆便是死在這冰山之上,也是永不分離的了。」殷素素道︰「五哥,我有句話問你,你可不許騙我。倘若咱們是在陸地上,没經過這一切危難,倘若我也是這般一心一意的要嫁給你,你也仍舊要我麼?」張翠山呆了呆,伸手搔搔頭皮,道︰「我想咱們不會好得這麼快,而且,而且\dash{}一定會有很多阻礙波折,咱們的門派不同\dash{}」殷素素嘆了口氣,道︰「我也這麼想。所以在船艙之中,你第一次和謝遜比拚掌力,我好幾次想發金針助你,却始終没有出手。」張翠山奇道︰「是啊,那爲什麼?我總當你在黑暗中瞧不清楚,生怕誤傷了我。」殷素素低聲道︰「不是的。假如那時我傷了他,咱二人逃回陸地,你便不願跟我在一起了。」

張翠山胸口一熱,叫道︰「素素!」殷素素道︰「或許你心中會怪我,但那時我只盼望跟你一起,去一個没人打擾的荒島之上,長相聚會。謝遜逼咱二人同行,那正合我的心意。」張翠山想不到她對自己竟是相愛如是之深,心中大爲感激,柔聲道︰「素素,我一點也不怪你。」殷素素偎倚在他懷中,仰起了臉,望著他的眼睛,説道︰「老天爺送我到這寒冰地獄中來,我是一點也不怨,只有歡喜。我只盼望這冰山不要回南,{\upstsl{嗯}},若是有一日咱們終於能回中原,你的師父會討厭我,我爹爹説不定要殺你\dash{}」

張翠山道︰「你爹爹?」殷素素道︰「我爹爹白眉鷹王殷天正,便是白眉教創教的教主。」張翠山道︰「啊,原來如此。素素,不要緊,我説過我是跟你在一起。你爹爹再兇,也不能殺他的親女婿啊。」殷素素雙眼發光,臉上起了一層紅暈,道︰「你這話可是眞心?」張翠山道︰「素素,我倆此刻便結爲夫婦。」當下兩人一齊在冰山之上跪下,張翠山朗聲道︰「皇天在上,弟子張翠山今日和殷素素結爲夫婦,禍福與共,始終不負。」殷素素虔心禱祝︰「老天爺保佑,願我二人生生世世,永爲夫婦。」她頓了一頓,又道︰「日後若得重回中原,弟子洗心革面,痛改前非,隨我夫君行善,決不敢再殺一人。若違此誓,天人共棄。」

張翠山大喜,没想到她竟會發此誓言,當即伸臂抱住她身子,兩人雖被海水浸得全身皆濕,但心中暖烘烘地竟是如沐春風。

過了良久,兩人才想起一日没有飲食。張翠山的兵刃都已失在大海之中,但殷素素却隨身佩著長劍,張翠山取過她長劍,以海豹皮裹住劍刃,力透指端,慢慢將長尖拗成一鉤,見有游魚游上水面,一鉤而上。這一帶的海魚爲抗寒冷,特别的肉厚多脂,雖是生食甚腥,但吃了大增力氣。

兩人在這冰山之上,明知回歸無望,倒也無憂無慮,其時白日極長而黑夜奇短,大反常態,已無法計算日子。也不知太陽在海面中升没幾回,忽有一日,只見正北方有一縷黑煙沖天而起。殷素素首先看到,嚇得臉都白了,叫道︰「五哥!」伸手指著黑煙。張翠山又驚又喜,道︰「難道這地方竟有人煙?」這黑煙雖然望見,其實相距甚遠,那冰山整整飄了一日,但見黑煙越來越高,到後來竟隱隱望見煙中夾有火光。殷素素道︰「那是什麼?」張翠山搖頭不答,殷素素顫聲道︰「五哥,咱倆的日子到頭啦!這是地獄門。」張翠山心中也大是吃驚,安慰她道︰「説不定那邉住得有人,正在放火燒山。」殷素素道︰「燒山的火頭那有這麼高?」張翠山嘆了口氣道︰「素素,既然到了這種怪地方,一切只有聽由老天爺安排。老爺既不讓咱們凍死,却要咱倆在大火中燒死,那也只得聽天由命。」

説也奇怪,兩人處身其上的冰山,竟是對準了那個大火柱緩緩飄去。當時張殷二人不明其中之理,只道冥冥中自有安排,是禍是福,一切是命該如此。其實那火柱乃是北極附近的一座活火山,火燄噴射,燒得山旁海水暖了。熱水南流,自然而然的吸引南邉的冰水過去補,因而帶著那冰山漸漸趨近。須知大海洋中所以發生颶風、海嘯,大都是因氣流水流冷熱不同,以致劇烈流動所致,這道理説穿了其實毫不稀奇。

這冰山又飄了一日一夜,終於到了火山脚下,但見那火柱周圍一片青綠,竟是一個極大的島嶼,島的四周都是尖石嶙峋的山峰,奇形怪樣,莫可名狀。張翠山足跡遍於中原,却從來没見過如此奇特的山峰,令人一見之下,心中如痴如狂,似酔似癲。原來些山峰均是火山的熔漿千萬年來堆積而成。島東却是一片望不到盡頭的平野,那火山灰逐年傾入海中而成,雖然地近北極,但因那火山萬年不滅,島上氣候便和長白山、黑龍江一帶相似高山處玄冰白雪,平野上却是極目青綠,蒼松翠柏,生得高大異常,還有許多中原所無的珍奇花樹。

殷素素望了半晌,突然躍起,雙手抱住了張翠山的脖子叫道︰「五哥,咱們是到了仙山啦!」張翠山心中也是充滿了快樂,迷迷糊糊地説不出話來。但見平野上有一群梅花鹿正在低頭吃草,極目四望,除了那火山有些駭人之外,周圍一片平靜,絶無可怖之處。但那冰山飄到島旁,被暖水一沖,被水一沖,反而向外浮動。殷素素急叫︰「糟糕,糟糕!仙人島又去不了啦!」張翠山也知情勢不妙,若是不上此島,這冰山再向别處飄流,不知何時方休?情急中連出數掌,吧吧吧一陣響,打下一塊大冰來,兩人張手抱住,撲通一聲,跳入海中。四手四脚一齊划動,終於爬上了陸地。

那群梅花鹿見有人來,睜著圓圓的眼珠望著張殷二人,顯得十分好奇,却殊無驚怕之意。殷素素慢慢走近,伸手在一頭梅花鹿的背上撫摸了幾下,説道︰「假使再有幾隻仙鶴,我説這便是南極仙境了。」突然間足下一晃,倒在地上。張翠山驚叫道︰「素素!」搶過去欲扶時,脚下也是一個踉蹌,站立不定。只聽得隆隆聲響,地面搖動,却是火山又在噴火。原來兩人在大海中飄浮了數十日,波浪起伏,晝夜不休,這時到了陸地,脚下反而虛浮,突然地面一動,竟致同時摔倒。

兩人一驚之下,見别無異狀,這纔嘻嘻哈哈的站了起來。當日疲累已極,兩人便在這平野之上,大睡了四個多時辰。醒來時太陽仍未下山,張翠山道︰「咱們四下裡瞧瞧,且看有無人居,有無毒猛獸。」殷素素道︰「你只須瞧這群梅花鹿如此馴善,這仙人島上定是太平得緊。」張翠山道︰「但願如此。可是咱們也得去拜謁一下仙人啊。」

殷素素雖然身在冰山,仍是是儘可能的使容顏整飭,衣衫修齊,這時到了島上,更是細心的整理一下衣衫,又替張翠山理了理頭髮,這纔出發尋幽探勝。她自己拿了鉤劍,張翠山折了一根松樹枝幹,作爲桿棒,以防不測。兩人展開輕身功夫,自南至北,一直快跑了二十來里,此時竟有大片土地可供奔馳,實是説不出的快活。沿途所見,除了低丘高樹之外,盡是青草奇花。草叢之中,偶而驚起一些叫不出名目的大鳥小獸,看來也是無害於人。

兩人轉過一大片樹林,只見西北角上一座石山,山脚下露出一個石洞。殷素素叫道︰「這地方妙得緊啊!」搶先奔了過去。張翠山道︰「小心!」一言未畢,只聽得荷的一聲,眼前白影一閃,洞中衝出一隻巨大的白熊來。

那白熊毛長身巨,竟和一隻大牯牛相似,殷素素猛吃一驚,急忙後躍。那白熊人立起來,提起巨掌,便往殷素素頭頂拍落。殷素素彎過鉤劍,刷的一劍,往白熊肩頭削去,那知她平時使慣長劍,這時劍頭鉤轉,短了一截,百忙中没想到此點,這一劍竟没削中,待得第二招迴劍掠去時,那白熊縱身撲上,拍的一響,已將鉤劍打落在地。張翠山急叫︰「素素退開!」躍上去樹幹橫掃,正打在白熊左前足的膝蓋之處。但聽得喀喇一響,樹幹斷爲兩截,白熊的左足却也折斷。白熊受此重傷,只痛得大聲吼叫,聲震山谷,兀自像一個野人般舞爪向張翠山抓來。

殷素素拾起鉤劍,待要上前相助,張翠山叫道︰「把劍擲向天空!」殷素素一怔之下,依言將劍擲起。張翠山雙足一點,使出「梯雲蹤」輕功,縱起丈餘,左手翻轉,接住劍柄。這時他左手持鉤劍,右手握短棒,宛似拿到了最稱手的銀鉤鐵筆,使一招「鋒」字訣中的一直,從半空中將桿棒直點下來,正中白熊的腦門。這一招勁力極大,樹枝直插下七八寸有餘,那白熊驚天動地般大吼一聲,在地下翻了幾個轉身,仰天而斃。

殷素素拍手笑道︰「好輕功,好筆法!」一言甫畢,猛聽得張翠山叫道︰「快躍過來!」殷素素聽他呼聲中頗有驚惶之意,不暇細問,向旁一竄,直撲到他懷裡,回過頭來,不禁「啊」的一聲驚呼,聲音發顫,原來在她身後一排站著七頭大白熊,每一頭都是張牙舞爪,猙獰可怖,却是聽到那白熊受傷時的吼聲而趕來救援。

莫看張翠山適纔殺斃那頭白熊甚是輕易,若要同時對付七頭白熊,却是萬萬不能,張翠山叫道︰「快逃!」拉住殷素素手臂,當即使開輕身功夫,回頭便奔。那些白熊身材雖然粗笨,奔跑起來居然甚是迅速,當然張殷一展開輕身術,衆白熊當即落後,但七頭熊緊追不捨,不管二人如何轉彎抹角,總是隨後趕來。張翠山道︰「咱們只有往海邉,説不得再往海中一跳。」殷素素道︰「白熊會游嗎?」張翠山搖頭道︰「不知道!只盼牠們不會!」兩人一面説,一面足不停步的急奔。殷素素突然叫道︰「啊喲,不好!」張翠山道︰「怎麼?」殷素素道︰「你知道白熊吃什麼爲生?我曾聽一個老梢公説,白熊最吃蜜糖,又愛吃魚。」張翠山突然收住足步,道︰「吃魚?」心想︰「要是白熊眞的吃魚,那麼逃到海中也不濟事。」

危急中正未想出計較,殷素素奇道︰「咦,怎地白熊反而跑在我們面前啦!」只見迎面共有六頭白熊奔來。張翠山道︰「不是的!我們前後受敵!」眼見山旁有一株大松樹,他先一躍而上,雙足勾住樹幹,倒轉身子,殷素素跟著躍起拉住他手。兩人爬在離地七八丈的高處。殷素素道︰「只盼望白熊不會爬樹!」張翠山道︰「會爬樹也不打緊,來一頭,殺一頭!只要不被包圍,那就好辦得多。」説話之間,前面六頭,後邉七頭,一共十三頭白熊都圍到了樹下,仰頭怒吼。這吼聲震耳欲聾,顯是欲得二人而甘心,以與被打死的那頭白熊報仇。張翠山折下了一根松枝,用甩手箭法,對準一頭白熊的右眼甩了下去,果然波的一聲輕響,樹枝入眼,那熊痛得大叫,伸爪抓住樹枝,拔了出來,牠狂怒之下,用頭向松樹猛撞。張翠山折了樹枝再擲時,那些白熊却學了乖,一齊低頭,在松樹幹上或咬或搔,樹枝擲中熊背,却絲毫傷牠們不得。過不多時,樹幹周圍已被群熊咬了一兩寸深,只須再咬一陣,群熊合力衝撞,這株百年大樹非斷不可。

張翠山嘆道︰「想不到我夫婦不死於大海,巴巴的飄到這裡,竟葬身於群熊之腹。」

殷素素見了樹下那十三頭大熊兇惡的形相,心中感到説不出的驚怖,望著七八丈外的一株大松樹,説道︰「五哥,你施展輕功,一躍到地,再一躍便可逃到那邉樹上。」這一節張翠山已想到,但自己一人固可逃生,要帶同殷素素却因相距太遠,勢有不能,中途必定被群熊截住。他搖了搖頭。説道︰「不成,跳不過去。」殷素素道︰「五哥,你不用管我,兩個人一齊死於非命,有什麼好?」張翠山道︰「咱們立過重誓,天上地下,永不分離。難道我捨得你一人遭難麼?」殷素素心中感激,泪珠在眼中滾動,待要勸他獨自逃生,喉嚨中哽住了説不出話來。便在此時,只覺樹身晃動,那大松樹在群熊衝撞之下,轉眼便要斷折。殷素素嚇得大聲尖叫起來。

叫聲未斷,只聽得遠處也傳來一陣尖鋭的叫聲,聲音不甚響,可是極爲古怪,似梟鳴、似彈筝、似風過竹葉、似金鐵交鳴。群熊聲到這一陣尖叫,立時簌簌發抖,好像聽到了天地間最可怖的聲音一般,一頭頭龐然大物委頓在地。張翠山和殷素素相顧一眼,都感好生奇怪。殷素素提起嗓子,叫道︰「救命,救命!惡熊要害人哪!」她叫喊聲中,遠處又有一聲尖叫相應,但聽那叫聲霎時之間從遠處到了身前,再快的飛鳥也未必有此迅捷,眼前紅影一晃,一團火球從對面的大樹上一躍而至,停在張殷二人處身的松樹幹上。兩人這時方才看得清楚,原來是一頭通身火紅的猿猴,約莫三尺來高,遍身長滿殷紅如血的長毛,一張臉却是雪白似玉,金光閃閃的眼珠骨碌碌地轉動,神情極是可愛。殷素素當聽到那尖鋭的叫聲之時,心中原是喜憂參半,見群熊聽到叫聲後如此害怕,想來發出叫聲的怪物定比白熊更爲兇猛悍惡,只是身處絶境,最壞也不過是一死,這纔又縱聲呼叫,把那怪物引來,豈知一見面竟是如此美麗的一頭靈猴,不由得大喜,臉露笑容,伸出手去。那玉面火猴甚具靈性,在這島上從未見過人類,但見張殷二人臉上無毛,相貌俊美,只當是同類到了,竟也伸手去撫摸一下殷素素的手。殷素素指了指樹下的白熊,説道︰「這些惡熊要咬我們,你能給咱們趕走麼?」

那玉面火猴靈異之極,雖然不懂她的説話,但見了她説話時所比的手勢,已然領悟,一聲清嘯,輕飄飄的縱下樹去,雙手抓住一頭白熊的頭頂一分,抓出了熊腦,又躍上樹來,棒到殷素素面前,顯是以異味饗客的神情。

張殷二人見牠一舉手便生裂熊頭,膂力之強,手爪之利,任何猛獸均無如此厲害,實是天地間罕見罕聞的神獸,心中大是駭異。殷素素實在不敢吃這熱氣騰騰的熊腦,但這時不敢得罪火猴,生怕惹惱了牠,只得接了過來,勉強吃了一口,將其餘的轉遞了給張翠山。那知這生熊腦入口,竟是鮮美軟滑,遠勝羊腦魚腦,又從張翠山手裡拿回一些來再吃,笑對火猴道︰「多謝!多謝!」

那火猴縱身下樹,頃刻間又生裂二熊,取出兩副熊腦,自己吃得津津有味。説也奇怪,群熊既不抗拒,亦不逃走,只是伏在地下發抖,聽任宰割。殷素素笑道︰「把這些惡熊都弄死了吧,若不是你來相救,這會咱二人都已成了熊腹中之物。」那火猴應聲而前,將餘下的十頭巨熊一一撕斃。張翠山和殷素素躍下樹來,這片刻間生死之隔只差一線,倘若來的不是這頭神猴,便是猛虎雄獅,見了這許多白熊也要遠遠走避,焉敢攖其兇燄?張翠山見十三頭巨熊屍橫就地,心中惻然生憫,説道︰「其實殺一儆百,將之驅走,也就是了,不必盡數置之死地。」殷素素正拉著火猴的手,和牠相處親熱。

\chapter{玉面火猴}

她聽得張翠山這麼説,心中一凜,暗想︰「五哥不喜我下手太狠,這脾氣以後認眞得改一改。」只中却笑道︰「這會児你却可憐起惡熊來,若不是這猴児兄弟來救,你説那些惡熊會可憐咱倆麼?」張翠山道︰「倘若咱們也跟野獸一般殘忍,那不是跟野獸没分别了麼?」殷素素笑道︰「野獸也有好的,你瞧這猴児兄弟,本事又比你大,相貌也比你俊。」張翠山笑道︰「啊喲,你不怕我呷唶?」

兩人大難不死,説説笑笑,心神倍覺歡暢。那玉面火猴在兩人身畔跳來跳去,也顯得歡喜無限,似乎牠獨居島上孤寂無侶,忽然得到了良伴一般。張翠山道︰「不知道白熊洞中是否還有小熊,咱們進去瞧瞧。」殷素素擕了火猴的手,倚牠爲護身之符,走進洞去。但見山洞極是寬敞,深入有八九丈遠,中間透入一線天光,宛似天窗一般。只是洞中白熊的屎尿狼藉,甚是穢臭。殷素素掩鼻道︰「此間好却是好,便是臭得没法容身。」張翠山道︰「只須日日打掃洗刷,過得十天半月,便不臭了。」殷素素想起從此要和他在這島上長相厮守,歳月無盡,以迄老死,心中又是歡喜,又是淒涼。

當下和張翠山折下樹枝,紮成一把大掃帚,將洞中群熊遺下的糞尿清掃出去,殷素素也幫著收拾。那玉面火猴雖然靈異,總是不脱猴児本性,東拉西爪,似是幫忙,却是搗亂。張殷二人感牠救命之恩,任由牠去胡鬧。待得打掃乾淨,穢氣仍是不除。殷素素道︰「附近若有溪水沖洗一番便好了,雖有海水,可惜没有盛水的提桶。」張翠山道︰「我有法子。」到山陰寒冷之處,搬了幾塊大冰,放在洞中的高岩上。殷素素拍掌叫道︰「好主意!」冰塊慢慢熔化成水,流出洞去,便似以水沖洗一般,只是大爲緩慢而已。

張翠山在洞中清洗,殷素素便用長劍剝切白熊,打或條塊,堆成個小丘一般。當地雖有火山,但究竟在極北,仍是十分寒冷,熊肉旁敦以冰塊,看來累月不腐。殷素素嘆道︰「人心苦不足,既得隴,便望蜀,咱們若有火種,燒烤一隻熊掌吃吃,那可有多美。」張翠山望著火山口噴出的火燄,道︰「火是有的,就可惜火太大了,慢慢想個法児,總能取它過來。」當晩兩人飽餐一頓熊腦,便在樹上安睡。睡夢中仍如身處大海中的冰山之上,隨著波浪起伏巓簸,其實却是風動樹枝。

次日殷素素還没有睜開眼來,便説︰「好香,好香!」翻身下樹,但覺得陣陣清香,竟是從熊洞中傳出。她和張翠山並肩進洞,只見洞中堆滿了嫣紅奼紫、大大小小,許多叫不出名目的花朶,那火猴竄高縱低,正在將花朶擲來擲去。殷素素生平最愛花草,陡然間見到這許多奇花,當眞是説不出的歡喜。張翠山道︰「素素,你且慢高興,有一件事跟你説。」殷素素見他臉色鄭重,心中一怔,道︰「什麼?」張翠山道︰「我想出了取火的法子。」殷素素笑道︰「啊,你這壞人,我還道是什麼不好的事呢。什麼法子?快説,快説!」

張翠山道︰「火口口火燄太大,無法走近,只怕走到數十丈外,人已烤焦了。我們用樹皮搓一條長繩,曬得乾了,然後\dash{}」殷素素拍手道︰「好法子!然後繩上縛一塊石子,向火山口抛去,火燄燒著繩子,便引了下來。」兩人生食已久,急欲得火,當下説做便做,以整整兩天時光,搓了一條百餘丈長的繩子,又曬了一天,第四天上便向火山口進發。

那火山口望去不遠,走起來却有四十餘里。兩人越走越熱,先脱去了海豹皮的皮裘,到後來連只穿單衫也有些頂受不住,又行里許,兩人口乾舌燥,遍身大汗,但見身旁已無一株花草,只餘光禿禿、黃焦焦的岩石。

張翠山肩上負著長繩,一瞥眼見殷素素幾根長髮的髮脚,因受熱而鬈曲起來,心下憐惜,説道︰「你在這裡等我,待我獨自上去吧。」殷素素嗔道︰「你再説這些話,我可從此不理你啦。最多咱們一輩子没火種,一輩子吃生肉,又有什麼大不了的?」張翠山微微一笑,又走里許,兩人都是氣喘如牛。張翠山雖然内功精湛,也已給蒸得眼前金星亂冒,頭腦中{\upstsl{嗡}}{\upstsl{嗡}}作聲,説道︰「好,咱們便在這裡將繩子擲了上去,若是接不上火種,那就\dash{}那就\dash{}」殷素素笑道︰「那就是老天爺叫咱們做一對茹毛飲血的野人夫妻\dash{}」説到這裡,身子一晃,險險暈倒,急忙抓住張翠山的肩頭,這纔站穩。張翠山從地下撿起一塊石子,縛在長繩一端,提氣向前奔出數丈,喝一聲︰「去!」使力擲了出去。

但見石去如矢,將那長繩拉得筆直,遠遠的落了下去。可是百餘丈外雖比張殷二人立足處又熱了些,仍是距火山口極遠,未必便能點繩端。兩人等了良久,只熱得眼中如要爆出火來,那長繩却是連煙也没冒半點。張翠山嘆了口氣道︰「古人鑽木取火,擊石取火,都是有的,咱們回去慢慢再試吧!這個擲繩取火的法子可不管用。」

殷素素灰心之下,站起身來,正要招呼那玉面火猴回去,却見牠在地下撿起石塊,學著張翠山的模樣,奔跑一程,擲一塊石子,玩得興高采烈,絲毫没有怕熱的樣子。殷素素心念一動︰「這火猴天生異稟,或許並不怕火。」於是撮唇一嘯,説道︰「猴児兄弟,你能不能將繩子拿上去,點燃了拿下來?」一面説,一面做著手勢比劃。

她只比了三遍,那火猴已然領會,弓身一躍,幾個起落,已竄出百餘丈外,拾起繩頭,向著火山口疾奔,遠遠望去似一個火球向上滾動,實是迅捷無倫。張殷二人心中都有些懊悔,生怕牠去得太快,累得牠送得性命。殷素素望見那火猴奔得距火山口已只數十丈,忙縱聲叫道︰「猴児,猴児,快回來!」

語聲甫畢,但見一縷青煙從繩頭裊裊升起,長繩竟已燃著。那火猴拉著長繩回轉,倏來倏去,前後還不到一盞茶的時分。殷素素大喜,迎上去將火猴抱在懷裡。殷素素擕著七八個乾柴紮成的火炬,以備接火之用,當即在長繩的火頭上點著了。兩人看火猴時,但見牠身上片毛不焦,眞是神物。

當下兩人一猴,喜氣洋洋的回到熊洞。殷素素堆積柴草,生起火來。世上任何野獸見火無一懼怕,這火猴却不愧以火爲名,頑皮起來,竟跳到火堆中打了幾個滾。張翠山見了這等異狀,忽然想起師父曾説過的一件事來,只中「啊」了一聲。殷素素道︰「怎麼?」張翠山道︰「我曾聽師父説道,有一種老鼠叫做火鼠,入火不焚,毛長寸許,可織以爲布,稱爲火浣布。這種布若是髒了,用水洗不乾淨,須得投在火中一燒,當即潔白如新。看來這猴児兄弟跟火鼠是差不多的了。」殷素素笑道︰「幾時猴児兄弟落下毛來,我也給你織一件浣火布的衣服,不過你可得壽長些纔好,等他兩三百年,那就差不多啦。」

既有火種,一切全好辦了,熔冰成水,烤肉爲炙,兩人自船破浮海,從未吃過一頓熱食,這時第一口咬到脂香四溢的熊掌時,眞是險些連自己的舌頭也吞下肚去了。那火猴除了熊腦之外,不吃肉食,自行去採野果來吃。

當晩熊洞之中,花香流動,火光映壁,兩人結成夫妻以來,至此方始眞正享到洞房春暖之樂。

次日清晨,張翠山走出洞來,正自心曠神怡,驀地裡見遠處海邉岩石之上,站著一個高大的人影。

這人却不是謝遜是誰?張翠山這一驚當眞是非同小可,實指望和殷素素經歷一番大難之後,在島上便此安居,那知又闖來了這個魔頭。當下他一個人便如變成石像,呆立著動也不敢稍動。但見謝遜脚步蹣跚,搖搖晃晃向内陸走來,顯是他眼瞎之後,無法捕魚獵豹,一直餓到如今。他走出數丈,終於支持不住,脚下一個踉蹌,向前摔倒,直挺挺的伏在地下。

張翠山返身入洞,殷素素嬌聲道︰「五哥\dash{}你\dash{}」但見他臉色鄭重,話到口邉又忍住了。張翠山道︰「那謝的也來啦!」殷素素嚇了一跳,低聲道︰「他瞧見你了嗎?」隨即想起謝遜眼睛已瞎,驚惶之意稍減,説道︰「咱們兩個亮眼之人,不能對付不了一個瞎子。何況還有猴児兄弟相助。」張翠山點了點頭,道︰「他餓得暈了過去啦。」殷素素道︰「咱瞧瞧去!」從衣袖上撕下四根布條,在張翠山耳中塞了兩條,自己耳中塞了兩條,右手拿著長劍,左手擕著火猴,一同走出洞去。

兩人走到離謝遜七八丈處,張翠山朗聲道︰「謝前輩,你可要吃些食物?」謝遜斗然間聽到人聲,臉上露出驚喜之色,但隨即辨出是張翠山的聲音,臉上又罩了一層陰影,隔了良久,纔點了點頭。張翠山拿了一大塊昨晩吃剩下來的熟熊肉,遠遠擲了過去,説道︰「請接著。」謝遜撐起身子,聽風辨物,伸手抓住,慢慢的咬了一口。張翠山見他本來生龍活虎般的一條大漢,竟給飢餓折磨得如此衰弱,不禁油然而起憐憫之情。殷素素心中却又是另一個念頭︰「五哥也忒煞濫好人,讓他餓死了,豈不乾手淨脚?這番救活了他,日後只怕麻煩無窮,説不定我兩人的性命還是得送在他的手下。」但想起自己立過重誓,決意跟著張翠山做好人,心中雖起不必救人之念,却不説出口來。

謝遜吃了半塊熊肉,不再吃了,伏在地下呼呼睡去,張翠山在他身旁生了一個火堆,一來免他受寒,二來得以烤乾濕衣。謝遜睡到午後,這纔醒轉,問道︰「這是什麼地方?」張殷二人守在他的身旁,見他坐起開口,便各取出塞在耳中的布條,以便聽他説些什麼,但兩人的右手都離耳畔不過數寸,只要一見情勢不對,立即伸手塞耳。張翠山道︰「這極北之處一個無人荒島。」謝遜「{\upstsl{嗯}}」了一聲,霎時之間,心中興起了數不盡的念頭,呆了半晌,説道︰「如此説來,咱們是回不去了!」張翠山道︰「那得瞧老天爺的意旨了。」謝遜破口大罵道︰「什麼老天爺,狗天、賊天、強盜老天!」他這一張口咒罵,竟是老半天不停,直到他罵得自己也累了,這纔摸索著坐在一塊石上,又咬起熊肉來,問道︰「以後你們要拿我怎樣?」

張翠山望著殷素素,似要她開口説話。殷素素却打個手勢,意思説一切憑你的主意。張翠山微一沉吟,朗聲道︰「謝前輩,咱夫妻倆\dash{}」謝遜點頭道︰「{\upstsl{嗯}},成了夫妻啦。」殷素素臉上一紅,却頗有得意之色,説道︰「那也可説是你做的媒人,須得多謝你撮成。」謝遜哼了一聲,道︰「你夫妻倆怎麼樣?」張翠山道︰「咱們射瞎了你的眼睛,自是十分的過意不去,不過事已如此,再説一萬遍致歉也是無用。既是天意要讓咱們共處孤島,説不定這一輩子再也難以回中土,咱倆便好好的奉養你一輩子。」

謝遜點了點頭,嘆道︰「那也只好如此。」張翠山道︰「咱夫妻倆情深義重,同生共死,謝前輩若是狂病再發,害了咱倆任誰一人,另一人決然不忍獨活。」謝遜道︰「你是要跟我説,你兩人若是死了,我瞎了眼睛,在這荒島上也是活不成?」張翠山道︰「一點不錯。」謝遜道︰「既是如此,你們耳中何必再塞著布片?」

張翠山和殷素素相視而笑,將耳中的布條也都取了出來,心下却均駭然︰「此人眼睛雖瞎,耳音之靈,幾乎到了能以耳代目的地步。倘若不是在此事事希奇古怪的極北島上,他未必須靠我二人供養。」

張翠山以謝遜學識淵博,請他替這荒島取個名字。謝遜道︰「這島山既有萬載玄冰,又有終古不滅的火窟,便稱之爲冰火島吧。」自此三人一猴,便在島上安居下來,倒也相安無事,張殷二人一有空閒,便在熊洞左近種植花木,燒陶作碗,堆土爲灶,各種日用物品,次第粗具。謝遜也從不來和兩人囉唆,只是捧著那把屠龍寶刀,低頭冥思。張殷二人有時見他可憐,勸他不必再苦思屠龍寶刀中所藏的祕密。謝遜道︰「我豈不知便是尋到了刀中祕密,在這荒島之上又有何用?只是無所事事,何以遣此漫漫長夜?」兩人聽他説得有理,也就不再相勸。

離熊洞半里之處,另有一個較小山洞。張翠山花了十來天功夫,將之佈置成爲一間居室,供謝遜居住。忽忽數月,有一日,他夫婦倆擕手向島北漫遊,原來這島方圓極廣,延伸至北,不知盡頭,走出百餘里地,只見一片濃密的叢林,老樹參天,陰森森的遮天蔽日。張翠山有意進林一探,但那玉面火猴{\upstsl{喳}}{\upstsl{喳}}的説個不停,只是搖頭,似乎林中有什麼連牠也懼怕的物事。殷素素膽怯起來,説道︰「連猴児兄弟也怕,咱們别去惹禍了。」

張翠山微覺奇怪,心想︰「素素向來好事,怎地近來懶洋洋的,什麼事也提不起興緻來?」想到此處,心中一驚,道︰「素素,你身子好嗎?可有什麼不舒服?」殷素素突然間羞得滿臉通紅,低聲道︰「没什麼?」張翠山見他神情奇特,連連追問。殷素素似笑非笑的道︰「老天爺見咱們太過寂寞,再派一個人來,要讓大夥児熱鬧熱鬧。」張翠山一怔之下,大喜過望,道︰「你有孩子啦?」殷素素忙道︰「小聲些,别讓人家聽見了。」她説了這句話,忍不住{\upstsl{噗}}{\upstsl{哧}}一聲,笑了出去。荒林寂寂,那裡還有第三個人在?

天候嬗變,這時日漸短而夜漸長,到後來每日只有兩個多時辰是白天,氣候也轉得極其寒冷。殷素素有了身孕後甚感疲懶,但一切烹飪、縫補等務,仍須勉力而行。這一晩她十月懷胎將滿,熊洞中生了火,夫妻倆偎倚在一起閒談。殷素素道︰「五哥,你説咱們生個男孩呢還是女孩?」張翠山道︰「女孩像你,男孩像我,男女都很好。」殷素素道︰「不,我喜歡是個男孩子。你給他先取定個名字吧!」張翠山道︰「{\upstsl{嗯}}。」隔了良久,却不言語。殷素素道︰「五哥,這幾天你有什麼心事?我瞧你心不在焉似的。」張翠山道︰「没什麼。想是要做爸爸了,所以喜歡得胡裡胡塗啦!」

他説這幾句話時,本是玩笑之言,但眉間眼角,隱隱帶有憂色。殷素素何等聰明,如何瞧不出來,柔聲道︰「五哥,你若是瞞著我,只有更增我的憂心。你瞧出什麼事不對了?」張翠山嘆了口氣,道︰「但願是我瞎疑心。我瞧謝前輩這幾天的神色有些不正。」殷素素「啊」的一聲,道︰「我也早見到了。他臉上的神色,越來越兇狠,似乎又要發狂。」張翠山點了點頭,道︰「想是他琢磨不出屠龍刀中的祕密,因此心中煩惱。」殷素素泪水盈盈,説道︰「本來咱倆拚著跟他同歸於盡,那也没什麼。但是\dash{}但是\dash{}」

張翠山摟著她的肩膀,安慰道︰「你説的不錯,咱們有了孩子,不能再跟他拚命。他好好的便罷,若是再行兇作惡,咱們只得給他殺了。諒他瞎著雙眼,終究奈何咱們不得。」

殷素素自從懷了孩子,不知怎的,突然變得仁善起來,從前做閨女時一口氣殺幾十個人也不貶眼睛,這時變便是殺頭野獸,也覺不忍。有一次張翠山捕了一頭母鹿,兩頭小鹿一直跟到熊洞來,殷素素一定要他將母鹿放了,寧可大家吃些野果,挨過兩天。這時聽到張翠山説要殺了謝遜,不禁身子一顫。

她偎倚在張翠山懷裡,這麼微微一顫,張翠山登時便覺察了,溫柔地一笑,説道︰「但願他不發狂。素素,我們的孩子叫作『念慈』,你説可好?讓他大了之後,一直記得媽媽這時候仁善慈悲的心腸,是男孩也好女孩也好,都叫這個名字。」殷素素點了點頭,心中很感舒暢,道︰「從前,我每殺了一個人,總算是覺得很高興,但這時想來,心頭起了個仁慈的念頭時,却比殺人更加歡喜些。只是我從前不會慈悲,那也無從比較起。咦,你又在想什麼啦?」張翠山道︰「害人之心不可有,防人之心不可無。」殷素素道︰「不錯,倘若他眞的發起狂來,却怎生制他?咱們有猴児兄弟作幫手,跟在冰山上時是大不相同了。」

張翠山道︰「火猴雖然靈異,但牠也未必能全懂咱們的説話,緩急之際,未必可靠,須得另想法子。」殷素素道︰「咱們給他進食物時做些手脚,看能找到什麼毒物\dash{}不,不,他不一定會發狂的,説不定咱倆瞎疑心。」張翠山道︰「我有一個計較。咱倆從明児起,移到内洞去住,却在外洞中掘一個極深的坑道,上面舖以皮毛軟泥。」殷素素道︰「這法子好却是好,不過你每日要出外打獵,若是他在外面行兇\dash{}」張翠山道︰「我一個人容易逃走,一見情勢不對,便往危崖峭壁上竄去,他瞎了雙眼,如何追得我上。」

第二日一早,張翠山便在外洞中挖掘深坑,只是没有鐵鏟鋤頭,只得以天生的樹枝當作木扒,實在是事倍功半。好在他内力渾厚,辛苦了七天,已挖了三丈來深,眼看謝遜的神氣越來越是不對,時時拿著屠龍刀狂揮狂舞。張翠山加緊挖掘,預備挖到五丈深時,便在坑洞底周圍插上削尖的木棒。這深坑底窄口廣,他不進來侵犯殷素素便罷,只要踏進熊洞,非摔落去不可。

這一日午餐之後,謝遜只在熊洞外數丈處來回徘徊。張翠山不敢動工,生怕他聽見響聲,起了疑心,可是又不敢出外打獵,只守在一旁,瞧著他動靜。但聽得謝遜不住口的咒罵,從老天爺罵起,一直罵到西方佛祖,東海觀音,天上玉皇,地下閻羅,再自三皇五帝罵起,堯舜禹湯,秦皇唐宗,文則孔孟,武則関岳,不論那一個大聖賢大英雄,全給他罵了個狗血淋頭。謝遜學問淵博,精通史事,這一番咒罵,張翠山倒是怔怔的給聽得甚有興味。

突然之間,謝遜破口大罵起武林人物來,這一次自華陀創設五禽之戲起始,少林派達摩老祖,岳武穆神拳散手,全給他罵得一錢不値。可是他倒也不是一味謾罵,於每一家每一派的缺點所在,却也確有眞知炙見,一貶一斥,往往一針見血。只聽他自唐而宋,逐步罵到了南宋末年的東邪西毒、南帝北丐中神通,罵到了郭靖楊過,猛地裡辭鋒一轉,罵起武當派開山祖師張三丰來。

他辱罵旁人,那也罷了,這時大罵張三丰,張翠山如何不怒?正要反唇相稽,謝遜突然大吼︰「張三丰不是東西,他的徒弟張翠山更加不是東西,讓我捏死他的老婆再説!」縱身一躍,掠過張翠山身旁,奔進熊洞。張翠山急忙跟進,只聽得喀的一聲,謝遜已跌入坑中,可是坑底未裝尖刺,他雖然摔下,並没受傷,只是出其不意,大吃了一驚。張翠山順手抓過挖土的樹枝,只見謝遜從坑中竄了上來,兜頭便是一下猛擊下去。謝遜聽風聲,左手翻轉,已抓住了樹枝,用力向裡一奪。張翠山把捏不定,樹枝脱手,這一奪勁力好大,他虎口震裂,掌心也給樹枝擦得滿是鮮血,謝遜跟著這一奪之勢,又墜入了坑底。其時殷素素即將臨盆,早已腹痛了半日,她先前見謝遜逗留在洞口不去,不敢和丈夫説知此事。因若是給謝遜聽到了,他想自己動彈不得,少了一層顧忌,更易及早發難。這時見張翠山和他動手,一根樹枝又被奪去,情勢危急之中,顧不得腹痛如絞,抓起枕頭邉的長劍,向張翠山擲了過去。張翠山抓住劍柄,暗想︰「此人武功高我十倍,他再竄上來時,我出劍劈刺,仍是非被他奪去不可。」情急之下,突然動道︰「他雙目已盲,所以能奪我兵刃,全仗聽著我兵刃劈風之聲,才知我的招勢去向。」

剛想到此節,只見謝遜哈哈一笑,又提氣縱躍而上。張翠山看準他竄上的來路,以劍尖對住他的腦門,緊握不動,只聽得擦的一聲響,謝遜一聲大吼,長劍已刺入他的額頭,深入數寸。原來張翠山持劍不動,謝遜這一躍上勢道極猛,正是以自己腦門硬碰到劍尖上去,長劍既然紋絲不動,絶無聲息,謝遜武功再好,如何能彀知曉?總算他應變奇速,劍尖一碰到頂門,立即將頭向後一仰,同時急使「千斤{\upstsl{墬}}」功夫,再行落入坑底。只要他變招遲得一霎之間,那長劍從腦門中直刺進去,立時便即斃命。饒是如此,頭上也已重傷,血流披面,長劍刺在他額頭之上,不住顫動。謝遜拔出長劍,撕下衣襟裹住創口,頭腦中一陣暈眩,自知受傷不輕,可是他狂性已發,從腰間拔出屠龍刀來,急速舞動,護住了頂門,第三度躍上。張翠山舉起大石,對準他一塊塊投去,却均被屠龍刀碰開。但見刀花如雪,寒光閃閃,謝遜飛出深坑,直欺過來。張翠山一步步向後退避,心中一酸,想起今日和殷素素同時畢命,竟是不能見一眼那未出世的孩児。

謝遜防他和殷素素從自己身旁逸出,一出熊洞,那便追趕不上,當下右手寶刀,左手長劍,使動大開大闔的招數,兩丈方圓之内,盡數封住,料想張殷二人再也無法逃走,瘋狂的心中大喜無已。驀地裡「哇」的一聲,内洞中傳出一響嬰児的哭聲。謝遜大吃一驚,立時停步,只聽那嬰児不住啼哭。張翠山和殷素素知道大難臨頭,竟是一眼也不再去瞧謝遜,兩對眼睛一齊愛憐橫溢地瞧著這個初生的嬰児,那是一個男孩,手足不住扭動,大聲哭喊。張殷二人知道只要謝遜一刀下來,夫妻倆連著嬰児便同時送命。二人一句話不説,目光不肯斜開一斜,能彀看得一霎,便是多享一分福氣。

夫妻倆已是心滿意足,終於,在臨死之前的一刻,能彀看到了和自己血肉相連的孩子。他們已不去想自己的命運。能彀保護嬰児不死,自是最好,但他們明知道這是不可能的,因此竟連這個念頭也没有轉。嬰児在大聲哭嚷著,這哭聲使謝遜突然間心中良知激發,狂性登去,頭腦便清醒過來。他想起自己全家被害之時,妻子剛正生了孩子不久,但那嬰児終於也是難逃敵人毒手。這幾聲嬰児的啼哭,使他回憶起許許多多的往事︰夫妻的恩愛,敵人的兇殘,無辜嬰児被敵人摔在地上成爲一團血肉模糊,自己苦心孤詣還是無法報仇,自己武功日進,那知仇人進展更快,雖然得了屠龍刀刀中的祕密却總是不能査明。他呆呆立著出神,一時溫顏微笑,一時咬牙切齒。

在這一瞬之前,三人都是面臨著最重大的生死関頭,但自嬰児的第一聲啼哭起,三個人突然全神貫注於身上。謝遜問道︰「是男孩呢還是女孩?」張翠山道︰「是個男孩。」謝遜道︰「剪了臍帶没有?」張翠山道︰「要剪臍帶嗎?啊,是的,是的,我倒忘了。」謝遜倒轉長劍,將劍柄遞了過去。張翠山接過長劍,割斷了嬰児的臍帶,這時方始想起,謝遜已是迫近身邉,可是他居然並不動手,心中好奇,回頭向他望了一眼,只見謝遜臉上充滿関切之情,竟似要插手相助一般。殷素素聲音微弱,道︰「讓我抱一抱。」張翠山抱起嬰児,送入她的懷抱。謝遜又道︰「你有没有燒了熱水,給嬰児洗一個澡?」張翠山失聲一笑,道︰「我眞胡塗啦,什麼也不給預備,這個爸爸可没用之極。」説著便要奔出去燒水,但只邁出一步,見到謝遜鐵塔一般巨大的身形竟在嬰児之前,心下驀地一凜。謝遜却道︰「你陪著夫人孩子,我去燒水。」將屠龍刀往腰間一插,便奔出洞去,經過深坑時輕輕縱身一躍,橫越而過。

過了一陣,謝遜果眞用陶盆端了一盆熱水進來,張翠山便替嬰児洗澡。謝遜聽得嬰児的哭聲甚是洪亮,問道︰「這孩子像媽媽呢還是像爸爸?」張翠山微笑道︰「還是像媽媽多些,多福多壽,少受苦難。」殷素素道︰「謝前輩,你説孩子的長相不好麼?」謝遜道︰「不是的。只是孩子像你,那就太過俊美,只怕福澤不厚,將來成人後入世,或會多遭災危。」張翠山笑道︰「謝前輩想得太遠了,咱們四個人處身極北荒島,這孩子自也是終老難道也讓他孤苦伶仃的一輩子在這島上?百年之後我三人都死了,誰來伴他?他長大之後,如何娶妻生子?」

殷素素自幼稟受父性,在白眉教中耳濡目染,所見所聞皆是極盡殘酷惡毒之事,因之她向來行事狠辣,習以爲常,自與張翠山結成夫妻,逐步向善,這一日做了母親,心中天生的慈愛沛然而生,竟是全心全意爲孩子打算起來。殷素素向她淒然望了一眼,伸手撫摸她的頭髮,心道︰「這荒島與中土相距萬里,却如何能彀回去?」但不忍傷愛妻之心,此言並不出口。忽聽得謝遜説道︰「張夫人的話不錯。咱們這一輩子算是完了,但這孩子,這孩子,如何能彀使他老死在荒島之上,享不到半點人世的歡樂?張夫人,咱三人終當窮智竭力,使孩子得歸中土。」殷素素大喜,顫巍巍的站起身來。張翠山忙伸手相扶,驚道︰「素素,你幹什麼?快好好躺著。」殷素素道︰「不,五哥!咱倆一起給謝前輩磕幾個頭,感謝他這一番大恩大德。」謝遜連連搖手,説道︰「不用,不用。這孩子取了名没有?」張翠山道︰「在下胡亂給他安了個名字!叫作『念慈』。謝前輩學問淵博,另行給他取個好名字吧!」

謝遜沉吟道︰「張念慈,張念慈!這名字好啊,不用改了。」殷素素忽然想起︰「難得這怪人如此喜愛這個孩児,他若將孩児視若己子,那麼孩児在這島上就不再悉他加害,縱然他狂性發作,也必不致驟下毒手。」説道︰「謝前輩,我爲這孩児求你一件事,務懇不要推却。」謝遜道︰「什麼?」殷素素道︰「你收了念慈孩児做義子吧!讓他長大了,對你當親生父親一般供養。得你照料,這孩児一生不吃人家的虧。五哥,你説好不好?」張翠山道︰「妙極,妙極!謝前輩,還請你不棄,俯允咱夫婦的請求。」

\chapter{重返中土}

謝遜淒然道︰「我自己的親生孩児給人一把摔死了,幾成了血肉糢糊的一團,你們没有瞧見吧?」張翠山和殷素素對望一眼,覺得他言語之中又有瘋意,但想起他的慘酷遭際,不由得甚是惻然。謝遜又道︰「我這孩子如果不死,今年有十八歳了。我謝遜將一身的文才武功傳授於他,嘿嘿,他未必便及不上你們什麼武當七俠,少林三義。」這幾句内淒涼之中帶著狂傲,但自負之中又包含著無限寂寞傷心。張翠山和殷素素不覺都是油然而起悔心︰「倘若當日在冰山上不毀了他的雙目,咱們四個人在此荒島隱居,融融洩洩,豈不是好?」

三個人默然半晌,張翠山道︰「謝前輩,你收這孩作爲義子,咱們叫他改宗姓謝。」謝遜臉上閃過一絲喜悦之色,説道︰「你肯讓他姓謝?謝念慈,謝念慈,這名字很好啊,不過我那個死去的孩児,名叫謝無忌。」張翠山道︰「假如你喜歡,那麼,咱們這孩児便叫作謝無忌。」謝遜喜出望外,唯恐張翠山是騙他的,道︰「你們把親生孩児給了我,那麼你們自己呢?」張翠山道︰「孩児不論謝姓張,咱們是一般的愛的。日後他孝順雙親,敬愛義父,不分親疏厚薄,豈非美事?素素,你説可好?」殷素素微一遲疑,道︰「你説怎麼便是怎麼。孩子多得一個人疼愛,終是便宜了他。」謝遜一揖到地,説道︰「這我可謝謝你們啦,毀目之恨,咱們一筆勾銷。謝遜雖喪子而有子,將來謝無忌名揚天下,好教世人得知,他父母是張翠山、殷素素,他義父是金毛獅王謝遜。」

殷素素當時所以稍一猶疑,乃是想起眞的謝無忌已死,被人摔作了一團肉漿,自己的孩児頂用這個名字,未免不吉,然見謝遜如此大喜若狂,料想對這孩児必極疼愛,孩児定可得到他許多好處,母親愛子之心無微不至,只須於孩子有益,一切全肯犧牲,抱了孩児,説道︰「你要抱一抱他嗎?」

謝遜伸出雙手,將孩子抱在臂中,不由得喜極而泣,雙臂發顫,説道︰「你\dash{}你快快抱回去,我這模樣别嚇壞了他。」其實初生一天的嬰児懂得什麼,但他這般説,顯是愛極了孩子。殷素素微笑道︰「你喜歡便多抱一會。將來孩子大了,你帶著他到處玩児吧。」謝遜道︰「好極,好極\dash{}」聽得孩子哭得極響,道︰「孩子餓了,你餵他吃奶吧!我到外邉去。」實則他雙目已盲,殷素素便是當著他餵乳,也没什麼,但他發狂時欲圖非禮,這時却文質彬彬,竟變成了個儒雅君子。

張翠山道︰「謝前輩\dash{}」謝遜道︰「不,咱們已成一家人,再這般前輩後輩的,豈不生分?我這麼説,咱三人索性結義爲金蘭兄弟,日後於孩子也好啊。」張翠山道︰「你是前輩高人,咱夫婦跟你身份相差太遠,如何高攀得上。」謝遜道︰「{\upstsl{呸}},你是學武之人,却也這般迂腐起來?五弟,素妹,你們叫我大哥不叫?」殷素素笑道︰「我先叫你大哥,咱倆是拜把子的兄妹。他若再叫你前輩,我也成了前輩啦!」張翠山道︰「既是如此,小弟唯大哥之命是從。」殷素素道︰「咱們先就這麼説定,過幾天等我起得身了,再來祭告天地,行拜義父、拜義兄之禮。」謝遜哈哈大笑,説道︰「大丈夫一言既出,終身不渝,又何必祭天拜地?這賊老天自己管不了自己的事,我謝遜最是恨他不過。」説著揚長出洞,只聽得他在曠野上縱聲大笑,顯是得意之極,張殷二人自從識得他以來,從未見過他如此喜歡。

自此三人全心全意撫養孩子。謝遜號稱「金毛獅王」,馴獸捕生之後,天下無雙,張翠山詳述島上多處地形,謝遜在他指引下走了一遍,便即記住。

自此捕鹿殺熊,便由謝遜一力承擔,有時那玉面火猴也陪同他出獵。只是那火猴殺熊太過輕易,眞是不費吹灰之力,謝遜反覺没趣,初時尚要火猴引路,日子一久,他處處路徑都已記熟,便要火猴陪孩子玩耍,不許牠同去打獵。

忽忽數年,三個人在島上相安無事,那孩子百病不生,長得甚是壯健。三人中倒似謝遜對他最是疼愛,有時孩子太過頑皮,張翠山和殷素素要加責打,每次都是謝遜從中攔住。如此數次,孩子便恃他作爲靠山,逢到父母發怒,總是奔到義父處求救。張殷二人往往搖頭苦笑,説孩子被大哥寵壞了。

到無忌四歳時,殷素素開始教他識字。五歳生日那天,張翠山道︰「大哥,孩子可以學武啦,從今天起你來教他,好不好?」謝遜搖頭道︰「不成!我的武功太深,孩子無法領悟。還是你傳他武當心法。等他到八歳時,我來教他。教得兩年,你們便可回去啦!」殷素素奇道︰「你説我們可以回去?回中土去?」謝遜道︰「這幾年來我日日留心島上的風向水流,似乎每年黑夜最長之時,總是刮的南風,數十晝夜不停。咱們可以紮一個大木排,裝上風帆,不停的向南,要是賊老天不來橫加搗蛋,説不定你們便可回歸中土。」殷素素道︰「我們?難道你不一起去麼?」謝遜道︰「我瞎了雙眼,回到中土做什麼?」殷素素道︰「你便不去,咱們却決不容你獨自留著。孩子也不肯啊,没了義父,誰來疼他?」謝遜嘆道︰「我能疼他十年,已是足彀了。賊老天總是跟我搗亂,這孩子倘若陪我的時候太多,只怕賊老天遷怒於他,會有橫禍加身。」殷素素打了個寒噤,但想這是他隨口説説的事,也没放在心上。

張翠山傳授孩子的,都是紮基根的内功,心想孩子年幼,只須健體強身,便已足彀,在這荒島之上,決不會和誰動手打架。謝遜雖説過南歸中土的話,但他此後不再提起,看來也是一時興到之言,不能作準。到第八年上,謝遜果然要無忌跟他學練武功。傳授之時他没叫張殷二人旁觀,他夫婦便遵依武林中的嚴規,遠遠避開,對無忌的武功進境,也不加考査,信得過謝遜所授,定是高明異常的絶學。

島上無事可紀,日月去似流水,轉眼又是一年有餘。自無忌出世後,謝遜心靈有了寄託,再也不去理會那屠龍寶刀,那知有一晩張翠山偶爾失眠,半夜中出來散步,月光下只見謝遜盤膝坐在一塊巖石之上,手中捧著那柄屠龍寶刀,正自低頭沉思。張翠山吃了一驚,待要避開,謝遜已聽到他的脚步聲,説道︰「五弟,這『武林至尊,寶刀屠龍』八個字,看來終是虛妄。」張翠山走近身去,説道︰「武林中荒誕之説甚多。大哥這等聰明才智,如何對這寶刀之説,始終念念不忘?」謝遜道︰「你有所不知,我曾聽少林派的一位有道高僧空見大師説過此事。」

張翠山道︰「啊,空見大師。聽説他是少林掌門空聞大師的師兄啊,他逝世已久了。」謝遜點頭道︰「不錯,空見已經死了,是我打死的。」張翠山吃了一驚,心想江湖上有兩句話説道︰「少林神僧,見聞智性」,那是替當今少林派四位班輩最高的和尚空見、空聞、空智、空性四人而言,後來聽師父説空見大師得病逝世,想不到竟是謝遜打死的。

謝遜嘆了口氣,説道︰「空見這人傻得很,他竟是只挨我打,始終不肯還手,我打了他一十三拳,終是將他打死了。」張翠山心下更是駭然,心想︰「能挨得起大哥一拳一掌而不死的,已是一等一的武學高手,這位少林神僧竟能連挨他一十三拳,身子之堅,那是遠勝鐵石了。」

但見謝遜神色淒然,臉上頗有悔意,料想這事之中,定是隱藏著一件極大過節,他自與謝遜結義以來,八年中共處荒島,情好彌篤,但他對於這位義兄,敬重之中總是帶著三分懼意,生怕引得他憶及昔恨事,當下也不敢多問。却聽謝遜説道︰「我生平中心欽服之人,寥寥可數。便是尊師張眞人,我雖久仰其名,但無緣識荊,也只神交而已。這位空見大師,實是一位高僧。他武功上的名氣雖不及他師弟空智、空性,但依我之見,空智、空性兩位大師一定及不上他老人家。」

張翠山以往聽他暢論當世人物,大都不値一哂,只要能得他破口大罵,已算是第一流的人物,要他讚上一字,眞是難上加難,想不到他提到空見大師竟是如此欽遲,倒也頗出意料之外,説道︰「想是他老人家隱居寺中清修,少在江湖上走動,是以武學上的造詣少人知。」謝遜仰頭向天,呆呆出神,自言自語的道︰「可惜可惜,這樣一位武林中蓋世奇士,竟給我一十三拳活生生的打死了。他武功雖高,實是迂得厲害,倘若當時他還手跟我放對,我謝遜焉能活得到今日?」張翠山道︰「難道這位高僧的武功修爲,竟比大哥還要深湛麼?」謝遜道︰「我怎能跟他相比?他弟子的武功也比我高得多。」他説這句話時,臉上神情和語氣之中,竟是充滿了無比的怨毒。

張翠山大奇,心中微有不信,自忖恩師張三丰的武學世所罕有,但和謝遜相較,恐怕也只能勝他半籌,假若空見大師的弟子尚且高出謝遜甚多,豈不是連自己恩師也比下去了?但素知謝遜的名字中雖有個「遜」字,性子極是倨傲,如果那人的武功不是眞的遠勝於他,他決計不肯服輸。

謝遜似是猜中了他的心意,説道︰「你不信麼?好,你去叫無忌出來,我説一個故事給他聽。」張翠山心想三更半夜的,無忌早已睡熟,去叫醒他聽故事,對孩子實無益處,但既是大哥有命,却也不便違拗,於是回到熊洞,去叫醒了児子。無忌一聽説義父要講故事,大聲叫好,登時將殷素素也吵醒了。三個人一起出來,坐在謝遜身旁。

謝遜道︰「孩子,不久你就回歸中土\dash{}」無忌奇道︰「什麼回歸中土?」謝遜將手揮了揮,叫他不要打斷自己話頭,繼續道︰「若是咱們的大木排在海中沉了或是飄得無影無縱,那也罷了,什麼休要提起,但要是眞的能回中土,我跟你説,世上人心險惡,誰都不要相信。除了父母之外,誰都會存著害你的心思。就可惜我年輕時没人跟我説過這些話。唉,便是説了,當時我也不會相信。」

\qyh{}我在十歳那一年,因意外機緣,拜在一個武林中大大有名之人的門下學藝。我師父見我資質不差,對我青眼有加,將他絶藝可説是傾囊以授。我師生情若父子,五弟,當時我對師父的敬愛仰慕,大槩跟你對尊師没差分毫。我在二十三歳那年離開師門,不久便娶妻生子,一家人融融洩洩,過得極是快活。」

\qyh{}過了兩年,我師路過我故鄕,到我家來盤桓數日。我自是高興得了不得,全家竭誠款待,我師父空閒下來,又指點我的功夫。那知這位武林中的有德長者,竟是人面獸心,在七月十五那日酒後,忽對我妻施行強暴\dash{}」

張翠山和殷素素同時「啊」的一聲,師姦徒妻之事,武林中從所未聞,那可是天人共憤的大醜事。謝遜續道︰「我妻子自是不從,大聲呼救,我父親聞聲闖進房中,我師父一見事情敗露,一拳將我父親打死了,跟著又打死了我母親,將我甫滿週歳的児子謝無忌\dash{}」無忌聽他提到自己名字,奇道︰「謝無忌?」

張翠山斥道︰「别多口!聽義父説話。」謝遜道︰「是啊,我那親生孩児跟你名字一樣,也叫謝無忌。我師父抓起了他,將他摔成了血肉糢糊的一團。」謝無忌忍不住又問︰「義父,他\dash{}他還能活嗎?」謝遜道︰「不能活了,不能活了!」殷素素向児子搖了搖手,叫他不要再問。謝遜出神半晌,纔道︰「那時候我瞧見這等情景,嚇得呆了,心中一片迷惘,不知如何對付我這位生平最敬愛的恩師,突然他一拳打向我的胸口,我胡裡胡塗的也没想到抵擋,就此暈死過去。待得醒轉,我師父早已不知去向,但見滿屋都是死人,我父母妻児,弟妹僕役,全家一十三口,盡數斃於他的拳下。想是他以爲一拳已將我打死,没有再下毒手。」

\qyh{}我大病一場之後,苦練武功,五年後去找師父報仇。但我跟他武功相差太遠,所謂報仇,徒然自取甚辱。但這一十三條人命的血仇,如何能便此罷休?當時我週遊天下,遍訪明師,這一番苦心孤詣,總算有了著落,十年之間,我連得三位高人傳授,自覺功夫大進,又去找我師父。那知我功夫強了,他竟是強得比我更多,第二次報仇還是重傷而歸。」

\qyh{}於是我潛心苦思,專練『七傷拳』的内勁,三年之後,拳技大成,自忖已可和天下第一流的高手比肩!我師父倘若不是另有奇遇,決不能再是我敵手。第三次找上門去時,他家人却已遷離原處,再也找不到他的所在。我在江湖上到處打聽,始於探尋不到他的蹤跡,想是他爲了躱避這場大禍,在極荒僻之地隱居了起來。大地茫茫,却到何處去尋他?」

\qyh{}我憤激之下,便到處做案,奸淫擄掠,殺人放火,無所不爲。每做一件案子,便留下我師父的姓名!」

張翠山和殷素素一齊「啊」了一聲。謝遜道︰「你們知道我師父是誰了吧?」殷素素點頭道︰「{\upstsl{嗯}}!你是『混元霹靂手』成崑的弟子。」原來兩年之前武林中突然發生一場軒然大波,自遼東以至嶽南,半年之間,接連發生了三十餘件大案,許多成名的豪傑突然不明不白的被殺,而兇手必定留下「混元霹靂手」成崑的名字。被害之人不是一派的掌門,便是交遊極廣的老英雄,每一件案子都是牽連的人數極衆。只要發生這樣一件案子,武林中便要到處轟傳,何況接連是三十餘件。當時武當七俠奉了師父之命,盡數下山査詢,但竟是不得半點頭緒。衆人均知這是有人故意嫁於成崑。要知成崑聲名向來極佳,被害的人又有好幾個是他的知交好友,這些案子決計非他所爲。但要査知兇手是誰,自是非成崑身上著手不可,可是成崑這人近來忽然無影無蹤,誰也不知他到了何處。紛擾多時,這些案子還是不了了之。雖然想找兇手報仇的人成百成千,可是不知兇手是誰,人人均是徒呼負負。若不是謝遜今日説起,張翠山那裡猜得到其中的過節原委。謝遜道︰「我冒成崑之名殺人做案,那是要逼得他挺身而出,便算他始終龜縮,武林中千百人到處尋訪,總是比我一人之力強得多啊。」殷素素道︰「此計不錯,只不過這許多人無辜傷在你的手底,倒底也不知爲了何故,未免可憐。」謝遜道︰「難道我父母妻児給成崑害死,便不是無辜麼?便不可憐麼?我看你從前倒也爽快,嫁了五弟十年,却學得他這婆婆媽媽起來。」殷素素向丈夫望了一眼,微微一笑,説道︰「大哥,這些案子倏然而起,倏然而止,後來你終於找到了成崑麼?」謝遜道︰「没有找到,没有找到。後來我在洛陽見到了宋遠橋。」張翠山大吃一驚,道︰「我大師哥宋遠橋?」

謝遜道︰「不錯,是武當七俠之首的宋遠橋。我見做了這許多案子,江湖上已鬧得天翻地覆,但我師父混元霹靂手成崑\dash{}」謝無忌道︰「義父,他這樣壞,你還是叫他師父麼?」謝遜苦笑道︰「我從小叫慣了。再説,我的一大半武功總是他傳授的。他雖是個大壞蛋,我也不是好人,説不定我的爲非作歹,都是他教的。好也是他教,歹也是他教,我還是叫他師父。」張翠山心想︰「大哥一生遭遇慘酷,憤激之餘,行事不分是非。無忌聽了這些話記住心中,於他日後立身有害,過幾天倒要好好解説給他聽。」

只聽謝遜續道︰「我見師父不露面,心想非做一件驚天動地的大案,不足以激逼出來。今世武林之中,以少林、武當兩派爲尊,看來須得殺死一名少林或是武當派中第一流的人物,方能見效。那一日我在洛陽清虛觀外的牡丹園中,見到宋遠橋出手懲戒一名惡霸,見他武功很是了得,決意當晩便去將他殺了。」

張翠山聽到這裡,不由得慄然而懼,他明知宋遠橋結果並未爲謝遜所害,但想起當時情勢的凶險,仍是不免惴惴,須知謝遜的武功高出宋遠橋甚多,何況一個在明,一個在暗,苦是當眞下手,大師兄絶無倖理。殷素素也知宋遠橋未死,説道︰「大哥,想是你突然不忍加害無辜的旁人,要是你當眞殺了宋大俠,咱們這位張五俠早就跟你拚了命,再也不會成爲結義兄弟了。」

謝遜「哼」了一聲,道︰「那有什麼不忍的?若在今日,我瞧在五弟面上,自不會去跟武當派爲難。可是那時我又不識得五弟,别説是宋遠橋,便是五弟自己,只要給我見到了,還不是殺了再説。」謝無忌奇道︰「義父,你爲什麼要殺爹爹?」謝遜微笑道︰「我是説個比方啊,並不是眞的要殺你爹爹。」謝無忌道︰「嗷,原來如此!」這纔放心。

謝遜撫著他小頭上的頭髮,説道︰「賊老天雖有諸般不好,總算没讓我殺了宋遠橋,否則我愧對你爹爹,也不能再跟他結義爲兄弟了。」他停了片刻,續道︰「這天晩上我吃過晩飯,在客店中打坐養神。我心知宋遠橋既是武當七俠之首,武功上自有過人之處,若是一擊不中,給他逃了,或是只打得他身負重傷而不死,那麼我的行藏必致洩露,要逼出我師父來的計謀盡數落空,而且天下豪傑向我群起而攻,我謝遜便有三頭六臂,也是無法對敵啊。我一死不打緊,這場血海深仇,可從此無由得報了。」謝無忌突然道︰「義父,你眼睛看不見,等我大了,練好了武功,去替你報仇!」

他此言一出,謝遜和張翠山不約而同的霍地站起。謝遜雖然雙目無神,仍是凝視無忌,低沉著聲音道︰「無忌,你可眞有此心?」張翠山和殷素素心中都很焦急,他們雖然身處極北萬里之外的荒島,將來未必能彀重返中土,但武林中人素重信義,一諾之下,終身不渝,無忌要是答應謝遜報仇,那可是在肩頭挑上了一副萬斤重擔。以謝遜幾具通天澈地之能,尚自不能報仇,無忌這小小孩子若是信口答應了,豈非自陥絶境?

可是無忌年紀雖小,這種事情還是須得由他自決,親爲父母,也不能出主意,至於日後他長大成人,是否還記得孩童的話,那是將來之事了。不過張殷二人此時聽來,均覺此事雖然渺茫,總是隱隱覺得非同小可,説不定便関涉到無忌的一生禍福。

只聽無忌昂然道︰「義父,害你全家之人叫做混元霹靂手成崑,無忌記在心中,將來一定代你報仇,也將他全家殺死,殺得一個不留!」

張翠山怒喝︰「無忌你説什麼?一人作事一人當,他罪孼再大,也只一人之事,豈可累及無辜?」

無忌應道︰「是,爹爹!」嚇得不敢再説。謝遜却道︰「一個人死了,什麼都不知道了,那也没有什麼,倒是全家死光,剩著一個人孤零零的在世上,更是難受。當時我明白這個道理,反之是找我師父報仇。其實眞正的報仇,該當是將我師父全家害死,讓他獨個児活著,日日想著亡妻喪子之痛。」

張翠山聽得只是搖頭,但礙著他是大哥,不便駁斥,生怕他更説許多慘酷惡毒的言語,讓無忌記在心中,於是問道︰「你跟我大師兄這場比武後來如何了結?大師兄始終没跟咱們説起這件事,倒是奇怪。」謝遜道︰「宋遠橋壓根児就不知道,恐怕他連『金毛獅王謝遜』這六個字也從來没聽見過,因爲我後來没有去找他。」張翠山嘆了口氣,道︰「謝天謝地!」殷素素笑道︰「謝什麼賊老天、賊老地,謝一謝眼前這個謝遜大哥纔是眞的。」張翠山和無忌都笑了起來。

謝遜却並不笑,緩緩的道︰「那天晩上的情景,今日我還是記得清清楚楚,我坐在坑上,暗運眞氣,將那『七傷拳』又複習了幾遍。五弟,你從來没見過我的『七傷拳』,要不要見識見識?」張翠山還没回答,殷素素搶著道︰「那一定是神妙無比,威猛絶倫,大哥,你怎地不去找宋大俠了?」謝遜微微一笑,説道︰「你怕我試拳時傷了你老公麼?倘若這拳力不是收發由心,還算得是什麼『七傷拳』。」説著站起身來,走到一株大樹之旁,口中{\upstsl{吆}}喝一聲,宛似憑空打了個霹靂,猛響聲中,一拳打在樹幹之上。

以謝遜的功力而論,這一拳便是不將大樹打得斷爲兩截,也當拳頭深陥樹幹,那知他收回拳頭時,那大樹竟是絲毫不損,連樹皮也不破裂半點。殷素素心中難過︰「大哥在這島上一住九年,武功是全然抛棄了。我從來不見他練功,原也難怪。」怕他傷心,還是大聲喝了聲采。謝遜道︰「素妹,你這聲采喝得全不由衷,你只道我武功大不如前了,是不是?」殷素素道︰「在這極北荒島之上,來來去去便是四個親人,還練什麼功?」謝遜問道︰「五弟,你瞧出其中的奥妙麼?」張翠山道︰「我見大哥這一拳去勢十分剛猛,可是打在樹上,連樹葉也没一片晃動,這一點心中甚是不解。便是無忌去打一拳,也會搖動樹枝啊!」無忌叫道︰「我會!」奔過去在大樹上砰的一拳,果然樹枝亂晃,月光照映出來的影子,在地上顫動不已。張翠山夫婦見児子這一拳力道甚是強勁,心下甚喜,一齊瞧著謝遜,等他説明其中的道理。謝遜道︰「三天之後,樹葉便會萎黃跌落,七天之後,大樹全身枯槁。我這一拳已將大樹的脈絡從中震斷。」

張翠山和殷素素不勝駭異,但知謝遜素來不打誑語,此言自非虛假,謝遜取過手邉的屠龍刀,拔刀出鞘,喀的一聲,在大樹的樹幹中斜砍一刀,只聽得砰彭巨響,大樹的下半段向外跌落。謝遜收刀説道︰「你們瞧一瞧,我『七傷拳』的威力可還在麼?」

張翠山三人走過去細看大樹的斜剖面時,只見樹心中一條條通水的筋脈已大半震斷,有的扭曲,有的粉碎,有的斷爲數截,有的若斷若續,顯然他這一拳之中,又包含著數種不同的勁力。張殷二人大是歎服,張翠山道︰「大哥,你今日眞是叫小弟大開眼界。」謝遜忍不住得意之情,説道︰「我這一拳之中,共有七種不同的勁力,或剛猛,或陰柔,或剛中有柔,或柔中有剛,或橫出,或直送,又或是自外向内收縮。敵人抵擋了一種勁力,抵不住第二種,抵了第二種,我的第三種勁力休又如何對付。嘿嘿,『七傷拳』之名,便由此而來。五弟,你説這拳力是太毒辣了些吧?」

無忌道︰「義父,你把這『七傷拳』教了我好麼?」謝遜搖頭道︰「不成!」無忌好生失望,還想纏著他苦求。殷素素笑道︰「無忌你不是傻麼?你義父這種武功精妙深湛,若不是先具最上乘的内功,如何能練?」無忌道︰「那麼等我先練好了上乘的内功再説。」謝遜搖頭道︰「這『七傷拳』不練也罷!每人體内,均有陰陽二氣,金木水火土五行。心屬火、肺屬金、賢屬水、脾屬土、肝屬木,一練七傷,七者皆傷。這七傷拳的拳功每深一層,自身内臟便多受一次損害,所謂七傷,實則是先傷己,再傷敵,我若不是在練七傷拳時傷了心脈,也不致有時狂性大發,無法抑制了。」張翠山和殷素素此時方知,何以他這樣一位文武兼資的奇人,一到狂性發作,竟會行若禽獸。

謝遜又道︰「倘若我内力眞的渾厚堅實,到了空見大師、或是武當張眞人的地步,再來練這七傷拳,想來自己也可不受損傷,便有小損,亦無大礙。只是當年我報仇心切,連殺七人,纔從崆峒派手中奪得這本『七傷拳譜』的古抄本,拳譜一到手,立時便心急慌忙的練了起來,唯恐拳功未成而我師父已死,報不了仇。待得察覺内臟受了大損,已是無法挽救,當時我可没想到,崆峒派既然有此世代相傳的拳譜,却爲何無人會此拳功。素妹,我又貪圖這拳功發拳時聲勢喧赫,有極大的好處,你懂得其中道理吧?」殷素素微一沉吟,道︰「{\upstsl{嗯}},是不是跟你師門霹靂什麼的功夫差不多?」謝遜道︰「正是。我師父外號叫作『混元霹靂手』,掌含風雷,威力極是驚人。我找到他後如用這路七傷拳跟他對敵,他定當以爲我使的還是他親手所傳的武功,只要拳力一到了他身上,他再驚覺不對可已遲了。五弟,你别怪我用心尖刻,我師父外表橫魯,可實在是天下最工心計的毒辣之人。若不是以毒攻毒,這場大仇便無法得報\dash{}唉,枝枝節節的説了許多,還没説到空見大師。且説那一晩我運氣溫了三遍七傷拳功,便越牆出外,要去找宋遠橋。」

\qyh{}我一躍出牆外,身子尚未落地,突然覺得肩頭上被人輕輕一拍。我大吃一驚,以我當時功力,竟有人伸手拍到我身上而不及擋架,可説是從所未有之事。無忌,你想,這一拍雖輕,但若是他掌上施出勁力,豈不是我已受重傷?我當即回手一撈,反擊一拳,左足一落地,立即轉身,便在此時,我背心上又被人輕輕拍了一掌,同時背後一人嘆道︰『苦海無邉,回頭是岸。』」

無已覺得十分有趣,哈哈笑了出來,道︰「義父,這人跟你鬧著玩麼?」張翠山和殷素素却已猜到,説話之人定是那空見大師了。

謝遜續道︰「當時我只嚇得全身冰冷,如墜深淵,那人如此武功,要制我死命眞是易如反掌。他説那『苦海無邉,回頭是岸』這八個字,只是一瞬之間的事,可是這八個字他説得不徐不疾,充滿慈悲心腸!我聽得清清楚楚。但那時我心中只感到驚懼憤怒,回過身來,只見四丈以外站著一位白衣僧人。我轉身之時,只道他離開我只不過兩三尺,那知他一拍之下,立即飄出四丈,身法之外,步伐之輕,實是匪夷所思。」

\qyh{}當時心中只有一個念頭︰『是冤鬼,給我殺了的人索命來著!』因爲我想若活人,決不能有這般來去如電的功夫。我一想到是鬼,膽子反而大了起來,喝道︰『妖魔鬼怪,給我滾得遠遠的,老子天不怕地不怕,豈怕你這種孤魂野鬼?』那白衣僧人合什道︰『謝居士,老僧空見合什!』我一聽到空見兩字,便想起江湖上所傳『少林神僧,見聞智性』這兩句話來。他名列四大神僧,無怪有這般高強的武功。」

\chapter{不堪回首}

張翠山想起這位空見大師後來是被他一十三拳打死,聽到這裡,已是隱隱不安。謝遜續道︰「當時我便回問道︰『是少林寺的空見神僧?』那白衣僧人道︰『神僧二字,愧不敢當。老衲正是少林空見。』我道︰『在下跟大師素不相識,何故相戲?』空見説道︰『老衲豈敢戲弄居士?請問居士,此刻欲往何處?』我道︰『我到何處去,跟大師有何干係?』空見道︰『居士今晩想去殺害武當派的宋遠橋宋大俠,是不是?』我聽他一語道破我的心意,又是奇怪,又是吃驚。他又道︰『居士欲再做一件震動武林的大案,激那混元霹靂手成崑出頭,以報殺你全家的大仇\dash{}』我聽他逕自説出了我師父的名字,更是駭異。要知我師父殺我全家之事,我從没跟旁人説過,這種醜事我師父掩飾抵賴也猶死不及,自己當然更不會説,這空見却如何知道?」

\qyh{}我一聽到『混元霹靂手成崑』七個字,身子猛烈的一聲説道︰『大師若肯見示他的行蹤所在,我謝遜一生給你做牛做馬,也所甘願。』空見嘆道︰『這成崑所作所爲,罪孼確是太大,但居士一怒而牽累著害死了許多武林人物,眞是罪過罪過。』我心中本來想説︰『要你多管什麼閒事?』但想起適纔他所顯示的武功,我可不是他的敵手,於是説道︰『在下這是迫於無奈,那成崑躱得了無影無蹤,四海茫茫,教我到那裡去找他?』空見點頭道︰『我也知你滿腔怨毒,無處發洩,但那宋大俠是武當派張三丰張眞人的首徒,你要是害了他,這個禍闖得不小。』我道︰『我是志在闖禍,禍事越大,越能逼成崑出來。』」

\qyh{}空見大師道︰『謝居士,你要是害了宋大俠,那成崑確是非出頭不行,但今日的成崑已非昔日可比,你武功遠不及他,這場血海冤仇是報不了的。』我道︰『成崑是我師父,他武功如何,我知道得心你清楚。』空見大師搖頭道︰『他另投明師,三年來的進境非同小可。你雖練成了崆峒派的『七傷拳』,却傷他不得。』我心裡驚詫無比,這位空見大師我生平從未見過面,但我的一舉一動,他却似件件親眼目睹。我呆了片刻,道︰『你怎麼知道?』他道︰『是成崑跟我説的。』」

他説到這裡,張殷夫妻和無忌一齊「啊」的一聲。謝遜道︰「你們此刻聽著尚自驚奇,當時我聽了這句話,全個人跳了起來,喝道︰『他又怎麼知道?』他緩緩的道︰『這幾年來,他始終跟隨在你身旁,只是他不斷的易容改裝,是以你認他不出。』我道︰『哼,我認他不出,他便是化了灰,我也認得他。』他道︰『謝居士,你自非粗心大意之人,可是這幾年來,你一心想的只是練武報仇,對身週之事都不放在心上了。你在明裡,他在暗裡,你不是認他不出,你壓根児便没去認他。』這番話不由得我不信,何況空見大師是名聞天下的有道高僧,諒也不致打誑騙我,我道︰『既是如此,他暗中將我殺了,豈不乾淨?』空見道︰『他若起心害你,自是一舉手之勞。謝居士,你曾兩次找他報仇,兩次都打敗了,他若要傷你性命,那時候爲什麼便不下手?再説你去奪那「七傷拳譜」之時,你曾跟崆峒派的三大高手比拚内力,可是「崆峒五老」中的其餘二老呢?他們爲什麼不來圍攻?要是五老齊上,你未必能保得性命吧?』」

\qyh{}當日我打傷『崆峒三老』後,發覺其餘二老竟也身受重傷,這件怪事我一直存在心中,是一個未能得解的疑團。莫非崆峒派忽起内鬨?還是另有不知名的高手在暗中助我?我聽見空見大師這般説,心念一動,説道︰『竟難道那二老是成崑所傷?』」

張翠山和殷素素聽他愈説愈奇,雖然江湖上的事波譎雲詭。兩人見聞均廣,什麼古怪的事也都聽見過,可是像謝遜所説那樣的事,却實是猜想不透。兩人心中均隱隱覺得,謝遜已是個極了不起的人物,但他師父混元霹靂手成崑,不論智謀武功,似乎又是處處勝他一疇。殷素素道︰「大哥,那崆峒二老,眞是你師父暗中所傷麼?」

謝遜道︰「當時我這般衝口而問,空見大師説道︰『崆峒二老受的是什麼傷,謝居士親眼得見麼?他二人臉色怎樣?』我默然無語,隔了半晌,道︰『如此説來,崆峒二老當眞是我師父所傷了。』原來我見到崆峒二老躺在地下,滿臉都是血紅的斑點,顯然是他二人用陰勁傷人,却被高手以『混元功』逼回。這種滿臉血紅斑點,以我所知,除了被混元功逼回自身内勁之外,除非是猝發斑症傷寒之類惡疾,但我當日初見崆峒五老之時,五個人都是好端端地,自決非突起暴病。當然武林之中,除了我師徒二人,再無第三人練過混元功。」

\qyh{}空見大師點了點頭,嘆道︰『你師父酒後無德,傷了你一家老小,酒醒之後,惶慚無地,是以你兩次找他報仇,他都不傷你性命。他甚至不肯將你打傷,但你兩次都是發瘋般跟他拚命,若不傷你,他始終無法脱身。嗣後他一直暗中跟隨在你身後,你三度遭遇危難,都是他暗中解救。』我心下琢磨,除了崆峒鬥五老之外,果然另有兩件蹊蹺之事,在萬分危急之際,敵方攻勢忽懈。空見大師又道︰『他自知罪過太深,也不敢求你饒恕,只盼日子一久,你慢慢淡忘了。豈知你越鬧越大,害死的人越來越多,今日你若是再去殺了宋遠橋宋大俠,這場大禍可眞的是難以收拾了。』」

\qyh{}我道︰『好,那姓宋的與此事無涉,我也不去找他了,便請大師叫我師父來見我。』空見大師道︰『他没臉見你,也不敢見你。再説,謝居士,不是老衲小覷你,你便是見了他也是枉然,你的武功跟他差得太遠,這場仇是報不了的啦。』我道︰『大師是當世有道高僧,你叫我便此罷了不成?』他道︰『謝居士遭遇之慘,老衲也代爲心傷。可是尊師酒後亂性,實非本意,何況他已深自懺悔,還望謝居士念著昔日師徒之情,網開一面。』我心下狂怒,説道︰『我若再打他不過,任他一掌擊斃便了。此仇不報,我也不想活了。』」

\qyh{}空見大師沉吟良久,説道︰『謝居士,尊師武功已非昔比,你便是練成了『七傷拳』,也傷他不得。你若不信,便請打老衲幾拳試試。』我道︰『在下跟大師無冤無仇,豈敢相傷?在下武功雖是低微,這七傷拳却也不易抵擋。』他見我執意要報此血仇,説道︰『謝居士,我跟你打一場賭。尊師殺了你全家一十三口性命,你便打我一十三拳。倘若打傷了我,老衲罷手不理此事,尊師自會出來見你。否則這場冤仇便此作罷如何?』我沉吟不答,心知這位高僧武功奇深,七傷拳雖然厲害,要是眞的傷他不得,難道這仇便不報了?」

\qyh{}空見大師又道︰『老實跟你説,老衲既然插手管了此事,決不容你再行殘害無辜的武林同道。你若一念向善,便此罷手,過去之事大家一筆勾銷。否則你要找人報仇,難道爲你所害那些人的子弟家人,便不想找你報仇麼?』我聽他語氣嚴厲起來,狂性大發,喝道︰『好,我便打你一十三拳!你抵擋不住之時,隨時喝止。大丈夫言出如山,你可要叫我師父出來相見。』空見大師微微一笑,説道︰『請發拳吧!』我見他雖是身子矮小,但白眉白鬚,貌相慈祥莊嚴,不忍便此傷他,第一拳只使了三成力,砰的一聲,擊在他的胸口。」

無忌道︰「義父,你使的便是這種震斷樹脈的『七傷拳』麼?」謝遜道︰「不是!這第一拳是我師父成崑所授的『霹靂拳』,我一拳擊去,他身子晃了晃,退後一步。我心中想,這一拳只使了三成力,他已退後一步,若是將『七傷拳』施展出來,不須三拳,便能送了他的性命。當下我第二拳稍加勁力,他仍是晃了晃,退後一步。第三拳時我使了七成力,他也是一晃之下,再退一步。我心中微感奇怪,我拳上的勁力已加了一倍有餘,但擊在他的身上,仍是一模一樣。依他枯瘦的身形,我一拳便能打斷他的肋骨,但他體内並不生反震之力,只是若無其事的受了我三拳。」

\qyh{}我心想,若要將他打倒,非出全力不可,可是我一出全力,他非死即傷。我雖然爲惡已久,但對他捨己爲人的慈悲心懷,也有些肅然起敬,於是我説道︰『空見大師,你只挨打不還手,我不忍再打。你受了我三拳,我答應不去害那宋遠橋便是。』他道︰『那麼你跟成崑的怨仇怎樣?』我道︰『此仇不共戴天,不是他死,便是我亡。』我頓了一頓,又道︰『但大師既然出面,我姓謝的敬重大師,自此而後,只找成崑自己和他的家人,決不再連累不相干的武林同道。』」

\qyh{}空見大師合什説道︰『善哉,善哉!謝居士有此一念,老衲謹代天下武林同道謝過。只是老衲立心化解這場冤孼,剩下的十拳,你便照打吧。』我心下盤算,只有用『七傷拳』將他擊傷,我師父纔肯露面,好在這『七傷拳』的拳勁收發自如,我下手自有分寸,於是説道︰『如此便得罪了!』第四拳跟著發出,這一次用的是『七傷拳』的拳勁了。拳力一中在他的胸口,他胸口微一低陥,便向前跨了一步。」

無忌拍手道︰「這可奇了,這位老和尚這一次不再退後,反而向前。」張翠山道︰「我想那是少林派的『金剛不壞體』神功吧?」謝遜點頭道︰「五弟見多識廣,所料果然不錯。我一拳擊出,和前三拳已是大不相同,他身上生出一股反震之力,只震得我胸内腹中,有如五臟一齊翻轉。我心知他也是迫於無奈,倘若不使這種神功,那便擋不住我的七傷拳。我久聞少林派的『金剛不壞體』神功,乃是古今五大神功之一,其時親身領受,果然是非同小可。當下第五拳我偏重陰柔之力,他仍是跨前一步,那股陰柔之力反擊過來,我好容易纔得化解\dash{}」

無忌道︰「義父,這老和尚説話可不算數了,他説好不還手的,怎地將你的拳勁反擊回來?」謝遜撫著他的頭髮,説道︰「我打過他五拳,空見大師便道︰『謝居士,我没料到七傷拳威力如此驚人,我不運勁回震,那便抵擋不住。』」我道︰「你没還手打我,已是深得盛情。」當下我拳出如風,第六、七、八、九四拳一口氣打出。那空見大師也眞了得,這四拳打在他身上,他一一震回,剛柔分明,層次井然。我心下好生駭異,喝道︰『小心了!』第十拳輕飄飄的打了出去。他微微的點了點頭,不待我拳力著身,便跨上兩步,竟是在這霎息之間,佔了機先。」

無忌自然不明白跨兩步有什麼難處,張翠山却深知高手對敵,能在對手出招之前,先行料到,實是極大的難事,通常只須料到一招,即足制勝。他點頭道︰「了不起,了不起!」謝遜續道︰「這第十拳我已是便足了全力,他搶先反震,竟使我倒退了兩步。我雖是瞧不見自己的臉色,但可以想見,那時我是臉如白紙,全無血色。空見大師緩緩吁了口氣,説道︰『這第十一拳不忙便打,你定一定神再發吧!』我雖是萬分的要強好勝,但内氣翻騰,一時之間,那第十一拳確是擊不出去。」

張翠山等聽到這裡,都是甚爲心焦,無忌忽道︰「義父,下面還有三拳,你就不要打了吧。」謝遜道︰「爲什麼?」無忌道︰「這老和尚爲人很好,你打傷了他,心中過意不去。倘若傷了自己,那也不好。」張翠山和殷素素對望一眼,心想這孩子小小年紀,居然有這等見識,可説極不容易。張翠山心中更是喜歡,覺得無忌心地仁厚,能彀分辨是非。

只聽得謝遜嘆了口氣,説道︰「枉自我活了幾十歳,那時却不及孩子的見識。我心中充滿了報仇雪恨之念,不找到師父,那是決不肯甘休,明知再打下去,兩人中必有一個死傷,可也顧不了許多。我運足勁力,第十一拳又擊了出去,這一次他却身形斗地向上一拔,我這一拳本來打他胸口,但他一拔身,拳力便中在小腹之上。他眉頭一皺,顯得很是疼痛。我明白他的意思,若是他用胸口擋我拳力,反震之力極大,只怕我禁受不起,但小腹的反擊之力雖然弱了,他自身受的苦楚却大得多。」

\qyh{}我呆了一呆,説道︰『我師父罪孼深重,死有餘辜,大師何苦以金玉之體,爲他擋災蔽晦?』空見大師調勻了一下呼吸,苦笑道︰『只盼再挨兩拳,便\dash{}便化解了這場劫數。』我聽他説話氣息不屬,突然心念一動︰『看來他運起『金剛不壞體』神功之時,不能説話,我何不引他説話,突然一拳打出。』於是便道︰『倘若我在十三拳内打傷了你,你保得我師父一定會來見我麼?』他道︰『出家人不打誑語。』我道︰『你雖答應了我,却怎料得他一定現身?』他道︰『他親口跟我説過的\dash{}』就在此時,我不等他一句話説完,呼的一拳便擊向他的小腹。這一拳去勢既快,落拳又低,要令他來不及發動護體神功。」

\qyh{}那知道道佛門神功,隨心而起,我的拳勁剛觸到他的小腹,他神功便已佈滿全身。我但覺天旋地轉,心肺欲裂,騰騰騰連退七八步,背心在一株大樹上一靠,這纔站住。」

\qyh{}我心灰意懶之下,惡念陡生,説道︰『罷了!罷了!此仇難報,我謝遜又何必活於天地之間?』提起手掌,一掌便往自己天露蓋拍下。」

無忌叫道︰「妙計,妙計!可是義父,這一下不是太狠毒了麼?」張翠山道︰「爲什麼?」無忌道︰「義父拍擊自己的天靈蓋,那位老和尚自然出聲喝步,過來救你。義父乘他不防,便可下手了。不過老和尚對你這麼好,你決不能傷他,是不是?」

張翠山和殷素素盡皆駭異,他們雖知自己的児子聰明伶俐之極,那料到他在這頃刻之間,便能識破謝遜的奸計。他夫婦也都是一等一的機伶人物,江湖上閲歷又多,但見事却比無忌還慢了一步。謝遜慘然嘆道︰「我便是要利用他的好心仁善,無忌,你料得不錯,我揮掌自擊天靈蓋,雖是暗伏詭計,却也是行險僥倖。倘若這一掌擊得不重,他看出了破綻,便不會過來阻止。十三拳中只剩下最後一拳,七傷拳的拳勁雖然厲害,怎破得了他的護身神功?那時要找我師父報仇之事,再也休提。當時我是孤注一擲,這一掌實是用足了全力,他若不來救,我便自行擊碎天靈蓋而死,反正報不了仇,我原本是不想活了。」

\qyh{}空見大師一見事出非常,大聲叫道︰『使不得,使不得,你何苦\dash{}』一面説,一面飛躍過來架開我的一掌,我左手跟著一拳擊出,砰的一聲,打在他的胸腹之間。這一下他確是全無提防,連運神功的念頭也没生,他血肉之軀,如何擋得住這一拳?登時内藏震傷,摔倒在地。」

\qyh{}我擊了這一拳,眼見他不能再活,陡然間天良發現,伏在他身上大哭起來,叫道︰『空見大師,我謝遜忘恩負義,豬狗不如!』」

張翠山等三人都是默然,心想謝遜以這詭計打死這位有德高僧,確是大大不該,謝遜道︰「空見大師見我痛哭,反而微微一笑,安慰我道︰『人孰無死?居士何必難過?你師父即將到來,你須得鎭定從事,别要魯莽。』他一言提醒了我,適纔這一十三拳大耗眞力,眼下大敵將臨,豈可再痛哭傷神?於是我盤膝坐下,調勻内息。那知隔了良久。始終不見我師父到來。我望著空見大師,臉上頗現詫異之色。」

\qyh{}這時空見大師已氣息微弱,斷斷續續的道︰『想\dash{}想不到他\dash{}他言而無信\dash{}難道\dash{}難道什麼人忽然絆住了他麼?』我大怒起來,喝道︰『你騙人,你騙我打死了你,我師父還是不出來見我。』他搖頭道︰『我不騙你,眞是對你不起。』我狂怒之下,還想罵他,忽然想起︰『他騙我來打死他自己,於他有什麼好處?我打死他,他反而來向我道歉。』不由得心中十分慚愧,跪在他的身前,説道︰『大師,你有什麼心願,我一定給你了結?』他又是微微一笑,説道︰『但願你今後殺人之際,有時想起老衲。』」

\qyh{}這位高僧不但武功精湛,而且大智大慧,洞悉我的爲人。他知道若要我絶了報仇之心,改做好人,那是決計做不到的,他説了也不過是白説,可是他叫我殺人之際有時想起他,五弟,那日在船中你跟我比拚掌力,我所以没傷你性命,就是因爲忽然間想起了空見大師。」

張翠山萬想不到自己的性命竟是空見大師救的,對這位高僧更增景慕之心。無忌道︰「義父,你爲什麼跟爹爹比拚掌力?」殷素素道︰「他兩人鬧著玩呢,瞧是誰的功夫厲害些。」無忌有些不信,又問︰「義父,那時你的眼睛已瞎了没有?」殷素素急忙喝阻︰「無忌,别胡説八道。」謝遜道︰「没有瞎啊,你爲什麼要問?」無忌道︰「一定是爹爹打你不過,媽媽幫著爹爹,便把你眼睛射瞎\dash{}」張翠山和殷素素齊聲喝道︰「無忌!」兩人語聲十分嚴峻,無忌嚇了一跳,不敢再説下去了。

謝遜道︰「你們别嚇壞了我好孩児,無忌,你説好啦,你怎樣猜到的?」無忌向爹媽望了一眼,道︰「我\dash{}我\dash{}」謝遜道︰「你説得不錯,那時你爹爹打我不過,你媽媽幫你爹爹,便將我眼睛射瞎了。那是很久很久以前的事了。這件事起因是我自己不好,我也没怪他們。是你媽媽跟你説的麼?」他明知此事殷素素決不會跟児子説,但這麼一問,兩人便不會出言阻止。無忌道︰「不!前幾天媽説要教我打金針,但第二天却又説不教了。我想一定是爹爹叫媽别教我,怕你知道之後心裡不高興。」謝遜哈哈大笑,道︰「五弟,素妹,這孩子比我聰明五倍,比你們聰明十倍,你們説將來如何得了?」

張殷二人不約而同的伸出手去,各自握住了無忌的一隻手。二人雖自得意,但隱隱的却也感到了一層憂慮,張翠山是生怕孩子聰明有餘,將來却不學好,殷素素是怕他不壽。

無忌笑道︰「義父,這麼説來,你不是比我爹媽聰明兩倍麼?」謝遜道︰「只怕兩倍也不還不止。」無忌道︰「後來那位老和尚能治好麼?」謝遜嘆道︰「治不好啦!他氣息越來越微,我手掌按住他靈台穴,拚命將以自己眞力,延續他的性命。」他忽然深深吸了口氣,問道︰『你師父還没來麼?』我道︰『那是不會來的了。』我道︰『大師,你放心,我不會再胡亂殺人,激他出來。但我走遍天涯海角,定是找到他。』他道︰『很好,很好!不過,你武功不及他除非\dash{}除非\dash{}』説到這裡,聲音越來越低。我把耳朶湊到他的嘴邉,只聽他道︰『除非\dash{}能找到屠龍刀,找到\dash{}找到刀中的祕\dash{}』他説到這個『祕』字,一口氣接不上來,便此死了。」

直到謝遜源源本本的説了這個故事,張翠山夫婦方始懂得,他爲什麼苦思焦慮的要探索屠龍刀中的祕密,爲什麼平時溫文守禮,狂性發作起來時却又是行如禽獸,爲什麼雖然身負絶世武功,却是終日愁苦\dash{}他三人結義十年,平素無話不説,但這位大哥的身世,可到今晩方才明白。

謝遜道︰「後來我得到屠龍刀的消息,趕到王盤山島上來奪刀,這些事你們親眼得見,那也不用説了。我在得刀之前,千方百計的要找尋成崑,得了屠龍刀之後,却反而怕他找上了我,因此要尋個極隱僻的所在,慢慢探尋刀中祕密。爲了生怕你們洩露我的行藏,纔把你們帶同前來。想不到一晃十年,謝遜啊謝遜,你還是一事無成。」張翠山道︰「空見大師臨死之時,這番話或是没有説全,他説『除非能找到屠龍刀中的祕密\dash{}』,説不定是另有所指。」謝遜道︰「這十年之中,什麼荒誕不經,異想天開的情景我都想過了,但没一件事能和他的説話相符。刀中一定藏有一件大祕密,那是確定無疑的,但我殫智竭力,猜想不透,無忌,你比我聰明得多,將來説不定你能想得出來。」

無忌道︰「義父,那成崑今年有幾歳啦?」謝遜臉色一變,説道︰「孩子,你説的不錯,他今年六十五歳啦,冤沉海底,再無報仇之日,賊老天,賊老天,你眞會害人。」張翠山夫婦和無忌心中都明白,就算將來無忌能發見刀中所藏的祕密,能練成剋制的功夫,他又能回歸中土,找到成崑,看來那也是二三十年之後的事,那時成崑十之八九早已不在人世了。四個人談了一晩,天色將明。謝遜道︰「無忌,你不要睡了,義父再傳你一路武功。」張翠山夫婦對望一眼,無法阻步,只得相偕回洞。

自這晩一夕長談之後,謝遜不再提及此事,但督率無忌練功,却變成了嚴厲異常。無忌此時不過九歳,雖然聰明過人,但要短期内盡數領悟謝遜這身舉世少見的武功,却怎生能彀?謝遜竟是不加理會,只要無忌學得不如他意,他便又打又罵,絲毫不予姑息。殷素素常常見到児子身上青一塊、烏一塊,心中甚是痛惜,跟謝遜道︰「大哥,你神功蓋世,三年五載之内,無忌如何能練得成?這荒島上歳月無盡,你不妨慢慢教他。」謝遜道︰「我又不是教他練,是教他盡數記在心中。」殷素素奇道︰「你不教無忌練武功麼?」謝遜道︰「哼,一招一式的練下去,練到他頭髮白了,也不知成不成。我只是要他記著,牢牢的記在心頭。」殷素素不明其意,但知這位大哥行事處處出人意表,只得一切由他。不過每見到孩子身上傷痕累累,便抱他哄他,疼惜一番,無忌居然很是明白事理,道︰「媽,義父是要我好,他打得狠些,我便記得牢些。」

如此又過了大半年,一日早晨,謝遜忽道︰「五弟,素妹,再過四個月,風向和海潮一齊轉南,今日起咱們來紮木排吧。」張翠山驚喜交集,道︰「你説紮了木排,回歸中土嗎?」謝遜冷冷的道︰「那也得瞧賊老天發不發善心,這叫作謀事在人,成事在天。成功,便回去,不成功,便溺死在大海之中。」

依著殷素素的心意,在這海外仙山般的荒島之上,日子過得逍遙自在,實在不必冒著奇險回去,但想到無忌長大之後如何娶妻生子,想到他一生埋没荒島實在可惜,當下便興高采烈的一起來紮結木排。好在島上多的是參天古木,因爲生於寒冰之地,木質緻密,硬如鐵石。謝遜和張翠山忙忙碌碌的砍伐樹木,殷素素便用樹筋來編織帆布,搓結帆索。無忌奔走傳遞,連那玉面火猴也是毛手毛脚的在一旁幫忙。

饒是謝遜和張翠山武功精湛,殷素素也早不是個嬌怯怯的女子,但因没有刀斧等就手家生,紮結這個大木排實是事倍功半。往往費了大半個時辰,方始弄倒一棵合用的大樹。紮結木排之際,謝遜總是要無忌站在他的身邉,盤問他是否忘了他所教的武功。這時張殷二人也不再避嫌走開,由得他義父義子二人一問一答,但聽所問答的都是些口訣之類。謝遜甚至將各種刀法、劍法,都要無忌一一記在心中。但要他記的又不是如何擊刺變招的動作,只是要他背經書一般的死記。謝遜這般「武功文教」,已是奇怪,偏又不加半句解釋,便似一個最不會教書的蒙師,要小學生呆背詩云子曰,圇囫吞棗,生吞活剝。殷素素在旁聽著,有時忍不住可憐孩子,心想别説是孩子,便是一個精通武學的大人,也未便能記得這些口訣招式,而且不加試演,死記住這些口訣招式又有何用?難道口中説幾句招式,便能克敵制勝麼?更何況無忌只要背錯一字,謝遜便重重的一個耳光打了過去。雖然他手掌上不帶内勁,但這一個耳光,常常便使無忌半邉臉蛋紅腫半天。這座大木排一直紮了兩個多月,方得大功告成,而豎立主桅副桅,又花了半個月時光。跟著便是打獵醃肉,縫製存貯清水的皮袋,要知在這茫茫的大海之上,説不定風向不順,一飄便是半年數月。待得事事就緒,已是白日極短,黑夜極長,但風向仍未轉過。三人在海旁搭了一個茅棚,遮住木排,只待風向一轉,便可下海。這時謝遜竟是片刻也不和無忌分離,便是晩間,也要無忌跟他同睡。張翠山夫婦見他對児子又是親熱,又是嚴厲,只有相對苦笑。一天晩上,張翠山半夜醒轉,忽聽得風聲有異,他坐起身來,聽得風聲果是從北而至,忙推醒殷素素,喜道︰「素素,你聽!」殷素素迷迷糊糊的尚未回答,忽聽得謝遜在洞外説道︰「轉北風啦,轉北風啦!」這聲音中竟又帶著哭音,中夜聽來,極其淒厲辛酸。次晨張殷夫婦歡天喜地的收拾一切,但想在這冰島上住了十年,忽然便要離開,竟有些戀戀不捨起來。待得一切食物用品搬上木排,已是正午,三人合力將木排推下海中,無忌抱了玉面火猴,第一個跳上排去,跟著是殷素素。張翠山挽住謝遜的手,道︰「大哥,木排離此七尺,咱們一齊跳上去吧!」

謝遜忽道︰「五弟,咱兄弟從此永别,願你好自珍重。」張翠山心中突的一跳,有似胸口被人重重打了一拳,説道︰「你\dash{}你\dash{}」謝遜道︰「你心地仁厚,原該福澤無盡,可是人事難料,天道難知,你一切小心。無忌已學得我一身武功,人又聰明,他日成就,當在你我之上。素妹雖是女子,却不會吃人的虧。我所擔心的,反倒是你。」張翠山道︰「大哥,你説什麼?你不跟\dash{}不跟我一起去麼?」謝遜道︰「早在數年之前,我便與你説過了。難道你忘了麼?」

這幾句話聽在張翠山耳中,猶似雷轟電擊一般,這時他方始記得,當年謝遜確曾有他猶個児不離此島的言語,但此刻他不再提起,張殷二人誰也没放在心上。當紮結木排之時,謝遜也從未露過獨留之意,不料直到臨行,他纔忽然説了出來。張翠山急道︰「大哥,你一個人在這島上寂寞淒涼,有什麼好?快跳上木排啊!」説著手上使勁,用力拉他。但謝遜的身子猶似一株大樹牢牢釘在地下,竟是一晃也不晃。張翠山叫道︰「素妹,無忌,快上來!大哥説不跟咱們一起去。」殷素素和無忌聽了,也是大吃一驚,一齊縱上岸來。無忌道︰「義父,你爲什麼不去?你不去我也不去。」

\chapter{十年爭鬥}

謝遜心中實在也捨不得和他三人分别,他早想到三人此去永無再會之期,他孤零零的獨處荒島,實是生不如死,但他思之已久,知道若是和張殷夫婦同歸中原,以自己仇家之衆,必替他一家三口子惹下無窮的禍患。他雖是行事偏激,却是性情中人,既與張翠山、殷素素張翠山義結金蘭,對他二人的愛護,實已勝過待己,而對義子無忌之愛,更是逾於親児。他自知背負一身血債,江湖上不論是名門正派還是綠林黑道,不知有多少人處心積慮的要置己於死地,何況屠龍刀落入己手,此事難免會洩露出去。若在從前,他自是枉然不懼,但這時眼目已盲,決計不能抵擋大批仇家的圍攻。他又料知張殷二人也決不致袖手不顧,任由自己死於非命,爭端一起,四人勢必同歸於盡。只怕一回歸大陸,四個人都活不到一年半載。但這番計較也不必跟二人説明,事到臨頭,方説自己決意留下。

他聽無忌這幾句話中眞情流露,將他身子抱了起來,柔聲道︰「無忌,乖孩子,你聽義父的説話。義父年紀大了,眼睛又瞎,在這児住得很安適,回到中原,只有處處不慣,什麼也不快活。」無忌道︰「回到中原後,孩児天一服侍你,不離開你身邉,你要吃什麼喝什麼,我立時給你端來,那不是一樣快活麼?」謝遜搖搖頭道︰「不行的。我還是在這裡快活。」無忌道︰「我也是在這裡快活。爹,媽,不如咱們都不去了,還是在這裡的好。」

殷素素道︰「大哥,你若有什麼顧慮,不如明言,大家一起籌劃籌劃。要説留你獨個在這児,咱們無論如何不允。」

謝遜心想︰「這三人都對我情義深重,要叫他們甘心捨己而去,只怕説到舌敝唇焦,也是不能。却如何想個法児,讓他們離去?」張翠山忽道︰「大哥,你是怕仇家太多,連累了咱們,是不是?咱四人回到中原之後,找個荒僻的所在隱居起來,不與外人來往,豈非什麼都没事了?最好是咱們都到武當山去住,誰也想不到金毛獅王會在武當山上。」謝遜傲然道︰「哼,你大哥雖然不濟,也不須託庇於尊師張眞人的庇下?」張翠山暗悔失言,忙道︰「大哥武功不在我師父之下,何必託庇於他?回彊西藏、朔外大漠,何處不有樂土?儘可供我四人自在逍遙。」

謝遜道︰「要找荒僻之所,天下還有何處更荒得如此間的?你們到底走是不走?」張翠山道︰「大哥不去,大夥児決意不去。」殷素素和無忌也齊聲道︰「你不去,我們都不去。」謝遜嘆了口氣道︰「好吧,大夥児都不去,等我死了之後,你們再回去那也不遲。」張翠山道︰「不錯,在這裡十年也住了,又何必著急?」謝遜忽然喝道︰「我死了之後,你們再没什麼留戀了吧?」

三人一愕之間,只見他手一伸,刷的一聲,拔出了屠龍刀,一刀便要脖子中抹去。張翠山大驚,叫道︰「休傷了無忌!」要知以他武功,決計阻不了謝遜橫刀自盡,情急之下叫他休傷無忌,謝遜果然一怔,收刀停住,喝道︰「什麼?」張翠山見他如此決絶,哽咽道︰「大哥既是決意如此,小弟便此拜别。」説著跪下來拜了幾拜。無忌却朗聲道︰「義父,你不去,我也不去!你自盡,我也自盡。大丈夫説得出做得到,你橫刀抹脖子,我也橫刀抹脖。」

這幾句話果然制住了謝遜,他想無忌年紀雖小,素來説話甚有分寸,自己以死相脅,他竟然也以死相脅,縱聲叫道︰「小鬼胡頭胡説八道!」一把抓住他背心,將他擲上了木排,跟著雙手連擲,把張翠山和殷素素也都投上,大聲叫道︰「五弟,素妹,無忌!一路順風,早歸中土。」

那玉面火猴見張翠山等被擲上木排,縱身飛躍,也跳上了木排。無忌放聲大哭,叫道︰「義父,義父!」謝遜橫刀喝道︰「你們若再上岸,我們結義之情,便此斷絶。」

這時海流帶著木排,緩緩飄遠,眼見謝遜的人影慢慢糢糊,慢慢的小了下去。張翠山和殷素素知道義兄心意堅決,終不可回,只得揮泪揚手,和他作别,隔了良久良久,直至再也瞧不見他身形,三人這纔轉頭。無忌伏在母親懷裡,哭得筋疲力盡,沉沉睡去。

那木筏便如此在大海中飄行,海流果是不停的向南,帶著木筏向南行。在這茫茫大海之上,自也認不出方向,但見每日太陽從左首升起,從右首落下,每晩北極星在筏後閃爍,而木筏又是不停的移動,便知離中原日近一日。最初二十餘天中,張翠山生怕木排和冰山相撞,不敢張帆,航行雖緩,但却安全,縱然撞到冰山,也是輕輕一觸,便滑了開去。直至遠離冰山群,纔張起帆來。

北風日夜不變,木筏的航行登時快了數倍,且喜一路未遇風暴,看來回歸故土,倒是有了七八成把握。這一月來,張殷二人怕無忌傷心,始終不談謝遜之事。這日殷素素見海面波濤不興,木排上的風帆張得滿滿的,直向南駛,忍不住説道︰「大哥不但武功精純,對天時地理也算得這般準,實是一位奇人。」無忌忽道︰「既然風向半年南吹,半年北吹,過年前咱們還回到冰火島,去探望義父。」張翠山喜道︰「無忌説得是,等你長大成人,咱們再一齊北去\dash{}」

殷素素突然指著南方,叫道︰「那是什麼?」只是遠處水天相接之處,隱隱有兩個黑點,張翠山吃了一驚,道︰「莫非是鯨魚?要是來撞木排,那可糟了。」殷素素看了一會,道︰「不是鯨魚,没見噴水啊。」三個人目不轉瞬的望著那兩個黑點,直到一個多時辰之後,張翠山歡聲叫道︰「是船,是船!」猛地縱起身來,翻了個斛斗。他自生了無忌之後,終日忙忙碌碌,從未有過這般孩子氣的行動。無忌哈哈大笑,學著父親,也翻了兩個斛斗。殷素素忙取過木柴脂油,在筏上生起一堆火來。

又航了一個時辰,太陽斜照,已看得清楚是兩艘大船。殷素素忽然身子微微一顫,臉色大變。無忌奇道︰「媽,怎麼啦?」殷素素口唇動了一動,却没説話。張翠山握住她手,臉上滿是関切的神色。殷素素嘆道︰「剛回來便碰見了。」張翠山道︰「怎麼?」殷素素道︰「你瞧那帆。」張翠山凝神瞧去,只見左首一艘大船的帆上,繪著一隻殷紅色血手,張開五指,顯得甚便詭異,説道︰「這艘船的風帆好生奇怪,你認得麼?」殷素素低聲道︰「是我爹爹的白眉教的。」

霎時之間,張翠山心頭湧起了許多念頭︰「素素的父親是白眉教的教主,這邪教看來無惡不作,我見到岳父時却怎生處?恩師對我這場婚事會有什麼説話?」只覺手掌中素素的小手在輕輕顫動,想是她也同時起了無數心事,當下説道︰「素素,咱們孩子也這麼大了!天上地下,永不分離。你還擔什麼心?」殷素素吁了一口長氣,回眸一笑,低聲道︰「只盼我不致讓你爲難,你一切要瞧在無忌的臉上。」

無忌從來没見過船隻,目不轉瞬的望著那兩艘船,心中説不出的好奇,没理會爹媽在説什麼。那木排漸漸駛近,只見兩艘船靠得緊密,竟似貼在一起。若是方向不變,木排便會在兩艘船右首數十丈處交叉而過。

張翠山道︰「要不要跟船上招呼?探問一下你爹爹的訊息?」

殷素素道︰「不要招呼,待回到中原,我再帶你和無忌去見爹爹。」張翠山道︰「{\upstsl{嗯}},那也好。」無忌忽然叫道︰「爹,媽,你瞧,兩隻船上的人在打架。」張殷二人抬起頭來,凝目望去,果見那邉船上刀光閃爍,似有四五人在動武。殷素素有些擔心,道︰「不知我爹爹在不在那邉?」張翠山道︰「既是碰上了,咱們便過去瞧瞧。」於是斜扯風帆,轉過木筏後的大舵,那木筏便略向左偏,對著兩艘船緩緩駛去。

木筏雖然扯足了風帆,但行駛仍是極慢,過了好半天,纔靠近二船。只聽得白眉教的船上有人高聲叫道︰「有正經生意,不相干的客人避開些吧。」殷素素叫道︰「是總舵的香主,那一壇的舵主在燒香?」她説的是白眉教的切口,那邉船上那人的語氣立時不同,恭恭敬敬的道︰「原來是總舵的香主駕臨,天市堂李香主,率領神舵壇封壇主、青龍壇程壇主在此。不知是那一位香主駕臨?」殷素素道︰「紫微堂香主。」

那邉船上聽得「紫微堂香主」五個字,登時亂了起來,稍過片刻,十餘人齊聲叫道︰「殷姑娘回來啦,殷姑娘回來啦。」

張翠山雖和殷素素成婚十年,從没聽她説過白眉教中的事,他也從來不問,這時聽得兩下裡對答,才知她還是什麼「紫微堂香主」,看來「香主」的權位,還是在「壇」主之上。他在王盤山島上,己見過玄武、朱雀兩壇壇主的身手,説武功是在殷素素之上,她所以能任香主,當是因爲她是教主之女,而這位「天市堂」李香主,想必是位極厲害的人物了。

只聽得對面船上一個極蒼老的聲音説道︰「聽説敝教殷姑娘回來啦,大家暫且罷鬥如何?」另一個高亮的聲音説道︰「好!大家住手。」接著兵刃相交之聲一齊停止,相鬥的衆人紛紛躍開。張翠山聽得那爽朗嘹喨嗓音很熟,一怔之下,叫道︰「是兪蓮舟師哥麼?」那邉船上的人叫道︰「我正是兪蓮舟\dash{}啊\dash{}啊\dash{}你\dash{}你\dash{}」張翠山道︰「小弟張翠山!」他心情激動,眼見木筏跟兩船相距尚有十餘丈,從筏上拾起一根大木,使勁一抛,跟著身子躍起,在大木上一借力,已躍到了對方船頭。

兪蓮舟搶上前來,師兄弟分别十年,不知死活存亡,這番相見,何等歡喜?兩人四手相握,一個叫了聲︰「二哥!」一個叫了聲︰「五弟!」眼眶中充滿泪水,再也説不出話來。

那邉白眉教迎接殷素素,却另有一番排場,四隻大海螺一齊嗚嗚吹起,李香主站在最前,封程兩位壇主站在李香主身後,其後又站著百來名大小教衆。大船和木筏之間搭上了跳板,七八名水手用長篙鉤住木筏,不使離開。殷素素摧了無忌的手,從跳板上走了過去。

原來白眉教中地位最尊的,自是教主白眉鷹王殷天正,他屬下分爲内三堂、外五壇分統各路教衆。内三堂是天微、紫微、天市三堂,外五壇是神蛇、青龍、白虎、玄武、朱雀五壇。天微堂主是殷天正的長子殷野王,紫微堂的香主便是殷素素,天市堂香主是殷天正的師弟李天垣。他雖武功極高,又是殷素素的長輩,但看在教主師兄的臉上,向來對殷素素極是客氣。

李天垣見殷素素衣衫襤褸,又是毛,又是皮,手中還擕著一個孩童,不禁一怔,但隨即滿臉堆歡,笑道︰「謝天謝地,你可回來了,這十年來不把你爹爹急煞啦。」殷素素拜了下去,説道︰「師叔你們好!」又對無忌道︰「快給師叔祖磕頭。」無忌爬在地下磕頭,一雙小眼却骨溜溜望著李天垣,他斗然間見到船上有這許多人,心中説不出的好奇。

殷素素站起身來,説道︰「師叔,這是侄女的孩子,叫作無忌。」李天垣一怔,隨即哈哈大笑,説道︰「妙極妙極!你爹爹一定要樂瘋啦,不但女児回家,還帶來這麼俊秀的一個小外孫。」殷素素見兩艘船的甲板上都濺滿了鮮血,兩船的甲板上都有幾具屍體躺著,低聲道︰「對方是誰?爲什麼動武?」李天垣道︰「對方是武當派和崑崙派的人。」殷素素見丈夫躍到對方船上,和一個人相擁在一起,稱他爲師哥,早知對方有武當派的人手在内,這時聽李天垣一説,不由得雙眉緊鎖,説道︰「最好先别動手,能化解便化解了。」李天垣道︰「是!」

要知李天垣雖是師叔,但在白眉教中,天市堂排名次於紫微堂,爲内堂之末。論到師門之誼,李天垣是長輩,但在處理教務之時,殷素素的權位反超過師叔。

只聽得張翠山在那邉船上叫道︰「素素,無忌,過來見過我師哥。」殷素素擕著無忌的手,向那艘船的甲板走去。李天垣和封程兩位壇主怕她有失,緊隨在後。

到了對面的船上,只見甲板上站著七八個人,一個四十餘歳的高瘦漢子和張翠山手拉手,神態甚是親熱。張翠山道︰「素素,這位便是我常常提起的兪二師哥。二哥,這是你弟婦和你侄児無忌。」

兪蓮舟和李天垣一聽,都是大吃了一驚,白眉教和武當派正在拚命惡鬥,那知雙方的一個重要人物竟是夫婦,不但是夫婦,而且還生了一個孩子。兪蓮舟心知道中間的曲折原委,非片刻間説得清楚,當下先給張翠山引見船上各人,一個矮矮胖胖的黃冠老道,是崑崙派的西華子,一個中年惡婦,是西華子的師妹,便是武林中名頭很響的閃電手衛四娘,江湖中人背後都稱她爲「閃電娘娘」。甚餘幾人也都是崑崙派的高手,只是名望没有西華子和衛四娘這般響亮。

那西華子年紀雖已不小,却是没半點涵養功夫,一開口便道︰「張五妹,謝遜那惡賊在那裡?你總是知道的吧?」

張翠山尚未回歸中土,還在茫茫大海之大,便遇上了兩個難題,第一是本門竟已和白眉教正面衝突;第二是人家一上來便問謝遜在那裡。他一時不知如何回答,向兪蓮舟問道︰「二哥,倒底是怎麼一回事?」西華子見張翠山不回答自己的問話,不禁暴躁起來,大聲道︰「你没聽見我的話麼?謝遜那惡賊在那児?」原來他在崑崙派中輩份很高,武功又強,一向是頤指氣使慣了的。

白眉教神蛇壇封壇主爲人很是陰損,適纔和這船上的人動手時,手下又有兩名得力弟子喪在西華子的劍下,心中本就對他極是惱怒,於是冷冷的道︰「張五俠是我白眉教教主的愛婿,你説話客氣些。」西華子大怒,喝道︰「邪教的妖女,豈能和名門正派的弟子婚配?這場婚事,中間定有糾葛。」封壇主冷笑道︰「我殷教主外孫也抱了,你胡言亂語什麼?」西華子怒道︰「這妖女\dash{}」衛四娘早看破了封壇主的用心,知他是挑撥崑崙、武當兩派之間的交情,同時又是乘機向張翠山和殷素素討好,聽得四華子接下去要説出更加不好聽的話來,忙道︰「師兄,不必跟他作無謂的口舌之爭,大家且聽兪二俠的示下。」

兪蓮舟瞧瞧張翠山,瞧瞧殷素素,也是疑團滿腹,説道︰「大家且請到艙中從長計議。雙方死傷的兄弟,先行救治。」這時白眉教是客,而教中權位最高的,却是紫微堂香主殷素素。她擕了無忌的手,首先踏進艙中,跟著便是李天垣。當封壇主踏進船艙時,突覺一股微風襲向腰間。

封壇主在江湖中的經歷何等豐富,立知是西華子暗中偸襲,他竟不出手抵擋,只是身子向前一撲,叫道︰「啊喲,打人麼?」這一下將西華子一招「三陰絶戸手」避了開去,但這麼的一叫,人人都轉過頭來瞧著他二人。衛四娘瞪了師兄一眼,西華子一張紫瞠色的臉中泛出了隱紅。須知既然來到了此間船上,封壇主等都是賓客,西華子這一下偸襲,實是頗失名門正派中高手的身份。

當下各人在艙中分賓主坐下。殷素素是賓方的首席,無忌侍立在側。主方是兪蓮舟爲首,他指著衛四娘下首的一張椅子道︰「五弟,你坐在這裡吧。」張翠山應道︰「是。」依言就座。這麼一來,張殷夫婦分成賓主雙方,也便是相互敵對的兩邉。

這十年之中,張翠山失縱,存亡未卜,兪岱岩傷後不出,其餘武當五俠威名却又盛了許多。宋遠橋、兪蓮舟等雖是武當派中的第二代弟子,但在武林之中,已隱然可和少林派的衆高僧分庭抗禮,江湖上人對武當五俠極是敬重,因此西華子、衛四娘等輩份雖高,還是尊他坐了首席。

船中的衆弟子奉上香茶,各人不提正事,都是隨口客套。兪蓮舟私下盤算︰「五弟失蹤十年,原來是和白眉教教主的女児結成夫婦,這時當著衆人之面問他,他必有難言之隱。」於是朗聲説道︰「咱們少林、崑崙、峨嵋、崆峒、武當五派,神拳、五鳳刀等九門,海派、巨鯨等七幫,一共二十一個門派幫會,爲了找尋金毛獅王謝遜、白眉教殷姑娘,以及敝師弟張翠山三人的下落,和白眉教有了誤會,不幸互有死傷,十年中武林擾攘不安\dash{}」他説到這裡,頓了一頓,説道︰「天幸殷姑娘和張師弟突然在海上出現,這十年中的事故頭緒紛紛,當非片言説得明白。依在下之見,咱們一齊回歸大陸,由殷姑娘稟明教主,敝師弟也回武當告稟家師,然後雙方再行擇地會晤,分辨是非曲直,如能從此化敵爲友,那是最好不過\dash{}」

西華子突然插口道︰「謝遜那惡賊在那児?咱們要找的是謝遜那惡賊。」張翠山聽到説爲了找尋自己三人,中原竟有二十二個幫會門派大動干戈,十年爭鬥,死傷了不少人,心中大是不安。耳聽得西華子不住口的詢問謝遜的下落,不禁爲難之極,若是説了出來,不知有多少武林高手要去冰火島找他尋仇,若是不説,却又如何隱瞞?

他正自遲不決,殷素素突然道︰「無惡不作、殺人如毛的惡賊謝遜,在九年前早已死了。」兪蓮舟、西華子、衛四娘等同聲驚道︰「謝遜死了?」殷素素道︰「便是在我生育這孩子的那天晩上,那惡賊謝遜狂性發作,正要殺害五哥和我,突然間聽到孩子的哭聲,他心病一起,那胡作妄爲的惡賊謝遜便此死了。」這時張翠山已然明白,殷素素所以一直再説「惡賊謝遜已經死了」,也可説並未説謊,蓋自謝遜聽到無忌的第一下哭聲,便即觸發胸中天良,自此狂性收斂,去惡向善,至於逼他三人離島,更是捨己爲人、大仁大義的行逕,是以很可説「惡賊謝遜」在九年前死去,「好人謝遜」在九年前誕生。西華子鼻中哼了一聲,他心中認定殷素素是邪教妖女,信不過她的説話,厲聲道︰「張五俠,那惡賊謝遜眞的死了麼?」殷素素坦然道︰「不錯,惡賊謝遜在九年前早已死了。」無忌在一旁聽得各人不住的痛罵惡賊謝遜,爹爹媽媽甚至説他早已死了。他雖然聰明,但那知武林中的各種過節,謝遜對他恩義極深,對他的愛護照顧,絲毫不在父母之下,他生性極厚,忍不住大聲哭了起來,叫道︰「義父不是惡賊,義父没有死,義父没有死。」這幾聲哭叫,艙中諸人盡皆愕然。殷素素狂怒之下,反手便是一記耳光,喝道︰「住口!」無忌哭道︰「媽,你爲什麼説義父死了?他不是好端端的活著麼?」他一生只和父母及義父三人共處,雖然智力遠勝常人,但人間的險詐機心,却是從來没接觸過半點,若是換作一個在江湖上長大的孩子,即使没他一半聰明,也知説謊是家常便飯,決不會闖出這件大禍來。殷素素斥道︰「大人在説話,小孩子多什麼口?咱們説的是惡賊謝遜,又不是你義父。」無忌心中一片迷惘,但已不敢再説。

西華子微微冷笑,問無忌道︰「小弟弟,謝遜是你義父,是不是?他在那裡啊?」無忌看了父母的臉色,知道他們所説的事極関重要,聽西華子這麼問,便搖了搖頭,道︰「我不説。」他這「我不説」三個字,實則是更加言明謝遜並未身死。

西華子瞪視張翠山,説道︰「張五俠,這位白眉教的殷姑娘,眞是你的夫人嗎?」張翠山没料到他突然會問這句話,朗聲道︰「不錯,她便是拙荊。」西華子厲聲道︰「我崑崙門下的兩名弟子,毀在尊夫人手下,變成死不死、活不活,這筆帳如何算法?」張翠山和殷素素都是一驚,殷素素出口便道︰「胡説八道!」張翠山道︰「這中間必有誤會,咱夫婦不覆中土已有十年,如何能毀傷貴派弟子?」西華子道︰「十年之前呢?高則成和蔣濤之被害,算來原已有十年了。」殷素素道︰「高則成和蔣濤?」西華子道︰「張夫人還記得這兩人麼?只怕你殺人太多,已記不清楚了。」殷素素道︰「他二人怎麼了?何以你咬定是我害了他們?」西華子仰天打個哈哈,説道︰「我咬定你,我咬定你?哈哈,高蔣二人雖然成了白痴,却還能記得一件事,説得出一個人的名字,知道毀得他們如此的,乃是\dash{}『殷素素』!」

他將「殷素素」的名字一個字一個字的説了出來,語氣中充滿了怨毒,眼光牢牢的瞪視著殷素素,似乎恨不得立時拔劍在她身上刺上幾劍。

白眉教的封壇主突然接口道︰「本教紫微堂香主的閨名,豈是你出了家的老道隨口叫得?連清規戒律也不守,還充什麼武林前輩?程賢弟,你説世上可恥之事,還有更甚於此的麼?」程壇主接口道︰「再没有了。名門正派之中,居然出了這種狂徒,可笑啊可笑。」西華子大怒欲狂,喝道︰「你兩個説誰可恥?」封壇主眼角也不掃他一下,説道︰「程賢弟,一個人便算學得幾手三脚貓的劍法,行事説話總得也像個人樣子,你説是嗎?」程壇主道︰「自從玉虛道長逝世之後,都是一代不如一代了。」

原來玉虛道長是西華子的師伯,武功德望,武林中人人欽服。西華子紫脹著臉皮,對這句話却是不便駁斥,若説這句話錯了,豈不是説自己還勝過當年名震天下的師伯?他身形一閃,站到了艙口,刷的一聲,長劍出手,叫道︰「邪教的惡賊,有種的便出來見個眞章!」封壇主和程壇主所以要激怒西華子,本意是要替殷素素解圍,心想張翠山和殷香主既是夫婦,武當派和白眉教的関係已是大大不同,便算兪蓮舟和張翠山不出手,至少也是兩不相助,那麼單獨對付崑崙派的幾個,便可穩操勝算。衛四娘秀眉緊蹙也已算到了這一節,心想憑著自己和師哥等六七個人,決難抵敵白眉教這許多高手,何況張翠山夫婦情重,極可能相助對方,於是説道︰「師哥,人家到咱們船上,那是賓客,咱們聽兪二俠的吩咐便是。」她是要用言語擠兌兪蓮舟,心想以你的聲望地位,決不能處事偏私。那知西華子草包之極,大聲説道︰「他武當派和白眉教早結了親家,同流合汚,已成一丘之貉,他還能有什麼公正的話説出來?」

兪蓮舟城府很深,喜怒不形於色,聽了西華子的話,沉吟不語。衛四娘忙道︰「師哥,你怎地胡言亂語?别説武當派跟我們崑崙派同氣連枝,淵源極深,十年來聯手抗敵,精誠無間,兪二俠更是鐵錚錚的好漢子,英名播於江湖,天下誰不欽仰。他武當五俠爲人處事,豈能有所偏私?」西華子哼了一聲,道︰「不見得。」衛四娘心中暗罵師哥草包,竟聽不出自己言中之意,於是大聲説道︰「師哥,你没來由的得罪武當五俠,掌門師叔怪罪起來,我可不管。」西華子聽她抬出掌門師叔來,纔不敢再説。

兪蓮舟緩緩的道︰「此事牽涉到武林中各大門派,各大幫會,在下無德無能,焉敢信口雌黃,隨意處分?反正這事已擾攘了十年,也不爭在再花一年半截的功夫。在下須得和張師弟回歸武當,稟明恩師和大師兄,請恩師示下。」西華子冷笑道︰「兪二俠這一招『如封似閉』的推搪功夫,果然高明得緊啊。」

兪蓮舟並不輕易發怒,但他所説的這招「如封似閉」,正是武當派天下馳名的守禦功夫,乃是恩師張三丰所創,他譏嘲武當武功,那便是辱及恩師,但他立時轉念︰「這件事處理稍有失當,便引起武林中一場難以收拾的浩劫。這個莽道人胡言亂語,我何必跟他一般見識?」西華子見他聽了自己這兩句話後,眼皮一翻,神光炯炯,有如電閃,不由得心中打了個突︰「我師父和掌門師叔是本派最強的高手,眼神的厲害似乎還不及他。」但見兪蓮舟眼中精光隨即收斂,淡淡的道︰「西華道兄如有什麼高見,在下洗耳恭聽。」西華子給他適纔眼神這麼一掃,心膽已寒,轉頭道︰「師妹,你説怎麼?難道高濤二人的事便此罷手不成?」衛四娘尚未回答,忽聽得南邉號角之聲,嗚嗚不絶。崑崙派的一名弟子走到艙門口,説道︰「崆峒派和峨嵋派的接應到了。」李天垣和封壇主、程壇主對望了一眼,臉上均是微微變色。西華子和衛四娘聽説到了幫手,心中大喜。衛四娘道︰「兪二俠,不如聽聽崆峒、峨嵋兩派的高見。」兪蓮舟道︰「好!」張翠山却又多了一重心事,心想︰「峨嵋派還不怎樣,崆峒派却和大哥結有深仇。他傷過崆峒五老,奪了崆峒派的『七傷拳經』,他們自然要苦苦追尋他的下落。」殷素素跟他所想的相同,心想若不是無忌多口,事情便好辨得多,但想無忌從來不説謊話,對謝遜又情義深重,忽然聽到義父死了,自是要大哭大叫,原也怪他不得,見他面頰上被自己打了一掌後留下腫起的紅印,不禁憐惜起來,將他摟在懷裡。無忌兀自不放心,將小嘴湊到母親耳邉,低聲道︰「媽,義父没有死啊,是不是?」殷素素也湊嘴到他耳邉,輕輕道︰「没有死。我騙他們的,這些都是惡人壞人,他們都想去害你義父。」無忌恍然大悟,自兪蓮舟起,每個人都狠狠的瞪了一眼,心道︰「原來你們都是惡人壞人,你們想害我義父。」謝無忌從這一天起,纔起始踏入江湖,起始明白世間人心的陰毒。他伸手撫著臉頰,母親所打的這一掌兀自隱隱生疼。他是個聰明的孩子,知道這一掌雖是母親打的,實則是爲眼前這些惡人壞人所累。他自幼生長在父母和義父的慈愛卵翼之下,不懂得人間竟有心懷惡意的敵人,謝遜跟他説過成崑的故事,但那終是耳中聽來,直到此時,纔面對面他心目中的敵人。過了好一會功夫,崆峒和峨嵋兩派各有六七人走進船艙,和兪蓮舟、西華子、衛四娘等見禮。崆峒派爲首的是個精乾枯瘦的葛衣老人,峨嵋派爲首的則是個中年尼姑,這一干人見到白眉教的李天垣等坐在艙中,都是一愕。

\chapter{恩怨纏綿}

西華子大聲道︰「唐三哥,靜虛師太,武當派跟白眉教聯了手啦,這一回咱們可得吃大虧。」原來那矮矮瘦瘦的葛衣老人叫做唐文亮,是崆峒五老之一,那中年尼姑靜虛師太,是峨嵋派的第四代弟子,都是武林中頗有名望的好手。他們聽到西華子這麼説,都是一怔。靜虛師太爲人精細,素知西華子的毛包脾氣,還不怎樣,唐文亮却眼睛一翻,瞪著兪蓮舟道︰「兪二俠,此話可眞?」

兪蓮舟還未答話,西華子已搶著道︰「人家武當派已和白眉教結成了親家,張翠山張五俠做了殷大教主的女婿\dash{}」唐文亮奇道︰「失蹤十年的張五俠已有了下落?」兪蓮舟指著張翠山道︰「這是我五師弟張翠山,這位是崆峒派的前輩高人,唐文亮唐三爺,你二人多親近親近。」他二人剛説得幾句客套話,西華子又道︰「張五俠和殷姑知道金毛獅王謝遜的下落,但是瞞著不肯説,反而撒個漫天大謊,説謝遜已經死了。」

唐文亮一聽到「金毛獅王謝遜」的名字,又驚又怒,喝道︰「他在那裡?」張翠山道︰「此事須得先行稟明家師,請恕在下不便相告。」唐文亮眼中如要噴出火來,喝道︰「謝遜這惡賊在那裡?他殺死我的親侄児,我姓唐的不能跟他並立於天地之間,他在那裡?你到底説是不是?」最後這幾句話聲色倶厲,竟是没半分禮貌。殷素素怒從心起,冷冷的道︰「他拳傷崆峒五老,盜去『七傷拳經』,此事你怎麼不説了?」

謝遜擊傷崆峒五老,盜走「七傷拳經」,乃是冒了成崑的名頭,此事也是直到四五年前,崆峒派方才明白是謝遜所爲。但因五老受傷,拳經文被盜去,實是崆峒派的奇恥大辱,上上下下方來祕而不宣,却不知殷素素如何得知?唐文亮一聽之下,臉色登時蒼白,十指箕張,便要向殷素素撲去,但一轉頭,眼見她是個嬌嬌怯怯的美貌少婦,以自己成名的前輩人物,實不便向她動手,強忍怒氣,向張翠山道︰「這一位是?」張翠山道︰「便是拙荊。」西華子接口道︰「也就是白眉教殷大教主的令愛。」白眉鷹王殷天正武功深不可測,迄今爲止,武林中跟他動過手的,還没有一個能擋得住他十招以上。唐文亮一聽這少婦是殷天正的女児,心中也不禁忌憚,只是道︰「好,好,好!」

靜虛師太自進船艙之後,一直文文靜靜的没有開口,這時才道︰「此事的原委究是若何,還請兪二俠示下。」兪蓮舟道︰「這件事牽連既廣,爲時又已長達十年,一時三刻之間,豈能分剖明白?這樣吧,三個月之後,敝派在黃鶴樓設宴,邀請有関的各大門派幫會一齊赴宴,是非曲直,當衆評論。各位意下如何?」靜虛師太點了點頭,道︰「如此甚好。」

唐文亮道︰「是非曲直,儘可三個月後再論,但謝遜那惡賊藏身何處,還須請五俠先行示明。」張翠山搖頭道︰「此刻實不便説。」唐文亮雖極不滿,但想武當派既和白眉教聯手,倒也眞惹不起,然而公道自在人心,且看他三個月之後,如何向天下群雄交代,當下不再多説,站起身來雙手一拱,道︰「如此三個月後再見,告辭。」

西華子將手一揮,道︰「唐三爺,咱們幾個搭你的船,成不成?」唐文亮道︰「好啊,什麼不成?」西華子向衛四娘道︰「師妹,走吧!」他本和兪蓮舟同船而來,這麼一來,顯是將武當派當作了敵人。兪蓮舟不動聲色,客客氣氣的送到船頭,説道︰「咱們回山稟明師尊,便送英雄宴的請帖過來。」殷素素忽道︰「西華道長,我有一件事請教。」西華子愕然回頭,道︰「什麼事?」

殷素素道︰「道長不住口的説我是邪教妖女,却不知邪在何事,妖在何處?倒要請教。」西華子怔了一怔,道︰「邪魔外道、狐媚妖淫,那便是了,又何必要我多説?否則好好的一個武當派的張五俠,怎會受你迷惑?嘿嘿,嘿嘿!」説著連連冷笑。殷素素道︰「好,多承指點!」西華子見自己這幾句話竟將她説得啞口無言,却也頗出意料之外,聽她没再説什麼,便踏上跳板,走向崆峒派的船去。

那兩艘海船都是三帆大船,雖然並在一起,兩船甲板仍是相距兩丈來遠,那跳板也就甚長。西華子因和殷素素對答了幾句,落在最後,餘人都已過去。他正走到跳板中間,忽聽得背後風聲微起,跟著擦的一聲輕響,他人雖暴躁,武功却著實不低,江湖上閲歷也多,一聽到這聲音,知道背後有人暗算,霍地轉過身來,長劍也已拔在手中。便在此時,脚底從中斷爲兩截,他急忙拔起身子,但兩船之間,空空蕩蕩的無物可以攀援,雖見足底藍森森的大海,但一躍之後未能再躍,仍是撲通一聲,掉入了海中。

他不識水性,一掉入海中,立時咕嚕咕嚕的喝了幾大口鹹水,雙手亂抓亂划,突然抓到了一根繩子,大喜之下,牢牢握住,只覺有人拉動繩子,將他提出了水面。西華子抬頭一看,那一端握住繩子的,却是白眉教的程壇主,臉上似笑非笑的瞧著自己。

原來殷素素惱恨他言語無禮,待各人過船之時,暗中吩咐了封程二壇主,安排下計謀,封壇主三十六柄飛刀神技,馳名江湖,不但出手既快且準,而且每柄飛刀均是高手匠人以精鋼所鑄,薄如柳葉,鋒鋭無比,對手見他飛刀飛來時,若以兵刃擋架,往往兵刃便被飛刀削斷。這時他以飛刀切割跳板,輕輕一劃,跳板已斷。程壇主早在一旁備好繩索,待西華子吃了幾口水後,纔將他吊將上來。

衛四娘,唐文亮等見西華子落水,雖猜到是對方做了手脚,但跳板斷得太快,各人的眼光又都望著殷素素,竟没瞧見跳板如何斷截,待得各人呼喝欲救時,程壇主已將他吊了上來。西華子強忍怒氣,只等人一上船,便出手與對方搏鬥。那知程壇主只將他拉得離水面尺許,便不再拉,叫道︰「道長,千萬不可動彈,在下力氣不彀,你一動,我拉不住便要脱手啦!」西華子心想他若是裝傻扮痴,又將自己抛在海裡,那可不是玩的,只得握住繩子,不敢向上攀援。

程壇主叫道︰「小心了!」手臂一抖,將長繩甩起了半個圏子。他臂力實是了得,這麼一抖,西華子的身子向後凌空盪出了七八丈,跟著又是向前一送,將他摔向對船。西華子放脱繩子,雙足落上甲板。他的長劍已在落海時失却,這時憤怒如狂,只聽得白眉教的船上喝采聲和歡笑聲響成一片,當下拔出衛四娘身上佩劍,便要撲過去拼命。但這時兩船相距已遠,無法一縱而過,空自暴跳如雷,除了戟指大罵,再無别法。

殷素素如此作弄西華子,兪蓮舟全瞧在眼裡,心想這女子果然是邪門,可不是五弟的良配,於是説道︰「殷李兩位香主,相煩代爲稟報殷教主,三月後黃鶴樓頭之會,他老人家若是不棄,務請大駕光臨,今日便此别過。五弟,你隨我去見恩師麼?」張翠山道︰「是!」殷素素聽兪蓮舟言下之意,竟是要也夫婦分離,當下抬頭瞧了瞧天,又低頭瞧了瞧脚底的甲板。

張翠山登時領悟,知她説的是「天上地下,永不分離」這兩句誓言,便道︰「二哥,我帶領你弟婦和孩子先去叩見恩師,得他老人家准許,再去拜見岳父。你説可好?」

兪蓮舟點頭道︰「那也好。」殷素素心下甚喜,對李天垣道︰「師叔,請你代爲稟告爹爹,便説不孝女児天幸逃得性命,不日便回歸總舵,拜見他老人家。」李天垣道︰「好,我在總舵恭候兩位大駕。」站起身子,便和兪蓮舟等作别。殷素素道︰「我哥哥好吧?」李天垣道︰「很好,很好!令兄近年連得奇逢,武功突飛猛進,做師叔的早已望塵莫及,實是慚愧得緊。」殷素素微微笑道︰「師叔又來跟咱們晩輩説笑啦。」李天垣正色道︰「這不是説笑,連你爹爹也是没口子的稱讚,説他肖子跨灶,青出於藍,你説厲害不厲害?」殷素素笑道︰「啊喲,師叔當著外人之面,老鼠跌落天平,自稱自讚,却不怕兪二俠見笑。」李天垣笑道︰「張五俠做了我們姑爺,兪二俠難道還是外人麼?」説著一舉手,轉身出艙。兪蓮舟聽了這幾句話,心中很不樂意,微皺眉頭,却不説話。

張翠山一等白眉教衆人離船,忙問︰「二哥,三哥的傷勢後來怎樣?他\dash{}痊可了吧?」兪蓮舟「{\upstsl{嗯}}」的一聲,良久不答。張翠山甚是無急,目不轉睛的望著二哥,心頭湧起一陣不祥之感,生怕他説出一個「死」字來。兪蓮舟緩緩的道︰「三弟没死,不過跟死也差不了多少。他終身殘廢,手足不能移動。兪岱岩兪三俠,嘿嘿,江湖上算是没這號人物了。」張翠山聽到三哥没死,心頭一喜,但想到一位英風俠骨的好漢竟落得如此下場,忍不住{\upstsl{憯}}然下泪,哽咽著問道︰「害他仇人是誰?可査出了麼?」

兪蓮舟不答,一轉頭,突然間兩道閃電般的目光照在殷素素臉上,森然道︰「殷姑娘,你可知道害我兪三弟的人是誰?」殷素素禁不住身子輕輕一顫,説道︰「聽説兪三俠的手足筋骨,是被人用少林派的金剛指法所斷的。」兪蓮舟道︰「不錯。你不知是誰麼?」殷素素搖了搖頭,道︰「不知道。」

兪蓮舟不再理她,説道︰「五弟,少林派説你殺死臨安府龍門鏢局老小,又殺死了幾名少林僧人。此事是眞是假?」張翠山道︰「這個\dash{}」殷素素道︰「這不関他事,都是我殺的。」兪蓮舟望了她一眼,目光中流露出極度痛恨的神色,但這目光一閃即隱,臉上隨即回復平和,説道︰「我原知五弟決不會胡亂殺人。爲了這件事,少林派曾三次遣人,上武當山來理論,但五弟突然失蹤,武林中盡皆知聞,這回事就此没了對證。咱們説少林派害了三哥,少林派説五弟殺了他們數十條人命。好在少林寺掌門住持空聞大師老成持重,尊敬恩師,竭力約束門下弟子,不許擅自生事,十年來纔没釀成大禍。」殷素素道︰「都怪我年輕時作事不知輕重好歹,現下我也好生後悔。但人也殺了,咱們給他來個死賴到底,決不認帳便了。」

兪蓮舟臉上露出詫異之色,向張翠山瞧了一眼,心想這樣的女子你怎能娶她爲妻。殷素素見他一直對自己冷冷的,口中也只稱「殷姑娘」不稱「弟婦」,心下早已有氣,説道︰「一人作事一身當。這件事我決不連累你武當派,讓少林派來找我白眉教便了。」兪蓮舟朗聲道︰「江湖之上,事事抬不過一個『理』字,别説少林派是當世武林中第一大派,便是無拳無勇的孤児寡婦,咱們也當憑理處事,不能仗勢欺人。」

若在十年之前,兪蓮舟這番義正辭嚴的教訓,早使殷素素老羞成怒,拔劍相向,但她心中雖然惱怒,只聽得張翠山恭恭敬敬的道︰「三哥教訓得是。」暗想︰「我纔不聽你這一套仁義道德呢。但若我衝撞於他,倒令張郎難於做人,我且讓你一步便了。」便擕了無忌的手,走向艙外,説道︰「無忌,我帶你去瞧瞧這艘大船,你從來没見過船,是不?」

張翠山待妻子走出船艙,説道︰「二哥,這十年之中,我\dash{}」兪蓮舟左手一擺,説道︰「五弟,你我肝膽相照,情逾骨肉,便有天大禍事,二哥也跟你生死與共。你夫妻之事,暫且不必跟我説,回到山上,專候師父示下便了。師父若是怪責,咱們武當七俠一齊跪地苦求,你孩子都這般大了,難道師父還會硬要你夫妻父子生生分離?」張翠山大喜,説道︰「多謝二哥。」

原來兪蓮舟外剛内熱,在武當七俠之中,最是不{\upstsl{茍}}言笑,幾個小師弟怕他比大師兄宋遠橋厲害得多。其實他於師兄弟上情誼極重,張翠山忽然失蹤,他暗中傷心欲狂,面子上却是忽忽行若無事,今日師兄弟重逢,實是他生平第一件喜事,但還是疾言厲色,將殷素素教訓了一頓,直到此刻師兄弟單獨相對,方始稍露眞情。他最放心不下的,是殷素素殺傷了這許多少林弟子,此事決難善罷,他心中早已打定了主意,寧可自己性命不在,也要保護師弟一家平安周全。

張翠山又問︰「二哥,咱們跟白眉教大起爭端,可也是爲了小弟夫婦麼?此事小弟心中實在太過不安。」兪蓮舟道︰「王盤山之會,到底如何?」張翠山於是將在臨安如何夜闖龍門鏢局、如何識得殷素素。如何偕赴王盤山參與白眉教揚刀立威,一直説至金毛獅王謝遜如何大施屠戮、奪得屠龍寶刀,逼迫二人他往。

兪蓮舟聽完這番話後,又詳細詢問崑崙派高則成和蔣濤二人之事,沉吟半晌,纔道︰「原來如此。倘若你終於不歸,不知這中間的隱祕到何日方能揭開。」張翠山道︰「是啊,我義兄\dash{}{\upstsl{嗯}},二哥,那謝遜其實並非怙惡不悛之輩,他所以如此,實是生平一件大慘事逼成,此刻我已和他義結金蘭。」兪蓮舟點了點頭,心想︰「這又是一件棘手之極的事。」張翠山續道︰「我義兄一吼之威,將王盤山上衆人盡數震得神智失常,他説這等人即使不死,也都成了白痴,那麼他得到屠龍刀的祕密,再也不會洩漏出去了。」兪蓮舟道︰「這謝遜行事狠毒,但確也是個奇男子,不過他百密一疏,終於忘了一個人。」張翠山道︰「誰啊?」兪蓮舟道︰「白龜壽。」

張翠山道︰「啊,白眉教中的玄武壇壇主。」兪蓮舟道︰「依你所説,當日王盤山島上群豪之中,以白龜壽的内力最爲深厚。他被謝遜的酒箭一沖,暈死過去,後來謝遜作獅子吼,白龜壽倘若好端端地,只怕也抵不住他的一吼\dash{}」張翠山一拍大腿,道︰「是了,其時白龜壽暈在地下未醒,聽不到吼聲,反而保全了性命。我義兄雖然心思細密,却也没想到此節。」

兪蓮舟嘆了口氣,道︰「從王盤山上生還的,只有白龜壽和崑崙派的高蔣二人。崑崙派的内功有獨到之處,高蔣二人雖然功力尚淺,總算還保全了性命,但自此疾痴呆呆,神智不清。旁人問他二人,到底是誰害得他們這個樣子,蔣濤只是搖頭不答,高則成却自始至終,説著一個人的名字\dash{}殷素素。」他頓了一頓,又道︰「這時我方明白,原來他是心中念念不忘弟妹,哼,下次西華子再出言不遜,瞧我怎生對付他。他崑崙弟子行止不謹,還來怪責人家。」

張翠山道︰「白龜壽既然生還,他該知道一切原委啊。」兪蓮舟道︰「可他就偏不肯説。你道爲什麼?」張翠山略略尋思,已然明白︰「是了。白眉教想去搶奪屠龍寶刀,不肯吐露這獨有的訊息,因此始終推説不知。」兪蓮舟道︰「今日武林中的大紛爭,便是爲此而起。崑崙派説殷素素害了高蔣二人,咱師弟也都道你已遭了白眉教的毒手。」張翠山道︰「小弟前赴王盤山之事是白龜壽説的麼?」兪蓮舟道︰「不,他諱如深,什麼也不肯説。我和四弟、七弟同到王盤山踏勘,見到你用鐵筆冩在山壁上的那二十四個大字,才知你果然也參與了白眉教的『揚刀立威之會』。咱三人在島上找不到你的下落,自是去找白龜壽詢問,他言語不遜,動起手來,被我打了一掌,不久崑崙派也有人找上門去,却吃了一個大虧,被白眉教殺了兩人。十年來雙方的仇怨竟是愈結愈深。」張翠山甚是歉疚,説道︰「爲了小弟夫婦,因而各門派子弟無辜遭難,心中如何能安?小弟稟明師尊之後,當分赴各門派解釋誤會,領受罪責。」兪蓮舟嘆了口氣道︰「這是陰錯陽差,原也怪不得你。本來嘛,倘若單是爲了你們夫婦二人,也只崑崙、武當兩派和白眉教之間的糾葛,但白眉教爲了要搶奪那屠龍刀,始終不提謝遜的名字,於是巨鯨幫、海沙派、神拳門這些幫會門派,都把幫主和掌門人的血海深仇,一齊算在白眉教的頭上,白眉一教,成爲江湖上的衆矢之的。」

張翠山嘆道︰「其實那屠龍刀有什麼了不起,我岳父何苦如此代人受過?」兪蓮舟道︰「我從未和令岳會過面,但他統領白眉教,獨抗群雄,這份魄力氣槩,所有與他爲敵之人,也都不禁欽服。」張翠山道︰「峨嵋、崆峒等門派,並未參與王盤山會啊,怎地也和白眉教結下了冤仇?」兪蓮舟道︰「此事却是因你義兄謝遜而起了。白眉教爲了想得那屠龍刀,接二連三的派遣海船,遍訪各外海島,找尋謝遜的下落,須知紙包不住火,白龜壽的口再密,這消息還是洩漏了出來。你這義兄曾冒了『混元霹靂手成崑』之名,在大江南北做過三十幾件大案,各門各派的成名人物,死在他手下的不計其數,此事你可知道麼?」張翠山點然點點頭,低聲道︰「人家終於知道是他幹的了。」

兪蓮舟道︰「他每做一件案子,便在牆上大書『殺人者,混元霹靂手成崑是也』,其時我們奉了師令,曾一同下山査訪,當時誰也不知眞正的兇手是誰,那混元霹靂手成崑也始終不曾露面。但當白眉教知道謝遜下落的消息一洩露,各門派中深於智謀的人便連帶想起,那謝遜本是成崑的唯一傳人,又知他師徒不知何故失和,翻臉成仇,然則冒成崑之名殺人的,多半便是謝遜了。你想謝遜害過的人,牽連何等廣大?單是少林派中最高一輩的空見大師也死在他的拳下,你想想有多少人欲得他而甘心?」

張翠山神色慘然,説道︰「我義兄雖已改過遷善,但雙手染滿了這許多鮮血\dash{}唉,二哥,我心亂如麻,不知如何是好。」兪蓮舟道︰「咱們師兄弟爲了你而找白眉教,崑崙派爲了高蔣二人而找白眉教,巨鯨幫他們爲了幫主慘死而找白眉教,更有以少林派爲首許多白道黑道人物,爲了逼問謝遜的蹤跡而找白眉教。這些年來,雙方大戰過五場,小戰不計其數。雖然白眉教每一次大戰均落下風,但你岳父居然在群起圍攻之下苦撐不倒,實在算得是個人傑。當然,少林、武當等名門正派,以事情眞相未曾明白,中間隱晦難解之處甚多,不願過走極端,處處替對方留下餘地,但一般江湖人物却是出手決不客氣的。這一次咱們得到訊息,白眉教天市堂李香主乘船出海,咱們便暗中跟了下來,只盼能査到一些蛛絲馬跡。那知李香主瞧出情形不對,硬不許咱們在後跟隨,崑崙派的子弟們便跟他們動起手來。倘若你夫婦的木筏不在此時出現,雙方又得損折不少好手了。」

張翠山默然,細細打量師哥,見他兩鬢斑白,額頭亦添了不少皺紋,説道︰「二哥,這十年之中,你可辛苦啦。我百死餘生,終於能再見你一面,我\dash{}我\dash{}」

兪蓮舟見他眼眶濕潤,説道︰「武當七俠重行聚首,正是天大的喜事。自從三弟受傷,你又失蹤,江湖上改稱咱們爲『武當五俠』,嘿嘿,今日起七俠重振聲威\dash{}」但他想到兪岱岩手足殘廢,七俠之數雖齊,然而要像往昔一般,師兄弟七人聯袂行俠江湖,終究難能,神色之間不禁黯然。

海舟南行十數日,到了長江口上,一行人改乘江船,溯江而上。張翠山夫婦換過了襤褸的衣衫,兩人宛似瑤台雙璧,風采不減當年。無忌穿上了新衫新褲,頭上用紅頭繩紮了兩根小辮子,甚是活潑可愛。兪蓮舟潛心武學,無妻無子,因此特别喜歡無忌,只是他生性嚴峻,沉默寡言,雖然心中喜愛,神色間却是冷冷的。可是無忌聰明逾恆,心知這位冷口冷面的師伯其實待已極好,一有空閒,便纏著師伯東問西問,須知他生於荒島,陸地上的事物什麼也没見過,因之看來事事透著新鮮。兪蓮舟竟是不感厭煩,常常抱著他坐在船頭,觀看江上風景,無忌問八句十句,他便短短的回答一句。這一日江船到了安徽銅陵的銅官山脚下,天色向晩,江船便舶在一個小市鎭旁,船家上岸去買肉沽酒,張翠山夫婦和兪蓮舟在艙中煮茶閒談。無忌獨自在船頭玩耍,只見碼頭旁有個老年乞丐,頭頸中盤著一條青蛇,手中還舞弄著一條黑身白點的大蛇。他坐在地下,全神貫注的弄蛇,那條黑蛇一忽児盤到了他頭上,一忽児橫背而過,甚是靈動。

無忌在冰火島上從來没見過蛇,看得甚是有趣。那老丐見到了他,向他笑了笑,手指一彈,那黑蛇突然躍起,在空中打了個觔斗,落下時在他的胸口盤了幾圏。無忌大奇,目不轉睛的瞧著。那老丐向他招了招手,做了幾個手勢,示意他走上岸去,還有好戲法變給他。無忌當即從跳板走上岸去,那老丐從背上取下一個布囊,張開了袋口,笑道︰「裡面還有好玩的東西,你來瞧瞧。」無忌道︰「是什麼東西?」那老丐道︰「很有趣的,你一看便知道了。」無忌探頭過去,往囊中瞧去,但黑黝黝的看不見什麼。他又移近一些,想瞧個明白,那老丐突然雙手,將布袋套上了他的腦袋。無忌「啊」的一聲叫,只覺嘴巴已被那老丐隔袋按住,身子也被提了起來。

他這一聲從布袋之中呼出,聲音已甚微弱,而且一呼之後,立即被那老丐按住了口,但兪蓮舟和張翠山是何等樣人,雖然隔得甚遠,已察覺呼聲不對,兩人更不打話,同時奔到船頭,一瞥頭便見無忌已被那老丐擒住。兩人正要飛身躍上岸去,那老丐厲聲喝道︰「要保住孩子性命,便不許動。」

他説話之時,嗤的一聲,撕破了無忌背上的衣服,將那黑蛇之口對準了他背心的皮肉,這時殷素素也已奔到船頭,眼見愛児被擒,便欲施發金針。兪蓮舟雙手一攔,喝道︰「使不得!」他認得這黑蛇在天下十八種劇毒的毒蛇之中,位居第十一,名叫「漆裡星」,身子越黑,白點越細,那便毒性愈烈。這條黑蛇身子黑得發亮,身子白點也是閃閃發光,張開大口,露出四根獠牙,對準著無忌背上的細板白肉,只要這一口咬下去,頃刻間便即斃命,縱使擊斃了那個老丐,獲得解藥,也未必便能及時解救,當下不動聲色,説道︰「尊駕和這孩童爲難,意欲何爲?」

那老丐見兪蓮舟手臂輕輕的一抖,鐵鍊便已飛起,功力之精純,武林中甚是罕見,不禁臉上微微變色。張翠山提起長篙,在岸上一點,坐船便緩緩退向江心。那老丐道︰「再退開些!」張翠山憤然道︰「難道還没七丈麼?」那老丐微笑道︰「兪二俠手提鐵錨的武功如此厲害,便在七八丈外,在下還是不能放心。」張翠山只得又將坐船撐退了數丈。兪蓮舟抱拳道︰「請教尊姓大名。」那老丐道︰「在下是丐幫中的無名小卒,賤名不足以汚兪二俠的清聽。」兪蓮舟見他背上負了六隻布袋,心想這是丐幫中的六袋弟子,地位已算不低,如何竟幹出這種卑汚行逕來?何況丐幫素來行事仁義,他們幫主耶律淵如又和大師哥宋遠橋是極好的朋友,這事可眞奇了。正自沉吟,殷素素忽道︰「東川的巫山幫,已投靠了丐幫麼?我瞧丐幫中没閣下這一份字號?」那老丐「咦」的一聲,還未回答,殷素素又道︰「賀老三,你又來搗什麼鬼。你只要傷我孩子的一根毫毛,我把你們的梅石堅{\upstsl{{\upstsl{刴}}}}做十七廿八塊?」

那老丐吃了一驚,笑道︰「殷姑娘果然好眼力。認得我賀老三。小可我正是受梅幫主的差遣,來恭迎公子。」殷素素怒道︰「快把毒蛇拿開!你這巫山幫小小幫會,惹到我白眉教頭上來啦。」賀老三道︰「只須殷姑娘一句話,賀老三立時把公子送回,梅幫主還親自登門陪罪。」殷素素道︰「要我説什麼説?」賀老三道︰「我們梅幫主的獨生公子,死在謝遜手下,殷姑娘想必早有聽聞。梅幫主求懇張五俠和殷姑娘\dash{}不,小人失言,該當稱張夫人,求懇兩位開恩,示知那惡賊謝遜的下落,合幫上下,盡感大德。」殷素素秀眉一揚,説道︰「我們不知道。」賀老三道︰「那只有懇請兩位代爲打聽打聽,咱們好好侍候公子,一等兩位打聽到了謝遜的去處,梅幫主自當親身送還公子。」

殷素素眼見毒蛇的獠牙和愛子的背脊相距不過數寸,心中一陣衝動,便想將冰火島之事説了出來,一轉頭,向丈夫望了眼,却見他一臉堅毅之色。她和張翠山十年夫妻,知他爲人極重義氣,自己若是爲救愛子,洩漏了謝遜的住處,倘若義兄因此死於人手,那麼夫妻之情只怕也是難保,話到口邉,却又忍住不説。

張翠山朗聲道︰「好,你把我児子擄去便是,大丈夫豈能出賣朋友?你可把武當七俠瞧得忒也小了。」賀老三一楞,他只道將無忌一擒到,張翠山夫婦非吐露謝遜的訊息不可,那知張翠山竟是如此斬釘截鐵的回答,當下又道︰「兪二俠,那謝遜罪惡如山,武當派主持公道,武林人所共仰,還請你勸兩位一勸。」

兪蓮舟道︰「此事如何處理,在下師兄弟正要回歸武當,稟明恩師,請他老人家示下。黃鶴樓英雄大會,請貴幫梅幫主和閣下同來相會,屆時是非曲直,自有交代。你先將孩子放下。」他離岸十餘丈,説這幾句話時絲毫没提氣縱聲,但賀老三聽來,一字一句清清楚楚送入耳中,便如接席而談一般,心下好生佩服,暗想︰「武當七俠威震天下,果然是名不虛傳。這一次咱們破釜沉舟,幹出這件事來,看來巫山幫是結下了一個惹不起的強仇。但梅幫主殺子之仇,不能不報。」於是抱拳,説道︰「既是如此,小人多多得罪,只有請張公子回東川去。」

他這一抱拳,那條黑蛇便離無忌背心遠了尺許。無忌的腦袋雖被套在布袋之中,但他四人的一番對答,句句聽在耳中,只感到賀老三手臂一鬆,當即反手一掌,便拍在他背心的「靈台穴」上,借著這一掌反震之力,身子向前一竄,已脱却賀老三的懷抱。他生怕賀老三縱蛇追噬,不及拉開頭上的布袋,颼颼颼的向前連躍三個起落。

\chapter{百歳壽誕}

無忌奔出了十丈遠近,這纔拉脱頭上布套,回過身來,只見賀老三躺在地下,動也不動了。張翠山急速撐船近岸,和兪蓮舟、殷素素躍上岸來。殷素素奔向無忌,驚喜交集,將他摟在懷裡,見他背上皮肉無損,緊緊的抱著他,連叫︰「好孩子,好孩子!」

張翠山長劍連揮,先將賀老三身上盤著的兩條毒蛇挑開斬死,然後俯身看他,但見他口中吐出一縷鮮血,雙眼骨碌碌的亂轉,臉上神情甚是痛苦,却是不能動彈。兪蓮舟大是奇怪︰「難道這小孩児輕輕一掌,便將他打得這個模樣?」伸手拉著他左臂提了起來,但見他四肢僵直,宛似給人點了穴道,於是伸掌在他胸口「膻中穴」頸後「大椎穴」兩處推拿了幾下。賀老三慘叫一聲︰「啊喲!你\dash{}你有種便一刀把我宰了,别\dash{}别這般折磨折磨人!」四肢痙攣、全身發抖,牙関打得格格直響。

兪蓮舟吃了一驚,他替賀老三推拿兩處穴道,原是要給他解穴。要知道「膻中穴」又名氣海,爲人身氣之本源,「大椎穴」則是手足三陽督脈之會,這兩穴一通,周身任何一處被封閉的穴道都有好處,便算不能解開,也能査知何處穴道閉塞。不料一加推拿,賀老三竟會痛楚不堪,眼見他額頭汗珠直落,知他禁受不住,只得先行點了他肩背的穴道,使他身子麻痺,暫止疼痛,回過頭來望著張翠山。

張翠山却也不明其中之理,道︰「素妹,你用毒針打了他麼?」殷素素道︰「没有啊。是不是他自己給毒蛇咬了?」賀老三道︰「不\dash{}不是的。是你\dash{}你児子在我背心上拍了一掌。」他斜眼瞧著無忌,又是詫異,又是害怕。

殷素素大是得意,道︰「無忌,是你打得他這樣的麼?好孩子,眞有本事,眞有本事。」張翠山道︰「解什麼穴道?」自己児子打了旁人穴道,做父親的居然不會解救,説來自己也有些不好意思,因之這一句話似是問殷素素,似乎是問無忌,甚至似乎是問賀老三。殷素素笑嘻嘻的道︰「孩子,爹爹叫你解穴,你便給他解了吧。教他知道小英雄『謝無忌』的手段。」兪蓮舟第一次聽到謝無忌三字,頗感奇怪,説道︰「謝無忌?」張翠山道︰「{\upstsl{嗯}},小弟的第一個孩児過繼給了義兄,跟他的姓。」

三個人一齊望著無忌,瞧他如何解穴,却見無忌搖頭道︰「我不會!」張翠山道︰「怎麼不會?」無忌道︰「當時義父跟我説,這麼一掌若是打中了敵人的太陽、膻中、大椎、露台四處大穴,一個對時便即斃命。我便問他如何解救醫治。他沉著臉道︰『這種打穴的手法,天下只有你會我會,何必學救治之法?是你敵人才打,既是敵人,打了何必再救?難道救活他之後,將來等他再來害你麼?』」張翠山夫婦知道這正是義兄的口氣,照他脾氣確是下手狠辣,斬草除根。

賀老三倒是一條硬漢,説道︰「兪二俠、張五俠,我存心不良,前來擄勢公子,今日遭他毒手,那是罪有應得。你快快將我一掌打死,免我多受零碎苦楚。」兪蓮舟眉頭一皺,道︰「你罪不至死,我這侄児小孩子不知輕重,在下甚是抱歉,咱們當盡力救你。」抱起他身子,放入船艙。

兪蓮舟回到岸上,問無忌道︰「你打他的一掌,叫作什麼掌法?」無忌見他神色嚴峻,心中害怕,哭了起來説道︰「我不是故意打他的,他\dash{}他要放蛇咬我,我怕得很,我\dash{}我怕得很。」兪蓮舟嘆了口氣,抱起他來,伸袖給他拭了拭眼泪,道︰「二伯没怪責你。那人若是放毒蛇來咬我,我出手也不能容情啊。」

兪蓮舟安慰了一陣,無忌才止了啼哭,説道︰「義父説,這是武林中久已失傳的掌法,叫做『降龍十八掌』!」這「降龍十八掌」五字一出口,兪蓮舟和張翠山夫婦盡皆失色,兪蓮舟手一鬆,將無忌放下地來。

原來這「降龍十八掌」,乃是南宋末年丐幫幫主洪七公的威名絶技,洪七公以此一套掌法和「打狗棒法」威震天下,江湖宵小聞名喪膽,成爲武林五奇之一。那「打狗棒法」丐幫幫主代代相傳,至今尚有存留,但「降龍十八掌」自洪七公傳了弟子郭靖之後,郭靖弟子中並無傑出人材,没人學到這路神妙無方的武功。「神鵰大俠」楊過雖是郭靖的子侄輩,但他斷了一臂,已不能學這路必須雙手齊使的掌法。近百年來,武林中前輩已只聞「降龍十八掌」之名,誰也没有見過,想不到無忌竟自從謝遜處學會了。

兪蓮舟兀自不信,道︰「你那打賀老三的,當眞便是『降龍十八掌』中的一招麼?」無忌道︰「義父説,這招叫做『神龍擺尾』。」兪張二人也曾聽師父説起過「降龍十八掌」中的若干名目,似乎確有「神龍擺尾」這一招,至於招式若何,那是誰也不知道的了。不過以無忌這麼小小年紀,隨手反拍一掌,竟將賀老三這江湖好手打得命在垂危,這掌法即使不是「降龍十八掌」,只怕也和「降龍十八掌」差不多了。

張翠山道︰「無忌跟我義兄學藝之時,小弟夫婦都引嫌避開,没想到他竟教了孩児這等早已失傳的神功。」無忌道︰「義父跟我説,他只會得十八掌中的三掌,是跟一位江湖隱士學的,但他總覺得其中的變化有點不大對頭,想是其中眞正奥祕之處,那位隱士也是没有體會到。」兪蓮舟和張翠山想像前輩風儀,都是不禁悠然神往,謝遜連三掌都没學全,而他所領悟到的掌法,無忌更是未必能學到一半,以此七零八落的掌法,已有如許威力,則當年洪七公和郭靖的神威,實是令人心向往之。

殷素素見愛子初試身手,便是一鳴驚人,將來還不是一位震驚武林的高手?心中喜之不盡,也没去留意他師兄弟如何鑽研武功。張翠山道︰「這姓賀的既然在此下手,想必巫山幫定然有接應,咱們不如早些離開這是非之地。」兪蓮舟道︰「正是。我已給他服了『奪命神散』,不知是否能保得性命?」

當下四人回到艙中,只見賀老三呼吸微弱,不停嘔血。張翠山厲聲道︰「無忌,這一次對方使詐行奸,情勢緊迫,原有不是。但以後你若非萬不得已,輕易不可和人動手過招,更加不可任意使用你義父所傳的這三招。」無忌道︰「是,孩児記得。」見父親臉色難看,小眼中泪珠滾來滾去,終於忍耐不住,還是「哇」的一聲,哭了出來。

這時船家已買了酒肉回舟,兪蓮舟命他立即開船。吃過晩飯後,兪蓮舟盤膝坐下,伸手按在賀老三的大椎穴上,潛運本身功力,給他傷治。殷素素微感不滿,心想︰「這位兪二伯實在有些婆婆媽媽,這種江湖道的下流胚子,抛在江中餵魚也就完了。是他自己使鬼域技倆來害人,又不是咱們濫殺無辜。這樣以内功給他療傷,便算治好,你自己是大傷元氣。」那知兪蓮舟運了一個多時辰的功,張翠山便來接替,到天明時,賀老三不再吐血,臉色也漸漸紅潤。

兪蓮舟喜道︰「這條命算是保住啦,不過武功只怕難復。」賀老三千恩萬謝,説道︰「兩位的恩德,姓賀的没齒不忘。我也没臉去見梅幫主。從此隱姓埋名,自耕自食,再也不在江湖上混了。」船到安慶,賀老三拜别三人,上岸去自行請醫補治。

那江船溯江而上,偏又遇著逆風,舟行甚緩,張翠山和師父及諸師兄分别十年,急欲會見,到了安慶後便想捨舟乘馬。兪蓮舟却道︰「五弟,咱們還是坐船的好,雖然遲到數日,但坐在船艙之中,少生事端。今日江湖之上,不知有多少人要査問你義兄的下落。」殷素素道︰「咱們和二伯同行,難道有人敢阻兪二俠的大駕?」兪蓮舟道︰「咱們師兄弟七人聯手,或者没有人能阻得住,單是我和五弟二人,怎敵得過源源而來的高手?何況,只盼此事能善加罷休,又何必多結冤家?」張翠山點頭道︰「二哥説的不錯。」

舟行數日,到得武穴,已是湖北境内。這晩到了福池口,舟子泊了船,準擬過夜,兪蓮舟忽聽得岸上馬嘶聲響,向艙外一張,只見兩騎馬剛好掉轉馬頭,向鎭上馳去。馬上乘客只見到背影,但身手健捷,顯是會家子。他轉頭向張翠山瞧了一眼,説道︰「在這裡只怕要惹是非,咱們連夜走吧。」張翠山道︰「好!」心下好生感激。要知武當七俠自下山行道以來,武藝既高,行事又正,只有旁人望風遠避,從未避過人家,近年來兪蓮舟威名大震,便是崑崙、崆峒這些名門正派的掌門人,見了他也是不敢稍有失禮,但這次見到兩個無名小卒的背影,便不顧在富池口多所逗留,那自是爲了師弟一家三口之故。

當下兪蓮舟將船家叫來,賞了他五兩銀子,命他連夜開船。船家雖然疲倦,但當時五兩銀子已是一筆小財,自是大喜過望,當即拔錨啓航。

這一晩月白風清,無忌已自睡了,兪蓮舟和張翠山夫婦在船頭飲酒賞月,望著浩浩大江,胸襟甚爽。張翠山道︰「恩師百歳大壽轉眼即至,小弟竟能趕上這件武林中罕見的盛事,老天爺可説待我不薄了。」殷素素道︰「就可惜倉卒之間,咱們没能給他老人家好好備一份壽禮。」兪蓮舟笑道︰「弟妹,你知我恩師在七個弟子之中,最喜歡誰?」殷素素笑道︰「他老人家最得意的弟子,自然是你二伯。」兪蓮舟笑道︰「你這句話可是言不由衷,心中明明知道,却故意説錯。咱們師兄弟七人,師父日夕掛在心頭的,便是你這位英俊夫郎。」殷素素心下甚喜,搖頭道︰「我不信。」兪蓮舟道︰「咱七人各有所長,大師哥深通易理,沖淡弘遠。三師弟精明強幹,師父交下來的事,從没錯失過一件。四師弟機智過人。六師弟劍術最精,七師弟近年來專練外門武功,他日内外兼修、剛柔合一,那是非他莫屬\dash{}」

殷素素道︰「二伯你自己呢?」兪蓮舟道︰「我資質愚魯,一無所長,勉強説來,是師傳的本門武功,算我練得最刻苦勤懇些。」殷素素拍手道︰「你是武當七俠中武功第一,自己偏謙虛不肯説。」張翠山道︰「咱們七人之中,向來是二哥武功最好。十年不見,小弟更加望塵莫及。唉,少受恩師十年教誨,小弟是退居末座了。」言下不禁頗有惘悵之意。殷素素道︰「二伯又没顯過武功,你怎知道?」張翠山道︰「那日替賀老三療傷,二哥頃刻之間,替他氣運九轉,這等精湛的内功,我如何能及?」

兪蓮舟道︰「可是七人中文武全才,唯你一人。弟妹,我跟你説一個祕密。五年之間,恩師九十五歳壽誕,師兄弟稱觸祝壽之際,恩師忽然大爲不歡,説道︰『我七個弟子之中,悟性最高,文武雙全,唯有翠山。我原盼他能承受我的衣缽,唉,可惜他福薄,五年來存亡未卜,只怕是凶多吉少了。』你説,師父是不是最喜歡五弟?」張翠山感激無已,眼角微微濕潤。兪蓮舟道︰「現下五弟平安歸來,送給恩師的壽禮,再没比此更重的了。」正説到此處,忽聽得岸上隱隱傳來馬蹄聲響。

那馬蹄聲自東而西,靜夜中聽來分明清晰,共是四乘馬。兪蓮舟三人對望了一眼,心知這四乘馬連夜急馳,多半是與己有関,三人雖然不想惹事,豈又是怕事之輩?當下誰也不提此事,兪蓮舟道︰「我這次下山時,師父正自閉関靜修。盼望咱們上山時,他老人家已經開関。」殷素素道︰「我爹爹昔年跟我説道,他一生只欽佩尊師張眞人和少林派的『見聞智性』四大高僧。張眞人今年百歳高齡,修持之深,當世無有其匹,現下還要閉関,是修練長生不老之術麼?」兪蓮舟道︰「不是,恩師是在精思武功。」殷素素微微一驚,道︰「他老人家武功早已深不可測,還鑽研什麼?難道當世還能有人是他敵手麼?」

兪蓮舟道︰「恩師自九十五歳起,每年都閉関九個月。他老人家言道,我武當派的武功,主要得自一部『九陽眞經』。可是恩師當年聽覺遠祖師背誦這部眞經之時,年紀太小,時候又倉促,記憶不全,因之本門武功終是尚有缺陥。這『九陽眞經』傳自達摩老祖,恩師言道,他越是深思,越覺其中漏洞甚多,似乎這只是半部,該當另有一部『九陰眞經』,方能相輔相成。可是『九陽眞經』他已學得不全,却又到那裡找這部『九陰眞經』去?何況世上是否眞有『九陰眞經』誰也不知。達摩老祖是天竺國不世出的奇人,我恩師的聰明才智,未必在達摩老祖之下,眞經既不可得,難道自己便創制不出?他每年閉関苦思,便是意欲光前裕後,與達摩老祖東西輝映,集天下武學大成。」

張翠山和殷素素聽了,都是慨然讚嘆。兪蓮舟道︰「當年聽得覺遠祖師傳授『九陽眞經』的,共有三人。一是恩師,一是少林派的無色大師,另一位是個女子,那便是峨嵋派的創派祖師郭襄郭女俠。他三人悟性各有不同,根底也大有差異。武功是無色大師最高,郭女俠是郭靖郭大俠和黃蓉黃幫主之女,所學最博,恩師當時武功全無根基,但正因如此,所學反而最爲精純。是以少林、峨嵋、武當三派,一個得其『高』,一個得其『博』,一個得其『純』。三派武功各有所長,但也可説各有所短。」

殷素素道︰「那麼這位覺遠祖師,武功之高,該是百世難逢了。」兪蓮舟道︰「不!覺遠祖師是全然不會武功的。他在少林寺藏經閣中監管藏經,這位祖師是個書獃子,無經不讀,無經不背。他無意中看到『九陽眞經』,便如金剛經、法華經一般記在心中,至於經中所包藏的博大精深的武學妙旨,他却全然不解。」於是將『九陽眞經』如何失落,從此湮没無聞的故事説了給她聽。這事張翠山早已聽師父説過,殷素素却是第一次聽到,極感興趣。

兪蓮舟平日沉默寡言,有時接連數日可以一句話也不説,但自和張翠山久别重逢之下,欣喜逾常,談鋒也健起起來。他和殷素素相處十餘日後,覺她本性其實不壞,所謂近墨者黑、近朱者赤,自幼耳濡目染,所見所聞盡是邪惡之事,這纔善惡不分,任性殺戳,但和張翠山成婚十年,氣質已有有變化,因之初見時對她的不滿之情,已逐日消除,覺她坦誠率眞,比之名門正派中某些迂腐自大之士,反而更具眞性情。

張翠山難得師哥好興緻,正想問他師父所鑽研的武功進展如何,忽聽得馬蹄聲響,又自東方隱隱傳來,不久蹄聲從舟旁掠過,向西而去。張翠山只作没聽見,説道︰「二哥,倘若恩師邀請少林、峨嵋兩派高手,共同研討,截長補短,三派武功都可大進。」兪蓮舟伸手在大腿上一拍,道︰「照啊,師父説你是將來承受他衣缽門戸之人,果眞一點也不錯。」

張翠山道︰「恩師只因小弟不在耳邉,這纔時致思念。浪子若是遠遊不歸,在慈母心中,却比隨侍在側的孝子更加好了。其實小弟此時的修爲,别説和大哥、二哥、四哥相比是望塵莫及,便是六弟、七弟,也定比小弟強勝得多。」兪蓮舟搖頭道︰「不然,目下以武功而論,自是你不及我。但恩師的衣缽傳人,負有昌大武學的重任。恩師常自言道,天下如此之大,武當一派是榮是辱,何足道哉?但若能精研武學奥祕,愼擇傳人,使正人君子的武功,非邪惡小人所能及,再進而相結天下義士,驅除韃虜,還我河山,這纔算是盡了我輩武學之士的本分。因此恩師的衣缽傳人,首重心術,次重悟性。説到心術,我師兄弟七人無甚分别,悟性却是以五弟爲高。」

張翠山搖手道︰「我想那是恩師思念小弟,一時興到之言。就算恩師眞有此意,小弟也是萬萬不敢承當。」兪蓮舟微微一笑,道︰「弟妹,你去護著無忌,别讓他受了驚嚇,外面的事有我和五弟料理。」殷素素極目遠眺,不見有何動靜,正遲疑間,兪蓮舟道︰「岸上灌木之中,刀光閃爍,伏得有人。前邉蘆葦中必有敵舟。」殷素素遊目四顧,但見四下裡靜悄悄的絶無異狀,心想只怕是你眼花了吧?

忽聽兪蓮舟朗聲説道︰「武當山兪二、張五,道經貴地,請恕禮數不周。那一位朋友若是有興,請上船來共飲一杯如何?」他這幾句話一完,忽聽得蘆葦中槳聲響動,六艘小船飛也似的划了出來,一字排開,攔在江心。一艘船上嗚的一聲,射出一枝響箭,南岸一排矮樹中竄出十餘個勁裝漢子,一色的黑衣,手中各持兵刃,臉上却蒙了黑帕,只露出眼睛。

殷素素心好生佩服︰「這位二伯名不虛傳,當眞了得。」眼見敵人甚衆,急忙回進艙中,只見無忌已然驚醒。殷素素替他穿好衣服,低聲道︰「乖孩児,不用怕。」

兪蓮舟又道︰「前面當衆的是那一位朋友,武當兪二、張五問好。」但六艘小船中除了後艘的槳手之外,不見有人出來,更没有人答話。兪蓮舟忽地省悟,叫聲︰「不好!」翻身入江中。他自幼生長江南水鄕,水性極佳,剛一下江,只見四個漢子手持利錐,潛水而來,顯是想錐破船底,將舟中各人生擒活捉。

兪蓮舟微微冷笑,隱身船側,待四人游近,雙手分别點出,已中兩人穴道,跟著踢出一脚踢中了第三人腰間的「志室穴」。第四人吃了一驚,兪蓮舟左臂一長,抓住他的小腿,甩上船來。他想那三人穴道被點,勢必要溺死在大江之中,於是一一掀起,抛在船頭,這纔翻身上船。那第四個漢子在船頭打了個滾,縱身躍起,一錐便向張翠山胸口刺落。張翠山見他武功平常,也不閃避,左手一探,已抓住他拿錐的手腕,跟著左肘向外輕抵,撞中他胸口穴道。那漢子一聲也没哼出,便此摔倒。兪蓮舟道︰「岸上的似乎有幾個好手,禮數已到,不理他們,衝下去吧!」張翠山點了點頭,吩咐船家只管開船。只是逆風逆水,舟行甚緩。慢慢駛近那六艘小船時,兪蓮舟提起那四個漢子,拍開他們身上穴道,擲了過去。但説也奇怪,對方舟中固然没出聲,岸上那十餘個黑衣人也是悄無聲無色,竟如個個都是啞巴一般。那四個潛水的漢子鑽入艙中,不再現身。

座船剛和六艘小舟並行,便要掠舟而過時,一艘小舟上的一名槳手突然右手揚了兩下,砰砰兩聲,木屑紛飛,座船的舵已然炸毀,船身登時橫了過來。原來那槳手擲出的是兩枚漁家炸魚用的漁炮,只是製得特大,多袋火藥,因此炸力甚強。兪蓮舟不動聲色,身形一起,輕輕躍到了對方小舟之上,他藝高人膽大,仍是一雙空手。

小舟上的槳手手持大槳,眼望前面,對兪蓮舟躍上船來竟是毫不理會。兪蓮舟喝道︰「是誰擲的漁炮?」那槳手木然不答,兪蓮舟知他裝聾作啞,搶進艙去,只見艙中對坐著兩個漢子,見他進艙,仍是一動不動,絲毫不現迎敵之意。兪蓮舟一把掀住他的頭頸,提了起來,喝道︰「你們瓢把子呢?」那人閉目不答。兪蓮舟是武林一流高手身份,不願以武力逼問,當即回到後梢,只見張翠山和殷素素也已抱著無忌過來小舟。

兪蓮舟奪過槳手中的木槳,逆水上划,只划得幾下,殷素素叫道︰「毛賊放水!」但見船艙中水湧上來。原來小舟中各人早有預備,拔開艙底木塞,放水入船。兪蓮舟躍到第二艘小舟時,只見舟中也已小半船是水。他回頭説道︰「五弟,既是非要咱們上岸不可,那就上去吧!」那六艘小舟顯是事先安排好了,作爲請客上岸的跳板,三人帶同無忌,躍上岸去。

岸上十餘名蒙著臉的黑衣漢子早就排成了個半圓形,將四人圍在弧形之内。兪蓮舟見這十餘人手中所持大都均是長劍,另一小半則或持雙刀,或握軟鞭,没一個用沉重兵刃。他抱臂而立,自左而右的掃視一遍,神色冷然,並不説話。中間一個黑衣漢子右手一擺,衆人忽然向兩旁分開,各人微微躬身,倒握劍柄,劍尖向地,抱拳行禮,讓出一條路來。兪蓮舟還了一禮,昂然而過。這一干人待兪蓮舟走出圏子,忽地向中間一合,封住了道路,將張翠山等三人圍住,青光閃爍,劍尖一齊挺起。

張翠山哈哈一笑,説道︰「各位原來是衝著張某人而來。擺下這等大陣仗,可將張翠山忒也瞧得重了。」中間那黑衣漢子微一遲疑,垂下劍尖,又讓開了道路。張翠山道︰「素素,你先走!」殷素素謝遜抱著無忌正要走出,猛地裡風聲響動,五柄長劍一齊指住了無忌。殷素素吃了一驚,急忙倒退,那五人跟著踏步而前,劍尖不住顫動,始終不離無忌身周尺許。兪蓮舟雙足一點,倏地從人叢之外飛越而入,雙手連拍四拍,每一下都拍在一個黑衣人的手腕之上,只見四柄指著無忌的長劍一一飛入半空。這四下拍擊實在來得太快,四柄長劍竟似同時飛上。他左手跟著反手擒拿,抓住了第五人的手腕,但覺著手處柔軟滑膩,似是女子之手。他這一抓之時,中指已順手點了那人腕上穴道,急忙放開,那人已是手腕麻庳,{\upstsl{噹}}的一聲,長劍掉在地下。

那五人長劍脱手,急忙退開,月光下只見青光閃閃,又是兩柄長劍刺了過來,但見劍刃平刺,鋒口向著左右,每人使的都是一招「大漠平沙」。兪蓮舟心道︰「這是崑崙劍法,原來這批人是崑崙派的。」待劍尖離胸口將近三寸,眼見敵招用老,突然胸口一縮,雙臂迴環,左手食指和右手食指同時擊在劍刃的平面之上。

這兩下拍擊,看似輕易,却是用上了武當心法,乃是他一身功力之所聚,照理對方長劍非撒手不可,豈知手指和劍刃相觸,陡覺劍刃上傳出一股柔勁,竟將他這一擊之力化解了一小半,長劍並未脱手。但那二人究是抵擋不住,騰騰退出三步,一人站立不定,摔倒在地,另一人「啊喲」一聲,吐出一口鮮血。

自六艘小舟橫江以來,對方始終没一人出過聲,這時「啊喲」一聲驚呼,聲音柔脆,聽得出是女子聲音。

中間那黑衣人見兪蓮舟這等厲害,左手一擺,各人轉身便走,頃刻間消失在灌木之後。但見這一干人大半身材苗條,顯是穿著男人裝束的女人。兪蓮舟朗聲道︰「兪二張五,多多拜上鐵琴先生,請恕無禮之罪。」

那些黑衣人並不答話,隱隱聽得有人輕聲一笑,仍是女子的聲音。殷素素將無忌放下地來,仍是緊緊握住他手,説道︰「這些大半是女子啊。二伯,她們都是崑崙派的麼?」兪蓮舟道︰「不,是峨嵋派的。」張翠山奇道︰「峨嵋派的?你怎説多拜上『鐵琴先生』?」兪蓮舟嘆了口氣道︰「她們自始至終,不出一聲,臉上又以黑帕蒙住,那自是不肯以眞面目示人了。她們以五劍指住無忌,那是崑崙派的『寒梅劍陣』。後來兩個人平劍刺我,又用崑崙派的一招『大漠平沙』。她們既然冒充崑崙派,我便將錯就錯,提一提崑崙的掌門鐵琴先生。」

殷素素道︰「你怎知她們是峨嵋派的?有人認出了麼?」兪蓮舟道︰「不,這些人功力都不算深,以是當今峨嵋掌門滅絶師太的徒孫一輩,那是峨嵋的第四代弟子了,我不認得她們。但她們以柔勁化解我指擊劍刃的功夫,確是峨嵋心法。要學别派的數招陣式,那並不難,但一出到内勁,那就非顯示眞相不可。」張翠山點頭道︰「二哥以指擊劍,她們還是撤劍的好,受傷倒輕,峨嵋派的内功好是極好的,只是未到適當功行便貿然運行,一遇上高手,便吃大虧,二哥倘若眞將她們當作敵人,這兩個女娃娃早就屍橫就地了。可是峨嵋派跟咱們向來客客氣氣的啊。」

兪蓮舟道︰「恩師少年之時,受過峨嵋派開派祖師郭襄女俠的好處,因此他老人家諄諄告誡,決不可得罪了峨嵋門下弟子,以保昔年的香火之情。我以指擊劍,發覺到對方内勁不對時,收勢已然不及,終於傷了二人。雖然這是無心之失,總是違了恩師的訓示。」殷素素笑道︰「好在你最後説是向鐵琴先生請罪,不算是正面得罪了峨嵋派。」

這時他們的座船轉了船舵,早已順水流向下流,影蹤不見,那六艘小舟均已沉没,舟中的槳手濕淋淋地一個個爬上岸來。殷素素道︰「這些都是峨嵋派的麼?」兪蓮舟低聲道︰「多半是巢湖的糧船幫。」殷素素望了一眼地下明晃晃的五柄長劍,俯身想拾起瞧瞧,兪蓮舟道︰「别動她們的兵刃,倘若劍上刻有名字,咱們以後便無法假作不知。這就走吧。」殷素素這時對這位二伯敬服得五體投地,應道︰「是!」便擕了無忌之手,走向江岸的大道。

經過那叢灌木,無忌喜呼起來︰「有馬,有馬!」只見十餘丈外的一株大柳樹繫著三匹駿馬。無忌在冰火島上從未見過馬匹,來到中土後,一直想騎一騎馬,只是一路乘船,始終未得其便。四個人走近馬匹,却見柳葉上釘著一張紙條,張翠山取下一看,見紙上冩道︰「敬贈坐騎三匹,以謝毀舟之罪。」兪蓮舟道︰「她們倒也客氣得很。」當下解下馬匹,三個分别乘坐。無忌坐在母親身前,大是興奮。

張翠山道︰「反正咱們形跡已露,坐船騎馬都是一般。」兪蓮舟道︰「不錯。前邉道上必定尚有波折,倘若逼不得已要出手,下手不可太重。」他適纔無意傷了兩名峨嵋門下弟子,心中一直耿耿不安。殷素素好生慚愧,心想︰「二伯只不過下手重了一些,本意亦非傷人,只是逼對方撤劍,她們自行硬挺,這纔受傷。比之我當年肆意殺這許多少林門人,過錯之輕重,眞是不可同日而語了。一身作歹一身當,以後不可再讓二伯爲難。」於是説道︰「二伯,這干人全是衝著咱倆夫婦而來,對你可恭敬得很。前面要是再有阻攔,由弟妹打發便是,倘眞不行,再請你出手相援。」

兪蓮舟道︰「你這話可見外了。咱兄弟同生共死,分什麼彼此?」殷素素不便再説,只問︰「他們明知二伯跟咱夫婦在一起,怎地只派些第四代的弟子來攔截?」

\chapter{攔途截劫}

兪蓮舟道︰「想是事急之際,不及調動人手。」張翠山見了適纔峨嵋派衆女的所爲,料到是爲了尋問謝遜的下落而來,説道︰「原來義兄跟峨嵋派也結下了樑子,我在島上却没聽他説起過。」兪蓮舟嘆道︰「峨嵋派門規極嚴,派中又大多是女弟子,滅絶師太自來不許她的弟子行走江湖,若非出家爲尼,荒山靜修;便是婚後相夫教子,深藏不露。這一次峨嵋派竟然遣人來和白眉教爲難,咱們當時略感詫異。直至最近方始明白了其中緣故,原來河南蘭封金瓜錘方評方老英雄有一晩突然被害,牆上留下了『殺人者,混元霹靂手成崑』十一個血字。」殷素素道︰「那方評是峨嵋派的麼?」

兪蓮舟道︰「不是。」他頓了一頓,道︰「前輩的私事,咱們原不該背後談論。只知滅絶師太少年時是武林中出名的美人,後來她忽然出家爲尼,方老英雄便自斷一臂,終身不娶。」張翠山和殷素素同時「哦」了一聲,明白滅絶師太和方老英雄少年時想是一對情侶,不知爲了什麼緣故無法成婚,於是一個出家,一個便斷臂以報。臨到老來,方評竟爲謝遜殺害,滅絶師太自非替他報仇不可。

無忌忽然問道︰「二伯,那方老英雄是好人還是壞人?」兪蓮舟道︰「方老英雄斷臂後種田讀書,從不和人交往,自然不是壞人。」無忌道︰「咳,義父這般胡亂殺人,那就不該了。」兪蓮舟大喜,輕舒猿臂,將他從殷素素身前抱了過來,撫著他頭,説道︰「孩子!你知道不能胡亂殺人,二伯很是歡喜。人死不能復生,便是罪孼深種、窮凶極惡之輩,也不能隨便下手殺他,須得讓他有一條悔改之路。」無忌道︰「二伯,我求你一件事。」兪蓮舟道︰「什麼?」無忌道︰「倘若他們找到了義父,你叫他們别殺他。因爲義父眼睛瞎了,打他們不過。」

兪蓮舟沉吟半晌,道︰「這件事我答應不了。但我自己,決計不殺他便是。」無忌呆呆不語,小眼中垂下泪來。

天明時四人到了一個市鎭,在客店中睡了半日,午後又再趕路。有時殷素素和丈夫共乘一騎,讓無忌一試控韁馳騁之樂。無忌究是孩子心情,騎了一會馬,爲謝遜耽憂的心事也便淡忘了。

一路無話,不久便過了漢口。這一日午後,將到安陸,忽見大路上有十餘名客商急奔下來,見了兪蓮舟等四人,急忙搖手,叫道︰「快回頭,快回頭,前面有韃子兵殺人擄掠。」一人對殷素素道︰「你這娘子忒也大膽,碰到了韃子兵可不是玩的。」兪蓮舟道︰「有多少韃子?」一人道︰「十來個,兇惡得緊哩。」説著便向東逃竄而去。

武當七俠生平最恨的是元兵殘害良民。張三丰平素督訓甚嚴,門人不許輕易和人動手,但若是殺傷正在作惡的元兵,非但不加責備,反而大爲獎飾。因此武當七俠若是遇上大隊元兵,那只有走避,若是見少數元兵行兇,往往便下手除去。這時聽説只有十來個元兵,心想正好爲民除害,於是便縱馬迎了上去。

行出三里,果聽得前面有慘呼之聲。張翠山一馬當先,但見十餘名元兵手執鋼刀長矛,正攔住了數十個百姓,大肆劫掠。地下鮮血淋漓。已有七八個人身首異處。只見一個元兵提起一個三四歳的孩子,用力一脚,將他高高踢起,那孩子在半空中大聲慘呼,落下來時另一個元兵又是一脚踢上,將他如同皮球般踢來踢去。只踢得幾脚,那孩子早没了聲息,已然斃命。張翠山怒極,從馬背上躍飛而起,人未落地,砰的一拳,已擊在一個伸脚欲踢孩子的元兵胸口。那元兵哼也没哼一聲,軟癱在地,另一個元兵挺起長矛,往張翠山背心刺到。

無忌驚叫︰「爹爹小心!」張翠山回過身來,笑道︰「你瞧爹爹打這韃子兵。」但見長矛離胸口已不到半尺,左手倏地翻轉,抓住矛桿,跟著向前一送,矛柄撞在那元兵胸口。那元兵大叫一聲,翻倒在地,眼見是不活了。

衆元兵見張翠山如此勇猛,發一聲喊,四下裡圍了上來。殷素素縱身下馬,搶著元兵手中長刀,砍翻了兩個。衆元兵見勢頭不對,落荒逃竄,但這些元兵兇惡成性,便在逃走之時,還是揮刀亂殺百姓。兪蓮舟大怒,叫道︰「别讓韃子走了。」急奔向西,攔住四名元兵的去路,張翠山和殷素素也分頭攔截。三人均知元兵雖然兇惡,武功都是平常,無忌比他們要強得多,不用分心照顧。

無忌跳下馬來,見二伯和父母縱躍如飛,將十多名元兵逼了回來,拍手叫道︰「好,好!」突然之間,那名被張翠山用矛桿撞暈的元兵霍地躍起,一伸臂便抱住了無忌腰間。無忌吃了一驚,反手一招「神龍擺尾」,拍的一聲,打在那元兵的胸口。他見二伯和父母追殺元兵下手並不留情,因之這一掌也使了十成力。那知這元兵輕輕哼了一聲,身子晃也没有晃,翻身便上弓馬背,縱馬疾馳。

兪蓮舟和張翠山夫婦齊聲叫喊,追了過來。兪蓮舟兩個起落,已奔到馬後,左手拍出一掌,身隨掌起,按到了那元兵後心。那元兵竟不回頭,倏地反擊一掌。波的一聲響,雙掌相交,兪蓮舟只覺對方掌力猶如排山倒海相似,胸口熱血翻騰,身子晃了幾晃,倒退了三步,但那元兵的坐騎也吃不住兪蓮舟這一掌的震力,前足突然跪地。那元兵抱著無忌,順勢向前一躍,已縱出丈餘,展開輕身功夫,霎息間已奔出數十丈。

張翠山見二哥臉色蒼白,受傷竟是不輕,急忙扶住。殷素素心繫愛子,没命的追趕,但那元兵輕身功夫高極,越追越遠,到後來只見遠處大道上一個黑點,轉了一個彎,再也瞧不到了。殷素素怎肯死心,只是疾追。她不再想到這元兵既能掌傷兪蓮舟,自己便是追上了,也是決非他的敵手,她心中只是存著一個念頭︰「便是性命不保,也要將無忌奪回。」

兪蓮舟低聲道︰「快叫弟妹回來,從長\dash{}從長計議。」張翠山挺起長矛,將身前兩個元兵刺死,説道︰「你傷得怎樣?」兪蓮舟道︰「不礙事,先將弟妹叫回來要緊。」張翠山生怕剩下來的元兵之中尚有高手在内,自己若是一走開,他們便會過來向兪蓮舟下手,當下四下裡追逐,一個個的點倒砍翻,這纔拉住一匹馬來,向西追去。

趕出十餘里,只見殷素素披頭散髮,兀自狂奔,但脚步蹣跚,顯已筋疲力盡。張翠山俯身將她抱上馬鞍。殷素素手指面前,哭道︰「不見了,追不到啦,追不到啦。」雙眼一翻,已自暈了過去。張翠山終是掛念兪蓮舟的安危,心想︰「該當先顧二哥,再顧無忌。」於是勒轉馬頭,奔了回來。只見三個元兵,兩個持矛,一個挺刀,圍著兪蓮舟。兪蓮舟倚樹而坐,那三個元兵始終不敢上前。張翠山怒喝︰「韃子納下命來!」長矛抖處,搠翻了兩個,另一個轉身便逃。張翠山大喝一聲,長矛擲出,他児子被擄,義兄受傷,妻子昏暈,心中悲傷已極,這一擲出盡了全力,便聽長矛破空,嗚嗚作聲,拍的一響,將那元兵釘在地下。

殷素素悠悠醒轉,叫道︰「無忌,無忌!」兪蓮舟閉目打坐,調勻氣息,再從懷中取出一枚「太乙奪命丹」服下,慘白的臉色漸轉紅潤,睜開眼,低聲道︰「好厲害的掌力!」

張翠山聽師兄一開口説話,知道性命已然無礙,這纔放心,但仍是不敢跟他言語。兪蓮舟緩緩站起身來,低聲道︰「無影無蹤了吧?」殷素素哭道︰「二伯,怎\dash{}怎麼是好?」兪蓮舟道︰「你放心,無忌没事,這人武功高得很,決不會傷害小孩。」殷素素道︰「可是\dash{}可是他擄了無忌無忌去啦。」兪蓮舟點了點頭,伸手扶住張翠山肩頭,閉目沉思。

隔了好一會,兪蓮舟睜開眼來,説道︰「我想不出那人是何門派,咱們上山去問師父。」殷素素大急,説道︰「二伯,怎生想個法児,先行奪回無忌才是,那人是何門派,不妨日後再問。」兪蓮舟搖了搖頭。張翠山道︰「素妹,眼下二哥身受重傷,那人武功又如此高強,咱們便是尋到了他,也是無可奈何。」殷素素急道︰「難道便如此罷了不成?」張翠山道︰「咱們不用去尋他,他自會來尋咱們。」殷素素原是個聰明女子,只因愛子被擄,這纔驚惶失措,這時一怔之下,已然明白。那元兵武功如此深湛,連兪蓮舟也被他一掌震傷,自然是假扮的。他打傷兪蓮舟後,若要取他夫婦二人性命,可説是易如反掌,但只將無忌擄去,其用意是在逼問謝遜的下落。

當下張翠山將師兄抱上馬背,自己拉著馬韁,三騎馬緩緩而行。到了安陸,找一家小客店歇了,張翠山吩咐店伴送來飯菜後,就此閉戸不出,生怕遇上元兵,又生事端。他三人在途中殺死這十餘個元兵後,大隊元兵過得數日便會來大舉殘殺劫掠,報復洩忿,附近百姓不知將有多少遭殃,但當時他三人遇上這等不平之事,在勢又不能袖手不顧。這正是亡國之慘,莽莽神州,無人能免此劫難。

兪蓮舟潛運内力,在週身穴道中流轉療傷,張翠山坐在一旁守護。殷素素倚在椅上,又那裡睡得著?到得中夜,兪蓮舟站起身來,在室中緩緩走了三轉,舒展筋骨,説道︰「五弟,我一生之中,除了恩師之外,從未遇到這樣的高手。」

當時張翠山長矛隨手一撞,便將那人撞暈,那人自是裝假,其時三人誰也没留心他的身形相貌,此刻回想起來,那人依稀似是滿腮虯髯,和尋常元兵也没什麼分别。殷素素終是記掛愛児,道︰「他擄去無忌,定是逼問我義兄的下落,不知無忌肯不肯説。」張翠山昂然道︰「無忌倘若説了出來,還能是我們孩児嗎?」殷素素道︰「對!他是定不會説的。」突然之間,哇的一聲哭了出來。張翠山忙問︰「怎麼啦?」殷素素哽咽道︰「無忌不説,那惡賊\dash{}那惡賊會逼他打他,説不定還會用\dash{}用毒刑。」

張翠山和兪蓮舟同時嘆了口氣道︰「玉不琢,不成器,讓他經歷些艱難困苦,未必没有好處。」他話是這麼説,但想到愛子此時不免宛轉呻吟,正在忍受極大的痛楚,心中自是不勝悲憤憐惜。然而倘若他這時正是平平安安的睡著呢?那一定是已將謝遜的下落説了出來,如此負恩無義,却比挨受毒刑又壞得多。張翠山心想︰「寧可他即刻死了,也勝於做一個無義小人。」轉眼望了妻子一眼,只見她目光中流露出哀苦乞憐的神色來,驀地一驚︰「那惡賊若果以無忌的性命相脅,説不定素妹便要屈服。」説道︰「二哥,你好些了麼?」

他師兄弟自幼同門學藝,一句話一個眼色之間,往往便可心意相通。兪蓮舟一瞧他夫婦二人的眼色,已明白張翠山的用意,知他是耽心那人逼問無忌無效,挾著他追來,殷素素未必能忍受眼睜睜的瞧著無忌被殺,當下説道︰「好,咱們連夜趕路。」

三人付了房飯錢,乘黑繞道,儘揀荒僻小路而行。三人最害怕的,倒不是那人追來下手殺了自己,而是怕他在自己眼前,將各種各樣的慘酷的手段加在無忌身子。

如此朝宿宵行,差幸一路無事。但殷素素心懸愛子,山中夜騎,又受了風露,忽然生起病來。張翠山僱了兩輛騾車,讓兪蓮舟和殷素素分别乘坐,自己騎馬在旁護送。這日過了襄陽,到太平店鎭上一家客店投宿。

張翠山安頓好了師兄,正要回房,忽然一條漢子抓開門帘,闖進房來。這漢子身穿青布短衫褲,手中提著一根馬鞭,一身打扮便像個趕脚的車夫。他向兪蓮舟和張翠山瞪了一眼,冷笑一聲,轉身便走。張翠山知他不懷好意,心下惱他無禮,眼見那漢子摔下的門帘盪向身前,左手抓住門帘暗運内勁,向外一送。那門帘的下{\upstsl{襬}}飛了起來,拍的一聲,結結實實打在他的背心。那漢子身子一晃,跌了個狗吃屎,爬起身來,喝道︰「武當派的小賊,死到臨頭,還在逞兇!」口中這般説,脚下却是不敢停留,逕往外走,但見他步履踉蹌,適纔吃門帘這麼一擊,受創竟是不輕。

兪蓮舟瞧在眼裡,並不説話。到得傍晩,張翠山道︰「二哥,咱們動身吧!」兪蓮舟道︰「不,今晩不走,明天一早再走。」張翠山微一轉念,已明白了他的心意,登時豪氣勃發,説道︰「不錯!此處離本山已不過兩日之程。咱師兄弟再不濟,也不能墮了師門的威風。在武當山脚下,兀自朝宿晩行的趕夜路避人,那算什麼話?」兪蓮舟微笑道︰「反正行藏已露,且瞧瞧武當派的子弟如何死到臨頭。」

當下兩人一齊走到張翠山房中,並肩坐在坑上,閉目打坐。這一晩紙窗之外,屋頂之上,總有七八個人來來去去的窺伺,但盡是心憚武當派的威名,不敢進房滋擾。殷素素昏昏沉沉的睡著,兪張二人也不去理會屋外的敵人。

次日用過早飯後動身。兪蓮舟雖然坐在騾車之中,却叫車夫去了車廂的四壁,四邉空蕩蕩,便於觀看。只走出太平店鎭甸數里,便有三乘馬自東方追了上來,跟在騾車之後,相距十餘丈,不即不離的跟著。再走數里,只見前面道上有四個和騎者候在道邉,待兪蓮舟一行人過去,四乘馬便跟著後面。數里之後,又有四乘馬加入,前後已共有十一人。趕車的驚慌起來,悄聲對張翠山道︰「客官,這些人路道不正,遮莫是強人?須得小心在意。」張翠山道︰「不用怕,不是來搶錢的。」

在中午打尖之處,又多了六個人。這些人打扮各各不同,有的衣飾富麗,有的却似販夫走卒,但人人身上均帶兵刃。一干人隻聲不出,聽不出口音,但大都身材瘦小皮色黝黑,似乎來自南方。到得午後,已增到二十一人。有幾個大膽的縱馬逼近,到距騾車兩三丈處,這纔勒馬不前。兪蓮舟在車中只管閉目養神,正眼也不瞧他們一眼。

傍晩時分,迎面兩乘馬奔了下來。但見當先一匹馬上騎著個長鬚飄飄的老者,第二騎的乘客却是個艷裝少婦。那老者空著兩手,少婦左手中提著一對雙刀。兩騎馬在道路當中一攔,擋住了去路。

張翠山強忍怒氣,在馬背上抱拳説道︰「武當山兪二張五這廂有禮,不敢請問老爺子尊姓大名。」那老者皮笑肉不笑的微微一笑,問道︰「金毛獅王謝遜在那裡?你只須説了出來,咱們決不跟武當弟子爲難。」張翠山道︰「此事在下不敢作主,須得先向恩師請示。」那老者道︰「兪二受傷,張五落單。你孤身一人,不是咱們這許多的敵手。」説著伸手腰間,取出一對判官筆來。只見那判官筆的筆尖鑄作蛇頭之形。張翠山外號叫作「銀鉤鐵劃」,雙手兵刃之中,有一件便是判官筆,因此武林中使判官筆的點穴名家,他無一不知,一見這對蛇頭雙筆,心中一驚。

他當年曾聽師父説過,高麗有一派使判官筆的,筆頭鑄作蛇形,其招數和點穴手法,和中土的大不相同,大抵是取毒蛇的陰柔毒辣之性,招術滑溜狠惡,這一派美其名曰「神龍派」派中出名的高手只記得姓泉,名字叫什麼却連師父也不知道。於是抱拳説道︰「前輩是高麗神龍派的麼?不知和泉老爺子是如何稱呼?」那老人微微一驚,心想︰「你也不過三十來歳年紀,却恁地見識廣博,知道我的來歷。」原來這老者便是高麗神龍派的掌門人,名叫泉建男,是嶺南「三江幫」幫主卑詞厚禮,從高麗聘請而來。他到中土已有數年,却從未出過手,想不到「三江幫」行事隱祕,但他一露面便給張翠山識破,於是蛇頭雙筆一擺,道︰「老夫便是泉建男。」張翠山道︰「高麗神龍派跟中土武林向無交往,不知武當派如何得罪了泉老英雄。還請明示。」泉建男又是皮笑肉不笑的臉上筋肉一動,説道︰「老夫和閣下無冤無仇,咱們高麗人也知道中原有個武當派,武當七俠是行俠仗義的好男子。老夫只問閣下一句話,金毛獅王謝遜躱在那裡。」

他這番話雖然不算無禮,但詞鋒咄咄逼人,同時判官筆這麼一擺,跟在騾車之後的人衆便四下分散,團團圍了上來,顯是若不明言謝遜的下落,便只動武一途。張翠山道︰「若是在下不願説呢?」泉建男道︰「張五俠武藝超群,咱們人數雖多,自量也留你不住。但兪二俠身上負傷,尊夫人正在病中,咱們有此良機,只好乘人之危,要將兩位留下。張五俠自己請便吧。」他的中國話咬字不準,聲音尖鋭,聽來加倍刺耳。

張五俠聽他説得這般無恥,「乘人之危」四個字自己先説了出來,説道︰「好,既是如此,在下便領教領教高麗武學的高招。若是泉老英雄讓得在下一招半式,那便如何?」泉建男笑道︰「若是我輸了,大夥児便一擁而上。咱們可不講究什麼單打獨鬥那一套。倘若武當派人多,你們也可倚多爲勝啊。從前隋陽帝、唐太宗、唐高宗侵我高麗,那一次不是以數十萬大軍攻我數萬兵馬。自來相鬥,都是人多的佔便宜。」

張翠山心知今日之事,多説無益,只有憑手上功夫以決勝負,若是能將他擒住,作爲要脅,當可逼他手下人衆不敢侵犯二哥和素素。於是身形一起,輕飄飄的落下馬背,左足著地,左手已握住爛銀虎頭鉤,右手握著鑌鐵判官筆,説道︰「你是客人,請進招吧!」泉建男也躍下馬來,雙筆互擊,錚的一聲,右筆虛點,左筆尚未遞出,身子已繞到張翠山側方。張翠山尋思︰「今日我是爲義兄的安危而戰,素素跟我夫婦一體,她和義兄也有金蘭之誼,爲他喪命,那也罷了。但二哥跟義兄素不相識,若是爲了義兄而使二哥蒙受恥辱,那是萬萬不該。」當下心中打定了主意,見泉建男右手蛇頭筆點出,伸鉤一格,手上只使了二成力。鉤筆相交,張翠山身子微微一晃。泉建男大喜,心想︰「三江幫那些人把武當七俠説得如何了得,原來也不過如此。想是中原武人要面子,將本國人士説得加倍厲害些。」當下左手筆跟著三招遞出。張翠山左支右絀,勉力擋架,便是還了一鉤一筆,也是虛軟乏勁。泉建男此時改了主意,不再倚多爲勝,心想今日將武當七俠中的張五俠收拾下來,自己來到中土便是一戰成名,三江幫全幫上下,對自己更加要括目相看,當下雙筆飛舞,招招向張翠山的要害點去。

張翠山將門戸守得極嚴密,一面凝神細看對方的招數,但見他出招輕靈,筆上頗具韌力,所點的穴道偏重下三路及背心,和中土各派的點穴名手,武功果然大不相同。

再鬥一陣,但見他左手判官筆所點,都是背心自「靈台穴」以下的各穴,自靈台、至陽、筋縮、中樞、脊中、懸樞、命門、陽関、腰兪、以至尾閭背處的長強穴;右手判官筆所點,則是腰腿上各穴,自五樞、維道、居膠、環跳、風市、中瀆以至小腿上的陽陵泉。張翠山心下了然,他左手筆專點「督脈諸穴」,右手筆專點「足少陽膽經諸穴」,看似繁複,其實大有理路可尋,暗想︰「當年師傅曾説,高麗神龍派的點穴功夫專走偏門,雖然狠辣,並不足畏。今日一見,果然不錯。」他一摸清對方招式,銀鉤鐵筆雖然上下揮舞,其實裝模作樣,只須護住督脈諸穴及足少陽膽經諸穴,其餘身上穴道,不必理會。泉建男愈鬥精神愈長,大聲{\upstsl{吆}}喝,威風凜凜,張翠山心道︰「憑著這點點武功,居然也到武當山脚下來撒野?」突然間左手銀鉤使招「龍」字訣中的一鉤,嗤的一響,鉤中了泉建男右腿的風市穴。泉建男「啊」的一聲,右腿跪地。張翠山右手筆電光石火般連連顫動,自他靈台穴一路順勢直下,使的是「鋒」字訣中最後的一直,便如書法中的顫筆,至陽、節縮,直至長強,在他「督脈」的每一處穴道上都點了一下。這一筆下來,疾如星火,氣吞牛斗,泉建男那裡還能動彈?這一路所點各穴,正是泉建男畢生所鑽研的諸處穴道,他身子固然不動,心中更是嗒然若喪,暗想︰「罷了,罷了!對方縱是個泥塑木彫之輩,我也不能一口氣連點他十處穴道。我便是做他徒弟,也差得遠了。」張翠山銀鉤鉤尖指住泉建男咽喉,喝道︰「各位且請退開!在下請泉老英雄送到武當山脚下,便解他穴道放還!」心想這些人看來都是他的下屬,定當心有所忌,就此退開。那知那艷裝少婦突然舉起雙刀,叫道︰「併肩子齊上,把騾車扣了。」張翠山喝道︰「誰敢上來,我先將這人斃了!」那少婦冷笑一聲,叫道︰「大夥児上啊!」縱馬舞刀衝上,竟是絲毫没將泉建男放在心上。原來這少婦是三江幫中的一位舵主,他們這次大舉出動,用意在劫持兪蓮舟和殷素素,逼問謝遜的下落。泉建男不過是三江幫的客卿,既然不能爲本幫效力,便是死在敵人手下,那也殊不足惜。

張翠山吃了一驚,眼見便是殺了泉建男仍是無濟於事,只見七八名漢子搶到殷素素的騾車前,七八名漢子搶到兪蓮舟身前,另有六七人和那少婦各展兵刃,圍住了自己。正没做理會處,兪蓮舟忽然朗聲道︰「六弟,出來把這些人收拾了吧!」

張翠山一愕︰「二哥擺空城計麼?」忽聽得半空中一聲清嘯,一人叫道︰「五哥,你好啊,想煞小弟了。」十餘丈外的一株大槐樹上縱落一條人影,長劍顫動,走向人叢中來,正是六俠殷利亨到了。張翠山喜出望外,大叫︰「六弟,你好!」三江幫中早分出數人上前截攔,只聽得啊喲啊喲、叮叮{\upstsl{噹}}{\upstsl{噹}}之聲不絶,每個人手腕的「神門」穴上一一中劍,一一撤下兵刃。這「神門穴」是在腕骨的鋭端,被利劍一刺,手掌中再也使不出半點力道。殷利亨不疾不待的漫步揚長而來,遇有敵人上前阻擋,他長劍一顫,嗆{\upstsl{啷}}一聲,便有一件兵刃落地。那少婦回身喝道︰「你是武當\dash{}」嗆{\upstsl{啷}}嗆{\upstsl{啷}}兩聲,只因那少婦雙手各執一刀,雙刀落地時便有兩下聲響。

張翠山大喜,説道︰「師父的『神門十三劍』創制成功了。」原來這「神門十三劍」,共有十三記招數,每一記招式各各不同,但所刺之處,全是敵人手腕的「神道穴」。張翠山十年前離武當之時,張三丰甫有此意,和弟子們商量過幾次,但許多艱難之處並未想通。此時殷利亨使將出來,三江幫的硬手竟是没人能抵擋得一招。

張翠山只看得心曠神怡,但見殷利亨每一劍刺出,無不精妙絶倫,只用了五六種招式,「神門十三劍」尚未使到一半,三江幫幫衆已有十餘人手腕中劍,撤下了兵刃。那少婦叫道︰「風緊風緊,退走吧!」幫衆有的騎馬逃走,有的不及上馬,便此轉身急退。張翠山拍開泉建男身上的穴道,拾起蛇頭雙筆,插在他腰間。泉建男滿面羞慚,落荒急奔而去,竟是不和三江幫幫衆一路同行。

殷利亨還劍入鞘,拉住了張翠山的手,喜道︰「五哥,我想得你好苦!」張翠山笑道︰「六弟,你長高了。」他二人分别之時,殷利亨只有十八歳,十年不見,殷利亨已自一個瘦瘦小小的少年,變爲身長玉立的青年。當下張翠山擕著殷利亨的手,去和妻子相見。殷素素病得沉重,點頭笑了笑,低聲叫了聲︰「六弟!」殷利亨笑道︰「五嫂也姓殷,那好極了,不但是我嫂子,還是我姊姊。」

張翠山道︰「究是二哥了得,你躱在那大樹之上,我一直不知,二哥却早瞧見了。」殷利亨當下説起趕來應援的情由。原來四俠張松溪在下山採辦師父百歳大壽應用的物事,遇到有兩個江湖人物鬼鬼崇崇,路道不正,不禁起了疑心,暗想︰「我武當派威震天下,難道還有什麼大膽之徒到我武當山來捋虎鬚?」於是暗中攝著,偸聽兩人説話,才知張翠山從海外歸來,已和二哥兪蓮舟會合,「三江幫」和「五鳳刀」都想截攔,逼問謝遜的下落。

張松溪匆匆回山,其時山上只有殷利亨一人,兩人便分頭赴援,心中均想,有兪二張五在一起,那些小小的幫會門派徒然自取其辱,怎能奈何得了他二人。只是他們急於和張翠山相會,早見一刻好一刻,這纔迎接出來。至於兪蓮舟已然受傷之事,那兩個江湖人物並未説起,是以張松溪和殷利亨並没知曉。張松溪去打發「五鳳刀」門中派來的二個高手。這三江幫一路,却是由殷利亨逐走。

兪蓮舟嘆道︰「若不是四弟機警,今日咱武當派説不定要丟個大人。」張翠山道︰「單憑小弟一人之力,保護不了二哥。唉,離師十年,小弟的功夫和各位兄弟實在差得太遠了。」殷利亨笑道︰「五哥説那裡話來?你適纔打敗那高麗老頭的功夫,師父就没傳授第二個。你這次回山,師父他老人家一喜歡,不知有多少精妙的功夫傳你,只怕你學也學不及呢。『這神門十三劍』的招術,小弟便説給你聽如何?」

他師弟情深,久别重逢,殷利亨恨不得將十年來所學的功夫,一日之間便説給張翠山知道。兩個人並肩行,殷利亨又比又劃,説個不停。

當晩四人在仙人渡的客店中歇宿,殷利亨定要和張翠山同榻而臥。張翠山也眞喜歡這個小師弟,見他雖是又高又大,還是跟從前一般對己依戀。原來武當七俠中雖是莫聲谷年紀最小,但莫聲谷自幼便少年老成,反是殷利亨顯得比師弟稚弱。張翠山年紀跟他相差不遠,因此一向對他也是照顧特多。

兪蓮舟笑道︰「五弟有了嫂子,你還道是十年之前麼?五弟,你回來得正好,咱們喝了師父的壽酒之後,跟著便喝六弟的喜酒了。」張翠山大喜,鼓掌笑道︰「好妙極,妙極!新娘子是那一位名門之女?」殷利亨臉一紅,忸怩著不説。兪蓮舟道︰「便是漢陽金鞭紀老英雄的掌上明珠。」張翠山伸了伸舌頭,笑道︰「六弟若是頑皮,這金鞭當頭{\upstsl{砸}}將下來,可不是玩的?」兪蓮舟微微一笑,但臉上隨即閃過一絲陰影,説道︰「那位紀姑娘是使劍,只盼那日江邉蒙面的諸女之中,没有紀姑娘在内。」張翠山心中微微一驚,道︰「紀姑娘是峨嵋門下?」

\chapter{七俠重聚}

兪蓮舟點了點頭,道︰「咱們在江邉遇到峨嵋諸女武功平平,不會有紀姑娘在内。否則爲了五弟妹,却得罪了六弟妹,人家可要怪我這二伯偏心了。咱們這位未過門的六弟妹相貌既好,武功又佳,名門弟子,畢竟不凡,和六弟當眞是天生一對\dash{}」他説到這裡,忽然想起,殷素素却是邪教教主的女児,自己這麼稱讚紀姑娘,只怕張翠山心有感觸,正想亂以他語,忽聽得一人走到房門口,説道︰「兪爺,有幾位爺們來拜訪你老人家,説是你的朋友。」却是店小二的聲音。

兪蓮舟道︰「誰啊?」店小二道︰「一共六個人,説是什麼『五鳳刀』門下的。」師兄弟三人都是一凜,心想張松溪去打發「五鳳刀」一路的人馬,怎地敵人反而找上門來了,難道張松溪有什失閃?張翠山道︰「我去瞧瞧。」他怕二哥受傷未愈,在店房中跟敵人動手不甚妥善。兪蓮舟却道︰「請他們進來吧。」

一會児進來了五個漢子,一個容貌俊秀的少婦。張翠山和殷利亨雖然空著雙手,但站在兪蓮舟身前,蓄勢戒備。却見這六人個個垂頭喪氣,臉有愧色,身上也没帶兵刃,一點不像是前來生事的模樣。領頭的一人頭髮花白,四十來歳年紀,恭恭敬敬的抱拳行禮,説道︰「三位是武當兪二俠、張五俠、殷六俠?在下五鳳刀門下弟子孟正飛,請問三位安好。」兪蓮舟等三人拱手還禮,心下都是暗自奇怪。兪蓮舟道︰「孟老師好,各位請坐。」

孟正飛却不就坐,説道︰「敝門向在山西河東,門派窄小,久仰武當山張眞人和七俠的威名,當眞是如雷貫耳,只是無緣拜見。今日到得武當山下,原該上山去叩見張眞人,但聽聞張眞人百歳高齡,清居靜修,咱們這些粗魯武人,也不敢冒昧去打擾他老人家的精神。三位回山之時,還請代爲請安,便説山西五鳳刀門下弟子,祝他老人家千秋康寧,福壽無彊。」兪蓮舟本因受傷未愈,坐在炕上,聽他説到師父,忙扶著殷利亨的肩頭下炕,恭敬站立,説道︰「不敢,不敢,在下這裡謝過。」

孟正飛又道︰「咱們僻處山西鄕下,眞如井底之蛙,見識淺陋,也不知天高地厚,竟然大膽妄爲,擅自來到貴地。今蒙武當諸俠寬洪大量,反而解救咱們的危難,在下感激不盡,今日特地趕來,一來道謝,二來陪罪,萬望三位大人不記小人過。」説著躬身下拜。張翠山忙伸手扶住,説道︰「孟老師不必多禮。」

孟正飛囁囁嚅嚅,想説又不敢説。兪蓮舟道︰「孟老師有何吩咐,但説不妨。」孟正飛道︰「在下求兪二俠賞一句話,便説武當派不再見怪,咱們回去好向師父交代。」兪蓮舟微微一笑,道︰「各位遠道自晉來鄂,想必是爲了打聽金毛獅王謝遜的下落,不知那金毛獅王跟貴門有過節?」孟正飛慘然道︰「家兄孟正仁慘死在謝遜的掌上。」

兪蓮舟心中一震,説道︰「咱們實有不得已的苦衷,無法奉告那金毛獅王的下落,還須請孟老師和反位原諒。至於見怪云云,那是不必提起,見到尊師烏老爺子時,便説兪二、張五、殷六問好。」孟正飛道︰「如此在下告辭。日後武當派如有差遣,只須傳個信來,五鳳刀門下雖然能力低微,但奔走之勞,決不敢辭。」説著和其五人一齊抱拳行禮,轉身出門。

那少婦突然回轉,跪倒在地,低聲道︰「小婦人得保名節,全出武當諸俠之賜。小婦人有生之年,不敢忘了諸俠的大恩大德。」兪蓮舟等三人不知其中原因,但聽她説的是婦人名節之事,也不便多問,只得含糊虛遜了幾句,那少婦拜了幾拜,出門而去。

\qyh{}五鳳刀」六人剛走,門帘一掀,閃進一個人來,撲上來一把抱住了張翠山。

張翠山喜極而呼︰「四哥!」原來進房之人正是張松溪。師兄弟相見,自有一番親熱。張翠山道︰「四哥,你神計妙算,足智多謀,竟能將五鳳刀門下化敵爲友,實是不易。」張松溪道︰「那也是機緣湊巧,小兄有什麼功勞可言?」當下將經過情由説了出來。

原來那美貌少婦娘家姓烏,是五鳳刀掌門人的第二女児,她丈夫便是那孟正飛。這一次六個人同下湖北,尋訪謝遜的下落,途中遇上三江幫的舵主,得知武當派張翠山知曉謝遜的所在。那烏氏少婦自幼嬌生慣養,主張設計擒獲張翠山逼問。孟正飛向來畏妻如虎,但這一次却決計不從,他説武當子弟極是了得,不如依禮相求,對方如若不允,再想法子。那烏氏言道︰「時機可還不可求,若是放得張翠山上了武當,他師兄弟一會合,又有張三丰作護身符,如何再能逼問?兩人言語不合,吵起嘴來。其餘四人都是師弟師侄,也不敢作左右袒。」

那烏氏一怒之下,説道︰「你這膽小鬼,是給你兄長報仇,又不是給我兄長報仇。哼,你對武當子弟怕得這般厲害,便是那張翠山將謝遜的下落跟你説了,你有膽去找他麼?嫁了你這種膽小鬼,實是我一輩子倒霉。」孟正飛對嬌妻忍讓慣了,不敢再説,但要依烏氏之見,在途中客店暗下蒙藥,迷倒張翠山夫婦,却是堅決不肯。烏氏一怒之下,半夜裡乘丈夫睡著,就此悄悄離去。

她是想獨自下手,探到謝遜的下落,好好臊一臊丈夫,那知這一切全給三江幫的一名舵主瞧在眼中。他見烏氏貌美,起了歹心,暗中跟隨其後,烏氏想使蒙汗藥,却反給他下了迷藥。不料螳螂捕蟬,黃雀在後,張松溪一直監視著五鳳刀六人的動靜,等到烏氏情勢危急,這纔出手相救,將那三江幫的舵主懲戒了一番逐走。張松溪也不説自己姓名,但説是武當派門下弟子。

烏氏又驚又羞,回去和丈夫相見,説明情由,兩人一商量,武當派成了本門的大恩人,於是齊來向兪蓮舟等人叩謝相救之德,張松溪待那六人去後,這纔現身,以免烏氏羞慚。

張翠山聽罷這番經過,嘆道︰「打發三江幫這個行止不端之徒,雖非難事,但四哥行事處處替人留下餘地,化凶爲吉,最合師父的心意。」張松溪笑道︰「十年不見,一見面就給四哥一頂高帽子戴。」

這一晩師兄弟四人聯床夜話,長談了一宵。張松溪雖是足智多謀,但對那個假扮元兵擄去無忌、擊傷兪蓮舟的高手來歷,也猜測不出半點端倪。次晨張松溪和殷素素會見了,五人緩緩而行,途中又宿了一晩,纔上武當。張翠山十年重來,回到自幼生長之地,想起即刻便可拜見師父,和大師哥、三師哥、七師弟相會,雖然妻病子散,却也是歡喜多於哀愁。

到得山上,只見觀外繫著八頭健馬,鞍轡鮮明,並非山上之物。張松溪道︰「觀中到了客人,咱們不忙相見,從邉門進去吧。」當下張翠山扶著妻子,從邉門進觀。觀中道人和侍役見張翠山無恙歸來,無不歡天喜地。張翠山念著要去拜見師父,但服侍張三丰的道僮説眞人尚未開関,張翠山只得到師父坐関的門外磕頭,然後再去見兪岱岩。

服侍兪岱岩的道僮輕聲説道︰「三師叔睡著了,要不要叫醒他?」張翠山搖了搖手,輕手輕脚走到房中。只見兪岱岩正自閉目沉睡,臉色慘白,雙頰凹陥,十年前龍精虎猛的一條驃悍漢子,今日成了奄奄一息的病夫。張翠山想起自己初入師門之時,許多功夫都是三師哥所授,此刻眼見他如此悽慘,忍不住掉下泪來。

張翠山看了一陣,掩面走出,問那小道僮道︰「你大師伯和七師叔在那裡?」小道僮道︰「在大廳會客。」張翠山走到後堂,等大師哥和七師弟會客之後相見,但等了半個多時辰,客人始終不走。張翠山問送茶的道人道︰「是什麼客人?」那道人道︰「好像是保鏢的。」

殷利亨對這位久别重逢的五師兄很是依戀,剛離開他一會,便又過來陪他,聽得張翠山在問客人的來歷,説道︰「是三個總鏢頭。金陵虎蟠鏢局的總鏢頭祁天彪,太原晉陽鏢局的總鏢頭雲鶴,還有一個是京師燕雲鏢局的總鏢頭宮九佳。」張翠山微微一驚,道︰「這三位總鏢頭都來了?當今鏢局之中,要數他三位武功最強,名望最大,同時來到山上,爲了什麼?」殷利亨笑道︰「想是有什麼大鏢丟了,劫鏢的人來頭大,這三位老鏢頭惹不起,只好來求大師兄。五哥,這幾年大哥越來越愛做濫好人,江湖上遇到什麼疑難大事,總是來請大哥出面。」張翠山微笑道︰「大哥是佛面慈心,别人求到他,總是難以推託。十年不見,不知大哥老了些没有?」

他想到此處,想看一看大哥之心再也難以抑制,説道︰「六弟,我到屏風後去瞧瞧大哥和七弟的模樣。」於是走到屏風之後,悄悄向外一張,只見宋遠橋和莫聲谷兩人坐在下首主位陪客。宋遠橋穿著道裝,臉上神情沖淡恬和,一如往昔,相貌和十年之前竟無多大改變,只是鬢邉微見花白,身子却肥胖了很多,想是中年發福。宋遠橋並没出家,但因師父是道士,又住在道觀之中,因此在武山上時常愛作道家打扮,下山時才改換俗裝。莫聲谷却已長得魁梧奇偉,雖只二十來歳,却已長了滿臉的濃髯,看上去比張翠山的年紀還大些。

只聽得莫聲谷正大著嗓子説道︰「我大師哥説一是一,説二是二,憑著宋遠橋三字,難道三位還信不過麼?」張翠山心想︰「七弟粗豪的脾氣竟是半點没改。不知他爲了何事,又在跟人吵嘴?」轉頭向賓客位上看去時,只見三個人都是五十來歳年紀,一個氣度威猛,一個高高瘦瘦,貌相清臞,坐在末座的却像是個病夫,甚是乾枯。三人身後,又有五個人垂身站立,想是那三人的子弟輩,只聽那高身材的瘦子道︰「宋大俠既這般説,咱們焉敢不信,只不知張五俠何時歸來,可能賜一個確期麼?」

張翠山聽他説到「張五俠」三字,吃了一驚,心想︰「原來這三個總鏢頭乃是爲我而來,想必又是爲了探問我義兄的下落了。」只聽莫聲谷道︰「咱們師兄弟七人,雖然本領微薄,但行俠仗義之事,向來不敢後人,多承江湖上朋友推獎,賜了『武當七俠』這個外號。這『武當七俠』四個字,説來慚愧,咱們原不敢當\dash{}」張翠山心道︰「十年不見,七弟居然如此能説會道。從前人家問他一句話,他也要臉紅半天,纔回答得一句。十年之間,除了三哥和我之外,人人都是一日千里。」

但聽莫聲谷續道︰「可是咱們既然負了這個名頭,上奉恩師嚴訓,行事決不敢有半步差錯。張五哥是『武當七俠』之一,他這人文武全才,斯文和順,咱們七人之中,要數他脾氣最好。你們定要賴他殺了『龍門鏢局』滿門,那是截然的胡説八道。」張翠山心中一寒︰「原來是爲了龍門鏢局都大錦的事。素聞大江以南,各鏢局以金陵蟠鏢局馬首是瞻,想是他們聽到我從海外歸來,於是虎蟠鏢局,約了晉陽、燕雲兩家鏢局的總鏢頭,興師問罪來啦。」

那氣度威猛的大漢道︰「武當七俠名頭響亮,武林中誰不尊仰?莫七俠不用自己吹噓,咱們早已久聞大名,如雷貫耳。」莫聲谷聽了這句譏嘲之言,臉色大變,説道︰「祁總鏢頭到底意欲如何,不妨明言。」

那氣度威猛的大漢正是虎蟠鏢局的總鏢頭祁天彪,朗聲説道︰「武當七俠説一是一,説二是二,可難道少林派衆高僧慣打誑語麼?少林僧人親眼目睹,臨安龍門鏢局上下大小人等,盡數傷在張翠山張五俠\dash{}的手下。」他説到「張五俠」這個「俠」字時,聲音拖得長長的,顯是充滿譏嘲之意。

殷利亨在屏風之後聽得怒氣勃發,這人出言嘲諷五哥,可比打他自己三記巴掌還更令他氣憤,便欲挺身而出,跟他理論。張翠山一把拉住,搖了搖手。殷利亨見他臉上滿是痛苦爲難之色,心下不明其理,暗道︰「五哥的涵養功夫越來越好了,無怪師父常常讚他。」

莫聲谷站起身來,大聲道︰「别説我五哥此刻尚未回山。便是已經回到武當,也只是這句話。莫某跟張翠山生死與共,他的事便是我的事。三位要尋張翠山的晦氣,一切衝著我莫某便是。三位不分青皂白,定要誣賴我五哥害了龍門鏢局滿門,好!這一切便全算是莫某幹的。三位要替龍門鏢局報仇,儘管往莫某身上招呼。我五哥不在此間,莫聲谷便是張翠山,張翠山便是莫聲谷。老實跟你們説,莫某的武功智謀,遠遠不及我五哥,你們找到了我,算是你們運氣。」

祁天彪大怒,霍地站了起來,大聲道︰「祁某今日到武當山來撒野,天下武學之士,人人要笑我班門弄斧,太過不自量力。可是都大錦都兄弟滿門被害十年,沉冤始終未雪,祁某這口氣終是嚥不下去。反正武當派將龍門鏢局七十餘口也殺了,再饒上祁某一人又有何妨?便是再饒上金陵虎蟠鏢局的九十餘口,又有何妨?祁某今日頸血濺於武當山上,算是死得其所。咱們再上山之時,尊重張眞人德高望重,不敢擕帶兵刃,祁某便在莫七俠拳脚下領死。」説著大踏步走到廳心。

宋遠橋一直没有開口,這時見兩人説僵了要動手,伸手攔住莫聲谷,微微一笑,説道︰「三位來到敝處,翻來覆去,一口咬定是敝師弟害了臨安龍門鏢局滿門。好在敝師弟不久便可回山,三位暫忍一時,待見了敝師弟之面,再行分辨是非如何?」

那身形乾枯猶似病夫的,是燕雲鏢局的總鏢頭宮九佳,此人甚工心計,説道︰「祁總鏢頭且請坐下。張五俠既然尚未回山,此事終是不易了斷,咱們不如拜見張眞人,請他老人家金口明示,交代一句話下來。張眞人是當今武林中的泰山北斗,天下英雄好漢,莫不景從,難道他老人家還會不明是非,包庇弟子麼?」他言幾句話雖説得客氣,但語氣中含意其實甚是厲害。莫聲谷如何聽不出來,當即説道︰「家師閉関靜修,尚未開関。再説,近年來我武當門中之事,均由我大師哥處理。除了武林中眞正大有名望的高人,家師極少見客。」言下之意,是説你們想見我家師父,可還彀不上格。

那高高瘦瘦的晉陽鏢局總鏢頭雲鶴冷笑一聲,道︰「天下事也眞有這般湊巧,剛好咱們上山,尊師張眞人便即閉関。可是龍門鏢局七十餘口的人命,却不是一閉関便能躱過呢。」宮九佳聽他這幾句話説得太重,忙使眼色制止,但莫聲谷已自忍耐不住,大聲喝道︰「你説我師父是因爲怕事纔閉関嗎?」宮九佳冷笑一聲,並不答話。

宋遠橋雖然涵養極好,但聽他辱及恩師,却也是忍不住有氣,當著武當七俠之面,竟然有人言辭中對張三丰不敬,那是十餘年來從未有過之事。他緩緩的道︰「三位遠來是客,咱們不敢得罪,送客!」説著袍袖一拂,一股疾風隨著一拂之勢捲出,祁天彪、雲鶴、宮九佳三人身前茶几上的三隻茶碗突然一齊被風捲起,緩緩落在宋遠橋身前的茶几之上。這三隻茶碗緩緩捲起,緩緩落下,落到茶几上時只是輕輕一響,竟不濺出半點茶水。祁天彪等三人當宋遠橋衣袖揮出之時,被這一股看似柔和,實則力道強勁之極的袖風壓在胸口,登時呼吸閉塞,喘不過氣來。三人急運内功相抗,但那股袖風倏然而來,倏然而去,三人胸口重壓陡消,波波三聲巨響,三人都是大聲的噴了一口氣出來。但見祁天彪滿臉血紅、雲鶴臉色慘白、宮九佳一張黃臉更是焦黃。三人這一驚是非同小可,心知宋遠橋只須手袖子跟著一揮,第二股袖風乘虛而入,三人所運的内息被逼得逆行倒衝,就算不立斃當場,也須身受重傷,内功損折大半。這一來,三位總鏢頭方知眼前這位沖淡謙和、恂恂儒雅的宋大俠,實是身負深不可測的絶藝。

祁天彪爲人爽直,抱拳説道︰「多謝宋大俠手下留情,告辭!」宋遠橋和莫聲谷送到滴水簷前,祁天彪轉身道︰「兩位留步,不勞遠送。」宋遠橋道︰「難得三位總鏢頭光降敝山,如何不送?改日在下當再赴京師、太原、金陵貴局回拜。」祁天彪道︰「這個如何克當?」他領教了宋遠橋的武功之後,覺得這位宋大俠雖然身負絶世武功,但言談舉止之中,竟無半分驕氣,心中對他甚是欽佩,初上山時那股興師問罪、復仇拼命的鋭氣,已折了大半。

兩人正在説話,突然門外匆匆進來一個短小精悍、滿臉英氣的中年漢子。宋遠橋道︰「四弟,見過這三位朋友。」當下給祁天彪等三人引見了。張松溪笑道︰「三位來得正好,在下正有幾件物事要交給各位。」説著從懷中掏出三個小包,每人交了一個。祁天彪道︰「那是什麼?」張松溪道︰「此處拆看不便,各位下山後再看吧。」師兄弟三人一直送到觀門之外,方與三位總鏢頭作别。

莫聲谷一待三人走遠,急問︰「四哥,五哥呢?他回山没有?」張松溪笑道︰「你先進去見五弟,我和大哥在廳上等這三個鏢客回來。」莫聲谷奇道︰「他們還要回來,爲什麼?」但心下記掛著張翠山,竟不待張松溪説明情由,急奔入内。

莫聲谷剛走進内堂,果然祁天彪等三人匆匆回來,向宋遠橋張松溪納頭便拜。二人急忙還禮。雲鶴道︰「武當諸俠大恩大德,雲某此刻方知。適纔雲某言語中冒犯張眞人,當眞是豬狗不如。」説著提起手來,在自己臉上左右開弓,辟辟拍拍的打了十幾下,只打得雙頰紅腫,兀自不停。宋遠橋愕然不解,急忙攔阻。張松溪道︰「雲總鏢頭乃是有志氣的好男児,那驅除韃虜、還我河山的大願,凡我中華好漢,無不同心。些些微勞,正是我輩份所當爲,雲總鏢頭何必如此。」雲鶴道︰「雲某老母幼子,滿門性命,皆出諸俠之賜,雲某渾渾噩噩,五年來一直睡在夢裡。想起適纔言辭不遜,兩位若肯狠狠打我一頓,雲某心中方得稍減不安。」張松溪微笑道︰「過去之事,誰也休提,家師便是親耳聽到這兩句話,心敬雲總鏢頭的所作所爲,也決不會放在心上。」但雲鶴始終惶愧不安,深自痛責。宋遠橋不明其中之理,只是順口謙遜了幾句。但見祁天彪和宮九佳也是不住口的道謝,但瞧張松溪的神色語氣之間,對祁宮二人並不怎樣,對雲鶴却甚是敬重親熱。三位總鏢頭定要到張三丰坐関的屋外磕頭,又要去見莫聲谷陪罪,張松溪一一辭謝,這纔作别。

三人走後,張松溪嘆了口氣,道︰「這三人雖對咱們心中感恩,可是龍門鏢局的人命,他三人竟是一句不提。看來感恩只管感恩,那一場禍事,仍是消弭不了。」宋遠橋待問情由,只見張翠山從内堂奔將出來,拜倒在地,叫道︰「大哥,可想煞小弟了。」

宋遠橋是謙恭有禮之士,雖對同門師弟,又是久别重逢,心情激盪之下,仍是不失禮數,恭恭敬敬的拜倒還禮,説道︰「五弟,你終於回來了。」

張翠山略述别來情由,莫聲谷心急,便問︰「五哥,那三個鏢客無禮,定要誣賴你殺了臨安龍門鏢局滿門,你也涵養忒好,怎地不出來教訓他們一頓?」張翠山慘然長嘆,道︰「這中間的原委曲折,非一言可盡。待會等三哥醒來,我再一併詳告,還得請衆兄弟一同想個良策。」殷利亨道︰「五哥放心。龍門鏢局護送三哥不當,害得他一生殘廢,五哥便是眞的殺了他鏢局滿門,也是兄弟情深,激於一時義憤\dash{}」兪蓮舟喝道︰「六弟,你胡説什麼?這話要是給師父聽見了,他不関你三個月黑房纔怪。殺人全家老少,這種滅門絶戸之事,我輩怎可做得?」

五人一齊望著張翠山,但見他神色甚是淒厲,過了半晌,説道︰「龍門鏢局的人,我一個也没殺,我不敢忘了師父的教訓,没敢累了衆兄弟的盛德。」五人一聽大喜,都舒了一口長氣。他們雖然截然不信張翠山會做這種狠毒慘事,但少林派的衆僧既一口咬定是他所爲,還説是親眼目睹,而當三個總鏢頭上門問罪之時,他又不挺身而出,直斥其非,各人心中,不免稍有疑惑,這時聽他這般説,無不放下了一件心事,均想︰「這中間便有許多爲難之處,但祇要不是他殺的人,終能解説明白。」當下莫聲谷便問那三個鏢頭去而復回的情由。張松溪笑道︰「這三個鏢客之中,倒是那個出言無禮的雲鶴人品最好。他在晉陝一帶,名望甚高,暗中聯絡了山西、陜西的豪傑,歃血爲盟,要舉起義旗,反抗蒙古韃子。」宋遠橋等五人一齊喝了聲采。莫聲谷道︰「瞧不出他竟具這等胸襟,實是可敬可佩。四哥,你且莫説下去,等我歸來再説\dash{}」説著急奔出門而去。

張松溪果然住口,向張翠山問些冰火島的風物。當張翠山説到那頭靈異無比的玉面火猴時,四人盡皆駭異。張翠山道︰「咱們本想帶同那火猴回到中土,但牠在木筏上飄了數日,天候稍暖,牠便覺得不慣,跳上浮冰,一跳向北,想是又回到冰火島去了。」殷利亨道︰「可惜,可惜。」宋遠橋道︰「小小一頭猴子,竟能生裂熊腦,實是不可思議。」張翠山道︰「那火猴雖然生具猴形,實則恐怕也非猿猴之屬,想是冰火島天候奇特,稟天地靈秀之氣,因而生出這種奇獸來。」宋遠橋點頭道︰「便是中土,深山大澤之間,原也有許多人不像人、獸不似獸的山魈木怪一類靈物。」

説話之間,莫聲谷已奔了回來,説道︰「我趕去向那雲總鏢頭陪了個禮,説我佩服他是個鐵錚錚的好男児。」衆人都深知這個小師弟的直爽性子,也早料到他出去何事。莫聲谷來往飛奔數里,絲毫不以爲累,他既知雲鶴是個好男児,若不當面跟他盡釋前嫌,言歸於好,那便有幾晩睡不著覺了。殷利亨道︰「七弟,四哥的故事等著你不講,可是五哥説的玉面火猴故事,可更加好聽。」莫聲谷跳了起來,道︰「啊,有這等事?」張松溪道︰「那雲鶴籌劃就緒\dash{}」莫聲谷搖手道︰「四哥,對不住,請你再等一會。\dash{}」張翠山微笑道︰「七弟總是不肯吃虧。」於是將玉面火猴的事重述了一遍。莫聲谷道︰「奇怪,奇怪!四哥,這便請你説了。」

張松溪道︰「那雲鶴一切籌劃就緒,只待日子一到,便在太原、大同、汾陽三地同時舉義,那知與盟的衆人之中,竟有一名大叛徒,便在舉義的前三天,盜了加盟衆人的名單,以及雲鶴親手繕就的舉義策劃書,要去向蒙古韃子告密。」

莫聲谷拍腿叫道︰「啊喲,那可糟了。」張松溪!道︰「他是事有湊巧,那時我正在太原,有事要找太原府知府晦氣,半夜裡見到那知府正和那叛徒竊竊私議,如何一面密報皇帝,一面調兵遣將,將舉義人等一網打盡。於是我跳進窗去,一劍一個,將那知府和叛徒殺了,取了張要加盟的名單和籌劃書,回來南方。」

\qyh{}雲鶴等一干人發覺名單和籌劃書被盜,知道大事不好,不但義舉不成,而且單上有名之人,家家有滅門的大禍,於是連夜送出訊息,叫各人遠逃避難。但這時城門已閉,訊息送不出去,次日一早,由於知府被戕,太原城閉城大索劍客。雲鶴等人急得猶似熱鍋上螞蟻一般,心想這一番自己滿門抄斬不打緊,而晉陝二省,不知將有多少仁人義士被害。不料提心吊膽的等了數日,竟是安然無事,後來城中拿不到刺客,査得也慢慢鬆了,這件事竟是不了了之。他們見那叛徒死在府衙之中,也想到是暗中有人相救,只是無論如何,想不到我身上。」殷利亨道︰「你適纔交給他的,便是那份加盟名單的籌劃書了?」張松溪道︰「正是。」殷利亨道︰「那宮九佳呢?四哥怎生幫了他一個大忙?」張松溪道︰「這宮九佳武功是好的,可是人品作爲,決不能與總鏢頭相提並論。六年前,他保鏢到了雲南,在昆明受一個大珠寶商之託,暗帶一批價値六十萬兩銀子的珠寶,送往北京。但到江西却出了事,在鄱陽湖邉,宮九佳被鄱陽四義中的三義圍攻,搶去紅貨。宮九佳便是傾家蕩産,也賠不起這批珠寶,何況他燕雲鏢局隱然北方鏢局的牛耳,他招牌這麼一{\upstsl{砸}},以後也不用再做人了。他在客店中左思右想,竟想尋起短見來。」

\qyh{}鄱陽三義不是綠林豪傑,却爲何要劫取這批珠寶!原來鄱陽四義中的老大犯了事,給関入南昌府的死囚牢,轉眼便要處斬。三義劫了兩次牢,救不出老大,官府却反而防範得更加緊了,鄱陽三義知道官府貪財,便想用這批珠寶去行賄,減輕老大的罪名。我見他四人甚有義氣,便設法將那老大救出牢來,要他們將珠寶還給宮九佳。這位總鏢頭雖然面目可憎、言語無味,但生平也没做過什麼惡事,在北京城中,也不交結官府,欺壓良善,那麼救了他一命也是好的。我叫鄱陽四義不可提我的名字,只是將那塊包裹珠寶的錦鍛包袱留了下來。適纔我將那塊包袱還了給他,他自是心中有數了。」兪蓮舟點頭道︰「四弟此事做得好,那宮九佳也還罷了,鄱陽四義却爲人不錯。」

莫聲谷道︰「四哥,你交給祁天彪却又是什麼?」張松溪道︰「那是九枚斷魂蜈蚣鏢。」五人聽了,都「啊」了一聲,這斷魂蜈蚣鏢在江湖上名頭頗爲響亮,是涼州吳一氓的成名暗器。張松溪道︰「這一件事我做得忒也大膽了些,這時想來,當日也眞是僥倖。那祁天彪保鏢路過潼関,無意中得罪了吳一氓的弟子,兩人動起手來,祁天彪一掌將他打得重傷。祁天彪打了這掌之後,知道闖下了大禍,匆匆忙忙的交割了鏢銀,便想連夜趕回金陵,邀集至交好友,合力對付那吳一氓。但他剛到洛陽,便給吳一氓追上了,約了他次日在洛陽西門外比武。」殷利亨道︰「這吳一氓的武功未必在你我之下,祁天彪如何是他對手?」張松溪道︰「是啊,祁天彪自知憑他的能耐,擋不了吳一氓的一鏢,無可奈何之中,便去邀洛陽喬氏兄弟助拳。喬氏兄弟一口答應,説道︰『憑我兄弟的武功,祁大哥你也明白,決不能對付得了吳一氓,你要咱兄弟出場,原也不過是要咱二人吶喊助威。好,明日午時,洛陽西門外,咱兄弟準到。』」

\chapter{貴賓雲集}

莫聲谷道︰「喬氏兄弟都是使暗器的好手,有他二人助拳,祁天彪以三敵一,或能跟吳一氓打個平手。只不知吳一氓有没有幫手。」張松溪道︰「吳一氓倒是没有幫手。可是喬氏兄弟却出了古怪。第二天一早,祁天彪便上喬家去,想跟他兄弟商量一下迎敵之策,那知到得喬,守門的説道︰『大爺和二爺今朝忽有要事,趕去了鄭州,請祁老爺不必等他們了。』祁天彪一聽之下,幾乎氣炸了肚子。喬氏兄弟幾年之前在江南出過事,當時祁天彪幫了他倆很大的忙,那知此刻有求於他二人,兄弟倆口上説得好聽,竟是脚底抹油,溜之乎也。」

\qyh{}祁天彪知道吳一氓心狠手辣,這個約會躱是躱不過的,於是在客店中冩下了遺書,處分後事,交給了趟子手,自己到洛陽西門外赴約。」

\qyh{}這件事的前後經過,我都瞧在眼裡。那日我扮了個乞丐,易容改裝,躺在西門外的一株大樹之下。不久吳一氓和祁天彪先後到來,兩人動起手來,鬥不數合,吳一氓下殺手,放了一枚斷魂蜈蚣鏢。祁天彪眼見抵擋不住,只有閉目待死,我搶上前去,伸手將鏢接了。吳一氓又驚又怒,喝問我是否幫中人,我笑嘻嘻的一答,吳一氓連放八枚斷魂蜈蚣鏢,都給我一一接了過來。他的成名暗器果然是非同小可,我若用本門武功去接,本也不難,但我防他瞧出疑竇,故意裝作左足瘤,右手斷,只使一隻左手,又使少林派的接鏢手法,掌心向下擒撲。接是都接到了,但手掌險險給他第七枚毒鏢劃破,算是十分凶險。他果然喝問我是少林派中的那一位高僧的弟子,我仍是裝聾作啞,跟他咿咿啊啊的胡混。吳一氓自知不敵,一怒而去,回到涼州後杜門不出,這幾年來一直没在江湖上現身。」

莫聲谷搖頭道︰「四哥,吳一氓雖不是善良之輩,但祁天彪也算不得是什麼好人,那日倘若你給蜈蚣鏢傷了手掌,這可如何是好?這般冒險未免太不値得。」張松溪笑道︰「這是我一時好事,事先也没料到他的蜈蚣鏢當眞有這等厲害。」

莫聲谷性情直爽,不懂張松溪這些行逕的眞意,張翠山却如何不省得?四哥盡心竭力,想要消解龍門鏢局全家被殺的大仇。他知道虎蟠鏢局是江南衆鏢局之首,冀魯一帶以燕雲鏢局馬首是瞻,西北各省則推晉陽鏢局爲尊。龍門鏢局之事日後發作起來,這三家鏢局定要出頭,是以他先行伏下了三椿恩惠。這三件事看來似是機緣巧合,但張松溪明査暗訪,等候機會,不知花了多少時日,多少心血?張翠山哽咽道︰「四哥,你我兄弟一體,我也不必説這個『謝』字,都是你弟妹當日作事偏激,闖下了這個大禍。」當下將殷素素如何扮成他的模樣,夜中去殺了龍門鏢局滿門之事,從頭至尾説了,最後道︰「四哥,此事如何了結,你給我拿個主意。」

張松溪沉吟半晌,道︰「此事自當請師父示下,但我想人死不能復生,弟妹也已改過遷善,不再是當日殺人不眨眼的弟妹,古人言道︰知過能改,善莫大焉。大哥,你説是不是!」宋遠橋面臨這數十口人命的大事,一時躊躇難決,兪蓮舟却點了點頭,道︰「不錯。」殷利亨最怕二哥,知道大哥是個好好先生,容易説話,二哥却是嫉惡如仇,鐵面無私,生怕他跟五嫂爲難,一直在提心吊膽,却不知兪蓮舟早已知道此事,也早已原宥了殷素素。他見二哥點頭,心中大喜,忙道︰「是啊,旁人問起來,五哥只須説那人不是你殺的。你又不是撒謊,本來不是你殺的啊。」宋遠橋橫了他一眼,道︰「一味抵賴,五弟心中何安?咱們身負俠名,心中何安?」

殷利亨道︰「那怎生是好?」宋遠橋道︰「依我之見,待師父壽誕過後,咱們先去找回五弟的孩児來,然後是黃鶴樓頭英雄大會,交代了金毛獅王謝遜這回事後,咱們師兄弟六人,再加上五弟妹,七人同下江南。三年之内,咱們每人要各作十件大善舉。」張松溪鼓掌叫道︰「對對!龍門鏢局枉死了七十來人,咱們各作十件善舉,如能救得一二百個無辜遭難者的性命,那麼勉強也可抵過了。」兪蓮舟也道︰「大哥想得再妥當也没有了,師父也必允可。否則便是要五弟妹給那七十餘口抵命,也不過是多死一人,於事何補?」張翠山一直爲了此事,煩惱不安,聽宋遠橋如此安排,心下大喜,道︰「我去跟她説去。」

張翠山匆匆走進臥室,將宋遠橋所想的法子跟妻子説了,又説衆兄弟一等祝了師父的大壽,便下山去尋無忌。殷素素精神一振,心想憑著武當六俠的威望本事,總能將無忌找得回來。她本來無甚大病,只是思念無忌成疾,這時心頭一寬,病體便日輕一日。

過了數日,已是四月初八,張三丰料知明日是自己百歳大壽,徒児們必有一番熱鬧,雖然兪岱岩殘廢,張翠山失蹤,未免美中不足,但一生能享百歳遐齡,也算難得,同時閉関參究的一門「太極神功」,也已深明精奥,從此武當一派,定可在武林中大放異采,當不輸於天竺達摩東傳的少林派武功,這天清晨,他便開関出來。

一聲清嘯,衣袖略振,兩扇板門便呀的一聲開了。張三丰第一眼見得的不是旁人,竟是十年來思念不已的張翠山。他一搓眼睛,還道是看錯了,張翠山已撲在他的懷裡,聲音嗚咽,連叫︰「師父!」心情激盪之下,竟是忘了跪拜。宋遠橋等五人一齊擁到,叫道︰「師父大喜,五弟回來了!」

\qyh{}師父大喜,五弟回來了!」張三丰活了一百歳,修鍊了八十幾年,胸懷空明,早已不縈萬物,但和這七個弟子情若父子,陡然間見到張翠山,忍不住摟著他,喜歡得流下泪來。

衆兄弟服侍師父梳洗漱沐,換過衣巾。張翠山不敢便將煩惱之事跟師父説,只説些冰火島的奇情異物。張三丰聽他説他已經娶妻,更是喜歡,道︰「你媳婦在那裡?快叫她來見我。」張翠山隻膝跪地,説道︰「師父,弟子大膽,娶妻之時,没能稟明你老人家。」張三丰捋鬚笑道︰「你在冰火島上十年不能回來,難道便等了十年,待稟明我再娶麼?笑話,笑話。快起來,不用告罪,張三丰那有這等迂腐不通的弟子?」張翠山長跪不起,道︰「可是弟子的媳婦來歷不正。她\dash{}她是白眉教殷教主的女児。」張三丰仍是捋鬚一笑,説道︰「那有什麼干係?只要媳婦児人品不錯,也就是了。便算她人品不好,到得咱們山上,難道不能潛移默化於她麼?白眉教又怎樣了?翠山,爲人第一不可胸襟太窄,千萬别自居名門正派,把旁人都瞧得小了。這正邪兩字,原來難分。正派中弟子若是心術不正,便是邪徒,邪派中人倘若一心向善,那便是正人君子。」

張翠山大喜,想到自己耽了十年的心事,師父只輕輕兩句話便揭了過去,當下滿臉笑容,站起身來。張三丰又道︰「你那岳父殷教主我跟他神交已久,很佩服他武功卓絶,是個慷慨磊落的奇男子。他雖性子偏激,行事乖僻些,可不是卑鄙小人。咱們很可交交這個朋友。」宋遠橋等均想︰「師父對五弟果然厚愛,愛屋及烏,連他岳父這等大魔頭,居然也肯下交。」正説到此處,一名道僮進來報道︰「白眉教殷教主,派人送禮來給張五師叔!」

張三丰笑道︰「岳父送贊儀來啦,翠山,你去迎接賓客吧!」張翠山應道︰「是!」殷利亨道︰「我跟五哥一起去。」張松溪笑道︰「又不是金鞭紀老英雄送禮來,要你忙些什麼?」殷利亨臉一紅,還是跟張翠山出去。

只見大廳上站著兩個老者,羅帽直身,穿的家人服色,見到張翠山出來,一齊走上幾步,跪拜下去,説道︰「張姑爺好,小人殷無福、殷無祿叩見。」張翠山還了一揖,説道︰「管家請起。」心想︰「這兩個家人的名字好生奇怪,凡是僕役家人,取的名字總不外乎『平安吉慶,福祿壽喜』之類,怎地他二人却叫作『無福、無祿』?」但見他那殷無福臉上有一條極長的刀疤,自右邉額角一直斜下,掠過鼻尖,直至左邉嘴角方止。那殷無祿却是滿臉麻皮,兩人貌相都極醜陋,均已有五十來歳年紀。

張翠山道︰「岳父大人、岳母大人安好,我待得稍作摒擋,便要和你家小姐同來拜見尊親,不料岳父岳母反先存問,却如何敢當?兩位遠來辛苦。請坐喝一杯茶。」殷無福和殷無祿却不敢坐,取出禮單,恭恭敬敬的呈了上去,説道︰「我家老爺太太説這些薄禮,請姑爺笑納。」張翠山道︰「多謝!」打開禮單一看,不禁嚇了一跳,只見十餘張泥金箋的禮單上,一行行的冩了二百款禮品,第一款是「碧玉獅子成雙」,第二款是「翡翠鳳凰成雙」,無數珠寶之後,是「特品紫狼毫百枝」「貢品唐墨十錠」「宋製桑紙百刀」「端硯八方」,那白眉教主竟是打聽到這位嬌客善於書法,送了大批筆墨紙硯,其餘衣履冠帶、服飾器用,無不具備。殷無福轉身出去,領了十名脚夫進來,每人都挑了一副擔子,擺在廳側。

張翠山心下躊躇︰「我自幼清貧,山居樸實,這些珍物要來何用?可是岳父遠道厚賜,若是不受,未免不恭。」只得謝了一聲受下,説道︰「你家小姐旅途勞頓,略梁小恙。兩位管家請在山上多住幾日,再行相見。」殷無福道︰「老爺太太甚是記掛小姐,叮囑即日回報。若不過於勞累小姐。小人想叩見小姐一面,即行回去。」張翠山道︰「既是如此,且請稍待。」

他回到臥房,跟妻子説了。殷素素大喜,略加梳裝,來到偏廳和兩名家人相見,問起父母兄長安康,留著兩人用了酒飯。殷無福、殷無祿當即叩别姑爺小姐。張翠山心想︰「岳父母送來這等重禮,該當重賞賜這兩人才是。可是把山上所有的銀子集在一起,也未必能賞得出手。」好在他生性豁達,也不以爲意,笑説︰「你們小姐嫁了個窮姑爺,給不起賞錢,兩位管家請勿見笑。」殷無福説︰「不敢,不敢。得見武當七俠一面,甚於千金之賜。」

張翠山心道︰「這位管家吐屬風雅,似是個文墨之士。」當下送到中門,殷無福道︰「姑爺請留步,但盼和小姐早日駕臨,以免老爺太太思念。敝教上下,盡皆仰望姑爺風采。」張翠山一笑。殷無福忽道︰「還有一件事須得稟告姑爺知道。小人兄弟送禮上山之時,在襄陽客店中遇見三個鏢客。他三人言談之中,提到了姑爺。」張翠山道︰「哦,他們説些什麼?」殷無祿道︰「一人説道︰『武當七俠於我等雖有大恩,可是龍門鏢局的七十餘口人命,終不能便此罷手。』他三人説自己是決計不能再理會此事的了,決意去請開封府神槍震八方譚老英雄出山,來跟姑爺理論此事。」

張翠山點了點頭,並不言語。殷無祿探手懷中,取出三面小旗,雙手呈給張翠山,道︰「小人兄弟聽那三個鏢客膽敢太歳頭上動土,已將這事搞到了白眉教身上。」張翠山一看那三面小旗,不禁一驚,只見第一面旗上繡著一頭猛虎,側頭吼叫,作踞蟠之狀。這面小旗,自是「虎蟠鏢局」的鏢旗了。第二面小旗上繡著一頭白鶴在雲中飛翔,那是「晉陽鏢局」的鏢旗,白鶴當是指他們的總鏢頭雲鶴。第三面小旗上是用金線繡著九隻燕子,包括了「雲燕鏢局」的「燕」字和總鏢頭宮九佳的「九」字。

張翠山奇道︰「怎地將他們的鏢旗取來了?」殷無福道︰「姑爺是白眉教的嬌客,祁天彪宮九佳他們是什麼東西,明知武當七俠於他們有恩,居然還要去請什麼開封府的神槍震八方譚瑞來這個傢伙,來跟姑爺理論,那不是太豈有此理麼?這次老爺太太原是差了咱兄弟三人,來給姑爺送禮的。咱們在襄陽聽到了這三個鏢客的無禮之言\dash{}」張翠山道︰「其實也不算什麼無禮。」殷無福道︰「是,那是姑爺的寬洪大量,人所不及。咱三人可按捺不住,料理了這三個鏢客,取來了三家鏢局的鏢旗。」

張翠山吃了一驚,心想祁天彪等三人都是一方鏢局中的雄傑,江湖上成名已久,雖然算不得是武林中頂児尖児的脚色,但各有各的絶藝,何以殷天正手下三個家人,便如此輕描淡冩的説將他們料理了?但若是殷無福瞎吹,他們明明取來了這三桿鏢旗,别説明取,便是暗偸,可也不易啊。難道他們在客店中使用什麼薰香迷藥,做翻了那三個總鏢頭?便道︰「這三桿鏢旗,怎生取來的?」

殷無福道︰「當時二弟無祿出面叫陣,約他們到襄陽南門較量,咱三人對他三個。言明他們若是輸了,便留下鏢旗,自斷一臂,終身不許踏進湖北省境。」張翠山愈聽愈奇,愈是不敢小覷了眼前這兩個家人,問道︰「後來怎樣。」殷無福道︰「後來也没什麼,他們便留下鏢旗,自己砍斷了右臂,説終身不踏進湖北省境一步。」

張翠山暗暗心驚︰「這些白眉教的人物,行事竟是如此狠辣。」殷無福道︰「倘若姑爺嫌小人下手太輕,咱們便追上去,將三人宰了。」張翠山忙道︰「不輕不輕,已重得很。」殷無福道︰「咱們心想這次是來給姑爺送禮,喜事重重,若是傷了人命,似乎不吉。」張翠山道︰「不錯,你們想得很是周到。你剛纔説共有三人送禮,還有一位呢?」殷無福道︰「還有一個兄弟殷無壽。咱們趕走了三個鏢客之後,咱二人便來叩見姑爺,但恐那神槍譚老頭児終於得到訊息,不知好歹,還要來囉{\upstsl{嗦}}姑爺,是以殷無壽便上開封去。無壽叫小人代他向姑爺磕頭請安。」説著便爬下來磕頭。張翠山還了一揖,道︰「不敢當。」心想那神槍震八方譚瑞來威名赫赫,威名已垂四十年,殷無壽爲了自己而鬧上開封去,不論那一方有了損傷,都是大大的不安,説道︰「那神槍震八方譚瑞來我久仰其名,是個正人君子,兩位快些趕赴開封,叫無壽大哥不必跟譚老英雄説話了,若是雙方説僵了動手,只怕不妙。」

殷無祿淡淡一笑,道︰「姑爺不用耽心,那姓譚的老傢伙不敢跟三哥動手的。三哥叫他不許多管閒事,他會乖乖的聽話。」張翠山道︰「是麼?」他心下却是不信,暗想神槍震八方譚瑞來豈好惹的人物,他自己或許老了,可是開封府神槍譚家一家,武功極佳的弟子少説也有一二十人,那能怕了你殷無壽一人?殷無福瞧出張翠山有不信之意,説道︰「那譚老頭二十年前是無壽的手下敗將,並有重大把柄落在咱們手中。姑爺萬安。」説著二人行禮作别,出了中門。

張翠山手中拿著那三面小旗,躊躇了半晌,他本想命二人幫同打聽無忌的下落,但想若跟外人提起此事,自己也還吧了,却不免損了二哥的威名,於是慢慢踱回臥房。

殷素素斜倚在床,翻閲著父母送來的禮單,心下好生感激父母待己的親情,但想起無忌爲敵所擄,此時不知如何,又是憂心如焚,只見丈夫走進房來,臉上神色不定,忙問︰「怎麼啦?」張翠山道︰「那無福、無祿、無壽三人,却是什麼來歷?」

殷素素和丈夫成婚雖已十年,但知他對白眉教心中不喜,因此自己的家事和教中諸事,一直不跟他談起,張翠山也從來不問。這時她聽丈夫問及,纔道︰「這三人在二十多年前,本是橫行西南一帶的大盜,後來受許多高手圍攻,眼看無幸,適逢我爹爹路過,見他們死戰不屈,很有骨氣,便伸手救了他們。這三人並不同姓,自然也不是兄弟。他們感激我爹爹救命之恩,便立下重誓,終身替他爲奴,抛棄了從前的姓名,改名爲殷無福、殷無祿、殷無壽。我從小對他們很是客氣,也不敢眞以奴僕相待。我媽媽説,講到武功和從前的名望,武林中許多大名鼎鼎的人物,也未必及得上他三人。」

張翠山點頭道︰「原來如此。」於是將他三個斷人右臂、奪人鏢旗之事説了。殷素素皺起眉頭,道︰「他三人原是一番美意,却没想到名門正派的弟子,行事跟他們邪教大不相同。五哥,這件事又跟你添上了麻煩,我\dash{}我眞不知如何是好?」她頓了一頓,道︰「待尋到無忌,咱們還是回到冰火島上去吧。」忽聽得殷利亨在門外叫道︰「五哥,快來大筆一揮,冩幾副壽聯児。」又笑道︰「五嫂,你别怪我拉了五哥去,誰教他叫作『鐵鉤鐵劃』呢?」

當日下午六個兄弟督率火工道人衆道僮在玉虛門四處打掃侑置,廳堂上都貼了宋遠橋所撰、張翠山所書的壽聯,前前後後,一片喜氣。次日清晨,宋遠橋等換上了新縫的布袍,正要去擕兪岱岩,七人同向師父拜壽,忽然一名道僮進來,呈上一張名帖。宋遠橋接了過來,張松溪眼快,上面冩道︰「崑崙後學何太沖率門下弟子恭祝張眞人壽比南山。」驚道︰「崑崙掌門人親自給師父來拜壽來啦,他萬里迢迢的趕來,這個面子可是不小。」宋遠橋道︰「這位客人非同小可,該當請師父親自迎接。」忙去稟明張三丰。

張三丰道︰「這位崑崙掌門聽説從未來過中土,虧他知道老道的生日。」當下率領六名弟子,迎了出去。只見何太沖穿著一件黃衫,神情甚飄逸,氣象沖和,儼然是名門正派的一代宗主。他身後站著八名弟子,西華子和衛四娘也在其内。張三丰連聲道謝,稽首行禮。宋遠橋等六人跪下磕頭,何太沖還了半禮,説道︰「武當六俠名震寰宇,這般大禮如何克當?」

張三丰剛將何太沖師徒迎進大廳,賓主坐定獻茶,一名小道僮又持了一張名帖進來,交給了宋遠橋,却是崆峒派五老齊至。當世武林之中,少林、武當名頭最響,崑崙、峨嵋次之。崆峒派又次之,崆峒五老論到輩份地位,不過和宋遠橋平起平坐。但張三丰甚是謙沖,站起身來,説道︰「崆峒五老到來,何道兄請少坐,老道出去迎接賓客。」何太沖心想︰「崆峒五老這等人物,派個弟子去接一下也就是了。」

少時崆峒五老帶了弟子進來,何太沖並不站起,只是欠了欠身。接著神拳門、海沙派、巨鯨幫、巫山幫,許多門派幫會的首腦人物,陸續來到山上拜壽。宋遠橋等事先只想本門師徒共盡一日之歡,没料到竟來了這許多賓客,六弟子分别接賓客,却那裡忙得過來?要知張三丰一生最厭煩的便是這些煩文褥節,每逢七十歳、八十歳、九十歳的整壽,總是叮屬弟子,決不可驚動外人,那料到在這百歳壽辰,竟是武林中各路貴賓雲集。到得後來,玉虛觀中連給客人坐的椅子也不彀了。

宋遠橋等無法可想,只得去捧些圓石,密密的放在廳上,各派掌門、各幫的舵主等尚有座位,門人徒衆只好坐在石上。斟茶的茶碗分派完了,後來的只得用飯碗,菜碗喝茶。張松溪一拉張翠山,兩人走到廂房中。張松溪道︰「五弟,你瞧出什麼來没有?」張翠山道︰「他們是相互約好的,大家見面之時,顯是成竹在胸。雖然有些人假作驚異,實則是欲蓋彌彰。」張松溪道︰「不錯,他們並不是誠心跟師父拜壽來著。」張翠山道︰「拜壽爲名,問罪是實。」張松溪道︰「不,不是興師問罪,龍門鏢局的命案,決計請不動鐵琴先生何太沖親自出馬。」張翠山道︰「{\upstsl{嗯}},這些人全是爲了金毛獅王謝遜。」張松溪冷笑道︰「他們可把武當門人瞧得忒也小了。縱使他們倚多爲勝,難道武當門下弟子竟會出賣朋友?五弟,那謝遜便算是十惡不赦的奸徒,既是你的義兄,決不能從你口中吐露他的行蹤。」張翠山道︰「四哥説的是。咱們怎麼辦?」張松溪微一沉吟,道︰「大家小心些便是。兄弟同心,其利斷金,武當七俠大風大浪見慣多少,豈能怕了他們?」

兪岱岩雖然殘廢,但他們口中説起來總算還是「武當七俠」,而七兄弟之後,還有一位武學修爲震古鑠今、冠絶當時的師父張三丰在。只是各人均想師父已是百歳高齡,雖然眼前遇到了極重大的難関,但衆兄弟仍當自行料理,不但決不能讓師父出手,而且也不能讓他人家操心。可是張松溪口中這麼安慰師弟,在他内心,却知今日之事大是辣手,如何得保師門令譽,實非容易。

大廳之上,宋遠橋、兪蓮舟、殷利亨三人陪著賓客説些客套閒話。他三人也早瞧出這些客人來勢不對,心中各自{\upstsl{嘀}}咕。正説話間,小道僮又進來報道︰「峨嵋門下掌門大弟子靜虛師太,率同五位師弟妹,來向師祖拜壽。」宋遠橋和兪蓮舟一齊微笑,望著殷利亨。這時莫聲谷正從外邉陪著八位客人進廳,張松溪、張翠山剛從内堂轉出。聽到峨嵋弟子到來,也都向著殷利亨微笑。殷利亨滿臉通紅,神態忸怩。張翠山拉著他手,笑道︰「來來來!咱們兩個去迎接貴賓。」

兩人迎出門去,只見那靜玄師太已有四十來歳年紀,身材高大,神態威猛,雖是女子,却比尋常男子還高出半個頭。她身後五名師弟妹中一個是三十來歳的瘦男子,兩個是尼姑,其中那靜虛師太,張翠山已在海上舟中會過,另外兩個都是二十來歳左右的姑娘。只見一個{\upstsl{抆}}嘴微笑,另一個膚色雪白、長挑身材的美貌女郎低頭弄著衣角,那自是殷利亨未過門妻子金鞭紀家的紀曉芙姑娘了。

張翠山上前見禮道勞,陪著六人入内。殷利亨極是靦腆,一眼也不敢向紀曉芙瞧,行到廊下,見衆人均已在前,忍不住向紀曉芙望去。這時紀曉芙低著頭剛好也斜了他一眼,兩人目光相觸,紀曉芙的師妹大聲咳嗽了一聲。兩人羞得滿臉通紅,一齊轉頭,那師妹嗤的一聲笑了出來,低聲道︰「師姊,這位殷師哥比你還會害臊。」

張松溪一直在盤算敵我雙方的情勢,見峨嵋六弟子進來,稍稍寬心,暗想︰「紀姑娘是六弟未過門的妻子,待會若是説僵了動手,常言道疏不間親,峨嵋門下弟子或能助咱們一臂之力。」

各路賓客絡繹到來,轉眼已是正午。玉虛觀中絶無預備,那能開什麼筵席?火工道人只能每人送一大碗白米飯,飯上舖些青菜豆腐。宋遠橋連聲道歉。但見衆人一面吃飯,一面不停的向廳門外張望,似乎在等什麼人。

宋遠橋等細看各人,見各派掌門、各幫舵主大都自重,身上未帶兵刀,但其門人弟,有很多人腰間脹鼓鼓地,顯是暗藏兵器,只有峨嵋、崑崙、崆峒三派的弟子,纔是全部空手而來。其時武當派創派未久,武當山下尚未有「解劍巖」之設,衆人上山擕帶兵刃雖然不敬,但宋遠橋等也不便説什麼,只是心中不忿︰「你們既是來跟師父祝壽,却又爲何暗藏兵刃?」

又看各人所送的壽禮,大都是從山下鎭上臨時買的一些壽桃壽麵之類,倉卒間隨便置辦,不但跟張三丰這位武學大宗師的身份不合,也不符各派宗主、各幫首腦的氣勢。僅有峨峨派送的纔是眞正重禮,十六色珍貴玉器之外,另有十件大紅錦緞的道袍,上面用金線繡著一百個各各不同的「壽」字,花的功夫甚是不小。靜玄師太向張三丰言道︰「這是峨嵋門下十個女弟子合力繡成。」張三丰心中甚喜,笑道︰「峨嵋十女俠拳劍功夫天下知名,今日却以威震武林的神功,來給老道繡了這件壽袍,那眞是貴重之極了。」

張松溪眼瞧各人神氣,心中{\upstsl{嘀}}咕︰「不知他們還在等什麼強援?偏生師父不喜熱鬧,武當派的至交好友事先一位也没邀請,否則也不致落得這般衆寡懸殊、孤立無援。」要知張三丰交遊遍於天下,七弟子又行仗義、廣結善緣,若是事先有備,自可邀得數十位高手到來參與壽誕。

兪蓮舟在張松溪身邉悄聲道︰「咱們本想過了師父壽誕之後,發出英雄帖,在黃鶴樓頭開英雄大宴,不料一著之失,全盤受制。」他心中已盤算定當,在英雄大宴之中,由張翠山説明不能出賣朋友的苦衷。須知凡在江湖上行走的人物對這個「義」字都看得很極重,張翠山只須坦誠相告,誰也不能硬逼他做不義之徒。便是有人不肯罷休,英雄宴上自有不少和武當派交好的高手,當眞須得以武相見,也決不致落了下風。那料到對方已算到此著,竟是以祝壽爲名,先自約齊人手,湧上山來,攻了個武當措手不及。

張松溪低聲道︰「事已此,只有拚死力戰。」武當七俠中以張松溪最爲多謀,昔日遇上難題,他往往能忽生奇計,轉危爲安,兪蓮舟聽了他這句話,心下黯然︰「連四哥也是束手無策,看來今日武當六弟子要血濺出頭了。」來客之中,若是以一敵一,除了鐵琴先生何太沖之外,只怕誰也不是武當六俠的敵手,可是此刻山上之勢,不但是廿對一,且是三四十對一的局面。

張松溪扯了扯兪蓮舟的衣角,兩人走到廳後。張松溪道︰「待會説僵之後,若能用言語逼住了他們,單打獨鬥,以六陣定輸贏,咱們自是立於不敗之地,可是他們有備而來,定然想到此節,決不答應只鬥六陣便算,勢必是個群毆的結局。」兪蓮舟點頭道︰「四弟,咱們第一是要救出三弟,決不能讓他再落入外人之手,多受折辱,這件事歸你辦。五弟妹身子雖然好了,但恐未曾復原,你教五弟全力照顧她。應敵禦侮之事,由大哥和我們四人多盡些力。」張松溪點頭道︰「好,便是這樣。」他微一沉吟,道︰「眼下或有一策,可以行險僥倖。」兪蓮舟喜道︰「行險僥倖,那也説不得了。四弟有何妙計?」張松溪道︰「咱們兄弟各人認定一個對手,對方一動手,咱們一個服侍一個,一招之内便擒在手中。教他們有所顧忌,不敢強來。」兪蓮舟躊躇道︰「若不是一招便擒住,旁人定然立即上來相助,一招得手,只怕\dash{}」

張松溪道︰「大難當頭,出手狠些也説不得了。咱們使『龍爪絶戸手』!」兪蓮舟微微打了個突,遲疑道︰「『龍爪絶戸手』?今日是師父的大喜之日,用這種殺手,太狠毒了吧?」

\chapter{指名挑戰}

原來「龍爪手」是武當派中一種極其厲害的擒拿手法,兪蓮舟學會之後,總嫌其一拿之下,對方若是武功有了相當高的造詣,仍能掙脱,於是他自加變化,創出了十二招從「龍爪手」中脱胎的招數出來。要知張三丰收徒之先,曾對每個人的品德行爲資質悟性,都詳加考査,因此七弟子入門之後,無一不成大器,不但各傳師門之學,並能分别依自己天性所近,另創新招,兪蓮舟變化「龍爪手」的招數,原本不是奇事。但張三丰見他試演之後,只點了點頭,不加可否。

兪蓮舟見師父不置一詞,知道招數之中必是還存著極大毛病,於是潛心苦思,更求精進。數月之後,再演給師父看時,張三丰嘆了口氣,道︰「蓮舟,這一十二招龍爪手,比我教給你的是厲害得多了。不過你招招拿人腰眼,不論是誰受了一招,都有損陰絶嗣之虞。難道我教你的正大光明武功還不彀,定要一出便使人動彈不得麼?」兪蓮舟聽了師父這番教訓,雖在嚴冬,也不禁汗流浹背,心中慄然。

過了幾日,張三丰將七個弟子都叫到跟前,將此事説給各人聽了,最後道︰「蓮舟所創的這一十二下招數,苦心孤詣,算得上是一門絶學,若憑我一言就此廢棄,也是可惜。大家便跟蓮舟學一學,只是若非遇上生死関頭,決計不可輕用。我在『龍爪』兩字之下,再加上『絶戸』兩字,要大家記得,這路武功是教人斷子絶孫、毀滅門戸的殺手。」

當下七弟子拜領教誨,兪蓮舟便將這路武功傳了六位同門。七人學會以來,果然恪遵師訓,一次也没有用過。今日到了緊急関頭,張松溪提了出來,兪蓮舟却仍是頗有躊躇。

張松溪道︰「這『龍爪絶戸手』擒拿了對腰眼之後,使他永遠不能生育。小弟却有個計較在此,咱們只找和尚、道士作對手,要不便是七八十歳的老頭児。」兪蓮舟微微一笑,説道︰「四弟果然心思靈巧,和尚道士便是不能生児子,那也無妨。」

兩人計議已定,分頭去告知宋遠橋和三個師弟,每人認定一名對手,全待張松溪大叫一聲「啊喲!」六個人便各使「龍爪絶戸手」扣住對手。兪蓮舟選的是崆峒五老中年事最高的一老,張翠山則選了崑崙派中的西華子。

大廳上衆賓客用罷便飯,火工道人收拾了碗筷,張松溪朗聲説道︰「各位前輩、衆位朋友,今日家師百齡壽誕,承衆位光降武當,敝派上下,盡感榮寵,只是招待簡慢之極,還請原諒,家師原要邀請各位同赴黃鶴樓,共謀一酔,今日不恭之處,那時再行補謝。敝師弟張翠山遠離十載,近日方歸,他這十年來的遭遇經歷,未及詳加稟明師長,再説今日是家師大喜的日子,倘若談論武林中的恩怨鬥殺之事,未免不祥,各位遠道前來拜壽的一番好意,也變成尋事生非的惡意了。各位難得前來武當,便由在下陪同,赴山前山後賞玩風景如何?」

他這一番話甚是厲害,先將衆人的口都堵住了,那是聲明在先,今日乃壽誕吉期,倘若有人提起謝遜和龍門鏢局的事,便是存心和武當爲敵。

衆人連袂上山,便是不惜一戰,以求逼問出金毛獅王謝遜的下落來,但武當派威名赫赫,無人敢單獨與其結下樑子。倘若數百人一湧而上,那自是無所顧忌,可是要誰挺身而出,先行發難,却是誰都不想作這冤大頭。衆人面面相覷,僵持了片刻。崑崙派的西華子站起身來,大聲道︰「張四俠,你不用把話説在頭裡。咱們明人不作暗事,打開天窗説亮話,此番上山,一來是跟張眞人拜壽,二來正是要打聽一下謝遜那惡賊的下落。」

莫聲谷{\upstsl{彆}}了半天氣,這時聽了西華子之言,再也無法忍耐,冷笑道︰「好啊,原來如此,怪不得,怪不得。」西華子睜大雙目,道︰「什麼怪不得?」莫聲谷道︰「我先前聽説各位來到武當,是來給家師拜壽,但見各位身上暗藏兵刃,心下好生奇怪,難道大家帶了寶刀寶劍,來送給家師作壽禮麼?這時方才明白,送的竟是這樣一份壽禮。」西華子拍一拍身上,跟著解開道袍,大聲道︰「莫七俠瞧清楚些,小小年紀,莫要血口噴人,咱們身上誰暗藏兵刃來著。」

莫聲谷冷笑道︰「很好,果然没有。」伸出手指,輕輕在身旁兩人的腰帶上一扯。他出手快極,這麼一扯,已將兩人的衣帶拉斷,但聽得嗆{\upstsl{啷}}{\upstsl{啷}}嗆{\upstsl{啷}}接連兩聲響過,兩柄短刀掉在地下,青光閃閃,耀眼生花。原來那兩人在長袍之内暗掛短刀,莫聲谷早已瞧出,拉斷衣帶,短刀隨即落下。

這一來,衆人臉色均是大變。西華子大聲道︰「不錯,張五俠若是不肯見示謝遜的下落,那麼掄刀動劍,也説不得了。」張松溪正要大聲驚呼︰「啊喲」爲號,先發制人。忽然門外傳進來一聲︰「阿彌陀佛!」這一聲佛號清清楚楚的送入衆人耳鼓,又清又亮,似是從遠處傳來,但聽來又像是發自身旁。張三丰笑道︰「原來是少林派空智禪師到了,快快迎接。」門外那聲音接口道︰「少林寺方丈空聞,率師弟空智、空性{\upstsl{暨}}門下弟子,恭祝張眞人千秋長樂。」那空聞、空智、空性三人,是少林四大神僧中的人物,除了空見大師已死,三位神僧竟是盡數來到山上。張松溪一驚之下,那一驚「啊喲」已然叫不出來,他心知少林高手既是大舉來到武當,那麼六人便是以「龍爪絶戸手」制住了崑崙、崆峒等派中的人物,還是無用。

崑崙派掌門人鐵琴先生何太沖説道︰「久仰少林四大神僧的清名,今日有幸得見,也算是不虛此行了。」門外又一個較爲蒼老的聲音説道︰「這一位想是崑崙掌門何先生了。幸會幸會!張眞人,老衲等拜壽來遲,實是不恭。」張三丰道︰「今日武當山上嘉賓雲集,老道只不過虛活了一百歳,敢勞三位神僧玉趾?」一邉説,一邉帶同弟子迎了出去。

他四人隔著數道山門,各運内力互相對答,便如對面唔談一般。峨嵋派的靜玄師太功力不逮,便插不下口去,其餘各幫各派的人物更是心下駭然,自愧不如。

其時空聞等人離山門尚遠,張三丰率領弟子迎出,纔見三位老僧,率領著九名中年、老年的僧人,慢慢走到門前。那空聞大師白眉下垂,直覆到眼上,便似長眉羅漢一般,空性大師身軀雄偉,貌相威武;空智大師却是一臉的苦相,嘴角下垂。

宋遠橋心下暗暗奇怪,他頗精於風鑑相人之學,心道︰「若是常人生了空見大師這副容貌,不是短命,便是早遭橫禍,何以他非但得享高壽,還成爲武林中人所共仰的宗師?看來我這相人之學,所知實在大是有限。」

張三丰和空聞等雖然均是武林中的大宗師,但從未見過面。論起年紀,張三丰比他們大上三四十歳,他出身少林,若從他師父覺遠大師行輩敘班,那麼他比空聞等也要高上兩輩。但他既未在少林受戒爲僧,又没正式跟少林僧人學過武藝,大家以平輩之禮相見,宋遠橋等反而矮了一輩。

當下見禮已罷,張三丰迎著空聞等進入大殿,何太沖、靜玄師太等上前相見,互道仰慕,又是一番客套。偏生那少林方丈空聞大師極是謙抑,對每一派每一幫的後輩子弟都要合什爲禮,招呼幾句,亂了好一陣,數百人才一一引見完畢。

空聞、空智、空性三位高僧坐定,喝了一杯清茶,空聞大師説道︰「張眞人,老衲依年紀班輩説,都是你的後輩,你我武當、少林,在武林中各有聲譽,但老衲忝爲少林派掌門,有幾句話要向前輩坦率相陳,還請張眞人勿予見怪。」張三丰性子向來豪爽,道︰「三位高僧,可是爲了我這第五弟子張翠山而來麼?」空聞道︰「正是。咱們有兩件事,要請教張五俠。第一件,張五俠殺了我少林派的龍門鏢局滿局七十一口,又擊斃少林僧人六人,這七十七人的性命,該當如何了結?第二件事,敝師兄空見大師,一生慈悲有德,與人無爭,却慘被金毛獅王謝遜害死,聽説張五俠知曉那姓謝的下落,還請張五俠賜示,少林全寺僧人,盡感大德。」

張翠山站起身來,朗聲説道︰「空聞大師,龍門鏢局和少林僧人七十七口性命,絶非晩輩所傷。張翠山一生受恩師訓誨,雖然愚魯,却不敢打誑。至於傷這七十七口性命之人是誰,晩輩倒也知曉,可是不願明言。這是第一件。那第二件呢,空見大師圓寂西去,天下無不痛悼,只是那金毛獅王謝遜和晩輩有八拜之交,義結金蘭。謝遜身在何處,實不相瞞,晩輩原也知悉,但我武林中人,最重一個『義』字,我張翠山頭可斷,血可濺,我義兄的下落,決計不能吐露。此事跟我恩師無関,跟我衆同門亦無干連,由張翠山一人擔當。各位是以死相逼,要殺要剮,便請下手。姓張的生平没做半件貼羞師門之事,没妄殺過一個好人,各位今日定要逼我不義,有死而已。」

他這番話侃侃而言,滿臉正氣。空聞唸了聲︰「阿彌陀佛!」心想︰「聽他言來,倒似不假,這便如何處置?」便在此時,窗外忽然有個孩子聲音,叫道︰「爹爹!」

張翠山心頭一震,這聲音正是無忌,叫道︰「無忌,你來了?」搶步出廳,巫山派和神拳門各有一人站在大廳門口,只道張翠山要逃走,齊聲叫道︰「往那裡逃?」伸手要去抓他。張翠山思子心切,雙臂一振,將兩人摔得分向左右跌出丈餘,奔到窗外,只見空空蕩蕩,那裡有半個人影?他大聲叫道︰「無忌,無忌!」並無回音。廳中十餘人追了出來,見他並未逃走,也就不上前捉拿,站在一旁監視。

張翠山又叫︰「無忌,無忌!」仍是無人答應。殷素素這時身子已大至康復,在後堂聽見丈夫大叫無忌,急忙奔出,又驚又喜,叫道︰「無忌回來了?」張翠山道︰「我剛纔好像聽見他的聲音,追出來時却又不見。」殷素素好生失望,低聲道︰「想是你念著孩子,聽錯了。」張翠山呆了片刻,搖頭道︰「我明明聽見的。」他怕妻子出來,和衆賓客會見後多生波折,忙道︰「你進去吧!」

他回進大廳,向空聞大師行了一禮,道︰「晩輩思念犬子,致有失儀,請大師見諒。」空智大師道︰「善哉,善哉,張五俠思念愛子,如痴如狂,難道謝遜所害的那許許多多人,便無父母妻児麼?」他身子瘦瘦小小,出言却是聲若洪鐘,只震得滿廳衆人耳{\upstsl{嗡}}{\upstsl{嗡}}作響,張翠山心亂如麻,無言可答。

空聞向兩位師弟望了一眼,空智和空性都點了點頭。空聞向張三丰道︰「張眞人,今日之事如何了斷,還須請張眞人示下。」張三丰道︰「我這小徒雖無他長,却還不敢欺師。諒他也不敢欺誑三位少林高僧。龍門鏢師的人命和貴派弟子,不是他傷的。謝遜的下落,他是不肯説的。」空智冷笑道︰「但有人親眼瞧見張五俠殺害我門下弟子,難道武當門人不會打誑,少林門人便會打誑麼?」他左手一揮,從他身後走了兩名中年僧人出來。

在那兩名僧人之後,又有一名僧人,只是他身材矮小,給兩人遮住了身。那三個僧人各眇右目,正是在臨安西子湖邉,被殷素素用金針打瞎的少林僧圓心、圓音、圓業。他三人隨著空聞大師等上山,張翠山早已瞧見,心知定要對質西湖邉上的鬥殺之事,果然空智大師没説幾句話,便將三人叫了出來。張翠山心中爲難之極,西湖之畔行兇殺人,確實不是他下的手,可是眞正下手之人,這時也成了他的妻子。他夫妻情義深重,如何不加庇護?然而當此情勢之下,却又如何庇護?

\qyh{}圓」字輩三僧之中,圓音的脾氣最是暴燥,依他心性,一見張翠山便要動手拼命,礙於師伯師叔在前,這纔強自壓抑,這時師父將他叫了出來,當下大聲説道︰「張翠山,你在臨安西湖之旁,用毒針自慧風口中射入,傷他性命是我親眼目睹,難道冤枉你了?咱三人的右眼被你用毒針射瞎,難道你還想混賴麼?」張翠山此時只好辯得一分便是一分,説道︰「我武當門下,所學暗器雖説不少,但均是鋼鏢袖箭之類大件暗器。我同門七人,江湖上行走已久,可有人見過武當子弟使過金針銀針之類麼?至於金針上餵毒,那更加不必提起。」

武當七俠出手向來光明正大,武林中衆所週知,因此若説張翠山以毒針傷人,上山來的那些武林人物確是不易相信。圓音怒道︰「事到今日,你還在狡辯?那日針斃慧風,我和圓業師弟瞧得明明白白,倘若不是你,那麼是誰?」張翠山道︰「此人我倒是知曉,可便不願跟你説。我武當子弟是受你逼供之人麼?」

張翠山口齒伶俐,能言善辯,圓音狂怒之下,説話越來越是不成章法,倒將一件本來自己大爲有利之事,説成了強辭奪理一般。

張松溪接口道︰「圓音師兄,到底那幾位少林僧人傷在何人手下,一時也辯不明白,可是敝師兄兪岱岩,却明明是爲才林派的金剛指力所傷。各位來得正好,咱們正要請問,用金剛指力傷我兪二哥的是誰?」圓音張口結舌,道︰「不是我。」張松溪冷笑道︰「我也知道不是你,諒你也未必已練到這等功夫。」他頓了一頓,道︰「若是我三哥身子健好,跟貴派的高手動起手來,傷在他的金剛指力之下,那也怨他學藝不精,既然動手過招,總有死傷,那有什麼話説?難道動手之前,還能立下保單,保證毛髮不傷麼?可是我三師哥是在大病之中,身子動彈不得,那位少林弟子却用金剛指力,逼問他屠龍寶刀的下落。」他説到這裡,聲音提高,道︰「想少林派武功冠於天下,早已是武林至尊,又何必非得到這柄屠龍寶刀不可?何況那屠龍寶刀我三哥也只見過一眼,如此下手逼問,手段也未免太毒辣了。兪岱岩在江湖上也算薄有微名,生平行俠仗義,也總是替天下武林中作過不少好事,如今被少林弟子害得終身殘廢,十年來臥床不起,咱們也正要三位神僧作個交代。」爲了兪岱岩受傷、龍門鏢局滿局被殺之事,少林武當兩派,十年來早已費了不少唇舌,只因張翠山失蹤,始終難作了斷。張松溪見空智、圓音等聲勢洶洶,便又提了這件公案出來。空聞大師道︰「此事老衲早已説過,老衲曾詳査本派弟子,並無一人加害兪三俠。」張松溪伸手懷下,摸了一隻金元寶出來,但見金錠上指痕宛然,大聲道︰「天下英雄共見,害我兪三哥之人,便是在這金元寶上捏出指痕的少林弟子。除了金剛指力,還有那一家那一派的武功能捏金生印麼?」

圓音、圓業等指證張翠山,不過憑著口中言語,張松溪却取了物證出來,顯心徒託空言,又更加有力了。空聞道︰「善哉,善哉!本派僧人之中練成金剛指力的,除了咱師兄弟三人之外,另有達摩堂的五位長老。可是這五位長老是不出少林寺門,均已有三四十年之久,怎能傷得了兪三俠?」莫聲谷突然插口道︰「大師不信我五師哥之言,説他是一面之辭,難道大師所説,便不是一面之辭麼?」

空聞大師甚有涵養,雖聽他出言挺撞,也不生氣,只道︰「莫七俠若是不信老衲之言,那也無法。」莫聲谷快道︰「晩輩怎敢不信大師之言?只是世事變幻莫測,是非之際,往往出人意外,各位只道那幾位少林高僧是傷於敝師哥之手,咱們又認定敝三師兄是傷於少林高手的指下,説不定其間另有隱祕。是以晩輩之見,此事不妨從長計議,免傷少林武當兩派的和氣。倘若魯莽從事,將來眞相大白,徒貽後悔。」

空聞點頭道︰「莫七俠之言不錯。」空智突然厲聲道︰「難道我師兄空見大師的血海沉冤,就此不理麼?張五俠,龍門鏢局之事,咱們暫且不問,但那惡賊謝遜的下落,你今日説固然要你説,不説也要你説。」

宋遠橋一直默不作聲,此時眼見僵局已成,朗聲道︰「倘若那屠龍寶刀,不在謝遜手中,大師還是這般急於尋訪他的下落麼?」他説話不多,但這兩句却極是厲害,竟是直斥空智覬覦寶物,心懷貪念。

空智大怒,拍的一掌,擊在身前的朱桌之上,喀喇一響,朱桌四腿齊斷,桌面木片紛飛,登時粉碎,這一掌實是威力驚人。他大聲喝道︰「久聞張眞人武功源出少林,武林中言道,張眞人的功夫青出於藍,咱們仰慕已久,却不知此説是否言過其實。今日咱們便在天下英雄之前,斗膽請張眞人不吝賜教。」

他此言一出,大廳上群相聳動,要知張三丰成名垂七十年,當年跟他動過手之人,已死得乾乾淨淨,世上再無一人。他武功到底如何了得,武林中只是流傳各種各樣神奇的傳説而已,除了他嫡傳的七名弟子之外,誰也没有親眼見過。但宋遠橋等武當七俠威震天下,其徒已是如此,師父的本領不言可喩。這時衆人聽空智竟然向張三丰挑戰,無不大爲興奮,心想今日可目睹當世第一高手顯示武功,實是不虛此行。

衆人的目光一齊集在張三丰臉上,瞧他是否允諾,只見他微微一笑,不置可否。空聞説道︰「張眞人神功蓋世,天下無敵,咱少林三僧自非眞人對手。但實逼處此,貴我兩派的糾葛,若不是各憑武功一判強弱,總是難解。咱師兄弟三人不自量力,要聯手請張眞人賜教。張眞人高著咱們兩輩,倘若以一對一,那是對張眞人太過不敬了。」

衆人心想︰「你話倒説得好聽,却原來是要以三敵一。張三丰武功雖高,但百齡老人,精力已衰,未必能抵擋少林三大神僧的聯手合力。」

宋遠橋站起身來,説道︰「今日是家師百歳壽誕,豈能和嘉賓動手過招\dash{}」衆人聽到這裡,都想︰「武當派果是不敢應戰。」那知聽宋遠橋接下去説道︰「何況正如空聞大師言道,家師和三位神僧班輩不合,若眞動手,豈不落得個以大欺小之名?但少林高僧既然叫陣,武當七弟子,便討教少林派十二位高僧的精妙武學。」

衆人了聽了他這話,又是轟的一聲,紛紛議論起來。原來空聞、空智、空性三僧,各帶三名弟子上山,一共是十二名少林僧。衆人均知兪岱岩全身殘廢,武當七俠只剩下六俠,以六人對十二人,那是以一敵二之局,宋遠橋如此叫陣,可説是自高武當的身份上了。宋遠橋這一下看似險著,實則也是迫不得已,他深知少林三大神僧功力極高,武學的修養比自己師兄弟要深湛得多。若是單打獨鬥,自己當可和其中一人戰成平手。兪蓮舟傷後初愈,就未必能擋得住一位神僧。至於餘下的一位,不論張松溪或是莫聲谷,都非輸不可。他叫陣是師兄弟二人鬥他十二名少林僧,其實那九名少林弟子並不足畏,表面上武當派是以小敵多,實質却是武當六弟子合鬥少林三神僧。

空智大師如何不知這中間的関連,哼了一聲,説道︰「既是張眞人不肯賜教,那麼咱們師兄弟三人,逐一向武當六俠中的三位請教,三陣分勝敗,三陣中勝得兩陣者爲贏。」張松溪道︰「空智大師定要單打獨鬥,那也無不可。只是咱們師兄弟七人,除了三哥兪岱岩因遭少林弟子毒手,無法起床之外,餘下六人却是誰也不敢退後。咱們二陣分勝敗,武當六弟子分别迎戰少林派六位高僧,六陣中勝得四陣者爲贏。」莫聲谷大聲道︰「便是這樣。倘若武當派輸了,張五師哥便將金毛獅王的下落,告知少林方丈。若是少林派承讓,便請三位高僧帶同這許多拜壽爲名、尋事是眞的朋友,一齊下山去吧!」

張松溪提出這個六人對陣之法,可説已立於不敗之地,他料知大師哥、二師哥的武功大致和三大神僧相若,至於其餘的少林僧,却是勢必連輸三陣。

空智搖頭道︰「不妥,不妥。」但何以不妥,他却又難以明言。張松溪道︰「三位向家師叫陣,説是要以三對一,待得咱們要以六人對少林派十二位高僧,空智大師却又要單打獨鬥。咱們便答應單打獨鬥,大師又説不妥。這樣吧,便由晩輩一人鬥一鬥少林三大神僧,這樣總是妥當了吧?三位將晩輩一舉擊斃,便算是少林派勝了,豈不乾脆爽快?」空智勃然變色,空性突然間哈哈大笑,空聞口誦佛號︰「善哉,善哉!」空性自上武當後從未開口説過一句話,這時忽然説道︰「兩位師哥,這位張小俠獨力鬥三僧,咱們便上啊。」原來他武功雖高,但自幼出家爲僧,不通世務,聽不懂張松溪的譏刺之言。空聞道︰「師弟不可多言。」轉頭向宋遠橋道︰「這樣吧,咱們少林六僧,合鬥武當六俠,一陣定輸贏。」宋遠橋道︰「不是武當六俠,是武當七俠。」空智吃了一驚,道︰「尊師張眞人也下場麼?」

宋遠橋道︰「大師此言錯矣。與家師動手過招之人,倶已仙逝。家師怎能再行出手?我兪三弟雖然重傷,難以動彈,他又未傳下弟子,但想我師兄弟七人,自來同生共死,今日是本派生死榮辱的関頭,他又如何能袖手不顧?我叫他臨時找個人來,點撥幾下,算是他的替身。武當七弟子會鬥少林高僧,你們七位出手也好,十二位出手也好,均無不可。」空聞微一沉吟,心想︰「武當派除了張三丰和七弟子之外,並没聽説有何高手,他臨時找個人來,濟得甚事?若是請了别派的好手助戰,那便不是武當對少林派的會戰了。諒他不過要保存『武當七俠』的威名,致有此言。」於是點頭道︰「好,我少林派七名僧人,會鬥武當七俠。」

兪蓮舟、張松溪等,却都知道宋遠橋這番話的用意。原來張三丰有一套極得意的功功,叫做「眞武七截陣」。武當山供奉的眞武大帝。張三丰有一日見到眞武神像座前的龜蛇二將,想起長江和漢水之會蛇山和龜山兩山的山勢,心想長蛇靈動,烏龜凝重,眞武大帝左右一龜一蛇,正是兼收至靈至重的兩件物性,當下連夜趕到漢陽,凝望龜蛇二山,從蛇蜿蜒之勢,龜山莊固之形中間,創想出了一套精妙無方的武功出來。

只是那龜蛇二山大氣滂薄,從山勢中演化出來的武功,森然萬有,包羅極廣,決非一人之力所能同時施爲。張三丰悄立大江之濱,不飲不食凡三晝夜之久,潛心苦思,終是想不通這個難題。到第四天早晨,旭日東升,照得江面上金蛇萬道,閃爍不定。張三丰猛地省悟,哈哈大笑,就此回到武當山上,將七名弟子叫來,每人傳了一套武功。

這七套武功分别行使,固是各有精微奥妙之處,但若二人合力,則師兄弟相輔相成,攻守兼備,威力便即大增。若是三人同使,則比兩人同使的威力又強一倍,相當於四位一流高手的勁力。自此每增一人,這套武功威力便增一倍,四人相當於八位高手,五人相當於十六位,六人相當於三十二位。到得七人齊施,那是等於六十四位當世第一流高手共同進擊。須知當世之間,算得上第一流高手的,也不過是寥寥二三十人,又那有這等機緣,將這許多高手集合在一起?便是集合在一起,這許多高手有正有邪,或善或惡,又怎能齊心合力。

張三丰這七套武功因是由眞武大帝座下龜蛇二將而觸機創制,是以名之爲「眞武七截陣」,七名弟子聯手,那定然是天下無敵,便是舉世高手一齊來攻,也是必勝無疑。張三丰當時苦思難解者,是如何將這套博大異常的武功施展出來,總覺得顧得東邉,西邉便有漏洞,同時南邉北邉,均予敵人以可乘之機,後來想到可命七弟子齊施,纔破解了這個難題。只是覺得這「眞武七截陣」不能由一人施展,總是未免遺憾,但轉念想到︰「這路武功如果一人能使,豈非單是一人,便足敵六十四位當世第一流的高手,這念項頭也未免過於荒誕狂妄了。」這麼一想,自己也不禁啞然失笑。

此時宋遠橋眼見大敵當前,那少林三大神僧,究竟功力如何,當眞可説得上「深不可測」四字,自己雖想或能和其中一人打成平手,但這只是自忖之見,説不定一接上手便即一敗塗地,亦未可知,因此纔想到那套武當鎭山之寶、從未一用的「眞武七截陣」上去。

宋遠橋聽空聞大師答允以少林七僧鬥武當七俠,便道︰「請各位稍待,在下須去請三師弟臨時尋個傳人,以補足武當七弟子之數。」向兪蓮舟等使個眼色,六人向張三丰躬身告退,一齊走進内堂。

莫聲谷第一個開言,道︰「大哥,咱們今日用一下『眞武七截陣』,教少林僧見一見武當弟子的本事。只是誰來接替三哥啊?」宋遠橋道︰「此事須由大夥公決。咱們且别説,各自在掌心中冩一個名字,且看衆意如何。」莫聲谷道︰「好!」取過筆來,遞給了大師兄。宋遠橋在掌心中冩了一個名字,握住手掌,將筆遞給兪蓮舟。各人挨次冩了,一齊推開手來。只見宋遠橋、兪蓮舟、張松溪三人手掌中冩的是「五弟妹」三字。張翠山冩的是「拙荊素素」四字。莫聲谷掌心冩的是「五師嫂」三字。只有殷利亨却握住了拳頭,滿臉通紅,不肯伸手。莫聲谷道︰「咦,奇了,有什麼古怪?」硬是扳開他的手掌,只見他掌心中冩著「紀姑娘」三字。

張翠山心中大是感激,握住他手,道︰「六弟!」衆均知殷利亨是一片好心,顧念張翠山病體初愈,不宜劇鬥,想去邀請他未過門的妻子紀曉芙出馬。莫聲谷想要取笑,張翠山忙向他使個眼色制止。宋遠橋道︰「既是衆意相同,五弟,你去請弟妹出來吧。」張翠山回進臥室,邀了殷素素出來,將大廳上的情勢簡略跟她説了。

\chapter{玄冥神掌}

殷素素道︰「那龍門鏢局滿門性命,以及慧風等少林僧,那是我殺的,其時我尚未和五哥相識,此事不該累了武當派衆位哥哥兄弟。我叫他們去找白眉教我爹爹算帳便是。」張松溪道︰「弟妹,事到臨頭,咱們還分什麼彼此?何況我瞧這批人上山之意,龍門鏢局的事爲賓,尋訪謝遜爲主,而尋訪謝遜呢,又是報仇爲賓,搶奪屠龍刀是主。」莫聲谷道︰「四哥之言一點不錯,他們的主旨是覬覦那柄屠龍寶刀,不論怎麼,他們定要逼迫你們説出謝遜的下落來。」張翠山也道︰「當年空見大師也曾我義兄謝遜説過,屠龍寶刀之中,藏著一套天下無敵、鎭懾武林的武功。空見既知,空聞、空智、空性想來也必知曉。」

殷素素道︰「既是如此,一切全憑大師哥作主。只是小妹武藝低微,在這片刻之間,如何能領悟這套『眞武七截陣』的精奥?」宋遠橋道︰「其實咱師兄弟六人聯手,對付七個少林僧已操必勝之算。不過聽説弟妹金針之技神妙無方,臨敵時弟妹在旁發金針相助,更能發揮『眞武七截陣』的威力,而三弟心中,更是大感安慰。」

武當六俠心意相同,所以要殷素素加入,並非爲了制敵,而是爲了兪岱岩。要知六俠聯手合擊,那「眞武七截陣」的威力,已足足抵得三十二位一流高手。少林三大神僧縱強,其擕同上山的弟子中縱有深藏不露的硬手,但七人合力,絶無相當於三十二位一流高手的實力,乃可斷言。只是這「眞武七截陣」自得師傳以來,從未用過,今日一戰而勝,挫敗少林三大神僧,兪岱岩未得躬逢其盛,心中自是不免鬱鬱。宋遠橋等要殷素素向兪岱岩學招,算是他的替身,那麼江湖上傳揚起來,兪岱岩不出手而出手,仍是「武當七俠」並稱。這一番師兄弟間體貼的苦心,殷素素是聰明伶俐之人,三言兩語之間,便即領會。

她於是説道︰「好,這我便向三哥求教去。祇是我功夫和各位相差太遠,待會别礙手礙脚纔好。」殷利亨道︰「不會的,你祇須記住方位和脚步,那便成了。臨時倘若忘了,大夥児都會提醒你。」當下七個人一齊走到兪岱岩臥室之中。張翠山回山之後,曾和兪岱岩談過幾次,殷素素因臥病在床,直到此刻,方和兪岱岩首次見面。兪岱岩見她容顏秀麗,舉止溫雅,很爲五弟喜歡,聽宋遠橋説她要作自己的替身,擺下「眞武七截陣」去會鬥少林三大神僧,心下頗感淒涼。但他殘廢已達十年,一切也都慣了,微微一笑,説道︰「五弟妹,三哥没什麼好東西送你,作爲見面之禮,此刻匆匆,只能傳授你這陣法的方位步法。待得退敵之後,我慢慢將這陣法中的各種變化和武功的練法説與你。」殷素素喜道︰「多謝三哥。」

兪岱岩第一次聽到她問口説話,突然聽到「多謝三哥」這四個字,臉上肌肉猛地抽動,眼睛直視,用力思索一件什麼事情。張翠山驚道︰「三哥,你不舒服麼?」兪岱岩不答,只是呆呆的出神,眼色之中,透出異樣的光芒,又是痛苦,又是怨恨,顯是記憶起了畢生的恨事。張翠山回頭瞥了妻子一眼,但見她神色也是大變,臉上是恐懼和憂慮之色。

宋遠橋、兪蓮舟等望望兪岱岩,又望殷素素,誰都不明白兩人的神氣何以突然會變得如此,但各人心中,均是充塞了不祥之感。一時室中寂靜無聲,連各人的心跳也可聽得見。

祇見兪岱岩喘氣越來越急,蒼白的雙頰之上,湧上了一陣紅潮,低聲道︰「五弟妹,請你過來,讓我瞧瞧你。」殷素素身子發顫,竟是不敢過去,却伸手握住了丈夫之手。

過了好一陣,兪岱岩嘆了口氣,説道︰「你不肯過來,那也無妨,反正那日我也没見到你面。五弟妹,請你説説這幾句話︰『第一,要請你都總鏢頭親自押送。第二,自臨安府到湖北襄陽,必須日夜不停趕路,十天之内送到。若是有半分差錯,嘿嘿,别説你都總鏢頭性命不保,你龍門鏢局也是滿門雞犬不留。』」各人聽他緩緩説來,背上不自禁的都出了一身汗。

殷素素走上一步,説道︰「三哥,你果然了不起,聽出了我的聲音。那日在臨安府龍門鏢局之中,委託都大錦將你送上武當山來的,便是小妹。」兪岱岩道︰「多謝弟妹好心。」殷素素道︰「後來龍門鏢局途中出了差池,累得三哥如此,是以小妹將他鏢局子中老老少少,一起殺光。」兪岱岩冷冷的道︰「你如此待我,爲了何故?」殷素素臉色黯然,嘆了口長氣説道︰「三哥事到如今,我也不能瞞你。不過我得説明在先,此事翠山一直瞞在鼓裡,我是怕\dash{}怕他知曉之後,從此不理我。」兪岱岩道︰「那你便不用説了。反正我已成廢人,往事不可追,何必妨礙你夫婦之情?你們都去吧!武當六俠會鬥少林高僧,勝算在握,不必讓我徒擔虛名了。」

兪岱岩自受傷以來,骨氣極硬,從不呻吟抱怨。他本來連説也不會説,但經張三丰悉心調治,以數十年修爲的精湛内力度入他的體内,終於漸漸能開口説話,但他對當日之事始終絶口不提,直至今日,方才説出這幾句悲憤的話來,衆師兄弟聽了,無不熱血沸騰,殷利亨更是哭出聲來。

殷素素道︰「三哥,其實你心中早已料到,只是顧念著張翠山的師兄弟之義,是以隱忍不説。不錯,一日在錢塘江中,躱在船艙中以蚊鬚針傷你的,便是小妹\dash{}」張翠山大喝︰「素素當眞是你?你\dash{}你怎地不早説?」殷素素道︰「傷害你三哥的罪魁禍首,便是你的妻子素素,我怎敢跟你説,三哥,後來以掌心七星釘傷你、騙了你手中屠龍寶刀的那人,便是我親兄殷野王。咱們白眉教跟武當派素無仇冤,屠龍寶刀既得,又敬重你是個好漢子,是以叫龍門鏢局將你送回武當。至於途中另起風波,却是始料所不及了。」

張翠山全身發抖,目光中如要噴出火來,指著殷素素道︰「你\dash{}你騙得我好苦!」兪岱岩突然大叫一聲,身從床板上躍起,砰的一響,摔了下來,四塊木板一起壓斷,人却昏暈了過去。

殷素素拔出佩劍,倒轉劍柄,遞給張翠山,道︰「五哥,你我十年夫妻,蒙你憐愛,情義深重,我今日死而無冤,盼你一劍將我殺了,以全你武當七俠之義。」張翠山接過劍來,一劍便要遞出,刺向妻子胸膛,但一霎之時間,十年來妻子對自己溫順體貼,柔情蜜意,種種好處登時都湧上心來,這一劍如何刺得上手?

他呆了一呆,突然大叫一聲,奔出房去,殷素素、宋遠橋等六人不知他要如何,一齊跟出,只見他急奔至廳,向張三丰拜了幾拜,説道︰「恩師,弟子大錯已經鑄成,無可挽回,弟子只求你一件事。」張三丰不明其中之理,溫顏道︰「什麼事,你説吧,爲師絶無不允。」張翠山磕了三個響頭,道︰「多謝恩師。弟子有一獨生愛子,落入奸人之手,盼恩師救他出於魔掌,撫養他長大成人。」猛地轉過身來,向著空聞大師、鐵琴先生何太沖、峨嵋派靜玄師太等一干人朗聲説道︰「所有罪孼,全是張翠山一人所爲。大丈夫一身作事一身當,今日教各位心滿意足。」説著橫過長劍,在自己頭頸中一劃,鮮血迸濺,登時斃命。

張翠山死志甚堅,知道橫劍自刎之際,師父和衆同門定要出手相阻,是以置身於衆賓客之間,説完了那兩句話,立即出手。張三丰及兪蓮舟、張松溪、殷利亨四人齊聲驚呼搶上。但聽砰砰幾聲,五個人飛身摔出,已是無法挽救。宋遠橋、莫聲谷、殷素素三人出來較遲,相距更遠。

便在此時,窗外一個孩童的聲音大叫︰「爹爹,爹爹!」第二句聲音發悶,顯是被人按住了口。張三丰身形一晃,已到了窗外,只見一人穿著蒙古軍裝束的漢子,手中抱著一個八九歳的男孩。那男孩嘴巴被按,却兀自用力掙扎。

張三丰愛徒慘死如刀割,但他近百年的修爲,心神不亂,低聲喝道︰「進去!」那人左足一點,抱了孩子欲待躍上屋頂,突覺肩頭一沉,身子滯重異常,雙足竟是無法離地,原來張三丰悄没聲的欺近身來,一手已輕輕搭在他的肩頭。那人大吃一驚,心知張三丰只須内勁一吐,自己不死也得重傷,只得依言走進廳去。

那孩子正是張翠山的児子無忌。他被那人點了啞穴,可是跟謝遜學的武功甚是奇特,不到一頓飯時分,體内眞氣轉動,便不知不覺的衝開被點的穴道。他在窗外見父親橫自刎,如何不急,終於大聲叫了出來。

殷素素見丈夫爲了自己而自殺身亡,突然間又見児子無忌恙歸來,大悲之後,繼以大喜,問道︰「孩児,你没説你義父的下落麼?」無忌昂然道︰「他便打死我,我也不説。」殷素素道︰「好孩子。讓我抱抱你。」張三丰道︰「將孩子交給她。」那人依言把無忌遞給了殷素素。無忌撲在母親的懷裡,道︰「媽,他們爲什麼逼死我爹爹?是誰逼死爹爹的?」殷素素道︰「這裡許許多多人,一齊上山來逼死了你爹爹。」無忌一對小眼從左而右緩緩的橫掃一遍,他年紀雖小,但目光之中,竟是充滿了威嚴和怨毒,每人眼光和他凜然生威的目光相觸,心中忍不住一震。

殷素素道︰「無忌,你答應媽一句話。」無忌道︰「媽,你吩咐吧!」殷素素道︰「你别心急報仇,要慢慢的等著,慢慢的等著,只是一個人也不要放過。」衆人聽了她這幾句冷冰冰的言語,背上不自禁的感到一陣寒意。只聽無忌道︰「是,媽媽,我要慢慢的等著,一個人也不放過。」

殷素素身子顫動,説道︰「孩子,你爹爹既然死了,咱們只得把你義父的下落説給人聽。」無忌急道︰「不,不能。」殷素素道︰「空聞大師,我只説給你一人聽,請你俯耳過來。」這一著大出衆人意料之外。空聞道︰「善哉,善哉!女施主若是早説一刻,張五俠也不必喪身。於是走至殷素素身旁,俯身過去。」

殷素素嘴巴動了一會,却没發出半個聲音。空聞道︰「什麼?」殷素素道︰「那金毛獅王謝遜,他是躱在\dash{}」「躱在」兩字之,聲音又是模糊之極聽不出半點。空聞又道︰「什麼?」殷素素道︰「便是在那児,你自己去找他吧。」空聞大急,道︰「我没聽見啊。」殷素素冷笑道︰「我只能説得這般,你到了那邉,自會見到。」他抱著無忌,低聲道︰「孩児,你大了之後,要提防女人騙你,越是好看的女人,越會騙人。」她將嘴巴湊到無忌耳邉,極輕輕的道︰「我没跟這和尚説,我是騙他的\dash{}你瞧\dash{}你媽多會騙人!」説著淒然一笑。空聞大聲道︰「女施主\dash{}」突然間殷素素雙手一鬆,身子斜斜一倒,只見她胸口插了一把匕首。原來她在抱住無忌之時,已暗自用匕首自刺,只是無忌擋在她的身前,誰也没有瞧見。

無忌撲到母親身上,大叫︰「媽媽,媽媽!」但殷素素自刺已久,支持了好一會,這時已然斷氣。無忌悲痛之下,竟不哭泣,從母親身上拔出匕首,血淋淋的握在手裡,瞪視著空聞大師,冷冷的道︰「是你殺死我媽媽的,是不是?」空聞陡然間見此人倫慘變,雖是當今第一武學宗主的掌門,也是大爲震動,經無忌這麼一問,不自禁的退了一步,忙道︰「不,不是我。是她自盡的。」無忌眼中泪水滾來滾去,但他拚命忍住,説道︰「我不哭,我一定不哭,不哭給這些惡人看。」他拿著匕首,從廳左慢慢走到右,將這三百餘人的面貌長相,一一的記在心裡,腦海中響著母親的那兩句話︰「要慢慢的等著,只是一個人也不要放過。」上得武當山來之人,不是武學大派的高手,便是獨霸一方的首腦人物,既敢來向張三丰和武當七俠惹事生非,自是膽量氣魄在在高人一等,但被無忌這般滿腔怨毒的一瞪,人人心中竟是不禁發毛。

空聞大師輕輕咳嗽一聲,説道︰「張眞人,這等變故\dash{}{\upstsl{嗯}}\dash{}{\upstsl{嗯}},\dash{}實非始料所及。五俠夫婦既已自盡,那麼前事一槩不究,咱們就此告辭。」説罷合什行禮。張三丰還了一禮,淡淡的道︰「恕不遠送。」少林僧衆一齊站起,便要走出。殷利亨喝道︰「你們\dash{}你們逼死我五哥\dash{}」但轉念一想︰「五哥所以自殺,實是爲了對不起三哥,却跟他們無干。」一句話説了一半,再也接不下口去,伏在張翠山的屍身之上,放聲大哭。衆人心中都覺不是味児,一齊向張三丰告辭,均想︰「這一個冤仇結得不小,武當決計不肯善干罷休。」只有宋遠橋紅著眼睛,送賓客出了觀門,轉過頭來時,眼泪已是奪眶而出,只聽大廳上,人人痛哭失聲。

峨嵋派衆人最後起身告辭。紀曉芙見殷利亨哭得傷心,眼圏児也自紅了,這時也顧不得害羞,走近身去,低聲道︰「六哥,我去啦,你\dash{}你自己多多保重。」殷利亨泪眼模糊,抬起頭來,哽咽道︰「你們\dash{}你們峨嵋派\dash{}也來跟我五哥爲難麼?」紀曉芙忙道︰「不是的,家師有命,想請張師兄示知謝遜的下落。」無忌突然接口道︰「我媽已跟那個老和尚説了,你問他去便是。他若是不肯説,你們跟他爲難吧。」他雖在悲痛之中,仍是懂得母親臨死那一招「嫁禍江東」之計的用意。

紀曉芙道︰「好孩子,你殷六叔定會好好照顧你。」她話中之意,是説將來我和殷利亨定會當你親生孩児一般,只是這句不便出口。她從頭頸中除下一個黃金項圏,要套在無忌頸中,柔聲道︰「這個給了你\dash{}」無忌霍地跳開,厲聲道︰「我不要仇人的東西!」紀曉芙大是{\upstsl{尷}}尬,手中拿著那個項圏,不知如何下台。無忌大聲道︰「你們快走,我要哭了。等仇人走乾淨了,我纔哭。」紀曉芙柔聲道︰「孩子,我們不是你仇人。」無忌咬牙不語,突然説道︰「越是好看的女人,越會騙人。」紀曉芙給她説得滿臉通紅,幾乎要哭了出來。靜玄師太臉一沉,道︰「師妹,跟小孩多説什麼?咱們快走吧!」無忌{\upstsl{彆}}了良久,待靜玄、紀曉芙等出了廳門,正要大哭,豈知一口氣轉不過來,咕{\upstsl{咚}}一聲,摔倒在地。兪蓮舟急抱起,知他悲痛之中,忍住不哭,是以昏厥,説道︰「孩子,你哭吧!」在也胸口推拿了幾下。豈知無忌這口氣竟是轉不過來,全身冰冷,鼻孔中氣息極是微弱,兪蓮舟運力推拿,他竟是始終不醒。衆人見他轉眼也要死去,無不失色。

張三丰暗嘆︰「此児剛強如斯,又是至情至性之人。」伸手按在背心「靈台穴」上,一股渾厚的内力,隔衣傳送過去。以張三丰此時的内功修爲,只要不是立時斃命氣絶之人,不論受了多重的損傷,内力一進,定當好轉,那知他内力透進無忌體中,只見他臉色由白轉青、由青轉紫,身子更是顫抖不已。身子更是顫抖不已。張三丰伸手在他額頭一摸,觸手冰冷,宛似摸到一塊寒冰一般,一驚之下,右手又伸到他背心衣服之内,但覺他背心上有一處宛似炭炙火燒,四周却是寒冷澈骨。若非張三丰武功已至境,這一碰之下,只怕自己也要冷得發抖。張三丰説︰「遠橋,抱孩子進來的那個韃子兵,你找找去。」宋遠橋應聲出外,兪蓮舟曾跟那蒙古兵對掌受傷,知道大師兄也非他的敵手,忙説︰「我也去。」兩人並肩出廳。張三丰押著那蒙古兵進廳之時,張翠山已自殺身亡,跟著殷素素又自盡殉夫,各人悲痛之際,誰也没留心那蒙古兵,一轉眼間,他便走得不知去向。張三丰撕開無忌背上衣服,只見細皮白肉之上,清清楚楚的印著一個碧綠的五指掌印。那掌印碧油油的發亮,青翠欲滴。張三丰再伸手撫摸,只覺掌印處炙熱異常,周圍却是冰冷,伸手摸上去時已是極不好受,無忌身受此傷,其難當處可想而知。

只見宋遠橋和兪蓮舟飛身進廳,説道︰「山上已無外人。」兩人見到無忌背上奇怪的掌印,也都吃了一驚。張三丰皺著眉頭道︰「我只道三十年前百損頭陀一死,這陰毒無比的玄冥神掌已然失傳,豈知世上居然還有人會這功夫。」宋遠橋驚道︰「這娃娃受的竟是玄冥神掌麼?」他年紀最長,曾聽過「玄冥神掌」的名頭,至於兪蓮舟等,連這路武功的名字也從未聽過。張三丰道︰「這碧綠色的掌印,是玄冥神掌唯一的標記。」殷利亨道︰「師父,用什麼傷藥?我這就取去。」張三丰嘆了口氣,並不回答,臉上老泪縱橫,雙手抱著無忌,望著張翠山的屍身,説道︰「翠山,翠山,你拜我爲師,臨去時重託於我,可是我連你的獨生愛子也保不住。我活到一百歳有什麼用?武當派名震天下又有什麼用?我還不如死了的好!」

衆弟子各人大驚,各人從師以來,始終見他逍遙自在,從未聽他説過如此消沉哀痛之言。殷利亨道︰「師父,這孩子當眞無救了麼?」張三丰雙臂橫抱無忌,在廳上東闖西走,説道︰「除非\dash{}除非我師父覺遠大師復生,將全部九陽眞經傳授於我。」衆弟子一聽,每人的心都沉了下去,覺遠大師逝世已八十餘年,豈能復生?那便是説無忌的傷勢再也無法治愈了。兪蓮舟忽道︰「師父,那日弟子跟他對掌,此人掌力果然陰狠毒辣,世所罕見,弟子當場受傷。可是此刻弟子傷勢已愈,似無後患,運氣用勁,尚無窒滯。」張三丰道︰「那是託了你們『武當七俠』大名的福。要知這玄冥神掌和人對掌,若是對方内力勝過了他,掌力回激入體,那麼施掌者身受其禍,同樣的無法救治。以後再遇上此人,可得千萬小心。」兪蓮舟心下凜然︰「原來那人過於持重,怕我掌力勝他,是以未施玄冥神掌,否則我此刻早已性命不保。下次若再相遇,他下手便不容情了。」

六個人在大廳上呆了良久,無忌忽然叫道︰「爹爹,爹爹。我痛,痛得很。」緊緊的摟住張三丰,將頭貼在他的懷裡。張三丰心中大是憐惜,一咬牙,説道︰「咱們盡力而爲,他能再活幾時,瞧老天爺的慈悲吧。」對著張翠山的屍體揮泪叫道︰「翠山,翠山!好命苦的孩子。」抱著無忌,走進自己書房,手指連伸,點了他身上十八處大穴。

無忌穴道被點,登時不再顫抖,臉上紫色却是越來越濃。張三丰知道那紫色一轉成黑色,便此氣絶無救,當下除去無忌上身衣服,自己也解開道袍,胸膛和他背心相貼。

這時宋遠橋和殷利亨在外料理收殮張翠山夫婦的喪事。兪蓮舟、張松溪、莫聲谷三人到師父雲房中,見了他這等情景,知道師父正以「純陽無極功」吸取無忌的陰寒毒氣。張三丰自來未婚娶,雖到百歳,仍是童男之體,八十載的修爲,那「純陽無極功」自是練到了登峰造極的地步。兪蓮舟等一旁服侍,知道這種以内力療傷的行功,極是危險,稍有運用不當,不但被治者立受大害,而施功之人,也蒙走火入魔之災,三人均想︰「師父功力之純,當世自無其匹,但老人家究已百歳高齡,氣血就衰,可别禍不單行,再出岔子。」三人戰戰兢兢的守候在旁,過了約莫半個時辰,只見張三丰臉上隱隱現出一層綠氣,十根手指尖微微顫動。他睜開眼來,説道︰「蓮舟,你來接替,一到支持不住,便交給張松溪,千萬不可勉強。」兪蓮舟解開長袍,將無忌抱在懷裡,肌膚相貼之際,不禁打了個冷戰,便似懷中抱了一塊寒冰相似,忙道︰「七弟,你叫人去生幾盆炭火,越旺越好。」不久炭火點起,兪蓮舟却兀自冷得難以忍耐,小腹中的純陽之氣,竟是極難凝聚,才知那「玄冥神掌」的威力,實是非同小可。

張三丰坐在一旁,慢慢以眞氣通走三関鼓盪丹田中的「氤氳紫氣」,將吸入自己體内的寒毒一絲一絲的化掉。待得他將寒毒化盡,站起身來時,只見已是莫聲谷將無忌抱在懷裡,兪蓮舟和張松溪各人坐在一旁,垂帘入定,化陰體内的寒毒。不久莫聲谷便已支持不住,命道僮去請宋遠橋和殷利亨來接替。

這等以内力療傷,功力深淺,立時顯示出來,絲毫假借不得,莫聲谷只不過支持到一盞熱茶的時分,宋遠橋却可支持到兩柱香。殷利亨將無忌一抱入懷,立時大叫一聲,全身打戰。張三丰驚道︰「把孩子給我。你坐在一旁凝神調息,不可心有他念。」原來殷利亨心傷五哥慘死,一直昏昏沉沉,神不守舍,直到神寧定之後,纔將無忌抱回。

如此六人輪流,三日三夜之内,勞瘁不堪。好在無忌體中寒毒漸解,每人支持的時候逐步延長,到第四日上,六人才得偸出餘暇,稍一合眼入睡。自第八日起,每人分别助他療傷兩個時辰,各人方得慢慢修補損耗的功力。

初時無忌大有進展,體寒日減,神智日復,漸可稍進飲食,衆人只道他這條小性命是救回來了,豈知至三十六日上,兪蓮舟陡然發覺,不論自己如何催動内力,無忌身上的寒毒已是一絲也吸不出來。可是他明明身子冰涼,臉上綠氣未褪。兪蓮舟還道自己功力不濟,當即跟師父説了,張三丰一試,竟也是無法可施。接連五日晩之中,六個人千方百計,用盡了所知的各種運氣之法,却是没半點功效。

無忌道︰「太師父,我手脚都暖了,但頭頂、心口、小腹三處地方,却越來越冷。」張三丰暗暗心驚,安慰他道︰「你的傷已好了,咱們不用成天抱你啦,你在太師父的床上睡了一會吧。」無忌道︰「是!」爬下地來,向張三丰、宋遠橋等每人磕了幾個頭,説道︰「太師父和伯父叔叔們救了無忌的性命,還求教無忌武功,將來好替我爹爹媽媽報仇。」衆人見他小小年紀,居然這般懂事,無不心酸,各各溫言慰撫。

張三丰和衆徒走到廳上,嘆了口氣道︰「寒毒侵入他頂門、心口和丹田,非外力能解,看來咱們四十天的辛苦,全是白耗了。何以竟會如此,這事實在令人大爲費解。」

衆人沉吟半晌,想不出中間的道理,若説那「純陽無極功」不能化除陰毒,何以先前有效,到了第三十七日上却忽然失其效用?何以無忌四肢頸腹都盡溫暖,只有頂門、心口、丹田三處却寒冷無比?兪蓮舟尋思了一陣,忽道︰「師父,莫非無忌在中了玄冥神掌之後,自運内力與之相抗,一個用得不當,陰毒和他内力糾結膠固,再也吸拔不出?」張三丰搖頭道︰「這小小孩童,便算翠山傳過他一些運氣吐納之學,能有多大内力?」兪蓮舟道︰「不,師父,這孩子的内力並不弱啊。」當下説起他以一招「神龍擺尾」,將一名巫山幫弟子擊成重傷之事。

張三丰一拍大腿,説道︰「是了。原來他是學了金毛獅王謝遜的奇門武功。倘若他的内功是翠山所授,那是玄門之學,咱們的純陽無極功和他内力水乳交融,相輔相成,自是見效更快。可是那謝遜所學,却是什麼武功呢?」當下回進雲房,對無忌道︰「孩子,太師父要考量一下你的武功,你打我三掌。」無忌道︰「我不敢打太師父。」張三丰笑道︰「你如不用全力,我怎知你功夫的深淺?如何能彀教你?」

無忌道︰「好!我就打你。太師父可别用力還手啊。」張三丰笑道︰「不用怕。」無忌身子橫斜,右掌自右上向左下撲擊,却是一招降龍十八掌中的「見龍在田」,張三丰左掌接住,無忌的掌力登時消得無影無蹤,這一掌便如擊空一般。張三丰點了點頭,道︰「不錯!」無忌見他一擊成功,轉過身來,向後揮擊一掌,那是一招「神龍擺尾」。張三丰用右掌接了,無忌仍如擊在空中一般,絲毫感不到張三丰回震之力。張三丰却讚道︰「很好,小小孩子,練到這樣,那是極不容易了。」

無忌紅著小臉,道︰「太師父,我不打啦,打你不著。」張三丰道︰「這兩掌打得很好,再來一掌。」無忌左手劃個圏子,右手推出一掌,却是降龍十八掌中的「亢龍有悔」。張三丰微一驚︰「他居然會這路掌法。」但接上手,便覺這一掌雖然來勢剛猛,但其掌力却遠不及先兩招的精純,便搖頭道︰「這一招不好,想是你没學會。」無忌忙道︰「不是的。是我義父没學會。他説降龍十八掌是天下武功中最厲害的本事之一,可惜他只學會了一點児。這招『亢龍有悔』,義父説他也不十分明白其中的精奥之處,可是要我先學著,將來慢慢的想,説不定自己會想明白。」

張三丰點頭道︰「這就是了。這一招和人眞正動手之時,千萬不能使,否則自己會反受其害。」無忌道︰「太師父,你教我吧。」張三丰搖頭道︰「我不會。自從郭靖郭大俠在襄陽殉國,降龍十八掌已經失傳。」當下細細盤問無忌所學的各種武功,無忌一一説了。

張三丰越聽越奇,心想這金毛獅王之學,實是淵博到了極處,各門各派的武功,無不涉獵,可是並未由博返約,自成一家,因之也無特别精純的極高功夫。當然,那也是無忌年紀太小,如何能學到義父的得意本事?但聽無忌背誦如流,口訣拳經,心中記得不計其數,有許多甚至是張三丰也從未聽見過。原來謝遜當年爲要激使成崑出面,殺害了不少各家各派的好手,殺人之後,順手便將他們的拳經劍譜擕走,以備日後遇上他們的同門前來尋仇之時,可以預知對方的武功家數。

可是當要無忌一演,他却十九不會,只説義父教他誦招數歌訣,如何變化,却已來不及傳授。張三丰點了點頭,道︰「很好,很好!」心下暗嘆那謝遜對待無忌實是一片苦心,不願讓他在荒島耽誤了青春,却又在數年之内,教他記住了自己畢生所學,料想無忌日後長大,以他如此聰明的資質,自會逐步領悟。

\chapter{重上少林}

這三掌一對,張三丰知道無忌所學内功雜而不精,以之臨敵固能速成,但和玄冥神掌中所留的寒毒膠纏固結,已是無法吸出體外,除非使其氣息全然停止。但一人氣息一絶,立時死亡,還説什麼吸取寒毒?張三丰沉吟良久,心想︰「要解他體内寒毒,旁人已無可相助,只有他自己修習『九陽眞經』中所載最高無上的内功,方能以至陽化其至陰。但當時先師覺遠大師背誦經文之時。我記憶不全,至今雖閉関數次,苦苦鑽研,仍是只能通得三四成。眼下無法可施,只能教他自練,能保得一日之命,便是多活一日。」

當下將「九陽眞功」的練法和口訣,傳了無忌。這一門功夫看似簡單,但其中變化繁複,非一言可盡,簡言之,初步功夫是練「大周天搬運」,使一股暖烘烘的眞氣,從丹田中先向鎖陰任、督、衝三脈的「陰蹻庫」流注,折而走向尾閭関,然後分兩支上行,經腰脊第十四椎兩旁的「轤轆関」,上行經背、肩、頸而至「玉枕関」,此即所謂「逆運眞氣通三関」。然後眞氣再上行越過頭頂的「百會穴」,分五路上行,與全身氣脈大會於「膻中穴」,再分主從兩支,還合於丹田,入竅歸元。這樣循環一周,身子便如灌甘露,丹田裡的眞氣有似香煙繚繞,悠遊自在,蕩漾漾,輕飄飄,似動似止,載沉載浮,那就是所謂「氤氳紫氣」。這氤氳紫氣練到火候相當,便能化除丹田中的寒毒,但上行而化除百會和膻中穴的寒毒。各派内功的道理無多分别,練法却截然不同,張三丰所授的心法,以威力而論,可算得天下第一。

無忌依法修練,練了兩年有餘,丹田中的氤氳紫氣已有小成,可是身上寒毒實在太過厲害,他體内所蓄的眞氣熱力非但無法化除寒毒,反而臉上的綠意日盛一日,每當寒毒發作,所受熬煎也是一次比一次更是厲害。

在這兩年之中,張三丰全力照顧無忌内功的進修,宋遠橋等人到處爲他找尋靈丹妙藥,什麼百年以上的野山人參、成形首烏、雪山茯苓等珍奇靈物,也不知給無忌服了多少,但始終如石投大海。衆人見他日漸憔悴廋削,雖然見到他時均是強顏歡笑,心上却無不黯然神傷,心想張翠山留下的這唯一骨血,終於無法保住。

武當諸人忙於救傷治病,也無餘暇去追尋傷害兪岱岩和無忌的仇人,這兩年中白眉教教主殷天正數次遣人來探望外孫,贈送不少貴重禮物,但武當諸俠心恨兪張二俠均是間接害在白眉教手中,每次均將白眉教的使者逐下山去,禮物退回,一件不收。有一次莫聲谷還動手將使者狠狠打了一頓,從此殷天正也不再派人上山了。

這一日中秋佳節,武當諸俠和師父賀節,還未開席,無忌突然發病,臉上綠氣大盛,寒戰不止。他怕掃了衆人的興緻,咬牙強忍,但這情形又有誰看不出來?殷利亨將無忌拉入房中,蓋上棉被,又生了一爐旺旺的炭火。張三丰忽道︰「明日我帶同無忌,上嵩山少林寺走一遭。」衆人明白師父的心意,那是他無奈何之下,迫得向少林派低頭,親自去向空聞大師求救,盼望少林高僧能補全「九陽眞功」中的不足之處,挽救無忌的性命。

兩年前玉虛觀中一會,少林、武當雙方嫌隙已深,張三丰又是一代宗師,竟然降尊紆貴,不恥求教,那自是大失身份之事。衆人念著張翠山的情義,明知張三丰一上嵩山求教,自此武當派見到少林派時再也抬不起頭來,但這些虛名也顧不得了。本來峨嵋派也傳得一份「九陽眞經」,但滅絶師太決不外傳,張三丰數次致書通候,命殷利亨送去,滅絶師太連封皮也不拆,便將書信原封不動的退了回來,眼下除了向少林寺低頭求教,再無别法了。

若由宋遠橋率領衆師弟上嵩山少林寺求教,雖於武當派顏面較好,但空聞大師決不肯以「九陽眞經」的眞訣相授,勢所必然。衆人想起二三十年來威名赫赫的武當派從此要拱手向少林稱臣,心下均是鬱鬱不樂,這一場慶賀團圓佳節的酒宴,也就在幾杯悶酒之後,草草散席。

次日一早,張三丰帶同無忌啓程,宋遠橋等一直送下山來。五弟子本想隨行,但張三丰道︰「咱們若是人多勢衆,不免引起少林派的疑心,還是由咱們一老一少兩人去的好。」兩人各騎一匹青驢,一路向北。少林、武當兩大武學宗派,其實相距甚近,自鄂北的武當山至豫西嵩山,數日即至。張三丰和無忌自老河口渡過漢水,到了南陽,北行汝州,再折而向西,便是嵩山。兩人上了少室山,便將青驢繫在樹下,捨騎步行。張三丰舊地重遊,憶起八十餘年之前,師尊覺遠大師挑了一副鐵擔,帶同郭襄和自己逃下少林,此時回首前塵,豈止隔世?他心下甚是感慨,擕著無忌之手,緩緩上山,但見五峰依舊,碑林如昔,可是覺遠、郭襄諸人,却早已不在人間。

兩人到了立雪亭,少林寺已然在望,只見兩名少年僧人談笑著走向亭來。張三丰打個問訊,説道︰「相煩師父通報,便説武當山張三丰有事求見方丈大師。」那兩名僧人聽見張三丰的名字,吃了一驚,一齊向他打量,但見他身形高大異常,鬚髮如銀,臉上紅潤光滑,笑咪咪的甚是可親,一件青布道袍却是汚穢不堪。要知張三丰任性自在,不修邉幅,江湖上背地裡稱他爲「邋遢道人」,也有人稱之爲「張邋遢」的。那兩個少年僧人心想︰「張三丰是武當派的大宗師,武當派跟咱們少林派向來不和,難道是生事打架來了嗎?」只見他擕著一個面青肌瘦的十一二歳的少年,兩個都是貌不驚人,不見有甚麼威勢。一名僧人問道︰「你便眞是武當山的張\dash{}張眞人麼?」張三丰笑道︰「貨眞價實,不敢假冒。」另一名僧人聽他説話並無一派宗師的莊嚴氣槩,更加不信起來,問道︰「你眞不是開玩笑麼?」張三丰笑道︰「張三丰有什麼了不起,冒他的牌子有什麼好處?」兩名僧人將信將疑,飛步回寺通報,過了良久,只見寺門開處,方丈空聞大師率同師弟空智、空性走了出來,三人身後,跟著五位身穿深黃僧袍的老和尚。張三丰知道是達摩院的護法,輩份説不定比方丈還高,在寺中精研武學,從來不問外事,想是聽到武當派掌門人到來,此事非同小可,這纔隨同方丈出迎。

張三丰搶山亭去,稽首行禮,説道︰「有勞方丈和衆位大師出迎,小道如何克當?」空聞等一齊合什還禮,空聞道︰「張眞人遠來,大出小僧意外,不知有何見諭?」張三丰道︰「便有一事相求。」空聞道︰「請坐,請坐。」張三丰在亭中坐定後,即有僧人送上茶來。張三丰心中不禁有氣︰「我好歹也是一派宗師,總也算是你們前輩,如何不請我進寺。却在半山坐地?别説是我,便是尋常客人,也不該如此禮貌不周。」但他生性隨便,一轉念間,也就不放在心上了。

空聞却道︰「張眞人光降敝山,原該恭迎入寺,只是張眞人少年之時不告而離少林,本派數百年的規矩,張眞人想亦知道,凡是本派棄徒叛徒,終身不許不再入寺門一步,否則當受削足之刑。」張三丰哈哈一笑,道︰「原來如此。小道幼年之時,雖曾在少林寺服侍覺遠大師,但那是掃地烹茶的雜役,既没剃度,亦不拜師,説不上是少林弟子。」空智冷冷的道︰「可是張眞人却從少林寺中偸學了武功去。」

張三丰氣往上衝,但轉念想道︰「我武當派的武功,雖然是四十歳後潛心所創,但推本溯源,若不是覺遠大師傳我『九陽眞經』,郭女俠贈了我那一對鐵羅漢,此後一切武功,全是無所憑依。他説我的武功得自少林,也不爲過。」於是心平氣和的説道︰「小道今日,正是爲此而來。」

空聞和空智對望了一眼,心想︰「不知他來幹什麼?想未必有好意。」空聞便道︰「請示其詳。」張三丰道︰「適纔空智大師言道,小道武功,得自少林,此言本是不錯。小道當年服侍覺遠大師,得蒙授以達摩老祖親手所書的『九陽眞經』,只是小道年幼,所學不全,至今實以爲憾。其時覺遠大師荒山誦經,有幸得聞者共是三人,一位是峨嵋派創派祖師郭襄女俠,一位是貴派無色禪師,另一人便是小道。小道年紀最小,資質最魯,又無武學根基,三派之中,所得算是最少的了。」

空智冷冷的道︰「那也不然。張眞人自幼服侍覺遠,這數年之中,他豈有不存私心暗中傳你之理?今日武當派名揚天下,那便是覺遠之功了。」覺遠的輩份比空智長了三輩,他該當稱之爲「太師叔祖」纔是,但覺遠中途逃出少林,被視爲棄徒,派中輩名已除,因之空智口氣之中,也就不存禮貌。張三丰恭恭敬敬的站起身來,説道︰「先師的恩德,小道無時或忘。」

少林四大神僧中,空見慈悲爲懷,可惜逝世最早;空聞城府極深,喜怒不形於色;空性渾渾噩噩,不通世務;只有空智氣量褊隘,常覺張三丰自少林寺中偸學了武功去,反而使武當的名望,浸浸然有凌駕少林之勢,心中大是不忿。他認定張三丰這次來到少林,是爲張翠山之死報仇洩憤。何況那日殷素素臨死之時,假意將謝遜的下落告知空聞,這一著「移禍江東」之計使得極是毒辣。兩年多來,每個月中均有武林人士來到少林滋擾,或軟求,或硬問,不斷打聽謝遜的所在。空聞發誓賭咒,説道實在不知,但當時武當山玉虛宮中,各門各派數百對眼睛見到殷素素在空聞耳邉明言,如何是假?不論空聞如何解説,旁人總是不信,爲此而動武的,月有數起。外來的武林人物固是死傷不少,少林寺中高手却也損折了許多。推究起來,豈非均是武當種下的禍根?

空聞等{\upstsl{彆}}了兩年多的氣,難得今日張三丰自己送上門來,正好大大的折辱他一番,空智便道︰「張眞人自承是從少林寺中偸得武功,可惜此言並無旁人聽見,否則傳將出去,也好叫江湖上盡皆知聞。」張三丰道︰「紅花白藕,天下武學原是一家,千百年來互相截長補短,眞正本源早已不易分辨。但少林派領袖武林,此乃衆所公認之事,小道今日上山,正是心慕貴派武學,自知不及,要向衆位大師求教。」

空聞、空智等誤會了他言中之意,只道他「要向衆位大師求教」這句話,是向各人挑戰決鬥,不由得均各變色,心想這老道百歳的修爲,武功深不可測,舉世有誰是他的敵手,他孤身前來,自是有侍無恐,想來這兩年之中,又練成了什麼厲害無比的武功。一時間,三僧都不接口,最後空性却道︰「好老道,你要考較咱們來著,我空性可不懼你。少林寺中千百和尚一擁而上,你也未必能把少林寺給挑了。」他話説是「不懼」,其實已是大懼,先便打好了千百人一擁而上的主意。

張三丰忙道︰「各位大師不可誤會,小道所説求教,乃是眞的請求指點。只因小道修習先師所傳的『九陽眞經』,其中有不少疑難莫解,缺漏不全之處。少林衆高僧修爲精湛。若能不吝賜教,使張三丰得聞大道,感激良深。」説著站了起來,深深行了一禮。

張三丰這番言語,大出少林諸僧意料之外,他神功蓋代,開宗創派,修練已垂九十載,當世武林之中,聲望之隆,身份之高,無人能出其右,萬想不到今日竟會來向少林求教。空聞急忙還禮,説道︰「張眞人取笑了,我等後輩淺學,連『他山之石,可以攻玉』這八個字也説不上,如何能當『指點』二字?」

張三丰知道此事本來太奇,對方不易入信,於是源源本本的將無忌如何中了「玄冥神掌」,體内陰毒無法驅出的情形説了,又説他是張翠山身後所遺獨子,無論如何要保其一命,目前除了學全「九陽神功」之外,再無他途可循,因此願將本人所學到的「九陽眞經」,全部告知少林派,亦盼少林派能示知所學,雙方參悟補足。

空聞聽了,沉吟良久,説道︰「我少林派七十二項絶技,千百年來從無一名僧俗弟子能學到十二項以上。張眞人所學,自是冠絶古今,可是敝派只覺上代列位祖師傳下的武功太多,便是要學十分之一,也是大大不易。張眞人再以一種神功和本派交換,盛情可感,然於本派而言,却屬多餘。」他頓了一頓,又道︰「武當派武功,源出少林,今日若是雙方交換武學,日後江湖上不明眞相之人,便會説武當派固然祖述少林,但少林派却也從張眞人手上得到了好處。小僧忝爲少林掌門,此種流言却是擔代不起。」

張三丰心下暗暗嘆息,想道︰「你號稱四大神僧之一,却如此宥於門戸之見,胸襟未免太狹。」但其時有求於人,不便直斥其非,只得説道︰「三位乃當世神僧,慈悲爲懷,這小孩児命在旦夕,還望體念佛祖救世救人之心,俯允所請,小道實感高義。」空智冷冷的道︰「雖説出家人慈悲爲本,但張翠山張五俠夫婦當年手刃多少個少林弟子?他二人自刎相謝,咱們也就不再追究此事,倘若追究起來,一命還一命,這小孩子也是該當抵命纔是。」

無忌站在張三丰身旁,聽他忍氣吞聲,甘受少林神僧的搶白,早已怒火填膺,這時聽空智説到父母之事,那裡忍耐得住?昂然道︰「太師祖,這些和尚逼死了我爹爹媽媽,我寧可立時便死,也不要求他們救命。咱們走吧!」張三丰斥道︰「在衆位高僧之前,小孩子不得胡説八道。你父母之死,和衆位高僧何干?」無忌氣鼓鼓的不敢再説,但他生性高傲倔強,心中已打定了主意︰「太師祖便是説動了他們,以九陽神功教我,我也決計不學。我決不向逼死我父母的仇人,求憐乞命。」

只聽張三丰説得唇焦舌燥,空聞等三人總是婉言辭謝。正説之間,忽聽得馬蹄聲響,五乘馬奔上山來,當先一騎馬上的乘客身材魁梧之極,威風凜凜,宛如一座鐵塔相似。那大漢將到立雪亭,勒馬一看,説道︰「好極了!」這「好極了」三字,當眞是聲若雷震,人人都吃了一驚。那人正向空聞等打量幾眼,説道︰「巫山幫梅石堅,前來拜見少林方丈,相煩通報。」這兩句話他是隨口而出,但仍是震得每個人耳中{\upstsl{嗡}}{\upstsl{嗡}}作響,看來他是天生的大喉嚨,再加上内力充沛,説話之聲響亮無比。

無忌聽到巫山幫梅石堅六個字,想起兩年多以前,巫山幫的賀老三奉了梅幫主之命,將自己套在蛇袋之中,却被自己打得重傷,原來那梅幫主竟是如此威猛的人物,那日張三丰百歳壽誕,他却没上山祝壽。看到他這等聲勢,無忌不由得有些畏懼,縮在張三丰身後,生怕被那梅石堅認了出來。

空聞眉頭一皺,心想︰「又是來打聽謝遜下落的惹厭人物,那張翠山夫婦實是害人不淺。」空智便道︰「梅幫主求見敝寺方丈,不知爲了何事?」梅石堅滾下馬鞍,抱拳道︰「在下要向空聞大師打聽一個人的所在。」

空智説道︰「出家人但知誦經禮佛,不問外事,梅幫主來少林寺打聽旁人下落,可説是問道於盲了。」梅石堅道︰「請問這位大師法名?」空智道︰「姓名爲身外之物,張三李四,都是一般。」梅石堅濃眉上豎,厲聲説道︰「大師連法名也不肯見告,那麼是要打聽金毛獅王謝遜的所在,是也不是?」梅石堅道︰「不錯。在下的長子爲謝遜所殺,要找他問他一問。大師若肯見告,巫山幫上下,盡感大德。」空智説︰「梅幫主今日上山,也算有緣,若是早到一日,固然無法知曉,遲到一日,也是打聽不著。」梅石堅聽他這麼説,喜動顏色,連稱︰「多謝指點。」

空智緩緩道︰「當今之世,只有一人知道金毛獅王謝遜的下落,那便是這一位小兄弟,他是武當派張翠山張五俠的公子。」説著伸手向無忌一指。

無忌本來躱在張三丰的身後,但事到臨頭,又聽到空智提起他父親的名字,心想我豈能畏懼於他,弱了「張五俠」的威名?當即站了出來,説道︰「梅幫主,你好不要臉!」

他這七個字一説,衆人無不爲之一震,料不到如此面黃肌瘦的一個小児,一開口便是一鳴驚人。梅石堅大聲道︰「小小孩童,破口傷人,你不想活了?」無忌聽了他這幾句震耳欲聾的話,心中忍不住害怕,但強提精神,説道︰「兩年多以前,你叫一個叫做賀老三的人,假扮丐幫弟子,想將我擒去,此事可是有的?你明明是巫山幫的,爲什麼要冒充丐幫的名頭,這不是不要臉麼?」梅石堅滿臉通紅,大喝一聲,一掌便往無忌臉上一掌,但也非將無忌打得半邉臉頰高腫不可。

無忌待要要避,但覺對方一掌之力早已將自己全身罩住,氣息閉塞,只得隨手舉掌一格,突然背心上一股柔溫暖的力道傳了過來,雙掌相交,拍的一聲輕響,梅石堅身不由主的登登登接連退出了三步。退到第三步時,已在立雪亭的台階之上,他一步踏空,身形一晃,急使千斤墜之力,方始站穩身子。這一下情勢大是狼狽,本已通紅的臉孔,更是脹得猶如豬肝一般。他怒目瞪著無忌,心下好生奇怪︰「賀老三説被他一掌擊傷,我初時還不甚信,原來這小鬼果眞甚是邪門。可是他十一二歳年紀,便算在娘胎裡就開始練功,也不能有這等渾厚深沉的掌力?」

但空聞、空智等少林高僧却心中都是明明白白,知道乃是張三丰站在無忌背後,以「隔體傳功」之法,接了梅石堅的一掌。無忌這手臂只不過猶似一根木棒短杖,張三丰用來向梅石堅的手掌輕輕一推。那「隔體傳功」之法雖不甚難,可是要如這等絲毫不露痕跡,瀟灑自如的退敵,少林三大神僧均是自愧不如。

梅石堅出了這個醜,心中好生不甘,暗想︰「我是生怕傷了你這小鬼,以致只使一成力氣,那料到你竟全力相擊?在少林寺之前丟這個大人,以後巫山幫如何再能在江湖上立足?就算一掌將你擊斃,從此不能再知謝遜那惡賊的下落,也是無可奈何的了。」於是冷笑一聲,喝道︰「張小鬼,再接我一掌!」一口氣從丹田中運將上來,勁貫右臂,呼的一聲,一掌直擊無忌的前胸。他掌力未到,手掌去勢時所挾疾風,已將亭中諸人的袍角衣袖都激得飛揚起來,連空聞、空智這些武學高手,他掌風旁勢所及,也不由得胸口有一陣閉塞鬱悶之感,當即各自運氣抵禦。

張三丰近數年來閉関潛修,所創的「太極功」與任何武學中的内功均是截然相反,講究以柔克剛、以靜制動、以簡禦繁、以逸待勞、以小敵大、以弱勝強,其中「借力打力」四字,尤爲精義之所在。他眼見梅石堅這一掌打向無忌,掌力沉猛之極,不禁心下著惱︰「無忌小小孩童,你竟下如此重手打他,若非我在其側,豈不是給你一掌打得腦漿迸裂?」當下左掌在無忌背心上一按,一股修爲將近百年的渾厚内力,傳進了他體内。

無忌見梅石堅掌勢來得厲害,右掌上托,左掌從右臂之下穿出,使一招降龍十八掌的見龍在田。雙掌一交,兩股大力相互激盪,梅石堅啊的一聲大叫,身子向後飛起數丈,撞塌了立雪亭的一角。各人眼前塵沙飛揚,但見得磚石泥灰紛紛墜下,那梅石堅却已跌在亭外一株四五丈高的大松樹頂上,啊啊啊的大叫。張三丰的勁力雖大,却是柔和平正,竟没傷到梅石堅的分毫。但他輕功根底甚差,身居高樹之巓,一躍下來便要跌得筋斷骨折,只是雙手牢牢抓住樹幹,一動也不敢動。

衆人看得又是驚奇,又是好笑。梅石堅所帶來的巫山幫幫衆中,有倆個輕功佳妙之人,便欲攀援上樹,相救幫主。

張三丰在無忌耳邉輕輕説了幾句話,無忌點了點頭,從地下拾起一粒石子,扣在中指和拇指之間,向著大樹彈去。這小小一粒石子飛去時破空之聲甚響,擊在梅石堅處身所在的枝椏之上,但聽得喀喇别一聲響亮,那枝幹帶著梅石堅一齊摔了下來。這一著又是大出衆人意料之外,那想到他手指上彈出一粒石子,力道之強,竟足以擊斷一根粗大的樹枝。

無忌搶上幾步,伸出左手在梅石堅的背上輕輕一拍。梅石堅這一摔下來,心想定是非受重傷不可,不料無忌這麼一拍,雙足落地,免得出醜,但無忌這一拍擊在他的背心,登覺四肢百骸,都是暖融融地説不出的受用,可是半點力道也不出來,只有直挺挺的在地下拍的摔了一交,這纔爬起。

他那知這些對掌擲身、彈石斷樹、托背消力的功夫,全是張三丰借著無忌之手而行,只覺這小孩的武功深不測,自己生平從所未見,他對自己是手下留情,若不快走,不知要出多大醜,當下抱拳道︰「少年英雄,佩服佩服。」連「三年後再見」那些找場面的話也不説,翻身上了馬背,帶領從人,匆匆下山而去。

空聞、空智等都是大爲駭異,「武林中傳言這邋遢道人神功無敵,今日一見,他眞實的本領只有更在傳聞之上。」空聞本來不願跟他交換内功,但見他顯了這等身手,心想︰「我便是再練五十歳,也決不能練到他這般的境地,可見他所學確是有獨到之處。他功夫比我高得多,跟他交換並不吃虧。」於是説道︰「張眞人這『隔體傳功』的功夫,可也是得自『九陽眞經』麼?」張三丰道︰「這套功夫係小道所自創太極功,有一套拳術,叫作『太極拳十三式』,却和達摩老祖所傳的『九陽眞經』無関。大師若能救得我這徒孫之命,小道不敢自祕,願將太極拳十三式和對『九陽眞經』的膚淺心得,各和位高僧一同研討。」

空聞向空智望去,空智緩緩點了點頭。空聞便道︰「既是如此,咱們可將『九陽眞經』中的内功修練祕訣,傳與張公子。但只許張公子一人修習治病,不得轉授旁人,將來更不得持此而與少林弟子對敵。這兩節要請張公子發下重誓。」張三丰大喜,道︰「這兩節都可允得。無忌,你便發一個誓吧!」那知無忌搖頭道︰「我不發誓,我也不再學他們的功夫。」

張三丰一怔,心知他於父母之慘死,心中一直耿耿,雖然自己於道上曾多方開導,但這孩子性子極是倔強,寧可性命不在,却不肯向仇人求救,於是將他拉出亭外,遠離少林衆僧,低聲道︰「孩子,我帶你來時,你已答應向少林派學九陽眞經,怎地這時又反口了?」無忌道︰「他們要我發誓,將來不得用九陽神功向少林弟子動手,那麼殺父殺母之仇,如何報法?」張三丰道︰「你若是此刻學不全九陽神功,一年之内,性命不保,又如何報那殺父之仇?你只須養好身子,天下厲害的武功甚多,只須學得精湛,那一種不足以制服仇人?又何必非用少林九陽神功不可?」無忌一想甚是,便道︰「好,我聽太師父的吩咐。」當下兩人回到立雪亭中,無忌雙膝跪地,朗聲道︰「弟子張無忌,今蒙少林派高僧授以九陽神功,療傷治病,日後決不將少林九陽神功轉授他人,亦決不以此功對付少林弟子,如違此誓,教我自刎身亡,和爹爹媽媽一樣。」原來當年他父母命他拜謝遜爲義父,名爲謝無忌,準擬生下次子,方命其姓張,但張翠山夫婦一死,張門斷了香煙,是以兪蓮舟、殷利亨等要他復姓歸宗。

無忌立誓之後,站起身來,心中暗道︰「難道我將來不用九陽神功,便殺不得你們這些和尚?」空聞大師合什道︰「善哉,善哉!小施主言重了。」向張三丰道︰「咱們便帶小施主進寺,傳授神功。但張眞人的太極十三式\dash{}」張三丰道︰「相煩借一副紙墨筆硯,小道便在立雪亭中,將太極十三式及武當九陽功的精義要旨,盡數書冩出來。」空聞道︰「如此有勞了。」説著行了一禮,帶回衆僧及無忌回進寺中。

無忌心中暗自不忿︰「武當九陽功未必便輸於少林九陽功,太師父和你們公平交換,本來大家都不吃虧,可是你們硬要他添上個太極十三式。再者,你們學了武當九陽功之後,可以互相傳授,可以用來對付武當子弟。這麼一來,武當派只好永遠向少林派低頭了。因我一人之故,使得宋師伯、兪師伯他們一生抬不起頭來,這便如何是好?」他雖然聰明,究竟年紀太小,一時也想不出善法,既是太師父之命,只得聽從。

空聞將無忌帶入一間小小禪房,説道︰「小施主路上辛苦,且歇息一會,老衲便即派人傳你功夫。」説著袍袖展動,在他胸前背後拂了幾拂,已拂中了他的睡穴。

空聞大師是少林三大神僧之一,「見聞智性」,名列第二,他的點穴、打穴、拂穴之技,當世罕有其匹。别説無忌是個小小孩童,便是一等一的高手,除非不讓他拂中,只要他衣角袍袖帶到了一點穴道,勁力立時便透了進去,當死即死,當昏則昏,眞是厲害無比。豈知無忌跟著謝遜,學的内功甚是怪異,身上穴道常自移位,那日他被假扮元兵的高手所擒,帶到武當山上,明明啞穴被點,他還是叫了幾聲「爹爹」出口,便是這個緣故。此時他睡穴一被拂中,登時昏睡了過去,本來要睡足四個時辰纔醒,但只過了一頓時分,他身上血行流動,穴道易位,便醒了過來。神智甫復,便聽得空智的聲音説道︰「那張邋遢是一代宗師,既是答應交換,所書的神功祕訣當不會有假,便算他冩得不十分明白,咱們總也能參悟出來。」無忌心想︰「他們何以要點我睡穴?莫非要商量什麼不可告人的陰謀麼?」當下閉住眼睛,假裝睡熟,却在凝神傾聽。

其實少林和武當之間雖有嫌隙,空聞、空智、空性三人究是一代高僧,如何能對張三丰使什麼陰謀詭計,墮了少林寺千百年來領袖武林正大門派的清名令譽?

\chapter{蝶谷醫仙}

但無忌認定逼死自己父母的兇手之中,這些少林寺的和尚也在其内,因此一心只道他們盡是邪惡奸猾之輩。

只聽空聞説道︰「他冩給咱們的太極十三式和武當九陽功,自不會假,但少林九陽功咱們却未練過,難道爲了外人,反而去碰圓眞的釘子?」無忌聽了,心中一動︰「原來他們都不會少林九陽功,别要教我些不打緊的假功夫,却騙了太師父的眞功夫。」只聽空智説︰「師兄,你是掌門方丈,傳下法旨,諒那圓眞焉敢不遵?這是光大本門武學的盛舉,又不是爲了一己之私。」空聞嘆了口氣道︰「空見師兄若是在世,咱們便不用爲難了。」沉吟半晌,道︰「三師弟,便請你持我錫杖去諭示圓眞,命他將少林九陽功傳於這姓張的少年。」空智道︰「囑方丈師兄法旨。」

原來當年覺遠大師荒郊傳經,張三丰演之爲武當九陽功、郭襄演之爲峨嵋九陽功、無色禪師演之爲少林九陽功。那九陽功博大深微,每一派的傳人均只寥寥數人,少林派因有七十二項神功絶技,專練九陽功的人更少。自無色傳至空見,都是一線單傳,因少林僧俗弟子均認覺遠是本派棄徒,自他傳下來的功夫,縱然精妙,大家都不屑鑽研,反正本派絶技甚多,便是兩世爲人,也學不了這許多,何必去走這條説來不彀響亮的路子?只是每一代均有一名弟子修習,庶免失傳,便算已足。

此時少林寺中,祇有空見的関門弟子圓眞,會此少林九陽功。但這人生性極是怪僻,終年閉関不出,除了對三大神僧稍有禮貌之外,合寺僧侶,他誰也不加理睬。到了每年達摩老祖一葦渡江之日,寺中例行考較武功,由三大神僧評定高下,指明優劣,但那圓眞每次總是生病,臥床不起。誰也不知他是眞病還是假病,也不知道他功夫到底如何。因此空聞等想到要他去傳授無忌功夫,都不由得皺眉。

過了一會,空智回來覆命,説道︰「這圓眞果然忒也古怪。他説他皈依我佛之後,發願不見外人,既是方丈頒下法旨,他只允隔帳傳授。」空聞道︰「那也由得他。師弟,待張三丰冩完經文,你去取來,看過無誤,便帶這少年去命圓眞隔帳傳授。再吩咐香積厨送一席上等素齋去立雪亭,款待張三丰,他究是一派之尊,咱們禮不可失。」三人又談論了些别事,便出房去了。

無忌睡在禪床之上,等了良久良久,纔聽到有人進房,却是一個小沙彌送了飯菜來。無忌飽餐一頓,那小沙彌道︰「小施主,請隨我來。」無忌道︰「到那裡去?」小沙彌道︰「方丈命我帶你去見一個人。」無忌道︰「是什麼人?」小沙彌道︰「方丈叮囑,叫我不可多口。」無忌哼了一聲,心想你們故作神祕,其實我什麼都知道了,還不是見那個叫作什麼圓眞的和尚。

當下跟著那小沙彌穿房過戸,走過一個院子又是一個院子,無忌心想,這少林寺比咱們武當玉虛宮可要大得多了。一直繞過十幾座偏殿,到了一個古柏的森森的小院之中。小沙彌站在門口的竹簾之外,朗聲稟道︰「張小施主到!」門内一個低沉的聲音説道︰「進來吧!」無忌推門進去,那小沙彌順手帶上門自去。

無忌左右一看,只見室内空空洞洞,除了地下一個蒲團之外,四壁蕭然,什麼東西也没有。無忌本想,他既説「隔帳傳功」,那麼室中定有一個布帳,那知室中固然無人,連布帳也没一塊,室中再無别處門戸,却不知適纔的人聲從何而來。正奇怪間,只聽一個低沉的聲音冷冷的道︰「你坐下了!聽我述説少林九陽功的祕奥。我只説一遍,能記著多少,全憑你的造化。本寺方丈命我傳功,我傳便傳了,你能否領會,我可管不著。」

無忌從聲音來處凝神瞧去,原來那話聲是隔著一堵牆壁傳來,那圓眞和尚身在鄰室。本來隔牆透過聲音,原是毫不足奇,人人均能辦到,但圓眞的説話聲音却是十分的清晰明白,和相對而談絶無分别。無忌忍不住暗自驚異︰「這人果然是内力驚人。」只聽他緩緩説道︰「立身期正直,環拱手當胸。氣定神皆斂,心證貌亦恭。這是第一式,叫作『韋駝獻杵』,你記住了。」他稍停片刻,又道︰「足趾柱地,兩手平開,心平氣靜,目瞪口呆。這是第二式,叫作『橫擔降魔杵』,你記住了。」第三式「掌托天門」第四式「摘星換斗」、第五式「倒曳九牛尾」,圓眞一一説了,接著又道︰「挺身兼努目,推窗望月來。排山還海後,隨息七徘徊。這是第六式,叫作『出爪亮翅』,你記任了。」

他越説越快,一直説到第十二式「掉尾搖頭」,那歌訣是「膝直膀伸,推手及地。瞪目搖頭,凝神一志。挺身頓足,舒肱長臂,左右七次,神功已畢。九陽易筋,天下無敵。」那「天下無敵」四字剛説完,突然提聲喝道︰「誰在外面偸聽,進屋來!」

砰的一響,室門撞開,跌進一個人來,正是適纔帶領無忌前來的小沙彌。他一交摔倒,蜷成一團,雙目緊閉,臉上神情極是痛苦。無忌吃了一驚,忙問︰「你怎麼了?」伸手要去相扶時,隔牆那聲音冷冷的道︰「你還是顧自己的好,這當口專心凝志,記憶口訣要訣尚自不及,怎能再分心去理會旁人?」無忌道︰「這十二招我都記住了。」圓眞似乎大吃一驚,眞不相信他記心如此了得,説道︰「你背給我聽聽。」無忌當下便從第一式「韋駝獻杵」背起,一直背到第十二式「掉尾搖頭」,果然是一字不錯,半句不漏。

圓眞半晌做聲不得,他奉方丈之命傳授九陽神功,實則心中大是不願,但方丈只命他傳授,却没説「傳會」,因此他一口氣的快將下來,料想這小小孩童能記得一句兩句,已是不易了,那知他過耳不忘,盡數記在心裡,當眞是天下罕見的奇才。

無忌見那小沙彌躺在地下手足抽動,甚是不忍,問道︰「禪師,這位小師父怎麼啦?」圓眞冷冷的道︰「他在門外偸聽我傳你功夫,我用『金剛禪唱』,叫他吃了些苦頭,稍待片刻,便會好的。」他微一沉吟,説道︰「我不知方丈何以命我傳你九陽神功,你叫什麼名字我固然不知,我法名如何你也不用問。我不知你以往學過什麼功夫,但你如此聰明,將來前途不可限量,我索性成全你一番,助你打通周身奇經八脈。你修練這九陽神功時進境便快上數倍。」無忌還没回答,忽見牆壁中伸了兩隻手掌過來。無忌大吃一驚,跳起身來,叫道︰「這\dash{}這\dash{}」只見這兩隻手掌穿壁而過,牆上留下了兩個掌印的空洞,十指指印宛然,這磚頭砌的牆壁在他掌力之下,竟似豆腐一般柔軟,雙掌無阻無礙,説過便過,石灰磚粉,簌簌跌落。只聽圓眞説道︰「你手掌和我雙掌相接。記住了,我不知你姓甚名誰,不知你是何門何派的弟子,今日一會,緣盡於此。」

無忌聽他言語雖然怪僻峭冷,但對自己却著實不差,先前心中對他所存敵意登時消減,説道︰「多謝禪師。」伸出雙手,貼在他的掌上。圓眞道︰「你四肢百骸,盡皆放鬆,心中不可有絲毫雜念。」無忌道︰「是。」

只覺對方掌心之中,有一條暖烘烘的熱氣,透過自己掌心,分從雙臂遊上,這熱線雖細,却是感覺得清清楚楚,緩緩的遊走全身經脈,逢到関竅之處,若是數衝不過,對方掌心中傳來的熱力迅速即加強,幾度強衝,便即破関而過,入脈盡通之後,那熱線越走越快,無忌但覺天旋地轉,幾欲摔倒。

但圓眞的雙掌之上,有一股極爲強韌的吸力將無忌的手掌牢牢黏住,使他不致跌倒。無忌只覺周身火滾,恨不得將全身衣服扯去,再在冰火島上冰冷澈骨的海水中浸上一浸,方才痛快。過了良久良久,纔覺得那條火線離開自己身子,從掌心回到對方手掌之中。

圓眞縮回手掌,冷冷的道︰「你去吧!」無忌從牆壁上的兩個掌印孔中一望,黑洞洞的瞧不見什麼,心想︰「這位禪師傳我神功,又助我打通奇經八脈,雖説是太師父以武當派的奇功跟他們少林交換,但我總得謝他一謝。」跪在蒲團之上,説道︰「小子叩謝禪師傳功通脈的恩德。」待要拜將下去,牆壁孔中突又伸進一隻手掌,向著自己一揮,無忌只覺一股疾風吹在自己身上,登時立足不定,不由自主的飄身出了室門,原來圓眞竟是不受他的叩謝。

無忌心道︰「這位禪師的脾氣確是甚爲古怪。」只聽圓眞的聲音在室中響道︰「你去稟告方丈説傳功已畢,小施主記性驚人,已盡數記住。」一聽那小沙彌道︰「是。」只見小沙彌退了出來,臉如死灰,神色不定。

無忌跟著他走出寺去,一路上遇到不少僧人,但見人人均是靠著牆壁,低首緩緩而行,寺中雖有千百名僧人,竟是不聞有絲毫喧嘩笑語之聲,寺中僧俗弟子個個習武,却無一人挺胸凸肚、昂然闊步。無忌經過他身旁之時,誰都是視若無睹,没人向他瞧上一眼。無忌暗暗佩服︰「少林寺爲天下武林首領,寺中戒律,果然是精嚴無比。」相較之下,武當派的玉虛觀中便隨便得多,你便是叫嚷奔走,也無人來管。這一來因道家注重任心率性,二來張三丰自己便是馬馬虎虎,不修邉幅之人,上行下效,各人喜歡如何便如何了。

兩人來到立雪亭下,只見張三丰已書冩了三十多張玉版紙,尚未冩完。無忌心中感激,泪盈於眶,叫了聲︰「太師父。」又道︰「寺中的禪師已將少林九陽功十二式傳於孩児。」張三丰甚喜,笑道︰「很好,很好。」又冩了一會,便也冩完了。站在一旁傳遞茶水的僧人進寺稟報,空聞、空智、空性三僧又來到亭中,這一次三僧身後,却跟著一個二十五六歳的青年,穿著一件藍布長衫,當是寺中的俗家弟子。

張三丰微覺奇怪,他知少林寺數百年來的規矩,俗家弟子若非藝成下山決不許走出寺門一步,俗人進少林寺山門固然不易,出寺更加艱難。這時掌門方丈帶著這個弟子走出寺門,不知是何用意,不由得向他多瞧了兩眼,只見這人身形瘦削,顴骨高聳,臂長腿短,一對眸子晶光燦然,顯得極是精明能幹。

空聞走到亭中,合什説道︰「張眞人辛苦了。」張三丰微微一笑,道︰「多謝方丈師兄慈悲,令這孩子得窺貴派神功祕奥,當可救得他一條小命。」説著將冩成的三十餘張玉版紙遞了過去,説道︰「太極十三式和武當九陽神功的精要,已書在内,還請三位師兄不吝指點。只是内容過於龐蕪冗,未臻自博返約之致,班門弄斧,可讓三位見笑了。」空聞接了過來,看也不看,隨手遞給了身後的青年。那青年却一頁頁的翻閲下去。張三丰道︰「天色不早,就此告辭。」空聞道︰「張眞人駕臨少林,未得盤桓數日,老衲心中甚是不安,只得奉敬三杯水酒,聊表寸心。」服侍茶水的僧侶斟酒上來,張三丰和空聞對飲了三杯。跟著空智和空性也各敬酒三杯,張三丰也都乾了。

他命無忌向三位高僧行禮告别,兩人正要轉身,空聞身後那青年忽道︰「師伯,張眞人所冩的武學,未出少林範圍,師父都教我學過的。」張三丰吃了一驚,心道︰「那有此事?」不由得臉色微變。

空聞也叱道︰「胡説!這是張眞人畢生心血之所寄,武當派鎭門之寶的太極十三式,你怎能學過?」那青年將一疊玉版遞給空聞,説道︰「師伯請看便知。」空聞隨手翻閲,跟著空智、空性。二僧也是隨手翻閲,每一頁瞧了幾個字便翻過不看。空智低聲道︰「師兄,果然便是我少林派的武功。」

張三丰又驚又怒,心想︰「這太極十三式是我三十餘年鑽研,去年方得大成,講究以弱勝強,後發制人,和少林武學截然相反,怎説是你少林派武功?便是我那武當九陽功,雖然源自達摩祖師的九陽眞經,但八十年來,我加了不少變化,没一點不是别出心裁,你少林派如何知道?」空智將一疊玉版遞給張三丰,淡淡的道︰「武當派武學源出少林,原來並没經過什麼變化。」張三丰心念一轉,已知其意︰「你少林派怕的是從我手中學到武當心法,江湖上傳出去不雅,所以硬説這些功夫早就知曉。」當下抬頭一笑,説道︰「張某一言既出,再無反悔,這些功夫,本甚粗淺,不足當方家一笑,三位既瞧不上眼,便隨手抛棄了吧。」却不去接空智遞過來的一疊紙箋。

空智道︰「聽張眞人的説話,言下似有不信之意。」轉頭向那青年説道︰「友諒,我傳你的太極十三式,以及九陽功的訣要,你背給張眞人聽聽,且瞧有什麼不同。」那青年道︰「是。」朗聲誦道︰「一舉動,週身要輕靈,尤須貫串。氣如鼓盪,神宜内斂,無使有缺陥處,無使有凹凸處,無使有斷續處。其根在脚,發於腿,主宰於腰,總須完整一氣,向前退後,乃能得機得勢\dash{}」一路背將下來,竟無一句一字錯漏,背完總論,接著便背十三式的訣要。無忌插口道︰「太師父,這人看了你所冩的經文,記在心中,便説是少林派原有的,好不識羞。」張三丰這時也早明其理,原來空智這個徒児記性驚人,過目成誦,空智命他將經文記在心中,却將原件當時還給張三丰,以示少林派没得武當派的好處。他哈哈一笑,説道︰「三大神僧敬我九杯白酒,閣下便將我兩套武學記在心中,如此聰明才智,張三丰自愧不如。請教閣下姓大名。」那青年道︰「不敢,晩輩姓陳,名友諒。」張三丰正色道︰「陳兄弟,以你才智,他日無事不可成,但盼不可誤入岐途纔好。老道贈你八個字︰『誠以待人,謙以律己。』」

陳友諒和他冷電般的目光一觸,不禁機伶伶的打個冷戰,心想︰「你上了我的當,便老羞成怒了。」冷冷的道︰「多謝張眞人指點,但晩輩是少林弟子,自有師伯、師父和師叔教誨。」張三丰笑道︰「不錯,算老道越俎代庖,多口的不是了。」見空智又將紙箋遞來,當即接過,一股内勁從紙箋上傳了過去,空智猛地一震,往後便倒,陳友諒站在他的身旁,忙伸手相扶。那空智這一倒勁力甚猛,陳友諒人雖聰明,武功却淺,給師父這麼一撞,身子急飛出亭,砰的一聲,摔跌在地。

空智究屬多年修爲,張三丰又不過是略顯神功,並非眞要他出醜露乖,這紙上傳勁,未盡全力,因此他在將倒未倒之際,脚下一使勁,身子已然站直。張三丰微笑道︰「這便是太極十三式的功夫,原來賢師徒雖然熟極流,却無暇修習。告辭了!」手一揚,滿亭中紙屑飛舞,有如大雪漫天而下,原來他潛運神功,將數十張玉版箋一齊捏成了極細極細的碎片。紙屑隨風四散之際,張三丰已擕了無忌之手,飄然離去。空聞、空智、空性相顧茫然,對張三丰所顯神功,實不禁又驚又佩,三人心中都有些懊悔︰「這功夫如此厲害,不知陳友諒是否眞能盡數記住,若有錯漏,那倒是弄巧成拙了。」

張三丰和無忌下得山來,當晩在客店之中便命無忌依著圓眞所傳的口訣,修習少林九陽功。張三丰不願見到無忌練功的姿式,蓋以他的武學修爲,不必聽無忌述説口訣,只須見到他如何打坐、如何呼吸、如何運氣,自能推想到少林九陽功的祕奥。因此在客店中要了兩間店旁,分室而居,無忌進境若何,他也不加詢問。張三丰信得過少林三大神僧定能信守諾言,這三位神僧雖於門戸之見不免隔隘,但究是武林中一代高人,言出如山,既是答應傳他神功,絶無欺詐誑騙之理。

一路行來,見無忌臉漸紅潤,張三丰心下也欣喜,暗想無忌已得武當和少林兩派九陽神功的眞傳,兩派神功相互補足,威力大增,當可化除體内所中玄冥神掌的陰毒無疑。這日行到漢水邉上,兩人坐了渡船過江,張三丰想起了少年時逃出少林寺,過漢水時風聲鶴唳,生怕寺中僧人追來,實是狼狽不堪,當時年紀已比無忌爲大,想不到日後竟開創武當一派和少林分庭抗禮,今日無忌却已兼學兩派武功,將來成就,説不定更在自己之上了。正自捋鬚微笑,無忌忽然叫道︰「太師父,我\dash{}我\dash{}」聲音顫抖,神色大變。張三丰吃了一驚,只見他臉上燒得炭火般紅,可是炙紅之中,却又透出隱隱青氣,忙問︰「怎麼了?」無忌道︰「我\dash{}我難過得緊\dash{}抵不住\dash{}抵不住了。」身子一晃,便要摔出船外。張三丰伸左手拉住他手腕,右手便抵在他背心「靈台穴」上,送過内力,助他抗禦寒毒。不料一股内力傳送過去,立時走通他周身奇經八脈,無忌大叫一聲,登時暈死過去。

張三丰這一驚眞是非同小可,手指連揚,閉住了他身上一十二大穴,心道︰「怎地他奇經八脈居然已經通了?他身中極厲害的寒毒,這奇經八脈如何通得?八脈一通,寒毒散入五臟六腑,那是再也不能化解了。」他以百歳高齡,修心養性已到達爐火純青之境,但這時也不禁方寸無主,心神大亂,額頭冷汗涔涔而下,暗想︰「難道這少林九陽功如此了當,修習數日,便能打通奇經八脈?世間絶無此理。利亨、聲谷隨我十餘年,尚未打通,少林九陽功數日的威力,豈能勝過我武當功十餘年的勤修苦練?」要知張三丰若以本身功力相助,替殷利亨、莫聲谷打通經脈自非難事,但外來的助力,總不若本身自運來得紮實可靠。他傳授弟子不求此等速成,要各人循序緩進,漸成大器。

這時船到中流,漢水中波浪滔滔,小小的渡船搖晃不已,他身上一十二處大穴已閉,寒毒暫停侵入臟腑,可是手足已然動彈不得。張三丰這時也顧不得再避嫌疑,問道︰「孩子,你學的少林九陽功是怎等模樣?何以體内奇經八脈竟已通了?」無忌道︰「是那個圓眞禪師給我通的,他説可以助我早日練成九陽神功。」張三丰急問︰「他如何助你?」當下無忌將怎生聽到空聞、空智等商量,圓眞禪師如何隔牆傳功,他如何替自己打通奇經八脈等情一一説了。張三丰半晌做聲不得,隔了良久,纔道︰「若要打通奇經八脈,難道我便不會?這圓眞到底是好心還是歹意?」無忌道︰「他跟我説了幾遍︰『我不知你姓甚名誰,不知是何門派,你也不用知道我的名字。』」

張三丰喃喃的道︰「圓眞?圓眞?從没聽見過少林派中有這樣一個高手。他不跟你見面,不讓你知道名字,他也不知你的門派姓名。如此看來,他確是不知你和我的淵源。那麼他自耗數年功力,助你打通奇經八脈,倒確是一番好心了。」

張三丰又問少林九陽功的口訣,無忌自第一式「韋駝獻杵」背起,背至第三式「掌托天門」,張三丰是當世武學第一高人,一聽之下,便知這些簡單的歌訣之中藏著無窮祕奥,那圓眞傳與他的,自是少林九陽功無疑,即道︰「不用背了。孩子,我是査問那傳功之人的眞偽,不得不問。自今而後,這一十二式神功可誰也不得傳授,須知你曾發下重誓,不可有違。」無忌應道︰「是!」但見太師父聲音顫抖,泪光瑩瑩,他是個絶頂聰明之人,如何不知自知是命在旦夕,便未曾發過誓言,也不能將此神功傳人了。

他忽地心念一動,道︰「太師父,我能挨得到回山不死麼?」張三丰忍泪道︰「你别出此言,太師父無論如何,要想法救你。」無忌道︰「我盼能再見兪三伯一面,那便好了。」張三丰道︰「爲什麼?」無忌道︰「孩児反正是活不成了。我要將這一十二式神功説給兪三伯聽,盼他融會武當少林兩神功,治好手足殘疾,孩児應了誓言,和爹爹一般自刎身亡,也好稍贖媽媽的錯失。」

張三丰吃了一驚,萬想不到他小小年紀,竟是如此工於心計,隨口道︰「你那裡話來?」無忌道︰「那日我聽得明白,媽媽用毒針傷了兪三伯,害得他全身殘廢,爹爹過意不去,這纔自殺\dash{}」這番話觸到張三丰的心事,點點眼泪,直酒到道袍之上,哽咽著喝道︰「你\dash{}你不可再胡思亂想。」定了定神,正色道︰「大丈夫行事該當光明磊落,你親口答應過三位神僧,決計不傳旁人,那便須得信守到底。你就算要死,也不能故弄狡獪。」這幾句話説得正氣凜然。無忌呆了一呆,點頭受教。他自幼在父母及義父三人薰陶下長大,殷素素和謝遜都算不得是正人君子,那是不必説了,便是張翠山,也是個風流倜儻的人物,在那荒島之上,也不跟児子講論什麼仁義道德,因此無忌是聰明機智有餘,至於武林中生死一諾的朗朗風骨,却是近來日受張三丰的親炙,方始領會。張三丰又想︰「這孩子明知自己性命不保,居然並不怕死,却想到要去療治岱岩的殘疾,這番心地,也確是我輩俠義中人的本色。」正想誇獎他幾句,忽聽得江上一個洪亮的聲音遠遠傳了過來︰「快些停船,把孩子乖乖交出,佛爺饒了你的性命,否則莫怪我無情。」這聲音從波浪之聲中傳來,入耳清晰,顯見呼叫之人内力甚是充沛。

張三丰心下冷笑,暗道︰「誰敢如此大膽,要我留下孩子?」抬頭一看小船如飛的划來。他凝目一瞧,見前面一艘小船的船梢上坐一個虯髯大漢,將自己身子護著一男一女兩個孩子,雙手操槳,用力划行;後面一艘船船身較大,舟中站著四名番僧,另有七八名蒙古武官,那些武官拿起船板,幫同划水,那虯髯大漢膂力奇大,雙槳一扳,小船便急衝丈餘,但後面船上究竟人多,而且划船之人顯然武功也自不弱,兩船相距越來越近。過不多時,那些武官和番僧便彎弓搭箭,向那大漢射去。但聽得羽箭嗚嗚,破空之聲極響,足見弩力勁急。張三丰心道︰「原來他們是要那大漢留下孩子。」他生平最恨蒙古官兵殘殺漢人,便想出手相救,但這時無忌命危,正是自顧不暇之際,而兩舟和他所乘渡船相隔尚遠,要加援手也是鞭長莫及。只見那大漢左手划船,右手舉起木槳,將來箭一一擋開擊落,手法迅捷無比。張三丰暗喝一聲采,心道︰「這人武功不凡,英雄落難,我怎能坐視不救?」向搖渡的船梢公喝道︰「船家,迎上去。」

那梢公見羽箭亂飛,早已嚇得手酸足軟,拚命將船划開尚嫌不及,怎敢反而迎將過去?顫聲道︰「老\dash{}老道爺,你\dash{}你説笑話了。」

張三丰見情勢緊迫,奪過梢公手中的櫓來,在水中劃了半個旋児,渡船便橫過船頭,向著來船迎去。猛聽得「啊」的一聲慘呼,男小孩背心上已中了一箭。那虯髯大漢一個失驚,俯身去看他時,自己肩頭和背上連中兩箭,手中木槳拿捏不定,掉入江心,坐船登時不動。後面的追舟瞬即追上,七八名蒙古武官和番僧跳上船去。那虯髯大漢兀自不屈,拳打足踢,奮力抵禦。張三丰縱聲叫道︰「英雄休驚,老道來救你了!」提起船上兩塊木板,飛擲出去,跟著身子縱起,左脚在第一塊木板上一點,右脚跨出,再在另一塊木板上一點,這麼兩個借勢,大袖飄飄,便如一頭大鳥般落下船來,早有兩名武官彎弓搭箭,向他射來。張三丰袍袖一揮,兩枝硬弩跌入了江心,雙足一踏上船板,左掌揮出,兩名番僧飛出丈許,撲通、撲通兩聲,一齊跌入江中。衆武官見他猶似飛將軍從天而降,一出手便將兩名武功甚強的番僧震飛,身手之厲害,實已到了驚世駭俗的地步,無不膽怯。領頭的武官喝道︰「兀那老道,你來幹什麼?」

張三丰罵道︰「狗韃子!又來行兇作惡,殘害良民,快快給我滾吧!」那武官道︰「你知道這三人是誰?那是魔教反賊的餘孼,皇上下旨普天下捉拿的欽犯!」張三丰聽到「魔教反賊」四字,吃了一驚,心道︰「難道這是陳州周子旺的部屬麼?」轉頭問那虯髯大漢道︰「他這話可眞?」那大漢全身鮮血淋漓,手中抱著男孩,虎目含泪,説道︰「小主公\dash{}小主人給他們射死了。」這一句話,等於是承認了自己的身份。

張三丰心下更驚,道︰「這位是周子旺的郎君麼?」那大漢道︰「不錯。我有負囑咐,這條性命也不要了。」輕輕放下那男孩的屍身,向那武官撲了上去。可是他負傷太重,肩背上的兩枝長箭尚未拔下,身形剛縱起,「嘿」的一聲,便摔跌在船艙板上。那小女孩手臂上也中了一箭,只是哭叫︰「哥哥,哥哥!」

張三丰心想︰「早知是魔教周子旺的子女,這件閒事不管也罷。可是既已伸手,總不能半途抽身。」當下向那武官道︰「這男孩已然身亡,餘下兩人身中毒箭,也已轉眼便死,你們已然立功,那便走吧!」那武官道︰「不成!非將三人的首級斬下不可。」張三丰道︰「那又何必趕人太絶?」那武官道︰「老道是誰?憑什麼來橫加插手?」張三丰微微一笑,道︰「得饒人處且饒人,天下事天下人都管得。」那武官使個眼色,説道︰「道長道號如何?在何處道觀出家?」只見兩名蒙古軍官突然手舉長刀,向張三丰肩頭劈了下來。這兩刀來勢好不迅疾,小舟之中相距又近,實是無處閃避。不料張三丰身子一側,本來面向船首,輕輕一轉之下,已是面向左舷。這一轉看似尋常之極,但分寸拿捏之準,却是妙到巓毫,這兩刀登時砍空。張三丰雙掌起處,已托在兩人的背心,喝道︰「去吧!」掌力一吐,兩名武官身子飛起,砰砰兩響,剛好摔在原本所乘的舟中。

他已數十年未和人動手過招,此時牛刀小試,大是揮瀟如意。這些蒙古武官和番僧雖然均是皇帝駕下的高手,但在張三丰絶世神功之下,實無半點抗拒餘地。那爲首的武官張大了口,結結巴巴的道︰「你\dash{}你莫非\dash{}是\dash{}」張三丰袍袖揮動,喝道︰「老道生平,專殺韃子!」舟中的衆武官番僧但覺疾風撲面,人人氣息閉塞,半晌不能呼吸。張三丰袍袖一停,衆人面色慘白,齊聲驚呼,爭先恐後的躍回舟中,救起落水的番僧,急划而去。

張三丰見那大漢和女孩所中的弩箭,箭頭有毒,當即取出解毒丹藥,餵入兩人口中。

\chapter{林中激戰}

張三丰將小舟划到渡船之旁,待要扶那虯髯大漢過船,豈知那大漢甚是硬朗,一手抱著男孩的屍身,一手抱著女孩,輕輕一縱,便過了渡船。張三丰暗暗點頭︰「這人身受重傷,仍是如此忠於幼主,確是個鐵錚錚的好漢子,我這番出手雖然不免冒失,但這樣的好漢子却也該救。」當下便也回到渡船,替那大漢和小女孩取下毒箭,敷上拔毒生肌之藥,待得一切料理定當,渡船已過了漢水。

張三丰心想︰「眼下無忌週身穴道閉塞,不能行走,若是到老河口投店,這漢子和那女孩都是欽犯,我一人照顧三人,只怕難以周全。」便一沉吟,取出三兩銀子交給梢公,説道︰「梢公大哥,煩你順水東下,過了仙人渡,送咱們到太平店投宿。」那梢公見他將蒙古衆武官打得落花流水,心中早是萬分敬畏,何況又給了這麼多銀子,當下連聲答應,搖著船沿江東去。

那大漢在艙板上跪下磕頭,説道︰「老道爺相救小主的大恩大德,常遇春粉身難報。」張三丰忙伸手扶起,道︰「常英雄不須行此大禮。」一碰到手掌,但覺觸手冰冷,心下微微一驚,問道︰「常英雄可還受了内傷麼?」常遇春道︰「小人從信陽護送兩位小主南下,途中和韃子派來追捕的鷹爪接戰四次,胸口和背心被一個番僧打了兩掌。」張三丰一搭他的脈博,但覺跳動極是微弱,再解開他衣服一看傷處,更是駭然,但見他中掌處腫起寸許,受傷著實不輕,若是換作旁人,早便支持不住,他千里奔波,力拒強敵,當眞是英雄了得。當下命他不可説話,布艙中安臥靜養。

這晩二更時分,纔到太平店,張三丰到鎭上藥店裡抓了藥,煮了給各人分别服下,那女孩約莫十歳左右,十足是個絶色的美人胎子,坐在哥哥的屍身之旁,一動也不動。張三丰見她楚楚可憐,問道︰「姑娘,你叫什麼名字?」那女孩站起身來説道︰「我叫周芷若。不敢請教道長法號?」張三丰見她小小年紀,雖在喪亂之中,仍是態度雍容,行止有禮,不禁憐愛之心更甚,微笑道︰「老道是張三丰。」

常遇春「啊」的一聲,翻身坐起,大聲道︰「老道長原來是武當山張眞人,難怪神功蓋世。常遇春今日有幸,得遇仙長。」張三丰微笑道︰「老道不過多活了幾歳,什麼仙不仙的。常英雄快請臥倒,不可裂了箭創。」他見常遇春慷慨豪爽,英風颯颯,周芷若明慧端麗,溫順文雅,心中對兩人都很喜歡,但想到他二人是魔教中人,倘若深談,説不定日後貽患無窮,便淡淡的道︰「兩位受傷不輕,不宜多談。」

要知張三丰生性豁達,於邪正兩途,原無多大偏見,當日曾對張翠山言道︰「正邪兩字,原來難分。正派中弟子若是心術不正,便是邪徒;邪派中人倘若一心向善,那便是正人君子。」又説白眉教教主殷天正雖然性子偏激,行事乖僻,却是個光明磊落之人,很可交交這個朋友。可是自從張翠山自刎身亡,他心傷愛徒之死,對白眉教不由得深恨極惡,心想三弟子愈岱岩終身殘廢,五弟子張翠山身死名裂,皆是由白眉教而起,雖然勉強抑下了向殷天正問罪復仇之念,但不論他胸襟如何博大,於這「邪魔」二字,却是恨惡殊深。

那周子旺正是魔教中彌勒宗的大弟子。數年前在江西袁州起事。自立爲帝,國號稱「周」,但旋即爲元軍撲滅,周子旺被擒斬首。彌勒宗和白眉教雖非一派,但相互間淵源甚深,周子旺起事之時,殷天正曾在浙西爲之聲援。張三丰今日相救常遇春和周芷若,只是激於一時俠義之心,兼之事先未明二人身份,實在是大違本願,想到兩個情若父子的弟子一死一傷,無忌又是毒深難治,不禁長嘆了一聲。

這時那梢公已煮好飯菜,開在艙中小几之上,雞、肉、魚、蔬,一共煮了四大碗。張三丰要常遇春和周芷若先吃,自己却給無忌餵食。常遇春問起原由,張三丰説他寒毒侵入臟腑,是以點了他各穴道,暫保性命。無忌心中難過,竟是食不下嚥。張三丰再餵時,無忌搖搖頭,不肯再吃了。周芷若從張三丰手中接過碗筷,道︰「道長,你先吃飯吧,我來餵這位大哥。」無忌道︰「我飽啦,不要吃了。」周芷若道︰「張大哥,你若不吃,老道長心裡不快,他也吃不下飯,豈不是害得他肚子餓了?」無忌一想不錯,當周芷若將飯送到他嘴邉時,便張口吃了。周芷若細細心心的將魚骨雞骨剔除,每一口飯中再加上肉汁,無忌竟吃得十分香甜,將一大碗飯都吃光了。

張三丰心中稍慰,但轉念又想︰「無忌這孩子命苦,自幼死了父母,如他這般病重,原該有個細心的女子服侍他纔是。」只是常遇春不動魚肉,只是將那碗青菜吃得乾乾淨淨,雖在重傷之下,兀自吃了四大碗白米飯。張三丰雖是道士,却不忌葷腥,見常遇春食量甚豪,便勸他多吃雞肉。常遇春道︰「張眞人,咱拜菩薩的不吃葷。」張三丰道︰「啊,老道倒忘了。」

原來魔教中人規矩極嚴,每日只吃一頓晩餐,戒食葷腥,自唐朝以來,即是如此。北宋末年,魔教大首領方臘在浙東起事,當時官民均稱之爲「食菜事魔教」,食菜和奉事魔王,是魔教的兩大規律,傳之已達數百年。只是歷朝官府對魔教誅殺極嚴,武林中人也對之極是岐視,因此魔教教徒行事甚爲隱祕,雖然吃素,却對外人假稱奉佛拜菩薩,不敢洩漏自己身份。

常遇春道︰「張眞人,你於我有救命之恩,何況你也早已知曉我的來歷,自也不用瞞你。我是事奉明尊的明教中人,朝廷官府固然當咱是十惡不赦之徒,名門正派的俠義道瞧咱們不起,甚至是打家劫舍、殺人放火的黑道中人,也説咱們是妖魔鬼怪。可是你明知咱們的身份來歷,還是出手相救,這番恩德,當眞是不知如何報答了。」原來魔教所事奉的大魔王叫做摩尼,教中人稱爲「明尊」,稱自己的教爲「明教」,人却稱爲之爲魔教。

張三丰道︰「常英雄\dash{}」常遇春忙道︰「老道爺,你不用英雄長、豪傑短啦,乾脆叫我遇春得了。」張三丰道︰「好!遇春,你今年多大歳數?」常遇春道︰「我剛好二十歳。」張三丰見他雖然濃髯滿腮,但言談舉止之中,顯然年紀甚輕,是以有此一問,於是點頭道︰「你不過剛長大成人,雖然投入魔教,但陥溺未深,及早回頭,一點也没遲了。我有一句不中聽的話勸你,盼你不要見怪。」常遇春道︰「老道長見教,自是金玉良言,我怎敢見怪?」張三丰道︰「好!我勸你即日洗心革面,棄了邪教。你若是不嫌武當派本領低微,老道便命我大徒児宋遠橋收你爲徒。日後你行走江湖,揚眉吐氣,誰也不敢輕賤於你。」

宋遠橋是七俠之首,名震天下,尋常武林中人要見他一面亦是不易。武當諸俠直到近年方始收徒,但揀選甚嚴,若非根骨資質,品行性情無一不佳,決不能投入武當門下,常遇春出身魔教,常人一聽早就皺起眉,竟蒙張三丰垂青,那自是極大的福緣了。

豈知常遇春朗聲道︰「遇春蒙張眞人瞧得起,實是感激之極,但遇春身屬明教,終身不敢背教。」張三丰又勸了幾句,常遇春却堅決不從。張三丰見他執迷不悟,不由得暗自歎息,於是將無忌抱在手裡,説道︰「既是如此,咱們便此别過。」他不願和魔教中人打交道,那「後會有期」四字也忍住了不説。

張三丰抱了無忌,便要出船上岸,常遇春又拜謝。周芷若向無忌道︰「張家大哥,你要天天吃飽飯,免得老道爺操心。」無忌眼泪奪眶而出,哽咽道︰「多謝你好心,可是\dash{}可是我没幾天飯可吃了。」張三丰心下黯然,舉起袍袖,給他擦去腮上流下來的眼泪。周芷若驚道︰「什麼?你\dash{}你\dash{}」張三丰道︰「小姑娘,你良心甚好,但盼你日後走上正途,千萬别陥入邪魔纔好?」周芷若道︰「多謝老道教誨。」

常遇春忽道︰「張眞人,你老人家功行深厚,神通廣大,這位小爺雖然中毒不淺,總能化解吧?」張三丰道︰「是!」可是伸在無忌身下的左手,却輕輕搖了兩搖,意思是説他毒重難愈,只是不讓他自己知道。常遇春見他搖手,吃了一驚,説道︰「小人内傷不輕,正要去求一位神醫療治,何不便和這位小爺同去?」張三丰搖頭道︰「他奇經八脈已通,寒毒散入臟腑,非尋常藥石可治,普天下再無一人醫得」常遇春道︰「可是那位神醫却有起死人,肉白骨的能耐。」

張三丰一怔之下,猛地裡想起了一人,説道︰「你説的莫非是『蝶谷醫仙』?」常遇春道︰「正是他,原來老道長也知道我胡師伯的名頭。」張三丰心下却好生躊躇︰「素聞這『蝶谷醫仙』胡青牛是魔教一派,向爲武林人士所不齒,何況他脾氣怪僻無比,只要魔教中人患病,他盡心竭力的醫治,一文不收,教外之人求他,便是黃金萬兩堆在他的面前,他也不屑一顧。無忌寧可毒發身亡,我也決不容他陥身魔教。」要知那胡青牛以「青牛」兩字爲名,取意於「老子騎青牛出関而化胡」這句話,扣了這個「胡」字,那魔教原是由西域胡人傳入中土,另一含意義是青牛吃草,兼有「食菜事魔」和「嘗百草以治病」的意思,他我行我素,不加隱瞞,江湖上多知他是魔教中頗具身份的長老。

常遇春見他躊躇,明白他的心意,説道︰「張眞人,胡師伯雖是從來不給教外人治病,但張眞人有相救周姑娘的大恩,胡師伯非破例不可。他若當眞不肯動手,遇春決不和他干休。」張三丰道︰「這位胡先生醫術如神,我是聽到過的,可是無忌身上的寒毒,實非尋常\dash{}」常遇春大聲道︰「這位小爺反正不成了,最多治不好,左右也是個死,又有甚麼可想?」他性子爽直之極,心中想到甚麼,便説了出來。

張三丰聽到「左右也是個死」六個字,心頭一震,暗想︰「這莽漢子的話倒也不錯,眼看無忌最多不一月之命,只好死馬當作活馬醫。」於是説︰「如此便拜託你,可是咱們話説明在先,胡先生決不能勉強無忌入教,倘若當眞治好了,咱武當派也不領貴教之情。」他知道魔教中人行事詭秘,若是一糾纏上身,陰魂不散,不知道將有多少後患,張翠山弄到身敗名裂,便是一個活生生的例子。

常遇春昂然道︰「張眞人可把咱們明教中人,忒也瞧得小。」他轉頭向周芷若道︰「周姑娘,你暫且跟張眞人去,好不好?」周芷若尚未回答,張三丰愕然道︰「甚麼?」常遇春︰「張眞人不願去見我胡師伯,這個是我知道的,自來邪正不並立,張眞人是當今大宗師,如何能去相求於邪魔外道?我胡師伯脾氣古怪,若是見到張眞人,説不定禮貌不週,雙方反而弄僵。這位張兄弟只好由我帶去,但張眞人又未免不放心。是以我請周姑娘到武當山暫住,待張兄弟身子安好了,我送他上山,再來接周姑娘回去。打開天窗説亮話,那便是將周姑娘作抵押。」

張三丰一生和人相交,肝膽相照,自來信人不疑,可是張無忌是他愛徒唯一的骨血,要他交在以詭怪邪惡出名的魔教弟子手中,確是萬分的放心不下。

張三丰一時躊躇未答,常遇春又道︰「咱們周子旺大哥仁義過人,在信陽舉事失敗,滿門二十三口,全死於韃子之手,連周大哥七十八歳的老母,也是難免一刀。小人拚了性命,搶著他一子一女出來,豈知小公子又中韃子的毒箭身亡。這位姑娘是周大哥在世上獨一無二的親骨肉,周大哥身在明教,仇敵遍於天下,不但韃子要追捕他女児,他無數強仇若是知道訊息,非跟你張眞人找麻煩不可。張眞人,武當派雖然威震天下,但你還得小心。」

張三丰心下不禁啞然失笑,自己尚未答允收留周芷若,這個直心腸的漢子却已在諄諄叮囑起來,要跟周芷若爲難的人固多,江湖上要捉拿張無忌來加以逼問的人又豈是少了?只是無忌眼下毒入膏肓,當眞「左右也是個死」,多大的兇險也顧不得了。他也無法多想,便道︰「遇春,咱們一言爲定,我替你好好照顧周姑娘,你替我好好照顧無忌。待他體内陰毒去盡,便請你同他上武當山來。」常遇春道︰「受人之託,忠人之事,小人必當全力而爲。」

他跳上岸去,在一棵樹下用刀掘了個土坑,將周公子屍身上的衣服除得一絲不掛,這纔埋葬,和周芷若兩人跪在墳前,拜了幾拜。周芷若痛哭了一場,常遇春却站著默不作聲。要知「裸葬」乃明教的教規,教衆以爲每人出世時一絲不掛,離世時也當如此,張三丰不知其理,只覺這些人行事處處透著邪門詭異。

次日天明,張三丰擕同周芷若,與常遇春、無忌分手。無忌自父母死後,視張三丰如祖父一般,見他忽然離去,不由得泪如泉湧。張三丰溫言道︰「無忌,你病好之後,常大哥便帶你來到武當,乖孩子,分别數月,不用悲傷。」無忌手足動彈不得,只點了點頭,眼泪仍是不斷的流將下來。周芷若回上船去,從懷中取出一塊小手帕,替他抹去了眼泪,對他微微一笑,將手帕塞在他的衣襟之中,這纔回到岸上。張三丰心中一動︰「這小姑娘如此美麗,他年定是個絶色佳人。無忌若得傷愈,我決不容他二人再行相見,否則不幸二人互有情意,豈不是重𨂻翠山的覆轍?」

無忌目送太師父帶同周芷若在岸上西去,只見周芷若不斷回頭揚手,直走到一排楊柳背後,這纔不見。無忌霎時之間,只覺孤單淒涼,難當無比,忍不住又哭了起來。常遇春皺眉道︰「張兄弟,你今年幾歳?」無忌哽咽道︰「十二歳。」常遇春道︰「好啊,十二歳的人,又不是小孩子了,哭哭啼啼的,不怕醜麼?我在十二歳上,已不知挨過幾百頓好打,從來不作興流過半滴眼泪。男子漢大丈夫,只流鮮血不流泪。你再{\upstsl{妞}}児般的哭個不停,我可要拔拳打你了。」

無忌見他形相兇猛,心中好生害怕,暗想︰「我太師父剛去,你便對我如此狠惡,日後不知要吃你多少苦頭?」口中朗聲道︰「我是不捨得太師父纔哭,人家打我,我纔不哭呢。你敢打我便打好了,你今日打我一拳,他日我打還你十拳。」常遇春笑道︰「今日我打了你,他日你與你太師父學好了武功,這武當派的神拳,我可挨得起十拳嗎?」無忌波的一聲,笑了出來,覺得這個常大哥雖然相貌兇惡,倒也不是壞人。

當下常遇春取出銀兩、僱了一艘江船,直航漢口。到了漢口後,另換長江江船,沿江東下,原來那蝶谷醫仙胡青牛所隱居的蝶谷,是在皖北的女山湖畔。

長江自漢口到九江,流向東南,直到九江後,便折向東北而入皖境。兩年之前,無忌曾乘船溯江而上,但其時有父母相伴,又有兪蓮舟同行,旅途中何等快活,今日父母雙亡,自己全身穴道封閉,悽悽惶惶的隨常遇春東下求醫,其間苦樂,實有天壤之别。只是生怕常遇春發怒,心中雖然傷感,却也不敢流泪。每日子午兩時,體内寒毒發作,每一次均有大半個時辰的痛楚難當,無忌咬牙強忍,只咬得上下口唇傷痕斑斑,而且陰寒侵襲,一日甚於一日。

好容易到得集慶(即今之南京)下游的瓜埠,常遇春捨舟起旱,僱了一輛大車,向北進發,數日間到了鳳陽以東的明光。常遇春知道這位胡青牛胡師伯脾氣古怪,不喜旁人得知他隱居的所在,待大車行到離女山湖畔的蝴蝶谷尚有二十餘里地,便命大車轉頭,自己將無忌負在背上,大踏步而行。

他只道這二十餘里路程轉眼即至,豈知他身中番僧的兩記陰掌,内傷著實不輕,只走出里許,便是全身筋骨酸痛,氣喘吁吁的步履爲艱。無忌好生過意不去,道︰「常大哥,咱們慢慢走吧。你别累壞了身子。」常遇春焦躁起來,怒道︰「我平時一口氣走一百里路,也半點不累,難道那兩個賊和尚打了我兩掌,便叫我寸步難行?」他睹氣加快脚步,奮力而行,但一個身有内傷之人,這般心躁氣浮的勉強用力,只走出數十丈,幾乎四肢百骸骨節一齊都要散開一般。他兀自不服氣,不肯坐下休息,一步步的向前挨去。

這般走法,那就慢得緊了,行到天黑,尚未走得一半,而且山路崎嶇,越來越是難走,總算挨到了一座樹林之中,常遇春將無忌放下地來,仰天八叉的躺著休息。他懷中帶著些給無忌吃的糖果糕餅,兩人分著吃了。常遇春休息了半個時辰,又要趕路,無忌極力相勸,説在林中安睡一晩,待天明了再走。常遇春心想今晩便是趕到,半夜三更的去驚吵胡青牛,説不定他一怒之下,反而不肯醫治,只得依了無忌,兩人在一棵大樹下相倚而睡。

睡到半夜,無忌身上寒毒又發作起來,劇顫不止。他生怕吵醒了常遇春,一聲不響,強自忍耐,便在此時,忽聽得遠遠有兵刃相交之聲,又有人{\upstsl{吆}}喝道︰「往那裡走?」

\qyh{}堵住東邉,逼他到樹林中去。」

\qyh{}這一次不能再讓這賊禿走了。」跟著脚步聲響,幾個人快速異常的奔向樹林中來。常遇春一驚而醒,一手拔出單刀,一手抱起無忌,以備且戰且走。無忌低聲道︰「常大哥,似乎不是衝著咱們而來。」常遇春點了點頭,心中已是打好了主意,寧可力戰而死,也要保護無忌週全,只是自己受傷後武功全失,不由得大是焦急。

他躱在大樹後向外望去,只見影影綽綽,七八個人圍著一人相鬥。黑暗中看不清各人的身形,不知雙方各是些什麼人物,但見中間那人赤手空拳,雙掌飛舞,却逼得敵人無法欺近身去,鬥了一陣,衆人身形移動,一步步打近,常遇春和無忌藏身處的大樹旁來。一輪眉月從雲中鑽出,清光瀉地,只見中間那人身穿白色僧衣,是個五十來歳的高瘦僧人。圍攻他的衆人中却是有僧有道,有俗家打扮的漢子,還有兩個女子。常遇春凝神觀鬥,越看越是心驚,見圍攻的人個個武功精奇。兩個和尚一執禪杖,一持戒刀,禪杖橫掃,戒刀斜劈之際,一股股疾風帶得林中落葉四散飛舞。一個道人手持長劍,身形變動奇快,忽而在左,忽而在右,長劍抖動,在月光下閃出一團團劍花。一個矮小的漢子雙手各握一柄單刀,在地下滾來滾去,以地堂刀法進攻那僧人的下盤。

兩個女子身形苗條,各執長劍,劍法也是極盡靈動輕捷。酣鬥中的一個女子轉過身來,半邉臉龐照在月光之下,無忌險些失聲而呼︰「紀姑娘!」原來這女子正是殷利亨的未婚妻子紀曉芙。無忌初時見八個人圍攻一個和尚,覺得以多欺少,甚不公平,心中盼望那和尚能突圍而走,但這時認出紀曉芙後,心想那和尚既和紀姑姑爲敵,自是個壞人,一顆心便去幫在紀曉芙一邉了。

常遇春低聲的自言自語︰「八個人打一個,太不要臉,不知都是些什麼人?」無忌低聲道︰「兩個女子是峨嵋派的,{\upstsl{嗯}},兩個和尚都是少林派的。他『大漠飛沙』使得多狠,正是崑崙派的絶招。這使地堂刀的漢子却不知是什麼門派。」常遇春道︰「是崆峒的吧。」無忌搖頭道︰「不是崆峒派的地堂刀法,右手用刀、左手使拐,這兩人却使雙刀。」常遇春心下暗自佩服︰「當眞名門子弟,見識畢竟不凡。」他那知無忌的武功却主要學自謝遜,此人武學博大精深,因一心和各家各派爲敵,各家各派的武功便無所不窺。無忌日受親炙,雖談不上通曉,但見識却是不差。無忌見紀曉芙等久鬥那和尚不下,越看越是欽佩那和尚武功了得,但見他掌力雄渾,忽快忽慢,虛虛實實,打到快時,連他手掌的去路來勢都瞧不清楚,别説捉摸他的招數門派了,忽聽得一名漢子喝道︰「用暗青子招呼!」當即一名漢子和一個道人分向左右躍開,跟著便是嗤嗤聲響,彈丸和飛刀不斷向那和尚射去。這麼一來,那和尚便有點児難以支持。那使劍的道人喝道︰「彭和尚,咱們又不是要你性命,你拚命幹麼?你把白龜壽交了出來,大家一笑而散,豈不是妙?」常遇春吃了一驚,低聲道︰「這位便是彭和尚?」

無忌在江船之中,聽父母和兪蓮舟説起在王盤山揚刀立威大會,以及白眉教和各幫會結仇的來由時,知道白龜壽是白眉教在王盤山僅得生還的玄武壇壇主。崑崙派雖也有兩人僥倖不死,但已被謝遜的吼聲震成了白痴,因此十多年來各門各派和白眉教鬥爭不休,便是要白龜壽吐露謝遜的蹤跡。無忌心道︰「莫非這彭和尚也是我媽教中的人物?」却聽彭和尚朗聲道︰「那白壇主已被你們打得重傷,我彭和尚莫説他日後均是白眉教中人,便是毫無干連,也不能見死不救。」那道人道︰「什麼見死不救?咱們又不是取他性命?只是跟他打聽一個人。」彭和尚道︰「你們要問謝遜的下落,爲何不去問少林寺方丈?」圍攻他的一個少林僧叫了起來︰「這是白眉教妖女殷素素嫁禍少林的奸計,誰能信得?」無忌聽那少林僧提到亡母的名字,又是驕傲,又是傷心,暗想︰「我媽雖已逝去兩年,仍能作弄得你們頭昏腦脹。」但見彭和尚和衆人一問一答,手下却是絲毫没緩。那道人想引得他説話分心,便可乘虛而入,豈知彭和尚武功固強,心智也是高人一等,這等小小玄虛,焉能騙得了他?只是圍攻他的人是集中了數派的精英人物,竟無一個庸手,他數次想突圍而逃,却也不能。猛聽得站在外圍放射暗器的道人叫道︰「啊喲,不好!暗器打光了!」六個人一聽叫聲,同時伏地,但見白光閃動,五柄飛刀,激射而至。原來他「暗器打光了」這句話是個暗號,叫圍攻的衆人伏地相避。這五把飛刀勁道威猛之極,成梅花之形,對準了彭和尚的胸口射到。若在尋常之時,彭和尚只須低頭彎腰,或是向前撲跌,要不然便使鐵板橋仰身,使飛刀在胸前掠過,但這時地下六般兵刃一齊上撩,封住了他下三路,却如何能矮身閃躱?

無忌心頭一驚,只見彭和尚身形突然縱起,躍高丈許,五柄刀一齊從他脚底飛過,飛刀雖是避過,但少林僧的禪杖戒刀、崑崙派道人的長劍一齊向他腿上擊到。彭和尚身在半空,迫得使用險招,左掌拍出,波的一響,擊在一名少林僧的光頭之上,跟著右手一勾,已搶過了他手中的戒刀,順勢在禪杖上一格,借著這股力道,身子已飛出數丈。那少林僧被他一掌重手擊在天靈蓋上,立時斃命。餘人怒叫追去,只見彭和尚足下一個踉蹌,險險摔倒,七個人又將他重重圍住。那使禪杖的少林僧勢如瘋虎,一柄禪杖直上直下的猛{\upstsl{砸}},叫道︰「彭和尚,你殺了我師弟,我跟你拚了。」那崑崙派的道人道︰「他腿上已中了我的蝎尾鉤,轉眼便要毒發身亡。」果見彭和尚足下虛浮,掌叉已見散亂。常遇春急道︰「他\dash{}他是我周大哥的師父啊,怎生救他一救纔好?」無忌知他熱腸過人,雖是自己身負重傷仍要衝出去救人,除了徒然送命之外,殊無半點補益,心念一動。低聲道︰「常大哥,你想去救彭和尚,是不是?」常遇春道︰「不救不行的,他中了餵毒的暗器,可是我\dash{}我\dash{}」無忌道︰「我教你個法児,可使你恢復原來神力支持得半個時辰,只是不免損耗元氣。」常遇春適纔聽他指明各派的武功,信得過他既是張三丰的傳人,必有特殊本事,喜道︰「好兄弟,快説。救人要緊,耗些元氣怕什麼?」無忌道︰「你找塊尖角石子來。」常遇春在地下一摸便摸到了,道︰「這塊成麼?」

無忌點頭道︰「很好。你在自己腰下兩旁,雙腿之側的一個地方,用尖角石子猛力擊一下。」常遇春指著腿旁,道︰「是這裡麼?」無忌道︰「再下一點児,對啦,還要偏左半寸,好,用力擊下去。」常遇春依言一擊,只覺右腿登時酸麻,無忌道︰「這是提神打穴法,再打左腿。」常遇春有些遲疑,但他雖未學過點穴打穴之法,却知武學中確有這一門功夫,心想武當名震天下,打穴之法決計差不了,於是又在左腿上用石子猛力一擊。

不料擊了這兩下之後,下半身登時麻痺,雙腿再也作不了半分主,只見彭和尚一躍數丈,摔倒在地,常遇春大急之下,便要衝出去相救,但兩隻脚那裡動得了?驚道︰「張兄弟,怎\dash{}怎麼了?」無忌心下暗笑︰「我騙得你自己打了『環跳』雙穴,這『環跳穴』一下,自是動不得了。」口中却假作驚惶︰「啊喲,你不會打穴,只怕力道使得不對。再等一會児,多半便行。」常遇春並非蠢笨之人,一轉念間,已知著了這刁鑽古怪的小兄弟的道児,但想他也是一番好意,不由得又驚又急,又是好氣,又是好笑。

只見彭和尚倒在地下,似已毒發身亡,那七人却也不敢走近身去。崑崙派的道人道︰「許師弟,你放他兩柄飛刀試試。」那放飛刀的道人右手一揚,拍拍兩響,一柄飛刀射入彭和尚右肩,一柄射入他的左腿。彭和尚毫不動彈,顯已死去。那崑崙道人道︰「可惜,可惜,已經死了,却不知他將白龜壽藏在何處?」七個人圍了上去察看。

忽聽得砰砰砰砰砰,五聲急響,五個人同時向外摔跌,彭和尚已威風凜凜的站立起身,肩頭和腿上的飛刀却兀自插著,原來他腿上中了餵毒暗器,知道難以支持,便裝假死,誘得敵人近身,以連發的「五行拳」,在五個男敵的胸口各印了一掌。却放過了紀曉芙和另一個峨嵋女弟子。紀曉芙和她同門師姊丁敏君一驚之下,急忙躍開,看那五個同伴時,個個口噴鮮血,兩名漢子功力較遜,已是跪倒在地。但彭和尚這一急激運勁,也已搖搖欲墜,站立不定。那崑崙道人叫道︰「丁紀兩位姑娘,快用劍刺他。」

\chapter{殺絶活口}

雙方敵對的九人之中,一名少林僧已死,彭和尚和五個敵人同受重傷,只有紀曉芙和丁敏君都是毫髮無損。丁敏君聽那崑崙道人叫喊,心道︰「難道我不會用劍,還要你説?」長劍一招「虛式分金」逕往彭和尚足脛削去。

彭和尚長嘆一聲,心想︰「因你二人是女流之輩,出家人使掌擊打你們胸口,涉嫌輕薄,這纔下手留情,不料一念之仁,反招來殺身大禍。」眼見她劍尖削到,只有閉目待死,却聽得叮{\upstsl{噹}}一響,兵刃相交,張眼一看,却是紀曉芙伸劍將師姊一劍格開了。

丁敏君一怔,道︰「怎麼?」紀曉芙道︰「師姊,彭和尚掌下留情,咱們可也不能趕盡殺絶。」丁敏君道︰「我又不要殺他,只是留他下來,要他吐露白龜壽的所在。」紀曉芙道︰「他身中餵毒暗器,傷勢已重,先解了他的毒再説。」走到崑崙道人面前,道︰「西靈師兄,請把蝎尾鉤的解藥給我。」原來那道人道號西靈子,那使飛刀的道人叫作西捷子,都是西華子的師弟。

西靈子道︰「你先將他綁了,這和尚鬼計多端,甚是難防\dash{}」一面説,一面不住喘氣,強忍胸口翻湧上來的鮮血,他中了彭和尚這一記「五行掌」,受傷極是沉重。紀曉芙微一沉吟,點了點頭,取出絲條,走到彭和尚身旁,柔聲道︰「彭大師,委屈你一下。」彭和尚只覺腿上中毒之處,不住麻將上來,心知若無解藥,轉眼便得送命,反正不給他綁,她長劍一揮,挑斷自己脚筋,更加多受痛苦,若是出掌偸襲,旁邉却有個丁敏君仗劍監視,只得苦笑一下,由得她綁住了手足。西靈子從懷中取出解藥,喘著氣説了用法。紀曉芙先替彭和尚拔下兩柄飛刀,再在他腿上起下蝎尾鉤,敷上解藥。

丁敏君厲聲道︰「彭和尚,我師妹心慈,救了你一命,那白龜壽在那裡,這該説了吧?」彭和尚仰天大笑,説道︰「丁姑娘,你可將我彭瑩玉忒也看得小了。武當派張翠山張五俠寧可自刎而死!也決不説出他義兄的所在,彭瑩玉心慕張五俠的義肝烈膽,雖然不才,也要學他一學。」這幾句話只聽得無忌胸中熱血湧了上來,對彭和尚更增幾分好感。要知張翠山自刎身亡,在武當、峨嵋、少林諸人士雖覺惋惜,總不免説道︰「好好一位少年英俠,却受了邪教妖女之累,一失足成千古恨,終至身敗名裂,使得武當一派,同蒙羞辱。」無忌是個十分聰明之人,在太師父和各位師叔伯的言談神色中間,瞧得出他們傷心之餘,對母親頗有怒恨怪責的意思,只覺得父親一生什麼都好,就是娶錯了母親,却從無一人與彭和尚這般對他父親衷心欽佩。

丁敏君冷笑道︰「張翠山瞎了眼睛,竟去和魔教妖女締婚,這叫作自甘下賤,有什麼好學的?他武當派\dash{}」紀曉芙插口道︰「師姊\dash{}」丁敏君道︰「你放心,我不會説到殷六俠頭上。」她長劍一晃,指著彭和尚的右眼,説道︰「你若是不説,我先刺瞎你的右眼,再刺瞎你的左眼,然後刺聾你的右耳,又刺聾你的左耳,再削掉你的鼻子,總而言之,我不讓你死便是。」她劍尖和彭和尚眼珠距離不到半寸,晶光閃耀的劍尖顫動不休。彭和尚睜大了眼睛,一瞬也不稍瞬,淡淡的道︰「素仰峨嵋派滅絶師太行事心狠手辣,她調教出來的弟子自也差不了,彭瑩玉今日落在你手裡,你便請施展峨嵋派的拿手傑作吧!」

丁敏君蛾眉上揚,厲聲道︰「好賊禿,你膽敢辱我師門?」長劍向前一送,登時刺瞎了彭瑩玉的右眼,跟著劍尖便指在他左眼皮上。彭瑩玉哈哈一笑,一隻左眼却睜著大大的瞪視著她,丁敏君被他瞪得心中發毛,喝道︰「你又不白眉教的,何必爲了白龜壽送命?」

彭瑩玉凜然道︰「大丈夫做人的道理,我便是跟你説了,你也不會明白。」丁敏君見他雖無絲毫反抗之力,但神色之間,對自己却是大爲輕蔑,憤怒中長劍一送,便去刺他的左眼。紀曉芙揮劍格開,道︰「師姊,這和尚硬氣得很,不管怎樣,他總是不肯説的了,殺了他也是枉然。」丁敏君道︰「他罵師父心狠手辣,我便心狠手辣給他瞧瞧。這種魔教中的妖人留在世上只有多害好人,殺得一個,便是積一番功德。」紀曉芙道︰「這人也是條硬漢子,師姊,依小妹之見,便饒了他吧。」丁敏君朗聲道︰「這裡少林派的兩位師兄,一死一傷,崑崙派的兩位道長身受重傷,海沙派的兩位大哥傷得更是厲害,難道他下手還不彀狠麼?我廢了他左邉的招子,再來逼問。」那「問」字剛出口,劍如電閃,疾向彭和尚的左眼刺去。

紀曉芙長劍一橫,輕輕巧巧的將丁敏君這一劍格開了,説道︰「師姊,這人已然無力還手,這般傷害於他,江湖上傳將出去,於咱們峨嵋派聲名不好。」丁敏君長眉一揚,喝道︰「站開些,你别管我。」紀曉芙道︰「師姊,你\dash{}」丁敏君道︰「你既叫我師姊,便得聽師姊的話,不用再囉裡囉唆。」紀曉芙道︰「是!」丁敏君長劍抖動,又向彭和尚的左眼刺去,這一次又加了三分勁。

紀曉芙心下不忍,又是伸劍一格,她見師姊劍勁凌厲,出劍時也用上了内力,雙劍一交,{\upstsl{噹}}的一響,火花飛濺,兩人各自震得手臂發麻,退了兩步。丁敏君大怒,喝道︰「師妹,你三番兩次,迴護這魔教中的妖僧,到底是何居心?」紀曉芙道︰「我是勸你别這般折磨他,要他説出白龜壽的下落來,儘管慢慢問他便是。」丁敏君冷笑道︰「難道我不知你的心意。你倒撫心自問︰武當派殷六俠幾次催你完婚,爲什麼你總是推三阻四,爲什麼你爹爹也來催你時,你寧可離家出走?」紀曉芙道︰「咦,小妹自己的事,跟這件事又有什麼干係?師姊怎地扯在一起。」

丁敏君道︰「我們大家心裡明白,當著這許多外人之前,也不用揭誰的瘡疤。你是身在峨嵋,心向魔教。」紀曉芙氣得滿臉慘白,顫聲道︰「我平時敬你是師姊,從無半分得罪你啊,爲何今日這般羞辱於我?」丁敏君道︰「好,倘若你不是心向魔教,那你便一劍把這和尚的左眼給我刺瞎了。」紀曉芙道︰「本門自小東邪郭祖師開主宗派,派中歷代宗祖,自守不嫁的女子很多,小妹不過心慕先師高德,不願出嫁,那也事屬尋常,師姊何必苦苦相逼?」丁敏君道︰「我不聽你這些假撇清的言語。你不刺他眼睛,我可要一句一句,將你的事都抖露出來了?」紀曉芙似乎做了什麼虧心之事,不敢再行倔強,柔聲道︰「師姊,望你念在同門之情,勿再逼我。」

丁敏君笑道︰「我又不是要你去做什麼爲難的事児。師父命咱們打聽金毛獅王謝遜的下落,眼前和尚正是唯一可資著手之處。他不肯吐露眞相,又殺傷了咱們這許多同伴,我刺瞎他右眼,你刺瞎他左眼,那可説是天公地道,你爲什麼不動手?」紀曉芙低聲道︰「小妹心軟,下不了手?」丁敏君冷笑道︰「你心軟?師父常讚你劍法狠辣,性格剛毅,最像師父,一直有意把衣缽傳你,你怎麼心軟?」

她同門師妹吵嘴,旁人都聽得没頭没腦,這時才隱約聽出來,似乎峨嵋派掌門滅絶師太對紀曉芙特别喜愛,有相授衣缽眞傳之意,丁敏君不免心懷嫉妬,這次不知抓到了她什麼把柄,便存心要她當衆出醜。張無忌的小小心靈中的極重恩怨,想起自己父母自殺那日,紀曉芙待己甚好,這時眼見她受過,恨不得跳出去打丁敏君幾個耳光。

只聽丁敏君道︰「紀師妹,我來問你,三年之前,師父在峨嵋金頂召聚本門徒衆,傳授她老人家手創的『滅劍』和『絶劍』兩套劍法,你爲什麼不到?爲什麼惹得師父她老人家大發雷霆,以致將長劍震斷,説從此世上没這兩套劍法?」紀曉芙道︰「小妹在甘州忽患急病,動彈不得,此事早已稟明師父,師姊何以忽又動問?」丁敏君冷笑道︰「此事你瞞得過師父,却瞞不過我。我下面還有一句話問你,你若是將這和尚的眼睛刺瞎了,我便不問。」

紀曉芙低頭不語,心中好生爲難,輕聲道︰「師姊,你全不念咱同門學藝的情誼?」丁敏君道︰「你刺不刺?」紀曉芙道︰「師姊,你放心,師父便是要傳我衣缽,我也決計不敢相受。」丁敏君怒道︰「好啊!這麼説來,倒是我在喝你的醋啦,我什麼地方不如你,要來承你的情,要你推讓?你到底刺呢不刺?」紀曉芙道︰「小妹便是做了不對的事,師姊如要責罰,小妹難道還敢不服的麼?這児有别門别派的朋友在此,你如此逼迫於我\dash{}」説到這裡,不禁流下泪來。

丁敏君冷笑道︰「嘿,你裝著這副可憐巴巴的樣児,心中却不知在怎樣咒我呢。三年之前,你在甘州,當眞是生病麼?『生』是倒有個『生』字,却只是生娃娃吧?」

紀曉芙聽到這裡,一轉身,拔足便奔。丁敏君早料到他要逃走,飛步上前,長劍一抖,攔在她的面前,説道︰「我勸你乖乖的把彭和尚左眼刺瞎了,否則我便要問你那娃娃的父親是誰?問你爲什麼以一個名門正派的弟子,却這麼維護一個魔教的妖僧?」紀曉芙氣急敗壞的道︰「你\dash{}你讓我走!」丁敏君長劍指在她的胸前,大聲道︰「我問你,你把娃娃養在那裡?你是武當派殷利亨殷六俠的未婚妻子,怎地跟旁人生了孩子?」

這幾句石破天驚的話問了出來,聽在耳中的人都是禁不住心頭一震。張無忌心中一片迷惘︰「這位紀姑姑是個好人啊,怎能對殷叔叔不住?」他只是個十三歳的孩子,對這些男女之事自是不大了然,但便是常遇春、彭和尚、西靈子這些人,也是大感奇異。

紀曉芙臉色慘白,向前疾衝,豈知丁敏君説動手便眞動手,刷的一劍,已在她右臂上深深劃了一劍,直削至骨。紀曉芙受傷不輕,再也忍耐不住,左手拔出佩劍,説道︰「師姊,你再苦苦相逼,我可要對不住啦。」丁敏君知道今日既已破臉,自己又揭破了她的隱祕,她勢必要殺己滅口,自己武功不及這位師妹,當眞性命相搏,那是凶險之極,是以一上來乘機先傷了她的右臂,聽她這麼説,當下一招「笑指天南」,直刺她的小腹。

紀曉芙右臂劇痛,眼見師姊出的又是毫不容情的毒招,當即左手執劍,還招擋開。兩人這一搭上手,以快打快,迅即拆了二十餘招。旁觀衆人個個都是武林的好手,但個個身受重傷,既無法勸解,亦不能相助那一個,只有眼睜睜瞧著,心中均是暗自佩服︰「峨嵋爲當今武學四大宗派之一,劍法果是超逸絶倫,名不虛傳。」她師姊二人互知對方劍法,攻守之際,分外緊湊,也是分外的激烈。

紀曉芙右臂傷口血流不止,越鬥鮮血越是流得厲害,她連使殺著,想將丁敏君逼開,以便奪路而去,但她左手使劍甚是不慣,再加受傷之後,原有的武功已留不了三成。總算丁敏君對這位師妹向來甚是忌憚,不敢過份進逼,只是纏住了她,要她流血過多,自然衰竭。眼見紀曉芙脚步蹣跚,劍法漸漸散亂,已是支持不住,丁敏君刷刷兩招,紀曉芙右肩上又接連中劍,半邉衣衫上全染滿了鮮血。

彭和尚忽然大聲叫道︰「紀姑娘,你來將我的左眼刺瞎了吧,彭和尚對你已是感激不盡。」要知紀曉芙甘冒生死之險,迴護敵人,已是極爲難能,何況丁敏君用心威脅她的,更是一個女子瞧得比性命更重的清白名聲?但這時紀曉芙便是去刺瞎了彭和尚的左眼,丁敏君也已決計饒她不過,心知今日若不乘機下手除去,日後可是禍患無窮。

彭和尚見丁敏君劍招狠辣,大聲叫罵︰「你這不要臉的丁敏君,無怪江湖上送你一個綽號叫作『毒手無鹽丁敏君』,果然是心如蛇蝎,貌似無鹽。要是世上的女子個個都似你一般醜陋,令人一見便作嘔,天下男子人人都要去作和尚了。」其實丁敏君雖非絶色的美女,却也是頗具姿容,面目俊俏,甚有楚楚之致。彭和尚深通世情,知道普天下女子的心意,不論她是醜是美,你若罵她一聲難看,她非恨你切骨不可。他眼見情勢危急,只得隨口胡謅,給她取了個「毒手無鹽」的渾號,盼她一怒之下,轉來對付自己,紀曉芙便可乘機脱身,至少也能設法包紮傷口。

那知丁敏君的心思甚是細密,暗想待我殺了紀曉芙,還怕你這臭和尚逃到那裡去?是以對他的辱罵竟是充耳不聞。彭和尚又朗聲道︰「紀女俠冰清玉潔,江湖上誰不知聞?可是『毒手無鹽丁敏君』却偏偏自作多情,妄想去勾搭人家武當派殷利亨,殷利亨不睬你,你自然想加害紀女俠啦。哈哈,你顴骨這麼高,嘴巴大得像隻血盆,焦黃的臉皮,身子却又像根竹竿,人家英俊瀟灑的殷六俠怎會瞧得上眼?你也不自己照照鏡子,便向人亂抛媚眼\dash{}」丁敏君聽到這裡,只氣得全身發顫,一個箭步,縱到彭和尚身前,挺劍便往他嘴中刺去。

原來丁敏君顴骨確是微高,嘴非櫻桃小口,皮色不彀白皙,又生就一副長挑身材,這一些微嫌美中不足之處,旁人若非細看,本是不易發覺,但彭和尚自來目光極是鋭敏,不論是誰,只要給他見過一面,此人身材容貌上的特色,他便終身不忘。丁敏君對自己容貌上這些小小缺憾,原是常感不快,此時給彭和尚張大其辭的胡説一通,却教她如何不怒?何況殷利亨其人,她從未見過,「亂抛媚眼」云云,眞是從可説起?

她一劍將要刺到,樹林中突然閃出一人,大喝一聲,擋在彭和尚身前。這人來得快極,丁敏君不及收招,一劍已然刺出,那人比彭和尚矮了半個頭,這一劍正好透額而入。便在這電光石火般的一瞬之間,那人也是一掌拍出,掌力到處,擊中丁敏君的胸口,砰然一聲,將她震得飛出數步,一交摔倒,口中狂噴鮮血,一柄長劍却插在那人額頭,眼見他也是不活了。

崑崙派的西靈子走近兩步,驚呼︰「白龜壽,白龜壽!」原來替彭和尚擋了這一劍的,正是白眉教玄武壇壇主白龜壽。他身受重傷之後,得知彭和尚爲了掩護自己,受到少林、崑崙、峨嵋、海沙四派的好手圍攻,於是力疾趕來,替彭和尚代受了這一劍。他掌力雄渾,臨死這一掌却也擊得丁敏君肋骨斷折數根。

紀曉芙驚魂稍定,撤下衣襟包好了臂上傷口,伸劍挑斷綁著彭和尚手足的絲條,一言不發,轉身便走。彭和尚道︰「且慢,紀姑娘,受我彭和尚一拜。」説著行下禮去,紀曉芙閃在一旁,不受他這一拜。彭和尚拾起西靈子遺在地下的長劍,道︰「這丁敏君毀謗姑娘金名,不能再留這活口。」説著挺劍便向丁敏君咽喉刺下。紀曉芙左手揮劍格開,道︰「她是我同門師姊,她雖對我無情,我可不能對她無義。」彭和尚道︰「事已如此,若不殺她,這女子日後定要對姑娘大大不利。」紀曉芙垂泪道︰「我是天下最不祥最不幸的女子,一切認命罷啦!彭師傅,你别傷我師姊。」

彭和尚道︰「紀女俠所命,焉能不遵?」紀曉芙低聲向丁敏君道︰「師姊,你自己保重。」説著還劍入鞘,出林而去。

彭和尚對西靈子等一干人説道︰「我彭和尚跟你們並無深仇大冤,金毛獅王謝遜也不是非殺你們不可,但今晩這姓丁的女子誣衊紀女俠之言,你們都已聽在耳中,傳到江湖之上,却教紀女俠如何做人?我不能留下活口,乃是情非得已,你們可别怪我。」説著一劍一個,將西靈子、西捷子、一名少林僧、兩名海沙派的好手,盡數刺死,跟著又在丁敏君的臉上劃了一劍。丁敏君只嚇得心膽倶裂,但重傷之下,却又抗拒不得,罵道︰「賊禿,你别零碎折磨人,一劍將我殺了吧。」彭和尚笑道︰「像你這種皮黃闊口的醜女,我是不敢殺的。只怕你一入地獄,將陰世裡千千萬萬的惡鬼都嚇得逃到人間來,又怕你嚇得閻王判官上吐下瀉,豈不作孼?」説著大笑三聲,擲下長劍,抱起白龜壽的屍身,又大哭三聲,揚長而去。丁敏君喘息良久,纔以劍鞘拄地,緩緩出林。

這一幕驚心動魄的林中夜鬥。常遇春和張無忌二人清清楚楚的瞧在眼裡,直到丁敏君出林,兩人方鬆了一口氣。無忌道︰「常大哥,紀姑娘是我殷六叔的未婚妻子,那姓丁的女子説過\dash{}説過跟人生了個娃娃,你説是眞是假?」常遇春道︰「這姓丁的女子胡説八道,别信她的。」無忌道︰「對,下次我跟殷六叔説,叫他好好的教訓教訓這丁敏君,也好代紀姑姑出一口氣。」常遇春忙道︰「不,不!千萬不可跟你殷六叔提這件事,知道嗎?一提那可糟了。」無忌奇道︰「爲什麼?」常遇春道︰「這種不好聽的言語,你跟誰也别説。」無忌「{\upstsl{嗯}}」了一聲,過了一會,又道︰「常大哥,你怕那是眞的,是不是?」常遇春嘆道︰「我也不知道啊。」

到得天明,常遇春穴道已解,將無忌負在背上,眼見林中橫七豎八的屍首,心想︰「那謝遜絶跡江湖,已是十餘年,但武林中人,仍是源源不絶的爲他送命。這件禍事,不知何日方解?」他在林中一動不動的休息了大半夜,精神已復,步履之際也輕捷得多了。走了數里,轉到一條大路上來。常遇春心想︰「胡師伯在蝴蝶谷中隱居,住處甚是荒僻,怎地到了大路上來,莫非走錯路了?」正想找個鄕人打聽,忽聽得馬蹄聲響,四名蒙古兵手舞長刀,縱馬下來,大呼︰「快走,快走!」奔到常遇春身後,舉刀虛劈作勢,驅趕向前。常遇春暗暗叫苦︰「想不到今日終於又入虎口,却陪上了張兄弟一條性命。」這時他武功全失,連一個尋常的元兵也鬥不過,只得一步步的挨將前去。但見大路上百姓絡繹不斷,都被元兵趕畜牲般驅來,常遇春心中又存了一線之機︰「看來這些韃子正在虐待百姓,未必定要捉我。」

他隨著一衆百姓行去,到了一處三叉路口,只見一個蒙古軍官騎在馬上,領著六七十名士卒,元兵手中各執大刀。衆百姓行過他身前,便跪下磕頭,一名漢人通譯喝問︰「姓什麼?」那人答了,旁邉一名元兵或是在他屁股上用力踢上一脚,或是一記耳光,那百姓匆匆走過。問到一個百姓答稱姓張,那元兵當即一把抓過,命他站在一旁。又有一個百姓手挽的籃子中有一柄新買的菜刀,那元兵也將他抓在一旁。

無忌一見情勢不對,在常遇春耳邉悄聲道︰「常大哥,你快假裝摔一交,摔在草叢之中,解下腰間的佩刀。」常遇春登時省悟,隻膝一彎,撲在長草叢中,除下了佩刀,假裝哼哼{\upstsl{啷}}{\upstsl{啷}}的爬將起來,一步步挨到那軍官身前。那漢人通譯罵道︰「賊蠻子,不懂規矩,見了大人不快磕頭?」

常遇春想起故主周子旺全家慘死於蒙古韃子的刀下,這時寧死也不肯向韃子磕頭。一名元兵見他倔強,伸脚在他膝彎裡橫掃一腿。常遇春站立不穩,撲地跪下。那漢人通譯喝道︰「姓什麼?」常遇春還未回答,無忌搶著道︰「姓謝,他是我大哥。」那元兵在無忌屁股上踢了一脚,喝道︰「滾吧!」

常遇春滿腔怒火,爬起身來,心中暗暗立下重誓︰「此生若不將韃子逐回漠北,我常遇春誓不爲人。」負著無忌,急急向北行去,只走出數十步,忽聽得身後慘呼哭喊之聲大作。兩人回過頭來,但見被元兵拉在一旁的十多名百姓,個個身首異處,屍橫就地。原來當時朝政暴虐,百姓反叛著甚多,蒙古大臣有心要殺盡漢人,却又是殺不勝殺,當朝太師巴延便頒下一條虐令,殺盡天下張、王、劉、李、趙五姓漢人。因漢人之中,以張、王、劉、李四姓之人最多,而趙姓則是宋朝皇族,這五姓之人一除,漢人自必元氣大傷。後來皇帝不許,纔取消了這條暴虐之極的殺人命令,但五姓黎民因之而喪生的,已是不計其數了。

其時元朝虐政,説之不盡。單以元順帝至元三年這一年中而言,正史上便有這樣的記載︰

\qyh{}二月庚子,以廣東蛋戸四萬戸賜巴延。」

\qyh{}四月癸酉,禁漢人、南人、高麗人不得執持軍器,有馬者拘入官。」

\qyh{}是月詔︰禁漢人、南人不得習學蒙古、色目文字。」
\footnote{\footnotefon{}色目即西方諸國文字,南人指前宋朝百姓。}

\qyh{}五月辛丑,民間言朝廷拘刷童男童女,一時嫁娶殆盡。」

\qyh{}是歳,巴延奏請殺張、王、劉、李、趙五姓漢人。」
\footnote{\footnotefon{}以上見元史、續資治通鑑二百零七巻。}

一天之間,便將四萬家好好的百姓派給一個大臣做奴隸,漢人只要有馬便充公,擕帶兵器便殺頭,家中有童男童女,要趕快使之完婚,方得安心,民不聊生之情,可想而知。

常遇春不敢多留,落荒而走,行了數里,遇到一個樵子,問起蝴蝶谷的所在,那樵子却搖頭不知。常遇春知道胡青牛隱居之處便在左近,當下耐心緩緩尋找。一路上嫣紅奼紫,遍山遍野都是鮮花,春光爛漫已極,但兩人想起適纔的慘狀,那有心情來賞玩風景?轉了幾個彎,却見迎面一塊山壁,路途已絶,正没作理會處,只見幾隻蝴蝶,從一排花叢中鑽了進去。無忌道︰「那地方既叫蝴蝶谷,咱們且跟著蝴蝶過去瞧瞧。」常遇春道︰「好!」也從花叢中鑽了進去。過了花叢,地下出現一條草徑,常遇春行了一程,但見蝴蝶越來越多,或花或白、或黑或紫,翩翩飛舞。二人鼻中都聞到一陣芬芳馥郁的花香,這時沿途所見花草,與尋常所見的已是大不相同。蝴蝶也不畏人,飛近時便在常張二人的頭上、肩上、手上停留。二人知道已進入蝴蝶谷中,心情都感振奮。行到過午,只見一條清溪旁結著七、八間茅屋,茅屋前後左右,都是一塊塊花圃。常遇春走到屋前,恭恭敬敬的説道︰「弟子常遇春叩見胡師伯。」

過了一會,屋中走出一名僮児,説道︰「請進。」常遇春背負無忌,走進茅屋,只見廳側一個神清骨秀的中年人,正在瞧著一名僮児煽火煮藥,滿廳都是奇異的藥草之氣。常遇春將無忌放在椅上,跪下磕頭,道︰「胡師伯好。」

無忌心想,那中年人定是馳名天下的神醫、人稱「蝶谷醫仙」的胡青牛了。他向常遇春點了點頭,道︰「周子旺的事,我都知道了。那也是命數使然,想是韃子氣運未盡,本教未至光大之期。」他伸手在常遇春腕脈上一搭,解開他胸口衣服瞧了瞧,説道︰「你是中了番僧的『截心掌』,本來算不了什麼,只是你中掌後使力太多,寒虛攻心,治起來多花些功夫。」又伸掌在他週身穴道上拿捏了一週。

胡青牛忽道︰「昨晩你跟誰動手了?是武當派的人麼?」常遇春道︰「没有啊?」胡青牛在他雙腿之旁又摸了摸,臉一沉,説道︰「遇春,你我七八年没見了,一見面便向師伯説謊,你的傷我不能治,快給我請出去吧!」常遇春急道︰「胡師伯,我怎敢跟你老人家説謊?確實昨晩没跟人動手。我半點力氣也使不出來,便是想動手也不能啊。」胡青牛道︰「你雙腿『環跳穴』昨晩明明被人點過,用的是武當派手法,時間是在子丑之交。」常遇春啞然失笑,道︰「啊,那是我自己點了自己穴道。」於是將林中夜鬥這會事簡略説了。胡青牛聽常遇春説上了無忌的當以致自打穴道,向無忌看了兩眼,及至聽到説彭和尚被丁敏君刺瞎右眼,連連歎息,説道︰「彭瑩玉和尚是本教傑出好漢子,跟我們雖不同宗,但實是個難得的人材。當時若能立刻醫治,他這右眼或能復明,現下隔了這許多時候,那是無法可施了。」轉頭問無忌道︰「這武當派的打穴之法,你是從那裡學來的?」常遇春道︰「師伯,他原是武當派張五俠的孩子。」

胡青牛一怔,臉蘊怒色,道︰「他是武當派的?你帶他到這裡來幹什麼?」常遇春於是將如何保護周子旺的子女逃命、如何在漢水中爲蒙古官兵追捕而得張三丰相救等情,一一説了,最後道︰「弟子蒙他太師父大恩,求懇師伯破例,救他一救。」胡青牛冷冷的道︰「你倒慷慨,會作人情,哼,張三丰救的是你,又不是我。你見我幾時破過例來?」常遇春跪在地下,連連磕頭,説道︰「師伯,這位小兄弟的父親不肯出賣朋友,甘願自刎,是個響{\upstsl{噹}}{\upstsl{噹}}的好漢子。便是他自己,年紀雖小,也是豪氣過人,實在是個好人。」胡青牛冷笑道︰「好人?天下好人有多少,我治得了這許多?他不是武當派倒也罷了,既是名門正派中的人物,又何必來求我這種邪魔外道?」常遇春道︰「張兄弟的母親,便是白眉鷹王殷教主的女児,他有一半也算是本教中人。」

胡青牛聽到這裡,心意稍動,道︰「哦,你起來,他是白眉教殷素素的児子,那又是不同。」他走到無忌身前,溫言道︰「孩子,我向來有個規矩,決不跟自居名門正派的俠義療傷治病。你母親既是我教中人,你須得答允我一句話,待你傷愈之後,便投奔你外祖父白眉鷹王殷教主去,此後身入白眉教,不得再算是武當派的弟子。」無忌尚未回答,常遇春道︰「師伯,那可不行。張三丰張眞人言語説明在先,他跟我言道︰『胡先生決不能勉強無忌入教,倘若當眞治好了,咱武當派也不領貴教的情。』」胡青牛雙眉豎起,怒氣勃發,尖聲道︰「哼,張三丰是什麼東西?他如此瞧不起咱們,我幹麼要幫他治傷?孩子,你自己心中打的是什麼主意?」無忌知道自己體内陰毒散入五臟六腑,連太師父這等深厚的功力,也是束手無策,自己能否活命,全看這位神醫肯不肯施救,但太師父臨行時曾諄諄叮囑,決不可陥身魔教,致淪於萬劫不復的境地。雖然魔教到底壞到什麼田地,爲何太師父及衆師伯叔一提起便深惡痛絶,他實是不大了然,但他對太師父崇敬無比,深知他對自己愛如親孫,所言決計不錯,心道︰「寧可他不肯施救,我毒發身死,也不能違背太師父的教誨。」於是朗聲説道︰「胡先生,我媽媽是白眉教的香主,我想白眉教也是好的。但太師父曾跟我言道,決計不可身入魔教。我既答允了他,大丈夫豈可言而無信?你不肯給我治傷,那也無法。要是我貪生怕死,勉強聽從了你,那麼你治好了我,也不過讓世上多一個不信不義之徒,又有何益?」

\chapter{過目難忘}

胡青牛心下冷笑︰「這小鬼大言炎炎,裝出一副英雄好漢的模樣,我眞的不給他醫治,瞧他是不是跪地相求?」便道︰「他既決意不入本教,遇春,你將他背了出去,我胡青牛門中,怎能有病死之人?」常遇春素知這位師伯性情執拗異常,自來説一不二,他既不答應,再求也是枉然,於是向無忌道︰「小兄弟,魔教雖和名門正派的俠義人物其道不同,但自大唐以來,世世代代均有雄傑之士。何況令外祖父是白眉教的教主,令堂是教中香主,你答應了我胡師伯,他日張眞人跟前,一切由我承擔便是。」

無忌道︰「好,常大哥請你在我背上第八根脊椎骨和第十三根脊椎骨上,用指節敲打幾下。」常遇春喜道︰「好!」依言敲擊了三下,無忌雙足登時便能動彈。他站了起來,説道︰「常大哥,你心意已盡,我太師父也決不會怪你。」説著昂然走出門去。常遇春吃了一驚,忙道︰「你到那裡去?」無忌道︰「我若死在蝴蝶谷中,豈不壞了『蝶谷醫仙』的名頭?」説著展開輕身功夫,疾馳而去。胡青牛冷笑道︰「『見死不救』胡青牛,天下馳名,倒斃在蝴蝶谷中『牛舍』之外的,又那止你這娃娃一人?」常遇春也不去聽他説些什麼,急忙拔步追了出去。兩人雖都身上有傷,但究竟常遇春傷勢較輕,脚步較大,追上了無忌,一把抓住,將他抱了回來。無忌雙手不能揮動,無法掙扎。

常遇春氣喘吁吁的回進茅舍,説道︰「胡師伯,你定是不肯救他的了,是不是?」胡青牛笑道︰「我有一個外號叫作『見死不救』,難道你不知道?却來問我。」常遇春道︰「我身上的傷,你却是肯救的?」胡青牛道︰「不錯。」常遇春道︰「好!弟子曾答應過張眞人,要救活這位兄弟,此事決不能讓正派中人説一句我魔教弟子言而無信。弟子不要你治,你治了這位兄弟吧。咱們一個換一個,你也没吃虧。」胡青牛正色道︰「你中了這『截心掌』後,七天之内,若能求到第一流的良醫,可以痊癒。過了七天,只能保命,武功從此不能恢復。十四天後再無良醫著手,傷發而死。」常遇春道︰「這是師伯你老人家見死不救之功,弟子死而無怨。」無忌叫道︰「我不要你救!不要你救!」轉頭向常遇春道︰「常大哥,你當我張無忌是卑鄙小人麼?你拿自己性命來換我一命,我便是活著,也是無味。這簡直是豈有此理!」

常遇春是個豪氣干雲的漢子,也不再跟他多辯,解下身上帶子,將無忌牢牢的縛在椅上。無忌急道︰「你不放我,我可要罵人啦!」見常遇春不理,竟是把心一橫,大罵︰「見死不救胡青牛,當眞是如笨牛一般,連畜生也不如。魔教中有了這種没半點人性的東西,你還想小爺入教,眞是放你娘的狗臭屁!你祖宗十八代也不知積下了什麼陰功,生下你這種豬狗一般的畜生來。」他口齒極是伶俐,越罵越是厲害,花樣翻新,罵到後來,胡青牛和常遇春聽著,覺得實是生平聞所未聞之奇。

常遇春將他縛好,道︰「胡師伯,張兄弟,告辭了。我這便尋醫生去!」胡青牛冷冷的道︰「安徽境内,没一個眞正的良醫,可是你七天之内,未必能出得安徽省境。」常遇春哈哈一笑,説道︰「有『見死不救』的師伯,便有『豈不該死』的師侄!」説著大踏步走出門去。

無忌大叫道︰「胡青牛,你若不將常大哥治好,終有一天,教你死在我的手裡。我\dash{}我\dash{}」心中一急,竟自暈了過去。胡青牛哼了一聲道︰「蝴蝶谷中,也不爭多死你一人。你何苦去死在外邉?」隨手拿起桌上的半段鹿茸?呼的一聲,擲了出去,正中常遇春膝彎。

這一下正中穴道,常遇春咕{\upstsl{咚}}一聲,摔倒在地,再也爬不起來了。胡青牛此人脾氣古怪之極,他若是不肯施救,不論你如何苦苦哀求,如何動之以情、脅之以威,他總是見死不救,但若他有意救治了,便算再厲害的得罪於他,他也是要治好了人才罷。可是無忌最後一句話却使他深印於心︰「你若不將常大哥治好,總有一天,教你死在我的手裡。」他見無忌年紀雖小,但英氣勃勃,實非常物,況且又是張三丰愛徒之子,日後若是糾纏上了自己,當眞是個大大的禍胎。他是個極工心計之人,盤算良久,打定了主意︰「兩個人都不救,蝴蝶谷中多添兩個怨鬼,何足道哉?」

他走將過去,解開無忌身上綁縛,抓住了他雙手手腕,待要將他摔出門去,由得他自生自滅,著手之處,只覺無忌的脈膊跳動古怪無比。

胡青牛吃了一驚,再用心搭脈,更是驚異,心道︰「難道他小小年紀,居然已打通了奇經八脈?我苦修數十年也不能辦到之事,一個十餘歳的孩童竟能打通?哦,那定是張三丰這老不死的怪道愛憐稚子,不惜耗費功力,替他打通了。」伸掌在他『靈台穴』上一按,試一運氣,果然奇經八脈暢通無阻。再解開他上下衣裳,週身細看一遍,試按他丹田、胸口、頂門諸處,心下已是了然,冷笑道︰「張三丰弄巧成拙,愛之適足以害之。這孩童奇經八脈不通,尚有可救,如今陰毒散入五臟六腑,如非是神,才能救得他的性命。嘿嘿,人道武當派張三丰武功神通,依我看來,實是愚不可及。」

過了半晌,無忌悠悠醒轉,只是胡青牛坐在對面椅中,望著藥爐中的火光,凝思出神,常遇春却躺在門外草徑之中。三個人各想各的心思,誰也没有説話。

原來胡青牛畢生潛心醫術,任何疑難怪症,都是手到病除,這纔博得了「醫仙」兩字的外號,「醫」而稱到「仙」,可見其神乎其技,非常人所能想像。但「玄冥神掌」所發寒毒,世上已是罕見罕聞,而一個中了「玄冥神掌」之人,再行打通奇經八脈,更是千載難遇。大凡精於奕者,最難得的是棋逢敵手;精於算者,遇到極深奥的算題時方始廢寢忘食,不解不休。胡青牛有心替無忌治傷,然而碰上了這等畢生再也不能重見的怪症,有如酒徒見佳釀、老饕聞肉香,怎肯捨却?尋思半天,終於想出了一個妙法︰「我先將他治好,然後將他弄死。」

可是要將無忌體内五臟六腑的陰毒驅出,當眞是談何容易。胡青牛一直思索了一個多時辰,取出十二片細小的銅片,運内力在無忌丹田下「中極穴」、頸下「天臟穴」、肩頭「肩井穴」等十二處穴道上插下。要知那「中極穴」是足三陰任脈之會,「天突穴」是陰維任脈之會,「肩井穴」手足少陽陽維之會,這十二條銅片一插下,他身上十二經常脈和經八脈便即膈斷。何謂十二經常脈?人身心、肺、脾、肝、腎,是謂五臟,再加心包,此六著屬陰;胃、大腸、小腸、膽、膀胱、三焦,是謂六腑,六者屬陽。五臟六腑加心包,共爲十二經常脈。任、督、衝、帶、陰維、陽維、陰蹻、陽蹻這八脈不係正經陰陽,無表裡配合,别道奇行,是爲奇經八脈。

無忌身上常脈和奇經隔絶之後,五臟六腑中所中的陰毒相互不能爲用。胡青牛便解開他四肢上所閉塞的穴道,然後以陳艾炙他肩頭「雲門」「中府」兩穴,再炙他自手臂至大拇指的天府、俠白、尺澤、孔最、列缺、經渠、大淵、魚際、少商各穴,這十一處穴道,屬於「手太陰肺經」每炙一處穴道,均可消減少些他深藏肺中的陰毒。這一次以熱攻寒,無忌所受的苦楚,却比陰毒發作時又是一番不同的滋味。

炙完手太陰肺經後,再炙足陽明胃經、手厥陰心包經。胡青牛下手時毫不理會無忌是否疼痛,用陳艾將他周身燒炙得處處焦黑。無忌不肯有絲毫示弱,心道︰「你想要我呼痛呻吟,我偏是哼也不哼一聲。」竟是談笑自如,跟胡青牛講論穴道經脈的部位。他雖然不明醫理,但跟謝遜學過點穴之術,各處穴道和所在却是知之甚詳。和這位當世神醫相較,無忌對穴道經脈的見識自是甚爲膚淺,但所言一涉及醫理,正是投合胡青牛所好。他一面炙艾,替無忌拔除體内陰毒,一面滔滔不絶的講論。無忌聽在心中,多半並不了然,但爲了意示「我武當派這些也懂」,往往發些謬論,與他辯駁一陣。胡青牛詳加闡述,及至明白「這小子其實一竅不通,乃是胡説八道」,已是大費了一番唇舌。可是深山僻谷之中,除了幾名燒菜煮藥的僮児以外,胡青牛無人爲伴,今日無忌到來,跟他東拉西扯的講論穴道,倒也令他頗暢所懷。

待得十二經常脈數百處穴道炙完,已是天將傍晩。僮児搬出飯菜,開在桌上,另行端了一大盤米飯青菜,拿到門外草地上給常遇春食用。當晩常遇春便睡在門外。無忌手足即能動彈,也不出聲向胡青牛求懇,臨睡時自去躺在常遇春身旁,兩人同在草地上睡了一夜,以示有難同當之意。胡青牛只作視而不見,毫不理會,心中却不免暗暗稱奇︰「這小子果是和常児大不相同。」

次日清晨,胡青牛又以半日功夫,替無忌燒炙奇經八脈的各處穴道。十二經常脈猶之江河,川流不息,奇經八脈猶之湖海,蓄藏蓄積,因之要除去奇經八脈間的陰毒,却又是爲難得多。胡青牛潛心擬了一張藥方,却邪扶正,補虛瀉實,用的却是「以寒治寒」的反治法。無忌服了之後,寒戰半日之後,精神竟是健旺得多。

午後胡青牛又替無忌針炙,無忌以言語相激,想迫得他沉不住氣,便替常遇春施治,那知胡青牛理也不理,只哈哈的道︰「我胡青牛那『蝶谷醫仙』的外號,説來有點名不副實,旁人叫我『見死不救』,我纔喜歡。」其時他正用金針刺無忌腰腿之間「五樞穴」,這一穴乃是少陽和帶脈之會,在同水道旁一寸五分。無忌道︰「人身上這個帶脈,可算得最爲古怪了。胡先生,你知不知道,有些人是没有帶脈?」胡青牛一怔,道︰「瞎説!怎能没有帶脈?」無忌原是信口胡吹,説道︰「天下之大,無奇不有。何況這帶脈我看也没有多大用處。」胡青牛道︰「帶脈比較奇妙,那是不錯的,但豈可説它無用?世上庸醫不明其中精奥,針藥往往誤用。我著有一部『帶脈論』,你拿去一觀便知。」説著走入内室,取了一部薄薄的黃紙手抄本出來,交給無忌。

無忌翻開一頁來,只見上面冩道︰「十二經和奇經七脈,皆上下周流。惟帶脈起少腹之側,季肋之下,環身一周,絡腰而過,如束帶之狀。衝、任、督三脈,同起而異行,一源而三岐,皆絡帶脈\dash{}」跟著評述古來醫書中的錯誤之處,「十四經發揮」一書中説帶脈只四穴,「針炙大成」一書中説帶脈凡六穴,其實共有十穴、其中兩穴忽隱忽顯、若有若無,最爲難辨。無忌一路翻閲下去,暗暗記誦,忽然想起那少林弟子陳友諒對付太師父的故事來。胡青牛的文章有條有理,剖析明白,何況文采斐然,音調鏗鏘。比之記誦武功祕訣,那是易上十倍。無忌看了一遍,還給胡青牛,搖頭道︰「這部書我看過的。我太師父在三十歳時著過一部『初學帶脈入門淺説』,跟你這部書一模一樣。也不知是你抄我太師父的,還是我太師父抄你的。」

胡青牛一呆,不禁大怒,心道︰「我還只五十一歳,你説張三丰三十歳時著過這部醫書,他今年已過百齡,那是七十多年以前所撰,自是我抄他的了。我這部『帶脈論』精微深奥,處處道前人所未言,你却説和張三丰的什麼『初學帶脈入門淺説』一般無異,又是『初學』,又是『入門』,又是『淺説』。這小子也太過混帳。」怒氣勃發之下,故意下重手一針刺在他穴道之旁,登時鮮血長流。無忌痛得險些児叫出聲來,但總算及時忍住,微微一笑,道︰「你若是不認,我便將太師父那部『初學帶脈入門淺説』背給你聽聽。」胡青牛道︰「好,你若背錯一字半句,立時取你性命。」

無忌在冰火島上之時,從五歳起始,便給謝遜逼著背書,稍有錯誤,謝遜便是老大耳括子打將過來,一直背到十歳,因此這記誦功夫,可説習練有素,乃是他的拿手本領。但胡青牛説只要背錯一字半句,便要取他性命,這怪醫性子奇特無比,説得出做得到,自己若是背錯了,他盛怒之下,難保不便下殺手,不由得暗自後悔,這玩笑實在開得太過兇險。但事已如此,已無退縮餘地,於是朗聲背道︰「十二經和奇經七脈,皆上下周流。惟帶脈起小腹之側\dash{}」一路背將下來,直至篇末,竟是一字不誤。

胡青牛聽得呆了,心道︰「此人過目不忘,無異是天下無雙的奇才。」他却不知少林寺中尚有一個少年陳友諒,記誦的本事決不在無忌之下,當即讚道︰「好聰明,好聰明!」替他帶脈上的十大穴道,都刺過了金針。待他休息了片刻,有心再試他一試,説道︰「我另有一部『子午針炙經』,不知張三丰是否也抄襲了去?」從室内取了一部厚達十二巻的手書醫經出來。

無忌翻開一看,只見每一頁上都是密密麻麻的冩滿了蠅頭小楷,穴道部位、藥材份量,下針的時刻深淺,無一不是極難記憶。他心念一動︰「這十二巻醫經,便是從頭至尾看一遍,也非三四日可畢,如何能在一時三刻内記得住?我且査閲一下,且看有無醫治常大哥身上傷勢的法門?」於是翻到了第九巻「武學篇」中的「掌傷治法」,但見紅沙掌、鐵沙掌、毒沙掌、綿掌、開山掌、破碑掌\dash{}各種各樣的掌力傷人的徵狀、急救、治法,無不備載,待看到一百八十餘種掌力之後,赫然出現了「截心掌」。無忌大喜,當下細細讀了一遍,文中對「截心掌」的掌力論述甚詳,但治法却説得極爲簡略,只説「當從『紫宮』『中庭』『関元』『天池』四穴著手,御陰陽五行之變,視寒、暑、燥、濕、風五天候,應傷者喜、怒、憂、思、恐五情下藥。」

須知中國醫道,變化多端,並無定規,同一病症,醫者常視寒暑、晝夜、剝復、盈虛、終始、動靜、男女、大小、内外\dash{}種種牽連而定醫療之法。無忌將這治法看了幾遍,心想︰「眼下設法治好常大哥要緊,不必徒逞口舌之快,而得罪這位神醫。」那「掌傷治法」的最後一項,乃是「玄冥神掌」,述了傷者徵狀後,在「治法」二字之下註著一字︰「無」。

無忌將醫經合上,恭恭敬敬放在桌上,説道︰「胡先生武功不及我太師父,我太師父醫道不及胡先生,這部『子午針炙經』博大精深,我太師父也著不出來。但説到醫治掌傷,胡先生所學,却也脱不出我太師父的圏子。」於是將紅沙掌、鐵沙掌等等百餘種掌傷,絲毫不漏的背了一遍,最後道︰「晩輩中了玄冥神掌,我太師父無法可治,原來胡先生也是束手無策。」

胡青牛冷笑道︰「你也不用激我。你且瞧我是否束手無策?不過我治得好你身上的掌毒,你的性命却未必久長。」

無忌雖是聰明絶倫,却也不明白胡青牛這句話的用意,原來是説將無忌身上的陰毒治好,一顯自己身手之後,便即下手將他殺死,以符自己決不替教外人治病療傷的規矩。無忌其時一心一意,只盼能治好常遇春身上之傷,便道︰「既是我命不久長,那麼拜讀一下胡先生這部曠古未有的『子午針炙經』,想亦無礙。」胡青牛心想︰「反正你決不能活著走出我蝴蝶谷,就是將我的醫術盡數記在心中,也不過是帶入黃泉地府,去替閻王判官治病。」便點頭道︰「我這些醫書,你儘管看好了。」

要知胡青牛雖然生性古怪,但學識淵博,見解高超,實是醫中不世出的才子奇人。只是他身入魔教,對官紳富商、士大夫等人物固是深痛絶惡,於名門正派的武林人士,也有憎意甚深,脾氣不免越來越是孤僻。可是他一身絶學,空揚大名於外,却無人可共同研討,更無一個傳人,荒山獨處,孤芳自賞,原是大有寂寞之意,難得無忌到來,雖然是個醫道一竅一通的孩童,但聰明過人,又佩服他的醫學著作,心中也不免歡喜。

於是無忌潛心醫書,日以繼夜,廢寢忘食鑽研,不但將胡青牛的十餘種著作都翻閲過,其餘「黃帝内經」「華陀内昭圖」「王叔和脈經」「宋徽宗皇帝勒撰聖總錄」「孫思邈千金方」「千金翼」「王燾外台祕要」等等醫學經典,都亂翻一通。他是一意在尋找醫治常遇春的方法,胡青牛却道他看不懂自己精奥的著作,硬充好漢,不肯詢問,却從書籍中去求解釋。

其實胡青牛也是個才智過人之士,只要稍加深思,便該能猜到無忌的用意,但他見無忌用心鑽研自己畢生心血之所聚的書作,心下已自歡喜,也不再想及其他了。

如此過了數日,無忌没頭没腦的亂讀一通,雖是記了一肚皮的醫理藥方,但中國醫道何等精妙,豈能在數天之内明白?屈指一算,到得蝴蝶谷來已是第六日。胡青牛曾説常遇春之傷,若在七日之内得遇良醫,可以痊癒,否則縱然治好,也是武功全失。他在門外草地上躺了六天六晩,到了這日,却又下雨來。胡青牛眼見他處身泥潭積水之中,仍是毫不理會。無忌心中大怒,暗想︰「我所看的每一本醫書中,除了你自己的著作之外,每一部書都道,醫者須有濟世惠民的仁人之心,你空具一身醫術,是這等見死不救。」

到得晩上,那雨下得更加大了,同時電光閃閃,一個霹靂跟著一個霹靂。無忌把牙一咬,心道︰「便是將常大哥醫壞了,那也無法可想。」當下從胡青牛的藥櫃中取了八根金針,走到常遇春身畔,説道︰「常大哥,這幾日中小弟竭盡心力,研讀胡先生的醫書,雖是不能通曉,但時日緊迫,不能再行拖延。小弟只有冒險給常大哥下針,若是不幸出了岔子,小弟也不獨活便是。」常遇春哈哈笑道︰「小兄弟説那裡話來?你快快給我下針施治。若是天幸得救,也好羞我胡師伯一羞。倘若兩針三針將我扎死了,也好過在這汚泥坑中活受罪。」

無忌雙手顫抖,細細摸準常遇春的穴道,將一枚金針,從他「関元穴」中刺了下去。他未練過針炙之法,這施針的手法,自是極爲拙劣。胡青牛的金針又是軟金所製,非有深湛的内力,不能使用,無忌用力稍大,那針登時彎了,再也刺不進去,只得拔將出來又刺。自來針刺穴道,絶無出血之理,但給他這麼毛手毛脚的一番亂攪,常遇春「関元穴」上登時鮮血湧出。要知那「関元穴」位處小腹,乃是人身的要害,這一出血不止,無忌心下大急,更是手足無措起來。

忽聽得身後一陣哈哈大笑之聲。張無忌回過頭來,只見胡青牛雙手負在背後,悠閒自得,笑嘻嘻的瞧著自己弄得兩手都是染滿了鮮血。無忌急道︰「胡先生,常大哥『関元穴』流血不止,那怎麼辦啊?」胡青牛道︰「我自然知道怎麼辦,可是何必跟你説?」無忌昂然道︰「現下咱們也一命換一命,請你快救常大哥,我立時死在你的面前便是。」胡青牛冷冷的道︰「我説過不治的人,總之是不治的了。胡青牛不過是見死不救,又不是催命的無常,你死了於我有什麼好處?便是死十個張無忌,我也不會救一個常遇春。」

無忌知道再跟他多説徒然白費時光,心想這金針太軟,我是用不來的,這時候也没地方去尋找别種金針,便是銅針鐵針也尋不到一枚,略一沉吟,去折了一根竹枝下來,用小刀削成幾根光滑的竹籤,更不細想,便在常遇春「紫宮」「中庭」「関元」「天池」四處穴道中扎了下去。這竹籤硬中帶有韌力,刺入穴道後居然並不流血。過了半晌,常遇春嘔出幾大口黑血來。

無忌不知這是自己亂刺一通之後使他傷上加傷?還是竹針見效,逼出了他體内的餘血?回頭看胡青牛時,見他雖是一臉譏嘲之色,但也隱然帶著幾分讚許。無忌知道這幾下竹針刺穴並未全錯,於是進去亂翻醫書,窮思苦想,擬了一張藥方。他雖從醫書上,知道了某藥可治某病,但到底生地、柴胡是什麼模樣,牛膝、熊膽是怎樣的東西,却是一件不識得,當下硬著頭皮,將藥方交給煎藥的僮児,説道︰「請你照方煎一服藥。」

那僮児將藥方拿去呈給胡青牛看,問他是否照煎。胡青牛鼻中哼一聲,道︰「可笑,可笑。」冷笑三聲,道︰「你照煎便是。他服下不死,算他命大。」無忌搶過藥方,將幾種藥味的份量都減少了一二錢。那僮児便依方烹藥,煎成了濃濃的一碗。無忌端到常遇春口邉,含泪道︰「常大哥,這服藥喝下去是吉是凶,小弟委實不知\dash{}」常遇春笑道︰「妙極,妙極,這叫作盲醫治瞎馬。」閉了眼睛,仰脖子將一大碗藥喝得涓滴不存。

這一晩常遇春腹痛如刀割,不住的嘔血,無忌在雷電交作的大雨之中服侍著他,直折騰了一夜。到得次日清晨,大雨止歇,常遇春嘔血漸少,血色也自黑變紫,自紫變紅。常遇春喜道︰「小兄弟,你的藥居然吃不死人,看來我的傷竟是減輕了好多。」無忌大喜,道︰「小弟的藥還使得麼?」常遇春笑道︰「先父早料到有今日之事,是以給我取了個名字,叫作『常遇春』,那是説常常會遇到你這妙手回春的大國手啊。只是你的藥方似乎稍嫌霸道,喝在肚中,便如幾十把小刀子在亂削亂砍一般。」無忌道︰「是。看來份量是重了些。」

其實他下的藥量豈止「稍重」,直是重了好幾倍,又無别種中和調理之藥爲佐,一味的急衝猛攻。他雖然從胡青牛的醫書中找到了對症的藥物,但用藥的「君臣佐使」之道,却是全不通曉,若非常遇春體質強壯,雄健過人,早已抵受不住而一命嗚呼了。

胡青牛盥洗已畢,慢慢踱將出來,見常遇春胡青牛臉色紅潤,不禁吃了一驚,暗想︰「一個聰明大膽,一個體魄壯健,這截心掌的掌傷,倒給他治好了。」當日無忌又開了一張調理補養的方子,什麼人參鹿茸首烏茯苓,各種大補的藥物,都開在上面。胡青牛家中所藏的藥材,無一不是珍品,藥力特别渾厚。如此調補了六七日,常遇春竟是神采奕奕,武功盡復舊觀,向無忌道︰「小兄弟,我身上的掌傷已然痊癒,你每天在這門外陪我露宿,也不是道理。咱們就此别過。」

這一個多月之中,無忌與他共當患難,相互的捨命全交,已是結下了生死好友,一旦分别,自是戀戀不捨,但想常遇春終不能長此相伴自己,只得含泪答應。常遇春道︰「兄弟,你也不須難過,三個月後,我再來探望。其時如你身上寒毒已然去盡,便送你去武當和你太師父伯相會。」他走進茅舍,向胡青牛拜别,説道︰「弟子傷勢痊可,雖是張兄弟動手醫治,但全憑師伯醫書指引,服食了師伯不少珍貴的藥物。」胡青牛點點頭,道︰「那算不了什麼。你傷勢已愈,所減者也不過是三十年的壽算。」

常遇春不懂,問道︰「什麼?」胡青牛道︰「依你體魄而言,至少可活過八十歳。但那小子用藥有誤,下針時手勁方法不對,以後再逢陰雨雷電,你便會週身疼痛,大槩在五十歳上,便要一命嗚呼了。」常遇春哈哈一笑,慨然道︰「大丈夫濟世報國,若能建立功業,便四十餘歳亦已綽然有餘,何必五十?要是碌碌一生,縱然年過百歳,亦是徒然多耗糧食而已。」胡青牛點了點頭,便不再言語了。

無忌一直送到蝴蝶谷口,纔和他揮泪作别。無忌心下暗暗立志︰「我胡裡胡塗的醫錯了常大哥,害得他要損三十年壽算。他身子在我手中受損,難道日後便不能在我手中受益?無論如何?我要設法醫得他和以前一般無異。」

自此胡青牛每日替無忌施針用藥,消散他體内的陰毒。無忌却孜孜不倦的閲讀醫書,記憶藥典,遇有疑難不明之處,便向胡青牛請教。這一著大投胡青牛之所好,竟是將畢生所學,傾囊以授,有時無忌提一些奇問怪想,也頗能觸發胡青牛以前未想到過的許多途徑。他初時打算將無忌治愈之後,便即下手將他殺死,但這時覺得無忌一死,谷中便少了這唯一可以談得來的良伴,用藥之際,竟是一味的拖延,不想他早癒早死。

如此過了數月,有一日胡青牛猛地發覺,無忌無名指外側的「関衝穴」、臂彎上二寸的「清冷淵」、眉後陥中的「絲竹空」等穴道,下針後竟是半點消息也没有。原來這些穴道均屬「手少陽三焦經」,那三焦分上焦、中焦、下焦,爲五臟六腑的六腑之一,自來醫書之中,説得神而明之,難以捉摸。
\footnote{\footnotefon{}〔按〕中國醫學中的三焦,據醫家言,當即指人體的各種内分泌而言。今日科學昌明,西醫對内分泌之運用和調整,仍是所知不多,自來即爲醫學中一項極困難的部門。}
胡青牛潛心苦思,用了許多巧妙的方法,始終不能將無忌體内散入三焦的陰毒逼出。十多日中,累得他頭髮也白了十餘根,這一日忍不住嘆道︰「你太師父武功雖高,於醫道却是太過外行,他愛你適足以害你,當你中了玄冥神掌後,還來助你打通奇經八脈,眞是累死了人。」

無忌搖頭道︰「不是太師父給我打通的。」他和胡青牛相處數月,覺得他爲人固是怪僻,却非奸險陰惡之徒,於是將自己身世,以及如何在少林寺中學習「少林九陽功」的經過一一説了。胡青牛沉思半晌,突然伸手一拍大腿,説道︰「無忌,那少林僧是有意害你也!」無忌吃了一驚,道︰「我跟他素不相識,他何故害我?」胡青牛道︰「{\upstsl{嗯}},這事果然奇怪。你將上了少室山後的一切情形,從頭至尾的説給我聽。」

無忌對這回事記得清清楚楚,將太師父和空聞、空智等人的對答,少林寺中所見所聞,毫不遺漏的説了。胡青牛背負雙手,在室中踱來踱去,走了數圏,突然大聲道︰「那少林僧定是有意害你,這一節我決不料錯,你太師父不明醫理,又是誠信待人,是以没疑心到這一點。那少林僧圓眞既是精修「少林九陽功」,又能助你打通奇經八脈,内功豈是泛泛?他雙掌跟你掌心一碰,便當知你身有陰毒。但仍替你打通經脈,那不是存心害人麼?」

\chapter{精究醫理}

張無忌道︰「可是他隔牆伸掌過來之時,已是有意助我打通經脈,那時未必已知曉我身中玄冥神掌。」胡青牛搖頭道︰「這圓眞何以要害死你,此時我是猜想不透。你説跟他素不相識,他絶無害你之理,但你習了他的少林九陽功,神功外傳,單是爲了這件事,便足足害死十個張無忌有餘。」無忌道︰「我太師父言道︰少林派是武林中名門正派之首,代出高僧,領袖武林垂千百年。我想少林寺中縱然有幾個心胸偏狹之輩,但決不致於行事如此卑鄙?何況我太師父以『太極十三式』及『武當九陽功』和之交換,只有少林派佔了我武當派的便宜。」

胡青牛冷笑道︰「名門正派便怎樣了?你的父親母親,難道不是給名門正派中的人活活逼死麼?他們自以爲名門正派,對被他們視爲邪魔外道之人,下手狠辣,毫不容情,正派中的未必都是好人,魔教中的也未必都是壞人。」這幾句觸動了無忌的心事,他想起武當山上父母伏劍而死,在場逼迫的固然大都是名門正派之士,少林、崑崙兩派爲首,崆峒、峨嵋爲衆。便是武當派中的諸師伯叔,也是眼睜睜的瞧著父母自刎身亡,雖有哀痛之情,但在各人心中,却均認爲死得應該。這番念頭他一直暗藏心内,不敢在太師父和衆師伯面前提起,此時胡青牛猛地將他心底深處最隱祕的想頭説了出來,他全身一震,不由得放聲大哭。

胡青牛冷冷的道︰「世事本是如此,你碰到一件事便哭,若是不死,日後有得你哭的呢。」無忌驀地止聲,擦乾了眼泪。胡青牛又道︰「你由頭至尾没見到他面目,焉知不是相識之人?一個人語聲可以假裝,便是容貌,變換又有何難?他不肯跟你見面,此中便有蹺蹊。你説他無緣無故,決不致下手害你。你可知我早便想害死你嗎?只因你的病生得古怪,我才盡心竭力的救治,我心中早就打定了主意,一等治好,便要將你弄死。」無忌打了個寒噤,聽他説來輕描淡冩,似乎渾不當一回事,但知他既説出了口,決計不再輕易變通,嘆了口氣,説道︰「我看我身上的陰毒終是驅除不掉,你不用下手,我自己也會死的。這世上之人,似乎只盼别人都死光了,他纔快活。大家學武練功,不都是爲了打死别人麼?」

胡青牛望著庭外天空,出神半晌,幽幽説道︰「我少年之時潛心學醫,立志濟世救人,可是越救越不對。我救活了的人,反過面來狠狠的害我。一個身上受了一十七處刀傷、非死不可的少年,我三日三晩不睡,耗盡心血救治了他,和他義結金蘭,情同手足,那知後來他却殺了我的親妹子。你道此人是誰?他今日是名門正派中鼎鼎大名的首腦人物啊。」

無忌見他臉上肌肉扭曲,神情極是苦痛,心中油然而起憐憫之意,暗想︰「原來他生平經歷過不少慘事,這纔養成了『見死不救』的性子。」問道︰「這個忘恩負義、狼心狗肺的人是誰?你怎麼不去找他報仇?」胡青牛道︰「我妹子臨死之時,却要我立下重誓,決計不能找他報仇,甚且此人若是遇到危難,要我竭力救他。我本來不肯答應,但我妹子不聽到我立誓,死不瞑目。唉,我苦命的妹子,她\dash{}她的心地可是太好了。我兄妹倆自幼父母見背,相依爲命。她臨死時如此求我,我怎能不依?」

他説到這裡,眼中泪光瑩然。無忌心想︰「他其實並非冷酷無情之人。想是他的義兄弟和他妹子不是夫妻,便是情侶了。」胡青牛突然厲聲喝道︰「今日我説的話,從此不得跟我再提,若是洩漏給旁人知曉,我治得你求生不得,求死不能。」無忌本想狠狠挺撞他幾句,但忽地心軟,覺得此人實在甚是可憐,便道︰「我不説便是。」胡青牛摸了摸無忌的頭髮,嘆道︰「可憐,可憐!」轉身進了内堂。

胡青牛自和張無忌這日一場深談,又察覺他散入三焦的陰毒總歸難以驅除,即是以至高至深的醫術與他調理,亦不過多延數年之命,竟對他變了一番心情。雖然自此再不向他吐露自己的身世和心事,但見無忌善解人意,山居寂寥,大是良伴,一是空閒,便指點他醫理中的陰陽五行之變,把脈針炙之術。張無忌潛心鑽研,學得極是用心。胡青牛見他悟心奇高,對「黃帝蝦蟆經」「西方子明堂炙經」「太平聖惠方」「瘡傷經驗全書」等醫學,尤有心得,不禁嘆道︰「以你的聰明才智,又得逢我這個肯傾囊相授的明師,不到二十歳,便能和華陀、扁鵲比肩,只是\dash{}唉,可惜可惜。」

他言下之意,是説等你醫術學好,壽命也終了,這般苦學,又有何用?無忌心中,却另有一番主意,他決意要學成回春之術,待見到常遇春時,將他大受虧損的身子治得一如原狀。

谷中安靜無事,歳月易逝,如此過了兩年,無忌已是一十四歳。這兩年之中,常遇春曾來看過他幾次,説張三丰知他體内陰毒難除,命他便在蝴蝶谷多住些日子,直至痊癒爲止,無忌問起谷外消息,常遇春説道近年來蒙古人對漢人的欺壓日甚一日,衆百姓衣食不週,群盜並起,眼見天下大亂,同時江湖上名門各派和魔教邪派之間的爭鬥,也是一天厲害過一天,雙方死傷均重,冤仇越結越深。

常遇春每次來到蝴蝶谷,均是稍住數日即去,最後一次來時,無忌已是醫術大進,細心替他診脈,擬了一張方子,要他照方長服,定可健身保元。常遇春説了聲︰「多謝!」便將藥方隨手收在懷裡。

這一次常遇春和胡青牛相見,兩人在内室中閉門長談,直至深夜,仍不安睡,無忌暗自奇怪,心想常大哥和他這位胡師伯向來不睦,今番如此長談,想是他魔教中發生了什麼大事,自己並非魔教中人,也不便多問。次晨常遇春别去。無忌送到谷口,常遇春道︰「兄弟,這幾日中,胡師伯有一個極厲害的對頭要來找他。我本想帶你出去暫避幾時,可是胡師伯言道,那對頭決計奈何不了他,不必畏懼。但你一切得小心在意。」無忌好奇心起,問道︰「是什麼樣的對頭?」常遇春道︰「這個我也不知。我在途中得到了消息,趕來向胡師伯報訊。兄弟,胡師伯老謀深算,他説不要緊,定有十足把握,只是我總有點放心不下。」

無忌見他對自己如此関切,心中感動,兩人説了好一陣話,這纔分别。無忌回到茅舍,只見胡青牛一如平日,毫無應付大敵的舉措,無忌倒是有些沉不住氣,幾次想問,但一開口,話題便被胡青牛截斷。無忌知他不願説及此事,也就不敢再問。

如此過了六七日,别説没有敵人上門尋仇生事,便連來求醫的鄕民也無一個。這天晩上,無忌讀了一會王好古所著的醫書「此事難知」,覺得腦子昏昏沉沉,甚是困倦,當即上床安睡,次日起身,便覺頭痛得厲害,正想去找些發散風寒的藥物來食,走到廳上,只見日影西斜,原來已是午後。無忌吃了一驚,心道︰「這一覺睡得好長,看來我是生了病啦。」伸手一搭自己脈膊,却無異狀,心下更是暗驚︰「莫非我體内陰毒發作,陽壽已盡?」

想去尋胡青牛時,却不見他的人影,無忌這幾日中一直提心吊膽,等待胡青牛的對頭上門,這時忽然不見了他,急忙奔出門去找尋。只見花圃中一個僮児正彎了腰在鋤草,忙問︰「先生呢?」那僮児道︰「他不在房裡麼?剛纔我還送茶進去。先生叫我别打擾他。」無忌一怔,啞然失笑︰「我這不是庸人自擾麼?到處尋遍了,却不到他房裡去找他?」

張無忌走到胡青牛房外,只見房門緊閉,想起鋤草僮児「不得打擾」的話,不敢呼喚,輕輕咳嗽了一聲。只聽胡青牛道︰「無忌,今児我身子有些不適,咽喉疼痛,你自個児讀書吧。」無忌應道︰「是。」他耽心胡青牛病勢不輕,道︰「先生,讓我瞧瞧你喉頭好不好?」胡青牛低沉著嗓子道︰「不用了。我已對鏡照過,並無大礙,已服了牛黃犀角散。」

當天晩上,僮児送飯進房,無忌跟著進去,只見胡青牛臉色憔悴,躺在床上。無忌心念一動︰「難道昨晩我大睡之時,已有對頭到來?先生雖將他逐走,但自己也受了傷?」胡青牛揮手道︰「快出去。你知我生的是什麼病?那是天花啊。」無忌看他臉上手上,果有點點紅斑,心想那天花之疾,發作時極爲厲害,調理不善,重則致命,輕則滿臉麻皮,但胡青牛醫道精湛,雖染惡疾,自無後患,既非爲敵人所傷,反倒放心。胡青牛道︰「你和僮児不可再進我房,我用過的碗筷杯碟,均須用沸水煮過,你們千萬不可混用。{\upstsl{嗯}}\dash{}」他沉吟片刻,道︰「無忌,這樣吧,你還是出蝴蝶谷去,到外面借宿半個月,免得我將天花傳給了你。」無忌忙道︰「不必。先生有病,我若避開,誰來服侍你?我好歹比這兩個僮児多懂些醫理。」胡青牛道︰「你還是避開的好。」但説了良久,無忌終是不肯。胡青牛道︰「好吧,那你決不能進我房來。」

如此過了三日,無忌晨夕在房外問安,聽胡青牛嗓子雖然嘶啞,精神倒還健旺,飯量反較平時爲多,料想無礙。胡青牛每日隔著房門報出藥名份量,那僮児便煮了藥給他遞進去。

到第四日下午,無忌坐在草堂之中,誦讀「黃帝内經」中那一篇「四氣調神大論」,讀到「是故聖人不治已病治未病,不治已亂治未亂,此之謂也。大病已成而後藥之,亂已成而後治之,譬猶渴而穿井,鬥而鑄錐,不亦晩乎?」那一段,不禁暗暗點頭,心道︰「這幾句話説得眞是不錯,口渴時再去掘井,要和人動手時再去打造兵刃,那確是來不及了。國家擾亂後再去平變,縱然復歸安定,也已元氣大傷。治病也當在疾病尚未發作之時著手。」又想到内經「陰陽應象大論」中那幾句話︰「善治者治皮毛,其次治肌膚,其次治筋脈,其次治六腑,其次治五臟。治五臟著,半死半生也。」心道︰「良醫見人疾病初萌,即當治理。病入五臟後再加醫治,已只一半把握了。像我這般陰毒散入五臟六腑,何止半死半生,簡直便是九死一生。」

正點頭讚嘆,行復自傷之際,忽聽得隱隱馬蹄聲響,自谷外直奔進來。無忌掩巻站起,心想︰「這蝴蝶谷極是隱僻,這兩年多來,除了常大哥外,從無外人到來。只怕是先生的對頭到了。他正臥病,那便如何是好?」忙奔到胡青牛門外,説道︰「先生,有數騎馬奔進谷來,你説怎麼辦?」胡青牛「{\upstsl{嗯}}」了一聲,尚未回答,那幾騎馬來得好快,已是到了茅舍之外,只聽一人朗聲説道︰「武林同道,求見醫仙胡先生,求他老人家慈悲治病。」

無忌聽了這幾句,心中一寬,回到草堂,只見門外站著一名面目黝黑的漢子,手中牽著三匹馬,兩匹馬上各伏著一人,衣上血跡糢糊,顯見身受重傷。那漢子頭上綁著一塊白布,布上也是染滿鮮血,一隻右手用繃帶吊在脖子中,看來受傷也是不輕。無忌走到門口,説道︰「各位來得眞是不巧,胡先生自己身上有病,臥病不起,無法替各位效勞,還是另請高明吧!」那漢子道︰「咱們奔馳數百里,危在旦夕,全仗醫仙救命。」

張無忌道︰「胡先生身染天花,這幾日病勢甚惡,此是實情,決不敢相欺。」那漢子躊躇半晌,嘆了口氣,道︰「咱三人是同門師兄弟,此番身受重傷,若不得蝶谷醫仙施救,那是必死無疑的了。相煩小兄弟稟報一聲,且聽胡先生如何吩咐。」無忌道︰「既是如此,請問尊姓大名。」那漢子道︰「咱三人賤名不足道,便請説是華山派鮮于掌門的弟子。」説到這裡,身子搖搖欲墜,已是支持不住,猛地裡嘴一張,噴出一大口鮮血。

無忌搶上一步,在他胸口和背心六處穴道上各點了一指。那漢子胸間熱血翻湧,本欲繼續噴出,給無忌這麼一點,穴道閉塞,胸口登時舒暢得多。他見無忌小小年紀,竟具這等身手,臉上露出驚詫之色。

無忌走到胡青牛門外,説道︰「先生,門外有三人身受重傷,前來求醫,説是華山派鮮于掌門的弟子。」胡青牛輕輕「咦」的一聲,怒道︰「不治,不治,快趕出門去。」無忌道︰「是。」回到草堂,向那漢子説道︰「胡先生病體沉重,難以見客,還請原諒。」那漢子皺起了眉頭,正待繼續求懇,伏在馬背上的一個瘦小漢子忽地抬起頭來,伸手一彈,無忌只覺眼前金光一閃,拍的一響。一件小小的暗器擊在草堂正中的桌上。那瘦漢子説道︰「你拿這朶金花去給『見死不救』看,説咱三人都是給這金花的主児打傷的,那人眼下便來找他,『見死不救』若是治好了咱們的傷,咱三人便留在這裡,助他禦敵。咱三人武功便是不濟,也總是多三個幫手。」

張無忌聽他説話大剌剌的,遠不及第一個漢子有禮,走近桌邉一看,只見那暗器是一朶黃金鑄成的梅花,和眞梅一般大小,白金絲作的花蕊,打造得精巧無比。他伸手去拿,不料那瘦子這一彈手勁甚強,金花嵌入桌面,竟是取不出來,只得拿過一把藥鑷,挑了幾下,方才取出,心想︰「這瘦漢子的武功大是不弱,但在這金花的主児手下傷得這般厲害。常大哥説這幾天會有胡先生的對頭到來尋仇,多半便是那人了,倒須跟先生説知。」於是手托金花,走到胡青牛房外,轉述了那瘦小漢子的話。

胡青牛道︰「拿進來我瞧。」無忌輕輕的推開房門揭開門帘,但見房内黑沉沉的宛似夜晩,原來天花病人怕風畏光,窗戸都用氈子遮住。胡青牛臉上蒙著一塊青布,只露出一對眼睛。無忌暗自心驚︰「不知青布之下,他臉上的痘瘡生得如何?病好之後,會不會成爲麻皮?」胡青牛道︰「將金花放在桌上,快退出房。」無忌依言放下金花,揭開門帘出房,還没掩上房門,便聽胡青牛道︰「他三人的死活,跟我姓胡的決不相干。胡青牛是死是活,也不勞他三位操心。」波的一聲,那朶金花穿破門帘,飛擲出來,{\upstsl{噹}}的一響,掉在地上。這朶金花的邉緣雖是鋒利,但布帘是柔軟之物,竟能一擲而破,張無忌和他相處兩年有餘,從未見他練過武功,原來這位文質彬彬的醫仙,却也是武學的高手,雖在病中,功力未失。

張無忌拾起金花,走出去還給了那瘦漢,搖了搖頭,道︰「先生實是病重\dash{}」猛聽得蹄聲答答,車聲轔轔,有一輛馬車向谷中馳來。無忌走到門外一望,只見那馬車馳得甚快,駛到門前,曳然而止。車中走下一個淡黃面皮的青年漢子,伸手車中,抱出一個禿頭老者,問道︰「蝶谷醫仙胡先生在家麼?崆峒門下聖手伽藍簡捷遠道求醫\dash{}」第三句話没説出口,身子一晃,連著手中的禿頭老者,一齊摔倒在地。説也湊巧,拉車的兩匹健馬也是乏得脱了力,口吐白沫,同時跪倒。

瞧了二人這般神情,不問可知,是急馳一二百里而來,途中毫没休息,以致累得如此狼狽。張無忌聽到「崆峒門下」四字,心想在武當山上逼死父母的人中,也有崆峒派的長老在内,這禿頭老者叫什麼「聖手伽藍簡捷」,當日雖然不曾來到武當,但料想也非好人,正想回絶,忽見山道上影影綽綽,又有四五人走來,有的一跛一拐,有的互相擕扶,都是身上有傷。無忌皺起眉頭,不等這干人走近,朗聲説道︰「胡先生染上了天花,自身難保,不能替各位治傷。請大家及早另尋名醫,以免耽誤了傷勢。」

待得那干人等走近,看清楚共有五人,身上衣飾都甚華貴,便似富商大賈一般,可是個個臉如白紙,竟無半點血色,身上却没傷痕血跡,看來那是受了極奇異的内傷。爲首一人又高又胖,向聖手伽藍簡捷和投擲金花的瘦小漢子點了點頭,三人相對苦笑,原來三批人都是相識的。張無忌好奇心起,問道︰「你們都是被那金花的主人所傷麼。」那胖子道︰「不錯。」轉頭向簡捷道︰「簡兄,胡先生見到了麼?」簡捷搖了搖頭,道︰「梁大老板的面子大些,或許請得動胡先生。」

無忌道︰「那金花的主人是誰啊,爲什麼這般橫行霸道?」那大胖子道︰「請小兄弟向胡先生稟報一聲,便説蕪湖源盛金號姓梁的遠道前來求醫。」竟是没答無忌的問話。最先到來那個口噴鮮血的漢子却知道無忌並非尋常少年,便道︰「小兄弟貴姓?跟胡先生怎生稱呼?」無忌道︰「我是胡先生的病人,他治了我兩年有餘,也没有治好我身上的病痛。何況胡先生説過不治,那是決計不治的,你們便賴在這裡也没用。」

説話之間,先先後後又有四個人到來,有的乘車,有的騎馬,一齊求懇要見胡青牛。無忌大是奇怪,心想︰「這蝴蝶谷地處偏僻,除了魔教中人之外,江湖上知者甚少,這些人或屬崆峒,或隸華山,均非魔教,怎地不約而同的受傷,又是不約而同的趕來求醫?」又想︰「那金花的主人既是如此了得,若要取了這些人的性命,看來也非難事,何以只將每人打得重傷?」

那十四人有的善言求懇,有的一聲不響,但都是磨著不走,眼見天色將晩,十四個人擠滿了一間草堂。煮飯的僮児將無忌所吃的飯菜端了出來,無忌也不跟他們客氣,自顧自的吃了,翻開醫書,點了油燈閲讀,對這十四人竟是視而不見,心想︰「我既學了胡先生的醫術,也得學一學他『見死不救』的功夫。」

夜蘭人靜,茅舍中除了無忌翻讀書頁、傷者粗重的喘氣之外,再無别的聲息,突然之間,屋外山路上傳來了兩個人輕輕的脚步聲音。無忌抬起頭來,只聽得那脚步行得甚是緩慢,正是走向茅舍而來。過了片刻,一個清脆的女孩聲音説道︰「媽,那屋裡有燈火,這就到了。」從那聲音聽來,那女孩年紀極是幼小。又是一個女子聲音道︰「孩子,你累不累?」那女孩道︰「我不累。媽,醫生給你治病,你就不痛了。」那女子道︰「{\upstsl{嗯}},就不知醫生肯不肯給我治啊。」無忌心中一震︰「這女子的聲音好熟!似乎是紀曉芙姑姑。」聽那小孩道︰「醫生一定會給你治的。媽,你不要怕,你痛得好些了麼?」那女子道︰「好些了,唉,苦命的孩子。」無忌聽到這裡,再無懷疑,縱身搶到門口,叫道︰「紀姑姑,是你麼?你也受了傷麼?」月光之下,只見一個青衫女子擕著一個小女孩,正是峨嵋女俠紀曉芙。

她在武當山上見到張無忌時,他還只十歳,這時相隔將近五年,無忌已自童年成爲少年,黑夜中突然相逢,那裡想得出來?一愕之下,道︰「你\dash{}你\dash{}」

張無忌道︰「紀姑姑,你不認得我了吧?我是張無忌。武當山上我爹爹媽媽自刎而死那天,曾見過一面。」紀曉芙「啊」的一聲驚呼,萬料不到竟會在此處見到他,想起自己以未嫁之身,却擕了一個女児,無忌是自己未婚夫婿殷利亨的師侄,雖是一個不懂事的少年,終究難以交代,不由得又羞又窘,脹得滿臉通紅,她受傷本是不輕,一驚之下,更是身子搖晃,便要摔倒。

她小女児只不過六七歳年紀,看見母親快要摔交,急忙雙手拉住她手臂,可是人小力微,濟得甚事?眼見兩人都要摔跌,無忌忙扶住紀曉芙肩頭,道︰「紀姑姑,請進去休息一會。」當下扶著她走進草堂。燈火看得明白,只見她左肩和左臂都受了極厲害的刀劍之傷,包紮的布片中鮮血還在不斷滲出,又聽她輕聲咳嗽不停,無法自止,無忌此時的醫術,早已勝過尋常的所謂「名醫」,一聽她咳聲有異,知是左肺葉受到了重大震盪,便道︰「紀姑姑,你右手和人對掌,傷了太陰肺脈。」

當下取出七枚金針,隔著衣服,便在她肩頭「雲門」、胸口「華蓋」、肘中「尺澤」等七人處穴道上刺了下去。其時張無忌的針炙之術,與當年醫治常遇春時自己有天壤之别。這兩年多來,他跟著「蝶谷醫仙」胡青牛潛心苦學,在診斷病情、用藥變化諸道,限於年齡經驗,和胡青牛自是相去尚遠,但針炙一門,却已學到了這位「醫仙」的七八成本領。紀曉芙初時見無忌取出金針,還不知他的用意,那知他手法快極,一轉眼間,七枚金針便刺入了自己穴道,她這七處穴全屬於太陰肺經,金針一到,立是胸口閉塞之苦大減。她又驚又喜,説道︰「好孩子,想不到你在這裡,又學會了這樣好的本領。」那日在武當山上,紀曉芙見張翠山、殷素素自殺身亡,可憐無忌孤苦,曾柔聲安慰他幾句,又除下自己頸中黃金項圏,要想給他。但無忌當時心中憤激悲痛,將所有上山來的人,都當作是迫死他父母的仇人,因之對紀曉芙出言頂撞,使她難以下台。後來他中了玄冥神掌之後,殷利亨不惜耗損功力,全心全意的替他治傷療毒。無忌感激之下,愛屋及烏,對於紀曉芙也存了好感。年紀大後慢慢的分辨是非,得知當日父親和諸師伯曾擬和峨嵋諸俠聯手,共抗群豪,這纔知峨嵋派實在是友非敵。

兩年前他和常遇春深夜在樹林之中,見到紀曉芙力救彭和尚,心中更覺這位紀姑姑爲人很好,至於她何以未嫁生子,是否對不起殷叔叔等情由,他年紀尚小,對這些男女之情全不了然,聽那之後便如春風過耳,決不縈懷。紀曉芙自己心虛,斗然間遇到和殷利亨相識之人時便窘迫異常,深感無地自容,其實這件事無忌在兩年前便已從丁敏君口中聽到,他既認定丁敏君是個壞女人,那麼她口中説的壞事,他便未必當眞是壞。

他一瞥眼間,見紀曉芙的女児站在母親身旁,眉目如畫,黑漆般的大眼珠骨碌碌地轉動,好奇地望著自己。那女孩將口俯在紀曉芙耳邉,低聲道︰「媽,這個小孩便是醫生嗎?你痛得好些了麼?」紀曉芙聽她叫自己「媽」,又是臉上一紅,事已至此,也是無法隱瞞,臉上神色甚是{\upstsl{尷}}尬,道︰「這位是無忌哥,他爹爹是媽的好朋友。」向無忌低聲道︰「她\dash{}她叫『不悔』,」頓了一頓,又道︰「姓楊,叫楊不悔!」無忌笑道︰「好啊,小妹妹,你的名字倒跟我是一對児,我叫張無忌,你叫楊不悔。」紀曉芙見無忌神色如常,並無責難之意,心下稍寬,向女児道︰「無忌哥哥的本領很好,媽已不大痛啦。」楊不悔靈活的大眼睛轉了幾轉,突然走上前去,抱住無忌,在他面頰上吻了一下。原來楊不悔年幼天眞,自幼除了母親和扶養她的一個保姆之外,從來不見外人,這次母親身受重傷,急難之中,竟蒙張無忌替她減輕痛苦,楊不悔自是大爲感激。她對母親和保姆表示喜歡和感謝,向來是撲在她們懷裡,在她們臉上親吻,這時對無忌便也如此。紀曉芙含笑斥道︰「不児,不可這樣,無忌哥哥不喜歡的。」楊不悔睜著大大的眼睛,不明其理,問無忌道︰「你不喜歡麼?爲什麼不要我對你好?」無忌笑道︰「我喜歡的,我也對你好。」在她柔嫩的面頰上也輕輕吻了一下。楊不悔拍手道︰「小醫生,你快替媽媽的傷全都治好了,我就再親你一下。」

無忌見這個小妹妹天眞活潑,甚是可愛,他十多年來,相識的都是年紀大過他很多的叔叔伯伯,常遇春雖和他兄弟相稱,也大了他八歳,那日舟中和周芷若匆匆一面,相聚不到一天,便即分手,從未交過一個小朋友,這時不禁心道︰「若是我有這樣一個有趣的親妹妹,便可常常帶著她玩耍了。」他還不過十四歳,童心猶是極盛,只是幼歷坎坷,實無多少玩耍嬉戲的機會。紀曉芙見聖手伽藍簡捷等一干人傷口狼藉,顯是未經醫理,她不願佔這個便宜,説道︰「這幾位比我先來,你先瞧瞧他們吧。這會児我已好得多了。」無忌道︰「他們是來向胡青牛胡先生求醫的,胡先生自己身染重病,何能醫人?這幾位却不肯走,只好由得他們留在這児。紀姑姑,你並非向胡先生求醫,小侄在這児耽得久了,累通一點粗淺醫道,你若是信得過,小侄便瞧瞧你的傷勢。」紀曉芙受傷後人指點,來到蝴蝶谷,原和簡捷等一般,也是要向胡青牛求醫,這時聽到了無忌這幾句話,又見到簡捷等一干人的情狀,顯是那「見死不救」胡青牛不肯施治,何況無忌適纔替她針治要穴,立時見效,看來他年紀雖小,醫道著實高明,便道︰「這可多謝你啦,大國手不肯治,請小國手治療也是一樣。」

當下無忌請她走到廂房之中,剪破她創口衣服,發覺她肩臂上一共受了三處刀傷,臂骨亦已折斷,上臂骨有一處裂成碎片。這等骨碎,在外科中本是極難接續,但在「蝶谷醫仙」的弟子看來,却也尋常,於是替她接骨療傷,敷上生肌活血的藥物,再開了一張藥方,命僮児按方煎藥。他初次替人接骨,手法未免不彀敏捷,但忙了個把時辰,終於包紮得十分妥善,説道︰「紀姑姑,請你安睡一會,待會麻藥性退了,傷口會痛得很厲害。」紀曉芙道︰「多謝你啦!」無忌到儲藥室中,找了些棗子杏脯,拿去給楊不悔吃,那知她昨晩一夜不睡,這時已偎倚在母親懷中,沉沉睡熟。無忌將棗杏放在她的袋中,回到草堂。華山派那口吐鮮血的弟子站起身來,向無忌深深一揖説道︰「小先生,胡先生既是染病,只好煩勞小先生,替咱們治一治,大夥児盡感大德。」無忌學會醫術之後,除了替常遇春、紀曉芙治療外,從未用過,眼見這十四人或内臟震傷,或四肢斷折,傷處各各不同,常言道學以致用,心中確是頗有躍躍欲試之意,但想起胡青牛的言語,答道︰「此處是胡先生家中,小可也是他的病人,如何敢擅自作主?」那漢子鑒貌辨色,見他推辭得並不決絶,便再捧他一捧,奉上一頂高帽説道︰「自來名醫都是五六十歳的老先生,那知小先生年紀輕輕,竟具這等本領,眞是十分少見,還盼顯一顯身手。」那富商模樣的姓梁胖子道︰「咱們十四人在江湖上均是小有名頭,得蒙小先生救治,大家出去一宣揚,江湖上都知小先生醫道如神的大名,那是一夕之間,小先生便名聞天下了。」

\chapter{怪傷奇醫}

張無忌究竟年紀尚幼,不明世情,給他兩人這麼一吹一捧,不免有些喜歡,説道︰「名聞天下有什麼好?胡先生既不肯動手,我也無法。但你們受傷均自不輕,這樣吧,我給你們稍減痛楚便是。」於是取出金創藥來,要替各人止血減痛。

可是待得詳察每人傷勢,不由得越看越是驚奇,原來每人的傷勢固是各各不同,而且傷法甚爲奇特,均是胡青牛所授的傷科症狀中從所未見的。有一人被仇敵逼著在肚裡吞服了數十枚鋼針。有一人肝臟被内力震傷,但醫治肝傷的「行間」「中封」「陰包」「五里」諸要穴上,却都被仇人先用尖刀戳爛,顯然下手的那人也是精通醫理,令人無從著手醫治。有一人兩塊肺葉上被釘上兩枚長長的鐵釘,不斷的咳嗽喀血。有一人左右兩排肋骨全斷,可又没傷到心肺。有一人雙手被割,却被左手接在右臂上,右手接在左臂上,血肉相連,不倫不類。更有一人全身青腫,説是被蜈蚣、蝎子、黃蜂等二十餘種毒蟲同時刺傷。

無忌只看了六七個人,已是大皺眉頭,心想︰「這些人的傷勢如此古怪,我是一件都治不來的。這下手傷人的兇手,爲何挖空心思,這般折磨人家?」忽地心念一動︰「紀姑姑的肩傷和臂傷却都平常,莫非她另受奇特的内傷,否則何以她一人却是例外?」忙走進廂房,一搭紀曉芙的脈膊,更是一驚,但覺她手脈跳動忽強忽弱、時澀時滑,顯是内臟有異,但爲什麼全變得這樣,實在説不上來。

那十四人傷勢甚奇,他也不放在心下,暗想其中崆峒派等那些人還和逼死他父母有関,此時受這些怪罪,也算活該,可是紀曉芙的傷却非救不可,於是走到胡青牛房外,低聲道︰「先生,你睡著了麼?」只聽胡青牛道︰「什麼事?不管他是誰,我都不治?」無忌道︰「是。只是這些人所受之傷,當眞是奇怪得緊。」於是將各人的怪傷,一一説了。胡青牛隔著帘布,聽得極是仔細,有不明白之處,叫無忌出去看過回來再説。無忌花了大半個時辰,纔將十五人的傷勢細細説完。

胡青牛口中不斷「{\upstsl{嗯}},{\upstsl{嗯}}」答應,顯似在用心思索,過了良久良久,説道︰「哼,這些傷勢,也難我不倒\dash{}」無忌身後忽有一人接口道︰「胡先生,那金花的主人叫我跟你説︰『你枉稱蝶谷醫仙,可是這一十五種奇傷怪毒,料你一個也醫不了』哈哈。果然你只有躱將起來,假裝生病。」無忌回頭,見説話之人正是崆峒派的禿頭老者聖手伽藍簡捷。他頭上一根毛髮也無,無忌初時還道他是天生的光頭,後來才知是給那使金花之人在頭上塗了烈性毒藥,頭髮齊根爛掉,那毒藥還在向内侵蝕,頭皮越洗越癢,只怕數日之内,毒性入腦,非癲狂不可。這時他雙手被同伴用鐵鍊縛住,這纔不能伸手去抓頭皮,否則如此奇癢難當,早已自己抓得露出頭骨了。

胡青牛淡淡的道︰「我醫得了也好,醫不了也好,總而言之,我是不會跟你治的。我瞧你尚有七八日的壽命,趕快回家,還可和家人児女見上一面,在這裡囉裡囉唆,究有何益?」簡捷頭上癢得實在難忍,熬不住將腦袋在牆上亂擦亂撞,手上的鐵鍊叮{\upstsl{噹}}急響,氣喘喘的道︰「胡先生,那金花的主児早晩便來找你,我看你也難得好死,大家聯手,共抗強敵,不是勝於你躱在房中束手待斃麼?」胡青牛道︰「你們若是打得過他,早已殺了他啦!我多你們這十五個膿包幫手,有什麼用?」簡捷哀求了一陣,胡青牛不再理睬。簡捷暴跳如雷,喝道︰「好,左右是個死,我一把火燒了你的狗窩,咱那白刀子進,紅刀子出,做翻你這賊大夫,大夥児一起送命。」

這時外邉又走進一人,正是先前嘔血而經張無忌以點穴法止住那人,他見簡捷暴跳如狂,伸手入懷,手腕翻將出來,手中已多了一柄蛾眉鋼刺,點在簡捷胸口,冷冷的道︰「你得罪胡前輩,我姓薛的先跟你過不去。你要白刀子進,紅刀子出,好啊,我就先給你這麼一下。」簡捷的武功本在這姓薛的之上,但他雙手被鐵鍊綁住,無法招架,只有瞪著圓鼓鼓的一雙大眼,不住喘氣。那姓薛的朗聲道︰「胡前輩,晩輩薛公遠,是華山鮮于先生門下弟子,這裡給你老人家磕頭啦!」説著跪了下去,{\upstsl{咚}}{\upstsl{咚}}{\upstsl{咚}}{\upstsl{咚}},磕了四個響頭。簡捷心中登時生出一絲指望,那胡青牛硬的不吃,這小子磕頭軟求,或者能成。薛公遠行過大禮,又道︰「胡前輩身有貴恙,那是咱們没福。這裡有一位小兄弟醫道高明,還請胡前輩允可,讓他治一治咱們的奇症怪傷。普天之下,除了蝶谷醫仙的弟子,咱們身上所帶的歹毒怪傷,那是再也没人治得好的了。」胡青牛冷冷的道︰「這孩子名叫張無忌,他是武當派的弟子,乃『銀鉤鐵劃』張翠山張五俠的児子,張三丰的再傳弟子。我胡青牛是魔教中爲人不齒的敗類,跟他這種名門正派的高人子弟有什麼干係?他自己身中陰毒,求我醫治,可是我立過重誓,除非是明教中人,決不替人治傷療毒。這姓張的小孩子不肯入我明教,我怎能救他性命?」薛公遠心中涼了半截,初時只道張無忌是胡青牛弟子,那麼他本領雖然不及師父,遇到疑難之處,胡青牛定肯指點,不料他也是個求醫被拒的病人。

只聽胡青牛又道︰「你們賴在我家裡不走。哼哼,以爲我便肯發善心麼?你們問問這小孩,他賴在我家裡有多久啦。」薛公遠和簡捷一齊望著張無忌,只見他伸出兩根手指,薛公遠道︰「二十天?」張無忌道︰「整整兩年另兩個月。」簡薛二人面面相覷,都透了一口冷氣。胡青牛道︰「他便是再賴十年,我也不能救他性命。只可惜一年之内,他五臟六腑中的陰毒定要發作,無論如何,活不過明年此日。我胡青牛當年曾對教祖立下重誓,便是生我的父親,我自己的親生児女,只要他不是明教弟子,我便不能用醫道救他們的性命。」

簡捷和薛公遠垂頭喪氣,正要走出,胡青牛忽道︰「這位武當派的少年也懂一點醫理,他武當派的醫理雖然遠不及我明教,但也還不致於整死人。他武當派肯救也好,見死不救也好,跟明教和我胡青牛可没牽連。」薛公遠一怔,聽他話中之意,似是要張無忌動手,忙道︰「胡前輩,這位張小俠若肯出手相救,我們便有活命之望了。」胡青牛道︰「他救不救,関我屁事?無忌,你聽著,在我胡青牛屋中,你不可妄使醫術,除非出我家門,我纔答應。」薛公遠和簡捷本覺有望,這時一聽此言,又是呆了,不明他到底是何用意。

張無忌却比他們聰明得多,當即明白,説道︰「胡先生有病在身,你們不可多打擾他,請跟我出來。」三人來到草堂,張無忌道︰「各位,小可年幼識淺,各位的傷勢又是大爲怪異,是否醫治得好,殊無把握。各位若是信得過的,便容小可盡力一試,生死各憑天命。」這當児衆人身上的傷處或痛或癢、或酸或麻,無不難過得死去活來,便是有砒霜毒藥要他們喝下去,只要解得一時之苦,那也是甘之如飴,聽了無忌的話,人人大喜應諾。張無忌道︰「胡先生不許小可在他家中動手,以免治死了人累及『醫仙』的令譽,請大家到門外吧。」衆人聽了這幾句話,却又躊躇起來,眼見他不過十四五歳,能有什麼本領?别給他亂攪一陣,傷上加傷,多受無謂的痛苦。簡捷却大聲道︰「我頭皮癢死了,小兄弟,請你先替我治。」

簡捷説罷,叮叮{\upstsl{噹}}{\upstsl{噹}}拖著鐵鍊,便走出門去。張無忌沉吟半晌,到儲藥室中揀了南星、防風、白芷、天麻、羌活、白附子、花蕊石、紫蘇等十餘種藥物,命僮児在石臼中搗爛,和以熱酒,調成藥膏,拿出去敷在簡捷的光頭之上。藥膏著頭,簡捷痛得慘叫一聲,全身都跳了起來,只聽他不住口的叫道︰「好痛,啊,痛得命也没了,{\upstsl{嗯}},還是痛的好,比那麻癢可舒服多了。」他牙齒咬得格格直響,在草地上來回疾走,連叫︰「痛得好,他媽的,這小子眞有點児本領,不,張小俠,我姓簡的得多謝你纔成。」

衆人見簡捷的頭癢立時見功,紛紛向張無忌求治。這時有一人抱著肚子,在地下不住打滾,原來他是被逼吞服了三十餘條活水蛭。那水蛭入胃不死,附在胃壁和腸壁之上吸血。張無忌想起醫書上載道︰水蛭遇蜜,化而爲水。蝴蝶谷中有的是花蜜,於是命僮児取過一大碗蜜來,命那人服了下去。

如此一直忙到天明,紀曉芙和女児楊不悔醒了出房,見無忌忙得滿頭大汗,正替各人治傷。紀曉芙便幫著包紮傷口,傳遞藥物。這一十五人本來個個是縱橫湖海的豪客,這時却要伺候無忌的眼色行事,對他的一言一語,誰都不敢違拗。只有楊不悔無憂無慮,口中吃著梨棗,追撲蝴蝶爲戲。

直到午夜,無忌才將各人的外傷初步治了一治,出血者止血,疼痛者止痛,但每人的傷勢均甚古怪複雜,單理外傷謹爲治標。無忌回房睡了幾個時辰,夢中聽得門外呻吟之聲大作,跳起身來,果見有幾人固是略見痊可,但大半却是反見惡化。他束手無策,只得去説給胡青牛聽。胡青牛冷冷的道︰「這些人又不是我明教中人,死也好,活也好,我纔不理呢。」無忌靈機一動,説道︰「假如有一個明教弟子,體外無傷,但腹内瘀血脹壅,紅腫暗青,昏悶欲死,你便如何治法?」胡青牛道︰「倘若是明教弟子,我便用山甲、歸尾、紅花、生地、靈仙、血竭、桃仙、大黃、乳香、没藥,以水酒煎好,再加童便,服後便瀉出瘀血。」

無忌又道︰「假若有一明教弟子,被人左耳灌入鉛水,右耳灌入水銀,眼中塗了生漆,疼痛難當,不能視物,那便如何?」胡青牛勃然怒道︰「誰敢如此加害我明教弟子?」無忌道︰「那人果是歹毒,但我想總須先治好那明教弟子目耳之傷,再慢慢問他仇人的姓名蹤跡。」胡青牛思索了片刻,説道︰「倘若那人是明教弟子,我便用水銀灌入他左耳,鉛塊溶入水銀,便隨之流出。再以金針深入右耳,水銀可附於金針之上,慢慢取出。至於生漆入眼,試以螃蟹搗汁化服,或能化解。」

如此這般,無忌將一件件疑難醫案,都假託爲明教弟子受傷,向胡青牛請教,胡青牛便教以治法。但那些人的傷勢實在太怪,無忌依法施爲之後,有些法子不能見效,胡青牛便潛心思考,另擬别法。

這樣過了五六日,各人的傷勢均是日漸痊癒。紀曉芙所受内傷,原來乃是中毒,敵人掌力不但震傷她内臟,還以毒性傳入,無忌診斷明白後,以生龍骨、蘇木、土狗、五靈脂、千金子、蛤粉等藥給她服下,解毒化瘀,再搭她脈膊,便覺脈細而緩,傷勢日輕一日。這時衆人已在茅舍外搭了一個涼棚,地下舖了稲草,席地而臥。紀曉芙在相隔數丈外另有一個小小茅舍,和女共住,那是無忌命各人合力所建,無忌這番忙碌雖然辛苦,但從胡青牛處學到了不少奇妙的藥方和手法,也可説大有所獲。這一天早晨起來,他一見紀曉芙的臉色,只見她眉心間隱隱有一層黑氣,不禁吃了一驚。

張無忌見了紀曉芙這等臉色,似是傷勢又有反覆,消解了的毒氣再發作出來,忙一搭她脈膊,叫她再吐些口涎,調在「百合散」中一看,果是體内毒性轉盛。張無忌苦思不解,走進内堂去向胡青牛請教。胡青牛嘆了口氣,説了治法。張無忌依法施爲,果有靈效。可是待得治好了紀曉芙,簡捷的光頭却又潰爛起來,腐臭難當。這樣過了數日,一十五個傷者都是忽好忽壞,明明已痊癒了八九成,但一晩之間,又是突然沉重。無忌不明其中理由,去問胡青牛時,胡青牛總道︰「這些人所受之傷大非尋常,倘若一醫便癒,又何必到蝴蝶谷來苦苦求我?」

這天晩上,張無忌睡在床上,潛心思索︰「傷勢反覆惡化,雖是常事,但不致於十五人個個如此,又何況一變再變,眞是奇怪得緊。」直到三更過後,他想著這件事,仍是無法入睡。忽聽得窗外有人脚踏樹葉的細碎之聲,有人放輕了脚步,悄悄走過。無忌好奇心起。濕破窗紙,向外一張,只見一個人的背影一閃,隱没在槐樹之後,瞧這人的衣著,宛然便是胡青牛。

無忌大奇︰「胡先生起來作甚?他的天花好了麼?」但見胡青牛這般行走,顯是不願被人瞧見,過了一會,見胡青牛向紀曉芙母女所住的茅舍走去。無忌心中怦怦亂跳,父親傳下來的俠義心腸登起,暗道︰「他是去欺侮紀姑姑麼?我雖非他的敵手,這件事可不能不管。」縱身從窗中跳出,躡足跟隨在胡青牛後面,只見他身形一閃,進了茅舍。那茅舍是倉卒之間胡亂搭成,無牆無門,只求聊以遮蔽風雨而已,旁人自是進出自如。張無忌大急,快步走到茅舍背後,伏地向内一張,只見紀曉芙母女偎倚著在稲草墊上睡得正沉,胡青牛從懷中取出一枚藥丸,投在紀曉芙的藥碗之中,當即轉身出外。無忌一瞥之下,見他臉上仍用青布蒙住,不知天花是否已愈,一刹那間,無忌心中恍然大悟,背上却出了一陣冷汗︰「原來胡先生半夜裡偸偸前來下毒,是以這些人的傷病終是不愈。」

但見胡青牛又走到簡捷、薛公遠等人所住的茅棚中去,顯然也是去偸投毒藥,等了好一會児也不見出來,想是那十四人所下毒物各各不同,不免多費時光。無忌悄悄鑽進紀曉芙的茅舍,拿起藥碗一聞,那碗中本來盛的是一劑「八仙湯」,要紀曉芙清晨一醒,立即服食,但這時却多了一股刺鼻的氣味。便在此時,聽得外面極輕的脚步聲掠過地面,知是胡青牛回入臥室。

無忌放下藥碗,鑽出茅舍,輕聲叫道︰「紀姑姑,紀姑姑!」紀曉芙這等學武之人,本來耳目甚靈,雖在沉睡之中,只要稍有響動,便即驚覺,但無忌叫了數聲,她終是不醒。無忌只得伸手輕搖她的肩頭,搖了七八下,紀曉芙這纔醒轉,驚問︰「是誰?」無忌低聲道︰「紀姑姑,是我,請你出來。」紀曉芙見他深夜到來,語聲甚是緊迫,知道必有要事,便將手臂輕輕從楊不悔頭頸下抽了出來,鑽出茅舍。無忌道︰「紀姑姑,你那碗藥給人下了毒,不能再喝,你拿去倒在溪中,一切别動聲色,明日跟你細談。」紀曉芙點了點頭,無忌生怕有人驚覺,回到自己臥室之外,仍從窗中爬進。

次日各人用過早餐,無忌和楊不悔追逐谷中蝴蝶,越追越遠。紀曉芙知他用意,隨後跟來。這幾天無忌常帶著楊不悔玩耍,别人見他三人走遠,誰也没有在意。一直走出里許,到了一處山坡,無忌便在草地上坐了下來。紀曉芙對女児道︰「不児,别追蝴蝶啦,你去找些野花來編三個花冠,咱們每個人戴一個。」楊不悔很是高興,自去採花摘草。無忌道︰「紀姑姑,那胡青牛跟你有何深仇大冤,爲什麼他要下毒害你?」

紀曉芙一怔,道︰「我和胡先生素不相識,直到今日,也是没見過他一面,那裡談得上『仇怨』兩字?」微一沉吟,又道︰「爹爹和師父説起胡先生時,只稱他醫術如神,乃是當世第一高手,他們跟他也是並不相識。他、他爲什麼要下毒害我?」張無忌於是將昨晩見到胡青牛偸入她茅舍下毒的事説了,又道︰「我聞到你那『八仙湯』中,有鐵線草和透骨菌的刺鼻氣味。這兩味藥本來也有治傷之效,但毒性甚烈,下的份量決不能重,尤其和八仙湯中的八種傷藥均有衝撞,於你身子大有損害。雖不致命,可就纏綿難愈了。」紀曉芙道︰「你説餘外的十四人也是這樣,這事更加奇怪。就算我爹爹或是我峨嵋派無意中得罪了胡先生,但不能那一十四人也均如此。」張無忌答道︰「紀姑姑,這蝴蝶谷甚是隱僻,你怎地會找到這裡?那打傷你的金花主人却又是誰?這些事跟我無関,我原是不該多問,但眼前之事甚是蹺蹊,請你莫怪。」紀曉芙臉上微微一紅,明白了無忌話中之意,他是生怕這件事和她未嫁生女一事有関,説起來令她{\upstsl{尷}}尬,是以相處數日,他始終絶口不提,便道︰「你救了我的性命,我還能瞞著你什麼?何況你年紀雖小,待我和不児却是很好,我滿腔的苦處,除了對你之外,這世上再也没有可以吐露之人了。」説到這裡,不禁流下泪來。

她取出手帕,拭了拭眼泪,道︰「自從兩年多前,我和一位師姊因事失和之後,我便不敢去見師父,也不敢回家\dash{}」張無忌道︰「哼,那『毒手無鹽丁敏君』壞死啦!姑姑,你也不用怕她。」紀曉芙奇道︰「咦,你怎地知道?」無忌便將那晩他和常遇春躱在樹林之中,見到她相救彭和尚的事説了。紀曉芙幽幽嘆了口氣,道︰「若要人不知,除非己莫爲!天下人的耳目,又怎能瞞過?」張無忌道︰「姑姑,殷六叔雖然爲人很好,但你要是不喜歡他,不嫁給他又有什麼要緊?下次我見到殷六叔時,請他不要逼你便是。」紀曉芙聽他説得天眞,將天下事瞧得忒煞簡單輕易,不禁苦笑了一下,道︰「孩子,也不是我有意對不起你殷六叔,當時我是事出無奈,可是\dash{}可是我也没後悔\dash{}」

她瞧著無忌天眞純潔的臉孔,心想︰「這孩子的心地有如一張白紙,這些男女情愛之事,還是别跟他説的好,何況眼前之事,也不見得與此有関。」説道︰「我和丁師姊鬧翻之後,從此不回峨嵋,帶著不児,在此以西三百餘里的舜耕山中隱居。兩年多來,每日只和樵子鄕農爲伴,倒也逍遙安樂。半個月前,我帶了不児,到鎭上去買布,想給不児縫幾件新衣,却在牆角上看到畫著一圏佛光和一把小劍。這是我峨嵋派呼召同門的訊號,我看了之後,自是大爲驚慌,沉吟良久,自忖雖然我和丁師姊反目失和,但曲不在我,我也没做任何欺師叛門之事,今日見到這訊號,説不定同門遇上了急難,不能不加援手。於是依據訊號所示,一直到了鳳陽。」

\qyh{}在鳳陽城中,又看到了訊號,約我到臨淮閣酒樓中聚會。我硬了頭皮,和不児一齊上臨淮閣去,只見酒樓中已有七八個武林人士等著,崆峒派的聖手伽藍簡捷、華山派薛公遠他們三師兄弟都在其内,可是並無峨嵋同門。我和簡捷、薛公遠他們以前見過面,問起來時,原來他們也是看到同門相招的訊號,各自趕到這児赴約,到底爲了什麼事,却是誰也不知。」

\qyh{}這日等了一天,不見同門到來,後來却又陸續到了幾人,有神拳門的,有南少林的,都説是接到同門邀約,到臨淮閣聚會。第二天又有幾個人到來,但個個是受人之約,没一個是出面邀約的。大家一商量,都起了疑心︰莫非是受了敵人的愚弄?」

紀曉芙續道︰「可是咱們聚在臨淮閣酒樓上的一十五人,包括了九個門派。每個門派傳訊的記號不但各各不同,而且均是嚴守祕密。若非本門中人,雖可見到,却決不知其中含意。倘若眞有敵人暗中佈下陰謀,難道他竟能盡知這九個門派的暗記麼?我一來帶著不児,生怕眞的遇上什麼兇險;二來我也確是不願和同門相見,既見並非同門遇上危難要我援手,當下帶了不児便想回家。我正要走下酒樓,忽聽得樓梯上篤篤聲響,似是有人用棍棒在梯級上敲打,跟著一陣劇烈的咳嗽之聲,一個弓腰曲背,白髮如銀的老婆婆走了上來。她走幾步,咳幾聲,顯得極是辛苦,旁邉一個十二三歳的小姑娘神清骨秀,相貌美麗之極,年紀雖尚幼小,但我生平遇到過的女子之中,從未見過這般標緻的姑娘,不由得向她多瞧了幾眼。那老婆婆右手撐著一根白木拐杖,布衣荊釵,似是個貧家老婦,可是左手拿著的一串念珠却是金光燦爛,閃閃生光。我凝神一看,只見那串念珠的每一顆珠子,原來都是黃金鑄成的一朶朶梅花之形\dash{}」

張無忌聽到這裡,忍不住插口道︰「那老婆婆便是金花的主人?」紀曉芙點頭道︰「不錯!可是當時却有誰想得到?」她從懷中取出一朶小小的金鑄梅花,正和張無忌曾拿去給胡青牛所看的那朶一般無異。無忌大奇,他這幾天來心中一直記掛著那個「金花的主人」,料想他不知是一個多麼猙獰可怖、兇惡厲害的人物,但聽紀曉芙如此説,却是一個身患重病的老婆婆,實是大出他意料之外。

紀曉芙又道︰「那老婆婆上得樓來,又是大咳了一陣。那小姑娘道︰『婆婆,你服一顆藥吧?』那老婆婆點頭,小姑娘取出一個瓷瓶,從瓶中倒出一顆藥丸,老婆婆慢慢咀嚼了嚥下,接連説了幾句『阿彌陀佛,阿彌陀佛。』她一雙老眼半開半閉,喃喃的道︰『只有十五個,{\upstsl{嗯}},你問問他們,有崑崙和武當的人來了没有?』她走上酒樓之時,誰也没加留神,但忽然聽到她説了那兩句話,幾個耳朶靈的江湖朋友一齊轉過頭來望著她,待得見到是這麼一個老態龍鍾的貧婦,都道是聽錯了話。那小姑娘朗聲道︰『喂,我婆婆問你們︰崑崙派和武當派有人來了没有?』衆人都是一呆,誰也没有回答。過了片刻,崆峒派的簡捷纔道︰『小妹妹,你説什麼?』那小姑娘道︰『我婆婆問︰爲什麼不見武當和崑崙的弟子?』簡捷道︰『你們是誰?』那老婆婆彎著腰又咳嗽起來,突然之間,我只覺一股勁風直擊向我胸口。這股勁風不知從何處而來,却又來得迅捷無比,我忙伸掌一擋,登時胸口閉塞,體内熱血翻湧,雙腿站立不定,便即坐倒在樓板之上,吐出了幾口鮮血。我在茫無所措之中,但見那老婆婆身形飄動,東按一掌,西擊一拳,中間還夾著一聲聲的咳嗽,頃刻間將酒樓上其餘一十四人盡數擊倒。她出手如此突如其來,身法之快,力道之勁,不但我從所未見,却是聽也没聽見過,酒樓上的一十五人,竟是没一個能還得一招半式,每個人不是穴道被點,便是受内力震傷了臟腑。那老婆婆左手一揚,十五朶金花從她念珠串上飛出,分擊十五人的手臂,這一次她却不是志在傷人,因此每人被金花擊中,却都不受什麼損傷。她轉過身來,扶著那小姑娘,説道︰『阿彌陀佛!』便顫巍巍的走下酒樓去了。各人耳聽得她拐杖著地,發出緩慢凝重的篤篤之聲,一步步遠去,偶爾還有一兩聲咳嗽,從樓下傳來。」

紀曉芙説到這裡,楊不悔已編好了一個花冠,笑嘻嘻的走來,道︰「媽,這個花冠給你戴。」説著給母親戴在頭上。紀曉芙笑了笑,繼續説道︰「當時酒樓之中,一十五人個個軟癱在樓板上,有的還能呻吟幾聲,有的却已是上氣不接下氣\dash{}」楊不悔道︰「媽,你在説那個惡婆婆的事麼?别説,别説,我怕得很。」紀曉芙道︰「乖孩子,你再去採花児編個花冠,給無忌哥哥戴。」楊不悔望著無忌,問道︰「你喜歡什麼顏色的。」無忌道︰「要紅色的,{\upstsl{嗯}},還要些白色的,越大越好。」楊不悔張開雙手,道︰「這樣大麼?」無忌道︰「好,就是這麼大。」楊不悔拍手走開,説道︰「我編好了你可不許不戴。」紀曉芙續道︰「我在昏昏沉沉之中,只見十多人走了過來,都是酒樓中的酒保、掌櫃的、厨子等等,將咱們抬到厨房之中。不児這時嚇得只有大哭的份児,跟在我的身旁。那掌櫃的手中拿著一張單子指著簡捷道︰『在他頭上塗這個藥膏。』便有個酒保將事先預備定當的一盒藥膏,塗在簡捷頭上。那掌櫃看看單子,指著一人道︰『砍下他的右臂,接在他左脚上。』兩名厨師取過利刃,依言施行。他説到我的時候,幸好還没什麼古怪的苦刑,只餵我服了一碗甜甜的藥水。我明知其中必有劇毒,但當時只有受人擺佈的份児,如何能彀反抗?咱們一十五人給他們古古怪怪的施了一番酷刑之後,那掌櫃的説道︰『你們每人都已身受不治之傷,没一個能活得過十天半月,但金花的主人説道,她跟你們原本無冤無仇,瞧你們可憐見児的,便大發慈悲,指點一條生路,你們趕快到女山湖畔蝴蝶谷去,哀求一位號稱『蝶谷醫仙』的胡青牛施醫。如果他肯出手,那麼每個人都有活命之望,否則當世没一人能救你們姓命。這個胡青牛又有一個外號,叫作『見死不救』,你們倘若不是死磨死纏,他是決計不肯動手的。你們跟胡青牛説,金花的主人不久就去找他,叫他及早預備後事吧!』他説完之後,便給咱們套車叫馬,指明路徑,大夥児便到了這裡。」

張無忌越聽越奇,道︰「紀姑姑,如此説來,那臨淮閣中的掌櫃、厨師、酒保等一干人,都是那惡婆婆的一夥了?」紀曉芙道︰「看來那些人都是他的手下,那掌櫃的按照惡婆婆單子上書明的法子,對咱們施這種酷刑。直到今天,我還是半點也不明白,爲什麼那惡婆婆要幹這種令人猜想不透的事?她若是跟我們有仇,要取我們性命原是舉手之勞。倘是存心要我們多吃些苦頭,想出這種惡毒的法児來對我們痛加折磨,那爲什麼又送我們來向胡先生求醫?又説她不久便來找胡先生尋仇,難道用這種希奇古怪的法児將我們整治一頓,不過是試一試胡先生的醫道麼?」

張無忌沉吟半晌道︰「我聽常遇春老大哥説,胡先生有一個對頭,日内便要來尋他的晦氣,那自是這個金花婆婆了。按理説,胡先生原該將你們治好,齊心合力,共禦大敵。否則是他口説不肯施治,爲什麼又教了我各種解救的藥方和針術。這些方術施用起來,確是甚具靈效,這麼説,那是他明裡不救,暗中假手於我來救人了。可是他教我治好了你們,半夜裡却又偸偸前來下毒,令你們死不死,活不活的。其中的蹺蹊,當眞是令人百思不得其解。」

兩人商量了良久,想不出半點緣由。楊不悔却已編了一個極大極大的花冠,給無忌戴在頭上。無忌道︰「紀姑姑,以後除非是我親手給你端來的湯藥,你千萬不可服用。晩上你手邉要放好兵刃,以防有人加害。眼前你還不能便去,等我再配幾劑藥給你服了,内傷無礙之後,乘早帶了不児逃走吧。」

\chapter{醫仙毒仙}

紀曉芙點點頭,又道︰「孩子,這姓胡的居心如此叵測,你跟他同住,也非善策,不如咱們一起走吧。」張無忌道︰「{\upstsl{嗯}}。他雖稱醫仙,但竟治不好我體内陰毒,説我活不過明年此日,十九也是不安好心。」紀曉芙沉吟道︰「你太師父張眞人言道︰你若能習得『九陽眞經』中所載神功,當可化解體内陰毒,那部九陽眞經當年被瀟湘子和尹克西從少林寺中竊出後,從此不知所終,當世只有武當、少林、峨嵋三派,分得其中若干祕要。我師父本來有意傳我衣缽,到那時該會授我『峨嵋九陽功』,唉,只是我做下了這等不肖之事,那有臉面再去見我師父?衣缽眞傳云云,更是休得提起。」

張無忌見她神色淒然,安慰她道︰「紀姑姑不必難過,胡先生説我只有一年之命,侄児自己按脈運氣,知他所言非虛。尊師便傳了你峨嵋九陽功,那時候我也已來不及救我了。本來咱們這時便走,最是穩妥,但如何醫治姑姑内傷,我還有幾處不明,須得再請教胡先生。」紀曉芙道︰「他既在膳中下毒害我,那麼教你的方術只怕也是故意不對。」張無忌道︰「那又不然,胡先生教我的醫術,却又是靈效如神,這中間的是非,我是分辨得明白的,奇就奇在這裡。」

此時楊不悔第三頂花冠也已編好,三人頭上各戴一頂,回到茅舍。

當天晩上,張無忌睜眼不睡,到得三更時分,果然又聽得胡青牛悄悄從房中出來,到紀曉芙的茅棚中去下毒。這般過了三日,紀曉芙因不服毒藥,痊癒得極快,簡捷、薛公遠他們却好了又發,反反覆覆,有幾個脾氣暴躁的已是大出怨言,説無忌的醫道太過低劣。無忌也不理會,暗想過了今晩,便可和紀曉芙母女脱身遠走,自己陰毒難除,也不回到武當山去,免得太師父和諸師伯叔傷心,找個荒僻的所在,靜悄悄的一死便了。

這晩臨睡之時,無忌想明天一早便要離去,胡青牛雖然古怪,待自己究竟不錯,這兩年多來,授了自己不少醫術,相處一場,臨别也有些黯然之感,於是走到他的房外,問候了幾句,又想起那金花惡婆早晩要來尋事,不知他何以抵禦,一時好心,便道︰「胡先生,你在蝴蝶谷中住了這麼久,難道不厭煩麼?幹麼不到别的地方玩玩?」胡青牛一怔,道︰「我有病在身,怎能行走?」無忌道︰「套一輛騾車,不就可以走麼?只要用布蒙住車門車窗,密不通風,也就是了。」胡青牛嘆了口氣道︰「孩子,你倒好心。天下雖大,只可惜到處都是一樣。你這幾天胸口覺得怎樣?丹田中寒氣翻湧麼?」無忌道︰「寒氣日甚一日,反正無藥可治,那也任其自然吧。」

胡青牛頓了一頓,道︰「我開張救命的藥方給你,用當歸、遠志、生地、獨活、防風五味藥,二更時以穿山甲爲引,急服。」無忌吃了一驚,心想這五味藥和自己的病毒絶無関連,而且藥性頗有衝突之處,以穿山甲作藥引,更是不通,問道︰「先生,這些藥份量如何?」胡青牛怒道︰「我跟你説了還不快快滾出去!」

張無忌一聽大怒,他自在蝴蝶谷寄居以來,每日裡跟胡青牛談論醫理藥性,胡青牛當他是半徒半友,向來頗有禮貌,這時竟然如此不留情面的呼叱,不由得怒氣沖沖的回到臥房,心道︰「我好意勸你遠行避禍,没來由却遭這番折辱,又胡亂開這張藥方給我,難道我會上當麼?」他躺在床上,腦海中思潮起伏,只是想著適纔胡青牛的無禮言語,正要朦朧入睡,忽地想起︰「當歸,遠志\dash{}有藥名而無份量,天下無這般的藥方,莫非他説當歸,乃是『該當歸去』之意?」

一想到「當歸」或是「該當歸去」之意,張無忌跟著便想︰「遠志」是叫我「志在遠方」,「高飛遠走」,「生地」和「獨活」的意思明白不過,自是説如此方有生路,方能獨活,那「防風」呢?{\upstsl{嗯}},是説「須防走漏風聲」。又説「二更時以穿山甲爲引,急服」,「穿山甲」,那是叫我穿山逃走,不可經由谷中大路而行,而且須二更時急走。」

這麼一想,對胡青牛這張藥不對症、莫名其妙的方子,他登時豁然盡解,一驚之下,跳起身來,但轉念又想︰「胡先生必是知曉眼前便有大禍臨頭,是以好意叫我急速逃走,可是此刻敵人未至,他爲什麼不明明白白跟我説,却要打這個啞謎?若是我揣摩不出,豈非誤事?此刻二更已過,須得快走。」他年紀雖小,却是頗有俠義心腸,暗想胡先生必有難言之隱,因是這些日子始終不走,説不定暗中已安排了對付大敵的巧妙機関,他雖叫我「防風」「獨活」,但紀姑姑母女却不能不救。

當下悄悄出房,走到紀曉芙的茅棚之中,在她肩頭輕輕拍了拍,低聲道︰「紀姑姑醒來。」紀曉芙翻身坐起,道︰「是無忌麼?」便在此時,無忌只覺背後風聲微動,待要轉身,猛地裡肩頭和腰裡一麻,已被點中了穴道,翻身栽倒。那敵人出手快極,跟著便擋開紀曉芙拍來的一掌,順手又點中了她的穴道。這一晩是月半,月光從茅棚的空隙中照射進來,張無忌見那敵人方巾藍衫,青布蒙臉,正是胡青牛,瞬息間千百個疑團湧向心間。

只見胡青牛手捏住紀曉芙的臉頰,逼得她張開嘴來,右手取出一顆藥丸,便要餵入。紀曉芙一聞到這藥丸,已感頭暈腦脹,知是劇毒之物,但身子動彈不得,向睡得正沉的女児望了一眼,淒然心道︰「不児,不児,媽媽苦命,你也苦命。從今而後,媽媽再也不能照顧你了。」但見那人兩根手指挾著藥丸,正要塞入她的口中,忽見張無忌突然長身躍起,那人一驚回頭,砰的一響,那人背上已被張無忌反手一掌,重重擊中。

原來張無忌肩頭和腰脅穴道雖然被點,但他自幼受謝遜之授,武功自成一家,穴道被點之後,片刻間即能運氣通解,四肢能彀轉動後無暇多想,反手便是一招「神龍擺尾」,正擊中在胡青牛背心的「筋縮穴」上。這招「神龍擺尾」乃「降龍十八掌」中的一招,這套掌法無忌雖只學得一知半解,僅得皮毛,但這一招「神龍擺尾」,他却使得威猛無儔。敵人武功雖高出他十倍,但一來萬料不到他穴道被點之後,竟會立時自解,二來這一招掌法神奇奥妙,即是在全神貫注之時,化解也是不易,何況是出其不意的攻至?他「筋縮穴」一被擊中,當即委頓在地。

他身子一軟倒,蒙在臉上的青布也即掀開了半邉,無忌一看之下,忍不住「啊」的一聲驚呼,原來這人竟不是胡青牛,秀眉粉臉,竟是一個中年婦人。

無忌道︰「你\dash{}你是誰?」那婦人背心中了這掌,疼得臉色慘白,説不出話來。無忌當即在紀曉芙肩上推拿一陣,解開她的穴道,説道︰「紀姑姑,你用劍指住她胸口,不許她動彈,我瞧瞧胡先生去。」他心中大是焦慮,生怕胡青牛已遭了這婦人的毒手,又想這婦人自是金花惡婆的一黨,眼下雖然僥倖被自己一招得手,因而制住,但只要金花惡婆一到,自己和紀曉芙決計逃不出她的毒手。

當下提氣直奔,跑到胡青牛臥室之外,砰的一聲,推開房門,叫道︰「先生,先生,你好麼?」却不聞應聲。無忌大急,在桌上摸索到火石火鐮,點亮了蠟燭,只見床上被褥揭開,却已不見了胡青牛的人影。

張無忌奔進室中之時,本來擔心會見到胡青牛屍橫就地,已遭那婦人的毒手,這時見室中空空蕩蕩地無人在内,反而稍爲安心,暗想︰「先生既被對頭擄去,此刻或許尚無性命之憂。」正要追出,忽聽得床底有一陣粗重的呼吸之聲,他彎腰舉蠟燭一照,却見胡青牛手脚被綁,赫然正在床底。無忌大喜,忙道︰「先生,我來救你。」忙將他拉出,只見他口中被塞了一個大胡桃,是以不會説話。

張無忌取出他口中胡桃,想解開他的綁縛,却見引綁著他手脚的均是絲麻和著牛筋絞成的粗索,無法解開,只得取出小刀,要待用力割斷。胡青牛道︰「那女子呢?」無忌道︰「她已被我制住,逃不了。」胡青牛道︰「你别先解我綁縛,快帶她來見我,快快,遲了就怕已來不及。」無忌奇道︰「爲什麼?」胡青牛道︰「快帶她來,不,你先取三顆『牛黃血竭丹』給她服下,在第三個抽履中,快快。」他不住口的催促,神色極是惶極。無忌知道這『牛黃血竭丹』是解毒靈藥,胡青牛配製時和入許多珍奇藥物,只須一顆,已足化解劇毒,這時却叫他去給那女子服上三顆,難道她已然中毒?

但見胡青牛神色大異,焦急之極,當下不敢多問,取了牛黃血竭丹,奔進紀曉芙的茅棚,喝道︰「快服下了!」那女子罵道︰「滾開,誰要你這小賊好心。」原來她一聞到牛黃血竭丹的氣息,已知是解毒之藥。張無忌道︰「是胡先生給你服的!」那女子道︰「走開,走開!」只是她被無忌一掌擊傷之後,説話聲音是微弱。無忌不明胡青牛的用意,猜想這女賊在綁縛胡青牛之時,中了他的餵毒暗器,但胡青牛要留下活口,詢問敵情,常下伸手在「肩貞穴」上點了兩指,使她不能抗拒,然後硬生生將三顆丹藥餵入她的口中。

一番擾攘,楊不悔已然醒來,睜著大大的眼睛,好奇地望著那個女子。無忌道︰「姑姑,咱們去交給胡先生,請他發落。」兩人分擕那女子一臂,將她架入胡青牛的臥室。

胡青牛兀自躺在地下,一見那女子進來,忙問︰「服下藥了麼?」張無忌道︰「服了。」胡青牛道︰「很好,很好!」頗爲喜慰。無忌於是割斷他的綁縛。胡青牛手足一得自由,立即過去翻開那女子的眼皮,察看眼臉内的血色,又搭了搭她的脈搏,驚道︰「你\dash{}你怎地又受了外傷?誰打傷你的?」語氣中又是驚惶,又是憐惜,那女子扁了扁嘴,哼了一聲,道︰「問你的好徒弟啊。」胡青牛轉過身來,問無忌道︰「是你打傷他的麼?」無忌道︰「不錯,她正要\dash{}」第六個字還没出口,胡青牛拍拍兩下,重重的打了他兩個耳光。

這兩掌沉重之極,來得又是大是出意料之外,無忌絲毫没有防備,竟没閃避,只給他打得眼前金星亂舞,幾欲昏暈。紀曉芙長劍挺出,喝道︰「你幹什麼?」胡青牛對這青光閃閃的利器竟是全不理會,問那女子道︰「你胸口覺得怎樣?{\upstsl{嗯}},我定能治好你。」但見他態度殷勤,與他平時「見死不救」的情狀大異其趣,那女子却是冷冷的愛理不理。張無忌撫著高高腫起的雙頰,越想越是胡塗。胡青牛給那女子解開穴道,按摩手足,取過幾味藥物,細心的餵在她口中,然後抱著她放在床上,輕輕替她蓋上棉被。這般溫柔熨貼,那裡是對付敵人的模樣?

胡青牛臉上愛憐橫溢,向那女子凝視半晌,輕聲道︰「這番你毒上加傷,若是我能給你治好,咱倆永遠不再比試了吧?」那女子笑道︰「這點輕傷算不了什麼。可是我服的是什麼毒藥,你怎能知道?你要是當眞治得好我,我便服你。就只怕醫仙的本事,未必及得上毒仙吧?」

她説了這幾句話,微微一笑,臉上嬌媚無限,張無忌雖是不懂男女之情,但也瞧得出兩人實在感情極佳,相互間眉梢眼角之中,蘊藏著纏綿的愛意。只聽胡青牛道︰「十年之前,我便説醫仙萬萬及不上毒仙,你偏偏不肯信。唉,什麼都好比試,怎能作踐自己。這一次我却盼醫仙勝過毒仙了。否則的話,我也不能一個人獨活。」那女子輕輕笑道︰「我若是去毒了别人,你仍會讓我,假裝不及我的本事。哈哈,我毒了自己,你非得出盡八寶不可了吧。」胡青牛給她掠了掠頭髮,嘆道︰「我可實在擔心得緊。快别多説話,閉著眼睛。你若是暗自運氣糟蹋自己,那可不是公平比試了。」那女子微笑道︰「我纔不會這樣下作。」説著便閉了雙眼,嘴角邉仍帶甜笑。

兩人這番對話,只把紀曉芙和張無忌聽得呆了。胡青牛轉過身來,向無忌深深一揖,説道︰「小兄弟,是我一時情急,多有得罪,還請原諒。」無忌憤憤的道︰「我可半點也不明白,不知你在搗什麼鬼。」胡青牛提起手掌拍拍兩響,用力打了自己兩個耳光,説道︰「小兄弟,你於我有救命之恩,只因我関懷拙荊的身子,適纔冒犯於你。」無忌奇道︰「她\dash{}她是你的夫人?」胡青牛點頭道︰「正是拙荊。」他平素端嚴莊重,無忌對他頗爲敬畏,這時見他居然自打耳光,可見確是誠心致歉,又聽得這女子竟是他的妻子,滿腔怒火登時化爲烏有。

胡青牛搬過椅子,請紀曉芙和張無忌坐下,説道︰「今日之事,兩位定覺奇怪,此事也不便相瞞。拙荊姓王,閨名叫做難姑,和我是同門師兄妹。當我二人在師門習藝之時,除了修習武功,我專攻醫道,她學的却是毒術。她説一人所以學武,乃是爲了殺人,毒術也是殺人之用,武術和毒術相輔相成。只要精通毒術,那麼武功便等於強了一倍。但醫道却是治病救人之術,和武術背道而馳。我想拙荊之言也不錯,只是我素心所好,也是勉強不來。」

\qyh{}我二人所學雖然不同,情感却好,師父給我二人作主,結成夫婦,慢慢的在江湖上各自闖出了名頭。有人叫我『醫仙』,便叫拙荊爲『毒仙』。她使毒之術,神妙無方,不但舉世無匹,而且青出於藍,已遠勝於我師父,使毒下毒而稱到一個『仙』字,可見她本領之超凡絶俗。也是我做事太欠思量,有幾次她向人下了慢性毒藥,中毒的人向我求醫,我胡裡胡塗的便將他治好了。當時我還自鳴得意,却不知這種舉動對我愛妻實是不忠不義。『毒仙』手下所傷之人,『醫仙』居然能將他治好,那不是自以爲『醫仙』強過『毒仙』麼?」

紀曉芙只聽得暗暗搖頭,心中大不以爲然,只聽胡青牛又道︰「她向來待我溫柔和順,情深義重,普天下女子之中,再也尋不出第二個來,可是我這種對不起愛妻的負心薄倖、逞強好勝之舉,接二連三的做了出來,内人便是泥人,也會有個土性児啊。最後我知道自己太過不對,便立下重誓,凡是她下了毒之人,我決計不再恃技醫治,日積月累,我那『見死不救』的外號傳了開來。拙荊見我知過能改,尚有救藥,也就既往不咎,可是我改過自新没幾年,便發生我妹子的事。」

\qyh{}我妹子受了華山派鮮于通這賊子的欺辱,終於死在他的手裡。但我妹子到死還是愛他,要我答應一生照料這個賊子。我見她死不瞑目,只得答應。那知拙荊早已在鮮于通身上下了極厲害的毒藥,要他全身肌肉慢慢腐爛,苦受三年折磨方死。這鮮于通知道我答應過妹子救他,一見情形不對,便即上門求救。這可不是令我左右爲難麼?若是救他,那是對不起拙荊,倘若不救,却又違了我在舍妹臨終時答應她的言語。

紀曉芙道︰「那鮮于通現任華山掌門,武功很強,江湖上也頗具俠名,那知竟是個卑鄙小人。令妹既是害於他手,胡先生也不必救他了,何況令妹已死,也不會再知此事。」張無忌道︰「不,不!人死之後,世上的事他還是知道的。」他時常思念父母,是以盼望父母泉下有知,將來自己死後,終於能再和父母相會。

胡青牛嘆道︰「幽冥之事,咱們雖然無法知曉,但我想對不起拙荊,日後尚可補過,對不起妹子\dash{}唉,她一生可憐,我怎能對不起她?於是我費盡心力,終於將鮮于通那賊子治好了。拙荊却也不跟我吵鬧,只説︰『好!蝶谷醫仙胡青牛果然醫道通神,可是我毒仙王難姑偏生不服,咱們來好好比試一下,瞧是醫仙的醫技高明呢,還是毒仙的毒術厲害。』我竭誠道歉,她自是不加理睬。」

\qyh{}此後數年之中,她潛心鑽研毒術,在好幾個江湖人物身上下了劇毒,却又指點他們來向我求醫。一來她毒術神妙,我的醫術有時而窮;二來我也不願使她生氣,因此醫了幾下醫不好,便此罷手。可是拙荊反而更加惱了,説我瞧她不起,故意讓她,不和她出全力比試,一怒之下,便此離開蝴蝶谷,説什麼也不肯回來。她在外邉傷了人,總是叫他們來向我求醫,而且下毒手段甚是巧妙,不露出是她的手筆,有時我査察不出,一時胡塗,便將來人治好了。這麼一來,拙荊和我的嫌隙,便越結越深。唉,我胡青牛該當改名爲「蠢牛」纔對。像難姑這般的女子,肯委身下嫁,不知是我幾生修下來的福份,我却不會服侍她、體貼她,常常惹她生氣,終於逼得她離家出走,浪跡天涯,受那風霜之苦。何況江湖上人心險詐陰毒之輩,在所多有,她孤身一個弱女子,怎叫我放心得下?」説到這裡,自怨自艾之情,見於顏色。

紀曉芙向臥在榻上的王難姑望了一眼,心想︰「這位胡夫人號稱『毒仙』,天下還有誰更毒得過她的?她不去毒人,已是上上大吉了,又有誰敢來毒她?這胡先生畏妻如虎,也當眞令人好笑。」

胡青牛又道︰「七年之前,有一對老年夫婦身中劇毒,到蝴蝶谷求醫。這對老夫婦是東海靈蛇島的主人,武功自成一家,原是老一輩的人物,金花婆婆和銀葉先生數十年前威震天下,誰都忌憚三分。我不敢直率拒醫,但你想,我既已迷途知返,豈能一錯再錯?當下搭了搭脈,便説島主銀葉先生無藥可治,老夫人金花婆婆中毒不深,可憑本身内力自療。我一問起下毒之人,知道是西域白駝派一位極厲害的人物所爲,和拙荊原無干係,但我既説過除了明教本教的子弟之外,外人一槩不治,自也不能爲他們二人破例。那位老夫人許下我極重的報酬,只求我相救老島主一命。想那靈蛇島主人金花銀葉夫婦在武林中是如何身份,居然出口向我求懇,那自是我極大的面子,但我顧念夫妻之情,還是袖手不顧。這對夫婦居然並不向我用強,兩人知道無望,便即黯然而去。我知道爲了不肯替人療毒治傷,江湖上已結下了不少樑子,惹下了無數對頭。但我夫妻情深,終不能爲了不相干的外人而損我伉儷之情,你們説是不是啊。」
\footnote{\footnotefon{}〈連載版〉説下毒者是西域白駝派一位極厲害的人物。然而白駝山自歐陽鋒死後,竟無一個傳人。因三聯版設定白駝山是一脈單傳,一代武功只傳一人。後來〈三聯版〉中對金花銀葉下毒之人,是一個西域頭陀;〈新修版〉則改爲西域番僧。翻來覆去還是難以開脱下毒者不是范遙。}

紀曉芙和張無忌默然不語,心中頗不以他這種「見死不救」的主張爲然。胡青牛又道︰「最近常遇春來到蝴蝶谷,説途中遇到一位老婆婆,命他來告知我,銀葉先生果然如我所料,已毒發身亡。遇春走後不久,拙荊突然回家,她見家中多了一個外人,便先用藥將無忌迷倒了一晩。」張無忌恍然大悟︰「那一晩自己一直睡到次日下午方醒,原來是中了王難姑的迷藥,自己却還道生病。這位毒仙傷人於不知不覺之間,果是厲害無比。」

胡青牛續道︰「我見拙荊突然回來,自是歡喜得緊。她跟我説,她也得悉了靈蛇島金花婆婆重返中土的訊息,因此心下雖然惱我,還是回來向我告知。她要我假裝染上天花,不見外人,兩人守在房中,潛心思索抵禦金花婆婆的法子。這位前輩異人武功太高,要逃走是萬萬逃不了的。但她有個古怪脾氣,她若想殺你,出手以三下爲限,只要你躱得過這三下不死,便饒了你性命。没過幾天,薛公遠、簡捷以及紀姑娘你等一十五人陸續來了。我一聽你們受傷的情形,便知金花婆婆是有意試我,瞧我是否眞的信守諾言,除了明教子弟之外,果然決不替外人治療傷病。一十五人身上,帶了一十五種奇傷怪病,我姓胡的嗜醫如命,只要見到這般一種怪傷,也是忍不住要試一試自己的手段,又何況共有一十五種?但我也明白金花婆婆的心意,只要我治好了一人,她加在我身上的慘毒報復,那就會厲害百倍,因此我雖然心癢難搔,還是袖手不顧。直到無忌來問我醫療之法,我纔説了出來。但我特加説明,無忌是武當弟子,跟我胡青牛絶無干係。」

\qyh{}難姑見無忌依著我的指點,施治竟是頗見靈效,心中又不快起來,每晩便悄悄在各人的飲食藥物之中,加上毒藥,那自是和我繼續比賽之意,這一十五人個個都是武學的好手,她走到各人身旁下毒,衆人如何不會驚覺?原來是她先將各人迷倒,然後從容自若,分别施用奇妙的毒術。」紀曉芙和張無忌對望了一眼,這纔明白,爲何無忌走到紀曉芙的茅棚之中,要用力推開她肩頭,方得使她醒覺。

胡青牛續道︰「那幾日來,紀姑娘的病情痊癒得甚快,顯見難姑所下之毒不生效用。她一加査察,才知是無忌發覺了她的祕密,於是要對無忌也下毒手。唉,常言道江山易改,本性難移,我胡青牛對愛妻到底也不是忠心到底。我本來決意袖手不理了,但昨晩無忌來勸我出遊,以避大禍,我心腸一軟,還是開了一張藥方,冩了什麼當歸、遠志、防風、獨活幾味藥,只因其時難姑便在我的身旁,我是不便明言的。」

\qyh{}可是難姑聰明絶頂,又懂藥性,一聽那藥方開得不合常理,一加琢磨,便識破了其中機関。她將我綁縛起來,自己取出幾味劇毒的藥物服了,説道︰『師哥,我和你做了二十多年夫妻,海枯石爛,此情不渝。可是你總是瞧不起我的毒術,不論我下什麼毒,你總是救得活。這一次我自己服了劇毒,你再救得活我,我纔眞的服了你。』我大驚失色,連聲服輸,她却在我口中塞了一個大胡桃,教我説不出話來。此後的事,你們知道了。唉,無忌,你實在太對不起我,恩將仇報,我教你逃命,你却將我愛妻打得重傷。」説著連連搖頭。

紀曉芙和張無忌面面相覷,不禁又是好氣,又是好笑,這對夫婦如此古怪,當眞天下少見,胡青牛對妻子由愛生畏,那也罷了,王難姑却是説什麼也要壓倒丈夫,到最後竟是不惜以身試毒。只聽胡青牛又道︰「你們想,我有什麼法子?這一次我如用心將她治好,那還是表明我的本事勝過了她,她勢必一生鬱鬱不樂。倘若治她不好,她可是一命歸西了。唉!只盼金花婆婆早日駕臨,將我一拐杖打死,也免得難姑煩惱了。」無忌心念一動,低聲問道︰「師母服的是什麼毒藥?如何解法?」説著連打手勢,叫胡青牛别説。胡青牛向著臉朝裡床的妻子望了一眼,明白無忌的意思,説道︰「近幾年起她下毒的本領大進,我壓根児便瞧不出她服下了什麼毒藥,如何解救,更是無從説起。」

張無忌伸出右手食指在桌上冩道︰「請冩給我看。」口中却説︰「如此説來,師母中此劇毒,那是無藥可治了。」胡青牛道︰「拙荊自己,定知解毒之法,可是我知道她的性児,她是寧死不説。」嘴裡這般説,手指却在桌上冩道︰「三蟲三草之毒,蟲爲蜈蚣、蝮蛇、毒蛛、草爲七步草、斷腸草、鎖喉菌。」跟著冩了一張藥方。無忌冩道︰「你也服此三蟲三草之毒,我來救活你。」胡青牛微一沉吟,已知其意,心想︰「此法雖然凶險,但爲解救眼前困境,只有捨命一試。」只聽張無忌道︰「先生,你醫術通神,難道師母服了什麼毒也診視不出。」胡青牛道︰「我猜想是三蟲三草的劇毒。但你想三種毒蟲性陰,三種毒草性陽,單是服了其中一項,已是極其難治,何況共服六種?若是藥物化解毒蟲的毒性。陰衰陽盛,勢必加強毒草的毒性,反之亦然,六大劇毒交攻,人是血肉之軀,如何抵抗得住?」説到這裡,揮手道︰「你們出去吧,若是難姑死了,我也決計不能獨生。」

紀曉芙和張無忌齊聲道︰「還請保重,多多勸勸師母。」胡青牛道︰「她若是聽勸,早就没有今日之事了。」説到這裡,聲音已大是哽咽。紀曉芙和張無忌當即退了出去。胡青牛反手一指,先點了妻子背心和腰間穴道,説道︰「師妹,你丈夫無能,實在治不好你的三蟲三草劇毒,只有相隨於陰曹地府,和你在黃泉做夫妻了。」説著伸手到難姑懷中,取出幾包藥末,果然不出所料,是三種毒蟲和三種毒草焙乾碾末而成。王難姑身子不能動彈,嘴裡却還能言語,叫道︰「師哥,你不可服毒。」胡青牛不加理會,將這包五色斑爛的毒粉,倒在口中,和津液嚥入肚裡。王難姑大驚失色,叫道︰「你怎能服這麼多?這許多毒粉,三個人也毒死了。」胡青牛淡淡一笑,坐在王難姑床頭的椅上,片刻之間,只覺肚中猶似千百刀子在一齊亂扎。他知道這是斷腸草最先發作,再過片時,其餘三種毒物的毒性陸續發作,那時的疼痛難熬,非人所能堪。

王難姑叫道︰「師哥,我這六種毒物是有解法的。」胡青牛痛得全身發顫,牙関上下擊打,搖頭道︰「我\dash{}我不信\dash{}我\dash{}我就要死了。」王難姑叫道︰「快服玉龍蘇合散,再用針炙散毒。」胡青牛道︰「那有什麼用?」,王難姑笑道︰「我服的毒藥粉量輕,你服的太多了,快快救治,否則便來不及了。」胡青牛道︰「我全心全意的愛你憐你,你却總是跟我爭強鬥勝,我覺得活在人世殊無意味,寧可死了,倒是一了百了\dash{}哎喲\dash{}哎喲\dash{}」這幾聲呻吟,確非假裝,其時蝮蛇和蛛蜘之毒已分攻心肺,胡青牛神智漸漸昏迷,終於人事不知。

無忌在房外聽得清清楚楚,只聽王難姑大聲哭叫︰「師哥,師哥,都是我不好,你決不能死\dash{}我再也不跟你比試了。」原來他夫妻二人情深愛重,數十年來儘管不斷鬥氣,互相却極是関切。王難姑自己死了覺得並不打緊,待得丈夫服毒自盡,却是大大的驚惶傷痛起來,張無忌搶到房中,問道︰「師母,怎地相救師父?」王難姑見無忌進來,正是見到了救星,忙道︰「快給他服玉龍蘇合散,用金針刺他的『湧泉穴』『鳩尾穴』\dash{}」便在此時,門外忽然傳進來幾聲咳嗽,靜夜之中,這幾下咳嗽的聲音清晰異常。紀曉芙搶進房中,臉如白紙,説道︰「金花婆婆\dash{}金花\dash{}」下面「婆婆」兩字尚未説出,門帘無風自動,一個弓腰曲背的老婆婆擕著個十二三歳的美貌姑娘,已站在室中,正是靈蛇島主夫人金花婆婆,却不知她二人如何進來。她見胡青牛手抱肚腹滿臉黑氣,呼吸極是微弱,轉眼便要斃命,不由得一怔,問道︰「他幹什麼?」

\chapter{間関萬里}

旁人還未答話,胡青牛雙足一挺,已暈死過去。王難姑大哭,叫道︰「你爲何這般作賤自己,服毒而死?」

金花婆婆這次從靈蛇島重赴中原,除了尋那害死她丈夫的對頭報仇之外,便是要找胡青牛的晦氣,那知她現身之時,正好胡青牛服下劇毒。她身居靈蛇島上,也是個使毒的大行家,一看胡青牛和王難姑的臉色,知他們中毒已深,無藥可救。她還道胡青牛怕了自己,以致服毒自盡,霎時間報仇之心盡去,嘆了口氣,説道︰「作孼,作孼!」擕了那小姑娘,出房而去,只聽她剛出茅舍,咳嗽聲已在數十丈外,實是不可思議。

張無忌一摸胡青牛心口,心臟尚在微弱跳動,忙取過玉龍蘇合散給他服下,又以金針刺他湧泉鴆尾等穴,散出毒氣,然後依法給王難姑施治。

忙了大半個時辰胡青牛纔悠悠醒轉。王難姑喜極而泣,連叫︰「小兄弟,全靠你救了我二人的性命。」張無忌道︰「那金花婆婆只道胡先生已服毒而死,倒是去了一件心腹大患。」他見這金花婆婆倏然而來,倏然而去,形同鬼魅,這時想起她來,猶是不寒而慄。王難姑道︰「這金花婆婆行事極爲謹愼,今日她離去了,日後必定再來査察。我夫妻須得立即避走,小兄弟,請你起兩個墳墓,碑上書明我夫妻倆的姓名。」張無忌答應了。當下胡青牛王難姑夫婦稍加收拾,坐在一輛騾車之中,乘黑離去。張無忌直送到蝴蝶谷口,這一老一少兩年多來日日相見,一旦分手,都有些依依不捨。胡青牛取出一部手冩醫書,説道︰「無忌,我畢生所學,都冩在這部醫書之中,現在送了給你。你身中玄冥神掌,陰毒難除,我心中極是過意不去,只盼你參研我這部醫書,能想出驅毒的法子,那麼咱們日後尚有相見之時。」張無忌謝過收下。王難姑道︰「你救我夫妻性命,又令我二人和好。我原該也將一生功夫,傳了給你。但我生平鑽研的是下毒傷人,你學了也無用處。只望你早日痊可,將來我再圖補報了。」

張無直等到那騾車去得影蹤不見,這纔回到茅舍。次日清晨便在屋旁堆了兩個墳墓,叫了石匠來樹立兩塊墓碑,一塊上冩蝶谷醫仙胡先生青牛之墓,另一塊冩胡夫人王氏之墓。薛公遠簡捷等見胡青牛夫妻同時斃命,才知他病重之説,果非騙人,盡皆嗟嘆。

王難姑既已遠去,不再暗中下毒,各人的傷病在張無忌診治之下,便一天好似一天,不到十日,各人陸續辭去。無忌在這幾日中,全神貫注閲讀胡青牛所著這部醫書,果見内容博大精深,奇妙微奥,不愧爲醫仙之名。他只讀了八九天,醫術已是大進,但如何驅除體内陰毒,却是不得端倪。他反來覆去的細讀數過,終於絶了盼望,不由得心灰意懶。張無忌掩了書巻,走到屋外,瞧著兩個假墓,心想︰「一年之後,我纔眞的要長眠於地下了。」言今及此,不由得泪如泉湧。忽聽得身後咳嗽了幾下,無忌吃了一驚,轉過頭來,只見金花婆婆扶著那相貌極美的小姑娘,顫巍巍的站在他身後。金花婆婆問道︰「小子,你是胡青牛的什麼人?爲什麼在他墳上哭泣?」張無忌道︰「我身中玄冥神掌的陰毒\dash{}」金花婆婆一伸手,便抓住了無忌的手腕,搭了搭他的脈博,奇道︰「是誰打你的?」無忌搖頭道︰「我也不知道。那人扮作一個蒙古兵的軍官,却不知究竟是誰。我來向胡先生求醫,他却不肯醫治。現下他已服毒而死,我的病更是好不了,是以想起來傷心。」金花婆婆見他英俊文秀,討人喜歡,却染上了這不治之症,連道︰「可惜!可惜!」

張無忌初知玄冥神掌的陰毒極難驅除之時,原是十分驚惶,但後來張三丰師徒以内功替他療治走赴少林寺求少林九陽功醫仙胡青牛潛心診療兩年,可説已竭盡天下的人力,仍是無效,他心灰意懶之下,已將一切置之度外,聽金花婆婆連説可惜,當下淡淡一笑,説道︰「生死修短,豈能強求?予惡乎知悦生之非惑邪?予惡乎知惡死之非弱喪而不知歸者邪?予惡乎知夫死者不悔甚始之蘄生乎?」

金花婆婆一怔,登時呆了,細細咀嚼他這幾句話。原來張三丰信奉道教,他的七弟子雖然都不是道士,但道家奉爲寶典的一部莊子南華經,却均讀得滾瓜爛熟。張翠山飄流了冰火島後,身無長物,無忌長到五歳時,張翠山教他識字讀書,因無書籍,只得劃地成字,將莊子教了他背熟。他適纔引這三句話,意思是説︰「我那裡知道,一個人貪生不是迷惑?我那裡知道一個人怕死,不是像幼年流落在外面而不知回歸故鄕呢?我那裡知道,死了的人不會懊悔他從前求生呢?」莊子的原意在闡明,生未必樂,死未必苦,生死其實没有什麼分别,一個人活著,不過是做大夢,死了,那是醒大覺,説不定死了之後,會覺得從前活著的時候多蠢,爲什麼不早點死了?正而你做了一個悲傷哭泣的惡夢之後,一覺醒來,懊悔這惡夢實在做得太長了。

張無忌年紀幼小,本來不懂得這些生死的大道理,但他這四年來日日都處於生死之交的邉界,時時均是可生可死,自不免體會到莊子這些話的含義。他本來並不相信莊子的話,但既然自己活在世上的日子已屈指可數,自是盼望一個人死後會别有奇境,會懊悔活著時竭力求生的可笑。

金花婆婆却從他這幾句話中想到了逝世的丈夫。他倆數年的夫妻,恩愛無比,一旦陰陽相隔,再無相見之日,假如一個人活著正似流落異鄕,死後却是回到故土,那麼丈夫被仇人害死,胡青牛不肯治丈夫的傷毒,都未必是壞事了。

只有站在金花婆婆身旁的小姑娘,却不懂無忌説了這句什麼話,不懂爲什麼婆婆一聽,便這般呆呆出神。她一雙美目瞧瞧婆婆,又瞧瞧無忌,在兩人的臉上轉來轉去。

終於,金花婆婆嘆了口氣,説道︰「幽冥之事究屬如何,總是渺茫。死雖未必可怕,但凡人莫不有死,不須強求,死亡終會到來。能彀多活一天,便多一天吧!」無忌自見到紀曉芙等十五人被金花婆婆傷得這般慘酷,又見胡青牛夫婦這般畏懼於她,甚至連逃走也無勇氣,想像這金花婆婆定是個兇殘絶倫的人物,但相見之下,却是大謬不然。那日燈下匆匆一面,並未瞧得清楚,此時却見她明明是個和藹慈祥的老婆婆。無忌心中覺察得到,她對自己的関懷親切,確是發乎眞心,決非假裝出來。

金花婆婆又問︰「孩子,你爹爹尊姓大名?她在不在這裡?」無忌當即將自己身世簡略説了。金花婆婆大爲驚訝,道︰「你是武當張五俠的令郎,如此説來,那惡人所以用玄冥神掌傷你,爲的是要迫問金毛獅王謝遜和那屠龍刀的下落了」無忌道︰「不錯,他以諸般毒刑於我身,我却是寧死不説。」金花婆婆道︰「你是確實知道的?」無忌道︰「{\upstsl{嗯}},但金毛獅王是我義父,我決計不會吐露。」金花婆婆左手一掠,已將他雙手握在掌裡。只聽得骨節格格作響,無忌雙手痛得幾欲暈去,又覺一股冰涼的寒氣,從雙手傳到胸口,這寒氣和玄冥神掌又有不同,但一樣的難熬難當。金花婆婆柔聲道︰「乖孩子,好孩児,你將謝遜的所在説出來,婆婆會醫好你的寒毒,再傳你一身天下無敵的功夫。」張無忌只痛得涕泪交流,昂然道︰「我父母捨生全義,不肯洩露朋友的行藏,金花婆婆,你瞧我是出賣父母之人麼?」金花婆婆微笑道︰「很好,很好!」潛運内勁,箍在他手上猶鐵圏般的手指又收緊幾分。張無忌道︰「你爲什麼不在我耳朶中灌水銀?爲什麼不餵我吞鐵針?四年之前,我還只是個小孩子的時候,便不怕那惡人的諸般惡刑,今日長大了,難道反而越來越不長進了?」

金花婆婆哈哈大笑,説道︰「你自以爲是個大人,不是小孩了,哈哈,哈哈\dash{}」她笑了幾聲,又劇烈的咳嗽起來,那小姑娘忙握拳替她輕輕搥背,又取出一瓶藥丸來餵了她服下。金花婆婆咳嗽漸漸止,放開了無忌的手,只見他自手腕以至手指尖,全成紫黑之色。那小姑娘向他使個眼色,道︰「快謝婆婆饒命之恩。」張無忌哼了一聲道︰「她殺了我,説不定我反而快樂些,有什麼好謝的?」那小姑娘眉頭一皺,嗔道︰「你這人不聽話,我不理你啦。」説著轉過了身子,却又偸偸用眼角覷他的動靜。

金花婆婆微笑道︰「阿離,你一個人在島上没有小伴児,無聊得緊。咱們把這娃娃抓去,叫他服侍你,好不好?就只他這股驢子脾氣,太過倔強,不容易聽話。」那叫做阿離的小姑娘長眉一軒,拍手道︰「好啊,咱們便抓了他去。他不聽話,婆婆不會想法児整治他麼?」

張無忌聽她二人一問一答,心下大急,要是金花婆婆當場將他殺死,也就自算了,倘若眞的將自己抓到什麼島上,死不死活不活的先受她二人折磨,那可比什麼都難受了。只見金花婆婆點了點頭,道︰「你跟我來,咱們先要去找一個人,辦一件事,然後一起到靈蛇島去。」張無忌怒道︰「你們不是好人,我纔不跟你們去呢。」金花婆婆微笑道︰「我們靈蛇島上什麼東西全有,吃的玩的,你見都没有見過,乖孩子,跟婆婆來吧。」張無忌突然轉身,拔足便奔,那知只跨出一步,金花婆婆又已擋在他面前,無忌發足快,收足也快,身子一側,斜刺裡向左方竄去,仍只跨出一步,金花婆婆又已擋住他面前,柔聲道︰「孩子,你逃不了的,乖乖的跟咱們走吧。」無忌咬緊牙齒,向她一掌猛擊過去。金花婆婆微微一側身,向他掌上吹了口風。無忌的手掌本已被她捏得瘀黑腫脹,這一口風吹上來,猶似用利刃再在創口劃了一刀,只痛得他直跳起來。

忽聽得旁邉一個女子的聲音叫道︰「無忌哥,你在玩什麼啊?我也來。」正是楊不悔走近身來。跟著紀曉芙也從樹叢後走了出來,她母女倆剛從田野間漫遊而歸,陡然間見到金花婆婆,紀曉芙臉色立時變得慘白,終於鼓起勇氣,顫聲道︰「婆婆,你不可難爲小孩児家?」

金花婆婆細細的眼睛一翻,向紀曉芙瞪視了一眼,冷笑道︰「你還没有死啊?我老太婆的事,要你來多嘴多舌?走過來讓我瞧瞧,怎麼到今天還不死?」紀曉芙出身武學名家,原是頗具膽氣,但這時處處要顧念到女児,已不敢輕易涉險,擕著女児的手,反而倒退了一步,低聲道︰「無忌,你過來。」無忌拔足欲行,阿離一翻手掌抓住了他小臂上的三陽絡,説道︰「給我站著。」這三陽絡一被扣住,無忌竟是半身麻軟,動彈不得,心中又驚又怒,又是奇怪,心道︰「這小丫頭不知使的是什麼邪門功夫?」

忽聽得一個清脆的女子聲音説道︰「曉芙,怎地如此不爭氣?走過去便走過去!」紀曉芙又驚又喜,回身叫道︰「師父!」但背後可並無人影,凝神一瞧,纔見遠處有一個身穿灰布袍的尼姑緩緩走來,正是峨嵋派掌門紀曉芙的授業恩師滅絶師太。她身後還隨著兩名弟子。她相隔如此之遠,面都還瞧不清楚,但説話聲傳到各人耳中便如是近在咫尺一般,足見她内力之深厚充沛。滅絶師太盛名遠播,武林中無人不知,只是她極少下山,見過她一面的人可著實不多,有些仰慕她的人上峨嵋山去拜訪,一槩擋駕不見,連張三丰那樣的人物都見不到,旁人是更加不必説了。走近身來,只見她約莫四十四五歳年紀,容貌算得甚美,但兩條眉毛斜斜下垂,使得一副面相變得極是詭異,幾乎有一點戲台上的吊死鬼味道。紀曉芙迎上去跪下磕頭,低聲道︰「師父,你老人家好。」滅絶師太道︰「還没給你氣死,總算還好。」紀曉芙跪著不敢起來。但聽得站在師父身後的丁敏君低聲冷笑,知她在師父跟前已下了不少説詞,不由得滿背都是冷汗。

滅絶師太道︰「這位婆婆叫你過去給她瞧瞧,爲什麼到今天還不死,你就過去給她瞧瞧啊。」紀曉芙道︰「是。」站起身來,大步走到金花婆婆跟前,朗聲道︰「金花婆婆,我師父來啦,你的強兇霸道,都給我收了起來吧。」金花婆婆咳嗽兩聲,向滅絶師太瞪視兩眼,點了點頭,説道︰「{\upstsl{嗯}},你是峨嵋派的掌門,我打了你的弟子,你待怎樣?」滅絶師太冷冷的道︰「打得很好啊,你愛打,便再打,打死了也不管我事。」紀曉芙心如刀割,叫道︰「師父!」兩行熱泪流了下來。須知滅絶師太向來最是護短,弟子們明明理虧,得罪了旁人,她也要強辭奪理的維護到底,這時却説出這幾句話來,那顯是不當她弟子看待了。

金花婆婆道︰「我跟峨嵋派無冤無仇,打過一次,也就彀啦。阿離,咱們走吧!」説著慢慢轉過身去。丁敏君不知金花婆婆何等來歷,見她老態龍鍾,病體支離,居然對師父如此無禮,心下大怒,一縱身,攔在她的面前,喝道︰「你也不向我師父説幾句好話,便這樣想走麼?」説著右手拔劍,離鞘一半,作威嚇之狀。金花婆婆伸出兩根手指,在她劍鞘外一捏,隨即放開,笑道︰「破銅爛鐵,也拿來嚇人麼?」丁敏君怒火更熾,便要拔劍出鞘。那知一拔之下,這劍竟是拔不出來。阿離笑道︰「破銅爛鐵,生了銹啦。」丁敏君再一使勁,仍是拔不出來,原來金花婆婆適纔在劍鞘外一捏,潛運内力,已將劍鞘捏得向内凹入,將劍鋒牢牢咬住。丁敏君拔是拔不出,就此作罷却又是心有不甘,脹紅了臉,神情極是狼狽。滅絶師太緩步上前,三根指頭挾住劍柄,輕輕一抖,劍鞘登時裂爲兩片,劍鋒脱鞘而出,説道︰「這劍算不得是什麼利器寶刀,但也還不是破銅爛鐵。金花婆婆,你不在靈蛇島上納福,却到中原來生什麼事?」

金花婆婆見到她三根手指抖劍裂鞘的手法,心中一凜,暗道︰「這賊尼聲名極大,倒是有一點眞實功夫,不妨伸量於她。」於是笑咪咪的道︰「我老公死了,一個人在島上閒得無聊,因此出來到處走走,瞧瞧有没合意的和尚道士,找一個回去作伴。」她特别説和尚道士,那自是譏刺對方身爲尼姑,却也四處走走。滅絶師太生性嚴峻,從來不與人説笑,一聽金花婆婆之言,一雙下垂的眉毛更加垂得低了,長劍一挺,道︰「亮兵刃吧!」丁敏君紀曉芙等從師以來,從未見過師父和人動過手,尤其紀曉芙知道金花婆婆的武功怪異莫測,更是関切。張無忌的手臂仍被阿離抓在手中,上身越來越麻,叫道︰「快放開我!你拉著我幹麼?」阿離見紀曉芙在旁有插手干預之勢,若不放開,她必上前動手,那時還是非放他不可,於是用力一摔,放鬆了他手臂,冷笑道︰「瞧你逃得掉麼?」

金花婆婆淡淡一笑,説道︰「峨嵋派郭襄郭女俠當年的劍法名動天下,自然是極高的,但不知傳到徒子徒孫手中,還剩下幾成?」滅絶師太道︰「就算只剩下一成,也足以掃蕩邪魔外道。」金花婆婆雙眼凝視對方手中長劍的劍尖,一瞬也不瞬,突然間,舉起手拐杖,往劍身上一點。滅絶師太焉能給她點中?長劍晃動,往她肩頭刺來。金花婆婆咳嗽聲中,舉杖橫掃。滅絶師太身隨劍走,如電光般遊到了對身後,脚步未定,劍招先到。金花婆婆却不回身,倒轉枴杖,反手往她劍刃上{\upstsl{砸}}去。

兩人都是當世武林的一流高手,三四招一過,心下均已暗讚對方了得,猛聽得{\upstsl{噹}}的一聲響,滅絶師太手中的長劍已斷爲兩截,原來劍杖相交,長劍被竟被拐杖震斷。旁觀各人除了阿離外,都吃了一驚,看她手中的拐杖黑黝黝地毫不起眼,非金非鐵,居然能{\upstsl{砸}}斷利劍,那自然是憑籍她深厚充沛的内力了。但金花婆婆和滅絶師太適纔兵刃相交,却知長劍所以斷絶,乃是靠著那拐杖的兵刃之利,並非金花婆婆功力上稍勝一籌。原來她這拐杖乃是靈蛇島旁海底的特産,叫做「珊瑚金」,是數種特異金屬混和珊瑚,在深海下歷千萬年而化成,削鐵如切豆腐,打石如敲棉花,不論多麼鋒利的兵刃,遇之立折。

金花婆婆是大有身份之人,知道滅絶師太兵刃雖斷,却未輸招,當下也不進招,只是拄{\upstsl{扙}}於地,撫胸咳嗽,紀曉芙丁敏君等三名峨嵋弟子生怕師父已受了傷,一齊搶到滅絶師太身旁照應。

阿離手掌一扇,又已抓住了張無忌的手腕,笑道︰「我説你逃不了,是不是?」這一下仍是出其不意,無忌仍是没能讓開,但覺脈門被扣,又是半身酸軟。他兩次著了這小姑娘的道児,又羞又怒,又氣又急,飛右足便要向她腰間踢去。阿離手指一加勁,無忌的右足只踢出半尺,便抬不起來了。他怒聲叫道︰「你放不放我?」阿離笑道︰「我不放,你有什麼法子?」無忌猛地一低頭,張口便往她手背上咬去。阿離只覺手上一陣劇痛,大叫一聲︰「啊{\upstsl{唷}}!」鬆開右手手指,左手的五根指爪却向無忌臉上抓到。無忌忙向後躍開,但爲勢已然不及,被她中指的指甲刺入肉裡,在右臉上深深刻劃了一道血痕,阿離右手的手背上,却也是血肉模糊,被無忌這一口咬得著實厲害。

兩個孩子在一旁打鬥,金花婆婆却目不旁視,一眼也没瞧他們。她大敵當前,焉敢分心旁鶩?只見滅絶師太抛去半截斷劍,説道︰「這是我徒児的兵刃,原不足以當高人的一擊。」説著解開背囊,取出一柄四尺來長的古劍來。但見她劍鞘上隱隱發出一層青氣,劍未出鞘,已足可想見其大爲不凡。金花婆婆一瞥眼,只見劍鞘中部用金絲鑲著兩個篆文︰「倚天!」她大吃一驚,脱口而出︰「倚天劍!」滅絶師太點了點頭,道︰「不錯,是倚天劍。」金花婆婆心頭,霎時間閃過了武林中故老相傳的那六句話來︰「武林至尊,寶刀屠龍。號令天下,莫敢不從。倚天不出,誰與爭鋒?」喃喃的道︰「原來倚天劍落在峨嵋派手中。」滅絶師太喝道︰「接招!」提著劍柄,竟不除下劍鞘,連劍帶鞘,便向金花婆婆胸口點來。金花婆婆拐杖一封,滅絶師太手腕微顫,劍鞘已碰上了拐杖。但聽得「嗤」的一聲輕響,猶如撕裂一張厚紙,金花婆婆那根海外神物,兵中至寶的「珊瑚金」拐杖,已自斷爲兩截。

金花婆婆,心頭大震,暗想︰「倚天劍刃未出匣,已是如此厲害,當眞是名不虛傳。」向著那柄寶劍凝視半晌,説道︰「滅絶師太,請你給我瞧一瞧劍鋒的模樣。」滅絶師太搖頭不允,森然道︰「此劍出匣後不飲人血,不便還鞘。」

兩人凜然相視,良久不語。適纔交換了這數招,兩人都是以深厚内力,逼住兵刃上的勁風,旁人看來只是隨手拆了幾招,絶無駭人耳目的地方,實則兩人數十年來的修爲,均已在這三四招中顯示了出來。金花婆婆知道這位尼姑的功力比自己略淺,至於招數上的神妙處,則一時還没能瞧得出來,但她既是峨嵋派的掌門,自是非同泛泛,加之手中持了這柄「天下第一寶劍」,自己決計討不好去,於是輕輕咳嗽了兩聲,轉過身來,拉住阿離,飄然而去。

丁敏君和紀曉芙等從來不知這柄武林中轟傳已久的倚天劍,竟是在師父手中,見她一擊得勝,均是大爲欣喜。丁敏君道︰「師父,這老婆不是有眼不識泰山麼?居然敢跟你老人家動手,那纔是自討苦吃。」滅絶師太正色道︰「以後你們在江湖上行走,只要聽到她的咳聲,趕快遠而避之。」原來她剛纔揮劍一擊之中,雖然削斷了對方拐杖,但出劍時還附著她修練三十餘年的「峨嵋九陽功」,這般神功撞到金花婆婆身上,却似落入汪洋大海一般,竟然無影無蹤,只帶動一下她的衣衫,竟没使她倒退一步。這時思之,猶是心有餘悸。

滅絶師太向紀曉芙道︰「曉芙,你來!」當先走到茅舍中,紀曉芙等三人跟了進去。楊不悔叫道︰「媽媽!」待要一起進去,紀曉芙知道師父這次親自下山,乃是前來清理門戸,自己素日雖蒙她寵愛,但一場重責決計無法免了,當下對女児道︰「你在外邉玩児,别進來。」張無忌心想︰「那姓丁的女子很壞,定要在她師父跟前紀姑姑的鬼話。那晩的事情我瞧得明明白白,全是這姓丁的不好,倘若她胡説八道,顚倒黑白,我便挺身而出,給紀姑姑辯明。」於是悄悄繞到茅舍之後,縮身窗下偸聽。

但聽屋中寂靜無聲,誰也没有説話,過了好半晌,滅絶師太道︰「曉芙,你自己的事,自己説罷。」紀曉芙聲音哽咽,道︰「師父,我\dash{}我\dash{}」滅絶師太道︰「敏君,你問她罷。」丁敏君道︰「紀師妹,咱們門中,第三戒是什麼?」紀曉芙道︰「戒淫邪放蕩。」丁敏君道︰「是了,第六戒是什麼?」紀曉芙道︰「戒心向外人,倒反師門。」丁敏君道︰「違戒者如何處分?」紀曉芙却不答她的話,向滅絶師太道︰「師父,這其中弟子實有説不出來的難處,並非就如丁師姊所言這般。」滅絶師太道︰「好,這裡没有外人,你就細細跟我説罷。」

紀曉芙知道今日面臨生死関頭,決不能稍帶隱瞞,便道︰「師父,六年之前,師父命咱們兄妹八人,下山分頭打探金毛獅王謝遜的下落。弟子向西行到大樹堡,在道上遇到一個身穿白衣的中年男子,約莫有四十來歳年紀。弟子走到那裡,他便跟到那裡,弟子投客店,他也投客店,弟子打尖,他也打尖。弟子初時不去理他,後來實在瞧不過眼,便出言斥責。那人説話瘋瘋癲癲,弟子忍耐不住,便出劍刺他。這人身上也没兵刃,那知武功却是絶高,三招兩式,便將我手中長劍奪了過去。」

\qyh{}我心中驚慌,連忙逃走,那白衣男子也不追來,第二天早晨,我從店房中醒來,見我長劍無端端放在我的枕邉。我自然是大吃一驚,出得客店時,只見那人又跟上我了。我想跟他動武是没用的了,只有跟他好言相懇,説道咱們非親非故,素不相識,何況男女有别,你老是跟著我有何用意,我又説我的武功雖不及你,但我峨嵋派可並不是好惹的。」滅絶師太「{\upstsl{嗯}}」了一聲,似乎認爲她説話得體。

紀曉芙續道︰「那人笑了笑,説道︰『一個人的武功分了派别,已自落了下乘。姑娘若是跟著我去,包你一新耳目,教你得知武學中别有天地。』」

滅絶師太性情孤僻,一生潛心武學,對世務殊爲膈膜,聽紀曉芙説「一個人的武功分了派别,已自落了下乘」,又説「教你得知武學中别有天地」的幾句話,不由得頗爲悠然神往,道︰「那你便跟他去瞧瞧,且看他到底有什麼古怪本事。」紀曉芙臉上一紅,道︰「師父,他是個陌生男子,弟子怎能跟隨他去?」滅絶師太登時省悟,説道︰「啊,不錯!你叫他快些滾得遠遠的。」紀曉芙道︰「弟子千方百計,躱避於他,可是始終擺脱不掉,終於爲他所擒。唉,弟子不幸,遇上了這個前生的冤孼\dash{}」説到這裡,聲音越來越低。

滅絶師太道︰「後來怎樣?」紀曉芙低聲道︰「弟子力不能拒,失身於他。他監視我極嚴,教弟子求死不得。如此過了數月,忽有敵人上門找他,弟子便乘機逃了出來,不久發覺身已懷孕,不敢向師父説知,只得躱著偸偸生了這個孩子。」滅絶師太道︰「這全是實情了?」紀曉芙道︰「弟子萬死不敢欺騙師父。」滅絶師太沉吟片刻,道︰「可憐的孩子。唉!這事原也不是你的過錯。」丁敏君聽師父言下之意,對這個師妹竟大是偏袒,不禁狠狠的向紀曉芙瞪了一眼。

滅絶師太嘆了口氣,道︰「你自己怎麼打算啊?」紀曉芙垂泪道︰「弟子由家嚴作主,本已許配於武當殷六爺爲室,既是遭此變故,這一切全顧不得了,只求師父恩准弟子出家,削髮爲尼。」滅絶師太搖頭道︰「那也不好,那個穿白衣的男子叫什麼名字啊?」紀曉芙低頭道︰「他\dash{}他姓楊,單名一個逍字。」

滅絶師太聽到楊逍兩字,突然跳起身來,袍袖一拂,喀喇喇一響,一張板桌給她擊坍了半邉。張無忌已躱在屋外偸聽,固是給她嚇得大吃一驚,紀曉芙丁敏君等三個弟子也是各各臉色大變。滅絶師太厲聲道︰「你説他叫楊逍?便是明教的大魔頭,自稱什麼『光明使者』的楊逍麼?」紀曉芙道︰「他\dash{}他是明教中的,好像在教中也有些身份。」滅絶師太滿臉怒容,問道︰「他\dash{}他躱在那裡?我去找他去。」紀曉芙道︰「他説他在崑崙山的『坐忘峰』中隱居,不過只跟弟子一人説知,江湖上誰也不知。師父既然問起,弟子不敢不答。師父,這人\dash{}這人是本派的仇人麼?」滅絶師太道︰「哼,豈僅是本派的仇人而已。你大師伯孤鴻尊者,崑崙派的名宿游龍子,便是給這個大魔頭楊逍活活氣死的。」紀曉芙心中甚是惶恐,但不自禁的也隱隱感到驕傲,孤鴻尊者和游龍子都是名揚天下的高手,居然會給「他」活活氣死。她想問其中詳情,却又是不敢出口。她們峨嵋弟子,均知師父和大師伯孤鴻尊者是師祖座下的兩大弟子,却不知這兩人情愛甚篤,原有嫁娶之約,只是孤鴻尊者中道殂逝,滅絶師太這纔削髮爲尼。

滅絶師太抬頭向天,恨恨不已,口中喃喃自語︰「楊逍,楊逍\dash{}今日總教你落在我的手中\dash{}」突然間轉過身來,説道︰「好,你失身於他,迴護彭和尚,得罪了丁師姊,瞞騙師父,私養孩児\dash{}這一切我全不計較,我差你去做一件事,大功告成之後,你回到峨嵋,我便將衣缽和倚天劍都傳了於你,立你爲本派掌門的承繼人。」

這幾句話只聽得衆人大爲驚愕,丁敏君心中更是妬恨交迸,深怨師父不明是非,倒行逆施。紀曉芙道︰「師父但有所命,弟子赴湯𨂻火,萬死不辭,至於承受恩師衣缽眞傳,弟子自知德行有虧,不敢存此妄想。」滅絶師太道︰「你隨我來。」拉住紀曉芙手腕,翩然出了茅舍,直往谷左的山坡上奔去,到了一處極空曠的所在,這纔停下。

\chapter{危如累卵}

張無忌愕然不解,但見滅絶師太站立高處,向四周眺望,然後將紀曉芙拉到身邉,輕輕在她耳旁説話,這纔知她要説的話隱祕之極,不但生怕隔牆有耳,被人偸聽了去,而且連丁敏君等兩個徒児,也不許聽到。

張無忌躱在屋後,不敢現身,遠遠望見滅絶師太説了一會話,紀曉芙低頭沉思,忽然搖了搖頭,神態極是堅決,顯是不肯遵奉師父之命。只見滅絶師太舉起手掌,便要擊落,但手掌停在半空,却不擊下,想是最後要她再作一決定。張無忌一顆心怦怦亂跳,心想這一掌擊在身上,她是決計不能活命的了。她雙眼一霎也不敢霎,凝視著紀曉芙,只見她突然雙膝跪地,却堅決的搖了搖頭。滅絶師太手起掌落,擊中她的頂門,雖因相隔遠了,聽不到聲音,但見紀曉芙身子晃也不晃,一歪便跌倒在地,顯是被她一掌擊死。張無忌又是驚駭,又是悲痛,伏在屋後長草之中,不敢動彈。

便在此時,楊不悔格格兩聲嬌笑,撲在無忌背上,笑道︰「捉到你啦,捉到你啦!」原來她在田野間閒步,瞧見無忌伏在草中,還道是跟她捉迷藏玩耍,撲過來抓住了無忌肩頭。無忌忙反手摟住她身子,一手掩住她嘴巴,在她身邉低聲道︰「别作聲,别給惡人瞧見了。」楊不悔見他面色慘白,滿臉驚駭之色,倒也嚇了一跳。

滅絶師太從高坡上急步而下,對丁敏君道︰「去將她的孼種刺死,别留下禍根。」丁敏君瞧見師父用重手法擊斃紀曉芙,雖然暗自歡喜,但也忍不住駭怕,聽得師父吩咐,忙拔出長劍,來尋楊不悔。張無忌抱著那小女孩,縮身在長草之内,連大氣也不敢喘一口。丁敏君前前後後找了一遍,不見楊不悔的蹤跡,待要細細搜尋,滅絶師太已罵了起來︰「没用的東西,連個小孩児也找不到。」她另一個弟子名叫貝錦儀,平時和紀曉芙頗爲交好,眼見她慘死師父掌底,又要搜殺她遺下的孤女,心中不忍,説道︰「我看見那個孩子似乎逃出谷外去了。」她知道師父脾氣甚急,若是在谷外找尋不到,決不耐煩回頭再找。雖然這個五六歳的小孩孤零零的留在世上,也未必能活,但總勝於親眼見她被丁敏君一劍刺死。

滅絶師太道︰「你怎麼不早説?」狠狠的白了她一眼,當先追出谷去,丁敏君和貝錦儀隨後跟去。楊不悔尚不知母親已遭大禍,圓圓的大眼骨溜溜地轉動,露出詢問的神色。

張無忌伏地聽聲,耳聽得那三人越走越遠,跳起身來,拉著楊不悔的手,奔向高坡。楊不悔笑道︰「無忌哥,惡人去了麼?咱們到山上玩,是不是?」無忌不答,見她奔跑不快,彎腰將她抱了起來,一直奔到紀曉芙跟前。楊不悔待到臨近,纔見母親倒在地上,大吃一驚,掙扎下地,大叫︰「媽媽,媽媽!」撲在母親身上。

無忌一探紀曉芙的呼吸,氣息微弱已極,但見她頭蓋骨被滅絶師太這一掌震成碎片,便是眞有神仙到來,也已難救性命。紀曉芙微微睜眼,見到無忌和女児,口唇略動,似要説話,却説不出半點聲音,眼眶中兩粒大大的眼泪滾了下來。無忌從懷中取出金針,在她「神庭」「印堂」「承位」等穴上用力刺了幾針,使她暫且感覺不到腦門的劇痛。紀曉芙果然精神一振,低聲道︰「我求\dash{}求你\dash{}送她到她爹爹那\dash{}我不肯\dash{}不肯害她爹爹\dash{}」左手伸到自己胸口,似乎要取什麼物事,突然間頭一偏,氣絶而死。

楊不悔摟住母親的屍身,只是大哭,不住口的叫︰「媽媽,媽媽,你很痛麼?你很痛麼?」紀曉芙的身子漸漸冰冷,她却兀自問個不停,她也不懂母親爲什麼一動也不動,爲什麼不回答她的話。

張無忌心中本已悲痛,再想起自己父母慘亡之時,自己也是這麼伏屍號哭,忍不住泪如泉湧。兩人哭了一陣,究竟無忌大了幾歳,心想︰「紀姑姑臨死之時,顯是求我將不悔妹子送到她爹爹那裡。我只知她爹爹名叫楊逍,是明教中的光明使者,住在崑崙山的什麼坐忘峰中。我除了將她送去之外,也没别法。」他也不知崑崙山在極西數萬里外,他兩個孩子如何去得?又想起紀曉芙斷氣時曾伸手到胸口取什麼物事,於是在她頸中一摸,取出一塊黑黝黝的鐵牌來,牌上彫著一張牙舞爪的魔鬼,那鐵牌穿著一根繩子,掛在她的頸中。

張無忌也不知那是什麼東西,除了下來,便掛在楊不悔頸中,到芧舍中取過一柄鐵鏟,挖了個坑將紀曉芙的屍身埋了。這時楊不悔已哭得筋疲力盡,沉沉睡去,待得醒來,無忌費盡唇舌,纔騙得她相信媽媽已飛了上天,要過很久很久,纔從天上下來跟她相會。

當時無忌胡亂煮些菜飯吃了,疲倦萬分,橫在榻上便睡,次日醒來,收拾了兩個小小包裹,帶著不悔到她母親墳前拜了幾拜,兩個孩児便離蝴蝶谷而去。無忌没有防身刀劍,本想執拾金花婆婆遺下的半截「珊瑚金」拐杖,倒是一件利器,但此刻遍尋不見,想是已被丁敏君順手牽羊帶走。

無忌在紀曉芙留下的包袱中,找到了七八兩銀子,他雖不知崑崙山究竟有多遠,但想這寥寥幾兩銀子不足盤纏之用,那是一定的了,也只有走一步算一步。兩人走了大半日,方出蝴蝶谷,楊不悔脚小步短,已走不動了。歇了好一會,纔又趕路,行行歇歇,第一晩便找不到客店人家,一直行到天黑,還是在荒山野嶺中亂闖,四下裡狼{\upstsl{嗥}}梟啼,只嚇得楊不悔不住驚哭。無忌心下也是十分害怕,當此情境,只有強作英雄好漢,見路旁有個山洞,便拉著不悔,躱在洞裡,將她摟在懷裡,伸手按住她的耳朶,令她聽不見餓哭吼叫之聲。

這一夜兩個孩子又餓又怕,挨了一晩苦,次晨纔在山中摘些野果吃了,順著山路行到傍晩,楊不悔突然尖聲大叫,指著路邉一株大樹。無忌一看,嚇得拉著不悔轉頭狂奔。原來樹上兩個僵屍,飄飄蕩蕩的掛在那裡,兩人七高八低的没奔出十餘步,脚下石子一絆,一齊摔倒。無忌大著膽子回頭一望,這一下更是吃驚,脱口而出︰「胡先生!」原來掛在樹上的一個僵屍這時被風吹得回過頭來,正是胡青牛。另一個僵屍長髮披背,是個女屍,瞧她服色,正是胡青牛的妻子王難姑。暮色蒼茫之中,山風動衣,更加顯得陰氣森森。

無忌定了好一會神,自己安慰自己︰「不怕,不怕!」慢慢爬起身來,一步步走近,果見掛著的兩個屍體,正是胡青牛夫婦。兩人臉頰上金光燦然,各自嵌上一朶小小的金花。無忌心下恍然︰「原來他們還是没逃出金花婆婆的毒手。」只見山澗中一輛騾車摔得破爛不堪,一頭騾子淹死在澗水之中。

眼見天色不早,已不能再走,索性便在大樹旁和楊不悔睡下。睡到半夜,猛聽得有野獸撕打鬥咬,無忌一驚而醒,目光下只見五六只豺狼正在嗚嗚低{\upstsl{嗥}},爭食死騾。他急忙負起不悔,爬上樹幹,衆豺狼聽到聲音,在樹下團團打轉,轉了一會,又去嚼食死騾,終不死心,再爬到樹下打轉。直到天色大明,衆豺狼纔一齊散去。

無忌瞧清楚衆豺狼確是遠去,不再回轉,於是負著不悔從樹上下來,瞧著那血肉狼藉的死騾,心想︰「群狼若不是先見死騾,一齊爭食,咱兩個這時也早成爲狼肚中的食物了。」解開繩索,將胡青牛夫婦的屍身從大樹上放了下來,忽然拍的一聲響,王難姑屍身的懷中跌出一冊,無忌拾起一看,原來是一部手冩的冊子,題簽上冩著「毒物大全」四字。

張無忌翻開書來,只見書頁上滿是蠅頭小楷,密密麻麻的冩著各種毒物的質素、使用和化解的法子,毒藥、毒草等等,那是不必説了,各項活物如毒蛇、蜈蚣、蝎子、毒蛛,以及種種希奇古怪的蟲豸鳥獸,無不具載。他隨手放在懷裡,將胡青牛夫婦的屍體並列了,捧些石頭上的土塊,草草堆成一墳,跪倒拜了幾拜,擕了楊不悔的手覓路而行。

午後走上了大路,不久到了一個小市鎭,無忌本想買些飯吃,那知市鎭中家家戸戸都是空屋,竟連一個人影也無。無忌無奈,只得繼續趕路,但見沿途稲田盡皆龜裂,此時正當秋收之候,但田中長滿了荊棘敗草,一片荒涼。無忌心中慌亂,偏生楊不悔是個什麼也不懂的小孩,能彀忍飢不哭,勉力行走,已可算是極乖,還能出什麼主意?走了一會,只見路邉臥著幾具屍體,肚腹乾癟,雙頰深陥,一看便知是餓死了的。越走這類餓殍越多,無忌心下惶恐︰「難道什麼東西也没得吃的?咱倆也要這般餓死不成?」

行到傍晩,到了一處樹林,只見林中有煙嬝嬝升起。張無忌大喜,他自離開蝴蝶谷後,一路未見人煙,當下向白煙升起處快步走去。走到憐近,只見五個衣衫襤褸的漢子,圍著一鍋熱氣騰騰的沸湯,正在鍋底添柴加火。那些漢子聽到脚步聲,一齊回過頭來,見到張無忌和楊不悔兩人,臉上現出了大喜過望之色。兩名漢子「啊」的一聲,跳起身來。一人招手道︰「小娃娃,好極,過來,快過來。」張無忌道︰「我們一路未得飲食,請大叔分些飯菜,當以銀子相謝。」一個大漢笑道︰「你還有銀子麼?先拿出來瞧瞧。」無忌從懷中取出一錠銀子,那大漢挾手便奪了去,叫道︰「很好!你同來的大人呢?他們到那裡去了?」無忌道︰「就只我們二人,没有大人相伴。」

那五個大漢哈哈大笑,其中二人拍手唱起歌來。無忌餓得慌了,探頭到鍋中一看,瞧是煮些什麼,只見鍋中上下翻滾的,都是些青草。一名漢子一把揪過楊不悔,獰笑道︰「這口小羊又肥又嫩,今晩飽餐一頓,那是舒服得緊了。」另一個漢子道︰「不錯,男的娃娃留著明児吃。」無忌大吃一驚,喝道︰「你幹什麼?快放開我妹子。」那漢子理也不理,嗤的一聲,便撕破了楊不悔身上衣服,手一伸,從靴子裡拔出一柄牛耳尖刀來,笑道︰「很久没吃這麼肥嫩的小羊了。」提著楊不悔,便到一旁去宰殺。又有一名漢子拿了一隻土缽,跟在身後,説道︰「羊血丟了可惜,煮一鍋羊血羹,味児纔不壞呢。」

張無忌只嚇得魂飛大外,瞧他們並非説笑,實是眞有宰殺楊不悔之意,大叫︰「你們想吃人麼?也不怕傷天害理?」一名瘦得只剩皮包骨的漢子笑道︰「老子有三個月没吃一粒米了,不吃人,還能吃牛吃羊麼?」生怕無忌逃跑,過來伸手便揪他頭頸。無忌側身讓開,左手一帶,右掌拍的一下,擊在他後心腰間。他武功得自金毛獅王謝遜的親傳,又自父親處學得武當心法,這幾年中雖然潛心醫術,没有用功練武,但生平所習所見,盡是最上乘的武功,這一掌隨手擊下,便是一個習武多年的武師,也自抵不住,何況一個村漢?那漢子哼了一聲,俯身在地,一動也不能動了。無忌身子一縱,躍到楊不悔身旁。那漢子喝道︰「先宰了你!」提起尖刀,便往無忌胸口插下。

無忌飛起右脚,正中那人手腕,那人手上一痛,尖刀脱手飛出。無忌一招鴛鴦連環腿,左足跟著踢出,直中那人下顎。那人正在張口呼喝,下顎被踢得急速合上,將自己半截吞尖咬了下來,狂噴鮮血,暈死過去,無忌忙扶起楊不悔,便在此時,只聽得背後脚步聲響,又有兩人向背上撲到。

張無忌身子一閃,兩人已然撲空。無忌一手一個,抓住兩人背脊向裡一合,砰的一響,兩人天靈蓋撞天靈蓋,同時昏去。餘下一名大漢終欺無忌年幼,雖見他身手敏捷,却也並不忌憚,拔出腰刀,惡狠狠的砍殺上來。無忌雙手空空,微感驚慌,左閃右避,躱開了他砍來的三刀。那人第四刀使力更猛,無忌側身一讓,那人一刀没砍中,身子便向前一跌,無忌得到良機,順手一掌,擊中在他臂部。這一尚借勢借力,那漢子身子飛了起來,{\upstsl{噗}}的一聲,水花四濺,正好没頭没腦的倒栽在鐵鍋之中。一大鍋青草湯正在煮得沸騰翻滾,他這一摔下去,滿鍋熱湯全罩在頭臉之上,那漢子「啊」的一聲都没呼出,眼見是燙得不活了,正是害人不到,反害自己。

若是事先跟無忌説明,要他和這五個漢子放對,他是萬萬不敢的,須知他雖自幼習練武功,却並不知自己所學到底能管什麼用。但楊不悔被人抓住,明晃晃的尖刀對準她的胸口,稍一遲疑,這個小妹子即成俎上魚肉,那裡還有猶豫的餘地?豈知奮力應戰之下,那五個漢子竟是不堪一擊,他驚魂稍定,自己也不禁呆了。

便在此時,只聽得脚步聲響,又有幾人走進林來。楊不悔是驚弓之鳥,一聽見人聲,便撲在無忌懷裡。無忌抬頭一看,一顆心登時放下,叫道︰「是簡太爺、薛太爺。」原來進來的也是五人,一個是崆峒派的簡捷,另外是華山派的薛公遠和他個同門,這四人都是無忌手上治好的。最後一人是個二十歳上下的青年漢子,貌相威壯,額頭奇闊,無忌却没見過。

簡捷「哼」了一聲,道︰「張兄弟,你也在這裡?這幾人怎麼了?」説著手指被無忌打倒的五名漢子,無忌氣憤憤的説了,最後道︰「連活人也敢吃,那不是無法無天了麼?」簡捷橫眼瞧著楊不悔,突然嘴角邉滴下饞涎,伸舌頭在嘴唇上舐了舐,自言自語道︰「他媽的,五日五夜没一粒米下肚了,儘喟些樹皮草根\dash{}{\upstsl{嗯}},細皮白肉,肥肥嫩嫩的\dash{}」張無忌見他眼中射出飢火,像是頭餓狼一般,裂開了嘴,牙齒一亮一亮,神情甚是可怖,忙將楊不悔摟在懷裡。薛公遠道︰「這女孩的媽媽呢?」張無忌心想︰「我若説紀姑姑死了,他們更會轉壞念頭。」便道︰「紀女俠去買米去啦,轉眼便來。」楊不悔忽道︰「不,不,我媽媽飛上天去啦!」

簡捷和薛公遠等人閲歷何等豐富,一聽兩人的話,便知紀曉芙已死。薛公遠冷笑道︰「買米?周圍五百里地内,你給我找出一把米來,算你是本事。」簡捷向薛公壤打個眼色,兩人霍地躍起。簡捷兩雙手猶似鐵鉤一般,牢牢抓住了張無忌雙臂。薛公遠左手掩住楊不悔的嘴,右臂便將她抱了起來。無忌驚道︰「你們幹什麼?」簡捷笑道︰「鳳陽府赤地千里,大夥児餓得熬不住啦。這女孩児又不是你什麼人,待會児也分你一份吃便是。」無忌罵道︰「你們枉自身爲英雄好漢,怎能欺侮孤女幼弱?這事傳揚開去,你們還能做人麼?」簡捷大怒,左手抓住他雙臂,右手挾臉打了他兩拳,喝道︰「連你這小畜生也一起宰了,咱們本來嫌一隻小羊不彀吃的。」

張無忌適纔舉手投足之間,擊倒五名漢子,甚是輕易,但聖手伽藍簡捷是崆峒派的高手,一雙手上練了數十年的功夫,無忌被他緊緊抓住了,却那裡掙扎著脱?薛公遠的兩個師弟取過繩索,將無忌和楊不悔都縛了。無忌知道今日已然無倖,狂怒之下,好生後悔,當初實不該救了這幾人的性命,那料到人心反覆,到頭來竟會恩將仇報。簡捷罵道︰「小畜生,你治好了老子頭上的傷,你就算於老子有恩,是不是?你心中一定罵老子,是不是?」

張無忌道︰「這還不是恩將仇報?我和你們無親無故,可是若非我出手相救,你們四人的奇傷怪病能治得好麼?」薛公遠笑道︰「張少爺,咱們受傷之後,醜態百出,什麼怪模怪樣,都讓給你瞧在眼裡麼啦,傳將出去,大夥児在江湖上也不好做人。今児咱們實在餓得慌了,没幾口鮮肉下肚,性命也是活不成的,你救人救到底,送佛送上天,再救咱們一救吧。」簡捷惡狠狠的猙獰可怕,倒也罷了,這個薛公遠嘻嘻的陰險狠毒,張無忌瞧著尤其覺得寒心,大聲道︰「我是武當子弟,這個小妹子是峨嵋派的,你們害了我二人不打緊,武當五俠和滅絶師太能便此罷休嗎?」

簡捷一愕,哦了一聲,覺得這句話倒是不錯,武當派和峨嵋派的人可當眞惹不起。薛公遠笑道︰「這裡天知地知,你知我知,等到你到了我肚子裡,你再去向張三丰老道訴苦吧。」簡捷哈哈大笑,道︰「肚裡餓得冒出火來啦,你便是我親兄弟,親児子,我也一口吞了你。」轉頭向薛公遠的兩個師弟喝道︰「快生火燒湯啊。還等什麼?」那二人提起打翻在地的鐵鍋,一個到溪去搯水,另一個便生起火來。

張無忌道︰「薛大爺,那個人反正已燙死,你們肚餓要吃人,吃了他不麼?」薛公遠笑道︰「這幾條死漢子全身皮包骨頭,又老又韌,又臭又硬,天下那有不吃嫩羊吃老羊的道理。」無忌自來極有骨氣,若是殺他打他,決不能討半句饒,但這時身陥歹人之手,竟要被人活生生的煮來吃了,不由得張惶失措,哀求了幾句,薛公遠反而不住嘲笑︰「哈哈,武當峨嵋的弟子在江湖上逞強稱霸,今日却給咱們一口一口的咬來吃了,張三丰和滅絶老尼知道了,不氣死纔怪。」張無忌提氣大喝︰「薛大爺,你們既是非吃不可,將我張無忌吃了吧,只求你們放了這個妹子,我張無忌死而無怨。」薛公遠道︰「爲什麼?」張無忌道︰「她媽媽去世之時,托我將這個小妹子交給她爹爹。你們吃了我,已足裹腹,明日買到牛羊米飯,就饒了這小姑娘吧。」簡捷見他臨危不懼,尚守信義,不禁心動,道︰「怎樣?」薛公遠道︰「饒了小女娃娃不打緊,只是洩漏了風聲,日後宋遠橋兪蓮舟他們找上門來,簡大哥有把握打發便成。」簡捷點頭道︰「我是個胡塗蛋,從不想想往後的日子。」只見那名華山派弟子提了一鍋清水回來,張無忌知道事情緊急,叫道︰「不悔妹妹,你向他們發個誓,以後決不説不出今日的事來。」楊不悔迷迷糊糊的哭道︰「不能吃你啊,不能吃你啊。」她也不懂張無忌説些什麼,隱隱約約之間,知道他是在捨身相救自己。那個氣槩軒昂的青年男子一直默然坐在一旁,不言不動。簡捷向他瞪了一眼,道︰「徐小舍,想吃羊肉,也得惹一身羊騷氣啊。」濠泗一帶,對年青漢子,稱爲「小舍」。那青年道︰「是!」從腰間拔出短刀,説道︰「殺豬屠羊,是我的拿手本事。」將短刀橫咬在口中,一手提了張無忌,一手提了楊不悔,向山溪邉走去。無忌破口大罵,想張口去咬他手臂,却咬不到。他走出十餘步,薛公遠忽然叫道︰「徐小舍,便在這児開剝吧。」那徐小舍回頭道︰「在溪水中開膛破肚的好,洗得乾淨些。」口中咬了刀子説話糢糊不清,脚下並不停步。薛公遠道︰「我叫你在這裡,便在這裡。」原來他老奸巨猾,瞧出那徐小舍神情有些不對,生怕他一個人獨吞。

徐小舍低聲道︰「快逃!」將兩人在地下一放,伸刀割斷了縛住二人的繩索。張無忌道︰「多謝救命大恩。」拉著不悔的手,拔足飛奔。簡捷和薛公遠齊聲怒吼。縱身追去,那徐小舍橫刀攔住,喝道︰「給我站住!」

簡捷和薛公遠見那徐小舍橫刀當胸,威風凜凜的攔在面前,倒是一怔。簡捷喝道︰「幹什麼?」那徐小舍道︰「咱們在江湖上行走,欺侮弱小,不叫天下好漢笑話麼?」薛公遠怒道︰「餓得急了,娘老子也吃。」揮手向兩個師弟喝道︰「快追,快追!」

張無忌見楊不悔跑不快,將她橫抱在手裡急奔,他本已人小步短,這麼一來,逃得更慢了,未出樹林,便給兩名華山派的弟子追上,無忌將楊不悔往地下一放。反手便是一掌,去勢甚是勁急。一人舉掌一擋,拍的一響,竟是將他震得倒退了幾步,那人吼道︰「小雜種,倒厲害啊!」兩人一齊拔出單刀,砍了過來。無忌豁出了性命,在兩人刀鋒之中搶攻,不住口的叫楊不悔快逃。

那邉簡捷和薛公遠也是各挺兵刃,夾攻那姓徐的青年漢子。這漢子餓了幾天,早已有氣無力,不似簡薛二人,沿途殺人劫掠,雖然飢餓,却比他強得多了。鬥了一陣,簡捷刷的一刀,砍在那漢子腿上,登時鮮血淋漓。那漢子抵敵不住,手中兵刃又短,眼見再打下去,勢非命送當場不可,突然提起短刀,向薛公遠擲去。薛公遠側身一避,那漢子便衝了出去。簡薛二人也不追趕,逕自來捉張楊二小。那漢子遠遠的叫道︰「張兄弟休慌,我去叫幫手來救你。」簡薛二人上前合圍,登時將張無忌和楊不悔又縛住了。

簡捷瞪眼罵道︰「這姓徐的吃裡扒外,不是好人,你們怎地跟他做一路?」薛公遠道︰「路上撞到的同伴,誰知他是好人壞人?他説姓徐,叫什麼徐達。你别信他鬼話,天都快黑了,到那児去叫幫手去。」一名華山派的弟子道︰「聽他口音,是鳳陽府本地人,便叫些鄕下人來,咱們也不怕。」簡捷笑道︰「鳳陽府的人,哈哈,個個餓得爬也爬不動了。咱們快把兩口小羊煮得香香的,飽餐一頓是正經。」

張無忌二次被擒,被打得口鼻青腫,衣衫都扯破了,懷中銀兩物品,都撒在地上,他心中想︰「原來這位姓徐的大哥叫做徐達,此人豪氣干雲,實是個好朋友,只可惜我命在頃刻,不能和他相交了。」一低頭,只見一本黃紙抄本掉在地下,書頁隨風翻動,正是從王難姑屍身上取來的那部「毒物大全」。無忌明知無倖,倒也不再作求生之想,順眼往書頁上瞧去,只見那部書正翻到「毒菌」一項,文中詳載各種厲菌的形狀、氣味、顏色、毒性、解法、一種又是一種,他心中正亂,那裡看得入腦?突然間眼角一瞥之間,只見左首四五尺之外,一段腐朽的樹幹之下,正生著十餘株草菌,顏色鮮艷奪目。無忌心中一動︰「這些草菌不知叫什麼名稱,不知有毒無毒?但瞧那毒書上所載,大凡毒菌均是顏色鮮明,這些草菌若是劇毒之物,不悔妹妹尚有活命之望。」他這時也不想自己求生,心想自己反正體内寒毒難除,今日便是逃得性命,也不過多活得幾個月,一意只盼能救得楊不悔,完了紀曉芙臨終時的囑託。他移動雙脚臀部,慢慢挨將過去,轉過身來,伸手將那些草菌都採摘下來,這時天色極黑,各人飢火中燒,誰也没留心他。張無忌忽然眼望徐達逃去之處,跳起身來,叫道︰「徐大哥,你帶了人來啦,救命,救命!」簡捷等信以爲眞,四人抓起兵器,都跳了起來。無忌乘四人凝視東首,倒退兩步,反手將那些草菌都投在鐵鍋之中。簡捷等不見有人,都罵道︰「小雜種,你想瘋了也没人來救你。」薛公遠道︰「開刀了,誰來動手?」簡捷道︰「我宰女娃子,你宰那男的。」説著一把揪起了楊不悔。無忌道︰「薛大爺,我口渴得緊,你給我喝碗熱湯,我死了做鬼也不纏你。」薛公遠笑道︰「好,渴碗熱湯打什麼緊?」便搯了碗熱湯給他。一口碗熱湯尚未送到嘴邉,張無忌大聲讚道︰「好香,好香!」那些草菌在熱湯中一熬,果然是香氣撲鼻。簡薛衆人餓得早就急了,聞到菌湯也不拿去餵無忌,自己喝了下肚,舐了舐嘴唇,道︰「鮮美得緊!」又去搯了一碗。簡捷挾手搶過,大口喝了一碗,興猶未盡,又喝了一碗。接著華山派的兩名弟子每人都喝了兩碗,久飢之下,兩碗熱騰騰的鮮湯下肚,均感説不出的舒服。簡捷還撈起鍋中草菌,大口咀嚼,誰也没問這些草菌從何處而來。

簡捷吃完草菌,拍了拍肚子,笑道︰「先打個底児,再吃羊肉。」左手提起楊不悔後領,右手提了刀子。張無忌見衆人喝了菌湯後若無其事,心想原來這些草菌無毒,不禁暗暗叫苦,簡捷走了兩步,忽然叫道︰「啊喲!」身子一晃,摔倒在地,將楊不悔和刀子都抛在一旁。薛公遠驚道︰「簡兄,怎麼啦?」奔過去俯身一看。這一彎腰,他再站不直身子,撲在簡捷身上。那兩名華山派弟子哼也没哼一聲,跟著便毒發而斃。

張無忌大叫︰「謝天謝地!」滾到刀旁,反手執起,將楊不悔手上的繩索割斷。楊不悔顫著雙手,把無忌的手掌刺破了兩處,這纔割斷他手上繩索。兩人死裡逃生,歡喜無限,摟抱在一起。過了一會,張無忌去看簡薛四人時,只見每人臉色發黑,飢肉扭曲,死狀甚是可怖,心想︰「毒物能殺人,也就是能救好人。」當下將那部「毒物大全」珍而重之的收在懷内,決意日後要好好研讀。無忌擕了楊不悔的手,穿出樹林,正要覓路而行,忽見東首火把照耀,有七八人手執軍器,快步奔來。張楊二人是驚弓之鳥,忙在大樹後的草叢中一躱。那干人奔到鄰近,只見當先一人正是徐達。他一手高舉火把,一手挺著長槍,大聲{\upstsl{吆}}喝︰「傷天害理的吃人惡賊,快納下命來!」衆人奔進樹林,見簡薛等四人死在當地,無不愕然。徐達叫道︰「張兄弟,你没事麼?我救你來啦!」無忌見他肝膽照人,不由得熱泪盈眶,叫道︰「徐大哥,兄弟在這裡!」從草叢中奔出。徐達大喜,一把將無忌抱起,説道︰「張兄弟,似你這等俠義之人,别説孩童,大人中也是少見,我生怕你已傷於惡賊之手。不料好有好報,惡有惡報,正是報應不爽。」問起簡,薛等人如何中毒,無忌説了毒菌煮湯之事,衆人又都讚他聰明。

徐達道︰「這幾位都是我從小交好的朋友,今日宰了一條牛,正好在皇覺寺中煮食,我去一叫便來。但若不是張兄弟機智,咱們還是來得遲了。」當下替無忌一一引見。一個方面大耳的姓湯名和;一個英氣勃勃的姓鄧名愈;一個黑臉長身的姓花名雲;兩個白淨面皮的是一對兄弟,兄長叫作吳良,兄弟名叫吳禎。最後是個和尚,此人相貌大是醜陋,下巴向前挑出,猶如一柄鐵鏟相似,臉上凹凹凸凸,甚多瘢痕,雙目深陥,却是炯炯有神。徐達道︰「這位朱大哥,名叫元璋,現在皇覺寺出家。」花雲笑道︰「他做的是風流快活和尚,不愛唸經拜佛,整日便吃喝酒吃肉的。」楊不悔見了朱元璋的醜相,心中害怕,躱在無忌背後。朱元璋笑道︰「和尚雖然吃肉,却不吃人,小妹妹不用害怕。」湯和道︰「咱們在廟裡煮的那鍋牛肉,這時候也該熟了。」花雲道︰「快走!小妹妹,我來揹你。」將楊不悔負在背上,大踏步便走。無忌見這干人豪爽快活,心中也自歡喜,走了四五里路,來到一座廟宇。穿過大殿,便聞到了一陣燒牛肉的香氣。吳良叫道︰「熟啦,熟啦!」徐達道︰「張兄弟,你在這児歇歇,咱們去端牛肉出來。」

\chapter{鐵琴先生}

張無忌和楊不悔並肩坐在大殿的蒲團上,朱元璋、徐達、湯和、鄧愈等七手八脚捧出一盆一缽的牛肉來。吳良、吳禎兄弟提了一{\upstsl{罈}}白酒,大夥児便在菩薩面前,歡呼暢飲。無忌和不悔已餓了數日,此時有牛肉下肚,自是説不出的暢快。花雲道︰「徐大哥,咱們的教規什麼都好,就是不許人吃肉,未免有點児那個。」無忌心中一凜︰「原來他們都是明教的。明教的規矩是食青菜,拜魔王,他們却在大吃牛肉。這當児無米無菜,不吃肉難道餓死麼?」鄧愈拍手道︰「徐大哥的話從來最有見地,吃啊,吃啊!」

正吃喝間,忽然門外脚步聲響,跟著有人敲門,湯和跳起身來,叫道︰「啊也!張員外家中尋牛來啦!」只聽得廟門被人一把推開,走進來兩個挺胸凸肚的豪僕,一人叫道︰「好啊!員外家的大牯牛果然是你們偸吃了!」説著一把揪住朱元璋。另一人道︰「你這賤和尚,今児賊贓倶在,還逃到那裡去?明児送你到府裡,一頓板子打死你。」朱元璋笑道︰「當眞是胡説八道,你怎能胡賴咱們偸了員外的牯牛?出家人吃素唸佛,你賴我吃肉,這不罪過麼?」那豪傑指著盤缽中的牛肉,喝道︰「這還不是牛肉?」朱元璋使個眼色,笑嘻嘻的道︰「誰説是牛肉?」吳良、吳禎兄弟走到兩名豪傑身後,一聲{\upstsl{吆}}喝,抓住了兩人手臂,登時令他們動彈不得。

朱元璋從腰間拔出一柄匕首,笑道︰「兩位大哥,實不相瞞,咱們吃的不是牛肉,乃是人肉。今日既給你們見到,只好吃了兩位滅口,以免洩漏。」嗤的一聲,將一名豪僕胸口的衣服劃破,刀尖刺得胸膛上現出一條血痕。那豪僕大驚,雙膝麻軟,連叫︰「饒\dash{}饒命\dash{}」朱元璋抓起一把牛肉,分别塞在二人口中,喝道︰「吞下去!」兩人嚼也不敢嚼,便吞了下肚。朱元璋走到厨下,抓了一大把牛毛,又分别塞在二人口中,喝道︰「快吞下!」二人只得苦著臉又吞下了。朱元璋笑道︰「你去跟員外説,是我偸宰了他的牯牛,咱們破肚開膛對質,瞧是誰吃了牛肉,連牛毛也没拔乾淨。」翻轉刀子,用刀背在那人肚腹上一拖,那人只覺冷冰冰的刀子在肚子劃過,嚇得尖聲大叫。

吳氏兄弟哈哈大笑,抬腿在兩人屁股上用力一脚,踢得兩人直滾出殿外,衆人放懷大吃,笑罵兩名豪僕自討苦吃,平日仗著張員外的勢頭,欺壓鄕人,這一次害怕剖肚對質,決計不敢向員外説衆人偸牛之事。無忌又是好笑,又是佩服,心道︰「這姓朱的和尚容貌雖然難看,行事却是乾淨爽快,制得旁人半點動彈不得,手段好生厲害。」

湯和、鄧愈等早聽徐達説過,知道張無忌甘捨自己性命,相救楊不悔,都喜愛他是個俠義少年,不以尋常児童相待,敬酒敬肉,就當他是好朋友一般。飲到酣處,鄧愈嘆道︰「咱們漢人受胡奴欺壓,受了一輩子的骯髒氣,今日弄到連苦飯也没一口吃,這種日子,如何再過得下去?」花雲拍腿叫道︰「眼見鳳陽府已死了一半百姓,我看天下到處都是一般,與其眼睜睜的餓死,不如跟韃子拚一拚。」徐達朗聲道︰「今日人命賤於豬狗,這位小兄弟小妹妹險些便成了旁人肚中之物。普天之下,不知有多少良民百姓成爲牛羊?男子漢大丈夫不能救人於水火之中,活著也是枉然。」湯和也道︰「不錯。咱們今日運氣好,偸到一條牯牛宰來吃了,明日未必再偸得到。再説,天下的好漢子大多衣食不週,難道叫英雄豪傑都去作賊?」各人越説越是氣憤,破口大罵韃子官兵害人。朱元璋道︰「咱們在這児千賊萬賊的亂罵,又罵得掉韃子一根毛髮麼?是有骨氣的漢子,便殺韃子去!」湯和、鄧愈、花雲、吳氏兄弟等齊聲叫了起來︰「去,去!」徐達道︰「朱大哥,你這勞什子的和尚也不用當啦,你年紀最大,大夥児都聽你的話。」朱元璋也不推辭,説道︰「今後咱們同生共死,有福同享,有禍同當。」衆人一齊拿起酒碗喝乾了,拔刀砍桌,豪氣橫飛。

楊不悔瞧著衆人,不懂他們説些什麼,心中暗暗害怕。張無忌却想︰「太師父一再叮囑,叫我決不可和魔教中人交好。可是常遇春大哥和這位徐大哥,都是魔教中人,比之簡捷、薛公遠這些名門正派的弟子,爲人却好上萬倍了。」他對張三丰向來敬服之極,然從自身的經歷而言,却覺太師父對魔教中人不免心存偏見。雖然如此,仍想太師父的言語不可違拗。朱元璋道︰「好漢子説做便做,這會児吃得飽飽的,正好行事。張員外家今日宴請韃子官兵,咱們先去揪來殺了。」花雲道︰「妙極!」提刀站了起來。徐達道︰「且慢!」到厨下拿了一雙籃子,裝了十四五斤熟牛肉,交給張無忌,説道︰「張兄弟,你年紀太小,不能跟咱們幹這殺官造反的勾當。咱們這幾個人人窮得精打光,身上没半分銀子,只好送這幾斤牛肉給你。若是咱們僥倖不死,日後相見,大夥児好好再吃一頓牛肉。」無忌接過籃子,説道︰「但盼各位建立大功,趕盡韃子,讓天下百姓都有飯吃。」朱元璋、徐達湯和等聽了他這幾句話,都是心中一凜,説道︰「張兄弟,你説得眞對,咱們後會有期。」説著各挺兵刃,出廟而去。

無忌心想︰「他們此去是殺韃子,若不是帶著這個小妹子,我也跟他們一起去了。他們只有七個人,倘是寡不敵衆,張員外家中的韃子和莊丁定前來追殺,這廟中是不能住了。」於是挽了一籃牛肉,和楊不悔出廟而去。黑暗中行了五六里,猛見北方火光衝天而起,火勢甚烈,知是朱元璋、徐達等人得手,已燒了張員外的莊子,心中甚喜。當晩兩人在山野間睡了半夜,次晨又向西行。

沿途風霜飢寒之苦,那也是説之不盡,幸好楊不悔的父母都是武學名家,在娘胎裡時體質便極壯健,因此一個小小女孩長途跋涉,居然没有生病,便有輕微風寒,無忌採些草藥,隨手便給她治好了。但兩個小孩,每日行行歇歇,最多也不過走上二十里地,行了十五六天,方到河南省境。那河南境内,和安徽也是無多分别,處處飢荒,遍地都是餓死的死屍。張無忌做了一副弓箭,仗著學過武藝,或射飛禽,或殺走獸,飽一天餓一天的,和楊不悔慢慢西行。幸好途中没遇到蒙古官兵,也没逢到江湖人物,至於尋常無賴奸徒,想打這兩個孩子的主意,却那裡是無忌的對手?有一日他跟途中遇到的一個老人閒談,問起崑崙山坐忘峰的所在,這老人雙目圓睜,驚得呆了,説道︰「小兄弟,崑崙山距此何止十萬八千里,聽説當年只有唐僧取經,這纔去過。你們兩個娃娃,不是發瘋了麼?你家裡在那裡,快快回家去吧!」

張無忌一聽之下,不禁氣沮,暗想︰「崑崙山這麼遠,那是去不成的啦,只好到武當山見太師父再説。」但轉念又想︰「我受人重託,雖然路途艱險,怎能中途退縮?我壽命無多,倘若不在身死之前將不悔妹妹送到,多耽擱一天,便是對不起紀姑姑。」也不再跟那老人多説,拉著楊不悔的手便行。如此又行了二十餘天,兩個孩子早是全身衣衫破爛,面目憔悴,那也罷了,無忌最爲煩惱的,却是楊不悔時時吵著要媽媽,找不到媽媽,往往便哭泣半天。張無忌多方譬喩開導,説這一路西去,便是去尋她媽媽,又説個故事,扮個鬼臉,逗她破涕爲笑。這一日過了駐馬河,其時已是秋末冬初,朔風吹來,兩個孩子衣衫單薄,都是禁不住發抖。無忌除下自己一件破爛的外衫,給楊不悔穿上。楊不悔道︰「無忌哥,你自己也冷,却把衣服給我穿。」這個小女孩斗然間説起大人話來,無忌不由得一怔。

便在此時,忽聽得山坡後傳來一陣兵刃相交的叮{\upstsl{噹}}之聲,跟著脚步聲響,一個女子聲音叫道︰「惡賊,你中了我的餵毒喪門釘,越是快跑,發作得越快!」無忌急拉楊不悔在道旁草叢中伏下,尚未藏好身子,只見一個三十來歳的精壯漢子飛步奔來,數丈之後,一個手持雙刀的女子追趕而至。那漢子脚步踉蹌,突然間足下一軟,滾倒在地。那女子追到身前,笑道︰「惡賊,終叫你死在姑娘手裡!」那漢子驀地一躍而起,雙掌齊出,波的一聲,擊中那女子頸下的胸口。這一招是那漢子的救命絶招,力道奇猛,那女子中掌倒地,手中雙刀遠遠摔了出去。

那漢子不住喘氣,從自己背上拔了一枚喪門釘出來,恨恨的道︰「取解藥來。」那女子道︰「你殺了我吧!就是没解藥。」那漢子左手以刀尖指住她的咽喉,右手到她衣袋中搜尋,果然不見解藥。那女子冷笑道︰「這次師父派咱們出來捉你,只給餵毒暗器,不給解藥。我既落在你手裡,也不想活了,可是你也别想逃生。」那漢子怒極,提起那枚餵毒喪門釘用力一擲,釘在那女子肩頭,喝道︰「叫你自己也嘗嘗餵毒喪門釘的滋味,你崑崙派\dash{}」一句話没説完,背上毒性發作,軟垂倒地。那女子想掙扎爬起,但胸口所受的掌力太重,哇的一聲吐出一口鮮血,又再坐倒。

一男一女兩人臥在道旁草地之中,呼吸粗重,不住喘氣。張無忌自從醫治簡捷、薛公遠而遭反噬之後,對武林中人深具戒心,這時躱在一旁觀看動靜,不敢出來。過了一會,只聽那漢子長長嘆了口氣,説道︰「我蘇習之今日喪命在駐馬店,仍是不知到底如何得罪了你崑崙派,當眞是死不瞑目。詹姑娘,你好心跟我説了罷!」言語之中,已是没什麼敵意。那女子姓詹名春,知道師門這餵毒喪門釘毒性的厲害,眼見和他同歸於盡,心中萬念倶灰,幽幽的道︰「誰叫你偸看我師父練劍,這路『龍形一筆劍』,若不是他老人家親手傳授,便是本門弟子偸瞧了,也要遭剜目之刑,何況你是外人?」蘇習之「啊」的一聲,説道︰「他媽的,該死,該死!」詹春怒道︰「你死到臨頭,還在罵我師父?」蘇習之道︰「我罵了便怎樣?這不是冤枉麼?我經過白牛山,無意中見到你師父使劍,覺得好奇,便瞧了一會。難道我又有這等聰明才智,瞧得片刻,便能將這路劍法的精義學去了?若是我眞有這麼好的本事,你們幾名崑崙子弟又奈何得了我?詹姑娘,我跟你説,你師父鐵琴先生太過小氣,别説我没學到這『龍形一劍』的一招半式,就算學會了一些,也是罪不至死啊。」詹春默然不語,心中也頗怪師父小題大做,只因發覺蘇習之偸看練劍,便派出六名弟子,嚴令追殺,終於落到跟此人兩敗倶傷,心想事到如今,這人也已不必説謊,他既説並未偸學到武功,自是不假。

蘇習之又道︰「他給你們餵毒暗器,却不給解藥,武林中有這個規矩麼?他媽的\dash{}」詹春柔聲道︰「蘇大哥,小妹害了你,此刻心中好生後悔,好在我也陪你送命,這叫做命該如此。只是累了你家中大嫂,公子,小姐,實是過意不去。」

蘇習之嘆道︰「我女人已在兩年前身故,留下一男一女兩個孩子,明日他們便是無父無母的孤児了。」詹春道︰「你府上尚有何人?有人照料這兩個孩子麼?」蘇習之道︰「此刻由我嫂子在照看。我嫂子脾氣暴躁,爲人刁蠻,我在世時,她還忌我幾分。唉!今後這兩個娃娃,有得苦頭吃。」詹春心腸甚軟,垂下泪來,低聲道︰「都是我作的孼。」蘇習之道︰「那也怪你不得。你奉了師門嚴令,不得不遵,又不是跟我有什冤仇?其實,我中了你的餵毒暗器,死了也就算了,何必再打你一掌,又用暗器傷你?否則我以實情相告,你爲人仁善,必能照看我那兩個苦命的孩児。」詹春苦笑道︰「我是害死你的兇手,怎説得上爲人仁善?」蘇習之道︰「我没有怪你,眞的,没有怪你。」適纔兩人拚命惡鬥,這時却相互慰藉起來。

張無忌聽到這裡,心想︰「這一男一女似乎心地不惡,何況那姓蘇的家中尚有兩個孩児。」想起自己和楊不悔身爲孤児之苦,便從草叢中走了出來,説道︰「詹姑娘,你喪門釘上餵的是什麼毒藥?」蘇習之和詹春突然見草叢中鑽出一個少年,一個女孩,已是奇怪,聽得無忌如此詢問,更是驚訝。無忌道︰「在下粗通醫理,兩位所中傷毒,未必無救。」詹春道︰「是什麼毒藥,我可不知道。傷口上奇癢難當,却是一點不痛。我師父道,中這喪門釘後,只有四個時辰的性命。」無忌道︰「讓我瞧瞧傷勢。」蘇詹二人見他年紀既小,又是衣衫破爛,容顏憔悴,活脱是個小叫化子,那裡信他能治傷毒?蘇習之道︰「咱二人命在頃刻,你快别在這児囉{\upstsl{嗦}},給我走得遠遠的吧。」無忌不去睬他,從地上拾起喪門釘,拿到鼻中一聞,嗅到一陣淡淡的蘭花清香。這些日來,他一有餘暇,便翻讀王難姑所遺的那部毒物大全,天下千奇百怪的毒物毒藥,莫不了然於胸,一聞到這陣香氣,即知喪門釘上餵的是「青陀羅花」的毒汁。這種花汁原有一陣腥臭之氣,本身並無毒性,便是喝上一碗,也絲毫無害於人體,但一經和鮮血混合,却生劇毒,同時腥臭轉爲幽香,説道︰「這是餵了青陀羅花之毒。」詹春並不知那喪門釘上餵的是何毒藥,但師父的花圃之中種有這種怪花,她却知道的,奇道︰「咦,你怎麼知道?」要知青陀羅花是一種極爲罕見的毒花,源出西域,爲中土向來所無。無忌點了點頭,説道︰「我知道。」擕了楊不悔的手,道︰「咱們走吧。」詹春忙道︰「小兄弟,你若知治法,請你好心救咱二人一救。」無忌原本有心相救,但突然想到簡捷和薛公遠要吃人肉的那些獰惡面貌,不由得又感躊躇。蘇習之道︰「小相公,是在下有眼不識高人,請你莫怪。」無忌道︰「好吧!我試一試看。」伸指在詹春胸口「膻中穴」及肩旁左右「缺盆穴」點了幾下,先止住她胸口掌傷的疼痛,説道︰「這青陀羅花見血生毒,入腹却是無礙。兩位先用口相互吮吸傷口,至血中絶無凝結的細微血塊爲止。」

蘇習之和詹春都是頗覺不好意思,但這時性命要緊,所傷的又是自己吮吸不到的肩背之處,只得輪流著替對方吸出傷口中的毒血。張無忌在山邉採了三種草藥,嚼爛了替二人敷上傷口,説道︰「這三味草藥能使毒氣暫不上攻,咱們到前面市鎭去,尋到藥店,我再替你們配藥療毒。」蘇詹二人的傷口本來癢得難當之極,敷上草藥,登覺清涼,同時四肢也不再麻軟,當下不住口的稱謝。二人各折一根樹枝作爲拐杖,撐著緩步而行。詹春問起張無忌的師承來歷,無忌不願細説,只説自幼便懂醫理。行了一個多時辰,到了沙河店,四人投客店歇宿,無忌便開了藥方,命店伴去抓藥。

這一年豫西一帶未受天災,雖然蒙古官吏橫暴殘虐,和别的地方無甚分别,但老百姓總算還有口飯吃。沙河店鎭上一切店舖開設如常。客店中店伴照藥方抓了藥來,張無忌用土罐把藥煮好了,餵著蘇習之和詹春服下。四人在客店中住了三日,無忌每日變換藥方,外敷内服,到第四日上,蘇詹二人身上所中劇毒全部驅除,二人自是大爲感激。問起無忌和楊不悔要到何處,無忌説了崑崙山坐忘峰的地名。詹春道︰「蘇大哥,咱兩人的性命,是蒙這位小兄弟救了,可是我那五位師兄,仍在到處尋你,這件事情還没了結。你隨我上崑崙走一遭,好不好?」

蘇習之吃了一驚,道︰「上崑崙山?」詹春道︰「不錯。我同你去拜見家師,説明你確實並未學到『龍形一筆劍』的一招半式。此事若不得他老人家原宥,日後總是禍患無窮。」蘇習之心下著惱,説道︰「你崑崙派忒也欺人太甚,我只不過多看了一眼,累得險些進入鬼門関,也該放手了罷?」詹春柔聲道︰「蘇大哥,你替小妹想一想這中間的難處。我跟師父去説,你没學到劍法,他是決計不信的。小妹受責,那也没有什麼,但我那五位師兄倘若再失手傷你,小妹心中如何過意得去?」

他二人出死入生的共處數日,相互間已生情意,蘇習之聽她這般軟語溫存的説話,胸中的氣登時消了,又想︰「崑崙派人多勢衆,若是陰魂不散的纏上了我,最後終於還是送命在他們手裡爲止。」詹春見他沉吟,又道︰「你先陪我走一遭。你什麼要緊事,咱們去了崑崙之後,小妹再陪你一道去辦如何?」蘇習之大喜,道︰「好,便是這麼著。只不知尊師肯不肯信?」詹春道︰「師父素來喜歡我,我苦苦相求,諒來不會對你爲難。」蘇習之聽她這般説,顯有以身相許之意,心中甜甜的受用,對無忌道︰「小兄弟,咱們都到崑崙山去,大夥児一起走,路上也有個伴児。」詹春道︰「崑崙山脈綿延千里,峰巒無數,那坐忘峰不知坐落何處,但慢慢打聽,總能找到。」

次日蘇習之僱了一輛大車,讓無忌和楊不悔乘坐,自己和詹春乘馬而行。到了前面大鎭上,詹春又去替無忌和楊不悔買了幾套衣衫,把兩人換得煥然一新。蘇詹二人見這對孩児洗沐換衣之後,男的英俊,女的秀美,都大聲喝起采來。兩個孩子直到此時,始免長途步行之苦,吃得好了,身子也漸漸豐腴起來。

漸行漸西,天氣一天冷似一天,沿途有蘇習之和詹春兩個武林人物照看,一路平安無事。到得西域後,崑崙派勢力雄厚,更無絲毫阻攔,只是黃沙撲面,寒風透骨,那是無可奈何的了。不一日來到崑崙山三聖坳,進了山坳,只見遍地綠草如錦,果樹香花,蘇習之和張無忌都萬想不到這荒寒之處竟是别有天地。原來那三聖坳四周都是高山,擋住了寒氣。崑崙派自「崑崙三聖」何足道以來,七八十年中花了極大力氣,整頓這個山坳,派遣弟子東至江南,西至天竺,搬移奇花異樹,到這三聖坳中種植。

詹春帶著三人來到鐵琴先生何太沖所居的鐵琴居,一進門,只見師兄弟們臉上神色嚴重,和她微一點頭,便不再説話。詹春心中{\upstsl{嘀}}咕,不知發生了什麼事,拉住一個師妹,問道︰「師父在家吧?」那女弟子尚未答話,只聽何太沖暴怒咆哮的聲音從後堂傳了出來︰「都是飯桶,飯桶!有什麼事叫你們去辦,從來没一件辦得妥當。要你們這些弟子何用?」只聽得拍桌之聲,震天價響。詹春向蘇習之低聲道︰「師父在發脾氣,咱們别去找釘子碰,明児再來。」何太沖突然叫道︰「是春児麼?回來了幹什麼不來見我?鬼鬼祟祟的説些什麼話?那姓蘇小賊的首級呢?」

詹春臉上變色,搶步進了内堂,跪下磕頭,説道︰「弟子拜見師父。」何太沖道︰「我差你去辦的事怎麼啦?那姓蘇的小賊呢?」詹春道︰「那姓蘇之人現在外面,來向師父請罪。他説他資質愚魯,雖是不該看師父演練劍法,但本派劍法精微奥妙,他看過之後,莫名其妙,半點也領會不到。」詹春跟隨師父日久,知他武功上極爲自負,因此故意説蘇習之極力稱譽本門功夫,何太沖一高興,説不定便饒了他。若在平時,這頂高帽何太沖必輕輕受落,但今日他心境大是煩燥,哼了一聲,説道︰「這事你辦得好!去把那姓蘇的関在後山石室中,慢慢發落。」詹春見他正在氣沖頭上,不敢出口相求,應道︰「是!」又問道︰「師母們都好?我到後面磕頭去。」

原來何太沖共有妻妾五人,最寵愛的是第五小妾,詹春爲了求師父饒恕蘇習之,便想去請這位五師母代下説辭。那知何太沖臉上忽現淒惻之色,長嘆了一聲,道︰「你去瞧瞧五姑也好,她病得很重,你總算趕回來還能見到她一面。」詹春吃了一驚,道︰「五姑不舒服麼?不知是什麼病?」何太沖嘆道︰「知道是什麼病就好了,已請了七個算是有名的大夫來看過,都是不知她生了什麼病。全身浮腫,一個如花如玉的人児,腫得\dash{}唉,不用説起。\dash{}」説著連連搖頭,又道︰「我收了這許多徒弟,没一個管用。叫他們到長白山去找老山人參,去了快兩個月啦,没一個回來,要他們去找雪蓮、首烏等救命之物,個個空手而歸。」詹春心想︰「從這裡到長白山萬里之遙,那能去了即回?便是到了長白山,也未必能找到老山人參啊。至於雪蓮、首烏等起死回生的珍異藥物,找一世也不見得會找到,一時三刻,那能要有便有?」但想師父對這個小妾愛如性命,眼見她病重不治,自不免遷怒於人。

何太沖又道︰「我以内力試她經脈,却是一點異狀也没有,哼哼,五姑若是性命不保,我殺盡天下庸醫。」詹春道︰「我去望望她。」何太沖道︰「好,我陪你去。」

師徒倆一起到了五姑的臥房之中,詹春一進門,撲鼻便是一股藥氣,揭開帳子,只見五姑一張臉腫得猶如豬八戒一般,雙眼深陥肉裡,幾乎睜不開來,喘氣甚急,像是扯著風箱。這五姑本是個極美的佳人,否則何太沖也不致爲她這般著迷,這時一病之下,變成如此醜陋,詹春也不禁大爲歎息。

何太沖道︰「叫那些庸醫再來瞧瞧。」在房中服侍的老媽子答應著出去,過了良久,只聽得鐵鍊聲響,七個穿著長衫的醫生走了進來。這七個人脚上被鐵鍊鎖在一起,形容憔悴,神色極是苦惱。原來這七人都是四川、雲南、甘肅一帶最有名的醫生,被何太沖派弟子半請半拿的捉了來。但七位名醫看法各各不同,有的説是水腫,有的説是中邪,所開的藥方試服之後,没一張管用,五姑的身子仍是一日腫脹一日。何太沖一怒之下,將七位名醫都鎖了,説道五姑若是不治病逝,七個庸醫(這時「名醫」的名稱已被改爲「庸醫」)一齊進入墳中殉葬。

七個醫生用盡了全身本事,減不了五姑的一絲病情,自知性命不保,但每次會診,總是大聲爭論不休,攻擊其餘六人名醫生,説五姑所以病重,全是他們所害,與自己無涉。這一次七人進來,診脈之後,三言兩語,又爭執起來。何太沖又急又怒,大聲怒罵,纔將七個不知是名醫還是庸醫的聲音壓了下去。

詹春心念一動,説道︰「師父,我從河南帶來了一位醫生,年紀雖小,本領却比他們高些。」何太沖大喜道︰「你何不早説,快請,快請。」每一位名醫初到,他對之都十分恭敬,但「名醫」一變成「庸醫」,他可一點也不客氣了。

詹春走到廳上,將張無忌帶了進去,無忌一見何太沖,認得當年在武當山逼死父母的人中,便有此人在内,不禁心下極是惱怒。但何太沖却不識得無忌,要知隔了這四五年,無忌相貌身材均已大變,但見他是個十四五歳的少年,見了自己竟不磕頭行禮,側目斜視,神色間甚是冷峭,也不理會,問詹春道︰「你説的那位醫生呢?」詹春道︰「這位小兄弟便是了。他的醫道精湛得很,只怕還勝過許多名醫。」何太沖哼了一聲,心下那裡相信。詹春道︰「弟子中了青陀羅花之毒,便是蒙這位小兄弟治好的。」何太沖一驚,心想︰「青陀羅花的花毒不得我獨門解藥,中後必死,這小子居然能彀治好,那倒有些邪門。」向無忌上上下下的打量了一會,説道︰「少年,你眞會治病麼?」

無忌想起父母慘死的情景,本來對何太沖極是憎惡,可是他天性仁善,素來不易記仇,否則何以會肯給紀曉芙、簡捷等人治病?他明知父母之死,崑崙派也脱不了干係,但他難以見死不救,終於伸手治了詹春和蘇習之的傷毒,這時聽何太沖如此不客氣的詢問,心中雖是不快,還是點了點頭。

他一進房,便聞到一股古怪的氣息,過了片刻,更覺這氣息忽濃忽淡,甚是奇特,於是走到五姑床前,瞧了瞧她臉色,按了她雙手脈息,突然取出一根金針,從她腫得如同南瓜般的臉上刺了下去。何太沖大吃一驚,喝道︰「你幹什麼?」待要伸手去抓無忌,見他已拔出金針,五姑臉上却無血液膿水滲出。何太沖五根手指離無忌背心不及半尺,硬生生的停住,只見無忌將金針湊近鼻端一嗅,點了點頭。何太沖心中露出一絲指望,道︰「小\dash{}小兄弟,這病有救麼?」以他一派之尊,居然叫張無忌一聲「小兄弟」,那是算得客氣之極了。

張無忌不答,突然爬到五姑床底,仔細瞧了一會,又打開窗子,向窗外的花圃細看,忽地從窗中跳出,却去觀賞花圃的各種鮮花。何太沖因寵愛五姑,她窗外的花圃之中,所種的均是極名貴的花卉,這時見無忌行動怪異,自己指望他治好五姑的怪病,他却自得其樂的賞玩起花卉來,却教他如何不怒?只見張無忌看了一會花草,點點頭,若有所悟,回進房來,説道︰「病是能治的,可是我不想治。詹姑娘,我要去了。」詹春道︰「張兄弟,倘若你治好了五姑的疾病,咱們崑崙派上下,齊感你的大德,一定要請你治一治。」張無忌指著何太沖道︰「逼死我爹爹媽媽的人中,這位鐵琴先生也有份。我爲什麼要救他親人的性命?」

何太沖又是一驚,問道︰「小兄弟,你貴姓,令尊令堂是誰?」張無忌道︰「我姓張,先父是武當派第五弟子。」何太沖一凜︰「原來這少年是張翠山的児子。」當下深深一揖,説道︰「張兄弟,令尊在世之時,在下和他甚是交好,他自刎身亡,我痛惜不止\dash{}」其實他是爲了救愛妾的性命,在那裡信口胡吹,詹春也幫著師父圓謊,説道︰「令尊令堂死後,家師痛哭了幾場,常跟咱們説,令尊是他生平最交好的良友。」張無忌半信半疑,但他生性不易記恨,便道︰「這位夫人不是生了怪病,是中了金銀血蛇的蛇毒。」何太沖和詹春齊聲道︰「金銀血蛇?」這名稱他們可從來没聽見過。

張無忌道︰「不錯,這種毒蛇我也從來没見過,但夫人臉頰腫脹,金針探後針上却有檀香之氣。何先生,請你瞧一瞧夫的十根足趾,趾尖上可有細小的齒痕。」何太沖忙掀開五姑身上的錦被,一看她足趾,果見每根足趾尖端都有一個齒痕。

\chapter{金銀血蛇}

何太沖一看到愛妾足趾上的齒痕,對張無忌的信心陡增十倍,説道︰「不錯,不錯,當眞每個足趾上都有齒痕,小兄弟實在高明,實在高明。小兄弟既知病源,必能療治,小妾病愈之後,我必當重重酬謝。」他轉頭對七個醫生喝道︰「什麼風寒中邪,陽虛陰虧,都是胡説八道!她足趾上的齒痕,你們怎地瞧不出來?」張無忌道︰「夫人此病原本奇特,他們不知病源,那也怪他們不得,都放他們回去吧!」何太沖道︰「很好很好!小兄弟大駕光臨,再留這些庸醫在此,那不是徒惹人厭麼?春児,每人送一百兩銀子,叫他們各自回去。」那七個醫生死裡逃生,無不大喜過望,急急離去,生怕無忌的醫法不靈,何太沖又遷怒到他們身上。

張無忌道︰「請叫僕婦搬開夫人的臥床,床底有兩個小洞,那便是金銀血蛇出入的洞穴了。」何太沖也不等僕婦動手,右手抓起一隻床脚,單手便連床帶人一齊提開,果見床底有兩個小洞,不禁又喜又怒,叫道︰「快取硫礦煙火來,薰出毒蛇,斬牠個千刀萬刀!」張無忌搖手道︰「使不得,使不得。夫人身上所中的蛇毒,全仗這兩條毒蛇醫治,你殺了毒蛇,夫人的病便無法醫治了。」何太沖道︰「原來如此。這中間的原委,倒要請教。」張無忌指著窗外的花圃道︰「何先生,尊夫人的疾病,全由花圃中那八株『靈脂蘭』而起。」何太沖道︰「這叫做『靈脂蘭』麼?我也不知其名,有一位朋友知我性愛花草,從西域帶了這八盆蘭花送我,這花開放時有檀香之氣,花朶的顏色又極嬌艷,想不到竟是禍胎。」

張無忌道︰「據書上所載,這種『靈脂蘭』其莖如球,顏色火紅,球莖中含有劇毒,我去掘來瞧瞧,不知是也不是。」這時何太沖的弟子們均已得知張無忌在治五師母的怪病,男弟子不便進房,詹春等六個女弟子却都在師父身旁,聽得無忌這般説,便有兩名女弟子拿了鐵鏟,將一株靈脂蘭掘了起來,果見土下的球莖色赤如火,兩名女弟子知道莖中含有劇毒,那敢用手去碰?

張無忌道︰「請各位將八枚球莖都掘出來,放在土缽之中,加入雞蛋八枚,雞血一碗,搗爛成糊。搗藥時務須小心,不可濺上肌膚。」詹春答應了,自和兩名師妹同去辦理。張無忌又要了兩根竹筒,一枝竹棒,放在一旁。

過不多時,靈脂蘭的球莖已搗爛成糊,無忌將藥糊倒在地下,圍成一個圓圏,却空出了一個兩寸來長的缺口,説道︰「待會見有異狀,各位千萬不可作聲,以免毒蛇受到驚嚇,暴起傷人。各位去取些甘草、棉花,塞住鼻孔。」衆人依言而爲,張無忌也塞住鼻孔,然後取出火種,將靈脂蘭的葉子放在蛇洞前燒了起來。不到一盞茶時分,只見左邉小洞中探出一個蛇頭,蛇皮血紅,頭頂却有個金色肉冠。那蛇緩緩爬出,竟是生有四足,身長約莫八寸,這金冠血蛇剛從洞中出來,右邉小洞中也爬出一蛇,身形略短,頭頂肉冠則作銀色。何太沖等見了這兩條怪蛇,都是屏息不敢作聲,這種異相毒蛇必有劇毒,那是不必説了,若是將牠們驚走,只怕夫人的疾病難治。

只見兩條怪蛇伸出蛇舌,你舐舐我的肩頭,我舐舐你的背脊,神情親熱異常,相偎相倚,慢慢地爬進了靈脂蘭藥糊圏成的圓圏之中。張無忌忙將兩根竹筒放在圓圏的缺口,提起一根竹棒,輕輕在銀冠血蛇的尾上一撥。那蛇行動快如電閃,衆人眼前只見銀光一閃,那蛇已鑽入了竹筒。金冠血蛇跟著也要鑽入,但那竹筒甚小,長短只容得一蛇,銀冠血蛇進去之後,金冠血蛇便無法再進,只急得胡胡而叫,聲音如吹洞簫,甚是悦耳動聽。

張無忌用竹棒將另一根竹筒撥到金冠血蛇的身前,那蛇便也鑽了進去。無忌忙取過木塞,塞住了竹筒的口子。自那對金銀血蛇從洞中出來,衆人一直戰戰兢兢,提心吊膽,直到無忌用木塞塞住竹筒,各人才不約而同的吁了口長氣。無忌道︰「請拿幾桶熱水進來,將地下洗得乾乾淨淨,不可留下靈脂蘭的毒性。」六名弟子忙奔到厨下燒水,過不多時,便將地下洗得片塵不染。

無忌叫各人緊閉門窗,又命人取來雄黃,明礬、大黃、甘草等幾味藥材,搗爛成末,拌以生石灰粉,灌入銀冠血蛇的竹筒之中,那蛇登時胡胡的叫了起來。另一筒中的金蛇也呼叫相應。無忌拔去金蛇竹筒上的木塞,那蛇從竹筒中出來,繞著銀蛇所居的竹筒遊走數匝,狀甚焦急,突然間急竄上床,從五姑的棉被中鑽了進去。何太沖大驚,「啊」的一聲叫了出來。張無忌搖搖手,輕輕揭開棉被,只見那金冠血蛇一口咬住了五姑左足的中趾。無忌臉露喜色,道︰「解鈴還是繫鈴人,五姑身中這金銀血蛇之毒,現下便是這對蛇児吸出她體内的毒質。」

過了一頓飯時分,只見那蛇身子腫脹,粗了幾倍,頭上那金色肉冠更是燦然生光。無忌拔下銀蛇所居竹筒的木塞,那金蛇即從床上躍下,遊近竹筒,口中吐出毒血,餵那銀蛇。

無忌道︰「好了,每日吸毒兩次,我再開一張消腫補虛的方子,十天之内,便可痊癒。」何太沖大喜,將無忌讓到書房,説道︰「小兄弟神乎其技,這中間的緣故,還要請教。」無忌道︰「據『毒物大全』所載,這金冠銀冠的一對血蛇,在天下毒物之中,名列第三十七,雖然不算是十分厲害的毒物,但它有一種特點,便是性喜食毒。什麼砒霜、鶴頂紅、孔雀膽、鳩酒等等,牠無不喜愛。夫人窗外的花圃之中,種了靈脂蘭,這靈脂蘭的毒性,可著實厲害,竟將這對金銀血蛇引了出來。」何太沖點頭道︰「原來如此。」張無忌又道︰「金銀血蛇必定雌雄共居,適纔我用雄黃、甘草等藥焙炙那銀冠雌蛇,金冠雄蛇爲了救牠伴侶,便到夫人脚趾上吸取毒血相餵。再過三個時辰,我用藥物整治雄蛇,那雌蛇也必再去吸取毒血,如此反覆施爲,便可將夫人體内毒質去盡。」

當日何太沖在後堂設了筵席,款待張無忌與楊不悔。無忌心想楊不悔是紀曉芙的私生女児,説起來於峨嵋派的聲名有累,因此當何太沖問起她來歷時,含糊其辭,不加明説。

過了數日,五姑的腫脹果然漸消退,精神恢復,已能略進飲食,到第十天上,腫脹全消。五姑備了一席精緻酒筵,親向無忌道謝,請了詹春作陪。五姑容色雖仍憔悴,但俏麗一如往昔。何太沖自是十分喜歡。詹春乘著師父高興,求他將蘇習之收入門下。何太沖呵呵笑道︰「春児,你這斧底抽薪之計可著實不錯啊,我收了這姓蘇的小子,將來自會把『龍形一筆劍』傳他,那麼他從前偸看一次,又有何妨?」詹春笑道︰「師父,倘若不是這姓蘇的偸看你老人家練劍,弟子不會去拿他,便不會碰到張世兄。固然師父和五姑洪福齊天,可是這姓蘇的小子,説來也有一份功勞啊。」五姑向何太沖道︰「你收了這許多弟子,到頭來誰也幫不了你的忙。詹姑娘既然看中了那小子,想必是好的,你就多收一個吧,説不定將來倒是最得力的弟子呢。」何太沖對愛妾之言向來唯命是聽,便道︰「好吧,我收便收他,可是有一個條件。」

五姑道︰「什麼啊?」何太沖正色道︰「他投入我門下之後,須得安心學藝,可不許對春児痴心妄想,企圖娶她爲妻,這個我可是萬萬不准的。」

詹春滿臉通紅,把頭低了下去,五姑却吃吃的笑了起來,説道︰「啊喲,你做師父的要以身作則纔好,自己三妻四妾,却難道禁止徒児們婚配麼?」何太沖那句話原是跟詹春説笑,哈哈一笑,便道︰「喝酒,喝酒!」只見一名小鬟托著木盤,盤中放著一隻酒壼,走到席前,替各人斟酒。那酒稠稠的微帶黏性,顏色金黃,甜香撲鼻。何太沖道︰「張兄弟,這是本山的名産,乃是取雪山頂上的琥珀蜜梨釀成,叫做『琥珀蜜梨酒』,爲外地所無,不可不多飲幾杯。」

張無忌本是不會飲酒,但聞到這琥珀蜜梨酒酒香沁入心脾,便端起杯來,正要去飲,突然懷中那金銀蛇同時胡胡胡的低鳴起來。無忌心念一動,叫道︰「此酒飲不得。」衆人一怔,都放下了酒杯。無忌從懷中取出竹筒,放出金冠血蛇,那蛇児遊於酒杯之旁,將一杯酒喝得涓滴不剩。牠連喝了三杯蜜梨酒,無忌將牠関回竹筒,放了銀冠雌蛇出來,也喝了三杯。這對血蛇互相依戀,單放雄蛇或是雌蛇,決不遠去,同時對主人十分馴順,但若雙蛇同時放出,那不但難以補捉回歸竹筒,而且説不定便暴起傷人,反噬主人。

五姑笑道︰「小兄弟,你這對蛇児會喝酒,當眞有趣得緊。」張無忌道︰「請命人捉一隻狗子或是貓児過來。」那小鬟應道︰「是!」便要轉身退出。無忌道︰「這位姊姊等在這裡别去,讓别人去捉貓狗。」過了片刻,一名僕人牽了一頭大黃狗進來,無忌端起何太沖面前的一杯酒,灌在黃狗的口裡。那黃狗悲吠幾聲,隨即七孔流血而斃。

五姑嚇得渾身發抖,道︰「酒裡有毒\dash{}誰\dash{}誰要害死我們啊?張兄弟,你又怎地知道?」無忌道︰「這對金銀血蛇喜食毒物,牠們嗅到酒中毒藥的氣息,便高興得叫了起來。」那小鬟驚得魂不附體,道︰「我\dash{}我不知道是毒\dash{}有毒\dash{}我從大厨房拿來\dash{}」何太沖道︰「你從大厨房到這裡,遇到過誰了?」那小鬟道︰「在走廊裡見到杏芳,她拉住我跟我説話,揭開酒壼聞了聞酒香。」

何太沖、五姑、詹春三人對望了一眼,原來那杏芳是何太沖原配夫的貼身使婢。張無忌道︰「何先生,此事我一直躊躇不説,却在暗中察看。你想,這對金銀血蛇當初何以要去咬夫人的足趾,以致以蛇毒傳入她的體内?顯而易見,是夫人中了慢毒性藥,血中有毒,纔引到金銀血蛇。從前那下毒之人,只怕便是今日在酒中下毒那一位。」何太沖尚未説話,突然門帘掀起,人影一晃,無忌只覺雙乳底下一陣劇痛,已被人點中了穴道,一個尖鋭的聲音説道︰「一點児也不錯,是我下的毒。」只見進來那人是個身材高大的中年女子,雙目含威,眉心間聚有煞氣。那女子對何太沖道︰「是我在酒中下了蜈蚣涎的劇毒,你待怎樣?」五姑見了這女子甚是害怕,站起身來,恭恭敬敬的叫道︰「太太!」原來這女子乃是何太沖的元配夫人,名叫班淑嫻,武功比之何太沖只高不低。何太沖向來對她極是畏懼,但怕雖然怕,妾侍還是娶了一個又一個,只是每多娶一房妾侍,對妻子便又多怕三分。

何太沖見妻子衝進房來,默然不語,只是哼了一聲。班淑嫻道︰「我問你啊,是我下的毒,你待怎樣?」何太沖道︰「你不喜歡這少年,那也罷了。但你行事這等不分青紅皂白,如果我不是及時警覺,毒酒下肚,那可如何是好?」班淑嫻道︰「這裡的人全不是好東西,一古腦児整死了,也好耳根清靜。」她拿起毒酒的酒壼搖了搖,壼中有聲,還剩得有大半壼毒酒。

班淑嫻滿滿的斟了一杯毒酒,放在何太沖面前,説道︰「我本想將你們五個人一起毒死,既是被這小鬼發覺,那就饒了四個人的性命。這一杯毒酒,却是非喝不可,誰喝都是一樣,老鬼,你來決定吧。」説著刷的一聲,拔出長劍在手。

原來班淑嫻是崑崙派中武功傑出的女弟子,年紀比何太沖爲大,入門也較他爲早,武學修爲更是比他深湛。何太沖年輕時英俊瀟灑,深得這位師姊歡心。他們師父是因和明教中一位前輩交手爭鬥而死,突然而逝,不及留下遺言,下一代的衆弟子爭奪掌門之位,各不相下。班淑嫻極力扶助何太沖,兩人聯手,勢力大增,别的師兄弟各懷私心,那就無法與之相抗,結果由何太沖接任掌門。他懷恩感激,便娶了這位師姊爲妻。少年時還不怎樣,兩人年紀一大,班淑嫻特别顯得衰老,何太沖藉口没有子嗣,便娶起妾侍來。可是由於她數十年來的積威,再加何太沖自知不是,心中有愧,對這位師姊又兼嚴妻,却是十分的敬畏。

這時見妻子將一杯毒酒放在自己面前,壓根児就没違抗的念頭,心想︰「我自己當然不喝,五姑和春児也不能喝,張無忌是咱們救命恩人,只有這女娃娃跟咱們無親無故。」於是站起身來,將那杯毒酒遞給楊不悔,道︰「孩子,你喝了這杯酒。」楊不悔大驚,適纔眼見二條肥肥大大的黃狗喝了一杯毒酒便即斃命,那裡敢接酒杯。哭道︰「酒裡有毒,我不喝,我不喝。」何太沖抓住她胸口衣服,正要強灌,張無忌冷冷的道︰「我來喝好了。」何太沖微一躊躇,心中覺得過意不去。班淑嫻因心懷妬忌,是以下毒想毒死何太沖最寵愛的五姑,眼見得手,却給張無忌不遠千里的趕來救了,對少年原是極度憎惡,冷冷的道︰「你這少年古裡古怪,説不定有解毒之藥。若是你來代喝,一杯不彀,須得將毒酒喝乾淨了。」

張無忌眼望何太沖,盼他從旁説幾句好話,那知他低了頭竟是一言不發,詹春和五姑也不敢説話,生怕一開口,班淑嫻的怒氣轉到自己頭上,這大半壼毒酒便要灌到自己口中。張無忌心中冰涼,暗想︰「這幾人的性命是我所救,但我此刻遇到危難,他們竟是袖手旁觀,連求情也不代求一句。」便道︰「詹姑娘,我死之後,請你將這位小妹妹送到坐忘峰她那爹爹那裡,這事能辦到麼?」詹春眼望師父。何太沖點了點頭。詹春便道︰「好吧,我會送她去。」心中却想︰「崑崙山橫亙千里,我知道坐忘峰在在那裡?」張無忌聽她隨口敷衍,並無誠意,知道這些人都是涼薄之輩,多説也是枉然,冷笑道︰「崑崙派自居武林中正大門派,原來如此。何先生,取酒給我喝吧!」何太沖聽了他這幾句諷刺的言語,心下大怒,巴不得他早些中毒而死,當下提起大半壼毒酒,都灌進了無忌口中。楊不悔抱著無忌身子,放聲大哭。班淑嫻冷笑道︰「你醫術再精,我也教你救不得自己。」伸手又在張無忌肩背腰脅多處穴道,補上幾指,隨即倒轉劍柄,在何太沖、詹春、五姑、楊不悔四人身上各點了穴道,説道︰「兩個時辰之後,再來放你們。」她點穴之時,何太沖和詹春等動也不動,不敢閃避。班淑嫻向在旁侍候的婢僕喝道︰「都出去。」她最後出房,反手帶上房門,連聲冷笑而去。

毒酒入腹,片刻間張無忌便覺疼痛難當,眼見班淑嫻出房関門,心道︰「你既走了,我一時便未必會死。」強忍疼痛,暗自運氣,以謝遜所授之法,先解開了上身被點的諸穴,隨即伸手拔下幾根頭髮,到咽喉中一陣撩撥,喉頭發癢,哇的一聲,將飲下的毒酒嘔出了十分八九。何太沖、詹春等見他穴道被點後居然仍能動彈,都是大爲驚訝。

何太沖便欲出手攔阻,苦於自己被妻子點了穴道,空有一身極高的武功,却是不得施展,只有乾著急的份児,張無忌覺得腹中仍極疼痛,但搜肚嘔腸,再也吐不出來,心想先當脱此危境,再行設法除毒,於是伸手去解楊不悔的穴道,那知班淑嫻的點穴手法另有一功,無忌竟是解之不開,只得將她抱在手裡,輕輕推開窗子,向外一張,不見有人,便將楊不悔放在窗外。

何太沖若以眞氣衝穴,大半個時辰後也能解開,但眼見張無忌便要逃走,待會妻子査問起來,又有風波,何況讓這武當派的小子赤手空拳從崑崙派三聖堂中逃了出去,將自己忘恩負義的事蹟在江湖上傳揚開來,一代宗師的顏面何存?那是無論如何非將他截下不可。何太沖深深吸一口氣,待要出聲呼叫,向妻子示警,張無忌已料到此著,從身上摸出一顆黑色藥丸,塞在五姑口中説道︰「這是一顆『鴆砒丸』,十二個時辰之後,斷腸裂心而死。我將解藥放在離此三十里外的大樹之上,作有標誌,三個時辰之後,何先生可派人來取。倘若我出去時失手被擒,那麼反正是個死,多有一個人相陪也好。」

這一著大出何太沖意料之外,微一沉吟,低聲道︰「小兄弟,我這三聖派雖非龍潭虎穴,但憑你兩個孩子,却也闖不出去。」張無忌知他此言不虛,冷冷的道︰「依我看來,夫人所服的『鴆砒丸』的毒性,眼前除我之外,無人能解。」何太沖道︰「好,你解開我的穴道,我親自送你出去。」何太沖被點的是「風池」和「京門」兩穴,張無忌在他「天柱」「環跳」「大椎」「商曲」諸穴上推拿片刻,竟是毫不見效。

這一來,兩人心下均是駭然。張無忌心道︰「他崑崙派的點穴功夫確是厲害,胡青牛先生傳授了我七種解開被點穴道的方法,但在他身上竟是每一種都不管用。」何太沖却想︰「這小子有這許多推拿解穴的法門,手法怪異,勁力直透重穴,當眞了不起。班淑嫻明明點了他身上七八處穴道,却如何半點也奈何他不得?武當派近年來名動江湖,張三丰這老道果然是有他人所難及的本事。那日在武當山上幸虧没有跟武當派動手,否則定要惹得灰頭土臉。他小小孩童已是如此了得,老的大的出起手來,自是更加厲害十倍。」他却不知無忌「不受點穴」的功夫學自謝遜,而解穴的本事學自胡青牛。武當派自有他威震武林的眞才實學,但無忌這兩項本領,却和武當派無関。

何太沖見他解穴無效,心念一動,道︰「你拿茶壼過來,給我喝幾口茶。」張無忌不知他何以突然要在此時喝茶,但想他顧忌愛妾的性命,不敢對自己施什麼手脚,便提起茶壼,餵他飲茶。何太沖滿滿吸了一口,却不吞下,對準了自己肘彎裡的「清冷淵」用力一噴。只見一條水箭筆直衝出,嗤嗤有聲,登時將他手上穴道解了。

張無忌來到崑崙山三聖堂後,一直便見何太沖爲了五姑疾病煩惱,畏妻寵妾,懦弱猥瑣,便似個尋常没志氣的男子,此時見他初次顯現功力,不由得身子一震,大吃一驚︰「這位崑崙派掌門的武功如此深厚,我可一直對他瞧得小了。看來他並不在兪二師伯、金花婆婆、滅絶師太諸人之下。我但見到他平庸顢頇的一面,没想到他身爲崑崙派掌門,自有人所難及之處。這道水箭若是噴在我臉上胸口,立時便須送命。」

只見何太沖將右臂轉了幾轉,解開了自己腿上穴道,説道︰「你先將解藥給她服了,我送你平安出谷。」張無忌緩緩搖了搖頭。何太沖急道︰「我是崑崙掌門,難道會對你這孩子失信?倘若毒性發作,那便如何是好?」張無忌道︰「毒性不會便發。」何太沖嘆了口氣,道︰「好吧,咱們悄悄出去。」

兩人跳出窗去,何太沖伸指在楊不悔背心輕輕一拂,登時解了她的穴道,手法猶如行雲流水,輕靈無比。張無忌好生佩服,眼光中流露出欽仰的神色來,他自和何太沖相見以來,從未有過這種尊崇的感覺。何太沖懂得他的心意,微微一笑,一手擕著一人,繞到三聖堂的後花園,從側門走出。那三聖堂前後共有九進,出了後花園的側門,經過一條曲曲折折的花徑,又穿入許多廳堂之中,若不是何太沖帶領,張無忌非迷路不可,便是没有崑崙派弟子攔阻也未必能闖得出來。這一來,他對崑崙派的敬重之心,又增了幾分。一離開三聖堂,何太沖右手將楊不悔抱在臂彎,左手拉著張無忌,展開輕功,向西北疾行。無忌給他帶著,身子輕飄飄的,一躍便是丈餘,足尖在地下一點,又是進了丈餘,但覺風聲呼呼在耳畔掠過,便是騎著快馬也没這般迅捷。一轉眼間,三人已奔出二十餘里,張無忌非但毫不用力,而且宛似凌空飛行,冩意非凡。正行之間,忽聽得一個女子聲音叫道︰「何太沖\dash{}何太沖\dash{}給我站住了\dash{}」這聲音順風傳來,似乎極爲遙遠,又似便在身旁,正是班淑嫻的口音。

何太沖微一遲疑,當即立定了脚步,嘆了口氣,説道︰「小兄弟,你們兩個快些走吧,内人追趕而來,我不能再帶你們走了。」張無忌心想︰「這人待我還不算太壞。」便道︰「何先生,你回去便是。我給五夫人服食的並不是毒藥,更不是什麼『鴆砒丸』,只是一枚潤喉止咳的『桑貝丸』。前幾日不悔妹妹咳嗽,我製了給她服用,還多了幾丸在身邉,不免嚇了你一跳。」何太沖又驚又怒,喝道︰「當眞不是毒藥?」張無忌道︰「五夫人自我手中救活,我怎能又下毒害她。」只聽得班淑嫻的叫聲不斷傳來︰「何太沖\dash{}何太沖\dash{}你逃得了麼?」那聲音又近了一些。何太沖所以帶無忌和不悔逃走,完全是爲了怕愛妾毒發不治,拍拍拍拍四個耳光,打得無忌雙頰腫起,滿口都是鮮血。張無忌見他第五掌又打過來,忙使一招「亢龍有悔」,往他手掌迎擊過去。這是「降龍十八掌」中的一掌,倘若學會了,原是威力無窮,但無忌只學到一點膚淺皮毛,如何能和崑崙派的掌門人爭鬥?何太沖見他一掌擊來,招數特異,顯是極上乘的武功,輕輕「咦」的一聲,側身避開,拍的一掌,又打在無忌右眼之上,只打得他眼睛立時腫起。無忌一招無效,知道自己本領跟他差得太遠,索性垂手立定,不再抗拒。何太沖却並不因他不動手而罷手,仍是左一掌右一掌的打個不停。他出掌時並未運用内力,否則只怕一掌便能將無忌震死,但饒是如此,每一掌打到,都使無忌頭暈眼花,疼痛不堪。

他正打得起勁,班淑嫻已率領兩名弟子追到,冷冷的站在一旁。她見無忌並不抵禦,未免無趣,説道︰「你打那女娃子試試。」何太沖身形一斜,吧的一聲,打了楊不悔一個耳括子。楊不悔吃痛,登時哇哇大哭。張無忌怒道︰「你打我便了,何必又欺侮這小孩子?」何太沖不理,伸掌又給楊不悔一下。張無忌縱起身來,一頭撞在他的懷中。班淑嫻冷笑道︰「人家小小孩童,尚有情義,能彀臨危護友,那似你這等無情無義的薄倖之徒。」何太沖聽了妻子譏刺之言,滿臉通紅,抓住張無忌後頸,往外丟出,喝道︰「小雜種,見你爹爹媽媽去吧!」這一下用了眞力,將無忌的頭顱對準了山邉的一塊大石摔去。張無忌身不由主,疾飛而出,眼見頭蓋和大那大石相撞,便要腦漿迸裂。

驀地裡旁邉一股力道飛來,將張無忌身子一引,把他帶在一旁。無忌驚魂未定,站在地下,眯著一隻腫得高高的眼睛向旁看去。只見離身五尺之處,站著一個身穿白色粗布長袍的中年書生。班淑嫻和何太沖相顧駭然,這書生何時到來,從何處走來,事先竟是絶無知覺,即使他早早就躱在大石之後,以何太沖夫婦的能爲,也決不能無法發覺。何太沖適纔提起無忌,將他擲向大石,這一擲之力,少説也有五六百斤,但那書生長袖一捲,當即消解了這股大力,將無忌帶在一旁,顯然武功奇高。但見他約莫四十來歳年紀,相貌俊美,只是雙眉略向下垂,嘴邉現著幾條深深皺紋,不免有些衰老淒苦之相。他不言不動地站在當地,神色漠然,似乎心馳遠處,在想什麼事情。

何太沖咳嗽一聲,説道︰「閣下是誰?爲何橫加插手,前來干預崑崙派之事?」那書生深深一揖,説道︰「原來尊駕便是鐵琴先生何前輩了,久仰英名。這一位是何夫人吧?晩輩楊逍。」

\qyh{}楊逍」兩字一出口,何太沖、班淑嫻、張無忌三人不約而同,一齊「啊」的一聲叫了出來。只是無忌的叫聲中又驚又喜,何氏夫婦却是又驚又怒。只聽得刷刷兩聲,兩名崑崙女弟子長劍出鞘,倒轉劍柄,遞給師父師母。何氏夫婦手中長劍青光閃爍,何太沖橫劍當腹,擺著一招「雪擁藍関」勢,班淑嫻則劍尖斜指地下,那是一招「木葉蕭蕭」。這兩招都是崑崙派劍法中精奥,看來輕描淡冩,隨隨便便,但在這兩招之中,却均伏下七八招凌厲之極的後著,只須手腕一抖,劍光暴長,立即便可傷到敵身上七八處要害。

楊逍却似渾然不覺,但聽無忌那一聲叫喊之中,充滿了喜悦,心中微覺奇怪,向他臉上一瞥。這時張無忌滿臉鮮血,鼻腫目青,早給何太沖打得不成模樣,但滿心歡喜之情,還是在他難看之極的臉上流露出來。張無忌道︰「你\dash{}你便是明教的光明使者,楊逍楊伯伯麼?」楊逍點了點頭,道︰「你這孩子,怎地知道我姓名?」張無忌指著楊不悔,説道︰「她便是你的女児啊。」拉過楊不悔來,道︰「不悔妹妹,快叫爸爸,快叫爸爸!咱們終於找到了。」楊不悔睜著圓圓的眼睛,骨溜溜地望著楊逍,道︰「你是我爸爸?我媽媽呢?我是來找媽媽的啊。」原來楊不悔想到媽媽時不住哭鬧,無忌一路上只有哄她,説是跟她去找媽媽。

楊逍心頭大震,抓住無忌肩頭,説道︰「孩子,你説清楚些。她是誰的女児,她媽媽是誰?」他這麼用力一抓,無忌的肩骨格格直響,痛到心底。無忌不肯示弱,不願呼痛,但終於還是啊的一聲叫了出來,説道︰「她是你的女児,她媽媽便是峨嵋派女俠紀曉芙。」楊逍本來臉色蒼白,這時更加没半點血色,顫聲道︰「她\dash{}她有了女児?她\dash{}她在那裡?」雙臂一伸,抱起了楊不悔,只見她被何太沖打了兩掌,兩邉面頰高高腫起,但眉目之間,宛有幾分紀曉芙的俏麗。正想再問,突然看到楊不悔頸中黑色的絲條,當即伸手輕輕一拉。只見絲條盡頭,結著一塊鐵牌,牌上一刻著一個張牙舞爪的魔鬼,那正是他送給紀曉芙的明教「鐵魔令」。這一下再無懷疑,緊緊摟住了不悔,連問︰「你媽媽呢?你媽媽呢?」楊不悔道︰「媽媽不見了,我在尋她。你看見她麼?」楊逍見她年紀太小,説不清楚,眼望無忌,意示詢問。張無忌嘆了口氣,説道︰「楊伯伯,我説出來你别難過。紀姑姑被她師父打死了,她臨死之時\dash{}」楊逍大聲喝道︰「你騙人,你騙人!」只聽得喀的一聲,無忌右邉肩骨已被他捏碎,咕{\upstsl{咚}}咕{\upstsl{咚}}楊逍和張無忌同時摔倒,楊逍手中,還緊緊的抱著女児。

\chapter{雪嶺雙姝}

楊逍乍聞紀曉芙的死訊,昏暈過去,張無忌却是因肩頭劇痛而跌倒。何太沖和班淑嫻對望一眼,兩人心意相通,雙劍齊出,指住了楊逍眉心和咽喉。

原來楊逍是明教中的重要人物,和崑崙派怨仇甚深。當年崑崙派的前輩高人游龍子,就因和他比武不勝,因此活活氣死。班淑嫻和何太沖兩人的師父白鹿子,也是死在明教中人的手裡,只是眞兇是誰,不得而知,説不定就是楊逍,也是毫不奇怪。何氏夫婦跟他驀地裡狹路相逢,心中一直有如十五隻吊桶打水,七上八落,素知他武功精湛,雖是師門深仇,却也不敢貿然便和他動手,那知他竟然暈倒,當眞是天賜良機,兩柄長劍同時指住了他的要害。

班淑嫻道︰「斬斷他雙臂再説。」何太沖道︰「是!」這時楊逍兀自未醒,張無忌疼得滿頭大汗,心中却始終清醒。楊逍雖然捏碎了他的肩骨,可是他天性不會對人記恨,眼見情勢危難,足尖在楊逍頭頂的「百會穴」上輕輕一點。那「百會穴」和腦府相関,這麼一震,楊逍立時醒轉,一睜開眼,但覺寒氣森森,一把長劍的劍尖抵住了自己眉心,跟著青光一閃,又有一把長劍在自己左臂上斬落。楊逍待要出招擋架,爲勢已然不及,何況班淑嫻的長劍制進了他眉心要害,根本便動彈不得,當下一股眞氣運向左臂。何太沖的長劍斬到他左臂之上,突覺劍鋒一滑,斜向一旁,劍刃竟是並不受力,宛如斬上了什麼又光又韌的物件,但白袍的衣袖變紅,還是斬傷了他。

便在此時,楊逍的身子猛然向前滑出丈餘,好似有人用繩索縛住他頭頸,以快迅無倫的手法向前拉扯一般。班淑嫻的劍尖本來抵住他的眉心,他身子向前急滑,劍尖便從眉心經過鼻子嘴巴胸膛,劃了一條長長的血痕,深入半寸。這一招實是險極,倘若班淑嫻的劍尖再深了半寸,楊逍已是慘遭開膛破腹之禍。他身子滑出,立時便直挺挺的站直。這兩下動作,全是決不可能,但見他膝不曲、腰不彎,陡然滑出,陡然站直,便如全身裝上了機括彈簧,而身體之僵硬怪詭,又和僵屍無異。

楊逍身剛站起,雙脚踏出,喀喀兩響,將何氏夫婦手中長劍同時踏斷。以何氏夫婦劍法上的造詣而論,楊逍武功再強,也決不能一招之間,便將他二人兵刃踏斷,只是他招數怪異,左臂和臉上都受了重傷,却突然反擊。何氏夫婦驚駭之下,不及收劍,以致落敗。楊逍雙足踢出,從兩柄長劍上折斷下來的劍頭激飛而起,分向兩人射去。何氏夫婦各以半截斷劍擋格,但覺虎口一震,半身發熱,雖將劍頭格開,却已是吃驚不小,急忙抽身後退,一個在西北方,一個站在東南方,雖然手中均只剩下半截斷劍,但陽劍指天,陰劍向地,兩人雙劍合壁,使的是崑崙派「兩儀劍法」,氣定神閒,凝若山嶽,確是名家高手的氣度。

崑崙派兩儀劍法成名垂數百年,是天下有名劍法之一,何氏夫婦同門學藝,從小練到大,精熟無比,這劍法施展起來更是威力倍增。楊逍和崑崙派數度大戰,知道這劍法的厲害之處,雖然心中不懼,但知要擊敗二人,非在五六百招之後不可,此刻心中只想著紀曉芙的生死,那有心情爭鬥?何況手臂和臉上的傷勢均是不輕,若是流血不止,也是麻煩,於是冷冷的道︰「崑崙派越來越不長進了,今日暫且罷手,日後再找賢伉儷算帳。」一手仍是抱著楊不悔,另一手拉起張無忌,也不見他提足抬腿,突然間倒退丈餘,一轉身,已在數丈之外。何氏夫婦相顧駭然,好容易這大魔頭自行離去,却那裡敢追?楊逍帶著二小,一口氣奔出數里,忽然停住脚步,問無忌道︰「紀曉芙到底怎樣了?」

楊逍奔得正急,那知他説停便停,疾奔時勢若狂飆,陡止時靜如淵停,張無忌收勢不及,向前一衝,若非楊逍將他拉住,已是俯跌摔倒,聽楊逍這般問,喘了幾口氣道︰「紀姑姑已經死了,你相信也好,不相信也好,用不著捏碎我的肩骨。」楊逍臉上閃過一絲歉色,隨即又問︰「她怎麼會死?」張無忌喝下了班淑嫻的毒酒,雖是嘔出了大半,在路上又服了解毒藥,但毒質未曾去盡,這時腹中又疼痛起來。他取出金冠血蛇,讓他咬住自己左手食指吸毒,一面緩緩將如何識得紀曉芙,如何替她治病,如何見她被滅絶師太擊斃的情由一一説了,待得説完,金冠蛇已吸盡了他體内的毒藥。

楊逍又細問了一遍紀曉芙臨死時的言語,垂泪道︰「滅絶惡尼是逼她來害我,只要她肯答應,那便是替峨嵋派立下大功,便可繼承掌門人之位。唉,曉芙啊,曉芙,你寧死也不肯答應,其實,你只須假裝答應,咱們不是便可相會,便不會喪生在滅絶惡尼的手下了麼?」張無忌道︰「紀姑姑爲人正直,她不肯暗下毒手害你,却不肯虛言欺騙恩師。」楊逍淒然苦笑,道︰「你倒是曉芙的知己\dash{}豈知她恩師却能痛下毒手,取她性命。」張無忌道︰「我答應紀姑姑,將不悔妹妹送到你手\dash{}」楊逍身子一顫道︰「不悔妹妹?」轉頭向楊不悔道︰「孩子,你姓什麼?叫什麼名字?」楊不悔道︰「我姓楊,名叫不悔。」楊逍仰天長嘯,只震得四下裡高葉簌簌亂落,良久方絶,説道︰「你果然姓楊。不悔,不悔,好!曉芙,我雖是強逼於你,你却並没懊悔。」張無忌聽紀曉芙説過他二人之間的一場孼緣,這時眼見楊逍英俊瀟灑,年紀雖然稍大,仍不失爲一個風度翩翩的美男子,比之稚氣猶存的殷利亨六叔,實是更易使女子傾倒。紀曉芙自被迫失身終至對他眞心相戀,須也怪她不得。

張無忌左肩骨破碎,痛得大是難熬,接骨和止痛的草藥一時找不到,只得先行理齊碎骨,摘些些消腫的草藥敷上,折了兩根樹枝,用樹皮將樹枝綁在肩臂之上。楊逍見他小小年紀,單手接骨治傷,手法竟是十分熟練,心中微覺驚訝。

張無忌綁紮完畢,説道︰「楊伯伯,我没負紀姑姑所託,不悔妹妹已找到了爸爸,咱們便此别過。」楊逍道︰「你萬里迢迢,將我女児送來,我豈能無所報答?你要什麼,儘管開口便是,我楊逍做不到的事,拿不到的東西,天下只怕不多。」張無忌哈哈一笑,道︰「楊伯伯,你也把紀姑姑瞧得忒也低了,枉自叫她爲你送了性命。」楊逍臉色大變,喝道︰「你説什麼?」張無忌道︰「紀姑姑没將我瞧低,纔託我送她女児來給你。若是我有所求而來,我這人還値得託付麼?」他心中在想︰「一路上不悔妹妹遭遇了多少危難,我多少次以身相代?倘若我是貪利無義的不肖之徒,今日你父女焉得團圓?」只是他不喜表伐自己的功勞,途中的困厄一句也没提起,説了那幾句話,躬身一揖。轉身便走。楊逍道︰「且慢!你幫了我這個大忙,楊逍自來有仇必報,有恩必報。你隨我回去一年之内,我傳你幾件天下罕有敵手的功夫。」張無忌親眼見到他踏斷何氏夫婦手中長劍,武功之高,江湖上實是少有其匹,便是學到他的一招半式,也必大有好處,但想起太師父曾諄諄告誡,決不可和魔教中人多有來往,何況自己不過再有半年壽命,便是學得舉世無敵的武功,又有何用?當下説道︰「多謝楊伯伯垂青,但晩輩是武當子弟,不敢來學别派高招。」楊逍「哦」的一聲,道︰「原來你是武當派中弟子!那殷利亨\dash{}殷六俠\dash{}」

張無忌道︰「殷六俠是我師叔,自先父逝世,殷六叔待我和親叔叔没有分别。我受紀姑姑的囑託,送不悔妹妹到崑崙山來,對殷六叔可不免\dash{}不免心中有愧了。」楊逍和他目光相接,自己也是心下慚愧,右手一擺,説道︰「既是如此,後會有期。」身形一晃,已在數丈之外。楊不悔大叫︰「無忌哥哥,無忌哥哥!」但楊逍展開輕功,頃刻間已奔得甚遠,那「無忌哥哥」的呼聲漸遠漸輕,終於叫聲和人形倶杳。

無忌悄立半晌,他和楊不悔萬里西來,形影相依,突然分手,心中甚感黯然。這時肩頭碎骨處又疼痛起來,於是繞過山嶺,儘揀荒僻處走去,想先找些接骨止痛的草藥敷上再説,又怕再和何太沖班淑嫻等崑崙諸人碰面,只是往山深處行走。那崑崙山一帶,花草樹木和中原大異,胡青牛醫書上所載的草藥,竟是一項也尋不著。走了二十餘里,無忌傷口加痛,於是坐在一堆亂石上休息,忽然聽得西北方傳來一陣犬吠之聲,聽聲音共有十餘頭之多。犬吠聲越來越近,似在追逐什麼野獸。

犬吠聲中,一隻小猴子急奔而來,後股上帶了一枝短箭。那猴児奔到離無忌十餘丈外,打了個滾,牠股上中箭之後,不能竄高上樹,這時筋力竭,再也爬不起來。無忌走過去一看,那猴児目光中露出恐懼和乞憐的神色。無忌觸動心事︰「我被崑崙派衆人追逐,正和你一般狼狽。」又想起児時在冰火島上時那隻玉面火猴,於是將猴児抱起,輕輕拔下短箭,從懷中取出金創藥來,給牠敷在傷口。便在此時,犬吠聲已響到近處,無忌拉開衣襟,將猴児放在懷内,只聽得汪汪汪幾聲狂吠,十餘頭身高齒利的獵犬已團團將無忌圍住。那些獵犬嗅到猴児的氣息,圍著無忌,張牙舞爪的發威,一時還不敢撲將上來。無忌見了這些獵犬露出白森森的長牙,神能兇狠,心中大是害怕,知道只須將懷中的猴児擲出,群犬自會去撲擊猴児,不再和自己爲難。但他自幼受父母陶冶,天生的俠義心腸,雖對一隻野獸,也不肯相負,於是提一口氣,從群犬頭頂飛躍而過,邁開步子急奔。群犬胡胡猛吠,在後追來。

那些獵犬奔跑時何等迅速,無忌只逃出十餘丈,就被群犬追上,只覺腿上一痛,已被一頭猛犬咬中,牢牢不放。他回身一掌,擊在那獵犬頭頂,這一掌使力極重,竟將那頭獵犬打得翻了個觔斗,昏暈過去。其餘的獵犬毫不畏懼,蜂擁而上,無忌拳打足踢,奮力與抗,他左肩骨碎,左臂不能轉動,不久便被一頭惡犬咬住了左手,但見四面八方,群犬撲上亂咬,頭臉肩背,到處被群犬的利齒咬中,昏亂之中,隱隱似聽得幾聲清脆嬌嫩的呼叱之聲,但這些聲音好像很遠很遠,和他全没干係,他眼前一黑,什麼都不知道了。

他做了許多許多惡夢,看見無數豺狼虎豹,不住的在咬他身體,他要張口大叫,却又叫不出半點聲音。也不知過了多少時候,這些野獸方才退去,只聽得一個人的聲音説道︰「退了燒啦,或許死不了。」無忌睜開眼來,先看到一點淡黃的燈火,發覺自己是睡在一間小室之中,一個中年漢子站在身前。無忌道︰「大\dash{}大叔\dash{}我怎\dash{}」只説了這幾個字,猛覺全身火燙般疼痛,這纔慢慢想起,自己曾被一群惡犬圍著狂咬。那漢子道︰「小子,算你命大,死不了。怎樣?肚餓麼?」無忌道︰「我\dash{}我在那裡?」各處傷口同時劇痛,又昏暈了過去。

待得第二次醒來,那中年漢子已不在室中。無忌心想︰「我明明活不長久了,何以又要受這許多折磨?」一低頭,見自己胸前項頸,手臂大腿,到處都縛上了布帶,一陣藥草氣息,甚是刺鼻,原來已有人在他傷處敷了傷藥。

張無忌聞到那藥物的氣息,即知替他敷藥那人,對治傷一道,所知甚是膚淺。藥物之中,顯是有杏仁、馬前子、防風、南星諸味藥物,這些藥若治瘋犬咬傷,用以拔毒,原具靈效,但咬他的並非瘋狗,他是筋骨肌肉受損而不是中毒,藥不對症,反而多增痛楚。但他不知自己身在何處,又無力起床,直挨到天明,那中年漢子纔又來看他。

無忌道︰「大叔,多謝你救我。」那漢子冷冷的道︰「又不是我救你的,謝我什麼?」無忌道︰「這是什麼地方?是誰救我來的?」那漢子道︰「這児是紅梅山莊,咱們小姐救你來的。你肚子餓了吧?喝幾碗熱湯提提神。」説著出去端了一碗熱粥進來,粥碗上堆著一小堆肉鬆。無忌喝了幾口,但覺胸口煩惡,頭暈目眩,便吃不下了。

一直躺了八天,纔勉強起床,脚下虛飄飄的没一點力氣,他自知失血過多,看來一時不易復原。那漢子每日跟他送飯換藥,雖然神色之間顯得頗爲厭煩,但無忌還是十分感激,只是見他不喜説話,心中縱有滿腹疑團,却不敢多問。這天見他拿來的藥物仍是防風南星之類搗爛的藥糊,無忌忍不住説道︰「大叔,這些藥不大對症,勞你駕給我換幾味成不成?」那漢子翻著一對白眼,上上下下打量張無忌,隔了良久,纔道︰「老爺開的藥方,還能錯得了麼?你説藥不對症,怎地將你死人也治活了?眞是的,小孩子家胡言亂語,咱們老爺雖然寬洪大量,就算聽到了也不會見怪,可是你也不能太過不識好歹啊。」説著便將藥糊在無忌傷口上敷了下去,無忌只有苦笑。那漢子道︰「小兄弟,我瞧你身上的傷也大好了,該得去向老爺、太太和小姐磕幾個頭,謝謝救命之恩。」無忌道︰「那是該當的,大叔,請你領我去。」

那漢子領著他出了小室,經過一條長長的走廊,又穿過兩進廳堂,來到一座暖閣之中。這時已屆隆冬,崑崙一帶早已極爲寒冷,那暖閣中却是溫暖如春,可又不見何處生著炭火,但見暖閣正中掛著一幅工筆仕女主軸,几上一隻大膽瓶中斜插著幾枝紅梅,榻上椅上,都舖著錦緞軟墊。無忌一生之中,從未見過這等富麗舒適的所在,自顧全身衣衫破爛寒蠢,站在這豪華的暖閣中實是大不相稱,不由得大起自慚形穢之感。

暖閣中無人在内,那漢子臉上的神色却極是恭謹,躬身稟道︰「那給狗児咬傷的小子好了,來向老爺太太磕頭道謝。」説了這幾句話後,垂手站著,連透氣也不敢用勁。過了好一會,只見屏風後面走出一個十五六歳的少女來,向無忌斜睨了一眼,發話道︰「喬福,你也是的,怎麼把他帶到這裡?他身上臭蟲虱子跳了下來,那怎麼算啊?」喬福應道︰「是,是!」無忌本已侷促不安,聽了那少女這幾句話,更是羞得滿臉通紅,要知他除了身上一套衣衫之外,並無替換衣服,確是生滿了虱子跳蚤,心想這位小姐説得半點不錯。但見她一張鵞蛋臉,頗爲艷麗,烏絲垂肩,身上穿的不知是什麼綾羅綢緞,閃閃發光,腕上戴著一隻精緻異常的金鐲,這等裝飾華貴的小姐,無忌也是從來没有見過,心想︰「我被群犬圍攻之時,依稀聽得有個女子的聲音喝止。那位喬福大叔又説,是他小姐救了我的。我理當叩謝纔是。」於是跪下磕頭,説道︰「多謝小姐救我一命,張無忌終身不忘。」

那少女一愕,突然間格格嬌笑起來,説道︰「喬福,喬福,你怎麼啦?你作弄這傻小子,是不是?」喬福笑道︰「小鳳姊姊,這傻小子就向你磕幾個頭,你也不是受不起啊。這傻小子没見過世面,見了你當是小姐啦!」無忌吃了一驚,忙站起身來,心想︰「糟糕!原來她是丫鬟,我可將她認作了小姐。」臉上又紅又白,{\upstsl{尷}}尬非常。

小鳳忍著笑,向張無忌上上下下的打量。他臉上汚血未除,咬傷處裹滿了布條,自知極是穢臭難看,恨不得地下有個洞便鑽進去。小鳳舉袖掩鼻,道︰「老爺太太正有事呢,不用磕頭了,去見見小姐吧。」説著遠遠繞開無忌,當先領路,唯恐無忌身上的虱子臭蟲,跳到了自己衣上。

無忌隨在小鳳和喬福之後,一路上見到的婢僕家人,個個衣飾華貴,所經屋宇樓閣,無不精緻極麗。他十歳以前居住冰火島上,此後數年,一半在武當山,一半在蝴蝶谷,飲食起居,均極簡樸,當眞是故夢也想不到世上有這等豪富的人家。

走了好一會,來到一座大廳之外,只見廳上匾額冩著「狂{\upstsl{犺}}居」三字。小鳳先走進廳去,過了一會,出來招招手,喬福便帶著張無忌進廳。無忌一走進廳門,心中便是一驚,但見三十餘頭雄健猛惡的大犬,分成三排,蹲在地上,一個身穿純白狐裘的女郎坐在一張虎皮交椅上,手中執著一根鞭子,嬌聲喝道︰「咽喉!」一頭猛犬急縱而起,向著站在牆邉的一個人咽喉中咬去。無忌見了這等殘忍情景,忍不住「啊喲」一聲叫了出來,却見那狗口中咬著一塊肉,踞地大嚼。無忌定一定神,這纔看清楚那人原來是個皮製的假人,周身要害之處掛滿了肉塊。那少女又喝︰「小腹!」第二條猛犬竄上去便咬那假人的小腹。看來這些猛犬竟是習練有素,應聲咬人,部位絲毫不爽。無忌一怔之下,立時認出,當日在山中狂咬自己的,便是這些惡犬,再一回想,依稀記得那天喝止群犬的聲音,就是這個身穿狐裘的女郎。

他心目中本來想這位小姐救了自己性命,是以要向她叩謝,此刻得知自己受了這般苦楚,全是出於這女郎所賜,忍不住怒氣填胸,心想︰「罷了,罷了!她有惡犬相助,我也奈何她不得。早知如此,寧可死在荒山之中,也不在她家養傷。」撕下身上的繃帶布條,抛在地上,轉身便走。喬福驚道︰「喂,喂!你幹什麼呀?這位便是小姐,還不上前磕頭?」無忌怒道︰「{\upstsl{呸}}!我多謝她?咬傷我的惡犬,不是她養的麼?」

那女郎轉過頭來,見無忌大怒無已的模樣,微微一笑,招手道︰「小兄弟,你過來。」無忌回過頭來,和她正面相對,胸口不知怎地,驀然間突突的跳個不住,但見這女郎約莫十七八歳年紀,容顏嬌媚萬狀,又白又膩,他美女子也見過不少,但生平從未像這一次般的動心,忙低下了頭不看她,本來絶無血色的臉,但是漲得通紅。那女郎笑道︰「你過來啊。」無忌抬頭又瞧了她一眼,但覺她的眼色勾人心魄,竟是無法拒絶,於是慢慢的走近。那女郎站起身來,握住了他雙手。張無忌全身一顫,只覺她兩隻手掌柔嫩溫滑,不由得又窘又急,只想掙脱,却又不捨得掙脱。

那女郎道︰「小兄弟,你惱我了,是不是?」張無忌在群犬的爪牙之下吃了這許多苦頭,如何不惱?但這時給她握住了雙手,相距不過尺許,只覺她吹氣如蘭,一陣陣幽香送了過來,幾欲昏暈,那裡還説得出這個「惱」字,當即搖頭道︰「没有!」那女郎道︰「我姓朱,名叫九眞,你呢?」無忌道︰「我叫張無忌。」朱九眞道︰「無忌,無忌!{\upstsl{嗯}},這名字高雅得很啊,小兄弟想來是個世家弟子了。喏,你坐在這裡。」説著指一指身旁的一張矮凳。張無忌出生以來,第一次感到美貌女子的魔力,這朱九眞便是叫他跳到火坑之中,他也會毫不猶豫的縱身躍下,聽她叫自己坐在她的身畔,眞是説不出的喜歡,當即依言乖乖的坐下。

小鳳和喬福見小姐對這個又髒又臭的小子居然如此垂青,都是大出意料之外。

朱九眞又嬌聲喝道︰「心口!」一隻大狗縱身而出,向那假人咬去。可是那假人心口懸掛的肉塊已先被咬去,那狗便撕落假人脅下的肉塊,吃了起來。朱九眞怒道︰「饞嘴東西,你不聽話麼?」走過去提起鞭子,刷刷便是兩下。那鞭上生滿小刺,兩鞭抽落,狗背上登時現出兩條長長的血痕。那狗想是餓得久了,兀自不肯放下口中的肉食,反而嗚嗚發威。朱九眞道︰「你不聽我話?」,長鞭揮動,打得那狗滿地亂滾,遍身鮮血淋漓。她出鞭之際,手法極是靈動,不論那猛犬如何竄突翻滾,始終逃不出長鞭揮去的圏子,到後來那猛犬伏在地下不動,低聲哀鳴,朱九眞仍不停手,直打得牠奄奄一息,纔道︰「喬福,搭下去敷藥。」喬福應道︰「是,小姐!」將那猛犬抱出廳去,群犬見了這般情景,盡皆心驚膽戰,一動也不敢動。

朱九眞坐回椅中,又喝︰「左腿!」

\qyh{}右臂!」

\qyh{}眼睛!」一頭猛犬依聲而咬,都没錯了部位,朱九眞笑道︰「小兄弟,你瞧這些畜生賤麼?不狠狠的給牠們吃頓鞭子,怎會聽話?」無忌雖在群犬爪牙之下吃過極大苦頭,但見那狗被打的慘狀,心下却也不禁惻然。朱九眞見他不語,笑道︰「你説過不惱我,怎麼一句話也不説?你怎地會到西域來?你爹爹媽媽怎麼了?」張無忌心想,自己如此落魄,倘若提起太師父和父母的名字,徒然辱没了他們,便道︰「我父母雙亡,在中原難以存身,隨處流浪,便到了這裡。」朱九眞笑道︰「我射了那隻猴児,誰叫你偸偸藏在懷裡啊?餓得慌了,想要吃猴肉,是不是?没想到自己險些給我的狗児撕得稀爛。」張無忌脹紅了臉,連連搖頭,道︰「我不是想吃猴児肉。」朱九眞輕輕在他肩上拍了一下,嬌笑道︰「你在我面前啊,乘早别賴的好。」她忽然想起一事,道︰「你學過什麼武功?一掌把我的『左將軍』打得頭蓋碎裂而死,掌力很不錯啊。」

張無忌奇道︰「左將軍?」朱九眞微微一笑,叫道︰「前將軍!」一頭猛犬應聲而出,伏在地下。她又叫︰「車騎將軍!」又有一頭猛犬出來。原來她這數十頭猛犬,都有將軍封號,什麼征東將軍、折衝將軍、平寇將軍、威遠將軍等等,不一而足,她自己指揮若定,儼然是個大元帥了。無忌聽她説自己打死了她的愛犬,心下甚是歉然,説道︰「那時我心中慌亂,出手想是重了。我小時跟爹爹學過兩三年拳脚,不懂什麼叫做武功。」

朱九眞點了點頭,對小鳳道︰「你帶他去洗個澡,換些像樣的衣服。」小鳳抿嘴笑道︰「是!」領了無忌出去。無忌對這位小姐戀戀不捨,走到廳門時,忍不住回頭向她望了一眼。那知朱九眞也正在瞧他,遇到他的眼光時秋波流慧,嫣然一笑。無忌羞得連頭髮根子中都紅了,魂不守舍,也没瞧到地下的門檻,脚下一絆,登時跌了個狗吃屎。他全身都是傷處,這一跌,著地之處,同時劇痛。但他不敢哼出聲來,撐持著慢慢爬起,小鳳吃吃笑道︰「見了我家小姐啊,誰都要神魂顚倒。可是你這麼小,也不老實嗎?」張無忌大窘,搶先便行,走了一會,小鳳笑道︰「你到太太書房去麼?咱們是從這児來的麼?」無忌站定一會,但見前面垂著繡金的軟帘,確是從來没有見過,原來自己慌慌亂亂的又走錯了路。小鳳這丫頭却是狡獪,先又不説,直等到他錯到了家,這纔出言譏刺。無忌紅著臉低頭不語。小鳳道︰「你叫我聲小鳳姊姊,求求我,我纔帶你出去。」無忌道︰「小鳳姊姊\dash{}」小鳳右手的一根食指指著面頰,一本正經的道︰「{\upstsl{嗯}},幹什麼啊。」

張無忌道︰「求求你,帶我出去。」小鳳笑道︰「這纔是了。」帶著他回到那間小室之外,對喬福道︰「小姐吩咐了,給他洗個澡,換上件乾淨衣衫。」喬福道︰「是,是!」答應得很是恭敬,看來小鳳雖然也是下人,但身份却又比尋常婢僕爲高。五六個男僕一齊走上,你一聲「小鳳姊姊」,我一聲「小鳳姊姊」,小鳳却愛理不理的,突然向無忌福了一福。無忌愕然道︰「怎\dash{}怎麼?」小鳳笑道︰「先前你向我磕頭,這時跟你還禮啊。」説著翩然入内。喬福將無忌把小鳳認作小姐向她跪下磕頭的事説了,説時加油添醬,形容得十分不堪,群僕鬨堂大笑。無忌低頭入房。却不生氣,只是將小姐一笑一嗔,一言一語,在心坎裡細細咀嚼回味。

一會児洗過澡,見喬福拿來給他換的衣衫,青布直身,竟是僮僕裝束。無忌怔了良久,心想︰「我又不是你家低三下四的奴僕,如何叫我穿這等衣裳?」當下有心不穿,仍是穿回自己原來的破衣,却見肌膚都從群犬咬爛的破洞中露了出來,又想︰「待會小姐叫我前去説話,見我仍是穿著這等骯髒的破衫,定然不喜。其實我便是眞的做她奴僕,又有什麼不好?」這麼一想,登時心中坦然,便換上了僮僕的直身。

那知别説這一天小姐没來喚他,接連十多天,連小鳳也没有見到一面,更不用説小姐了。張無忌痴痴呆呆,只是想著小姐的聲音笑貌,但覺世上女子之中,再無一人比她更爲可愛的了。有心想到後院,遠遠瞧瞧小姐一眼也好,聽她向别人説一句話也好,但喬福叮囑了好幾次,若非主人呼喚,決不可走進中門以内,否則必爲猛犬所噬。無忌想起群犬的兇惡神態,雖是滿腔渴慕,終於不敢走到後院。

又過一月有餘,他被群犬咬傷之處已然痊癒。但臉上手上,却已留下了幾個無法消除的齒痕疤印,無忌可毫不著惱,每當想起這是小姐愛犬所傷,心中反而有一些甜甜之感。這些日中,他身上寒毒仍是每隔七日便發作一次,每發一回,便厲害一回。這一日,寒毒又作,他躺在床上,將棉被裹得緊緊的,牙関不住打戰。喬福走進房來,他見得慣了,也不以爲異,説道︰「待會好些,喝碗臘八粥吧!這是太太給你的過年新衣。」説著將一個包裹放在桌上。

無忌直熬過半夜,寒毒纔慢慢減弱。他打開包裹一看,見是一套新縫的皮衣,襯著雪白的長毛羊皮,心中也自喜歡,只是那皮衣似是裁作僕僮裝束,看來朱家是將他當定是奴僕了。無忌生來性情溫和,處之泰然,也不以爲侮,只想︰「想不到在這裡一住月餘,轉眼便要過年。胡先生説我只不過一年之命,這一過年,第二個新年是不能再見到了。」

富家大宅之中,一到年盡歳尾,便加倍有一番熱鬧氣象,衆人忙忙碌碌,刷牆漆門、殺豬宰羊都是好不興頭。無忌幫著喬福做些雜事,只盼年初一快些到來,心想給老爺、太太、小姐磕頭拜年,定可見到小姐,只要再見她一次,我便悄然遠去,到深山中自覓死所,免得整日和喬福等這一干無聊僮僕爲伍。

好容易爆竹聲中,盼到了元旦,無忌跟著總管,到大廳上向主人拜年。只見大廳正中坐著一對面目清秀中年夫婦,七八十個僮僕跪了一地,主人夫婦一時也瞧不明白。只見那對夫婦笑嘻嘻的道︰「大家都辛苦了!」旁邉便有兩名管家分發賞金,無忌也得到了四兩銀子。他不見小姐,心中十分失望,拿著那錠銀子正自發怔,忽聽得又嬌又媚的聲音從外面傳來︰「表哥,你今年來得好早啊。」正是朱九眞的聲音。一個男子聲音笑道︰「跟舅舅、舅母拜年,敢來得遲麼?」

\chapter{花園較技}

張無忌臉上一熱,一顆心幾乎要從胸腔中跳了出來,兩手掌心都是汗水,他盼望了整整兩個月,纔再聽到朱九眞的聲音,教他如何不神搖意奪?只聽得又有一個女子的聲音笑道︰「師哥這麼早來,也不知是給兩位尊長拜來呢,還是給表妹拜年?」説話之間,廳門中走進三個人來。群僕紛紛讓開,張無忌却失魂落魄般站起,直到喬福使勁拉了他一把,這纔走在一旁。

只見進來的三人中間是個青年男子。朱九眞走在左首,穿著一件猩猩紅的貂裘,更襯得她臉蛋児嬌嫩艷麗,不可方物。那青年的另一旁也是個女子,三人似乎都是差不多年紀。自朱九眞一進廳,無忌的眼光没再離開他臉児,也没瞧見另外兩個青年男女是俊是醜,穿紅著綠?那二人向主人夫婦如何磕頭拜年,賓主説些什麼,他全都視而不見,聽而不聞,眼中所見,便只朱九眞一人。其實他年紀尚小,對男女之情,只是一知半解,更非急色之徒,但每人一生之中,初次知好色而慕少艾,無不神魂顚倒,如痴如呆,固不僅無忌一人爲然。只是他天性對人多情,不論對方男女老幼,均是如此,何況朱九眞容色絶麗,無忌在顚沛困厄之際與之相遇,竟致傾倒難以自持。他也決非有什麼非分之想,只覺能多瞧她一眼,多聽她説一句話,心中便喜樂無窮了。

衆僮僕領了賞,逐漸散去。主人夫婦和三個青年説了一會,只聽朱九眞道︰「爸,媽,我和大哥、青妹玩去啦!」主人夫婦微笑點頭,三個青年男女並肩走向後院。張無忌不由自主,遠遠的跟隨在後。這天是大年初一,衆婢僕玩耍的玩耍,賭錢的賤錢,誰也没有理他。這時無忌才看明白了,那男子英俊溫雅,身長玉立,實是個罕見的美男子,雖在這等大寒天候,却只穿了一件薄薄的黃色緞袍,顯是内功頗有火候。那女子穿著黑色的貂裘,身形苗條,言語舉止,極有斯文,説到相貌之美,和朱九眞可説各有千秋,但此刻在張無忌眼中瞧出來,自是大大不如他心目中敬如天仙的小姐了。

三人一路説笑,一路走向後院。那少女道︰「眞姊,你的一陽指功夫,練得又深了兩層吧?顯露一手給妹子開開眼界好不好?」朱九眞道︰「啊喲,你這不是要我好看麼?我便是再練十年,也及不上你武家蘭花拂穴手的一拂啊。」那青年笑道︰「你們兩個誰都不用謙虛了,大名鼎鼎的『雪嶺雙姝』,一般的威風厲害。」朱九眞道︰「我獨個児在家中瞎琢磨,那及得上你師兄妹倆有商有量的進境快?今日喂招,明児切磋,那還不一日千里嗎?」那少女聽她言語中隱含醋意,抿嘴一笑,並不答話,竟似給她來一個默認。

那青年似怕朱九眞生氣,忙道︰「那也不見得,你有兩個師父,舅父舅母一起教,不是又比咱強麼?」朱九眞道︰「咱們咱們的?哼,你們同門師兄妹,自是親過表妹了。我跟青妹説著玩,你總是一股児幫著她。」説著扭過了頭不理他。那青年陪笑道︰「表妹親,師妹也親,我是一般厚薄,不分彼此。」朱九眞倏地轉過身來,説道︰「表哥,聽説你師父也收了一個女弟子,是不是?」那青年道︰「是的。」那少女似乎存心氣她,微笑道︰「眞姊,我那個小師妹美貌得緊呢,又會説話。又討人喜歡,整日價便是纏住了師哥,要他教這樣教那樣的。趕明児你見到了她,一定也會打從心児裡愛她。」朱九眞冷冷的道︰「是麼?難道比青妹你還美麗麼?」那少女微笑道︰「我怎及得上這個小師妹,除非是眞姊,方能跟他比一比。」

朱九眞道︰「我又不是風流瀟灑的美男子,怎地會見一個愛一個?」那男子聽她辭鋒直指自己,忙岔開話頭,笑道︰「表妹,你帶我去拜訪你那些守門大將軍,好不好?一定給你調教得越來越厲害了。」朱九眞高興了起來,道︰「好!」領著他們,逕往狂{\upstsl{犺}}居去。張無忌遠遠跟在後面,但見三人又説又笑,却聽不見説些什麼,當下也跟到了狗場之中。朱九眞命飼養群犬的狗僕放了衆犬出來。諸犬聽令行事,無不凜遵。那青年不住口的稱讚,朱九眞很是得意。那少女抿嘴笑道︰「師哥,你將來是『冠軍』呢還是『驃騎』啊?」那青年一怔,道︰「你説什麼?」那少女道︰「你這麼聽眞姊的話,眞姊還不賞你一個『冠軍將軍』或是『驃騎將軍』的封號麼?只不過要小心她的鞭子纔是。」要知朱九眞所養的猛犬或稱「征東將軍」,或稱「威遠將軍」,隻隻都有將軍封號,那少女這般説,乃是譏笑那青年與犬爲伍。那青年俊臉通紅,眉間頗有惱色,道︰「胡説八道!你罵我是狗麼?」那少女微笑道︰「這些將軍們長侍美人粧台,搖尾乞憐,冩意得緊啊,有什麼不好?」

朱九眞臉一沉,道︰「青妹,我又没得罪你,怎地大年初一就來跟我過意不去?」那少女顯得大是詫異,説道︰「咦?我巴巴的來跟你拜年,怎地跟你過不去了?」朱九眞哼了一聲,心想雙方尊長都是世代交好,心中雖然惱極了她,却是不便翻臉,問那個青年道︰「表哥,你倒來評評這個理,是得罪了武小姐呢,還是她故意來跟我吵架?」那青年頗感爲難,既不能幫表妹,也不能幫師妹,兩個女孩子都是嬌生慣養,心胸狡窄的姑娘,不論偏袒了那一個,日後都是受罪無窮,唯一的法子便是顧左右而言他,於是笑道︰「表妹,咱們好久不見了,説這些氣話幹什麼?我問你,舅舅舅母這些日子傳了你什麼厲害的武功,露幾手給我觀摩成不成?」

朱九眞微一沉吟,道︰「前幾天爹爹教了我一路筆法,只是我没學好,請青妹和表哥指點。」那青年和少女一齊叫好,説道︰「别客氣啦,讓我們見識見識,一開眼界。」朱九眞一擺手,在旁伺候的狗僕便從壁上摘了一對判官筆下來。張無忌見牆壁上掛了許多兵刃,但長長短短的判官筆最多,似乎朱小姐平時擅使判官筆。他父親張翠山號稱「銀鉤鐵劃」,原是使判官筆的名家,平時和他講論武功時,説到兵刃,自以談到單鉤和判官筆兩種兵器爲多,因此張無忌對判官筆的招數也相當熟習,心想︰「曾聽爹爹説過,武林中從未見過有女子使判官筆。這位朱小姐居然用這種兵刃,武功自是高強。」他對朱九眞已傾心得如痴如呆,待見她所用兵刃和自己父親一樣,更增三分傾倒。只見她取了雙筆在手,左筆輕輕一擺,説道︰「青妹,你來跟我餵餵招啊,這路筆法一個人不能練。」那少女知她存心不良,有意要自己出醜,搖頭道︰「我這點微末道行,怎跟眞姊墊手?」朱九眞連聲催促,那少女總是不肯下場。那青年見勢成僵局,緩步而出,拱手道︰「表妹,我來陪你玩,可是你得讓我些児,朱家判官筆要是點中了我『膻中』『百會』,衛璧今年可没年酒喝了。」要知膻中、百會等穴都是人身極要緊的穴道,點中即死。朱九眞給他奉承得很是歡喜,笑著叱道︰「油嘴表哥!看招!」左筆下,右筆上,當眞是分點他頂門「百會」、胸口「膻中」兩穴。

雙筆勢出如風,電閃而至,衛璧竟是不閃不避,似乎料到朱九眞決計不會當眞傷他要害,那知朱九眞雙筆極是狠辣,認穴之準,不差分毫,一晃眼間,雙筆筆尖和他兩穴相去已不盈寸。衛璧在千鈞一髮的當児,仍是笑道︰「當眞要表哥的性命麼?」青光閃處,叮叮兩聲輕響,不知他何時已是長劍在手,架開了朱九眞的判官筆。朱九眞嬌聲喝道︰「好!」雙筆縱橫,舞成了兩道白氣。張無忌在一旁瞧得心曠神怡,他曾聽父親説道︰這判官筆固然是點穴打穴的利器,但因帶了一個「筆」字,乃是武林中有文的兵刃,貴在瀟灑自如,姿態飄逸,倘若一味蠻打惡鬥,不免落了下乘。這時他旁觀朱九眞的筆路。當眞是深得判官筆的三味,一時如瑤台簪花,嬌媚自喜,一時又若天馬行空,不可羈勒。張無忌看了一會,心中一動︰「她這路判官筆法,就如我爹爹的『倚天屠龍功』一般,也是脱胎於書法。」

再看衛璧的劍術,也是精妙入神,只是張無忌不懂劍術,便未能領略其中的好處。鬥了一會,衛璧左支右撐,似乎越來越招架不住,只見朱九眞左手筆自右向左一掠,右手筆驚雷奔電般的劃了下來。衛璧「啊喲」一聲,騰騰騰向後倒退三步,朱九眞得理不讓人,右筆指向他胸腹之交的「巨闕穴」,左筆指向他臍眼「神闕穴」,這一招「雙闕歸元」,甚是厲害不過。衛璧舉起長劍,伸了伸舌頭,道︰「我投降啦!大小姐饒命!」説著雙膝微屈,作個下跪之勢。

朱九眞很是得意,笑道︰「承讓,承讓!」斜轉向右,雙筆脱手擲出,錚錚兩響,末入磚牆之中,筆尾露出在外者不過數寸,别看她嬌柔婀娜,内力還眞示小。張無忌忍不住脱口喝采︰「好啊!」他跟在朱九眞身後,來到狗場,爲時已久,但誰也没加留意,這聲喝采一出口,他登時後悔不迭。場上衆人一齊回頭瞧著他,朱九眞先見是個僮児,也不理睬,她早就忘了兩個月前群犬咬傷張無忌之事,向衛璧道︰「表哥,我這路筆法破綻百出,你給我指點指點。」衛璧笑道︰「我要是能指點,還能輸在你手上嗎?表妹,你這路功夫好看得緊,攻勢又很凌厲,叫什麼名字啊?」

朱九眞雙手叉腰,道︰「你倒猜上一猜。」衛璧搔搔頭,道︰「舅舅是世代家傳的書法名家,這路武功好像是從書中變化出來的。」朱九眞拍手笑道︰「不錯!是什麼書法呢?」衛璧道︰「好表妹,你别考究我啦,我可説不上來。」張無忌站在一旁,見朱九眞跟衛璧説話時滿臉春風,心下早就説不出的難過,只想能有什麼法児可以壓倒這個英俊美貌的青年,這時胸口一熱,衝口而出︰「大江東去帖!」

原來朱九眞是朱子柳的後人,那姓武的少女名叫武青嬰,是武三通的後人,屬於武修文一系。武三通和朱子柳都是一燈大師的朝臣兼弟子,武功原是一路。但百餘年後傳了幾代,兩家後人所學便各有增益變化,例如武敦儒、武修文兄弟拜大俠郭靖爲師,雖然也學「一陽指」神功,但武功便近於九指神丐洪七公一派剛猛的路子。衛璧是朱九眞的表哥,拜武青嬰之父爲師,他人既英俊,性子又溫柔和順,是以朱九眞和武青嬰芳心可可,暗中都愛上了他。

朱武二女年齡相若,人均美艷,春蘭秋菊,各擅勝場,家傳的武學又是不相上下,兩三年前就被崑崙一帶的武林中人合稱爲「雪嶺雙姝」。她二人暗中早就較上了勁,偏生衛璧覺得熊掌與魚,難以取捨,因此只要三個人走上了一起,面子上客客氣氣,但二女唇槍舌劍,却誰也不肯讓誰,只是武青嬰較爲含蓄不露,反正她和衛璧同門學藝,日夕相見,比之朱九眞要多份便宜。

三個人突然聽到這個小僮児口中吐出「大江東去帖」五字,都是一愕,其實衛璧和武青嬰文武雙全,何嘗没瞧出這是「大江東去帖」,只是藏在心中不説而已。

這時見張無忌不過十四五歳年紀,相貌也無特異之處,居然説得出「大江東去帖」,三人心中先是均感奇怪,但衛璧和武青嬰一怔之下,登時明白︰「想來是在練功場中侍候老爺小姐的小厮,老爺傳授功夫之時,當然説過這路筆法的名字。」朱九眞却知父親傳功時機密之極,絶無第三人聽到,難道這小厮暗中窺探,偸學本門武功?這却非嚴加査究不可,當即喝道︰「你叫什麼名字?怎地知道這是『大江東去帖』?」張無忌聽得小姐又來問自己姓名,心中一酸︰「我早就跟你説了,原來你絲毫没放在心上。」説道︰「我叫張無忌。小人隨口瞎説,不知道對不對。」

朱九眞哦了一聲,道︰「你便是給衆將軍咬傷的那個小孩?」想起他曾一掌打碎「左將軍」的頭蓋骨,頗有武功根底,更起疑心︰「莫非他是我爹爹的仇人派來臥底的?否則我爹爹這門得意功夫的名字,他小小一個孩子怎能知道?」説道︰「啊,我想起來啦。」待要詳加査問,一瞥眼間,見衛璧和武青嬰並肩坐在一旁,低聲細語,不知説些什麼,心中妬意又生,不再理會無忌,大聲道︰「表妹,我和表哥都獻過醜啦,現下請你露一手絶藝給咱們瞧瞧。」武青嬰和衛璧款款深談,也不知有意還是無意,竟没理她。

朱九眞大怒,冷笑道︰「我這路筆法雖然平常,看來武家的武學却還擋不住。」武青嬰抬起頭來,冷冷的道︰「我師哥知道你要強好勝,存心讓你,虧你還得意呢。」朱九眞道︰「誰要他讓我?你問問他,他能不能拆解我這招『雙闕歸元』?」武青嬰道︰「你道咱們都是傻子,瞧不出這是蘇東坡的大江東去帖麼?我師兄倘若當眞不知,爲什麼這麼巧,遲不遲,早不早的,剛好等你使到一句『一尊還酬江月』的『月』字訣上,這纔罷手認輸?」朱九眞一呆,心想自己左筆掠,右筆直而鉤,再加一招「雙闕歸元」,正是最後一字的「月」字訣,原來他師兄妹早就知道了,那不是將自己當作傻子來耍弄麼?到了我背後,不知要如何的恥笑編排我了?想到這裡,更是老羞成怒,大聲道︰「就算識得,未必便能拆解?就算表哥存心讓我,青妹總不會讓吧?單是嘴上説説,哼!你瞧,連我家裡的小厮也會説,有什麼希奇?」

武青嬰站起身來,鐵青著臉,道︰「表哥,我回家去啦!人家把我比作低三下四的小厮,何苦賴在這裡受人家羞辱?」衛璧陪笑道︰「師妹,你别當眞,表妹跟你説笑呢。這泥腿小厮是什麼東西,這種人你府上要多少有多少,理他作什麼?」張無忌聽他言語中對自己如此輕賤,他脾氣再好,也是不禁有氣,却聽朱九眞道︰「好啊,你瞧不起我的泥腿小厮,青妹,你在三招之内,未必便打得倒他。」武青嬰道︰「哼,這樣的人也配我出手麼?眞姊,你不能這般瞧我不起。」

張無忌大聲道︰「武姑娘,我也是父母所生,難道不是人麼?你又是什麼高貴人物了?」武青嬰一眼也不瞧他,却向衛璧道︰「師哥,你讓我受這小厮的搶白,也不幫我。」衛璧見她楚楚的神態,心中早就軟了,而且在他心底,雖對雪嶺雙姝無分軒輊,可是知道師父武功深不可測,自己蒙他傳授的,最多不過十之一二,要學他絶世功夫,非討師妹的歡心示可,當下對朱九眞笑道︰「表妹,你這個小厮武功很不差嗎?讓我考考他成不成?」朱九眞明知他是在幫師妹,但轉念一想︰「這姓張的小子不知是甚麼來路,讓表哥迫出他的根底來也好。」便道︰「好啊,讓他領教一下武家的絶學,那是再妙也没有了,這人啊,連我也不知他到底是甚麼門派的弟子。」

衛璧奇道︰「這小厮學的,不是府上的武功麼?」朱九眞向張無忌道︰「你跟表少爺説,你師父是誰,是那一派的門下。」張無忌心想︰「你們這般輕視於我,我豈能説起父母的門派,羞辱太師父和死去的父母?何況我又没眞正練過武當派的功夫。」便道︰「我自幼父母雙亡,流落江湖,没學什麼武功,只有我義父指點過我一些。但他眼睛瞎了,也瞧不見我到底練得對不對。」朱九眞道︰「你義父叫什麼名字?是什麼門派的?」張無忌搖頭道︰「我不能説。」

衛璧長笑道︰「以咱們三人的眼光,還瞧他不出麼?」緩步走到場中,笑道︰「小子,你來接我三招試試。」説著轉頭向武青嬰使個眼色,意思是説︰「師妹莫惱,我狠狠打這小子一頓給你消氣。」豈知陥身在情網中的男女,對情人的一言一動、一顰一笑,無不留心在意,衛璧這一個眼色,盡教朱九眞瞧在眼裡。她見張無忌不肯下場,向他招招手,叫他過來,在他耳邉低聲道︰「我表哥武功很強,適纔你已見過了。你不用想勝他,只須擋得他三招,就算是給我面子。」説著在他肩頭輕輕拍了拍,意示鼓勵。

張無忌原知不是衛璧的敵手,若是一場跟他放對,徒然自取其辱,不過讓他門開心一場而已,但一站到了朱九眞的面前,已不禁意亂情迷,再聽她軟語叮囑,香澤微聞,那裡還有主意?心中只想︰「小姐命我給她掙面子,我豈能讓她失望。」迷迷惘惘的走到衛璧面前,獃獃呆呆的站著。衛璧笑道︰「小子,接招!」拍拍兩聲,打了他兩個耳光。這兩掌來得好快,無忌待要伸手擋架,臉上早已挨打,雙頰上都起了紅紅的指印。衛璧既知他並非朱家傳授的武功,不怕削了朱九眞和舅父的面子,下手便不容情,但這兩掌也没眞使上内力。否則早將他打得齒落頰碎,昏暈過去。

朱九眞叫道︰「無忌,還招啊!」張無忌聽得小姐的叫聲,精神一{\upstsl{挀}},呼的一拳打了出去。衛璧側身避開,讚道︰「好小子,還有兩下子!」一閃身躍到他的背後,張無忌急忙轉身,那知衛璧手出如電,已抓住了他的後領,提臂將他高高舉起,笑道︰「跌個狗吃屎!」用力往地上一摔。張無忌跟謝遜和父親學過幾年功夫,但一來時間甚短,二來當時年紀太小,三來謝遜只叫他記憶口訣和招數,不求實戰對拆,遇上了衛璧這等出自名門的弟子,竟是縛手縛脚,一點也施展不開。被他這麼一摔,想要伸出手足撐持,已然不及,砰的一響,額頭和鼻子重重撞在地下,鮮血長流。

武青嬰拍手叫好,格格嬌笑,説道︰「眞姊,我武家的功夫還成麼?」朱九眞又羞又惱,若説武家的功夫不好,不免得罪衛璧,説他好吧,却又氣不過武青嬰,只有寒著臉不作聲。張無忌爬了起來,戰兢兢的向朱九眞望了一眼,見她秀眉緊蹙,心道︰「我便是性命不在,也要給小姐掙這面子。」只聽衛璧笑道︰「表妹,這小子連三脚貓的功夫也不會,説什麼門派?」張無忌突然衝上,一脚往他小腹上踢去。衛璧笑著叫聲︰「啊喲!」身子向後微仰,避開了他這一脚,跟著左手倏地伸出,抓住他踢出後尚未收回的右脚,往外一摔。這一下只用了三成力,但無忌還是如箭離弦,平平往牆上撞去。他危急中身子用力一躍,這纔背脊先撞上牆,雖免頭破骨裂之禍,但背上已痛得宛如每根骨頭都要斷裂,如爛泥一團般堆在牆邉,再也爬不起來。

他身上雖痛,心中却仍是牽掛著朱九眞的臉色,迷糊中只聽她説道︰「咱們到花園中玩去吧!」話意中顯是氣惱之極。張無忌也不知從那裡來的一股力氣,翻身躍起,一縱上前,一掌便向衛璧打去。

張無忌這一掌,竟是使上了「降龍十八掌」中一招「亢龍有悔」。這降龍十八掌,在普天下掌法中威力第一,當年洪七公和郭靖恃此而傲視群雄,那是何等厲害?只可惜謝遜學到的已是破碎不全,而張無忌再學到的,更是這破碎不全掌法的一些皮毛,這時使將出來,連原來掌法的一成威力也及不到。饒是如此,這一掌擊出,仍是風聲虎虎。衛璧忙揮掌相迎,拍的一響,他竟是身子一晃,退了一步,武青嬰更是「咦」的一聲,大爲詫異。

原來她的祖上武修文雖拜郭靖爲師,但限於資質,這路降龍十八掌並未練成,傳到武青嬰之父武烈的手上,那降龍十八掌的招式仍是全然知曉的,其中威力却仍然一點也發揮不出。武青嬰常見父親在密室之中,比劃招式,苦苦思索,十餘年來從不間斷,但始終無甚收穫。須知自武修文至武青嬰,一百多年來已傳了五代,每一代都在潛心鑽研這套掌法的訣竅,可是百餘年無數心曲,盡付流水。這倒不是武家這些子孫魯鈍愚笨,實在降龍十八掌的精要能否把握,和聰明智慧無関,説不定越是聰明之人,越是練不成。只看黃蓉聰明而郭靖魯鈍,反而郭靖練成而黃蓉始終學不會,便知其理。郭靖並非祕技自珍之人,但楊過、耶律齊、郭芙、郭襄、郭破虜武氏兄弟諸小輩,無一能得其眞傳,降龍十八掌所以失傳,原因便在於此。

衛璧却不知張無忌這一掌的來歷,只是雙掌相交,但覺手臂酸麻,胸口氣血震盪,一斜身,揮拳往張無忌後心擊去。無忌手掌向後揮出,正是一招「神龍擺尾」。衛璧見他手掌來勢神妙無方,急向後閃時,肩頭已被他三根指頭掃中,雖不如何疼痛,但朱九眞和武青嬰都已看到,衛璧已是輸了一招。

在美人之前,這個台如何坍得起?衛璧初時和張無忌放對,眼看對方年紀既小,身份又賤,實是勝之不武,只不過拿他來耍弄耍弄,以博武青嬰一粲,因此拳脚下都只使二三成,這時連吃了兩次虧,大喝一聲︰「小鬼,你不怕死麼?」呼的一聲,一拳當胸打了過去,這招「長江三疊浪」中共含三道勁力,敵人如以全力擋住了第一道勁力,料不到第二道接踵而至,跟著第三道勁力又洶湧而來,若非武學高手,遇上了不死也得重傷。

這一招他是使出了全力,但他究非窮凶極惡之徒,只不過爲了挽回顏面,並不想眞的一拳便將表妹家中的僮児打死,是以將這招「長江三疊浪」中的第三道勁力扣住不發。張無忌見對方招數凌厲,左掌斜向下按,勁力似聚似散、如發如藏,乃是降龍十八掌中的一招「潛龍勿用」。這一招博大精深,奥妙無方,張無忌那能領會到其中的微旨?只是危急之際,順手便使出來。衛璧一掌打出,見他按掌相迎,姿式極是怪異,自己拳招中的,第一道勁力便如投入汪洋大海,登時無影無蹤,心中一驚之下,喀喇一響,那第二道勁力反彈過來,他右臂下臂已然震斷。幸好他一念之仁,第三道勁力扣住不發,否則張無忌不懂這招「潛龍勿用」的妙用,兩個人都要同時重傷在第三道勁力之下。

朱九眞和武青嬰齊齊驚呼,奔到衛璧身旁察看他的傷處。衛璧苦笑道︰「不妨,是我一時大意。」朱九眞和武青嬰心疼情郎受傷,兩人不約而同,揮掌向張無忌打去。無忌一掌震斷衛璧手臂,自己早是嚇得呆了,朱武二女雙掌打來,他避也不避,一中前胸,一中肩骨,登時吐出了一口鮮血。可是他心中的憤慨傷痛,尤在身體上的傷痛之上,暗想︰「我爲你拚命力戰,爲你掙面子,當眞勝了,你却又來打我!」衛璧叫道︰「兩位住手!」朱武二女依言停手,只見他提起左掌,鐵青著臉,一掌向張無忌打去。

張無忌身形急閃,避開了衛璧這一招。朱九眞叫道︰「表哥,你受了傷,何必跟這小厮一般見識?是我錯啦,不該要你跟他動手。」憑她平時心高氣傲的脾氣,要她向人低頭認錯,實是千難萬難,若不是眼見情郎臂骨折斷,惶急之際,決不能如此低聲下氣。豈知衛璧一聽,更是惱怒,冷笑道︰「表妹,你的小厮本領高強,你那裡錯了?只是我偏不服氣。」説著左臂橫推,將朱九眞推在一旁,跟著一拳便向張無忌打去。

張無忌要退後避讓,那知武青嬰雙掌向他背心輕輕一擋,使他無路可退,衛璧那一拳正中他的鼻梁,登時鼻血長流。原來武青嬰遠比朱九眞工於心計,她暗中相助衛璧,却不露相助的痕跡,要使衛璧臉上光采,心中感激。張無忌的武功本來遠遠不如衛璧,再加朱武二女一個明助,一個暗幫,頃刻之間,給三人拳打足踢,連中七八招,又吐了幾口鮮血。可是他骨氣甚硬,憤慨之下,仍是奮力招架,雖是以一敵三,但臨到拚命,將謝遜所授各種武功、父親教過的一些武當派拳法掌法,掃數使將出來,雖然功力不足,一拳一脚均無威力,但他所學的均是上乘家數,尤其「神龍擺尾」「亢龍有悔」「潛龍勿用」之三招,更是厲害,居然支持了一盞茶時分,仍是直立不倒。

朱九眞喝道︰「那裡來的臭小子,却到朱武連環莊來撒野,當眞是活得不耐煩了。」眼見衛璧舉起左掌,運勁劈落,當下左肩一撞,將張無忌的身上往他掌底推去。衛璧斷臂處越來越是疼痛,不願跟張無忌多所糾纏,是以這一掌劈下,已是用了十成力。無忌身不由主的向前一撞,但覺勁風撲面,自知抵擋不來,只有任他一掌劈死。驀地裡聽得一個威嚴的聲音喝道︰「且慢!」黃衫一晃,一個人在旁竄到,舉臂輕輕一格,擋開了衛璧這一掌。看他輕描淡冩的隨手一格,衛璧竟是立足不定,急退數步,眼見他身形後仰,便要坐倒在地,那身穿黃袍之人行動快極,早已縱到他的身旁,在他肩後一扶,衛璧這纔立定。朱九眞叫道︰「爹!」武青嬰叫道︰「朱伯父!」衛璧喘了口氣,纔道︰「舅舅!」原來這人正是朱九眞之父朱長齡。衛璧受傷斷臂,事情不小,狂{\upstsl{犺}}居的狗僕向前飛報,朱長齡匆匆趕來,見到三人已在圍攻張無忌。他站在旁邉看了一會,待見衛璧猛下殺手,這纔出手救了無忌一命。

朱長齡見無忌混身血汚,身子搖搖晃晃,但仍是咬牙站定,心中暗讚這小子極有骨氣,橫眼瞧著女児和衛武二人,滿臉怒火,突然間反手拍的一掌,打了女児一個耳光,大聲喝道︰「好,好!朱家的子孫越來越爭氣了。我生了這樣的乖女児,將來還有臉去見祖宗於地下麼?」

朱九眞自幼極得父母寵愛,連較重的呵責也没一句,今日在人前竟被老父重重打了一個耳光,一時眼前天旋地轉。不知所云,隔了一會,纔哇的一聲哭了出來。朱長齡喝道︰「住聲,不許哭!」聲音中充滿威嚴,聲音之響,只震得樑上灰塵簌簌而下。朱九眞心下害怕,當即住聲。

朱長齡道︰「我朱家世代相傳,以俠義自命,你高祖子柳公佐一燈大師,在大理國官居宰相,後來助守襄陽,名揚天下,那是何等的英雄?那知子孫不肖,到了我朱長齡手裡,竟會有這樣的女児,三個大人圍攻一個小孩,還想傷他性命。你説,羞也不羞,羞也不羞?」他雖是對著女児厲聲責備,但這些話衛璧和武青嬰聽在耳裡,句句猶如刀刺,不由得滿臉羞慚。張無忌見朱長齡一臉正氣,心下好生佩服,暗想︰「是非分明,那纔是眞正的俠義中人。」

只見朱長齡氣得面皮焦黃,全身發顫,不住呼呼喘氣。衛璧等三人眼望地下,不敢和他目光相對。

\chapter{眞假謝遜}

張無忌見朱九眞半邉粉臉腫起好高,顯見她父親這一掌打得著實不輕,但見她又羞又怕的可憐神態,想哭却不敢哭,只是用牙齒咬著下唇,便道︰「老爺,這不関小姐的事。」他話一出口,不禁嚇了一跳,原來自己説話嘶啞,幾不成聲,那是咽喉處受了衛璧的重擊之故。

朱長齡道︰「小兄弟會使『降龍十八掌』的功夫,想必是丐幫子弟了?」張無忌不願吐露自己身份門派,聽他當自己是丐幫子弟,便含含糊糊的答應。朱長齡又呵責女児道︰「這路掌法由丐幫幫主九指神丐洪七公傳下來,他老人家當年威震大江南北,和咱們朱武兩家都有極深的淵源。」轉頭向武青嬰道︰「郭靖郭大俠是你祖上修文公的師父,你既識得『降龍十八掌』,怎麼還可動手?」他一頓疾言厲色的斥責,竟對衛璧和武青嬰也是絲毫不留情面,張無忌聽著,反覺惶悚不安。

朱長齡又問起張無忌何以來莊中,怎地身穿僮僕衣衫,一面問,一面叫人取了傷藥和接骨膏來給無忌及衛璧治傷。朱九眞明知父親定要著惱,但又不敢隱瞞,只得將無忌如何收藏小猴、如何由群犬咬傷自己、如何救他來莊的情由説了。朱長齡越聽眉頭越皺,聽女児述説完畢,突然厲聲喝道︰「好啊,這位張兄弟是丐幫中的好朋友,你居然拿他當作厮僕,日後傳揚開去,江湖上好漢人人要説我『乾坤一筆』朱長齡是個無義之徒。你養這些惡狗,我只當你爲了玩児,那也罷了,那知大膽妄爲,竟然縱犬傷人?我今日不打死你這丫頭,我朱長齡還有顏面廁身於武林麼?」朱九眞見父親動了眞怒,雙膝一屈,跪在地下,説道︰「爹爹,孩児再也不敢了。」朱長齡兀自狂怒不休,衛璧和武青嬰一齊跪下求懇。張無忌道︰「老爺\dash{}」朱長齡忙道︰「小兄弟,你怎可叫我老爺?我疾長你幾歳,最多稱我一聲前輩,也就是了。」

張無忌道︰「是,是,朱前輩。這件事須怪不得小姐,她確是不知。」朱長齡道︰「你瞧,人家小小年紀,這等胸襟懷抱,你們三個那裡及得上人家?大年初一,武姑娘又是客人,我原不該生氣,可是這件事實在太不應該,那是黑道中卑鄙小人們的行逕,豈是我輩俠義中人的所作所爲?既是小兄弟代爲説情,你們都起來吧。」衛璧等三人含羞帶愧,站了起來。

朱長齡向餵養群犬的狗僕喝道︰「那些惡犬呢?都放出來。」三名狗僕答應了,將群犬放出。朱九眞見父親臉色不善,不知他有何舉動,低聲叫道︰「爹。」朱長齡冷笑道︰「你養了這些惡犬,縱犬傷人,好啊,你叫惡犬來咬我啊。」朱九眞哭道︰「爹,女児知錯了。」朱長齡哼了一聲,走入惡犬群中,雙掌揮動,拍拍拍拍四聲響過,四條巨狼般的惡犬已狗骨碎裂,屍橫就地。旁人嚇得呆了,都説不出話來。朱長齡拳打足踢、掌劈指戳,但見他身形飄動,一陣黃影在這狗場上繞了一圏,三十餘條猛犬已全被擊斃,别説噬咬抗擊,連逃竄幾步也來不及。衛璧和武青嬰相顧駭然,心想︰「雖知他武功極高,但從未見他出過手,想不到竟是這般厲害。不知何年何月,咱們才能練到這般地步。」朱長齡盡斃群犬,將無忌橫抱在臂彎之中,送到自己的房中養傷。不久朱夫人和朱九眞一齊過來照料湯藥。張無忌被群犬咬傷後失血過多,身子本已衰弱,這一次受傷不輕,又昏迷了數日,稍待清醒,便自己開了張療傷調養的藥方,命人煮藥服食,這纔好得快了。朱九眞常自伴在床邉,跟他猜謎説笑,持笛和歌,像大姊服侍生病的弟弟一般,細心體貼,無微不至。

張無忌傷愈起床後,朱九眞每日仍有大半天和他在一起。朱家的規矩,上午學武,下午練字,蓋朱家家傳武學,主要係脱胎於書法,書法愈精,武功跟著愈高。朱九眞的小書房窗明几淨,東壁懸著一幅杜牧書的「張好好詩」,北壁上兩張山水條幅之間,懸著懷素如和尚的「食魚帖」。朱九眞每日練字,給張無忌一副紙筆,也要他臨池學書,兩人相對而坐,但聞筆尖在宣紙上劃過時的沙沙微聲,有時冩得倦了,抬起頭來相對一笑,此時之樂,實是雖宣難言。朱九眞跟父親學武之時,居然對張無忌也不避忌,常常叫他在一旁觀看。空閒時拆解招數,也要張無忌作爲對手。朱家的武功雖和張無忌大不相同,但攻守搏擊之道,天下武學都是一例,朱長齡和朱九眞毫不藏私的向他指點。張無忌自從離冰火島來到中土後,一直顚沛離、流憂傷困苦,那裡有過這等安樂快活的日子?

轉眼到了二月中旬,這日,無忌正和朱九眞在房中冩字,丫鬟小鳳進來稟報︰「小姐,姚二爺從中原回來了。」朱九眞大喜,擲筆叫道︰「好啊,我等了他大半年啦,到這時才來。」拉著無忌的手,説道︰「無忌弟,咱們瞧瞧去,不知姚二叔有没替我買齊了東西。」兩人並肩走向大廳,無忌問道︰「姚二叔是誰?」朱九眞道︰「他是我爹爹的結義兄弟,叫做千里追風姚清泉。去年我爹爹託他到中原去送禮,我請他到杭州買胭脂水粉、到蘇州買繡花的針線和圖樣,又要買湖筆徽墨、碑帖書籍,不知他買齊没有。」要知這朱家莊僻處西域的崑崙山中,大姑娘家所用的精緻物事,千里之内都無買處,和中土相隔萬里之遙,來回一次,動輒便是兩年三年,若是有人前赴中原,朱九眞自要託他購買大批用品了。

兩人走近廳門,只聽到一陣嗚咽哭泣之聲,不由得都吃了一驚,進廳一看,更是驚詫,只見朱長齡和一個身裁高瘦的中年漢子都跪在地下,相擁而泣。那漢子身穿白裝喪服,腰中繫了一根草繩。朱九眞走近身去,叫道︰「姚二叔!」朱長齡放聲大哭,叫道︰「眞児,眞児!咱們的大恩人張五爺,張\dash{}張五爺\dash{}他\dash{}他\dash{}已死了!」朱九眞驚道︰「那\dash{}那怎麼會?他\dash{}失蹤十年,不是已安然歸來麼?」那身穿喪服的漢子正是千里追風姚清泉,嗚咽著説道︰「咱們住得偏僻,訊息不靈,原來張恩人在四年多以前,便已和夫人一齊自刎身亡。我還没有上武當,在途中已聽到消息。上山後見到宋大俠和兪二俠,才知實情,唉\dash{}」

張無忌越聽越驚,到後來更無疑惑,他們所説的大恩人張五爺,自是自己的生父張翠山了,眼見朱長齡和姚清泉哭得悲傷,朱九眞也是泫然落泪,忍不住便要撲上前去,吐露自己身份,但轉念一想︰「我一直自充是丐幫子弟,這時説明眞相,只怕朱伯伯和眞姊反而不信,説我冒充求恩,反而被他們瞧得小了。」過不多時,只聽得内院哭聲大作,朱夫人扶著丫鬟走出廳來,連連向姚清泉追問。

姚清泉悲憤之下,也忘了向義嫂見禮,當即述説張翠山自刎身亡的經過。張無忌雖然強忍,不致號哭出聲,但泪珠却已滾滾而下,只是大廳上人人均在哭泣流泪,誰也没留心到他。朱長齡突然手起一掌,喀喇喇一聲響,將面前的一張八仙桌打塌了半邉,説道︰「二弟,你明明白白説給我聽,上武當山去逼死恩公恩嫂的,到底是那些人?」姚清泉道︰「我一得到訊息,本來早該回來急報大哥,但想須得査明何人的姓名要緊。原來上武當山逼死恩公的,自少林派三大神僧以下,人數著實不少,小弟暗中到處打聽,這纔耽擱了日子。」

當下姚清泉將少林、崆峒、崑崙、峨嵋各派,海沙、巨鯨、神拳、巫山等等幫會中,凡是曾上武當山去勒逼張翠山的,諸如空聞大師、何太沖、靜玄師太等的名字,都説了出來。朱長齡慨然道︰「二弟,這些人都是當今武林中數一數二的好手,咱們本來是一個也惹不起。可是張五爺對咱們恩重如山,咱們便是粉身碎骨,也得給他報這個仇。」姚清泉道︰「大哥説得是,咱哥児倆的性命,都是張五爺救的,反正已多活了這十多年,交還給張五爺,也就是了。小弟最感抱憾的,是没能見到張五爺的公子,否則也可轉達大哥之意,最好是能請他到這児來,大夥児盡其所有,好好的侍奉他一輩子。」

朱夫人當下絮絮詢問這位張公子的詳情,姚清泉説只知他受了重傷不知在何處醫治,似乎今年還只八九歳年紀,大槩張三丰張眞人要傳以絶世武功,將來可能出任武當派的掌門人。朱長齡夫婦跪下拜謝天地,慶幸張門有後。姚清泉道︰「大哥叫我帶去的千年人參王、天山雪蓮、玉獅鎭紙、烏金匕首等等這些物事,小弟都在武當山上,請宋大俠轉交給張公子。」朱長齡道︰「這樣最好,這樣最好。」

朱長齡向女児道︰「我家身受大恩,你可跟張兄弟説一説。」朱九眞擕著無忌的手,走到父親書房,指著牆上一幅大中堂給他看。那中堂右端題著七個字道︰「張公翠山恩德圖。」張無忌見到父親的名諱,已是泪眼糢糊,只見圖中所繪是一處曠野,一個少年英俊的武士,左手持銀鉤、右手揮鐵筆,正和五個兇悍的敵人惡鬥。張無忌知道這人便是自己父親了,雖然面貌並不肖似,但依稀可從他眉目之間,看到自己的影子。地下躺著兩人,一個是朱長齡,另一個便是姚清泉,還有兩人却已身首異處。左下角繪著一個青年婦人,滿眼懼色,正是朱夫人,她手中抱著一個女嬰,無忌凝目細看,但見那女嬰嘴角邉有一顆小黑痣,那自是朱九眞了。這幅中堂紙色已變淡黃,爲時至少已在十年以上。朱九眞指著圖畫,向無忌解釋。原來其時朱九眞初生不久,朱長齡爲了躱避強仇,擕眷西行,但途中還是給對頭追上了。兩名師弟爲敵人所殺,他和姚清泉也被打倒,敵人正要痛下毒手,適逢張翠山路過,行俠仗義,將敵人擊退,救了他一家的性命。依時日推算,那自是張翠山在赴冰火島之前所爲。

朱九眞説了這件事後,神色黯然,道︰「咱們住得隱僻,張恩公從海外歸來的訊息,直至去年方才得知。爹爹因爲立誓不再踏進中原一步,忙請姚二叔擕帶貴禮物,前赴武當,那知道\dash{}」説到這裡,一名書僮進來請她赴靈堂行禮。朱九眞匆匆回房,換了一套最素淨的衣衫,和無忌同到後堂。只見後堂已排列了一個靈位,素燭高燒,靈牌上冩著「恩公張大俠諱翠山之靈位」。朱長齡夫婦及姚清泉跪拜在地,哭泣甚哀。無忌跟著朱九眞一同跪拜。朱長齡撫著他頭,哽咽道︰「小兄弟,很好,很好。這位張大俠慷慨磊落,實是當世無雙的奇男子,你雖跟他並不相識,無親無故,但拜他一拜,也是應該的。」當此情境,張無忌更不能自認便是這位「張恩公」的児子,心想︰「那姚二叔傳聞有誤,説我不過八九歳年紀。此時我便明説,他們也一定不信。」忽聽姚清泉道︰「大哥,那位謝爺\dash{}」朱長齡咳嗽一聲,向他使個眼色,姚清泉登時會意,説道︰「那些謝儀該怎麼辦?要不要替恩公發喪?」朱長齡道︰「你瞧著辦吧!」無忌心想︰「我明明聽你説的是『謝爺』,怎地忽然改爲『謝儀』?謝爺,謝爺?難道説的是我義父麼?」

這一晩張無忌想起亡父亡母,以及獨個児在冰火島上苦渡餘生的義父,思潮起伏,那裡睡得安穩?次晨起身,聽得脚步細碎,鼻中聞到一陣幽香,見朱九眞端著洗臉水,走進房來。張無忌一驚,道︰「眞姊,怎\dash{}怎麼你給我\dash{}」朱九眞道︰「傭僕和丫鬟都走乾淨了,我服侍你一下又打什麼緊?」張無忌更是驚奇,問道︰「爲\dash{}爲甚麼都走了?」朱九眞道︰「是我爹爹昨晩叫他們走的,每個人都領了一筆銀子,各自回自己家去,因爲在這児危險不過。」她頓了一頓,道︰「你洗臉後,爹爹有話跟你説。」

張無忌胡亂洗臉,朱九眞拿了梳子,給他梳頭,然後兩人一同來到朱長齡的書房。這所大宅子中本有一百多名婢僕,這時突然冷清清的一個也不見了。朱長齡見二人進來,説道︰「張兄弟,我敬重你是位少年英雄,本想留你在舍下住個十年八載,可是眼下突起變故,迫得和你分手,張兄弟千萬莫怪。」説著托過一隻盤子,盤中放著十二錠黃金,十二錠白銀,又有一柄防身的短劍,説道︰「這是愚夫婦和小女的一點敬意,請張兄弟收下。老夫若能留得這條性命,日後當再相會\dash{}」説到這裡,喉頭塞住了,再也説不下去。

無忌閃身讓在一旁,昂然道︰「朱伯伯,小侄雖然年輕無用,却也不是貪生怕死之徒。府上眼前既有危難,小侄決不能自行趨避。縱使不能幫伯父和姊姊什麼忙,也當跟伯父和姊姊同生共死。」朱長齡勸之再三,無忌只是不聽。朱長齡嘆道︰「唉,小孩子家不知危險。我只有將眞相跟你説了,可是你先得立下一個重誓,決不向第二人洩露機密,也不得向我多問一句。」張無忌跪下地下,朗聲道︰「皇天在上,朱伯伯向我所説之事,若是我向旁人洩露,多口査問,教我亂刀分屍,身敗名裂。」

朱長齡扶他起來,探首到窗外一看,隨即飛身上屋,査明四下裡無旁人偸聽,這纔回進書房,在無忌耳旁低聲道︰「我跟你説的話,你只可記在心中,却不許問我,不得向我説一句話,以防隔牆有耳。」無忌點了點頭,朱長齡低聲道︰「昨日姚二弟來報張恩公的死訊時,還帶了一個人來,此人姓謝名遜,外號叫作金毛獅王\dash{}」

張無忌大吃一驚,身子發顫。朱長齡又道︰「這位謝大俠和張恩公有八拜之交,他和天下各家各派的豪強都結下了深仇,張恩公夫婦所以自刎,便是爲了不肯吐露義兄的所在。他不知如何回到中土,動手爲張恩公報仇雪恨,殺傷了許多仇人,只是好漢敵不過人多,終於身受重傷。姚二弟爲人機智,救了他逃到這裡,對頭們轉眼便要追到,對方人多勢衆,咱們萬萬抵敵不住。我是捨命報恩,決意爲謝大俠而死,可是你跟他並無半點淵源,何必將一條性命陪在這児?張兄弟,我言盡於此,你快快去吧!敵人一到,玉石倶焚,再遲可來不及了。」

張無忌只聽得心頭火熱,又驚又喜,萬想不到義父竟會到了此處,問道︰「他在那\dash{}」朱長齡右手迭出,按住了他嘴巴,在他耳邉低聲道︰「不許説話。敵人神通廣大,一句話不小心,便危及謝大俠性命。你忘了適纔的重誓麼?」張無忌點了點頭。朱長齡道︰「我已跟你説得明白,張兄弟,我當你是好朋友,跟你推心置腹,絶無瞞隱。你即速動身爲要。」張無忌道︰「你跟我説明白後,我更加不走了。」

朱長齡長嘆一聲,説道︰「事不宜遲,須得動手了。」當下和朱九眞及無忌奔出大門,見朱夫人和姚清泉已候在門外,身旁放著幾個包袱,似要遠行。無忌東瞧西望,却不見義父的影蹤。朱長齡晃著火摺,點燃了一個火把。便往大門上點去。頃刻間火光衝天而起,火頭延向四處,原來這座大莊院的數百間房屋上,早已澆遍了火油。

西域天山崑崙一帶,自古盛産火油,常見油如湧泉從地底噴出,取之即可生火煮食。朱家莊廣厦華宅,連綿里許,但在火油助燃之下,焚燒極是迅速。張無忌眼見彫樑畫棟,頃刻間化爲灰燼,心下好生感激︰「朱伯伯畢生積儲,無數心血,盡爲焦土,那完全是爲了我爹爹和義父,這等血性男子,世間少有。」

當晩朱長齡夫婦、朱九眞、張無忌四人在一個山洞中宿歇,朱長齡的五名親信弟子手執兵刃,由姚清泉率領,在洞外戒備。這場大火直燒到第三日上方熄,幸而敵人尚未趕到。第三日晩間,朱長齡帶同妻女弟子,和姚清泉張無忌從山洞深處走去,經過黑越越的一條長隧道,來到幾間地下石室之中。這幾間石室中糧食清水等物,儲備充分,只是頗爲悶熱。朱九眞見無忌不住伸袖拭汗,笑道︰「無忌弟,你知不知道,爲什麼這裡如此炎熱?你可知咱們是在什麼地方?」無忌鼻中聞到一陣焦臭,登時省悟︰「啊,咱們便是在原來的莊院之下。」朱九眞道︰「你眞聰明。」

無忌對朱長齡用心的周密,更是佩服。敵人大舉來襲之時,眼見朱家莊已燒得片瓦不存,只有向遠處追索,決不會猜到謝遜竟是躱在火場之下。他見石室彼端有一處鐵門緊緊閉住,料想義父便藏在其中,心中雖是亟盼和義父相見,一敘别來之情,但想眼前步步危機,連朱長齡都不敢去和謝遜説話,自己怎能隨便,倘若誤了大事,自己送命不打緊,累了義父和朱家全家的性命,那是多大的罪過?

在地窖中住了半日,各人展開毛氈,正要安睡,忽然聽得一陣急速的馬蹄聲,遠遠傳來,不多時便到了頭頂。只聽得一人粗聲説道︰「朱長齡這老賊定是護了謝遜逃走啦,快追,快追!」各人雖在地底,上面的聲音却聽得清清楚楚,原來在地窖中有鐵管通向地面,傳下聲音。但聽得馬蹄雜沓,漸漸遠去。

這一晩從地窖經過的追兵,先後共有五批,有崑崙派的巨鯨幫的,其中兩批人却聽不出來歷,每一批少則七八人,多則十餘人,兵刃鏘鏘,健馬嘶吼,無不口出惡言,聲勢洶洶。無忌心想︰「我義父若非雙目失明,又受重傷,那將你們這些妖魔小醜放在心上。」待第五批人走遠,姚清泉拿起木塞,塞住鐵管之口,如此地窖中各人的説話,不致爲上面偶然經過之人聽見。但他話聲仍是壓低,輕聲道︰「我去瞧瞧謝大俠的傷勢。」朱長齡點了點頭。姚清泉伸手扳動鐵門的機括,鐵門緩緩開了。他左手提著一盞火油燈,走進鐵門。這時張無忌再也忍耐不住,站起身來,在姚清泉背後張望,只見一個身材高大的漢子向裡面而臥。張無忌乍見義父,熱泪盈眶,只聽姚清泉低聲道︰「謝大俠,覺得好些了麼?要不要喝水?」

突然間勁風響處,姚清泉手中的火油應風而滅,跟著砰的一聲,姚清泉被謝遜一掌擊出,飛出鐵門,重重摔在地下。只聽謝遜大聲叫道︰「少林派的,崑崙派的,崆峒派的衆狗賊,來啊,來啊,我金毛獅王謝遜豈能畏懼於你?」朱長齡叫道︰「不好,謝大俠神智迷糊。」走到門邉,説道︰「謝大俠,咱們是你朋友,並非仇敵。」謝遜哈哈笑道︰「什麼朋友?花言巧語,騙得倒我麼?」大踏步走出鐵門,一掌向朱長齡當胸擊來,這一掌勁力充沛,帶得室中那盞油燈火燄不住晃動。

朱長齡不敢擋架,轉身閃避,謝遜左手一拳便向朱夫人打去。朱夫人不會武功,眼見這一拳便要了她的性命,朱長齡和朱九眞迫不得已,雙雙舉臂架開他這一拳。張無忌見到這突如其來的變故,不禁嚇得呆了。

那謝遜雙掌如風,凌厲無比,朱長齡不敢與抗,只是退避。謝遜一掌擊不中朱長齡,掃在石牆之上,但見石屑紛飛,足見他掌力驚人,若是中在人體,當眞不死也得重傷。那謝遜長髮披肩,雙目如電,臉上血汚斑斑,口中荷荷而呼,掌勢越來越是猛烈。朱夫人和朱九眞嚇得躱在壁角,朱長齡見他拳掌攻到,只得將身邉的木桌推過去一擋。謝遜砰砰兩拳,登時將桌子打得粉碎。張無忌茫然失措,張大了口,呆立在一旁。眼看這個「謝遜」,根本不是他的義父金毛獅王謝遜。他義父雙眼早盲,這人却目光炯炯,極具威猛。只是這大漢呼的一掌打過去,朱長齡背靠石壁,已是退無可退,但並不出掌招架,叫道︰「謝大俠,我不是你敵人,我不還手。」那大漢毫不理會,一掌打在他的胸口。朱長齡神色極是痛苦,叫道︰「謝大俠,你相信了麼?」那大漢喝道︰「狗賊,再吃我一拳!」又是一拳打去。朱長齡噴出一口鮮血,顫聲道︰「你是我恩公義兄,便打死我,我也不還手。」那大漢狂笑道︰「不還手最好,我便打死你。」左一拳,右一掌,齊中胸腹。朱長齡「啊」的一聲慘呼,身子軟倒。

那大漢更不容情,又是一拳打去。張無忌搶上一步,拚命擋了他一拳,便覺這一拳勁力好大,一震之下,幾乎氣也透不過來,當下不顧生死,叫道︰「你不是謝遜,你不是\dash{}」那大漢怒道︰「你這小鬼知道什麼?」一脚向他踢去。無忌閃身避開,叫道︰「你冒充謝遜,不懷好意,假的,假的\dash{}」

朱長齡本已委頓在地,聽了無忌的叫聲,慢慢掙扎爬起,指著那大漢︰「你\dash{}你不是\dash{}你騙我\dash{}」突然一大口鮮血噴出,射在那大漢臉上,身子向前一跌,順勢便伸指點了他右乳下的「神封穴」。要知朱長齡重傷之後,已非那大漢的敵手,却藉著噴血傾跌,出其不意,以家傳的「一陽指」手法,點中了他大穴。「一陽指」點穴功夫天下無雙,那大漢武功雖強,竟也受制,動彈不得。朱長齡又在他腰脅間補上兩指,自己却支持不住,暈倒在地。朱九眞和張無忌急忙上前扶起。

過了一會,朱長齡悠悠醒轉,問無忌道︰「他\dash{}他\dash{}」張無忌道︰「朱伯伯,我再也不能隱瞞,你所説的恩公,便是家父。金毛獅王是我義父,我怎會認錯?」朱長齡搖了搖頭,不能相信。張無忌道︰「我義父雙眼已盲,這人眼目完好,便是最大的破綻。我義父是在冰火島上失明,此事外間無知曉,這人前來冒充,却不知我義父盲目這會事。」朱九眞拉住他手,道︰「無忌弟,你當眞是咱家大恩公的孩子?這可太好了,太好了。」朱長齡兀自不信,無忌只得將如何來到崑崙的情由,簡略説了。姚清泉旁敲側擊,問他武當山上和種情形,又詢問張翠山夫婦當日自刎的經過,聽他講得半點不錯,這纔相信。朱長齡仍感爲難,説道︰「倘若這孩子説的是謊話,咱們得罪了謝大俠,那可如何是好?」姚清泉拔出匕首,對著那大漢的右眼,説道︰「朋友,金毛獅王謝遜雙目已毀,你既要學他,便須學得到家些,今日先毀了你這對招子。我姓姚的上了你的大當,若不是這位小兄弟識破,豈非不明不白的送了我朱大哥的性命?」説著匕首向前一送,刃尖直抵他的眼皮。那大漢哈哈大笑,説道︰「有種的便一刀將我殺了。你當我開碑手胡豹是什麼人?能受你逼供的麼?」朱長齡「哦」的一聲,道︰「開碑手胡豹!{\upstsl{嗯}},你是崆峒派的。」胡豹大聲道︰「不錯,天下各門各派,都知你朱長齡要爲張翠山報仇。常言道得好︰先下手爲強,後下手遭殃。」

姚清泉喝道︰「你這人恁地惡毒!」匕首一抵,便往他心口刺去。朱長齡左手探出,一把抓住他的手腕,説道︰「二弟,且慢,倘若他眞的是謝大俠,咱哥児倆可是萬死莫贖。」姚清泉道︰「這位小兄弟已説得明明白白,大哥你若三心兩意,決斷不下,眼前大禍,可就難以避過。」朱長齡搖頭道︰「咱們寧可自己身受千刀,決不能錯傷了張恩公的義兄一根毫毛。」張無忌道︰「朱伯伯,這人決不是我義父。我義父外號叫作『金毛獅王』,頭髮是黃的,這人却是黑頭髮。」

朱長齡沉吟半晌,點了點頭,擕著他手,道︰「小兄弟,你跟我來。」兩人走出石室,再出了石洞,直到山坡後的一座懸崖之下。朱長齡和無忌並肩在一塊大石上坐下,説道︰「小兄弟,這人倘若不是謝大俠,咱們非殺了他不可,但在動手之前,我須得心下確無半點懷疑。你説不是不是?」張無忌道︰「這是你尊敬我爹爹和義父,唯恐有甚失閃,原是應當的。但這人絶非我義父,朱伯伯,你放心好了。」朱長齡輕輕嘆了口氣,道︰「孩子,我年輕之時,曾上過不少人的當。今日我所以不肯還手,以致身受重傷,還是識錯了人之故。一錯不能再錯,此事関係重大,我死不足惜,却無論如何,須得維護你和謝大俠的平安。我本該問個明白謝大俠到底身在何處,方能眞正放心,可是這件事我却又不便啓口。」張無忌心下激動,道︰「朱伯伯,你爲了我爹爹和義父把百萬家産都焚毀了,自己又受了這等重傷,難道我還有信你不過的。我義父的情形,你便是不問,我也要跟你説。」於是將父母和謝遜如何飄流到冰火島上、如何一住十年、如何三人結義回來的種種情由,一一對朱長齡説了。當然其中一大半經過,是他轉從父母口中得知,但也説得十分生動明白。

朱長齡一生飽經憂患,處事甚爲愼重,聽得無忌所言確無半點破綻,纔長長的舒了口氣,仰天説道︰「恩公啊恩公,你在天之靈,祈請明鑒,我朱長齡今日還不能死,定當竭盡所能,撫養無忌兄弟長大成人。只是強敵環伺,我朱長齡武藝低微,萬望恩公時加佑護。」説罷跪倒在地,向天叩頭。無忌又是傷心,又是感激,跟著跪下。

朱長齡站起身來,説道︰「現下我心中已無半分疑惑。唉!崑崙崆峒,少林峨嵋,那一派不是人多勢衆?小兄弟,先前我是決意拚了這條老命,殺得一個仇人是一個,以報令尊的大恩,但今日撫孤事大,報仇尚在其次,只是大地茫茫,却何處是避秦的桃源?連我這等偏僻之極的處所,他們也都找上來了,那裡更有一塊樂土?」他頓了一頓,又道︰「謝大俠孤零零的獨處冰火島上,這幾年的日子,想來也甚淒慘。唉,這位大俠對恩公恩嫂如此高義,我但盼能見他一面,死亦甘心。」

張無忌聽他説到義父人在冰火島受苦,極是難過,心念一動,衝口説道︰「朱伯伯,咱們一起往冰火島去,好不好?我在島上的十年何等逍遙快活,待等一回到中土,所見所受,不是兇殺流血,便是耽驚受怕。」朱長齡道︰「小兄弟,你很想回到冰火島去,是不是?」無忌躊躇不答,暗忖自己已活不多久,何況去冰火島途中海程艱險!未必能至,不該累得朱長齡一家身冒奇危,須知大海無情,只要稍有不測,那便葬身於洪波巨濤之中。

朱長齡握住他雙手,瞧著他臉,説道︰「小兄弟,你我不是外人,務請坦誠相告,你是不是想回冰火島去?」話聲誠懇已極。張無忌此時心中,確是苦厭江湖上人心的險惡,亟盼在身死之前,能再見義父一面,如能死於義父懷抱之中,那麼一生再無他求,在朱長齡面前,他也無法作偽,隱瞞自己心事,於是緩緩的點了點頭。

\chapter{萬丈深谷}

朱長齡不再多言,擕著張無忌的手,回到石室,向姚清泉道︰「那是奸賊,確然無疑。」姚清泉點了點頭,手執匕首,走進密室,只聽得那開碑手胡豹長聲慘呼,已然了帳。姚清泉從密室中出來,関上了鐵門,但見他匕首上鮮血殷然,順手在鞋底拂拭。朱長齡道︰「這賊子來此臥底,咱們的蹤跡看來已經洩露,此地不可再居。」當下領著各人,從石洞中出來,行了二十餘里,轉過兩座山峰,進了一個山谷,一棵大槐樹旁築著四五間小屋。此時天將黎明,各人進了小屋後,無忌見屋中放的都是犁頭、鐮刀之類農具,但鍋灶糧食,一應倶全。看來朱長齡爲防強仇,在宅第之旁,安排了不少避難的所在。朱長齡重傷之下,臥床不起。朱夫人取出土布長衫,以及草鞋、包頭,給各人換上,霎時間,大富之家的夫人小姐,變成了農婦村女,雖然言談舉止不像,但只要不加近視,也不致露出馬脚。

在農舍中住了數日,朱長齡因有祖傳的雲南傷藥,服後痊癒很快,幸喜敵人也不再追來。張無忌閒中旁觀,見姚清泉每日出去打探消息,朱夫人却率領弟子,收拾行李包裹,顯有遠行計。他知朱長齡是爲了報恩避仇,決意舉家前往海外冰火島,心中極是歡喜。這一晩他睡在床上,想起日後到了冰火島,如能天幸不死,終生得和這位美若天人的朱九眞姊姊在島上厮守,不禁面紅耳熱,一顆心怦怦跳動,又想朱伯伯、姚二叔和義父見面之後,三人結成好友,在島上無憂無慮的嘯傲歳月,既不怕蒙古韃子殘殺欺壓,也不必耽心武林強仇明攻暗襲,爲人若斯,自也更無他求了。他想得喜歡,雖在黑暗之中,臉上也露著微笑,直到中夜,仍未睡著,正朦朧間,忽聽得板門輕輕被人推開,一個人影閃進房來。無忌微感詫異,鼻中已聞到一陣淡淡幽香,正是朱九眞日常用以薰衣的素馨花香。他心中一動,突然間滿臉通紅,説不出的害羞。只見朱九眞悄步走到床前,低聲道︰「無忌弟,你睡著了麼?」無忌不敢回答,緊緊閉住雙眼,假裝睡熟。過了一會,忽有幾根溫軟的手指摸到他眼皮上,要探知他是否眞的睡著。

無忌又驚又喜,又羞又怕,只盼朱九眞快快出房,須知他心中對朱九眞敬重無比,只求每日能瞧她一眼,便已心滿意足,稚弱的心靈之中,固然從無半分褻瀆的念頭,便是將來娶她爲妻的盼望,也是從未有過。這時見她半夜裡忽然走進自己房來,如何不令他手足無措?他忽然又想︰「眞姊難道有什麼要緊事情,須得半夜裡來跟我説麼?」便在此時,突覺胸口「膻中穴」上一麻,接著肩貞、神藏、曲池、環跳諸穴上都一一被點。這一下大出無忌意料之外,那想得到朱九眞深夜竟來點自己的穴道?不由得大是懊喪︰「啊,眞姊是來試探我睡著之後,是否警覺?明児她解了我穴道,再來嘲笑我一番。早知如此,她進房時我便該躍起身來,嚇她一跳,免得她明日説嘴。」

只見朱九眞點了他穴道後,輕輕推開窗子,飛身而去。張無忌心道︰「我快些解穴,跟在她身後,扮鬼嚇她,倒也好玩。」於是即以謝遜所授的獨門解穴之法,衝解穴道。不料朱九眞家傳的「一陽指」功夫厲害無比,無忌直用了大半個時辰,方始解開被點的諸穴,這也因一來朱九眞功力不彀,二來她不欲使無忌受到些微損傷,因而使力極輕,否則倘若是練就一陽指力的高手來點,無忌解穴之法再妙,却也衝不開。待得他站起身來,匆匆穿上衣服,躍出窗去時,空山寂寂,樹影深深,那裡還尋得著朱九眞的芳蹤?

張無忌站在黑暗之中,頗是沮喪,但忽而轉念︰「眞姊明児要笑我無用,讓她取笑便是,何必跟她爭強鬥勝?我平日想博她個歡喜,也是不易,今晩倘若追到了她,只怕她反而要著惱了。」想到此處,登時心安理得。這時已是初春,山谷間野花放出清香,良夜人靜,無忌一時也睡不著,信步便順著山谷中一條小溪走去。山坡上積雪初溶,雪水順著小溪流去,偶爾挾著一些細小的冰塊,相互撞擊,錚錚有聲。

無忌走了一會,忽聽得左首樹林中傳出格格一聲嬌笑,正是朱九眞的聲音。無忌吃了一驚,心道︰「眞姊瞧見了我麼?」却聽得朱九眞低聲叱道︰「表哥,不許胡鬧,瞧我不老耳括子打你。」跟著是幾聲男子的爽朗笑聲,不問可知便是衛璧。無忌心頭一震,幾乎要哭了出來,做了半天的美夢登時破滅,登時心中雪亮︰「眞姊點我穴道,那裡是跟我鬧著玩?她是半夜裡跟她表哥相會,怕我知道。」霎時間手酸脚軟,又想︰「我是個無家可歸的窮小子,文才武功,人品相貌,那一樣都遠遠不及衛相公。眞姊和他是表兄妹之親,跟他原是郎才女貌,天造地設的一對。」

自己寬解了一會,輕輕嘆了口氣,忽聽得脚步聲響,有人從後面走來,便在此時,朱九眞和衛璧也低聲笑語,手擕手的並肩而來。張無忌不願和他們碰面,忙閃身在一株大樹後一躱,但聽得兩邉脚步聲漸漸湊近,朱九眞叫道︰「爹!你\dash{}你\dash{}」聲音顫抖,似乎很是害怕,原來從另一邉來的那人正是朱長齡,他對女児深夜中和外甥私會一事,顯得大爲忿怒,鼻孔中哼了一聲,道︰「你們在這裡幹什麼?」朱九眞強作無所掛礙,笑道︰「爹,表哥跟我這麼久没見了,他今日難得來到這裡,咱們隨便談談。」朱長齡道︰「你這小妮子忒也大膽,若是給無忌知覺\dash{}」朱九眞忙接口道︰「我輕輕點了他五處大穴,這時他正睡得香甜呢。待會去解開穴道,管教他決不知覺。」

無忌心道︰「朱伯伯也瞧出我喜歡眞姊,爲了我爹爹有恩於他,不肯令我傷心失望。其實我雖喜歡眞姊,却是絶無他念。朱伯伯,你待我當眞是太好了。」只聽朱長齡道︰「雖是如此,一切還當小心,不可功虧一簣,被他瞧出破綻。」朱九眞笑道︰「孩児理會得。」衛璧道︰「眞妹,我也該回去了,只怕師父等我。」朱九眞對他甚是依戀,道︰「我送你去。」朱長齡道︰「好,我也去再跟你師父談一會,咱們此去北海冰火島,大家須得萬事齊備,不可稍有差失。」説著三人一齊向西。

無忌聽得頗爲奇怪,知道衛璧的師父叫做武烈,是武青嬰的父親,聽朱長齡的口氣,好像武家父女和衛璧都要到冰火島去,怎麼事先没聽他説起?這件事知道的人多了,難保不洩漏風聲,别要累及義父纔好。他藏身樹後,低頭沉思,突然間想到了朱長齡的一句話︰「别要被他瞧出了破綻。」破綻,破綻,有甚麼破綻?

想到「破綻」兩字,腦海中一個糢糊的疑團,驀地裡鮮明異常的顯現在眼前︰那幅「張翠山恩德圖」中,爲什麼人人相貌逼肖,却將他尖臉的父親畫作了方臉?他父親的眉目很像,不錯,那因爲他父子倆眉目相似,可是他父親是尖臉蛋,不像無忌自己,臉作長方。

聽朱長齡説︰這幅書是十餘年前他親筆所繪,就算他丹青之術不佳,也不該將大恩公畫得面目全非。畫上的張翠山,倒像是長大了的張無忌一般。「啊,另有節難解之處。爹爹所用鐵筆形似毛筆,筆管極短,但畫中爹爹所使兵刃,却是尋常的判官筆。朱伯伯自己是使判官筆的大行家,什麼都可畫錯,怎能將爹爹所用的判官筆也畫錯了?」

張無忌想到此節,心中隱隱感到恐懼,在他内心,已是有了一個答案,可是這答案實在太可怕,無論如何不敢明明白白的去想它,只是安慰自己︰「千萬别胡思亂想,朱伯伯如此待我,怎可瞎起疑心?我這就回去睡吧,要是讓他們知道我半夜中出來,説不定會有性命之憂。」他想到「性命之憂」四字,全身爲之一震,自己也解釋不來,爲什麼無端端的會這樣害怕。

他呆了半晌,不自禁向著朱長齡父女所去的方向逃去,只見樹林中透出一星光,原來這樹叢之中,另有房屋。張無忌心中怦怦亂跳,放輕脚步,朝著火光悄悄前行,走到屋後,定了定神,探頭從窗縫中向内一張,只見朱長齡父女和衛璧對窗而坐,在和人説話,有兩個人背向無忌,看不見面目,但其中一個少女顯是「雪嶺雙姝」之一的武青嬰,另外那男子身材高大,傾聽朱長齡述説如何假裝客商,到山東一帶出海,他一聲不響的聽著,不住點頭。無忌心想︰「我這人不是庸人自擾嗎?這一位多半是武莊主,朱伯伯既然跟他交好,邀他同去冰火島,本來也是人情之常,我又何必大驚小怪。」

只聽得武青嬰道︰「爹,咱們都去冰火島,要是茫茫大海之中,找不到那小島,回又回不來,那可怎生是好?」無忌心想︰「這位果然是她爹爹武莊主。」只聽他説道︰「你若是害怕,那就别去。天下之事,不經艱難困苦,那有安樂時光?」武青嬰嬌嗔道︰「我不過問一問,又引得你來教訓人家。」武烈哈哈的一笑,朗聲説道︰「這一下原是孤注一擲。要是運氣好,咱們到了冰火島上,想那謝遜武功再好,也只一人,何況雙目失明,自不是咱們的敵手\dash{}」無忌聽到此處,一道涼氣從背脊上直衝了下來,全身打戰,答答兩響,牙齒互擊出聲。他用力咬緊牙関,只聽武烈繼續説道︰「\dash{}那屠龍刀還不手到拿來?那時『號令天下,莫敢不從?』我和你朱伯伯並肩成爲武林至尊。倘若人算不如天算,我們終於死在大海之中,哼,世上有那個人是不死的?」

衛璧説道︰「聽説金毛獅王謝遜武功卓絶,王盤山島上一吼,將數十名江湖好手一齊震死,崑崙派的兩名弟子被震成了白痴。依弟子之見,咱們到得島上,不用跟他明槍交戰,只須在食物中偸下毒藥,别説他是盲人,便算他雙目完好,瞧得清清楚楚,也決不會疑心他義児會帶人來害他啊。」朱長齡點頭道︰「璧児此計甚妙。只是咱們朱武兩家,上代都是名門正派的俠士,向來不碰毒藥,便是暗器之上,也從不餵毒。到底要用什麼毒藥,使他服食時全不知覺,我可一竅不通了。」衛璧道︰「我爹爹多在中原行走,定然知曉,請他購買齊備便是。」

武烈站起身來,拍了朱九眞的肩頭,笑道︰「眞児\dash{}」這時他回過頭來,張無忌看得清清楚楚,不由得吃了一驚。原來此人正是假扮他義父的「開碑手胡豹」,什麼將朱長齡打得重傷吐血、被姚清泉一刀殺死等等,全是誑騙無忌的巧妙機関,爲了這戲要演得逼眞,一掌擊出,碰到牆上是石屑紛飛,遇到桌椅是堅木破碎,所以要武功精強的武烈親自出馬。只聽武烈對朱九眞笑道︰「所以啊,這場戲還有得唱呢,你一路得跟那小鬼假裝親熱,直至送了謝遜的性命爲止。可千萬别露出馬脚。」

朱九眞道︰「爹,你須得答應我一件事。」朱長齡道︰「什麼?」朱九眞道︰「你叫我侍候這小鬼,這些日子來,吃的苦頭可眞不小。要到那冰火島,時候還長著呢,不知道還要受多少罪。等你取到屠龍刀後,我可要這小鬼一刀殺死!」

張無忌聽朱九眞這麼惡狠狠的説話,眼前一黑,幾欲暈倒,隱隱約約聽得朱長齡道︰「咱們這般用巧計騙他,誘出金毛獅王的所在,説來已有些不該。這小子也不是壞人,咱們殺了謝遜,取得屠龍寶刀之後,將這小子雙目刺瞎,留在冰火島上,也就是了。」武烈讚道︰「朱大哥就是心地仁善,不失俠義家風。」朱長齡嘆道︰「咱們這一步棋子,實在也是情非得已。武二弟,咱們出海之後,你們座船遠遠跟在我們後面,倘若太近,會惹起那小子的疑心,過份遠了,又怕失了連絡,這梢公舟師,可得費神物色纔是。」武烈道︰「是,朱大哥想得甚是周到。」

張無忌心中一片混亂︰「我從没吐露自己身份,怎地會給他們瞧破?{\upstsl{嗯}},想是我全力和衛璧及朱武二女周旋之時,使出了武當心法和降龍十八掌中的功夫,朱伯伯見多識廣,登時便識破了我的來歷。」又道︰「他知道我爹爹媽媽寧可自刎,也不吐露義父的所在,若是用強,決不能逼迫我洩露眞相。於是假造圖畫、焚燒巨宅、再使苦肉計使我感動。他不須問我一句,却使我反求他帶往冰火島去。朱長齡啊朱長齡,你的奸計,可眞是毒辣之至了。」

這時朱長齡和武烈兀自在商量東行的各種籌劃。張無忌不敢再聽,凝住氣息,輕輕提脚,輕輕放下,每跨一步,要聽得屋中並無動靜,纔敢再跨第二步。他知朱長齡、武烈兩人武功強極,自己只要稍一不愼,踏斷半條枝枯枝,立時便會給他們驚覺。這三十幾步路,跨得其慢無比,直至離那小屋已在十餘丈外,方才走得稍快。他慌不擇路,只是向山坡上的林木深處走去,越攀越高,越走越快,到後來竟是發足狂奔,一個多時辰之中,不敢停下來喘一口氣。他奔逃了半夜,到得天色明亮,只見自己處身在一個雪嶺的叢林之内。他回頭眺望,要瞧瞧朱長齡等是否追來,這麼一望,不由得叫一聲苦,只見一望無際的雪地之中,留著長長的一行足印。原來西域苦寒,這時雖然已是春天,但山嶺間積雪未溶。他昨晩倉惶逃命,不敢在山谷和平地上逗留,竭力的攀登山嶺,那知反而洩露了自己行藏。

便在此時,隱隱聽得前面傳來一陣狼{\upstsl{嗥}}的聲音,極是悽厲可怖,張無忌站在一塊突出的懸崖之上,向前遙望,只見山谷中有七八條大灰狼仰起了頭,向著他張牙舞爪的{\upstsl{嗥}}叫。顯是群狼腹飢,想要食之裹腹,只是和他站立之處隔著一條望不見底的萬丈峽谷,無法過來。他回頭再看,心中突的一跳,只見山坡上有五個黑影,慢慢向上移動。此時相隔尚遠,似乎這五人走得不快,實則奔行如風,不用一個時辰,便能追到,那自是朱武兩家一行人了。張無忌定了定神,打定了主意︰「我寧可被餓狼分屍而食,也不能落入他們手中,苦受他們折磨。」想到自己對朱九眞如此痴心敬重,那知她美艷絶倫的面貌之下,竟是藏著這樣一副蛇蝎心腸,他又是慚愧,又是傷心,縱身便往密林中奔去。

樹林中長草齊腰,雖然也有積雪,足跡却不易看得清楚。他奔了一陣,體中寒毒突然發作,雙腿也已累得無法再動,便鑽在一叢長草之中,從地下拾起一塊尖角的石頭,拿在手裡,若是朱長齡等追到,發覺了自己藏身所在,那麼便用尖石撞擊太陽穴自殺。

他心意已決,靈台清明,回想這兩個多月來,寄身朱家莊的種種經過,越想越是難受,心道︰「少林寺的高僧害我,那也罷了。崆峒派、華山派、崑崙派這些人恩將仇報,我也不放在心上,可是我對眞姊這般一片誠心,到頭來才知内中眞相原來如此\dash{}唉,唉,媽媽臨死之時叮囑我什麼話?怎地我全然置之腦後?」

他的母親素素臨死時對他説的那幾句話,清晰異常的在他耳邉響了起來︰「孩児,你大了之後,要提防女人騙你,越是好看的女人,越會騙人。」張無忌熱泪盈眶,眼前一片糢糊,心道︰「媽媽跟我説這幾句話之時,那柄匕首已插在她胸口。她忍著劇痛,如此叮囑我,我却將她這幾句血泪之言全不放在心上。若不是我會衝解穴道之法,鬼使神差的聽到了朱長齡的陰謀,以他們佈置的周密,我非將他們帶到冰火島上,害了義父的性命不可。」他心中一靜,對朱長齡父女所作所爲的含意,登時瞧得明明白白。朱長齡一料到他是張翠山的之子,便出手擊斃群犬,掌擊女児,使得張無忌深信他是一位是非分明、仁義過人的俠士。至於將這些連綿數十里的華厦付之一炬,雖然有些可惜,但比之「武林至尊」的屠龍寶刀,却又是不値什麼了。

張無忌又想︰「我在島上之時,每天都見義父抱著那柄刀児呆呆出神,十年之中,始終參解不透刀中的祕密。可是這朱長齡機智過人,計謀之深,遠遠勝我義父。我義父想不出,寶刀若是到了朱長齡手中,他却多半能想得出\dash{}」這時猛聽得脚步聲響,朱長齡和武烈二人已找到了叢林之中。

武烈低聲道︰「那小子定是躱在林内,不會再逃往遠處\dash{}」朱長齡急忙打斷他的話題,説道︰「唉,不知眞児説錯了什麼話,得罪了這位小兄弟。我眞是擔心,他小小年紀,若是在這大雪遍野的山嶺中有甚失閃,我便是粉身碎骨,也對不起張恩公啊。」他這幾句話説得憂心如搗,自責甚深,張無忌聽在耳裡,不由得毛骨悚然,暗想︰「他心尚不死,還在想花言巧語的騙我。」只聽得朱、武二人各持木杖,在長草叢中拍打,張無忌全身蜷縮,一動也不敢動,幸而那林中佔地甚廣,要每一處都拍打到,却是無法辦到。不久衛璧和雪嶺雙姝也趕到了,五人在這叢林中搜索了半天,始終没找到張無忌,各人都感倦累,便石上坐下休息。其實五人所坐之處,和張無忌相隔不過兩丈,只是林密草長,將無忌的身子掩蔽得極是嚴密。

朱長齡凝思片刻,突然大聲喝道︰「眞児,你到底怎地得罪了無忌兄弟,害得他三更半夜的不告而别?」朱九眞一怔,朱長齡忙向她使個眼色。張無忌伏在草叢,却將這眼色瞧得清清楚楚。朱九眞會意,便大聲道︰「我跟他開玩笑,點了他的穴道,不知怎樣,這位小兄弟却當了眞。」説著提高嗓子,縱聲叫道︰「無忌弟弟,無忌弟弟,你快出來,眞姊跟你陪不是啦。」聲音雖響,却仍是嬌媚婉轉,充滿了誘惑之意。她叫了一會,見無動靜,忽然哭了起來,説道︰「爹,你别打我,别打我。我不是故意得罪無忌弟弟啊。」朱長齡大聲怒喝,朱九眞不住口的慘叫,似乎給父親打得痛不可當。張無忌眼見他父女倆做戲,可是聽著這聲音,仍是心下惻然,暗道︰「幸而我瞧見你們的神情,否則聽了她如此尖聲慘叫,明知於我不利,也要忍不住挺身而出。」

朱氏父女知道張無忌是藏身在這樹林之内,一個怒罵,一個哀喚,聲音越來越是淒厲,張無忌雙手掩耳,那聲音還是一陣陣傳入耳中。他再也忍耐不住,把心一橫,縱身躍出,叫道︰「你們搗什麼鬼,難道還騙倒我麼?」朱長齡等五人齊聲歡呼︰「在這裡了!」張無忌道︰「眞姊,你好!」穿林而北,發足狂奔。朱長齡和武烈便如兩頭大鳥般向他身後撲去。張無忌死志早決,更無猶疑,筆直向那萬丈峽谷奔去,可是朱長齡的輕功勝他甚遠,待他奔到峽谷邉上,朱長齡已追到他的身後,伸手往他背心抓去。

張無忌只覺背心奇痛徹骨,朱長齡右手的五根手指已緊緊抓住他背脊,就在此時,他足底踏空,半個身子已在深淵之上。他左足跟著跨出,全身向前一撲。朱長齡萬没料到他寧可投崖而死,也不願落入他的手裡,被他一帶,跟著向前傾出。以他數十年的武功修爲,若是立時放手反躍,自可保住性命,可是他知道只須五根手指一鬆,那「武林至尊」的屠龍刀,便永遠再無到手的機會,這兩個月來的苦心籌劃、成爲一片焦土的巨宅華厦,盡數隨著這五根手指的一鬆而付諸東流。

當眞是時遲那時快,他策一猶豫,張無忌下跌之勢却是決不稍緩,朱長齡叫道︰「不好!」反探左手來和自後馳到接應的武烈相握時,却是差了尺許。他抓著張無忌的右手兀自不肯放開,兩人一齊自峭壁跌落,直摔向足底的萬丈深淵,只聽得武烈和朱九眞等人的驚呼聲自頭頂傳來,一霎間便聽不到了,兩人衝開瀰漫谷中的雲霧直向下墜。

朱長齡心知這一摔下去,自必變成肉泥,但他一生之中經歷過不少風浪,臨危不亂,只覺身旁風聲虎虎,不住的向下摔落,却是仍未著地。這峽谷兩邉相距並不甚寬,偶爾見到峭壁上有樹枝伸出,朱長齡左手去抓,但幾次都是差數尺,没能抓到,最後一次是抓到了,可是他二人下跌的力道太強,那枝樹枝吃不住力,喀喇一聲,一根手臂粗的松枝登時折斷。但就是這麼緩得一緩,朱長齡身子已有借力之處,雙足一絞,使招「烏龍絞柱」,牢牢的抱住那株松樹,提起無忌,將他放在樹上,唯恐他仍要躍下尋死,抓住他手臂不放。

張無忌見始終没能逃出他的掌握,灰心沮喪已極,恨恨的道︰「朱伯伯,不論你如何折磨我,若要我帶你去找我義父,那是一萬個休想。」朱長齡翻轉身子,在樹枝上坐穩了,抬頭一望,上面的峭壁相距極遠極遠,朱九眞等人固然見不到,呼聲也已聽不到了,饒是他大膽厲害,想起適纔的死裡逃生,也自不禁心悸,額頭上冷汗涔涔而下。他定了定神,笑道︰「小兄弟,你説什麼?我一點児也不懂,别胡思亂想。」張無忌道︰「你的奸謀既被我識破,那是無用的了。便是逼著我帶去冰火島,我東南西北的亂指一通,大家一齊死在大海之中,你當我不敢麼?」朱長齡心想這話倒是實情,眼前不能跟他破臉,總要著落在女児身上,另圖妙策。當下氣凝丹田,縱聲叫道︰「咱們都好好児的,放心好啦!」

這一聲叫了上去,只震得山谷鳴響,「放心好啦\dash{}放心好啦\dash{}放心好啦\dash{}」朱長齡猛地裡想起︰「啊喲,不好!這雪山之中,可不能如此呼叫。」只見山壁上白雪滾滾而下,幸好這一帶積雪不厚,並未造成雪崩,但朱長齡却也不敢再叫,一瞧四下的情勢,向上攀援決不可能,脚下仍是深不見底,便算到了谷底,十九也無出路。唯一的法子是沿著山壁斜坡,慢慢爬行出去,於是向張無忌道︰「小兄弟,你千萬不可瞎起疑心,總而言之,我決計不會逼迫你去找謝大俠,若有此事,教我姓朱的萬箭攢身,死無葬身之地。」

他立這個誓,並非虛言,實則他明知便是逼迫,也決計無用,只有誘得他心甘情厚的帶去,纔有指望。張無忌聽他如此立誓,心下稍寬。朱長齡道︰「咱們從這裡慢慢爬出去,你不能再往下跳,知道麼?」張無忌道︰「你既不逼我,我何必自己尋死?」朱長齡點點頭,取出短刀,剝下樹皮,搓成了一條繩子,兩端分别縛在自己和無忌腰裡,兩人沿著雪山斜坡,手脚著地,一步步向有陽光處爬去。

至於這般爬將出去,到底是步出生天,還是陥入絶境,朱長齡却也無法逆料,眼前之計,也只有走得一步算一步。那峭壁本就極陡,加上凍結的冰雪,更是滑溜無比,張無忌兩度滑跌,都是朱長齡使力拉住,纔不跌入下面的深谷。無忌心中並不感激,暗想︰「你不過是想得屠龍寶刀,那裡是眞的好意救我了?」

兩人爬了半天,手肘膝蓋都已被堅冰割得鮮血淋漓,總算山坡已不如何陡峭,兩人站起身來,一步步的向前掙扎而行,好容易轉過了那堵屏風也似的大山石,朱長齡只叫得一聲苦,不知高低。原來眼前茫茫雲海,更無去路,却是置身在一個三面皆空,當眞是死路一條。這大平台上白皚皚的都是冰雪,既無樹木,更無野獸,那裡有可吃的東西?

張無忌反而高興,笑道︰「朱伯伯,你花盡心機,却到了這個半天吊的石台上來。這會児就有一把屠龍寶刀給你,你拿著它却又如何?」朱長齡叱道︰「你别胡説八道!」盤膝坐下,吃了兩口雪,運氣休息半晌,心想︰「此時雖然疲累,精力尚在,若在這裡再餓上一天,只怕再也難以脱困了。」於是站起身來,説道︰「這裡既是前路已斷,咱們回去向另一邉找找出路。」張無忌道︰「我却覺得這児很是好玩,又何必回去?」朱長齡怒道︰「這児什麼也没得吃的,呆在這児幹麼?」張無忌笑道︰「不食人間煙火更好,便於修仙練道啊。」朱長齡心下大怒,但知若是逼得緊了,説不定他便縱身往崖下一跳,便道︰「好,你在這児多休息一會,我找到了出路,再來接你。别太走近崖邉,小心摔了下去。」張無忌笑道︰「我生死存亡,何勞你如此掛懷,你這時候還在妄想我帶你到冰火島去,勸你别操這份心了吧。」

朱長齡不答,逕自從原路回去,到了那棵大松樹旁,向左首探路而前。這一邉的山壁地勢更加兇險,只是不須顧到張無忌,他行得反而更快,或爬或走的行了半個多時辰,來到一處懸崖之上。眼前再無去路。朱長齡臨崖浩嘆,怔怔的呆了良久,纔没精打采的回到平台。張無忌不用詢問,一看他的臉色,便知没找到出路,心想︰「我身中玄冥神掌之後,陰毒難除,屈指計來,原是壽元將盡,不論死在那裡,都是一樣,只是這朱伯伯好端端的有福不會享,貪心一起,竟陪著我在冰天雪地中活活餓死,可歎可憐!」

他初時憎恨朱長齡陰狠奸險,墜崖出險之後還取笑他幾句,這時眼見生路已絶,朱長齡垂頭喪氣,心下反而憐憫他起來,溫言説道︰「朱伯伯,你年紀已大,什麼榮華快活都享過了,此刻便是與世長逝,又有何憾?不用難過吧。」

朱長齡將張無忌一直容讓三分,只不過不肯死心,盼望最後終能騙動了他。帶領自己前往冰火島去,這時眼見生路已斷,心想所以陥入這個絶境,全是爲了這個小子,一口怨氣那裡消得下去?雙眼中如要噴出烈火,惡狠狠的瞪視著他。張無忌見這個向來面目慈祥的溫厚長者,陡然間如同變成了一頭野獸,不由得大是害怕,一聲驚叫,站起來便逃。朱長齡喝道︰「這児還有路逃麼?」伸手向他背後抓去,決意盡情將他折磨一番,使他死不死、活不活的受盡痛楚,這纔將他弄死。

張無忌向前滑出一步,但見左側山壁黑黝黝的似乎有個洞穴,更不思索,便鑽了進去,嗤的一聲,褲管被朱長齡的手爪撕去了一塊,大腿也已抓破。張無忌捨命向前爬行,同時反手一掌,拍出一招「神龍擺尾」。他與朱長齡武功相差懸殊,可是朱長齡對這一招「神龍擺尾」却也頗爲忌憚,不敢過於逼近,但仍是彎著腰,一步步的追來。

\chapter{金在油中}

張無忌跌跌撞撞的急鑽,突然間砰的一下,額頭和山石相碰,撞得眼前金星亂舞。他知道這時朱長齡已撕破了臉,什麼毒辣的兇狠的手段都會做得出,自己雖是死不足懼,可是他倘若不是一下子便下殺手,而是讓自己吃彀零碎苦頭,這罪可就大了,因此拚命的向洞裡鑽去。他也没盼望能逃離朱長齡的毒手,只是能和他隔得遠一步,就儘量的遠遠離開。幸而那洞穴越走越小,爬進十餘丈後,他已是僅能容身,朱長齡却再也擠不進去了。張無忌又爬進數丈,祇見前面透進光亮,心中大喜,手足兼施,加速前行,朱長齡又急又怒,叫道︰「小兄弟,我不來傷你,别走啊。」張無忌却那裡理他?朱長齡運起掌力,往石壁上擊去,豈知這山石堅硬無比,一掌打在石上,只震得自己掌心劇烈疼痛,石壁竟是紋絲不損。他摸出短刀,想掘鬆山石,將洞口挖得稍大,但只挖得幾下,拍的一聲,一柄青鋼短刀斷爲兩截,山石上只劃出淺淺的兩條白痕。朱長齡狂怒之下,勁運雙肩,向前一擠,身子果是前進了尺許。可是再想前行,却已是萬萬不能,堅硬勝鐵的石壁壓在他胸口背心,竟是氣也喘不過來。

朱長齡但覺窒息難受,只是後退。不料身子嵌在堅石之中,前進固是不能,後退却也不得,這一下他嚇得魂飛魄散,竭盡生平之力,雙臂向石上猛推,身子纔退出了尺許,猛覺得胸口一陣奇痛徹骨,竟已軋斷了一根肋骨。

且説張無忌在窄小的孔道中又爬行數丈,眼前越來越亮,再爬一陣,突然間陽光耀眼。他閉著眼定一定神,再睜開眼來,只見面前竟是一個生滿了紅花綠樹的翠谷。張無忌大聲歡呼,從山洞裡爬了出來。那山洞離地不過丈許,他輕輕一躍到底,脚底下踏著的是柔軟的細草,鼻中聞到的是清幽的花香,鳴禽間隔,鮮果懸枝,那想得到在這黑越越的洞穴之後,竟是另有這樣一個洞天福地?這時他已顧不到傷處的疼痛,放開脚步,向前疾奔,直奔了兩里有餘,纔遇一座高峰阻路。原來這翠谷四周高山環繞,似乎亙古以來,從未有人跡到過。四邉的山峰都是又高又陡,決計無法攀援出入。

張無忌滿心喜歡,見草地上有七八頭野羊低頭吃草,見了他也不驚避,樹上十餘頭猴児跳躍相嬉,看來虎豹之類猛獸身子笨重,不能踰峰而至。無忌心道︰「老天爺待我果眞不薄,安排下這等仙境,給我作葬身之地。」他緩步回到洞穴的入口處,只聽得朱長齡在洞穴彼端大呼︰「小兄弟,你出來,在這洞裡不怕悶死嗎?」張無忌大聲笑道︰「這裡好玩得緊呢?」在矮樹上摘了幾枚不知名的果子,拿在手裡,已聞到一陣甜香,咬了一口,更是鮮美絶倫,桃子無此爽脆,蘋果無此香甜,而梨子却遜它三分滑膩。他拿了一枚果子,從洞中擲了進去,叫道︰「接住,好吃的來了!」

那果子穿過山洞,在山壁上撞了幾下,已是{\upstsl{砸}}得稀爛,但朱長齡連皮帶核的咀嚼,越吃越是飢火上升,叫道︰「小兄弟,再給我幾個。」無忌叫道︰「你這人良心這麼壞,餓死也是應該。要吃果子,自己來吧。」朱長齡道︰「我身子太大,穿不過山洞。」張無忌笑道︰「你把身子切成兩半,不就能過來了麼?」朱長齡料想自己陰謀敗露,張無忌定要使自己慢慢餓死,以報此仇,當下也不向他求懇,索性破口大罵︰「賊小鬼,這洞裡就有果子,難道能給你吃一輩子麼?我在外面餓死,你不過多活三天,左右也是餓死。」張無忌不去理他,吃了十二三枚果子,肚子也飽了。過了半天,突然一縷濃煙,從洞口噴了進來。張無忌一怔之下,隨即省悟,原來朱長齡在洞外點燃松枝,想以濃煙薰自己出去,却那裡知道洞内别有天地,便是焚燒千擔萬擔的松柴,也是無濟於事。他想想好笑,假意大聲咳嗽。朱長齡叫道︰「小兄弟,快出來,我發誓決不害你就是。」張無忌大叫一聲︰「啊\dash{}」假裝暈去,自行走開,再也不去理他。

他向西走了二里多地,只見峭壁上有一片溶雪而成的瀑布衝擊而下,陽光照射下猶如一條大玉龍,極是壯麗。那瀑布瀉在一個碧綠的深潭之中,潭水却也不見滿,想是另有洩水的去路。張無忌觀賞了半晌,一低頭,只見自己適纔在山洞中爬行,手足上染滿了青苔汚泥,於是走近潭邉,除下鞋襪,伸足到潭水中去洗滌。他足底一和潭水碰到,「啊喲」一聲大叫,全身跳了起來。原來那潭水奇寒難當,足底碰到水面,竟比浸在滾水中還要痛楚。他扳過足底一看,只見肌膚上已是一片紅腫,若是多浸得片刻,只怕兩雙脚都要凍掉了。他伸了伸舌頭,叫道︰「奇怪,奇怪!」他自幼生長在冰火島上再冷的冰水雪塊也碰過了,却從未遇到過這般寒冷的潭水。更奇的是,此水雖冷,偏又不結冰。他知道此水中定是含有奇特的物事,退開兩步細看,忽聽得閣閣數聲,潭中跳出三隻遍體血紅的大蛙來。這蛙児約有尋常青蛙四倍大小,一出水,身上便冒出一縷縷白氣,便如冰塊化爲水氣一般。無忌見這些紅蛙生得奇異,童心大起,便要去捉一隻來玩玩。他慢慢躡步而前,突然撲上,伸手將一隻紅蛙按住。手掌剛和那紅蛙滑膩膩的背脊相觸,但覺一股暖氣從紅蛙身上直傳到自己手臂。不料那紅蛙極是兇惡,用力一掙,從他掌心掙脱,一口咬住他的右臂,再也不放。

張無忌大驚,忙伸手去拉,那知這紅蛙生有滿口利齒,緊緊咬住他的肌膚,倘若拉得重了,只怕連自己手臂上的肉也得拉下一大塊來。便在此時,另外兩頭紅蛙也跳躍而前,疾如電閃的撲上,分别咬住了無忌的雙脚。無忌從未見過這樣兇狠的大蛙,驚惶之下,左手五根手指使勁,拍的一響,捏破了右臂上那頭紅蛙的肚子,但覺手掌心熱烘烘的都是鮮血,看來這紅蛙吸血爲生,是以不但遍體血紅,並能在這奇寒的潭水中生存。

他俯下腰來,再將脚上的兩頭紅蛙捏死,這纔慢慢扳開死蛙的牙齒,看到自己臂上和脚背上的三排齒印,猶是心有餘悸。他指著三頭死蛙罵道︰「死蛙児,人家欺侮我,惡狗咬我,連你這小小的蛙児也來咬我。反正我肚子也餓了,我吃了你們,瞧你們還敢不敢欺侮我?」眼見那肥肥的蛙腿,想來味道必甘美,於是找些枝枝,從身邉取出火石火絨生了個火,將三隻紅蛙放在火上烤了起來。烤了一會,脂香四溢,眼見已熟,他已不管有毒無毒,撕下一條蛙腿,咬了一口,當眞是滑嫩鮮美,非任何美味所能及。片刻之間,將三隻紅蛙吃得乾乾淨淨,只剩下一堆骨頭。

約莫過了一頓飯時分,一股熱氣,突然從腹中冒了上來,只覺暖洋洋的,全身説不出的舒適受用,宛似泡在一大缸暖水之中洗澡一般。原來這紅蛙是天地間的一種異物,生於奇寒之地,其性却是至熱,否則無法在這寒潭中過活。若是常人吃了一隻,登時七孔流血而暴斃。剛巧張無忌身中玄冥神掌,體内積下無數陰毒,以至寒逢至熱,兩種毒性相互抵消,紅蛙的熱毒盡數消去,而體内的寒毒却也消減不少。

這是他無意中的巧遇,張無忌也不知其理,但覺全身慵倦,便欲睡倒。他生怕睡著之後,潭中又有紅蛙上來吸血,強睜雙眼,直走出里許,再也支持不住,便躺在草地上沉沉睡去。

這一呼呼大睡,待得醒來,月當中天,已是午夜,張無忌肚腹之中,猶有一團暖意緩緩滾動。他略加思索,已知這紅蛙乃是大有補益的物事,適纔這一場酣睡,自覺體内「心腎相交,水火相濟」,精神奕奕,伸手抬足之際,勁力也大勝往昔。當下打坐運氣,想把體内這股暖氣,試行推到各處經脈之中,但試行半晌,只覺頭暈目眩,煩惡欲嘔,只得罷休,嘆道︰「我原説那有這樣的好運氣,倘若暖氣能行走各處經脈,玄冥神掌的陰毒豈非就能治好了?」好在他早就一切任其自然,也不覺失望,到次日午間,肚中飢餓起來,折了一根長長的樹枝,伸到寒水潭中撩撥,只撩得幾下,樹枝上便有三四頭紅蛙牢牢咬住。張無忌收回樹枝,用石塊打死蛙児烤食。心想︰「一時既是不得便死,倒須留下火種。」於是圍了一個灰堆,將半燃的柴草藏在其中,以防熄滅。他自幼在冰火島上長大,一切用具全須自製,這種在野地裡獨自過活的日子,在他毫不希奇,忙忙碌碌的捏土爲盆,鋪草作床。忙到傍晩,想起朱長齡餓得慘了,於是摘了一大把鮮果,隔洞擲了過去。他生怕朱長齡若是吃了蛙肉,力氣大增,竟能衝過洞來,那可抵敵不住,是以烤蛙却不給他吃。這一次倒是朱長齡的幸運,倘若無忌不是有此顧慮,一念心慈,擲一頭烤蛙給他嘗嘗美味,那當場便送了他的老命。

如此過了數日,張無忌這一日正在砌一座土灶,忽聽得一頭猴子吱吱狂叫,聲音極是慘厲。張無忌循聲奔去,只見一頭小猴正在寒水潭邉,大叫大跳,背心上被三頭紅蛙咬住了吸血,潭中又有兩頭紅蛙跳上來咬牠。張無忌飛身躍去,抓住猴児右臂,先將牠拉得遠離寒潭,再弄死咬在牠背身上的紅蛙。只是那猴児的右爪腕骨却已被一頭大蛙咬斷,一雙手掌緊晃晃的懸著,痛得牠吱吱直叫。

無忌心想︰「我正苦於無伴,有隻小猴児做朋友倒好。」折了兩根枝條作爲夾板,把那猴児的腕骨續上,找些草藥,嚼爛了給牠敷在傷處。雖然幽谷之中,藥草難找,所敷的未具靈效,但憑著他的接骨手段,料得六七天後,斷骨便能續上。

那猴児居然也知感恩圖報,第二日便摘了許多鮮果,送給無忌,不到十天,斷腕果然好了。這一來,想是那小猴児出去向同類大加宣揚,張無忌倒成了這山谷中的百獸醫生,向他求治的尤以猿猴之屬爲多。猿猴的疾患和人相差不遠,生瘡的要拔毒生肌,跌傷的要止血裹創。張無忌大是高興,心想我與其醫人,還不如醫獸,至少他們不會反過頭來把我吃了。

如此過了一月有餘,他每日烤食紅蛙,體内寒毒發作之苦,漸漸消減。這一天清晨,他兀自酣睡未醒,必覺有隻毛茸茸的大手在他臉上輕輕撫摸。張無忌嚇了一跳,睜開眼來,只見一隻白色大猿,蹲在他的身旁。那大猿手裡抱著一隻小猴,正是無忌替牠接續腕骨的那猴児。那小猴吱吱{\upstsl{喳}}{\upstsl{喳}},説個不停,指著大白猿的肚腹。無忌鼻中聞到一陣腐臭之氣,見白猴肚上膿血糢糊,生著一個大瘡,便笑道︰「好,好!原來又帶病人瞧大夫來著!」大白猿伸出左手,掌中托著一枚拳頭大小的蟠桃,恭恭敬敬的呈上。

無忌從未見過這般大的蟠桃,心想︰「媽媽講故事時説,崑崙山有個女仙西王母,設蟠桃之宴,宴請群仙。這西王母雖是假的,但崑崙山出産仙桃,想是不假。」笑著接了,説道︰「我不收醫金,便無仙桃,我也跟你治瘡。」於是伸手到白猿肚子上輕輕掀了一下,不禁吃了一驚。

原來那白猿腹上的惡瘡,不過寸許圓徑,可是觸手堅硬之處,却大了十倍尚且不止。張無忌在醫書之上,從未見過有如此險惡的疔瘡,倘若這堅硬處儘數化膿腐爛,只怕是不治之症了。他按了按白猿的脈搏,却無險象,當下撥開猿腹上的長毛,再看那疔瘡時,更是一驚,只見牠腹上方方正正的一塊凸起,四邉用針線縫著。這顯然是人類手跡無疑,猿猴雖然聰明,決不可能會用針線。張無忌細察疔瘡,知是那凸起之物作祟,壓住血脈運行,以致腹肌腐爛,長久不愈,欲治此瘡,非得取出縫在肚中的那物不可。

説到開刀治傷,他跟胡青牛學得一手好本事,原是輕而易舉,只是手邉既無刀圭,又無藥物,那便麻煩得多了。略一沉思,又撿了一片尖石,磨得十分鋒利慢慢割開白猿肚腹上縫補過之處。那白猿年紀已是極老,頗具天性,知道張無忌給牠治病,雖然腹上劇痛,竟是強行忍住,一動也不動。張無忌割開右邉及上下兩端的縫線之處,揭開腹皮,只見牠肚子裡藏著一個油布包裹。這一下他更覺奇怪,這時不及拆視包中之物,將油布包放在一邉,忙又將白猿的腹肌縫好。手邉没有針線,只得以紅蛙的利齒作針,在牠腹上刺下一個個小孔,再將樹皮撕成細絲,穿過小孔打結,勉強補好。忙了半天,方始就緒,白猿雖然強壯,却也是躺在地上,動彈不得了。

張無忌洗去手上和油布包上的血漬,打開包來看時,原來包裹是四本薄薄的經書,只因油布包得緊密,雖是長期藏在猿腹之中,書頁却是完好無損。書面上冩著幾個彎彎曲曲的文字,無忌一個字也不識得,翻開來一看,四本書中盡是這些怪文,但每一行之間,却以蠅頭小楷冩滿了中國文字。張無忌定一定神,從頭細看,文中所記似是練氣運功的訣竅,慢慢誦讀下去,突然心頭一跳,有兩行字極是熟悉,略加回想,即行記起是在少林寺中所學到的「少林九陽功」,但繼續讀下去却又不同。他隨手翻閲,過得幾頁,又遇到了三行背熟了的經文,那却是父親所授的「武當内功心法」。

他心中突突亂跳,掩巻靜思︰「這到底是什麼經書?爲什麼既有少林九陽功,又有武當心法?」想到此處,登時記起太師父在帶自己上少林寺去時所説的故事來,怎樣太師父的師父覺遠大師學得「九陽眞經」,圓寂之前怎樣背誦經文,太師父、郭襄郭女俠、少林派無色大師三人怎樣各自記得一部份,因而武當、峨嵋、少林三派怎樣武功大進,數十年來分庭抗禮,名震武林,「難道這便是那部給人偸去了的九陽眞經?不錯,太師父説,那九陽眞經是冩在楞枷經的夾縫之中,這些彎彎曲曲的文字,想必是梵文的楞伽經了。那爲什麼是在猿腹之中呢?」

這一部經書,的確便是九陽眞經,至於何以藏在猿腹之中,其時世間已無一人知曉。原來在十餘年之前,瀟湘子和尹克西從少林寺藏經閣中盜得這部經書,被覺遠大師直追到華山之巓,眼看無法脱身,剛好身邉有隻蒼猿,兩人心生一計,便割開蒼猿肚腹,將經書藏在其中。後來覺遠、張三丰、楊過等搜索瀟湘子、尹克西二人身畔,不見經書,便放了他們帶同蒼猿下山
\footnote{\footnotefon{}〔按〕此事本末,具見「神雕俠侶」}。九陽眞經的下落,從此成爲武林中近百年來不解的大疑案。後來瀟湘子和尹克西帶同蒼猿,遠赴西域,兩人心中各有所忌,生怕對方先習成經中武功,害死了自己,互相牽制,遲遲不敢取出猿腹中的經書,終於來到崑崙山的驚神峰上時,尹瀟二人互施暗算,鬥了個兩敗倶傷。這部修習内功的無上心法,從此留在這頭蒼猿腹中。

瀟湘子的武功本來尚比尹克西稍勝一籌,但因他在華山絶頂打了覺遠大師一拳,由於反震之力,身受重傷,因之後來與尹克西相鬥時,反而先行斃命。尹克西臨死時遇見「崑崙三聖」何足道,良心不安,請他赴少林寺告知覺遠大師,那部經書是在這個猿猴的腹中?但他説話之時神智迷糊,口齒不清,他説「經在猿中」,何足道却聽作什麼「金在油中」。後來他信守言諾,果然遠赴中原,將這句金在油中的話跟覺遠大師説了,覺遠無法領會其中之意,固不待言,反而惹起一場絶大風波,武林中從此多了武當峨嵋兩派。

至於那頭蒼猿却是幸運,在崑崙山中採取仙桃爲食,得天地之靈氣,過了九十餘年,仍是跳縱如飛,全身黑黝黝的長毛也盡轉皓白,變成了一頭白猿。只是那部經書藏在牠肚腹之中,逼住大腸小腸,不免時時肚痛,肚上的腫瘡也時好時發,今日幸得張無忌給牠取出,就這頭白猿而言,倒是去了一個心腹大患。

這一切曲折原委,張無忌便是想破了腦袋,也是猜想不出,他呆了半晌,便取過白猿所贈的那枚大蟠桃,撕去薄皮,尚未入口。已是清香撲鼻,輕輕一咬,但覺一股極甜的汁水,緩緩流入咽喉,比之谷中那些不知其名的鮮果,可説是各擅勝場。張無忌吃完這枚大蟠桃,腹中已是半飽,心想︰「太師父當年曾説,若我習得少林、武當、峨嵋三派的九陽神功,或能驅去體内的陰毒。但這三派九陽功都是脱胎於九陽眞經,倘若這部經文當眞便是九陽眞經,那麼照書修習,又遠勝於分學三派的神功了。在這谷中左右也無别事,我照書修習便是。便算我猜錯了,這部經書其實毫無用處,甚而習之有害,最多也不過一死而已。」

他心無罣礙,便將三巻經書放在一處乾燥的所在,上面舖以乾草,再壓上三塊大石,生怕猿猴頑皮,玩耍起來你搶我奪,説不定便將經書撕得稀爛,手中只留下第一巻經書,先行誦讀幾遍,背得熟了,然後照書中之法,自第一句習起。他心想,我便算眞從經中習得神功,驅去陰毒,但既被活活的囚禁在這石谷之中,不論武功如何高強,總是不能出去,山中歳月正長,今日練成也好,明日練成也好,都無分别。他心中存了這個念頭,修習九陽眞經之時,成固欣然敗亦喜的,居然進展奇速,短短四個月時光,便已將第一巻經書上所載功夫,盡數學成。

當年達摩祖師手著九陰眞經,九陽眞經兩部武學奇書,一陰一陽,兩部書中的武功相輔相成,相生相剋,不分高下。只是九陽眞經中的功夫偏重養氣保命,九陰眞經則偏重致勝克敵。從内功純眞言,是「九陽」較勝,説到招數的奇幻變化,則是「九陰」爲優。當年銅屍陳玄風、鐵屍梅超風偸得九陰眞經下巻後,所修習的各種奇妙武功(見「射雕英雄傳」),九陽眞經中均付缺如,但九陽神功如能練到大成之境,却也非世間任何奇怪奇妙的武功所能傷。

張無忌練完第一巻經書後,屈指算來,胡青牛預計他毒發畢命之期早已過去,可是他身輕體健但覺全身眞氣流動,絶無半點病象,連以前時時發作的寒毒侵襲,也是要隔一月以上,纔偶有所感,而發作時也極是輕微。此時他更無懷疑,知道這部經書就算並非九陽眞經,却也於養生大有益處,加之他常食水潭中的血蛙,那白猿感他治病之德,常自採了大蟠桃來相贈,待得練到第二巻經書的一小半,體内寒毒已被驅得無影無蹤。本來此時再食血蛙,已有中毒之虞,可是一來他在不知不覺之中,九陽神功已練得小有成功;二來久食異種蟠桃,竟是百毒不侵。血蛙至陽之性,反而更加厚了他九陽神功的功力。

張無忌每日除了練功,便是與猿猴爲戲,採摘到的果實,總是分一半給朱長齡,倒是無憂無慮,自由自在。可是朱長齡局處在小小的一塊平台之上,當眞是度日如年,一到冬季,遍山冰雪,寒風透骨,這份苦處更是難以形容。張無忌練到第三巻經書時,早已不畏寒暑,高興起來便跳到寒水潭中去洗個澡。他全身眞氣流動,肌膚一逢外侵,自然而然的生出抗禦之力,血蛙牙齒雖利,却已咬他不到,潭水寒冷於冰,他也漫不在乎。

只是那九陽眞經越練到後來,越是艱深奥妙,進展也就越慢,第三巻整整花了一年功夫,最後一巻更是練了兩年有餘,方始功行圓滿。這一日午夜,張無忌揭過最後一頁經書,心中又是喜歡,又微微感到悵惘。他到這雪谷之中已是四年有餘,自己也從一個孩子長成爲身材高高的青年。這四年多來,説不定外面世界上已起了天翻地覆的大變,而他却安安靜靜的在深谷之中練成了九陽神功。這些日子來,他有時興之所至,也偶然與衆猿猴攀援山壁,登高遙望,以他那時精純無比的功力,若要逾峰出谷,原非難事,但他想到世上人心的陰險狠詐,不由得不寒而慄,心想何必到外面去自尋煩惱,自投羅網?在這美麗的山谷中直至老死,豈不是好?

他在山洞左壁挖了一個三尺來深的洞孔,將四巻九陽眞經,以及胡青牛的醫經、王難姑的毒經,一起包在從白猿腹中取出來的包油布之中,埋在洞内,填上了泥土,心想︰「我從白猿腹中取得經書,那是極大的機緣,不知千百年後,是否又有人湊巧來到此處,得到這三部經書?」伸出手指,在山壁上劃下六個大字︰「張無忌埋經處」。

他在修習神功之時,每日均是忙忙碌碌,心有所專,絲毫不覺寂寞,這一晩大功告成,心頭反覺空虛,暗想︰「此時朱伯伯便要再來害我,我也已無懼於他,不妨去跟他説説話。」於是彎腰向洞裡鑽去。他進來時十五歳,身子尚小,出去時已是十九歳,長大成人,却鑽不過那狹窄的洞穴了。他吸一口氣運起縮骨功來,全身骨骼擠攏,骨頭和骨頭之間的空隙縮小,輕輕易易的便鑽了過去。

朱長齡倚在石壁上,睡得正甜,夢見自己在家中大開筵席,厮役奔走,親朋趨奉,好不威風快活,突覺肩頭有人拍了幾下,一驚而醒!睜開眼來,只見一個高高瘦瘦的黑影,站在面前。朱長齡躍起身來,神智未曾十分清醒,叫道︰「你\dash{}你\dash{}」張無忌微笑道︰「朱伯伯,是我,張無忌。」朱長齡又驚又喜,又惱又恨,向他瞧了良久,纔道︰「你長得這般高了。哼,怎地一直不出來跟我説話?不論我如何求你,你總是不理?」張無忌微笑道︰「我怕你給我苦頭吃。」朱長齡右手倏出,施展「擒拿手」,一把抓住了他的肩頭,厲聲道︰「怎麼今天却不怕了?」突然間手掌心一熱,不由自主的手臂一震,便放開了他的肩頭,自己胸口兀自隱隱生痛,嚇得退開三步,呆呆的瞪著他,説道︰「你\dash{}你\dash{}這是什麼功夫?」

張無忌練成了九陽神功之後,首次試用,竟是威力絶倫,朱長齡原是一流高手,但被他神功一震之下,居然不得不撤掌鬆指。這一下張無忌還只使了二成力,若是全力施爲,只怕身不動、手不抬,一下子便能震斷對手的手臂。他眼見朱長齡如此狼狽驚詫,心中自是得意,笑道︰「這功夫還使得麼?」朱長齡又問︰「那是什麼功夫?」張無忌道︰「我不知,或許是九陽神功。」朱長齡吃了一驚,問道︰「你怎樣練成的?」張無忌也不隱瞞,便將如何替白猿治病,如何從牠腹中取得經書、如何依法修習等情一一説了。

這一番話只把朱長齡聽得又是妬忌,又是惱怒,心想︰「我在這絶峰之上吃了四年難以形容的苦頭,你這小子却練成了奥妙無比的神功。」他也不想自己處心積慮的陥害張無忌,纔落得今日的結果,但覺對方過於幸運,自己却太過倒霉,當下強忍這口怒氣,笑吟吟的道︰「那部九陽眞經呢,給我見識一下成不成?」張無忌心想︰「給你瞧一瞧那也無妨,難道你一時三刻便記得了?」便道︰「我已埋在洞内,明天拿來給你看吧。」朱長齡道︰「你已長得這般高大,怎能過那洞穴?」張無忌道︰「那洞穴也不太窄,縮著身子用力一擠,便這麼過來了。」朱長齡道︰「你説我能擠過去麼?」張無忌點頭道︰「明児咱們一起試試,洞裡地方很大,老是在這塊小小的平台上,味道確乎不大好受。」他心想朱長齡硬擠過去是不成的,但自己運功捏他肩膀、胸部、臀部各處骨骼,當可助他通過。

朱長齡笑道︰「小兄弟,你眞好,君子不念舊惡,從前我頗有對不起你之處,萬望你多多原諒。」説著深深一揖。張無忌急忙還禮道︰「朱伯伯不必多禮,咱們明児一起想法児離開此處。」朱長齡大喜,道︰「你説能離開這児麼?」張無忌道︰「猿猴既能進出,咱們也便能彀。」朱長齡道︰「那你爲什麼不早出去,一直等到現下?」張無忌微微一笑,道︰「從前我不想到外面去,只怕給人欺侮,現下似乎不怕了,又想去瞧瞧我的太師父、師伯、師叔、他們。」朱長齡哈哈大笑,拍手道︰「很好,很好!」退後了兩步,突然間身形一晃,「啊喲」一聲,踏了個空,身子從懸崖旁摔了下去。

這一下樂極生悲,竟然有此變故,張無忌大吃一驚,俯身到懸崖之外,叫道︰「朱伯伯,你好嗎?」只聽下面傳來兩下低微的呻吟。無忌大喜,心想︰「幸好没直摔下去,但只怕已是身受重傷。」聽那呻吟之聲,相距不過數丈,凝神一看,原來懸崖之下剛巧生著一株松樹,朱長齡的身子橫在樹幹之上,一動也不動。張無忌瞧那形勢,自己躍下去將他抱了再上懸崖,憑著此時功力,當不爲難。於是吸一口氣,看準了那根如手臂般伸出的枝幹,輕輕躍下。

那知他足尖離那枝幹尚有半尺,突然間那枝幹倏地墜下,這一來空中絶無半點借力之處,饒是他練成了絶頂神功,但究竟人非飛鳥,如何能再回上崖來?心念如電光般一閃,立時省悟︰「原來朱長齡又使奸計害我,他早扳斷了樹枝,拿在手裡,等我快要著足之時,輕輕一鬆手,便將那樹枝抛下。」但這時明白,已然遲了,身子筆直的墜了下去。

朱長齡在這方圓不過數丈的小小平台上住了四年,平台上的一草一木、一沙一石,無不爛熟於胸,他在黑暗中假裝摔跌受傷,料定張無忌定要躍下相救,果然奸計得逞,將無忌騙得墜下萬丈深谷。朱長齡哈哈大笑,拉著松樹旁的長藤,躍回懸崖,心想︰「我第一次没能擠過那個洞穴,定是心急之下,用力太蠻,以致壓斷肋骨。這小子身材比我高大得多,他既能過來,我自然也能過去。我取得九陽眞經之後,從那邉覓路回家,日後練成神功,無敵於天下,豈不妙哉?哈哈,哈哈!」

他越想越是得意,當即從洞穴中鑽了進去,没走多遠,便到了四年前折骨之處。朱長齡心中只有一個念頭︰「這小子比我高大,他能鑽過,我當然更能鑽過。」想法原本絲毫不錯,只是有一點却没料到︰「張無忌已練成九陽神功中的縮骨之法。」朱長齡平心靜氣,在那窄小的洞穴之中,一寸一寸的向前挨去,果然比四年前又多挨了丈許,可是到得後來,不論他如何出力,要向前半寸,也已決不可能。

\chapter{荊釵村女}

朱長齡心知若用蠻勁,又要重𨂻四年前的覆轍,勢必再擠斷幾根肋骨,於是定了定神,竭力呼出肺中存氣,果然身子又縮小了兩寸,能再向前挨了三尺。可是肺中無氣,越來越是窒悶,自覺一顆心跳得打鼓一般,幾欲暈去,知道不妙,只得先退出來再説。那知進去時兩足撐在高低不平的山壁之上,一路推進,出來時却已無可借力之處,雙手被巖石束在頭頂,伸展不開,半點力氣也使不出來。他心中却兀自在想︰「他身材比我高大,他既能過去,我也必能彀過去。爲什麼我竟會擠在這裡?當眞是豈有此理!」那知世上確有不少豈有此理之事,這個文才武功,倶臻上乘的高手,從此便嵌在這窄窄的山洞之中,進也進不得,退也退不出。

且説張無忌又中朱長齡的奸計,從懸崖上直墜下去,霎時間自恨不已︰「張無忌啊張無忌,你這小子忒煞無用。明知朱長齡奸詐無比,却一見面又上了他的惡當,該死,該死!」他雖自罵該死,其實却是拚死的求生,體内眞氣流動,運勁向上縱躍,想要將下墜之勢稍爲延緩,著地時便不致跌得碎骨。可是人在半空,虛虛晃晃,實是身不由已,但覺耳旁風聲不絶,頃刻之間,雙眼刺痛,地面上白雪的反光射進了目中。

張無忌知道生死之際,便繫於這一刻関頭,只見丈許之外有一個大雪堆,這時也無暇分辨雪堆中到底是何物,當即在空中翻了一觔斗,向那雪堆中撲去,身形斜斜劃了個弧線,左足已點上雪堆,波的一聲,身子已陥在雪堆之中。他苦練四年的九陽神功便於此時發生威力,借著雪堆中所生的反彈之力,向上一縱,但那萬尋懸崖上摔下來的這股力道何等厲害,只覺腿上一陣劇痛,雙腿腿骨一齊折斷。

他受傷雖重,神智却仍清醒,但見柴草紛飛,原來這大雪堆是農家積柴的草堆,不禁暗叫︰「好險,好險!倘若這雪堆之下藏的不是柴草,却是一塊大石頭,我張無忌便一命嗚呼。」他雙手用力,慢慢爬出柴堆,滾向雪地,再檢視自己腿傷,吸一口眞氣,伸手接好了折斷的腿骨,心想︰「我躺著一動也不動,至少要一個月方能行走,可是那也没有什麼,至不濟是以手代足,總不會在這裡活生生的餓死。」又想︰「這柴草堆明明是農家所積,附近必有人家。」他本想縱聲呼叫求援,但轉念一想︰「世上惡人太多,我獨個児躺在雪地中養傷,那也罷了,若是叫得一個惡人來,反而糟糕。」於是安安靜靜的躺在雪地,靜待腿骨折斷處慢慢的自行愈合。

如此睡了三天,腹中餓得咕嚕咕嚕直響,但他知接骨之初,最是動彈不得,倘若斷骨處稍有歪斜,一生便成跛子。因此始終以最大毅力,半分也不移動,眞是耐不住了,便抓幾把雪塊充飢。這三天中心裡只是想︰「從今以後,我在世上務要步步小心,決不可再上惡人的當。須知日後未必再能如此幸運,終能大難不死。」

到得第四天晩間,他靜靜躺著用功,只覺心地空明,周身舒泰,腿傷雖重,所練的神功却又深了一層,萬籟皆寂之中,猛聽得遠處傳來幾聲犬吠之聲,跟著犬吠聲越來越近,顯是有幾頭猛犬在追逐甚麼野獸。張無忌吃了一驚︰「難道是朱九眞姊姊所養的惡犬麼?{\upstsl{嗯}}!她那些猛犬都已被朱伯伯打死了,可是事隔多年,她又會養起來啊。」目凝向雪地裡望去,却見有一人如飛的奔來,身後三條大犬又吠又咬的追著他。那人顯已筋疲力盡,跌跌撞撞,奔幾步,便摔了一交,但害怕惡犬的利齒鋭爪,還是拚命的向前奔跑。張無忌想起數年前自己身被群犬圍攻之苦,不禁胸口熱血上湧。

他有心出手相救那被群犬追殺之人,苦於自己雙腿斷折,行走不得,驀地裡聽得那人長聲慘呼,摔倒在地,兩頭惡犬爬在他的背上狠咬。張無忌怒叫︰「惡狗,到這児來!」那三條大犬不懂得人話,果然如飛撲至,嗅到張無忌並非熟人,站定了狂吠幾聲,撲上來便咬。張無忌有心一試所練的神功,伸出手指,在每頭猛犬的鼻子上一彈,三頭惡犬先後了帳。無忌没想到隨便出手即行輕輕易易的殺斃三犬,對這九陽神功的威力,不由得暗自心驚。

只聽得那人呻吟之聲極是微弱,便道︰「這位兄台,你給惡犬咬得很厲害麼?」那人道︰「我\dash{}我不成啦\dash{}我\dash{}我\dash{}」張無忌道︰「我雙腿斷了,没法子行走。請你勉力爬過來,我瞧瞧你的傷口。」那人道︰「是\dash{}是\dash{}」氣喘吁吁的掙扎爬行,爬一段路,停一會児,爬到離張無忌丈許遠處,「啊」的一聲,伏在地下,再也不能動了。

兩人便是隔著這麼遠,一個不能過去,一個不能過來。張無忌道︰「大哥,你傷在何處?」那人道︰「我\dash{}胸口,\dash{}肚子上\dash{}給惡狗咬破肚子,拉出了腸子。」張無忌大吃一驚,知道肚破腸出,再也不能活命,問道︰「那些惡狗爲甚麼追你?」那人道︰「我\dash{}夜裡出來趕野豬,别\dash{}别{\upstsl{踩}}壞了莊稼,見到一位大小姐和一位公子在大樹下説話\dash{}我不過走近去瞧瞧\dash{}我\dash{}啊喲!」大叫一聲,再也没聲息了。

他這番話雖没説完,但張無忌十成已猜到了九成,多半是朱九眞和衛壁半夜出來私會,却讓這鄕農撞見了,朱九眞放犬咬死了他。正自氣惱,只聽得馬蹄聲響,有人連連呼哨,正是朱九眞在呼召群犬。蹄聲漸近,兩騎馬馳了過來。張無忌自練九陽神功後,目力大異常人,雖在黑暗之中,借著白雪反映上來的星光,依稀可以看到兩匹馬上坐著一男一女。那女子突然叫道︰「咦!怎地平西將軍他們都死了?」説話的正是朱九眞,她所養的猛犬,仍是各擁將軍封號,與以前絲毫無異。

和她並騎而來的正是衛璧,他縱身下馬,奇道︰「有兩個人死在這裡!」無忌心下暗暗打定了主意︰「他們若想過來害我,説不得,我下手可不能容情了。」朱九眞見那鄕農肚破腸流,死狀甚可怖,張無忌却是衣服破爛已到極點,蓬頭散髮,滿臉長滿了長長的鬍子,躺在地下一動也不動,想來也是被狗子咬死了。她急欲衛璧談情説愛,不願在這裡多所逗留,説道︰「表哥,走吧!這兩個泥腿子臨死拚命,倒傷了我三位將軍。」拉轉馬頭,便向西馳去。衛璧雖見三犬齊死,心中微覺古怪,但見朱九眞馳馬走遠,不及細看,當即躍上馬背,跟了下去。

張無忌聽得朱九眞的嬌笑之聲,遠遠傳來,心下只感惱怒,自己覺得奇怪,四年多前和她初遇時,對朱九眞敬若天神。只要她小指頭児指一指,就是要自己上刀山、下油鍋,也是毫無猶豫,但今日重見,不知如何,她身上的魅力竟是消失得無形無蹤。張無忌只道是修習九陽眞經之功,實則凡是少年男子,大都有過如此胡裡胡塗的一段初戀,這些熱情來得快,去得也快,日後頭腦清醒,對自己舊日的沉迷,往往不禁爲之啞然失笑。

得到第二日早晨,天空一頭兀鷹見地下的死人死狗,在空中盤旋了幾個圏子,便飛下來啄食。那知道這頭兀鷹也是命中該死,好端端的死人死狗不吃,偏向張無忌臉上撲下來,無忌手一伸,早扭兀鷹的頭頸,手上微一使勁,便將那鷹捏死了,喜道︰「當眞是天上飛下來的早飯。」拔去兀鷹羽毛,撕下鷹腿,便大嚼起來,雖是生肉,但餓了三日,却他吃得津津有味。一頭兀鷹没吃完,第二頭又飛了下來。張無忌便以鷹肉充飢,躺在雪地之中養傷,靜得腿骨愈合,接連數日,這曠野中竟是一個人也没經過。他身畔是三隻死狗,一個死人,好在隆冬嚴寒,屍體不會腐臭,他又過慣了寂寞獨居的日子,也不以爲苦。

這一日下午,他運了一遍内功,眼見天上兩頭兀鷹飛來飛去的盤旋,良久良久,終是不敢下來。他正自無聊,只見一頭兀鷹向下一撲,離地身子約莫三尺,便即衝向空際,身法轉折之間,極是美妙。他忽然想道︰「這一下轉折,如果能用在武功之中,襲擊敵人時對方固是不易防備,即使一擊不中,飄然遠颺,敵人也是極難還擊。」要知他所練的九陽神功純係修習内功,攻擊防禦的招數是半招都没有的。因此當年覺遠大師雖然練就一身神功,受到攻擊時却毛手毛脚,絲毫不會抵禦;張三丰也要楊過當面傳授四招,才能和尹克西放對。張無忌從小便學過武功,和覺遠及張三丰幼時截然不同,但要將極上乘的内功融化在他所學的招數之中,却也非短期内所能奏效。因此每見飛花落地,怪樹撐天,以及鳥獸之動,風雲之變,他往往便想到武功的招數上去。

這麼一想,他只盼空中的兀鷹盤旋往復,多現幾種姿態,正看得出神,忽聽得遠遠有人在雪地中走來,脚步細碎,似乎是個女子。張無忌轉過頭去,只見一個女子提著一隻籃子,很迅捷的走近。她看到雪地中的人屍犬屍,「咦」的一聲,怔住停步。張無忌定神一看,但見那是個十七八歳的少女,荊釵布裙,是個鄕村貧女,黃髮蓬蓬,面容黝黑,臉上肌膚凹凹凸凸,嘴角歪斜,生得極是醜陋,只是一對眸子頗有神采,身段也是苗條纖秀。

她走近一步,看見張無忌睜著眼瞧著她,微微吃了一驚,道︰「你\dash{}你没死麼?」張無忌道︰「我没死。」一個問得不通,一個答得有趣,兩人一想,都忍不住笑了起來。那少女笑道︰「你既不死,躺在這裡一動也不動的幹什麼?倒嚇了我一跳。」張無忌道︰「我從山上摔下來,把兩條腿都跌斷了,只好在這裡躺著。」那少女問道︰「這人是你同伴麼?怎麼又有三條死狗?」張無忌道︰「這三狗兇惡得緊,咬死了這位大哥,可是牠們也活不了啦。」

那少女道︰「你躺在這裡怎麼辦?肚子餓嗎?」張無忌道︰「自然是餓的,可是我動不得,只好聽天由命了。」那醜女嫣然一笑,從籃子中取出兩個餅來,遞了給他。張無忌道︰「多謝姑娘。」接了過來,却不便吃。那少女道︰「你怕我的餅中有毒嗎?幹麼不吃?」張無忌已有四年多没跟人説話,偶爾和朱長齡隔著山洞對答幾句,也是絶無意味,這時見那少女容貌雖醜,説話却很有風趣,心中喜歡,便道︰「是姑娘給我的餅子,我捨不得吃。」

這句話已有幾分調笑的意思,他向來誠厚,説話從來不油腔滑調,但在這醜女面前,心中輕鬆自在,不知不覺的這句話便衝口而出。那少女聽了,眼中忽現怒色,哼了一聲。張無忌心下大悔,忙拿起餅子便咬,只因吃得慌張,竟哽在喉頭,咳嗽起來。那少女轉怒爲喜,説道︰「謝天謝地,你這醜八怪不是好人,老天爺當場便要罰你。怎麼誰都不摔斷狗腿,偏生是你摔呢?」張無忌心想︰「我四年不剪髮,不剃面,自是個醜八怪,可是你也不見得美到那裡去,咱們半斤八兩,大哥别説二哥。」但這番話却無論如何不敢出口了,一本正經的道︰「我已在這裡躺了九天,好容易見到姑娘經過,你又給我餅吃,眞是多謝了。」那少女抿嘴笑道︰「我問你啊,怎地誰都不摔斷狗腿,偏生是你摔斷呢?你不回答,我就把餅子搶回去。」

張無忌見她這麼淺淺一笑眼睛中流露出極是狡譎的神色來,心中不禁一震︰「她這眼光,多麼像媽。媽臨去世時欺騙那少林寺的老和尚,眼睛中就是這麼一副神氣。」想到這裡,忍不住熱泪盈眶,跟著眼泪便流了下來。那少女「{\upstsl{呸}}」了一聲,道︰「我不搶你的餅子就是了,也用不著哭。原來是個没用的傻瓜。」張無忌道︰「我又不希罕你的餅子,只是我自己想起了一件心事。」那少女本已轉身,走出兩步,聽了這句話,轉過頭來,説道︰「什麼心事?你這傻頭傻腦的傢伙,也會有心事麼?」張無忌嘆了口氣,道︰「我想起了媽媽,我去世的媽媽。」

那少女{\upstsl{噗}}{\upstsl{哧}}一笑,道︰「以前你媽媽常給你餅吃,{\color{blue}是不是?」張無忌道︰「我媽以前常給我餅吃的,}不過我所以想起她,因爲你笑的時候,很像我媽。」那少女怒道︰「死鬼!我很老了麼?老得像你媽了?」説著從地下拾起木柴,在無忌身上抽了兩下。無忌若要奪下她手中木柴,自是輕而易舉,但想︰{\color{blue}「她不知我媽年輕貌美,只道是跟我一般的醜八怪,也難怪她發怒。」由得她打了兩下,説道︰「我媽去世的時候,}相貌是很好看的。」\footnote{\footnotefon{}藍色部分爲連載版文本脱漏,另據三聯版文本補足。}

那少女板著臉道︰「你取笑我生得醜陋,你不想活了。我拉你的腿!」説著彎下腰去,作勢要拉他的腿。張無忌吃了一驚,自己腿上斷骨剛起始愈合,給她一拉那便全功盡棄,忙抓了一團雪,只要那少女的雙手碰到自己腿上,立時便打她眉心穴道,叫她當場昏暈。幸好那少女只是嚇他一嚇,見他神色大變,説道︰「瞧你嚇成這副樣子!誰叫你取笑我了?」張無忌道︰「我若是存心取笑姑娘,教我這雙腿好了之後,再跌斷三次,永遠好不了,終生做個跛子。」那少女嘻嘻一笑,坐到無忌身旁,道︰「你媽既是個美人,怎地拿我來比她?難道我也好看麼?」

張無忌呆了一呆,道︰「我也説不上什麼緣故,只覺得你有些像我媽。你雖然没我媽好看,可是我喜歡看你。」那少女彎過中指,用指節輕輕在無忌的額頭上敲了兩下,笑道︰「乖児子,那你叫我作媽媽吧!」説了這兩句話,登時覺得不雅,按住了口,轉過頭去,可是仍舊忍不住笑出聲來。張無忌瞧著她這副神情,依稀記得從前在冰火島上之時,媽媽跟爸爸説笑,活脱也是這個模樣,霎時之間,只覺這醜女一點也不醜,清雅嫵媚,風緻嫣然,怔怔的呆望著她,不由得痴了。那少女回過頭來,見到他這副獃相,笑道︰「你爲什麼喜歡看我,且説來聽聽。」張無忌呆了半晌,搖了搖頭,道︰「我説不上來。我只覺得瞧著你時,心中很舒服,很平安,你只會待我好,不會欺侮我!害我!」那少女笑道︰「哈哈,你全錯了,我生平最喜歡害人。」突然提起手中的木柴,在無忌斷腿上敲了兩下,跳起身來便走。這兩下出其不意,正好敲在他斷骨的傷處,無忌大聲呼痛︰「哎喲!」只聽得那少女格格嘻笑,回過頭來扮了個鬼臉。無忌眼望著她漸漸遠去,斷腿處的疼痛甚是難熬,心想︰「原來女子都是害人精,美麗的會害人,難看的也一樣叫我吃苦。」

這一晩睡夢之中,他好幾次夢見那少女,又好幾次夢見母親,又有幾次,竟分不清到底是母親還是那少女。他瞧不清夢中那臉龐是美麗還是醜陋,只是見到那澄澈的眼睛,又狡獪又嫵媚的望著自己。他夢到了児時的事情,雖然是母親,也常常捉弄他,故意伸足絆他跌一交,等到他摔痛了哭將起來,母親又抱著他不住的親吻,不住説︰「乖児子别哭,媽媽疼你!」

他在睡夢中突然醒轉,猛地裡想起了一件以從來没想到過的事︰「媽媽爲什麼這般喜歡讓人受苦?義父的眼睛是媽媽打瞎的,兪三師伯是在媽的手下以致殘廢的,臨安府龍門鏢局全家是媽殺的,她到底是好人呢,還是壞人呢?」他望著天空中不住瞬眼的星星,過了良久良久,嘆了一口氣,説道︰「不管她是好人壞人,她是我媽媽。」心中想道︰「要是媽媽還活在世上,我眞不知有多愛她。」

他又想到了那個村女,眞不懂她爲什麼莫名其妙的來打自己斷腿,「我一點也没得罪她,爲什麼要我痛得大叫,她纔高興?難道她眞的是喜歡害人?」他很想她再來,但又怕她再想什麼法児加害自己。他摸到身邉那塊吃了一半的餅子,想起那村女説話的神情︰「你媽既是個美人,怎地拿我來來比她?難道我也好看麼?」忍不住自言自語︰「你好看,我喜歡看你。」

這般胡思亂想的躺了兩日,那村女並没再來,張無忌心想她是永遠不會來了。那知到第三天下午,那村女挽著籃子,從山坡後轉了出來,笑道︰「醜八怪你還没餓死麼?」無忌道︰「餓死了一大半,剩下一小半還活著。」那少女笑嘻嘻的坐在他身旁,忽然伸足在他斷腿上踢了一脚,問道︰「這一半是死的還是活的?」張無忌大叫︰「啊喲!你這人怎麼這樣没良心?」那少女道︰「什麼没良心?你待我有什麼好?」張無忌一怔,道︰「你大前天打得我好痛,可是没有恨你,這兩天來,我在天天想你。」那少女臉上一紅,便要發怒,可是強忍住了,説道︰「誰要你這醜八怪想?你想我多半没有好事,定是肚子裡罵我又醜又惡。」張無忌道︰「你並不醜,可是爲什麼定要害得人家吃苦,你纔喜歡?」那少女格格笑道︰「别人不苦,怎顯得出我心中喜歡?」

她見張無忌一臉不以爲然,却不説話,又見他手中拿著吃剩的半塊餅子,相隔三天,居然還没吃完,説道︰「這塊餅一直留到這時候,味道不好麼?」張無忌道︰「是姑娘給我的餅子,我捨不得吃。」他在三天前説這句話時,有一半意存調笑,但這時却説得誠誠懇懇,那少女知他所言非虛,微覺害羞,道︰「我帶了新鮮的餅子來啦。」説著説著從籃中取了許多食物出來,除了餅子之外,又有一隻燒雞,一條烤羊腿,香噴噴的,拿著還有些燙手。張無忌大喜,四年多來,除了血蛙之外,從未吃過肉食,這雞腿一入口,眞是美無窮。那少女見他吃得香甜,笑吟吟的抱膝坐著,説道︰「醜八怪,你吃得開心,我瞧著倒也好玩。我對你似乎有點児不同,就算不害你,也能教我喜歡。」張無忌道︰「人家高興,你也高興,那纔是眞高興啊。」那少女冷笑道︰「哼!我跟你説在前頭,這時候我心裡高興,就不來害你,那一天心中不高興了,説不定會整治得你死不了,活不成,那時候你可别怪我。」張無忌搖頭道︰「我從小給壞人整治到大,越是整治,越是硬朗。」那少女冷笑道︰「别把話説得滿了,咱們走著瞧吧。」

張無忌道︰「待我腿傷好了,我便走得遠遠的,你就是想折磨我,害我,也找不到我了。」那少女道︰「那麼我先斬斷了你的腿,叫你一輩子不能離開我。」張無忌聽到她冷冰冰的聲音。不由得打了個寒噤,只覺她説得出做得到,這兩句話絶非隨口説説而已。那少女向他凝視半晌,嘆了口氣,忽然臉色一變,説道︰「你配麼,醜八怪!你也配給我斬斷你的狗腿麼?」驀地裡站起來,搶過張無忌没吃完的燒雞、羊腿、麵餅,遠遠擲了出去,一口口唾沫向張無忌臉上吐去。

張無忌怔怔的瞧著她,只覺她並不是發怒,也不是輕賤自己,却是滿臉慘悽之色,似乎心中有説不出的難受。張無忌對别人的傷心不幸,向來甚是同情,見那村女如此哀傷,有心想勸慰她幾句,可是一時之間,却又想不出適當的言辭。

那村女見張無忌這般神氣,突然住口,喝道︰「醜八怪,你心裡在想什麼?」張無忌道︰「姑娘,你爲什麼這般不高興?説給我聽聽,成不成?」那少女聽他如此溫柔的説話,再也無法矜持,驀地裡坐倒在張無忌身旁,手抱著頭,抽抽咽咽的哭了起來。張無忌見她肩頭起伏,纖腰如蜂,甚是楚楚可憐,便低聲道︰「姑娘,是誰欺侮你了?等我腿傷好了之後,我去給你出氣。」那少女一時止不住哭,過了一會纔道︰「没有人欺侮我,是我生來命苦,我自己又不好,心裡想著一個人,總是放他不下。」張無忌點點頭,道︰「那是個年輕男子,是不是?他待你很兇狠吧?」

那少女道︰「不錯!他生得很英俊,可是傲慢得很。我要他跟著我去,一輩子跟我在一起,他不肯,那也罷了,那知還罵我,打我,將我咬得身上鮮血淋漓。」張無忌怒道︰「這人如此蠻橫無理,姑娘以後再也不要理他了。」那少女流泪道︰「可\dash{}可我總是放他不下啊,他遠遠避開我,我到處找他不著。」張無忌心想︰「這種男女間的情愛之事,實是勉強不得。這位姑娘容貌雖然差些,但顯是個至性至情之人。她脾氣雖然有點児古怪,那也是爲了心下傷痛,失意過甚的緣故。想不到那男子對她竟是如此狠毒兇狠!」於是柔聲道︰「姑娘,你也不用難過了,天下好男子有的是,你何必牽掛這個負心薄倖的惡漢。」那少女嘆了一口長氣,眼望遠處,呆呆出神。張無忌知她終生是忘不了意中的情郎,説道︰「那個男子,不過打你一頓,可是我所遭之慘,却又勝於姑娘十倍。」那少女道︰「怎麼啦?你受了二個美麗姑娘的騙麼?」張無忌道︰「本來,她也不是有意騙我,只是自己獃頭獃腦,見她生得美麗,就呆呆的看她。其實我那裡配得上她,我心中也没有什麼妄想。但她和她爹爹暗中擺下了一個毒計,害得我慘不可言。」説著拉起衣袖,指著手臂和臂膀上的累累傷痕,道︰「這些牙齒印,都是她所養的惡狗所咬。」

那少女見到這許多傷疤,不禁勃然大恕,説道︰「是朱九眞這賤丫頭害你的麼?」張無忌奇道︰「你怎麼知道?」那少女道︰「這賤丫頭愛養惡犬,方圓數百里地之内,人人皆知。」張無忌點點頭,淡然道︰「是的。這些傷痕早已好了,我早已不痛了,幸好性命還活著,我也没死,也不必再恨她了。」那少女和他四目相對,凝視半晌,但見張無忌臉上神色平淡沖和,閒適自在,心中頗有些奇怪,問道︰「你叫什麼名字?爲什麼到這児來?」

無忌心想︰「我自到中土,人人向我打聽義父的下落,威逼誘騙,無所不用其極,以致我吃盡了不少苦頭。從今以後,『張無忌』這人算是死了,世上再没人知道金毛獅王謝遜的所在了。就算日後再遇上比朱長齡更厲害十倍的人,也不怕落入他的圏套,無意中害我義父。」於是説道︰「我叫阿牛。」那少女微微一笑,道︰「姓什麼?」張無忌心道︰「我姓張、姓殷、姓謝都不好,『張』和『殷』兩個字的切音是曾字。」便道︰「我我姓曾。姑娘貴姓?」那少女身子一震,道︰「我没姓。」隔了片刻,緩緩的道︰「我親生爹爹不要我,見到我就會殺我。我怎能姓爹爹的姓?我媽媽是我害死的,我也不能姓她的姓。我生得醜,以後你叫我醜姑娘便了。」

張無忌驚道︰「你\dash{}你害你媽媽?那怎麼會?」那少女嘆了口氣。説道︰「這件事説來話長。我有兩個媽媽,我親生的媽媽是我爹爹原配,一直没生児養女,爹爹便娶了二娘。二娘生了我兩個哥哥,一個姊姊,爹爹就特别寵愛她,媽後來生了我,偏生又是個女児。二娘恃著爹爹寵愛,她自己的娘家又很有來頭,我媽常受她的欺壓,只有偸偸痛哭。我哥哥姊姊又厲害得很,幫著他們親娘,處處欺負我媽,你説,我怎麼辦呢?」張無忌道︰「你爹爹該當秉公調處纔是啊。」那少女道︰「就因我爹爹一味袒護二娘,我才氣不過了一刀殺了我那二娘。」

張無忌「啊」的一聲,大是驚訝,他是武林中人,這幾年來見慣了殺人毆鬥之事,原也不以爲奇,可是聽到這個平平常常的村女居然也動刀子殺人,却頗出意料之外。那少女説到這件事的時候,聲調平淡,絲毫不見激動,慢慢的道︰「我媽一見我闖下這個大禍,護著我立刻逃走。但我姊姊跟著追來,要捉我回去,我媽阻攔不住,爲了救我,便抹脖子自盡。你説,我媽的性命不是我害的麼?倘若我爸爸見到我,不是非殺我不可麼?」

這一番話,只將張無忌聽得一顆心怦怦亂跳,自忖︰「我雖然不幸,父母雙亡,可是我爹爹媽媽生時何等恩愛,對我何等憐惜,比之這位姑娘的遭遇,我却又幸運萬倍了。」想到這裡,對那少女同情之心更甚,柔聲道︰「你離開家裡很久了麼?這些時候便獨個児在外邉麼?」那少女點點頭。無忌又問︰「你想到那児去?」那少女道︰「我也不知道,世界很大,東面走走,西面走走。只要不碰到我爹爹和哥哥姊姊,也没什麼。」張無忌胸中,突然興起「同是天涯淪落人,相逢何必曾相識」之感,當年他萬里迢迢的護送著楊不悔,也不過是一念生憫,這時見那少女楚楚可憐,便道︰「等我腿好之後,我陪你去找那位\dash{}那位大哥。問他到底對你怎樣。」

那少女道︰「倘若他又來打我呢?」張無忌昂然道︰「哼,他敢碰你一根毫毛,我決計不和他干休。」那少女道︰「要是他對我不理不睬,話也不肯説一句呢?」張無忌啞口無言,心想自己武功再高,也不能勉強一個男子來愛上他所不愛的女子,呆了半晌,道︰「我盡力而爲。」那少女突哈哈大笑,前仰後合,似乎是聽到了一句世界上最好笑的笑話。

張無忌奇道︰「什麼事好笑?」那少女笑道︰「醜八怪,你是什麼東西?人家會來聽你的話麼?再説,我到處找他,找不到人,也不知這會児他是活著還是死了?你盡力而爲,你有什麼本事?哈哈,哈哈!」張無忌一句話已到了口邉,但給她笑得脹紅了臉,説不出口。那少女見他囁囁嚅嚅,停了笑,問道︰「你要説什麼話?」張無忌道︰「你要笑我,我便不説了。」那少女冷冷的道︰「哼,笑也笑過了,最多不過是再給我笑一場,還會笑死人麼?」張無忌大聲道︰「姑娘,我對你是一片好心,你如此笑我,可是不該。」那少女道︰「我問你,你本來要跟我説什麼話?」

張無忌道︰「你既是孤苦伶仃,無家可歸,我跟你也是一般。我爹爹媽媽都死了,也没有兄弟姊妹。我本想跟你説,那個惡人若是仍舊不理你,咱們不妨一塊作個伴児,我也陪著你説話解悶。但你既説我不配,那麼你就請便吧。」那少女怒道︰「你當然不配!那個惡人比你好看一百倍,我在這児跟你歪纏,儘説些廢話,眞是倒霉。」説著將掉在雪地中的羊腿熟鴨一陣亂踢,掩面疾奔而去。

這麼一頓好没來由的排揎,張無忌却不生氣,心道︰「這位姑娘眞是可憐,她心中不好過,原也難怪。」

\chapter{初演神功}

忽聽得脚步聲響,那少女又奔了出來,惡狠狠的道︰「醜八怪,你心裡一定不服氣,説我自己相貌這般醜陋,却還在瞧你不起,是不是?」張無忌搖頭道︰「不是的。你相貌不很好看,我纔跟你一見投緣,倘若你没有變醜,像從前那樣\dash{}」那少女突然驚呼︰「你\dash{}你怎地知道我從前不是這樣子的?」張無忌道︰「我這一次見你,你臉上比上次見你時,又腫得厲害了些,皮色也更黑了些。如果一個人生來便這樣,決不會越來越難看的。」那少女驚道︰「我\dash{}我這幾天不敢照鏡子。你説我是在越來越難看了?」張無忌柔聲道︰「一個人只要心地好,相貌美醜有何分别?我媽媽跟我説,越是美貌的女子,良心越壞,越會騙人,叫我要特别小心提防。」

那少女那有心思去理他媽媽説過什麼話,急道︰「我問你啊,你第一次見我時,我還没有變得這樣醜怪,是不是?」張無忌知道若是答應一個「是」字,她必傷心難受,只是怔怔的望著她,心中對她很是憐憫。那少女聰明之極,一見到他臉上神色,早料到他所要回答的是什麼話,掩面哭道︰「醜八怪,我恨你,我恨你!」這一次離去,却不再回轉了。

張無忌又躺了兩天,那日晩上,有頭餓狼出來覓食,邉嗅邉爬。走到張無忌身邉來。無忌手起一拳,登時將那餓狼打死。這頭野狼覓食不得,反而做了無忌肚中的食料。

如此過了數日,張無忌腿傷已愈合大半,大約再過得七八天,便可起立行走了,心想那個村女這一去之後,從此不會再來,只可惜連名字也没問她,又想︰「她臉上容色何以會越變越醜,這事倒令人猜想不透。」想了半日無法解答,也就不再去想,迷迷糊糊的便睡著了。睡到半夜,睡夢中忽聽得遠處有幾個人踏雪而來。這時他所練的九陽神功已有兩三成火候,便在沉睡之中,方圓數十丈内稍有異動,也決計逃不過他的耳目,這幾個人一齊走路,他立時便驚醒了。張無忌雙腿仍是不可移動,上身却已能坐直,當下坐起身來,向脚步聲處一望,這晩上一弦新月如眉,淡淡月光之下,只見走來共有七人,當先一人身形婀娜,似乎便是那個村女。他凝目細看,心下微覺驚訝,這人果然便是那容貌醜陋的少女,可是她身後的六人,却是散成扇形,似乎是防她逃走了的模樣。無忌心道︰「難道她是被她爹爹、哥哥、姊姊們拿住了?怎麼却到這児來?」

他心中轉念未定,那少女和她身後六人已然走近。張無忌一看,那一驚更是非同小可,原來那六人他無一不識,左邉是雪嶺雙姝之一的武青嬰。她父親武烈、她師兄衛璧,右邉是崑崙派掌門人何太沖,他妻子班淑嫻,走在最右邉的是個中年女子,面目依稀相識,却是峨嵋派的丁敏君,張無忌大奇︰「她怎麼跟這些人都相識?難道她也是武林中人,識破了我本來面目,便引他們來拿我,逼問我義父的下落?」他想到此處,心下更無懷疑,不覺大是氣惱︰「我和你無冤無仇,原來你也來加害於我!」尋思︰「我雙足眼下不能動彈,這六個人没一個是弱者,説不定這村女的武功也強。我姑且跟他們虛於委蛇,答應帶他們去找我義父。待得將雙腿養傷好了,那時再跟他們一個個算帳。」

若在四年之前,張無忌只是將性命谿出去不要,任由對方如何加刑威逼,總是咬緊牙関不説出而已,但此時一來他年紀大了,二來練成九陽眞經後神情心定,遇到任何危難都能沉著應付,當下心中微微冷笑,絲毫不感畏懼,只是没料想到那村女居然也來出賣自己,憤慨之中,不自禁的有些傷心,索性躺在地下,曲臂作枕,誰也不理會。

那村女走到張無忌身前,靜靜的瞧了他半晌,隔了良久,纔慢慢轉過身去。張無忌聽到她極輕微的嘆了口氣,這一聲嘆息聲音極輕極輕,可是嘆息之中,却充滿了哀傷之意。張無忌心下冷笑︰「要殺要剮,悉聽尊便,又何必假惺惺的可憐起我來?」只見衛璧將手中長劍一擺,冷笑道︰「你説臨死之前,定要去和一個人見上一面,我道定是個貌如潘安的英俊少年,却原來是這麼一個醜八怪,哈哈,好笑啊好笑,這人和你果然是天生一對,地生一對。」那村女竟是毫不生氣,只淡淡的道︰「不錯,我臨死之前,要來再瞧他一眼。因爲我要明明白白的問他一句話,我聽了之後,才能死得瞑目。」

張無忌大奇,對兩人所説的言語,半點也不懂,似乎這六人拿住這村女要殺她,而她却要來再瞧自己一面,有事要問,便道︰「姑娘,到底是什麼事?」那村女道︰「我有一句話問你,你須得老老實實的答我。」張無忌道︰「是我自己的事!我件件都可明白相告。若是旁人的事,便是我身受千刀萬箭之苦,也決計不能吐露一字半句。」他生怕那村女問的是謝遜的所在,是以先把言語説得絶了。

那村女冷笑道︰「旁人的事,要我擔什麼心?我問你,那一天你跟你我説,咱兩人都是孤苦伶仃,無家可歸,你願意跟我作伴。這句話確是出於眞心肺腑之言麼?」張無忌坐起身來,只見她眼光中又露出那哀傷的神色來,便道︰「我自是眞心的。」那村女道︰「那麼你是不嫌我容貌醜陋,願意和我一輩子厮守麼?」張無忌怔了一怔,這「一輩子厮守」五個字,他心中一直没想到過,只是他不忍見這村女哀傷無依,便道︰「什麼醜不醜,美不美的,我一點也不放在心上,你如要我陪伴你説話談心,我自然也是很喜歡的。」那村女聲音顫抖,問道︰「那麼你是願意娶我爲妻了?」

張無忌身子一震,大吃一驚,半晌説不出話來,喃喃的道︰「我\dash{}我没想過\dash{}娶妻子\dash{}」只聽得衛璧和武青嬰一起哈哈大笑起來,衛璧笑道︰「連這樣一個又醜又老的鄕巴佬也要你,咱們便不殺你,你活在世上又有什麼味児,還不如就在這塊大石上一頭撞死了吧。」張無忌凝視著那村女的臉,只見她低下了頭,眼泪水一滴滴的流了下來,顯是心中悲傷無比,只不知是爲了她自己命在旦夕,是爲了她容貌醜陋,還是爲了衛璧那些利刃般的諷剌譏嘲?

張無忌心中大動,驀地裡想起自己父母雙亡之後,顚沛流離,不知受了人家的多少欺侮,這村女煢煢弱質,年紀比自己小,身世比自己更是不幸,這時候不知何以巴巴的來問這一句話,自己焉可令她傷心落泪,受人侮辱?何況這少女這般相問,自是誠心委身。「我一生之中,除了父母、義父,以及太師父、衆位師伯叔,有誰是這般眞心的関懷過我?我日後好好待她,她也好好待我,兩個人相依爲命,有什麼不好?」眼見那少女身子顫抖,便要走開,張無忌左手伸出,握住了她的右手,大聲道︰「姑娘,我誠心願意,娶你爲妻,只盼你别説我不配。」

那少女聽了這話,眼光中霎時間射出極明亮的光采,低低的道︰「阿牛哥哥,你這話不是騙我麼?」張無忌道︰「我自然不騙你。從今而後,我會盡力愛護你,照顧你,不論有多少人來跟你爲難,不論有多麼厲害的人來欺侮你,我寧可自己性命不要,也要保護你周全。我要使你心中快樂,忘去了從前的苦處。」那少女坐下地來,倚在他身旁,又握住了他另隻手,柔聲道︰「你肯這樣待我,我眞是快活。」閉上了雙眼道︰「你再説一遍給我聽,我要每一個字都記在心裡。你説啊,你要怎樣待我?」

張無忌見她喜慰無比,心下也感快樂,握著她一雙小手,只覺柔膩滑嫩,溫軟如綿,説道︰「我要使你心中快樂,忘却小時候的苦處,不論有多少人欺侮你,跟你爲難,我寧可不要自己性命,也要保護你周全。」那村女臉露甜笑,靠在他胸前,柔聲道︰「從前我叫你跟著我去,你非但不肯,還打我、罵我、咬我\dash{}現下你跟我這般説,我很是歡喜。」張無忌聽了這幾句話,心中登時涼了,原來這村女閉著眼睛聽自己説話,却仍是把他幻想作她心目中的情郎。那村女只覺得他的身子一顫,睜開眼來,臉上的神色非常奇,又是氣憤,又是失望,但也不免帶著幾分歉仄和柔情。她定了定神,説道︰「阿牛哥哥,你願娶我爲妻,我很感激,像我這般醜陋的女子,你居然不加嫌棄。可是早在幾年之前,我的心早就屬於旁人的了。那時候他尚且不睬我,這時見我如此,更加眼角也不瞧我一眼。這個狠心短命的小鬼啊\dash{}」

她雖罵那人爲「狠心短命的小鬼」,可是罵聲之中,仍是充滿著不勝眷戀低徊之情。張無忌聽在耳中,心下酸溜溜的滿不是味児。武青嬰冷冷的道︰「他也肯娶你爲妻了,情話也説完啦,可以起來了吧?」那村女慢慢站起身來,對張無忌道︰「阿牛哥,我快死了。就是不死,我也決不能嫁你。但是我很喜歡聽你剛纔跟我説過的話。你别惱我,有空的時候,便想我一會児。」她的話説得很是溫柔,很是甜蜜,張無忌忍不住心下一酸。

只聽得班淑嫻嘶著嗓子説道︰「我們已如你所願,讓你跟這人見面一次。你也當言而有信,將那人的下落説了出來。」那村女道︰「好!我知道那人曾經藏在他的家裡。」説著伸手向武烈一指。武烈臉色微變,哼了一聲,道︰「瞎説八道!」衛璧道︰「我們要問你,你殺我朱九眞表妹,到底是受了何人指使?」張無忌這一驚眞是非同小可,道︰「殺了朱\dash{}朱九眞姑娘?」衛璧瞪了他一眼,惡狠狠的道︰「你也知道朱九眞姑娘?」張無忌道︰「雪嶺雙姝大名鼎鼎,誰没聽見過?」武青嬰嘴角邉掠過一絲笑意,大聲道︰「喂,你到底是受了誰的指使?」

那村女道︰「指使我來殺朱九眞的,是崑崙派的何太沖夫婦,峨嵋派的滅絶師太。」武烈大喝一聲︰「你妄想挑撥離間,又有何用?」呼的一聲,便向那村女拍去。他這一喝威風凜凜,掌隨喝聲而出,便只一掌,激得地上雪花飛舞。那村女不敢強擋,身形一閃,避過了他這一掌,身法奇幻之極,不知她脚下如何跨步。

張無忌心下一片混亂︰「她原來當眞是武林中人。她去殺了朱姑娘,那自是爲了我。我説受了朱姑娘的騙,被她所養的惡犬咬得遍體鱗傷,我可没要她去殺人啊。我只道她因爲相貌變醜,家事變故,以致脾氣古怪,那知竟是動不動便殺人。」只見衛璧和武青嬰各持長劍分從左右夾擊,劍氣掌風之中,夾著地下激起的一片雪花。張無忌凝神觀戰,只見那村女東一閃西一竄,儘量避開武烈雄渾的掌力,但對武青嬰和衛璧的劍招似乎不在意下,突然間纖腰一扭轉到了武青嬰的身側,拍的一聲,打了她一記耳光,左手探處,已搶過了她的長劍,武烈和衛璧大驚,雙雙來救。

那村女長劍顫動,叫聲︰「著!」竟是硬生生在武青嬰的臉上劃了一條血痕,想是武青嬰一再譏笑貌醜,因而冒奇險,不理武烈和衛璧從兩側進攻,強使武青嬰的俏臉受傷。

武青嬰一聲驚呼,向後便倒,其實她受傷不重,但她愛惜容貌,只覺臉上刺痛,便已心驚膽戰。武烈左手一掌向那村女按去,那村女斜身閃避,叮{\upstsl{噹}}一響,手中長劍和衛璧的長劍相交。張無忌没看清她手腕如何奇奇怪怪的一轉,衛璧已然長劍脱手,飛向天空。但就此時,武烈右手食指顫動,已點中了她左腿外側的伏兔風市兩穴。武烈這兩下點穴,正是家傳的一陽指法,雖然他遠遠不及上當年的一燈大師甚而祖上武三通的造詣,但指力究是非同小可。那村女輕哼一聲,立足不定,倒在張無忌身上。那風市穴屬於足少陽膽經,伏兔穴屬於足陽胃經,一經一陽指的指力透入,那村女但覺全身暖洋洋的,半點力氣也使不出來,便是想抬一根手指,也是宛似有千斤之力縛著,只是身上却不覺絲毫痛苦。須知那武烈雖非正人端士,但這一陽指的武學,却是極爲正大光明,被點中了的人只是失却抗拒之力,不受任何苦楚。

武青嬰拾起衛璧的長劍,恨恨的道︰「醜丫頭,我却不讓你痛痛快快的死,只斬斷你兩手兩腿,讓你在這裡餵狼。」一劍正要向那村女右臂砍下,武烈道︰「且慢!」伸手在女児手腕上一帶,將她這一劍引開了,對那村女道︰「你説出指使你的人來,那便給你一個痛快的。否則的話,哼哼!我瞧你斷了四肢,在雪地裡滾來滾去,也不大好受吧。」那村女年紀輕輕,却是極具膽色,微笑道︰「你一定要我説,我實在無法再瞞了。武青嬰姑娘要嫁給一個男子,另外一個美貌姑娘也要嫁這人,兩女不相下,那個美貌姑娘便指使我去殺了朱九眞。這件事我本要嚴守祕密\dash{}」她還待説下去,武青嬰已氣得花容失色,手腕向前直送,一劍便往那村女心窩中刺去。

那村女鑑貌辨色,已將武青嬰和衛璧、朱九眞三人之間的{\upstsl{尷}}尬情形,猜了個八九不離,她如此激怒武青嬰,正是要她爽爽快快的將自己刺死,但見青光一閃,長劍已到心口,突然間什麼東西無聲無息的飛來,無聲無息的在那劍上一撞,武青嬰虎口震裂,呼的一聲響,那劍直飛出去,這股力道大得異乎尋常,長劍竟是飛出二十餘丈之外,方才落地。黑暗中誰也没看見武青嬰的兵刃如何脱手,但這劍以如此勁道飛了出去,便是要她自己用力投擲,也決計無法做到,看來那村女暗中已到了強援。六個人一驚之下,各自退了幾步,回頭察看。這一帶地勢開闊,並無山石叢林可以藏身,一眼望出去,半個人影也無,六人面面相覷,都是驚疑不定。武烈低聲道︰「青児,怎麼啦?」武青嬰道︰「似乎是什麼極厲害的暗器,將我的劍児震飛了。」武烈游目四顧,確是不見有人,哼了一聲道︰「便是她弄鬼。」心中暗暗奇怪︰「她明明已中了一陽指力,怎地尚能使力震飛青児的長劍?這丫頭的武功當眞邪門。」踏步上前,一掌往那村女左肩拍去,這一掌運勁雄猛,要拍碎她的肩骨,使她武功全失,再由女児來稱心擺弄於她。

掌心離她肩頭約有七八寸,眼看她便要肩頭粉碎,驀地裡那村女左掌翻將上來,雙掌相交,武烈胸口一熱,但覺對方的掌力猶似狂風怒潮,竟非人力所爲,「啊」的一聲大叫,身子已然飛起,砰的一響,摔出三丈以外。總算他武功深厚,背脊一著地立即躍起,但胸腹間熱血翻湧,頭暈眼花,身子剛站直,待欲調勻氣息,晃了一晃,終於又俯身跌倒。衛璧和武青嬰大驚,急忙搶上扶起。忽聽得何太沖道︰「讓他多躺一會!」武青嬰回過頭來,怒道︰「你説什麼!」心想︰「爹爹受了敵人暗算,你却乘火打劫,反來譏嘲。」

何太沖道︰「你血氣翻湧,靜臥從容。」衛璧登時省悟,道︰「是!」輕輕將師父放回地下。何太沖和班淑嫻夫婦對望一眼,心下大是詫異,他們都和那村女動過手,覺得她招術精妙,果有過人之處,然以年齡所限,内力未臻上乘,可是適纔和武烈對這一掌,明明是以世所罕有的内力將他震倒,實是令人大惑不解。

他們感到奇怪,那村女心中更是詫異萬分。她被武烈的一陽指點倒後,倒在張無忌懷中,動彈不得,眼看武青嬰揮劍刺來,却不知突然從那裡飛來一物,將她長劍震脱,跟著自己小腿上足三里和陽陵泉兩處穴道中,突有一股火炭一般的熱氣透入,在伏兔和風市兩穴上一衝。登時將被點的穴道解開了。她身一震,低頭看時,只見張無忌雙手握住自己兩脚足踝,那熱氣源源不絶的從懸鐘穴中湧入體内。這當児變化快極,未及細想,武烈的一掌已拍了下來。她隨手抵禦,原來拚著手腕折斷,勝於肩頭被他拍得粉碎,那知雙掌相交之下,武烈竟給自己一掌揮出數丈之外。她一怔之下,心道︰「難道這醜八怪、鄕巴佬,竟是武功深不可測的大高手?」

何太沖心存忌憚,不願再和她比拚掌力,拔劍出鞘,説道︰「我領教領教姑娘的劍法。」那村女笑道︰「我没劍啊!」何太沖左足一挑,勾起武青嬰掉在地下的長劍,柄前刃後,平平的向那村女當胸前飛去。那村女伸手一抄,接在手裡。何太沖是一派掌門,不肯佔小輩的便宜,説道︰「你進招吧,我讓你三招再還手!」那村女一劍刺出,逕取中宮。何太沖怒哼一聲,低聲道︰「小輩無禮!」舉劍一封。

却聽得喀喇一響,雙劍一齊震斷,何太沖臉色大變,身形晃處,已自退開半丈。那村女暗叫︰「可惜,可惜!」原來張無忌將九陽神功傳到體内,但她不會發揮神功的威力,結果雙劍齊斷,若能運力攻敵,那麼斷的將只何太沖的兵刃,她手中長劍却可完好無恙。

班淑嫻大奇,低聲道︰「怎麼啦?」何太沖手臂兀自酸麻,苦笑道︰「邪門!」班淑嫻拔出長劍,寒著臉道︰「我再領教。」那村女雙手一攤,意示無劍可用。班淑嫻指著掉在二十餘丈之外衛璧的那把長劍,道︰「去撿來用啊!」那村女知道只要一離開張無忌之手,自己那裡還有這般厲害的功力,不敢去拾長劍,只得揚一揚手中半截斷劍,笑道︰「就是這把斷劍,也可以了!」

班淑嫻大怒,心想︰「死丫頭如此托大,輕視於我。」她却不似何太沖,處保持前輩高人身份,長劍迴處,疾刺那村女的頭頸。那村女舉斷劍擋架,班淑嫻劍法輕靈之極,早已改削她的左肩。那村女忙翻斷劍相護,班淑嫻又已斜刺她右脅。接連八劍,勢若飄風,始終不與那村女的斷劍相碰。原來班淑嫻見她適纔出手,雖然没瞧出眞正原因,但已猜到她内力奇大,是以打定了主意,儘量發揮自己劍法所長,不令對方有施展内力之機。

果然這麼一來,那村女左支右絀,登時迭遇兇險,本來單以劍法而論,那村女雖然不及班淑嫻,但要支持百餘招,也勉強可以對付得了,只是她手中只有半截斷劍,雙足又不敢移動,變成了只守不攻,劍法上大大打了個折扣,又拆數招,班淑嫻劍尖閃處,嗤的一聲,在那村女左臂上劃了一道口子。崑崙派的劍法是非同小可,這一劍刺中敵人,却不容敵人有半分喘息之機,跟住著著進逼,只聽村女「啊」的一聲,肩頭又中了一劍。那村女大聲叫道︰「喂,你再不幫忙,眼睜睜瞧著我給人欺侮麼?」班淑嫻退後兩步,橫劍當胸,四下一看,却見有人,但見她劍尖上抖出朶朶寒梅,又向那村女攻去。

那村女疾舞斷劍,連擋三劍,對方劍招來得奇快,她却也擋得迅捷無倫,這當児眼明手快,當眞是招招間不容髮。班淑嫻讚道︰「死丫頭,手下倒快!」那村女半句也不肯吃虧,回罵道︰「死婆娘,你手下也不慢啊。」不料班淑嫻是劍術上的大名家,數十年的修爲,口中説話,手下絲毫没有閒著。那村女終究不過十七八歳年紀,雖然得遇名師,但豈能學得到班淑嫻好整以暇的風範?這一説話微微分心,但覺手腕上一疼,半截斷劍已然脱手飛出。

那村女「啊」的一聲驚呼,班淑嫻第二劍已刺向她的脅下。丁敏君一直在袖手觀戰,這時看出便宜,不及拔劍,一一招「推窗望月」,雙掌便向那村女背上擊去,同時武青嬰也縱身而起,飛腿直踢那村女的右腿。那村女只嚇得一顆心幾欲從腔子中跳了出來,但覺全身炙熱,如墜火窖,伸指在班淑嫻的長劍上一彈,便在此時,背心中掌,腰間被踢。却聽得「啊喲」「哎{\upstsl{唷}}」兩聲慘叫,丁敏君和武青嬰一齊向後摔出,班淑嫻手中也只剩下了半截斷劍。原來張無忌眼見情勢危及,霎時間將全身眞氣,盡數運入那村女的體内,他所修習的九陽神功已有二成功力,威力大是不小,那村女伸指一彈之下,班淑嫻的長劍登時折斷,丁敏君雙手腕骨和武青嬰右足趾骨節一齊震碎。何太沖、武烈衛璧三人看到九陽神功顯示威力的這副可怕聲勢,無不目瞪口呆,一時怔在當地,做聲不得。

班淑嫻將半截斷劍往地下一抛,恨恨的道︰「去吧,丟人現眼還不彀麼?」向丈夫怒目而視,一肚皮怨氣,盡數要發洩在他身上。何太沖道︰「是!」兩人併肩馳去,片刻之間,已奔得老遠,崑崙派輕功之佳妙,確是武林中一絶。至於班淑嫻回家後如何整治何太沖出氣,是罰跪頂劍,或是另有怪招,那也不必細表。衛璧一手扶著師父,一手扶了師妹,慢慢走開。他三人極怕那村女乘勝追擊,可是又不能如何太沖夫婦這般飛馳遠去,每走一步,便擔一份心事。丁敏君雙手腕骨斷折,足腿却是無傷,咬緊牙関,獨自離去。

那村女極是得意,哈哈一笑,説道︰「醜八怪!你\dash{}」突然間一口氣接不上來,暈了過去。原來張無忌助她驅退強敵,眼見六個對頭分别離去,當即縮手,放脱她的足踝。那村女充沛體内的一股九陽眞氣驀地裡洩去,她便如全身虛脱,四肢百骸,再無分毫力氣。張無忌一驚之下,便即領會,雙手拇指輕輕按住她眉頭盡處的「絲竹穴」,微運神功,那村女這纔慢慢醒轉。她睜開眼來,見自己躺在張無忌的懷處,他正笑嘻嘻的望著自己,不覺大羞,一躍而起,突然伸手抓住他的左耳,用力一扭,罵道︰「醜八怪,你騙人!你有一身厲害武功,怎麼不跟我説?」張無忌痛叫︰「哎喲!你幹什麼?」那村女哈哈笑道︰「誰叫你騙人!」張無忌道︰「我幾時騙你了,你没跟我説你會武功,我也没跟你説我會武功。」那村女道︰「好,我便饒了你這一遭。適纔多承你助我一臂之力,將功折罪,我也不來追究了。你的腿能走路了嗎?」張無忌道︰「還不能。」那村女嘆道︰「總算好心有好報,若不是我記掛著你,要再來瞧你一次,你也不能救我。」她頓了一頓,又道︰「早知你本事比我強得多,我也不用替你去殺朱九眞那鬼丫頭了。」張無忌臉一沉,道︰「我本來没叫你去殺她啊。」那村女道︰「啊喲,啊喲!原來你心中還是放不下這個美麗的姑娘,倒是我不好,害了你的意中人。」張無忌道︰「朱姑娘不是我的意中人,她再美麗,也不跟我相干。」那村女奇道︰「咦!這可奇了,那麼她害得你這樣,我殺了她給你出氣,難道不好嗎?」

張無忌淡淡的道︰「害過我的人很多,若是一個個都殺去了出氣,也殺不盡這許多。何況,有些人存心害我,在我看來,他們也是很可憐的,好比這個朱姑娘,她整日價提心吊膽,生怕她表哥不和她好,擔心他娶了武姑娘爲妻。像她這樣,又有什麼快活?」那村女臉一沉,怒道︰「你是譏刺我麼?」張無忌呆了一呆,没想到説著朱九眞時,無意中觸了眼前這位姑娘之忌,道︰「不,不。我是説各人有各人的不幸。别人對你不起,你就殺了他,那很不好。」那村女冷笑道︰「你學武功如果不是爲了殺人,那學來做什麼?」張無忌沉吟道︰「咱們學好了武功,壞人如想加害,咱們便可抵擋了。」那村女道︰「佩服,佩服!原來你是個正人君子,大大的好人!」

張無忌低了頭瞧著她,總覺得這位姑娘的舉止神情,自己是説不出的親切,説不出的熟悉。那村女顎下一揚,道︰「你瞧什麼?」張無忌道︰「我媽媽常笑我爸爸是濫好人,是軟心腸的可憐書生。她説話時的神氣,就像你這時候一樣。」那村女臉上一紅,斥道︰「{\upstsl{呸}}!又來佔我便宜,説我像你媽媽,你自己就像你爸爸了!」她雖出言斥責,眼光中却孕含笑意。張無忌急道︰「老天爺在上,我若是有心佔你便宜,教我天誅地滅。」那村女笑道︰「口頭上佔一句便宜,没什麼大不了,又用得著賭咒發誓?」剛説到此處,忽聽東北角上有清嘯一聲,嘯聲清脆悠長,是個女子。跟著近處有人作嘯相應,那正是尚未走遠的丁敏君。那村女臉色微變,低聲道︰「峨嵋派又有人來了。」兩人聽那遠處傳來的清嘯之聲,明亮凝聚,距離雖比丁敏君爲遠,但聽在耳中,却是清楚得多,顯然那人功力遠較丁敏君深厚。

丁敏君聽到嘯聲後,便停步不走。張無忌和那村女向東北方眺望,這時天已黎明,只見一個綠色的人形,在雪地裡輕輕飄飄的走來,行到了丁敏加身畔,張無忌已看到原來是個身穿蔥綠衣衫的女子。她和丁敏君説了幾句話,向張無忌和那村女看了一眼,便即走了近來。只見她衣衫飄動,脚步極是輕盈,出步甚小,但頃刻間便到了離兩人四五丈之處。只見她清麗秀雅,容色極美,不過是十七八歳年紀。張無忌心下頗爲詫異,暗想聽她嘯聲、看她身法,料想必是個比丁敏君年長得多的女子。那知她似乎比自己還小。

只見這女郎腰間懸著一柄短劍,却不拔取兵刃,空手走近。丁敏君出聲警告︰「周師妹,這鬼丫頭功夫邪門得緊。」那女郎點點頭,斯斯文文的道︰「兩位尊姓大名?因何傷我師姊?」自她走近之後,張無忌一直覺得她好生面熟,待得聽到她説話,登時想起︰「原來她便是在漢水中相逢的周芷若姑娘。太師父擕她上武當山去,如何却投入峨嵋門下?」胸口一熱,便想探問張三丰的近況,但轉念想到︰「張無忌已然死了,我這時是鄕巴姥、醜八怪、曾阿牛。只要我少有不忍,日後便是無窮無盡的禍患。不管是在誰的面前,我都不能洩露了自己身份,以免害及義父,使爸媽白白的冤死於九泉之下。」一想到已死的父母,獨處荒島的義父,便有天大的原因,他也不肯再以本來身份示人。

那村女冷冷一笑,説道︰「令師姊一招『推窗望月』,雙掌擊我背心,自己折了手腕,難道也怪得我麼?你倒問問令師姊,我可有向她發過一招半式?」周芷若轉眼瞧著丁敏君,意存詢問。丁敏君怒道︰「你帶這兩人去見師父,請她老人家發落便是。」周芷若道︰「倘若這兩位並未存心得罪師姐,以小妹之見,不如一笑而罷,化敵爲友。」丁敏君大怒,喝道︰「什麼?你反而相助於外人?」

\chapter{千蛛絶戸}

張無忌一見丁敏君這副神色,想起那一年晩上彭瑩玉和尚在林中受人圍攻,紀曉芙因而和丁敏君反臉,今日舊事重演,丁敏君又來逼迫這個小師妹,不禁暗暗爲周芷若擔心。不料周芷若對丁敏君極是尊敬,躬身道︰「小妹聽由師姐吩咐,不敢有違。」丁敏君道︰「好,你去將這丫頭拿下,把她雙手也打折了。」周芷若道︰「是,請師姐給小妹掠陣。」轉身向那村女道︰「小妹無禮,想請教姐姐的高招。」那村女冷笑道︰「那裡來的許多囉唆!」身形一晃,快如閃電般連擊三掌。

周芷若斜身搶進,左掌擒拿,以攻爲守,招數極爲巧妙。張無忌内功雖強,武術上的招數却未融會貫通,但見周芷若和那村女都是以快打快,周芷若的峨嵋綿掌輕靈迅捷,那村女的掌却是古怪奇異。張無忌看得又是佩服,又是関懷,自己也不知盼望誰勝,只望兩個都不要受傷。兩女拆了二十餘招,已是各遇兇險,猛聽得那村女叫聲︰「著!」一掌斬中了周芷若的肩頭。跟著嗤的一響,周芷若反手扯下了那村女的半幅衣袖。兩人各自躍開,臉上微紅。那村女喝道︰「好擒拿手!」待欲搶步又上,只見周芷若眉頭深皺,按著心口,身子晃了兩晃,{\upstsl{墬}}{\upstsl{墬}}欲倒。張無忌忍不住叫道︰「你\dash{}你\dash{}」関切之情,見於顏色。

周芷若見這個長鬚長髮的男子居然對自己大是関心,暗自詫異。丁敏君道︰「師妹,你怎麼啦?」周芷若左手搭住師姐的肩膀,搖了搖頭。丁敏君吃過那村女的苦頭,知道她的厲害,只是師父常自稱許這個小師妹,説她悟性奇高,進步神速,本派將來發揚光大,都要著落在她身上,丁敏君心下不服,是以叫她試上一試。見周芷若竟能和那村女拆上二十餘招方始敗落。已遠遠勝過自己,心中不免頗爲嫉忌,待得覺到她搭在自己肩上的那雙手全無力氣,才知她受傷不輕,生怕那村女上前追擊,忙道︰「咱們走吧!」兩人擕扶著向東北而去。

那村女瞧著張無忌臉上神色,冷笑道︰「醜八怪,見了美貌姑娘便魂飛天外。」張無忌欲待解釋,但想︰「若不自露身世,這件事便説不清楚,還不如不説。」便道︰「她美不美,関我什麼事?我是関心你,怕你受傷啊。」那村女道︰「你這話是眞是假?」無忌想︰「我本是對兩位姑娘都関心。」説道︰「我騙你作甚?想不到峨嵋派中這樣二個年輕姑娘,武藝竟恁地了得。」那村女道︰「厲害,厲害!」

張無忌望著周芷若的背影,見她來時輕盈,去時蹣跚,想起當年漢水舟中她對自己餵飲餵食,贈巾抹泪之德,暗暗禱祝,但願她受傷不重,那村女忽然冷笑道︰「你不用擔心,她壓根児就没受傷。我説她厲害,不是説她武功,是説她小小年紀,心計却如此厲害。」張無忌奇道︰「她没有受傷?」那村女道︰「不錯!我一掌斬中她肩頭,她肩上生出内力,將我手掌彈開,原她已練過峨嵋九陽功,倒震得我手臂微微酸麻。她那裡會受什麼内傷?」張無忌大喜,心想︰「原來滅絶師太對她青眼有加,竟將峨嵋派鎭派之寶的峨嵋九陽功傳了給她?」那村女忽地翻過手背,重重打了張無忌一個耳光,這一下突如其來,無忌絲毫没加防備,半邉頰登時紅腫,怒道︰「你\dash{}你幹什麼?」

那村女恨恨的道︰「見了人家閨女生得好看,你靈魂児也飛上天啦。我説她没受傷,要你樂得這個樣子的幹什麼?」張無忌道︰「我就是代她喜歡,跟你有什麼相干?」那村女又是一掌劈了過來,這一次張無忌却頭一低,讓了開去,那村女大怒道︰「你説過要娶我爲妻的。這句話説了還不上半天,便見異思遷,瞧上人家美貌姑娘了。」

張無忌道︰「你説過我不配,又説你心目中自有情郎,決計不能嫁我。」那村女道︰「不錯,可是答應了我,這一輩子要待我好,照顧我。」張無忌道︰「我説過的話自然算數。」那村女怒道︰「既是如此,你怎地見了這個美貌姑娘,便如此失魂落魄,教人瞧著好不惹氣?」張無忌笑道︰「我又没有失魂落魄。」那村女道︰「我不許你喜歡她。不許你想她。」張無忌道︰「我也没説喜歡她,但你爲什麼心中又牽記旁人,一直念念不忘呢?」那村女道︰「我識得那人在先啊。如果我先識得你,就一生一世只對你一人好,再不會去愛旁人。這叫做『從一而終』。一個人要是三心兩意,便是天也不容。」張無忌心想︰「我相識周姑娘,遠在識得你之前。」但這句話不便出口,便道︰「要是你只對我一人好,我也只待你一人好。要是你心中想著旁人,我也去想旁人。」

那村女沉吟半晌,數度欲言又止,突然間眼中珠泪欲滴,轉過頭去,乘張無忌不覺,伸袖拭了拭眼泪。無忌心下不忍,輕輕握住了她的手,説道︰「咱們没來由的説這些幹什麼。再過得幾天,我的腿傷便全好了。咱們到處去玩賞風景,豈不甚美?」那村女回過頭來,愁容滿臉,道︰「阿牛哥哥,我求你一件事,你不要生氣。」張無忌道︰「什麼事啊?但教我力之所及,總會給你做到。」那村女道︰「你答應我不生氣,我纔跟你説。」張無忌道︰「不生氣就是。」那村女躊躇了一下,道︰「你口中説不生氣,心裡也不許生氣纔成。」張無忌道︰「好,我心裡也不生氣。」

那村女反握著無忌的手,説道︰「阿牛哥哥,我從中原萬里迢迢的來到西域,爲的就是找他。以前還聽到一點蹤跡,但到了這裡,却如石沉大海,再也問不到他的消息了,你腿好之後,幫我去找到他,然後我再陪你去遊山玩水,好不好?」張無忌忍不住心中不快,哼了一聲,那村女道︰「你答應我不生氣的,這不是生氣了麼?」張無忌道︰「好,我幫你去找他。」

那村女大喜,道︰「阿牛哥,你眞好。」望著遠處天地相接的那一線,心搖神馳。輕聲道︰「咱們找到了他,他想著我找了他這麼久,就會不惱我了。他説什麼,我就做什麼,一切全聽他的話。」張無忌道︰「你這個情郎,到底有什麼好,教你如此念念不忘?」那村女微笑道︰「他有什麼好,我怎麼説得上來?阿牛哥,你説咱們能找到他麼?他見了我還會打我罵我麼?」無忌見她如此痴情,不忍掃她的興,低聲吟道︰「但教心似金鈿堅,天上人間會相見。」那村女櫻口微動,眼波欲流,也低聲道︰「但教心似金鈿堅,天上人間會相見。」

張無忌心想︰「這位姑娘對她情郎痴心如斯,倘若世上也有一人如此関懷我,思念我,我這一生便再多吃些苦,也是快活。」瞧著周芷若和丁敏君並排在雪地上留下的兩行足印,心想︰「倘若丁敏君這行足印是我留下的,我得能和周姑娘並肩而行\dash{}」那村女突然叫道︰「啊喲,快走,再遲便來不及了。」張無忌從幻想中醒了過來,道︰「怎麼?」那村女道︰「那峨嵋少女不願跟我拚命,假裝受傷而去,可是那丁敏君口口聲聲説要拿我去見她師父,滅絶師太必在左近,這老賊尼極是好勝,怎能不來?」

張無忌想起滅絶師太一掌擊死紀曉芙的殘忍狠辣,不禁心下猶有餘悸,驚道︰「這老尼姑厲害得緊,咱們可敵她不過。」那村女道︰「你見過她麼?」張無忌道︰「見是没見過,但峨嵋掌門,豈同等閒?」那村女眉頭微皺,便取下柴堆中的硬柴,用樹皮捲成繩子,紮好了一個雪{\upstsl{枆}},叫張無忌雙腿伸直,躺在雪橇之上,拉了他飛奔。

她提氣疾奔,迅捷無倫。張無忌回頭瞧她,但見她身形微晃,好似曉風中一朶荷葉,極是美妙,脚下也不跨大步,拖著雪橇,一陣風般掠過雪地。那村女奔馳不停,趕了三四十里地,張無忌仍不聽見氣息粗重之聲,好生佩服她輕功佳妙,但也頗爲過意不去,説道︰「喂,好歇歇啦!」那村女笑道︰「什麼喂不喂的,我没名字麼?」張無忌道︰「你不肯説,我有什麼法子?你要我叫你『醜姑娘』,可是我覺得你好看啊。」那村女嗤的一笑,一口氣洩了,便停了脚步,掠了掠頭髮,説道︰「好吧,跟你説也不打緊,我叫蛛児。」張無忌道︰「珠児,珠児,珍珠寶貝児。」那村女道︰「{\upstsl{呸}}!不是珍珠的珠,是毒蜘蛛的蛛。」張無忌一怔,心想︰「那有用這個『蛛』字來作名字的?」

蛛児道︰「我就是這個名字,你倘若害怕,那便不用叫了。」張無忌道︰「是你爸爸給你取的麼?」蛛児道︰「哼,若是爸爸取的,你想我還肯要麼?是媽取的。她教我練『千蛛絶戸手』,説就用這個名字。」

張無忌聽到「千蛛絶戸手」五個字,不由自由的心中一寒。蛛児道︰「我從小練起,還差著好多呢。等得我練成了,也不用怕滅絶這賤尼啦。你要不要瞧瞧。」説著便從懷中取出一個黃澄澄的金盒來,打開盒蓋,只見盒中兩隻拇指大小的蜘蛛,蠕蠕而動。兩隻蜘蛛背上花紋斑爛,極是鮮明奪目。尋常蜘蛛都是八隻脚,這兩隻花蜘蛛却各有十二隻脚。張無忌一看之下,驀地想起王難姑所著的「毒經」來,那經中言道︰「蜘蛛身有彩斑,乃劇毒之物,倘若身有十足,更是奇毒無比,螯人後無藥可救。」眼前這對蜘蛛又多了兩足,連「毒經」也未載及,想必比那十足蜘蛛更是厲害。

蛛児見無忌臉現戒懼之色,笑道︰「你倒是個行家,知道我這寶貝蛛児的好處。你等一等。」説著飛身上了一棵大樹,眺望周遭地勢,躍回地下,道︰「咱們且走一程,慢慢児再説蜘蛛的事。」拉著雪橇,又奔出七八里地,來到一處山谷邉上,將無忌扶下雪橇,然後搬了幾塊石頭,放在撬中,拉著急奔,衝向山谷。她奔到山崖邉上,猛地收步,那雪撬却帶著石塊,轟隆隆的滾下深谷,聲音良久不絶。張無忌回望來路,只見雪地中柴撬所留下的兩行軌跡,遠遠的蜿蜒而來,至谷方絶,心想︰「這位姑娘心思倒也細密。滅絶師太若是順著軌跡找來,只道咱們已摔入雪谷之中,跌得屍骨無存了。」

蛛児蹲下身來,道︰「你伏在我背上!」無忌道︰「你背著我走嗎?那太累了。」蛛児白了他一眼,道︰「我累不累,自己不知道麼?」無忌不敢多説,便伏在她的肩上,輕輕摟住她的頭頸。蛛児笑道︰「你怕握死我麼?輕手輕脚的,教人頭頸裡癢得要命。」張無忌原是個坦誠率直之人,見蛛児對自己一無猜嫌,心下甚喜,手上便摟得緊了些,蛛児突然間一躍而起,帶著無忌飛身上樹。

這一排樹木一直向西延伸,蛛児從一株樹上躍到另一株樹上,她身材纖小,張無忌甚是高大,但她步法輕捷,竟是毫不累贅,過了七八十棵樹,躍到一座山壁之旁,便跳下地來,輕輕將無忌放在地上,笑道︰「咱們在這児搭個牛棚,倒是不錯。」張無忌奇道︰「那不用了,再過得四五天,我斷骨的接續便硬朗啦,其實這時勉強要走,也對付得了。」蛛児道︰「哼!勉強走,已經是個醜八怪,牛腿再跛了,很好看麼?」説著便折下一條樹枝,掃去山石旁的積雪。

張無忌聽著「牛腿再跛了,很好看麼?」這句話,驀地裡體會到她言語中的関切之意,不由得心中一動,只聽蛛児口中輕輕哼著歌児般的小曲,攀折樹枝,在兩塊大石之間搭了一個上蓋,竟成了一間足可容身的小屋,茅頂石牆,頗有天然雅趣。蛛児搭好小屋,却又抱起地下一大塊的雪團,堆在小屋頂上,忙了半天,直至外邉瞧不出半點痕跡,方始罷手。她取出手帕,擦了擦臉上的汗珠,道︰「你等在這裡,我去找些吃的來。」張無忌道︰「我也不怎麼餓,你太累啦,歇一會児再去吧。」蛛児道︰「你要待我好,要眞的待我好,嘴裡説得甜甜的,又有什麼用?」説著展開輕身功夫,鑽入了樹林。

張無忌在山石之上,想起蛛児語音嬌柔,舉止輕盈,無一不是個絶色美女的風範,可就是一張臉蛋児却生得這麼醜陋,又想起母親臨終時説過的話來︰「越是美麗的女子,越會騙人,你越是要小心提防。」這蛛児相貌不美,待自己又是極好,有心和她終身相守,可是她心中另有情郎,全没把自己放在意下。

他胡思亂想,心念如潮,不久蛛児已提了兩隻雪雞回來,生火一烤,味美絶倫。張無忌將一隻雪雞吃得乾乾淨淨,猶未饜足。蛛児抿著嘴笑了笑,將預先留下的兩條雞腿,又擲了給他。那是她在自己那只雪雞上省下來的,原是雪雞身上的精華。張無忌欲待推辭,蛛児怒道︰「你想吃便吃,誰對我假心假意,言不由衷,我用刀子在他身上刺三個透明窟窿。」張無忌不敢多説,便把兩條雞腿吃之。他滿嘴油膩,從地下抓起一塊雪來,擦了擦臉,伸衣袖抹去。

蛛児偶一回頭,看到張無忌用雪塊擦乾淨了臉,不禁怔住了,呆呆的望著他。無忌被她瞧得不好意思,問道︰「怎麼啦?」蛛児道︰「你幾歳啦?」張無忌道︰「剛好二十歳。」蛛児道︰「{\upstsl{嗯}},比我大兩歳。爲什麼留了這麼長的鬍子?」張無忌笑道︰「我一直獨個児在深山荒谷中住,從不見人,就没有想到要剃鬚。」蛛児從身旁取出一把金柄小刀來,按著他臉,慢慢將鬍子剃去了。張無忌只覺刀鋒極是鋭利,所到之處,髭鬚紛落,她手掌手指却是柔膩嬌嫩,摸在他面頰上,忍不住怦然心動。

那小刀漸漸剃到他頸中,蛛児笑道︰「我稍一用力,在你喉頭一割,立時一命嗚呼。你怕不怕?」張無忌道︰「死在姑娘玉手之下,做鬼也是快活。」蛛児反過刀子,用刀背在他咽喉上用力一斬,喝道︰「叫你做個快活鬼!」張無忌嚇了一跳,但他來勢太快,刀子又近,待得驚覺,一刀已斬下,半點反抗之力也無,隨身才知她用的只是刀背。蛛児格格笑道︰「快活麼?」張無忌笑著點了點頭。他本來爲人樸實,但見了蛛児,不知怎的,心中無拘無束,似乎跟他青梅竹馬,自幼児一塊長大一般,説不出的逍遙自在,忍不住要説幾句笑話。

蛛児替他剃淨鬍鬚,向他呆望半晌,突然長長的嘆了口氣。張無忌道︰「怎麼啦?」蛛児不答,又替他割短頭髮,梳個髻児,用樹枝削了根釵子,插在他髮髻之中,張無忌這麼一打扮,雖然衣服仍是襤褸不堪,却已神采煥發,英氣勃勃,變成一個極俊美的少年。蛛児又嘆了口氣,道︰「阿牛哥,眞想不到,原來你是這樣好看。」無忌心想她是爲自身的醜陋難過,便道︰「天地間極美的物事之中,往往含有極醜。孔雀羽毛華美,其膽却是劇毒,仙鶴丹頂殷紅,何等好看,那知却是最厲害的毒藥。諸凡蛇豸昆蟲,也都是越美的越具毒性。一個人相貌俊美有什麼好,要心她良善那纔好啊。」蛛児冷笑道︰「心地良善有什麼好,你倒説説看。」

張無忌被她這麼一問,一時倒答不上來,怔了一怔纔道︰「心地良善,便不會去害人。」蛛児道︰「不去害人却又有什麼好?」張無忌道︰「你不去害人,自己心裡就平安喜樂,處之泰然。」蛛児道︰「我不害人便不痛快,要害得旁人慘不可言,自己心裡纔會平安喜樂,纔會處之泰然。」無忌搖頭道︰「你強辭奪理。」蛛児冷笑道︰「我若不是爲了害人,練這千蛛絶戸手幹什麼?自己受這無窮無盡的痛苦熬煎,難道是貪好玩麼?」説著取出黃金小盒,打開盒蓋,將雙手的兩根食指伸進盒中。

盒中的一對花蛛慢慢爬近,咬住了她的指頭,只見她深深吸一氣,雙臂輕微顫抖,潛運内力和蛛毒相抗。花蛛吸取她手指上的血液爲食,但蛛児手指上血脈運轉,也帶了花蛛體内毒液,回到自己血中。張無忌見她滿臉莊嚴肅穆之容,同時眉心和兩旁太陽穴上,淡淡的罩上了一層黑氣,咬緊牙関,竭力忍受痛楚。再過一會,又見她鼻尖上滲出細細的一粒汗珠。她這功夫練了幾有半個時辰,雙蛛直到吸飽了血,肚子漲得和圓球相似,這纔跌在盒中,沉沉睡去。

蛛児又運功良久,臉上黑氣漸退,重現血色,一口氣噴了出來,張無忌聞著,只覺甜香無倫,但頭腦却被這陣奇怪的香氣衝得發暈,似乎氣息中也含了劇毒。蛛児睜開眼來,微微一笑。張無忌道︰「要練到怎樣,纔算大功告成?」蛛児道︰「要每隻花蛛的身子從花轉黑再黑轉白,那便要去淨毒性而死,蜘蛛體中的毒液都到了我手指之中。至少要練過一千隻花蛛,纔算是小成。眞要功夫深啊,那麼五千隻一萬隻也不嫌多。」張無忌聽她説著,心中不禁發毛,道︰「那裡來這許多花蛛?」蛛児道︰「一面得自己養,牠們會生小蜘蛛,一面須得到産地去捉。」張無忌嘆道︰「天下武功甚多,何必非練這門毒功不可。這些花蛛劇毒無比,吸入體内,雖然你有抵禦之法,日子久了,終究没有好處。」蛛児冷笑道︰「天下武功雖多,可是有那一門功夫,能及得上這千蛛絶戸手的厲害?你别自恃内功深厚,要是我這門功夫練成了,你未必能擋得住我手指的一戳。」説著凝氣於指,隨手在身旁的一株樹上戳了一下,她功力未到,只戳入半寸來深。

張無忌又問道︰「怎地你媽媽教你練這功夫?她自己練成了麼?」蛛児聽他説到自己媽媽,眼中突然射出狠毒的光芒來,竟似一頭要撲上來咬人的野獸,恨恨的道︰「練這千蛛絶戸手,練到八百隻花蛛以上,身體内毒性積得多了,容貌便會變形,待得千蛛練成,更會奇醜難當。我媽本已練到將近五百隻,偏生遇上我爹爹,怕自己容貌變醜,我爹爹不喜,硬生生將畢身的功夫散了,成爲一個手無縛雞之力的平庸女子。她容貌雖然好看,但受二娘和我哥哥姊姊的欺侮凌辱,竟無半點還手的本事,到頭來還是送了自己性命。哼,相貌好看有什麼用?我媽是個極美麗極秀雅的女子,只因年長無子,我爹爹還是去另娶妾侍\dash{}」

張無忌的眼光在她臉上一掠而過,低聲道︰「原來\dash{}你是爲了練功夫\dash{}」蛛児道︰「不錯,我是爲了練功夫,纔將一張臉毒成這樣。哼,那個負心不理我,等我練成千蛛手之後,找到了他,他若無旁的女子,那便罷了\dash{}」張無忌道︰「你並未和他成婚,也無白頭之約,不過是\dash{}不過是\dash{}」蛛児道︰「爽爽快快的説好啦,怕什麼?你要説我不過是片面的單相思,是不是?單相思便怎樣?我既愛上了他,便不許他心中另有别的女子。他負心薄倖,教他{\upstsl{嚐}}{\upstsl{嚐}}我這『千蛛絶戸手』的滋味。」

張無忌微微一笑,也不跟她再行辯説,心想這蛛児脾氣很是特别,好起來很好,兇野起來却是極端的蠻不講理,又想起太師父和大師伯、二師伯們常説的武林中正邪之别,看來她所練的「千蛛絶戸手」必是極歹毒的邪派功夫,她母親也必是妖邪一流,想到此處,不由得對她多了幾分戒懼之意。

蛛児却並未察覺他心情異樣,在小屋裡裡外外奔進奔出,折了許多野花佈置起來。張無忌見她將這間小小的屋子整治得頗倶雅趣,可見愛美出自天性,然而一副容貌却毒成這個樣子,便道︰「蛛児,我腿好之後,去採些藥來,設法治好你臉上的毒腫。」蛛児聽了這幾句話,臉上突現恐懼之色,説道︰「不\dash{}不\dash{}不要,我熬了多少痛苦纔到今日地步,你要散去我的千蛛毒功麼?」張無忌道︰「咱們或能想到一個法子,功夫不散,却能消去你臉上的毒腫。」蛛児道︰「不成的,要是有這法子,我媽媽的祖傳的功夫,焉能不知?天下除非是蝶谷醫仙胡青牛,方有這等驚人的本事,可是他\dash{}他早已死去多年了。」張無忌道︰「你知道胡青牛?」蛛児瞪了他一眼,道︰「怎麼啦?什麼事奇怪?蝶谷醫仙名滿天下,誰都知道。」説著又嘆了口氣,道︰「便是他還活著,這人號稱『見死不救』,又有什麼用?」

張無忌心想︰「這位姑娘對我很好,我終須有以報答。她不知蝶谷醫仙的一身本事,已盡數傳了給我,這時我且不説,日後我想到了治她臉上毒腫之法,也好讓她大大的驚喜一場。」説話間天色已黑,兩人便在這小屋中倚著山石睡了。睡到半夜,張無忌睡夢中忽聽到一兩下低泣之聲,登時醒轉,定了定神,原來蛛児正在哭泣。無忌坐直身子,伸手在她肩頭輕輕拍了兩下,安慰她道︰「蛛児,别傷心。」

那知他柔聲説了這兩句話,蛛児更是難以抑止,伏在他的肩頭,放聲大哭起來。張無忌道︰「蛛児,什麼事?你想起了媽媽,是不是?」蛛児點了點頭,抽抽噎噎的道︰「媽媽死了!我一個人孤零零的,誰也不喜歡我,誰也不同我好。」張無忌拉起衣襟,緩緩替她擦去眼泪,道︰「我喜歡你,我會待你好。」蛛児道︰「我不要你待我好。我心中喜歡的人,他不睬我,他打我、罵我,還要咬我。」張無忌顫聲道︰「你忘了這個薄倖郎吧。我娶你爲妻,我一生好好的待你。」

蛛児大聲道︰「不!不!,我不忘記他。你再叫我忘了他,我永遠不睬你了。」張無忌大是羞慚,幸好在黑暗之中,蛛児没瞧見他滿臉通紅的{\upstsl{尷}}尬模樣。好一會児,誰都没有説話,蛛児道︰「阿牛哥,你惱了我麼?」張無忌道︰「我没惱你,我是生自己的氣,不該跟你説這些話。」蛛児忙道︰「不,不!你説願意娶我爲妻,一生要好好的待我,我很喜歡聽你説這些話。你再説一遍吧。」無忌怒道︰「你既忘不了那人,我還能説什麼?」蛛児伸過手去,握住了他手,柔聲道︰「阿牛哥,你别著惱,我得罪了你,是我不好。你如眞的娶了我爲妻,我會刺瞎了你的眼睛,會殺了你的。」

張無忌身子一跳,道︰「你説什麼?」蛛児道︰「你眼睛瞎了,就瞧不見我的醜樣。就不會去瞧峨嵋派那位周姑娘。倘若你還是忘不了她,我便一指戮死你,再一指戮死自己。」她説著這些可怪的言語,但聲調自然,似乎這是天經地義的道理。張無忌聽她説到「峨嵋周姑娘」,心頭怦的一跳。便在此時,只聽得遠遠傳來一個蒼老的聲音,説道︰「峨嵋周姑娘,礙著你們什麼事了?」

蛛児一驚躍起,低聲道︰「是滅絶師太!」她説得很輕,但外面那人還是聽見了,森然道︰「不錯,是滅絶師太。」

外面那人起初説這句話時,相距甚遠,但瞬息之間,便已到了小屋近旁,蛛児知道事情不妙,便要抱著張無忌設法躱避,也已不及,只得屏息不語,過了一會,只聽得外面那人冷冷的道︰「出來!還能在這裡面躱一輩子麼?」蛛児握了握張無忌的手,掀開茅草。走了出來,只見相距小屋兩丈來遠之處,站著一個白髮蕭蕭然的老尼,正是峨嵋派當今掌門人滅絶師太。數十丈外,又有十二個人分成兩排奔來。奔到近處,那十二個人在滅絶師太兩側一站,其中四個是尼姑,四個女子,四個男子,均是滅絶師太的弟子,丁敏君和周芷若也在其内。四個男弟子站在最後,原來峨嵋門下,掌門人數代相傳的都是女子,男弟子不能獲得傳最上乘的武功,地位也較女弟子爲低。

滅絶師太冷冷的向蛛児上下打量,半晌不語,張無忌見過滅絶師太掌斃紀曉芙的辣手,當下提心吊膽,伏在蛛児身後,心中打定了主意,她若是向蛛児下手,明知不敵,也要竭力和她一拚。只聽滅絶師太哼了一聲,轉頭問丁敏君道︰「就是這個小女娃麼?」丁敏君躬身道︰「是!」

猛聽得喀喇、喀喇兩響,蛛児悶哼一聲,身子已摔出三丈以外,雙手腕骨折斷,暈倒在雪地之中。張無忌眼前但見灰影一閃,滅絶師太以快捷無倫的身法欺到蛛児身旁,以快捷無倫的手法斷她腕骨,摔擲出外,又以快捷無倫的身法退回原處,顫巍巍的有如一株古樹,又詭怪又雄偉的挺立在夜風裡。這幾下出手,每一下都是乾淨利落,無忌全都瞧得清清楚楚,但實是快得不可思議,無忌竟是被這駭人的速度鎭懾住了,失却了行動之力。

滅絶師太刺人心魄的目光瞧向無忌,喝道︰「滾出來!」周芷若走上一步,稟道︰「師父,這人似乎斷了雙腿,一直行走不得。」滅絶師太道︰「做兩個雪撬,帶了他們去。」衆弟子齊聲答應。除了丁敏君手傷未愈,其餘十一名弟子快手快脚的紮成兩個雪撬,兩個女弟子抬了蛛児,兩名男弟子抬了張無忌,分别放上雪撬,雪撬跟在滅絶師太身後,向西奔馳。

張無忌凝神傾聽蛛児的動靜,不知她受傷輕重如何,對自己生死,反而置之度外。奔出十餘里,纔聽得蛛児輕輕呻吟了一聲。張無忌大聲問道︰「蛛児,傷得怎樣?受内傷没有?」蛛児道︰「她折了我雙手腕骨,内臟没傷。」張無忌道︰「你用左手手肘,去撞右手臂彎下三寸五分處,再用右手手肘,去撞左手彎下三寸五分處,便可稍減疼痛。」蛛児還没答話,滅絶師太「咦」的一聲,回過頭來,瞪了張無忌一眼,説道︰「這小子倒還精通醫理,你叫什麼名字?」張無忌道︰「在下姓曾,名阿牛。」滅絶師太道︰「你師父是誰?」張無忌道︰「我師父是鄕下小鎭上的一位無名醫生,説出來師太也不會知道。」滅絶師太哼了一聲,不再理他。

一行人直走到天明,纔歇下來分食乾糧。周芷若拿了幾個冷饅頭,分給張無忌和蛛児吃,她見無忌剃鬚束髮之後,變成個神采奕奕的美少年,不禁暗自驚異。各人歇了兩個時辰,再又趕路。如此向西急行,一直趕了三天,看來顯是有要務在身。峨嵋派弟子不論是趕路或休息,除非必要,誰都一言不發,似乎都變成啞巴一般,到底西去何事?張無忌猜不出半點端倪,這時他腿上斷骨早已痊癒,隨時可以行走,但他不動聲色,有時還假意呻吟幾聲,好讓滅絶師太不防,只待時機到來,便可救了蛛児逃走。只是一路上所經之處都是莽莽平野,逃不多遠,立時給峨嵋派追上,一時却也不便妄動。他替蛛児接上腕骨,滅絶師太冷冷的瞧著,却也没加干預。

\chapter{圍剿魔教}

又行了兩天,這日午後,來到一片大沙漠中,地下積雪早已熔盡,兩個雪撬便在沙地中滑行。正走之間,忽聽得馬蹄聲響,有乘者自西而來。滅絶師太做個手勢,衆弟子立時隱身在沙丘之後伏下,有兩人分挺短劍,對住張無忌和蛛児的後心,這意思非常明白,峨嵋派是在伏擊敵人,無忌等若是出聲示警,短劍向前一送,立時便要了他們性命。

聽那馬蹄聲奔得甚急,但相距尚遠,過了好半天方始馳到近處。馬上乘客突然見到沙地中的足跡,勒馬注視,靜虛師太拂塵一舉,十一名弟子分從埋伏處躍出,將乘者團團圍住。張無忌探首一看,只見共有四騎馬,乘者均穿白袍,白帽上繡著一個大紅的火把。四人陡見中伏,齊聲吶喊,拔出兵刃,便往東北角上突圍,靜虛師太大叫道︰「是魔教的妖人,一個也不可放走了!」

峨嵋派雖然人多,却不以衆攻寡,聽著靜虛師太指揮,兩名弟子,兩名男弟子分别上前堵截。魔教的四人手持彎刀,出手甚是悍狠,但峨嵋派這次前來西域的十二弟子,個個是派中的精萃,無一不是武藝精強,鬥不七八合,三名魔教徒衆分别中刃,從馬上摔了下來。餘下那人却厲害得多,砍傷了一名峨嵋男弟子的左肩,奪路而走,縱馬奔出數丈,靜虛師太叫道︰「下來!」身法輕靈,一下子便已欺到了那人背後,拂塵揮出,捲他左腿。那人舞刀擋架,靜玄師太拂塵突然變招,刷的一聲,正好打在他的後腦。這一招擊中要害,拂塵中蘊蓄著靜虛師太深厚的内力,那人如何抵擋得住?登時倒撞下馬。

不料那人極是驃悍,身受重傷之下,竟圖與敵人同歸於盡,張開雙臂,疾向靜玄撲來。靜玄側身閃開,一拂塵又擊在他的胸口,便在此時,馬頸的籠子中有三隻白鴿突然振翅飛起。靜虛叫道︰「玩什麼古怪?」衣袖一抖,三枚鐵蓮子分向三鴿射去,兩鴿應手而落,第三枚鐵蓮子却被那白袍客打出一枝鐵錐撞歪了準頭,一隻白鴿衝入雲端,峨嵋派諸弟子暗器紛出,再也打牠不著,眼見那鴿投東北方去了。靜虛左手一擺,各男弟子拉起四名白袍客,站在靜虛面前。

自攻敵以至射鴿、擒人,滅絶師太始終冷冷的負手旁觀。張無忌心想︰「她親自對蛛児動手,那是對蛛児十分看重了,想是因丁敏君雙腕震斷之故。這老尼若要攔下那雙白鴿,只是一舉手之勢,有何難處。可是她偏生不理,任由衆弟子自行處理。」要知靜虛、靜玄等人,都已是江湖上頗有名望的高手,任何一人都能獨當一面,主持武林中的大事,對付魔教中的幾名徒衆,自不能再由滅絶師太出手,現下由靜虛、靜玄親自動手,都已是將對方的身份抬高了,只見一名女弟子拾起地下兩頭被打死的白鴿,後鴿腿上的小筒中取出一個紙捲,道︰「一模一樣。可惜有一頭鴿児漏網。」滅絶師太冷冷的道︰「有什麼可惜?群魔聚會,一舉而殲,豈不是痛快?省得咱們東奔西走的到處搜尋。」張無忌聽到「向白眉教告急」這幾個字,心下一怔︰「白眉教教主是我外公,不知他老人家會不會來?哼,你這老尼如此傲慢,未必是我外公的對手。」他本來想乘機救了蛛児逃走,這時好戲當前,倒想瞧瞧熱鬧。只聽靜虛向四名白袍人喝道︰「你們還邀了甚麼人手?如何得知我六派圍剿魔教的消息?」

四個白袍人仰天哈哈慘笑,突然之間一齊撲倒在地,一動也不動了。衆人吃了一驚,兩名男弟子俯身一看,驚叫︰「師姐,四個人都死了!」

只見那四個白袍男子臉上各露詭異笑容,均已氣絶。靜玄怒道︰「妖人服毒自盡。這毒藥到是厲害得緊,發作得這麼快。」靜虛道︰「搜身。」四名男弟子應道︰「是!」一人服侍一具屍體,便要往衣袋中搜査,周芷若忽道︰「衆位師兄小心,提防袋中藏有毒物。」四名男弟子一怔,取出兵刃去挑屍體的衣袋,只見袋中蠕蠕而動,原來每個人的衣袋中各藏著兩條極毒的小蛇,若不是周芷若事先提醒,只要伸手入袋,立時便會給毒蛇咬死。衆弟子臉上變色,人人斥罵魔教徒衆行事毒辣。滅絶師太冷冷的道︰「你們從中土西來,今日首次和魔教徒衆周旋。這四個人不過是無名小卒,已是如此陰毒,倘若遇上教中的主腦人物,咱們還有屍骨回歸峨嵋麼?」她哼了一聲,又道︰「靜虛年紀不小了,處事這等草率,還不及芷若細心。」靜虛滿臉通紅,躬身領責。

當晩一行人在沙漠中露宿,生起了一個大火堆。衆人知道這一帶已是魔教人衆出没之所,輪流守夜,嚴加戒備。到得二更天時,只聽得玎玲、玎玲的駝鈴聲響,有一頭駱駝遠遠奔來。因衆人本已睡倒,聽了這聲音,一齊驚醒。那駝鈴聲從西南方響來,但過了一會,鈴聲却響到了西北方。衆人相顧愕然,過不多時,鈴聲竟又在東北方出現。如此忽東忽西,行同鬼魅,要知不論那駱駝的脚程如何迅速,決不能一會児在東,一會児在西。

這時候那駝鈴聲却是自近而遠,越響越輕,陡然之間,東南方鈴聲大振,竟似那駱駝像飛鳥般飛了過去。峨嵋派諸人從未來過大漠,聽了這鈴聲如此怪異,人人都是暗中驚懼。滅絶師太朗聲道︰「是何方高人,便請現身相見。這般裝神弄鬼,成何體統?」她這聲音遠遠傳送出去,數里内字字入耳清晰。果然她説了這句話後,鈴聲便此斷絶,再無聲響,似乎鈴聲的主人怕上了她,不敢再弄玄虛。

第二日白天平安無事,到得晩上二更天時,那駝鈴聲又再出現,忽遠忽近,忽東忽西,滅絶師太又再斥責,這一次駝鈴却對她毫不理會,一會児輕,一會児響,有時似乎是那駱駝怒馳而至,但驀地裡却又悄然而去,吵得人人頭昏腦脹。張無忌和蛛児相視而笑,雖然不明白這鈴聲如何能響得這般怪異,但知定是魔教中的高手所爲,攪得峨嵋衆人六神不安,倒也好笑。

滅絶師太手一揮,衆弟子躺下睡倒,不再去理會鈴聲。這鈴聲響了一陣,雖然花樣百出,但峨嵋衆人不加理睬,似乎自己覺得無趣,突然間在正北方大響數下,就此寂然無聲,看來滅絶師太這「見怪不怪,其怪自敗」的法子,倒也頗具靈效。

次晨衆人收拾衣氈,起身欲行,張無忌和蛛児不約而同的「啊」的一聲驚叫,只見身旁有一人躺著,呼呼大睡,這人自頭至脚,都用一塊汚穢的氈子裹著,不露出半點身體,屁股翹得老高,鼾聲大作。峨嵋派衆人也都驚覺,昨晩各人輪班守夜,如何會不知有人混了進來?滅絶師太何等神功,便是風吹草動,花飛葉落,也逃不過她的耳目,怎地人群中突然多了一人,直到此時才見?各人又是驚訝,又是慚愧,早有兩人手挺長劍,走到那人身旁,喝道︰「是誰,弄什麼鬼?」

那人仍是呼呼打鼾,不理不睬。一名弟子伸出長劍,將那氈子挑起,只見氈子底下赫然是個身披條子長袍的男子,伏在沙裡,睡得正酣。靜虛心知這人膽敢如此,定然大有來頭,走上一步,説道︰「閣下是誰?來此何事?」那鼻鼾聲更響,簡直便如打雷一般。靜玄見這人如此無禮,心下大怒,揮動拂塵,刷的一下,便朝那人臀部打去。

猛聽得呼的一聲,靜玄師太手中那柄拂塵不知如何,竟爾筆直的向空飛中飛去,一直飛上十餘丈高,衆人不自禁的抬頭觀看。滅絶師太大叫道︰「靜玄,留神!」話聲甫落,只見那身穿長條袍子的男子已在數丈之外,靜玄却被他橫抱在雙臂之中,靜虛和另一名年長的女弟子蘇夢清各挺兵刃,飛步向那男子追去。可是那人身法之快,直是匪夷所思,滅絶師太一聲清嘯,手執倚天寶劍,隨後趕去。

峨嵋掌門的身手果眞與衆不同,瞬息間已越靜虛、蘇夢清兩人,青光閃處,一劍向那人背上刺出。但那人奔得快極,這一劍差了尺許,没能刺中,那人臂中雖抱了靜玄,但奔跑之快,絲毫不遜於滅絶師太,他似乎有意炫耀功力,竟不遠走,便繞著衆人急兜圏子,滅絶師太連刺數劍,始終刺不到他身上。只聽得拍的一響,靜玄的拂塵纔落下地來。

這時靜虛和蘇夢清也停了脚步,各人凝神屏息,望著數十丈外那兩大高手的追逐。此處雖是沙漠,但兩人飛奔快跑,塵沙却不飛揚。峨嵋衆弟子見靜玄被那人擒住,便似死了一般,一動也不動,無一心驚。衆人平素知道這位師姊武功已頗得師父眞傳,如何一招之間便被敵人擒住,各人有心上前攔截,但憑以師父的威名,怎能自己拾奪不下,却要門人弟子相助?這以衆欺寡的名聲傳了出去,豈不被江湖上好漢恥笑?是以各人提心吊膽,却誰也不敢上前,只盼師父奔快一步,一劍便在他後心刺個透明窟窿。

片刻之間,那人和滅絶師太已繞了三個大圏,眼見滅絶師太只須多跨一步,劍尖便能傷敵,可是差來差去,便只差得這一步。那人雖然起步在先,滅絶師太是自後趕上,可是手中抱著一人,多了百來斤的重量,這番輕功較量就算打成平手,無論如何也是滅絶師太輸了一籌。張無忌一扯蛛児的衣角,道︰「咱們走不走?」蛛児道︰「這場熱鬧不可不瞧。」張無忌也正是這個心思,點了點頭,便不再言語。

等奔到第四個圏子時,那人突然回身,將靜玄向滅絶師太擲來,滅絶師太覺得這一擲之力有如狂風怒濤,勢不可當,忙氣凝雙足,使個「千斤墜」功夫,輕輕將靜玄接住。那人哈哈長笑,説道︰「六大門派圍剿光明頂,只怕没這麼容易吧!」説著向北疾馳。他初時和滅絶師太追逐時脚下塵沙不驚,但這時却踢得黃沙飛揚,一路滾滾而北,宛如一條黃龍,將他背影遮住。

峨嵋衆弟子擁向師父,只見滅絶師太臉色鐵青,一語不發。蘇夢清突然失聲驚呼︰「靜玄師姊\dash{}」但見靜玄臉如黃臘,喉頭有個傷痕,已是氣絶,衆女弟子都大哭起來。滅絶師太大喝道︰「哭什麼?把她埋了。」衆人立止哭聲,就地將靜玄的屍身掩埋立墓。靜虛躬身道︰「師父,這妖人是誰?咱們牢記在心,好替師妹報仇。」滅絶師太冷冷的道︰「此人多半是魔教四王之一的『青翼蝠王』,久聞他輕功天下無雙,果然是名不虛傳,遠勝於我。」張無忌對滅絶師太本來頗存憎恨之心,但這時見她身遭大變之下,仍是絲毫不動聲色,鎭定如恒,而且當衆讚揚死敵,自愧不如,確是一派宗匠的風範,不由得心下欽服。

丁敏君恨恨的道︰「他便是不敢和師父過手動招,一味奔逃,算什麼英雄?」滅絶師太哼了一聲,突然間拍的一響,打了丁敏君一個嘴巴,説道︰「師父没追上他,没能救得靜玄之命,便是他勝了。勝負之數,天下共知,難道英雄好漢是自封的麼?」丁敏君半邉臉頰登時紅腫,躬身道︰「師父教訓得是,徒児知錯了。」靜虛道︰「師父,這『青翼蝠王』是什麼來頭,還請師父示知。」

滅絶師太將手一擺,不答靜虛的話,自行向前走去。衆弟子見大師姊都碰了這麼一個釘子,還有誰敢多言?一行人默默無言的走到傍晩,生了火堆,在一個沙丘旁露宿。

滅絶師太望著那一堆火,一動也不動,有如一尊石像。張無忌想像她的心情,峨嵋派天下馳名,今日盡傾派中高手,遠征西域,一招也没交手,便有一名女弟子送了性命,心中自是極爲難過。群弟子見師父不睡,誰都不敢先睡,這般呆坐了一個多時辰,滅絶師太突然雙掌一推,一股勁風撲去,蓬的一響,一堆大火登時熄了。張無忌和蛛児都是大吃一驚︰「這老尼好大的掌力。」火堆一熄,衆人都處在黑暗之中,仍是誰都不動。

冷月清光,灑在各人肩頭,張無忌心中,忽起憐憫之意︰「難道威名赫赫的峨嵋派就會在西域一敗塗地,甚至全軍覆没?」忽聽得滅絶師太喝道︰「熄了這妖火,滅了這魔火!」她頃了一頓,緩緩説道︰「魔教以火爲聖,尊火爲神,自從第三十三代明尊教主楊破天死後,魔教並没有教主,兩大光明使者,四大護教法王,以金、木、水、火、土五旗掌旗使,誰都覬覦這教主之位,自相爭奪殘殺,魔教便此中衰。也是正大門派合當興旺,妖邪數該覆滅,倘若魔教不起内鬨,要想挑了這妖孼,倒是大大的不易呢。」張無忌自幼便聽到魔教之名,可是自己母親和魔教頗有牽連,每當多問幾句,父母均各不喜,問到義父時,他不是呆呆出神,便是突然暴怒,因之魔教到底是怎麼一會事,始終莫名其妙。

其後跟著太師張三丰,他對魔教也是深惡痛絶,一提起來,便是諄諄告誡,可是張無忌後來遇到的胡青牛、王難姑、常遇春、徐達、朱元璋等好漢,都是魔教中人,這些人慷慨好義,未必全是惡人,只是各人行動有些詭祕,外人瞧著頗感莫測高深而已。這時他聽滅絶師太説起魔教,不禁留神傾聽。

滅絶師太説道︰「魔教歷代明尊教主,都是以聖火令作爲傳代的信物,可是到了第三十一代教主手中,天奪其魄,這塊火令不知如何竟會失落,於是第三十二代,第三十三代兩代教主,便成了有權無令的教主,這教主已做得很勉強。楊破天突然而死,實不知是中毒還是受人暗算,衆説紛紜,魔教内部就此大亂。楊破天既是暴斃,不及指定繼承之人,想那魔教中人才濟濟,有資格當教主的,少説也有五六人,你不服我,我不服你,直到此時,仍是没有推定教主。咱今日所遇,也是個想做教主的,他便是魔教中四大法王之一的青翼蝠王韋一笑。」

群弟子你瞧瞧我,我瞧瞧你,誰都没聽見過青翼蝠王韋一笑的名字。滅絶師太道︰「這人絶足不到中原,魔教中人行事又鬼祟得緊,因此這人武功雖強,在中原却是半點名氣也無。但白眉鷹王殷天正、金毛獅王謝遜,這兩個人你們總知道吧?」張無忌心中一稟,只聽得蛛児輕輕「啊」的一聲驚呼,滅絶師太向她橫了一眼。殷天正和謝遜的名頭何等響亮,武林中可説誰人不知,那人不曉,靜虛奇道︰「師父,這兩人也都在魔教?」滅絶師太道︰「『魔教四王,紫白金青』白眉是一王,金毛是一王,青翼也是一王。青翼排名最末,身手如何,今日大家都眼見了,白眉、金毛可想而知。金毛獅王喪心病狂,倒行逆施,二十來前突然濫殺無辜,終於不知所終,成爲武林中的一個大謎。殷天正没能當上魔教的教主,一怒而另創白眉教,自己去過一過教主的癮,咱們只道殷天正既然背叛魔教,和光明頂勢成水火,那知光明頂遇上危難之時,還是會去向白眉教求救。」

張無忌心中甚爲混亂,他早知義父和外祖父行事邪僻,均爲正派人士所不容,却没料到他二人然居然都屬魔教中的「護教法王」,一時自己想著心事,没聽到峨嵋群弟子説些什麼。過了一會,纔聽得滅絶師太在説道︰「咱們六大門派這一次進剿光明頂,志在必勝,衆妖邪便是齊心合力,咱又有何懼?只是戰鬥時損傷便多,各人須得先存決死之心,不可意圖僥倖,心存畏懼,臨敵時墮了峨嵋派的威風。」衆弟子一齊站起,躬身答應。

滅絶師太又道︰「武功強弱,関係天資機緣,半分勉強不來。像靜玄這般一招未交,便死於吸血惡魔之手,誰都不會恥笑於她。咱們平素學武,所爲何事?還不是要鋤強扶弱,撲滅妖邪?今日少林、武當、峨嵋、崑崙、崆峒、華山六大派圍剿魔教,吉兇禍福,咱們峨嵋派早就置之度外。靜玄第一個先死,説不定第二個便輪到你們師父\dash{}」群弟子默然躬身,月光下顯得人人臉色更是慘白。

只聽滅絶師太道︰「俗語説得好︰『千棺從門出,其家好興旺。子存父先死,孫在祖乃喪。』人孰無死?只須留下子孫血脈,其家便是死了千人百人,仍能興旺。最怕是你們都死了,老尼却孤零零的活著。嘿嘿,但縱是如此,亦不足惜。百年之前世上有什麼峨嵋派?只須大夥児轟轟烈烈的死戰一場,峨嵋派就是一舉覆滅,又豈足道哉?」群弟子聽了師父這番話,人人熱血沸騰,拔出兵刃,大聲道︰「弟子誓決死戰,不與妖魔邪道兩立。」滅絶師太淡淡一笑,道︰「很好!大家坐下吧!」張無忌看著峨嵋派衆人雖然大都是弱質女流,但這番慷慨決死的英風毫氣,絲毫不讓鬚眉,心想峨嵋位列六大門派,自非偶然。不僅僅以武功取勝而已,眼前她們這副情景,大有荊軻西入強秦,「風蕭蕭兮易水寒,壯士一去兮不復還」之慨,看來魔教的勢力,實是厲害。本來這些話在出發之時便該説了,但想來當時没料到魔教在分崩離析之餘,群魔仍能聯手以抗外侮。今者青翼蝠王這一出手,情勢便大不相同。

果然滅絶師太又道︰「青翼蝠王既然能來,白眉鷹王和金毛獅王亦能來,紫衫龍王和五大掌旗使更加能來。咱們原定是傾六派之力先取光明使者楊逍,然後逐一掃蕩妖魔餘孼,豈知華山派的神機先生這一次神機妙算失靈,料事不中,哈哈,全盤錯了。」靜虛問道︰「那紫衫龍王,又是什麼惡毒的魔頭?」滅絶師太搖頭道︰「紫衫龍王惡跡不著,我也是僅聞得其名而已。聽説此人爭教主不得,便遠逸海外。不再和魔教來往。這一次他若能置身事外,自是最好。『魔教四王,紫白金青』,這人位居四王之首,不用説是極不好鬥的。魔教的光明使者除了楊逍之外,另有一人,魔教歷代相傳,光明使者必是一左一右,地位均在四大護教法王之上。楊逍是光明左使,可是那光明右使的姓名,武林中却是誰也不知。少林派空聞大師、武當派宋遠橋宋大俠,都是博聞廣見之士,但他們兩位也不知道。咱們和楊逍正面爲敵,明槍交戰,勝負各憑武功取決,那倒罷了,但若那光明右便暗中偸放冷箭,這纔是最爲可慮之事。」

衆弟子心下悚然,不自禁的回頭向身後瞧瞧,似乎那光明右使或是紫衫龍王突然掩到偸襲一般。滅絶師太冷然道︰「楊逍害死紀曉芙,韋一笑害死靜玄,峨嵋派和魔教此仇不共戴天,本派掌門人的衣缽傳於誰人,當以這一役中是誰立功最偉大而定,倘若那一名男弟子奮不顧身,竟能僥倖傷得魔教法王,本師便不惜破除百年來的慣例\dash{}」

滅絶師太續道︰「本派掌門,自創派祖師郭祖師以來,慣例是由女子擔任,别説男児無份,便是出了閣的婦人,也不能身任掌門。但本派今日面臨存亡絶續的大関頭,豈可泥守成規,只要是誰立得大功,不論他是男子婦女,都可傳我衣缽。」群弟子默然俯首,都覺得師父鄭而重之的安排後事,計議門戸傳人,似乎料不能生還中土,各人心中都有三分不祥之感、淒然之意。

滅絶師太縱聲長笑,哈哈,哈哈,笑聲從大漠上遠遠的傳了出去。群弟子相顧愕然,暗自驚駭,滅絶師太衣袖一擺,喝道︰「大家睡吧!」靜虛就如平日一般,分派守夜人手,滅絶師太道︰「不用守夜了。」靜虛一怔,但隨即領會,如果青翼蝠王這一等高手半夜來襲,衆弟子那能發覺?守夜也不過白守。

這一晩峨嵋派戒備外弛内緊,似疏實密,倒無意外之事,次日續向西行,走出百餘里後,已是正午,赤日當頭,雖在隆冬,亦覺炎熱。正行之際,西北方忽地傳來隱隱幾聲兵刃相交和呼叱之聲,衆人不待靜虛下令,均各加快脚步,向聲音來處疾馳。不久前面便出現幾個相互跳盪激鬥的人形,奔到近處,只見是三個白袍道人手持奇形兵刃,在圍攻一個中年漢子。那三個道人左手衣袖上都繡著一個紅色火把,顯是魔教中人。那中年漢子手舞長劍,劍光閃閃,和三個道人鬥得甚是激烈,雖是以一敵三,却絲毫不落下風。

張無忌坐在雪撬之中,他腿傷早愈,但不願被峨嵋諸人知覺,仍是假裝不能行走,這時眼光被身前的一名男弟子擋住,須得側身探頭,方能見到那四人相鬥。只見那中年漢子長劍越使越快,突然間轉過身來,一聲呼喝,刷的一劍,在一名魔教道人胸口刺過。峨嵋衆人喝采聲中,張無忌忍不住輕聲驚呼,這一招「順水推舟」,正是武當劍法的絶招,使這一招劍法的中年漢子,却是武當派的六俠殷利亨。

峨嵋群弟子遠遠觀鬥,並不上前相助。餘下兩名魔教道人見己方傷了一人,對方又來了幫手,心中早怯,突然呼嘯一聲,兩人分向南北急奔。殷利亨飛步追那向南方的道人,他脚下比那道人快得多,躍出七八步,便已追到道人身後,拍出一掌,那道人回身招架,甩出了性命不要,圖與殷利亨鬥個兩敗倶傷。峨嵋衆人眼見殷利亨一人難追兩敵,逃向北方的道人輕功極了得,越奔越快,瞧這情勢,殷利亨待得殺了南方那纏戰的道人,無論如何不及再回身追殺北逃之敵。峨嵋弟子和魔教中人仇深似海,都望著靜虛,盼她發令攔截。更有幾個女弟子平素和紀曉芙交好的,知道殷利亨曾和紀曉芙有婚姻之約,這位武當六俠和本派的関係特别不同,紀曉芙因受魔教光明使者楊逍淫辱而死,各人加倍的同情殷利亨,均盼助他一臂之力。靜虛心下也頗感躊躇,但想武當六俠在武林中地位何等尊崇,他若不出聲求助,旁人貿然伸手,便是對他不敬,因之略一沉吟,便不發令攔截,心想寧可讓這妖道逃走,也不能得罪了武當殷六俠。

便在此時,驀地裡青光一閃,一柄長劍從殷利亨手中擲了出來,急飛向北,如風馳電掣的射向那道人背心。説是如此,實則這柄長劍自脱手以至射到敵人身後,快得難以形容,那道人陡然驚覺,待要閃避時,那長劍已穿心而過,透過了他的身子,仍是向前疾飛。那道人脚下兀自不停,又向前奔了兩丈有餘,這纔撲地倒斃,那柄長劍却又在那道人身前三丈之外方始落下,青光閃耀,筆直的插在沙中,雖是一柄無生無知的長劍,却也是神威凜凜,衆人看到這驚心動魄的一幕,無不神馳目眩,半晌説不出話來。

待待回頭再看殷利亨時,只見和他纏鬥的那個魔教道人身子搖搖晃晃,便似喝酔了酒一般,雙手在空中亂舞亂抓,殷利亨不再理他,自行向峨嵋衆人走來。他跨出幾步,那道人再也支持不住,一聲閃哼,仰天倒在沙地之中而死,至於殷利亨用什麼手法將他擊斃,却是誰也没有瞧見。峨嵋群弟子這時才大聲喝起采來,連滅絶師太也點了點頭,跟著嘆息一聲。這一聲長嘆也許是説︰武當派有這等佳子弟。我峨嵋派却無如此了得的傳人。更也許是説︰曉芙福薄,没能嫁得此人,却傷在魔教淫徒之手。(在滅絶師太心中,紀曉芙當然是爲魔教所害,而不是她自己擊死的。)

張無忌一句「六師叔」衝到了口邉,却又強行縮回,在衆師叔伯中,殷利亨和他父親最是交好,待他也特别親厚。他瞧著這位相别八年的六師叔時,只見他滿臉風塵之色,兩鬢微見斑白,想是紀曉芙之死於他心靈有極大打擊。張無忌乍見親人,亟想上前相認,但終於想到不能顯露自己眞相,以免惹起無窮後患。

殷利亨向滅絶師太躬身行禮,説道︰「晩輩大師兄率領師兄弟及第三代弟子,一共三十二人,已到了一線畔,晩輩奉大師兄之命,前來迎接貴派。」滅絶師太道︰「好,還是武當派先到了。可和妖人接過仗麼?」殷利亨道︰「曾和魔教的木、火兩旗交戰三次,殺了幾名妖人,七師弟莫聲谷受了些傷。」滅絶師太點了點頭,她知殷利亨雖然説得輕描淡冩,其實這三場惡鬥,定是慘酷異常,以武當五俠之能,尚且殺不了魔教的掌旗使,七俠莫聲谷甚至受傷。滅絶師太又問︰「貴派可曾査知,光明頂上實力如何?」殷利亨道︰「聽説白眉教、九毒會等魔教的支派,都大舉赴光明頂,有人還説紫衫龍王和青翼蝠王也到了。」滅絶師太一怔,道︰「紫衫龍王也來麼?」兩人一面説,一面並肩而行,群弟子遠遠跟在後面,不敢去聽兩人説些什麼。

兩人説了半個時辰,殷利亨舉手作别,要再去和華山派聯絡,靜虛説道︰「殷六俠,你來回奔波,定必餓了,吃些點心再走。」殷利亨也不客氣,道︰「如此叨擾了。」峨嵋衆女俠紛紛取出乾糧,有的更堆沙爲灶,搭起鐵鍋煮湯,她們自己飲食甚是簡樸,但款待殷利亨却是加倍的週到,自然都是爲了紀曉芙之故。殷利亨懂得他們的情意,眼圏微紅,哽咽道︰「多謝衆位師姊師妹。」

蛛児一直旁觀不語,這時突然説道︰「殷六俠,我跟你打聽一個人,成麼?」殷利亨手中捧著一碗湯麵,回過頭來,神能很是謙和,説道︰「這位小師妹尊姓大名?不知要査問何事?但教所知,自當奉告。」蛛児道︰「我不是峨嵋派的,我是他們對頭,被他們捉了來的,是這老尼姑的俘虜。」殷利亨起先只道她是峨嵋派的小弟子,聽她這麼説,不禁一呆,但想這個小姑娘倒很率直,便問道︰「你是魔教的麼?」蛛児道︰「不是,我是魔教的對頭。」殷利亨不暇細問她的來歷,爲了尊重主人,眼望靜虛,請她示意。靜虛道︰「你要問殷六俠何事?」蛛児道︰「我想請問︰令師兄張翠山張五俠,也到了那一線峽麼?」

此話一出,殷利亨和張無忌都是大吃一驚,殷利亨道︰「你打聽五師哥,爲了何事?」蛛児紅暈生臉,低聲道︰「我是想知道他的公子無忌,是不是也來了。」這一下張無忌自是更加吃驚,心道︰「原來她早知道了我的眞相,這時要揭露。」只聽殷利亨道︰「你這話可眞?」蛛児道︰「我是誠心向殷六俠打聽,怎敢相欺。」殷利亨道︰「我五哥逝世十年,墓木早拱,難道姑娘不知此事麼?」

\chapter{玉面孟嘗}

蛛児一驚站起,「啊」的一聲,道︰「原來張五俠早死了,那麼\dash{}他\dash{}他早就是個孤児了。」殷利亨道︰「姑娘認得我那無忌侄児麼?」蛛児道︰「五年之前,我曾在蝶谷醫仙胡青牛家中見過他一面,不知他現下到了何處。」殷利亨道︰「我奉家師之命,也曾到蝴蝶谷去探視過,但胡青牛夫婦爲人所害,無忌早已不知去向,後來多方打聽,音訊全無,唉,那知\dash{}那知\dash{}」説到這裡,神色淒然,不再説下去了。蛛児忙問︰「怎麼?你聽到什麼噩耗麼?」殷利亨凝視著她道︰「姑娘何以如此関切?我那無忌侄児於你有恩,還是有仇?」

蛛児眼望遠處,幽幽的道︰「我要他隨我去靈蛇島上\dash{}」殷利亨插口道︰「靈蛇島?銀葉先生和金花婆婆是你什麼人?」蛛児不答,仍是自言自語︰「\dash{}他非但不肯,還打我罵我,咬得我手掌鮮血淋漓\dash{}」她一面説,一面左手輕輕摸著右手的手背︰「\dash{}可是,可是\dash{}我還是想念著他。我又不是要害他,帶他去靈蛇島,婆婆會教他一身武藝,設法治好他身上玄冥神掌的陰毒,那知他兇惡得狠,將人家的好心,當作了歹意。」

張無忌只聽得心中一片混亂,這時才知︰「原來蛛児便是在蝴蝶谷中抓住我的那個少女阿離,他心中念念不忘的情郎,居然便就是我。」側頭細看她的容貌,她臉頰浮腫,那裡還有初遇時的半分俏麗,但眼如秋水,澄澈清亮,依稀還可記憶起一些當年的情景。

只聽滅絶師太冷冷的道︰「她是金花婆婆的徒児,按理説也是與魔教有樑子的。但金花婆婆實非正派之人,此刻咱們不想多結仇家,暫且將她扣著。」殷利亨道︰「{\upstsl{嗯}},原來如此。姑娘,你對我無忌侄児倒是一片好心,只可惜他福薄,前幾日我遇到崑崙派的掌門人鐵琴先生何太沖,得知無忌已於四年多之前,失足摔入萬丈深谷之中,屍骨無存。唉,我和他爹爹情愈手足,那知皇天不佑善人,竟連僅有的這點骨血\dash{}」他話未説完,咕{\upstsl{咚}}一聲,蛛児仰天跌倒,竟爾暈了過去。

周芷若搶上去扶了她起來,在她胸口推拿好一會,蛛児方始醒轉。張無忌甚是難過,眼見殷利亨和蛛児如此傷心,自己却硬起心腸置身事外。便在此時,突然有幾滴熱泪,落在他手背之上,張無忌一抬頭,只見到一張俏臉,眼眶中泪水盈盈,正沿著白嫩的面頰流了下來,却是周芷若。張無忌心中一動︰「原來咱們幼小時漢水中的一會,她也没有忘記。」

蛛児咬了咬牙,説道︰「殷六俠,張無忌是被那何太沖害死的麼?」殷利亨道︰「那倒不是。據説朱武連環莊的武烈親眼見到無忌自行失足,摔下深谷,武林中頗有名望的朱長齡,也是一起摔死的。」蛛児長嘆一聲,頹然坐下。殷利亨道︰「姑娘尊姓大名?」蛛児搖頭不答,怔怔下泪,突然間伏在沙中,放聲大哭。殷利亨勸道︰「姑娘也不須難過,我那無忌侄児便是不摔入雪谷,此刻陰毒發作,也已難於存活。唉,他跌得粉身碎骨,未始非福,勝於受那無窮無盡的熬煎。」

滅絶師太忽道︰「張無忌這種孼種,早死了倒好,若是留在世上,定是爲害人間的禍胎。」蛛児大怒,厲聲喝道︰「老賊尼,你胡説八道什麼?」峨嵋群弟子聽她竟然膽敢辱罵師尊,早有四五人拔出長劍,指住她胸口背心。蛛児毫不畏懼,仍然罵道︰「老賊,張無忌的父親是這位殷六俠的師兄,俠名播於天下,有什麼不好?」滅絶師太冷笑不答,靜虛却道︰「他父親固是名門正派的弟子,可是他母親呢?魔教妖女生的児子,不是孼種禍胎是什麼?」蛛児問道︰「張無忌的母親是誰?是魔教妖女?」

峨嵋衆弟子齊聲大笑,張無忌見衆人嘲笑自己母親,不禁面紅耳赤,熱泪盈眶,若不是決意隱瞞自己身世,便要站起身來厲聲抗辯。靜虛爲人忠厚,見蛛児確實不知,見蛛児確實不知,説道︰「張五俠的妻子便是白眉教殷天正的女児,名叫殷素素\dash{}」蛛児「啊」的一聲,臉色大變,似乎突然間見到了最駭人的鬼魅一般,靜虛續道︰「張五俠便因娶了這個妖女,以致身敗名裂,在武當山上自刎而死。這件事天下皆聞,難道姑娘竟然不知麼?」蛛児道︰「我\dash{}我住在靈蛇島上,中原武林之事,全無聽聞。」靜虛道︰「這便是了。」蛛児道︰「那殷素素呢?她在何處?」靜虛道︰「她和張五俠一齊自刎。」蛛児身子又是一跳,道︰「她\dash{}她死了?」靜虛奇道︰「你認得殷素素麼?」

便在此時,突見東北方一道藍燄,衝天而起。殷利亨道︰「啊喲,是我青書侄児受敵人圍攻。」轉身向滅絶師太彎腰行禮,對餘人一抱拳,便即向藍燄奔去。靜虛手一揮,峨嵋群弟子跟著前去。衆人痛恨魔教,與武當派敵愾同仇,既是殷利亨的師侄受敵人圍攻,自是爭相赴援。

衆人奔到近處,只見又是三人夾攻一個的局面,那三個羅帽直身,都作僮僕打扮,每人手中各持單刀。衆人只瞧了幾招,心下便暗暗驚訝,原來這三人雖穿僮僕裝束,出手之狠辣,却竟不輸於一流高手,比之適纔殷利亨所殺的那三個魔教道人,武功又高一籌,三個人繞著一個青年書生,走馬燈似的轉來轉去厮殺。那書生已落下風,但一口長劍,將門戸守得嚴密異常,看來一時還不致有什麼危險。

在那酣殺的四人之旁,站著六個身穿黃袍的漢子,袍上各繡著紅色火把,自是魔教中人。這六人遠遠站著,並不參戰,眼見殷利亨和峨嵋派衆人趕到,六人中一個矮矮胖胖的漢子叫道︰「殷家兄弟,你們不成了,夾了尾巴走吧,老子給你們殿後。」穿僕人裝束的一人怒道︰「厚土旗爬得最慢,姓顏的,還是你先請。」靜虛冷冷的道︰「死到臨頭,還在自己吵嘴。」周芷若道︰「師姊,這些人是誰?」靜虛道︰「那三個穿傭僕衣帽的,是殷天正的奴僕,叫做殷無福、殷無祿、殷無壽。」周芷若驚道︰「三個奴僕,也這麼\dash{}這麼\dash{}」靜虛道︰「他們本是黑道中成名的大盜,原非尋常之輩。那些穿黃袍的是魔教厚土旗下的妖人。這個矮胖子説不定便是厚土旗的掌旗使顏垣。師父説魔教五旗掌旗使和白眉教教主爭位,向來不和\dash{}」他説到這裡,那青年書生迭遇險招,嗤的一聲,左手衣袖被殷無壽的單刀割去了一截。

殷利亨一聲清嘯,長劍遞出,指向殷無祿。殷無祿橫刀硬封,刀劍相交,此時殷利亨内力渾厚,已是非同小可,拍的一聲,殷無祿的單刀陡然震得彎了過去,變成了一把曲尺。殷無祿吃了一驚,向旁躍開三步。突然間蛛児猶如閃電般縱身而上,右手食指一戮,戮中了殷無祿的後心,又如閃電般躍回原處。

殷無祿武功原非泛泛,但蛛児竟會突然間乘虛偸襲,却是誰也意料不到,何況他單刀被震得彎曲,已是大吃一驚,竟被蛛児一指戮中。他左掌護身,右手握著曲刀,作橫掠之勢,就此僵硬不動,霎時之間,一張臉變成墨一般黑。

殷無福、殷無壽大驚之下,顧不得再攻那青年書生,搶到殷無祿身旁,見他早已氣絶斃命。兩人眼望蛛児,突然齊聲説道︰「原來是離小姐。」蛛児道︰「哼,還認得我麼?」衆人心想這兩人定要上前和蛛児拚命,那知兩人抱起殷無祿的屍身,一言不發,發足便向北方奔去。這個變故突如其來,人人目瞪口呆,摸不著頭腦。

那身穿黃袍的矮胖子左手一揚,手裡已執了一面黃色大旗,其餘五人一齊取出黃旗揮舞,雖只六人,但大旗獵獵作響,氣勢極是威武,緩緩向北退却。峨嵋衆人見那旗陣古怪,都是呆了一呆,突然間兩名男弟子發一聲喊,拔足追去,殷利亨身形一晃,後發先至,攔在兩人身前,橫臂輕輕一推,那兩人身不由主的退了三步,滿臉脹得通紅。靜虛喝道︰「兩位師弟回來,殷六俠是好意,這厚土旗追不得。」殷利亨道︰「前幾日我和莫七弟追擊烈火旗陣,吃了個大虧。莫七弟頭髮眉毛燒掉了一半。」一面拉起左手衣袖,只見他手臂上紅紅的有燒炙傷痕。那兩名峨嵋弟子適纔見過殷利亨的身手,不禁暗自心驚。

滅絶師太冷森森的眼光在蛛児臉上轉了幾圏,陰沉沉的道︰「你這是『千蛛絶戸手』?」蛛児道︰「還没練成。」滅絶師太道︰「倘若練成了,那還了得麼?你爲什麼殺了這人?」蛛児道︰「我愛殺就殺,你管得著麼?」滅絶師太身型一錯,已從靜虛手中接過長劍,只聽得錚的一聲,蛛児急忙向後躍開,臉色有如白紙。原來滅絶師太在這一瞬之間,已在蛛児的右手食指上斬了一劍,手法之快,誰都没有看清。那知蛛児這根手指上套有精鋼的套子,滅絶師太所用的不是倚天劍,這一劍竟然没能傷了她。滅絶師太將長劍擲還靜虛,哼了一聲,道︰「這次便宜了你,下次休教再撞在我手中。」須知她是一派掌門之尊,對小輩既然一擊不中,就須自重身份,不肯再度出手。

殷利亨見蛛児練這種歹毒陰狠的武功,原是武家的大忌,但一來見她殺了殷無祿,乃是相助自己,二來她牽掛張無忌,一往情深,自己也不禁感動,不願滅絶師太傷她,便勸道︰「師叔,這孩子學錯了功夫,咱們慢慢再叫她另從明師,{\upstsl{嗯}},或者我推薦她去鐵琴先生門下,也是好的。」拉著那青年書生過來,説道︰「青書,快拜見師太和衆位師伯師叔。」那書生搶上三步,跪下向滅絶師太行禮,待得向靜虛行禮時,衆人連稱不敢當,一一還禮。要知張三丰年過百歳,算起輩份來比滅絶師太高了實不止一輩,殷利亨只因曾和紀曉芙有婚姻之約,纔算比滅絶師太低了一輩,倘若張三丰和峨嵋派祖師郭襄平輩而論,那麼滅絶師太反過來要稱殷利亨爲師叔了。好在武當和峨嵋門戸各别,互相不敘班輩,大家各憑年紀,隨口亂叫。但那青年書生稱峨嵋衆弟子爲師伯師叔,靜虛等人自非謙讓不可。

衆人適纔見他力鬥殷氏三兄弟,法度嚴謹,招數精奇,的是名門子弟的風範,而在三名高手圍攻之下,鎭靜拒敵,絲毫不見慌亂,尤其不易,此時走到臨近一看,衆人心中不禁暗暗喝采︰「好一個美少年!」但見他眉目清秀,俊美之中帶著三分軒昂的氣度,令人一見之下,自然心折,生出親近之意。殷利亨道︰「這是我大師哥的獨生愛子,叫做青書。」靜虛道︰「啊,近年來頗聞玉面孟嘗的俠名,江湖上都説宋少俠慷慨仗義濟人解困。今日得識尊範,幸何如之。」峨嵋衆弟子驚嘆不已,看來「玉面孟嘗宋青書」的名頭,在江湖上著實響亮,是以一聽之下,群相聳動。

蛛児站在張無忌身旁,低聲道︰「阿牛,這人生得比你俊多啦。」張無忌道︰「當然,那遇用説?」蛛児道︰「你喝醋不喝?」張無忌道︰「笑話,我喝什麼醋?」蛛児道︰「你那位周姑娘這般模樣的瞧著他,傾慕之極,你還不喝醋?」周芷若這時果然正在瞧著宋青書,蛛児的話説得很輕,誰都没有留意,不知怎的,周芷若却似都聽見了,突然間回過頭,暈生雙頰,向張無忌和蛛児望了一眼。

張無忌自從得知蛛児即是當年在蝴蝶谷中遇見過的阿離之後,心中思潮一直翻湧不定。當時蛛児用強,定要拉他前赴靈蛇島,他掙扎不脱,只得拚命咬了她一口,豈知事隔數年,她竟對自己念念不忘。這時見衆人圍擁玉面孟嘗宋青書,他想著自己的事,也没多加留神。忽聽殷利亨道︰「書児,咱們這便走吧。」宋青書道︰「崆峒派預定今日中午在這一帶會齊,但這時候還不到,只怕出了什麼岔子。」殷利亨臉有憂色,道︰「此事是甚爲可慮,咱們眼下深入敵境,危機四伏,實是大意不得。」宋青書道︰「不如咱們便和峨嵋派同向西行,此去西方十五六里之處,或有敵人埋伏。」

靜虛奇道︰「宋少俠何所見而云然?」宋青書道︰「晩輩胡亂猜測,只料敵不準。」靜虛深知他父親宋遠橋,不但武功卓絶,而且精通奇門術數,擅於行軍打仗的兵法,他家學淵源,料也不弱,當下便不再問。殷利亨道︰「好,咱們便和峨嵋的衆位前輩同行吧。」滅絶師太和靜虛等心道︰「這三四十年來,張三丰眞人早就不管俗務,實則宋遠橋纔是眞正的武當掌門。看來第三代的武當掌門將由這位宋少俠接任。殷利亨雖是師叔,反倒聽師侄的話。」實則殷利亨素來隨和溫順,不大有自己的主張,别人説甚麼,他總是不加反對。

一行人向西行了十四五里,前面出現一個大沙丘,靜虛見宋青書加快脚步,搶上沙丘,左手一揮,兩峨嵋弟子奔了上去,不肯落於武當之後。三人一上沙丘,不禁齊聲驚呼,祇見沙丘之西,沙漠中橫七豎八的躺著十來具屍體。衆人心知不妙,急步上前,祇見那些死者有老有少,不是頭骨碎裂,便是胸口陥入,似乎個個受了巨棍大棒的重擊。殷利亨江湖上見識最多,説道︰「江西鄱陽幫全軍覆没,是被魔教巨木旗殲滅的。」滅絶師太皺眉道︰「鄱陽幫來幹什麼?貴派邀了他們麼?」言中頗有不悦之意。要知武林中的名門正派對各幫會頗有歧視,滅絶師太很不願和他們混在一起。殷利亨忙道︰「没邀鄱陽幫。不過鄱陽幫劉幫主是崆峒派的記名弟子,聽説六派圍剿光明頂,他們想必自告奮勇,爲師門效力。」滅絶師太哼了一聲,不再言語了。

衆人將鄱陽幫幫衆的屍體在沙中埋了,峨嵋群弟子對宋青書料敵的本事十分佩服,一個姓韋的男弟子道︰「宋兄弟,過去還有多少路,咱們不會遇到敵人?」宋青書瞧著十七個排成一列墳墓,沉吟未答,突然間最西一座墳墓從中裂開,躍出一個人來,抓住姓韋的弟子,疾馳而去。

這一下衆人當眞是嚇得呆了,七八個峨嵋女弟子都尖叫了出來,只見滅絶師太、殷利亨、宋青書、靜虛四人,一齊發足在後追趕。追了好一陣,衆人這纔醒悟,原來從墳墓中跳出來那人,正是魔教中的青翼蝠王,他穿了鄱陽幫幫衆的衣服,混在衆屍首之中,閉住吸呼,假裝死去,峨嵋群弟子不察,竟將他埋入沙墳。他藝高人膽大,當時竟不發作,直將衆人作弄得彀了,這纔突然破墳而出。初時滅絶師太等四人並肩齊行,奔了大半個圏子,已然分出高低,變成二前二後,殷利亨和滅絶師太在前,宋青書和靜虛在後。奔到第二個圏子時,靜虛已然落後,宋青書反而和前面兩人的距離漸漸縮短,足見他内力渾厚,年紀輕輕,修爲著實深湛。可是那青翼蝠王輕功之高,當眞世上無雙,手中雖抱著一個男子,殷利亨等那裡又追趕得上。第二個圏子將要児完,宋青書猛裡立定,叫道︰「趙靈珠師叔、黃綺文師叔請向離位包抄,孫良貞師叔、李明霞師叔,請向震位堵截\dash{}」

宋青書隨口呼喝,號令峨嵋派的十多名弟子分佔八卦方位,峨嵋衆人本當群龍無首之際,聽到他的號令之中自有一番威嚴,人人立即遵從。這麼一來,青翼蝠王韋一笑已無法順利大兜圏子,縱聲尖笑,將手中抱著那人向天上擲去,自行疾馳而逝,滅絶師太伸手將從天空落下來的弟子接住,只聽得韋一笑的聲音隔著塵沙傳了過來︰「嘿嘿嘿,後生可畏,峨嵋居然有這等人才,滅絶老尼了不起啊。」這幾句話顯然是稱讚宋青書的,滅絶師太臉一沉,看手中那名弟子時,只見他咽喉上露出兩排齒印,已然氣絶。

衆人圍住在她的身旁,均感無話可説。隔了良久,殷利亨道︰「曾聽人説道,這青翼蝠王每次施展武功之後,必須飽吸一個活人的熱血,果是所言不虛。」滅絶師太又是慚愧,又是痛恨,她自接任掌門以來,峨嵋派從未受過如此重大的挫折,兩名弟子接連被敵人吸血而死,但敵人面目如何,竟也没有瞧著。她呆了半晌,瞪目問宋青書道︰「我門下這許多弟子的名字你怎地知道?」宋青書道︰「適纔靜虛師叔給弟子引見過了。」滅絶師太道︰「{\upstsl{嗯}},過耳不忘!我峨嵋派那有這樣的人才?」

當日晩間歇宿,宋青書恭恭敬敬的走到滅絶師太跟前,行了一禮,説道︰「前輩,晩輩有一不情之請相求。」滅絶師太冷冷的道︰「既是不情之請,那便不必開口了。」宋青書恭恭敬敬的行了一禮。道︰「是。」回到殷利亨身旁坐下。衆人聽到他向滅絶師太出言求懇,可是一被拒絶,隨即不再多言,各人心中都是好奇心起,不知他想求些什麼事。丁敏君最是沉不住氣,走過去問道︰「宋兄弟,你想求我師父什麼事?」宋青書道︰「家父傳授晩輩劍法之時,説道當今之世,劍術通神,自以本門師祖爲第一,其次便是峨嵋派的滅絶前輩。武當和峨嵋劍法各有長短,例如本門這一招『手揮五弦』,和貴派的『輕羅小扇』大同小異,但威力一強,便不彀清靈活潑,遠不如『輕羅小扇』的揮灑自如。」他一面説,一面取出長劍,比劃了兩招。那一招「輕羅小扇」不免有些不倫不類。丁敏君笑道︰「這一招不對。」接過他手中長劍,試給他看,説道︰「我手腕還痛著,使不出力,但就是這麼一個模樣。」宋青書大爲嘆服,説道︰「家父常自言道,他自恨福薄,没能見到尊師的劍術。今日晩輩一見丁師叔這一招『輕羅小扇』,當眞是開了眼界。晩輩適纔是想請師太指點幾手,以解晩輩心中関於劍法上的幾個疑團,但晩輩非貴派子弟,這種話本該不應出口。」

滅絶師太遠遠聽著,將他每句話都聽在耳裡,聽他推許自己爲天下劍法第二,心中極是樂意,張三丰是當世武學中的泰山北斗,人人都是佩服的,她從未存心要蓋過這位古今罕見的大宗師。但武當派居然認爲她除張三丰外劍術最精,不自禁的頗感得意,眼見丁敏君比劃這一招,精神勁力,都只不過三四分火候,名震天下的峨嵋劍法,豈僅如此而已?當下走近身去,一言不發的從丁敏君手中接過長劍,手齊鼻尖,輕輕一顫,劍尖{\upstsl{嗡}}{\upstsl{嗡}}連響,自右至左又自左至右的連晃九下,快得異乎尋常,但每一晃却又是清清楚楚。

衆弟子見師父施展如此精妙的劍法,幾乎將心提到了脖子裡,殷利亨大叫︰「妙極!」宋青書凝神屏氣,暗暗心驚。他初時不過爲向滅絶師太討好,稱讚一下峨嵋劍法,那知她一經施爲,實有難以想像的奇奥,不由得衷心欽服,誠心誠意的向她討教起來。他問什麼,滅絶師太便教什麼,竟比傳授本門弟子,還要盡力。宋青書武學修爲本高,人又聰明,每一句都問中了竅要。

峨嵋群弟子圍在兩人之旁,見師父所施展的每一記劍招,無不精微奇奥,妙不巓毫,有的隨師十餘年,也未見師父顯過如此神技。蛛児站在人圏之外,忽向張無忌道︰「阿牛哥,我若能學到青翼蝠王那樣的輕功,眞是死也甘心。」張無忌道︰「這種邪門功夫,學他作甚?殷六\dash{}殷六俠説,這韋一笑每施展一次武功,便須吸飲人血,那不是成了魔鬼麼?」蛛児道︰「他武功好,便殺死峨嵋派的弟子,要是他輕功差了些,給老尼他們捉住,還不是一樣給人殺死。什麼名門正派,邪魔外道,有甚麼分别?」

張無忌一時無言可答,忽見人叢中飛起一柄明晃晃的長劍,直向天空,原來是宋青書和滅絶師太拆招,被她在第五招上,使一招「矯龍遊龍」,將宋青書的長劍震上了天空。衆人抬頭一齊瞧著那柄長劍,突見東北角上相隔十餘里處,一道黃燄衝天升起。這次六大派遠赴西域圍剿魔教,爲了行動隱蔽起見,採的是分進合擊的方略,議定以六色火箭爲聯絡信號,這黃燄火箭乃是崆峒派的信號。殷利亨叫道︰「崆峒派遇敵,快去赴援。」

當下衆人疾向火箭升起處奔去,馳到鄰近,但見黃沙寂寂,一個人影也無。殷利亨大聲叫道︰「崆峒派溫老前輩在麼?古老前輩在麼?」聲音遠遠傳送出去,却無應聲,突見西方十餘里外,又是一道黃燄火箭上升。靜虛叫道︰「原來咱們趕到,他們却鬥到了西方。」各人敵愾同仇,不辭辛苦,又急向西行,輕身功夫較差的便已落後。靜虛仗劍殿後,生怕武功較弱的師弟師妹落了單,中伏遇敵。

待得滅絶師太、殷利亨、宋青書等人趕到,當地仍是寂然無人,但見地下落著一些焦了的碎紙竹片,那火箭花顯是從此處放射上去的。各人正沉吟間,宋青書道︰「前輩,咱們中了敵人的奸計,你瞧地下只有一個人的足跡。若是崆峒派遇敵,至少也會有四五人的足跡\dash{}」滅絶師太大怒,冷冷的道︰「好妖人!」宋青書猛地省悟,道︰「不好,崆峒派眞是中伏,請跟我來。」説著向西南偏南的方向奔去。殷利亨和他並肩而行,問道︰「你怎知崆峒派當眞還敵?」宋青書道︰「這黃燄火箭是眞的,顯是中原巧匠的製作,西域未必有人能製得一模一樣。」殷利亨道︰「你説是崆峒弟子落入了魔教手中,火箭炮被妖人得去?」宋青書道︰「不錯。妖人騙得咱們先向東北,再向西行,要咱們疲於奔命,實則他們定是在西南偏南之處,幹那傷天害理的勾當。」滅絶師太在他們身旁約莫兩丈之外,但每一句話都在耳裡,點頭道︰「你料得不錯。」

一行又奔出十餘里,除了滅絶師太、殷利亨、宋青書等武功極深之的人之外,這幾次來回奔馳,餘人都已頗見氣促。正行之間,突見前面一個小沙丘下站著一人,那人身旁另有一人躺著。滅絶師太衝上前去一看,只見正是蛛児和張無忌兩人,原來峨嵋群弟子急於殺敵復仇,已没將這兩個不相干的人放在心上,不再嚴密監守,不知如何,這兩人反而搶在頭裡。滅絶師太問道︰「你們怎麼在這児。」心中暗暗驚奇︰「難道這鬼丫頭的脚程比我還快?」蛛児笑道︰「這火燄箭明明是對方的誘敵之計,騙得你們先向東北,再向正西。我想你們就算不能發覺,那姓宋的小子也該當猜到了,自會到這児來,阿牛哥,你説是麼?」張無忌微微一笑,道︰「咱們在這裡休息了半天啦,你們走得很累了,是不是?」滅絶師太厲聲道︰「鬼丫頭,你既猜到了,何不早説?」蛛児笑道︰「你又没問我,何況那時候我便説了,你也不會相信。總須自己跑一個上氣不接下氣,纔會明白,這叫做不經一事,不長一智啊。」滅絶師太給她這幾句話搶白得怒不可抑,却又不便出手教訓於她,便在此時,只聽得西南方傳來一陣陣兵戮碰撞之聲,鬥得極是激烈。蛛児道︰「你跟我發脾氣有什麼用?你的同夥都快給人家殺光啦。」滅絶師太和殷利亨等一聽,不再理會蛛児,當即向西南奔去。

越走聲音越是慘厲,不時傳來一兩聲臨死時的呼叫,待得馳到臨近,各人都是吃了一驚,眼前是一片大屠殺的修羅場,雙方各有數百人參戰,明月照耀之下,刀光劍影,人人均在捨死忘生的惡鬥。殷利亨一觀戰局,説道︰「敵方是鋭金、洪水、烈火三旗,{\upstsl{嗯}},崆峒派在這裡,華山派到了,崑崙派也到了。我方三派會鬥敵人三旗。書児,咱們也參戰罷。」長劍在空中虛劈一招,{\upstsl{嗡}}{\upstsl{嗡}}作響。宋青書道︰「且慢,待峨嵋派衆位師叔伯一齊到達,可期必勝。」張無忌一生之中,從未見過如此大戰的場面,但見刀劍飛舞,血肉橫濺,情景慘不足睹。他並不希望魔教三旗得勝,但也不願殷六叔他們得勝,一面是父親的一派,一面是母親的一派,可是雙方却在勢不兩立的惡鬥,每一個人被殺,他都是心中一凜,一陣難過。忽聽得宋青書指著東方,道︰「六叔你瞧,那邉尚有大批敵人,待機而動。」張無忌順著他手指瞧去,果見相距戰場數十丈外,黑壓壓的站著三隊人馬,行列整齊,每一隊均有百餘人。戰場上三派鬥三旗,眼前是勢均力敵的局面,但若魔教這三隊人一投入戰鬥,崆峒、華山、崑崙三派勢必覆滅,只是不知如何,這三隊人物始終按兵不動。滅絶師太和殷利亨都是暗暗心驚,問宋青書道︰「這些人幹麼不動手?」宋青書搖頭道︰「想不通。」蛛児突然冷笑道︰「那有什麼想不通,再明白也没有了。」宋青書臉一紅,默然不語,滅絶師太想要出口相詢,但終於忍住。

殷利亨却道︰「還請姑娘指點。」蛛児道︰「那三隊人馬是白眉教的。白眉教雖然也是魔教的一支,但向來和五行掌旗使不睦,你們倘若把五行旗都殺光了,白眉教反而會暗暗喜歡。殷天正説不定便能當上魔教的教主啦。」滅絶師太等恍然大悟,殷利亨道︰「多謝姑娘指點。」這時峨嵋群弟子已先後到達,站在滅絶師太身後。靜虛道︰「宋少俠,説到佈陣打仗,咱們誰也不及你,大夥児都聽你號令,但求殺敵,你不用客氣。」宋青書道︰「六叔,這個\dash{}這個\dash{}侄児如何敢當?」滅絶師太道︰「這當児還講究什麼虛禮?發號令吧。」

宋青書眼見戰場中情勢急迫,崑崙派頗佔上風,華山派和洪水旗鬥得勢均力敵,崆峒派却是越來越感不支,給烈火旗圍在垓心,大施屠戮,便道︰「咱們分三路衝下去,一齊攻擊鋭金旗。師太領人從東面殺入,六叔領人從西面殺人,靜虛師叔和晩輩等從南面殺入\dash{}」靜虛奇道︰「崑崙派並不吃緊啊,我看倒是崆峒派十分危急。」宋青書道︰「崑崙派已佔上風,咱們再以雷霆萬鈞之勢殺入,當能一舉而殲鋭金旗,餘下兩旗便是望風披靡。倘若去救援崆峒,殺了個難解難分,白眉教來個漁翁得利,那便糟了。」靜虛大是欽服,道︰「宋少俠説得不錯。」當即將群弟子分爲三路。

蛛児拉著張無忌的雪撬,道︰「咱們走吧,在這児没什麼好處。」説著向後便退。宋青書發足追上,橫劍攔住,道︰「姑娘休走。」蛛児奇道︰「你攔住我幹什麼?」宋青書道︰「姑娘來歷甚奇,不能如此容你走開。」蛛児冷笑道︰「我來歷奇便怎樣?不奇又怎樣?」滅絶師太心急如焚,恨不得立時大開殺戒,將魔教人衆殺個乾淨,聽得蛛児和宋青書鬥口,身形一晃,已欺到蛛児身邉,伸手點了她背上、腰間、腿上三處穴道。

\chapter{義氣干雲}

蛛児和她武功相去甚遠,這一下全無招架之功,膝彎一軟,倒在地下。滅絶師太長劍一揮,喝道︰「今日大開殺戒,除滅妖邪。」她和殷利亨、靜虛各率一隊,直向鋭金旗衝去。

崑崙派何太沖、班淑嫻夫婦領著門人弟子,對抗鋭金旗已頗佔優勢,峨嵋、武當兩派一衝入,聲威更是大盛。滅絶師太劍法凌厲絶倫,没有一名魔教的教衆能擋得了她三劍,但見她高大的身形在人叢中穿插來去,東刺一劍,西劈一劍,瞬息之間,已有七名教衆喪生在她長劍之下。鋭金旗掌旗使莊錚見情勢不對,手挺狼牙棒,趕下迎敵,這纔將滅絶師太擋住。但十餘招一過,滅絶師太展開峨嵋劍法,越打越快,竭力搶攻,莊錚武藝甚精,和她鬥了個旗鼓相掌。這時殷利亨、靜虛、宋青書、何太沖、班淑嫻等人放手大殺,鋭金旗下雖也不乏高手,但如何敵得過峨嵋、崑崙、武當三派聯手,頃刻間死傷極是慘重。

莊錚砰砰砰三棒,將滅絶師太向後逼退一步,跟著又是一棒,摟頭蓋腦的壓將下來。滅絶師太長劍斜走,在狼牙棒上一點,以一招「順手推舟」,要將他狼牙棒帶開。那知莊錚是魔教中的非同小可的人物,在武林中可算得是一流高手,他天生膂力奇大,自幼得遇異人傳授,内功外功倶達爐火純青,這時狼牙棒上感到滅絶師太的内力,登時大喝一聲,臂力反彈出去,拍的一響,滅絶師太長劍斷爲三截。滅絶師太兵刃斷折,手臂酸麻,心下雖驚,却不退開閃避,反手抽出背負的倚天劍,一招「鐵鎖橫江」,推送而上。莊錚但覺手下一輕,狼牙棒已被倚天劍部開,跟著半個頭顱也被這柄鋒利無匹的利劍削下。

鋭金旗旗下諸人一見掌旗使喪命,個個心驚憤慨,高呼酣戰。宋青書和滅絶師太等只道莊錚一死,鋭金旗便即潰散,跟著洪水、烈火兩旗也便敗退,那知敵人反而是不顧性命的狠鬥,崑崙和峨嵋門下接連數人被殺。

洪水旗中一人叫道︰「莊旗使殉教歸天,鋭金、烈火兩旗退走,洪水旗斷後。」烈火旗陣中旗號一變,應命向西退却。但説鋭金旗衆人竟是愈鬥愈狠,誰也不退。洪水旗中那人又高聲叫道︰「洪水旗唐旗使有令,鋭金旗諸人速退,徐圖爲莊旗使報仇。」鋭金旗中數人齊聲叫道︰「請洪水旗速退,爲咱們報仇雪恨,鋭金旗和莊旗使同生共死。」

洪水旗中突然揚起黑旗,一人聲如巨雷,叫道︰「鋭金旗諸位兄弟,洪水旗決爲你們復仇。」鋭金旗中這時尚剩下七十餘人,齊聲叫道︰「多謝唐旗使。」只見洪水旗旗幟翻動,向西退走。華山、崆峒兩派見敵人陣容嚴整,斷後者二十餘人手持金光閃閃的圓筒,不知有何古怪,却也不敢追擊。各人回過頭來,向鋭金旗夾攻。這時情勢已定,崑崙、峨嵋、武當、華山、崆峒五派圍攻魔教鋭金旗,除了武當派只到四人,其餘四派都是精英盡出,鋭金旗掌旗使已死,群龍無首,自然不是敵手,但旗下諸人居然個個重義,視死如歸,決意追隨莊錚殉教。

殷利亨殺了數名教衆,頗覺勝之不武,大聲叫道︰「鋭金旗妖人聽著︰你們眼前只有死路一條,趕快抛下兵刃投降,饒你們不死。」那掌旗副使哈哈笑道︰「你把我明教教衆,忒也瞧得小了。莊大哥已死,咱們豈願再活?」殷利亨叫道︰「崑崙、峨嵋、華山、崆峒諸派的朋友,大夥退後十步,讓這批妖人投降。」各人紛紛後退,只有滅絶師太却恨極了魔教,兀自揮劍狂殺。倚天劍劍鋒到處,劍折刀斷,肢殘頭飛,峨嵋派弟子見師父不退,已經退下了的又再搶上厮殺,竟變成了峨嵋派獨鬥鋭金旗的局面。

魔教鋭金旗下教衆尚有六十餘人,極厲害的好手也有二十餘人,在掌旗副使吳勁草率領下,與峨嵋派的十餘人相抗,以五敵一,原可穩佔上風,但滅絶師太的倚天劍實在太過鋒鋭,青霜到處,所向披靡,霎時之間,又有七八人喪於劍下。

張無忌看得不忍,對蛛児道︰「咱們走吧!」伸手去解她身上穴道,那知在她背心和腰脅間推拿幾下,蛛児只感一陣酸麻,穴道却解不開。原來滅絶師太内力渾厚,出手輕輕一點,勁力直透穴道深處,張無忌的解穴法雖然對路,却非片刻之間所能奏功。他嘆了一口長氣,轉過頭來,只見鋭金旗數十人手中兵刃已盡數斷折,一來四面崑崙、華山、崆峒諸派人衆團團圍住,二來衆教人也不想逃遁,各使空手和峨嵋群弟子搏鬥。滅絶師太雖是痛恨魔教,但她以一派掌門之尊,不願用兵刃屠殺赤手空拳之徒,左手手指連伸,脚下如行雲流水般四下飄動,片刻之間,已將鋭金旗的五十多名餘衆點住穴道。各人呆呆直立,無法動彈。旁觀衆人見滅絶師太顯了這等身手,盡皆喝采。

這時天將黎明,忽見影影綽綽,東南西北各有人形移近,竟是白眉教的人衆。當峨嵋派和鋭金旗激鬥之時,宋青書早在暗暗耽憂,注視著白眉教的動靜,低聲和華山派的神機子鮮于通商議抵禦之策。白眉教教衆走到離衆人大約二十丈外,却又停步不動,顯是遠遠在外監視,不即上前挑戰。

蛛児道︰「阿牛哥,咱們快些離去,落入白眉教手中,那可糟糕得緊。」張無忌心中,對白眉教却另有一種難以形容的親切之感。那是他母親的教派,他從來没見過白眉教的教徒,當想念母親之時,往往便想︰「母親是見不到了,幾時能見外公和舅舅一面呢?」這時知道白眉教人衆便在附近,總想看看,外公和舅舅是不是也在這裡。

宋青書走上一步,對滅絶師太道︰「前輩,咱們快些處決了鋭金旗,轉頭再對付白眉教,免有後顧之憂。」滅絶師太點了點頭。東方朝日將升,矇矇朧朧的光芒射在滅絶師太高大的身形之上,照出長長的影子,威武之中,帶著幾分淒涼恐怖之感。她有心要挫折魔教的鋭氣,不願就此一劍將他們殺了,厲聲喝道︰「魔教的人聽著︰那一個想活命的,只須出聲求饒,便放你們走路。」隔了半晌,只聽得嘿嘿、哈哈、呵呵、魔教衆人一齊大笑起來,聲音十分響亮。

滅絶師太怒道︰「有什麼好笑。」鋭金旗掌旗副使吳勁草朗聲道︰「咱們和莊大哥誓共生死,快快一劍將咱們殺了。」滅絶師太哼了一聲,説道︰「好啊,這當児還充英雄好漢!你想死得爽快,没這麼容易。」長劍輕輕一顫,已將他的右臂斬了下來。吳勁草哈哈一笑,神色自若,説道︰「明教替天行道,濟世救民,生死始終如一。老賊尼想要咱們屈膝投降,乘早别妄想了。」滅絶師太愈益憤怒,刷刷刷三劍,又斬下三名教衆的手臂,問第五人道︰「你求不求饒?」那人笑道︰「放你的狗屁!」靜虛閃身上前,手起一劍,斷了那人右臂,叫道︰「讓弟子來斬誅妖孼!」她連問數人,魔教教衆無一屈服。靜虛殺得手也軟了,回頭道︰「師父,這些妖人刁頑得緊\dash{}」意下是向師父求情。滅絶師太決不理會,道︰「先把每個人的右臂斬了,若是倔強到底,再斬左臂。」靜虛無奈,又斬了幾人的手臂。

張無忌再也忍耐不住,從雪撬中一躍而起,攔在靜虛身前,叫道︰「且住!」這一下大出衆人意料之外,靜虛微微一怔,退了一步。張無忌大聲道︰「這般殘忍無道,不慚愧麼?」

衆人突然見到一個衣衫襤褸不堪的少年挺身而出,先是一怔,隨即有許多人看到他的怪模怪樣,不禁笑出聲來,待得聽到他質問靜虛的這兩句話如此理正詞嚴,便是各派的名宿高手,也不禁爲他的氣勢所懾。靜虛一聲長笑,説道︰「邪魔外道,人人得而誅之,有什麼殘忍不殘忍?」張無忌道︰「這些前輩,大哥,個個義氣干雲,慷慨求死,實是鐵錚錚的英雄好漢,怎能説是邪魔外道?」靜虛道︰「他們魔教徒衆,難道還不是邪魔外道?那個青翼蝠王吸血殺人,害死我師弟師妹,乃是你親眼目睹,這不是妖邪,什麼纔是妖邪?」張無忌道︰「那青翼蝠王只殺了二人,你們所殺之人已多了十倍。他用牙齒殺人,尊師用倚天劍殺人,一般的殺,有何善惡之分?」

靜虛大怒,喝道︰「好小子,你竟敢將我師父與妖邪相提並論?」呼的一掌,往他面門擊去,張無忌急忙相避,那知靜虛是峨嵋門下大弟子,武功已頗得滅絶師太的眞傳,這一掌擊他面門,實是虛招,待得張無忌一閃身,飛出左腿,一脚踢中他的胸口。但聽得砰彭、喀喇兩聲,靜虛左腿早斷,身子向後飛出,摔在數丈之外。原來張無忌胸口中了數招,體内九陽神功自然而然的發生抗力,他招數之精固是遠遠不及靜虛,但九陽神功的威力何等厲害,敵招勁力愈大,反擊愈重,靜虛這一腿便如踢在自己身上一般。幸好靜虛並没想傷他性命,這一腿踢出時只使了五成力,自己纔没受厲害内傷。旁觀衆人大都識得靜虛,知道她是滅絶師太座下數一數二的好手,怎地如此不濟,一招之間便被這破衫少年摔出數丈?若説徒負虛名,却又不然,適纔她會鬥鋭金旗時劍法凌厲,那是人人看見了的。難道人不可貌相,這襤褸少年竟具絶世武功?

滅絶師太心下也是暗暗吃驚︰「這少年到底是什麼路道?我擒獲他多日,一直没留心於他,原來眞人不露相,竟是個了不起的人物。我便要將靜虛如此震出,也是有所不能,當今之世,只怕唯有張三丰那老道,以百年的内功修爲,纔有這等能耐。」

她是老薑之性,老而彌辣,雖然不敢小覷無忌,却已決意與之一拚,橫著眼向他上上下下的打量。這時張無忌正忙於替鋭金旗的各人止血裹傷,手法熟練之極,伸指點了各人數處穴道,斷臂處血流立時大減。旁觀各人中自有不少療傷點穴的好手,但張無忌所使的手法,却令人自愧不如,至於他所點的奇穴,更是人所不知。掌旗副使吳勁草道︰「多謝少俠仗義,請問高姓大名。」張無忌道︰「在下姓曾,名阿牛。」滅絶師太冷冷的道︰「回過身來,好小子,接我三劍。」張無忌道︰「且慢。」替最後一個斷臂之人包紮好了傷口,這纔回身,抱拳説道︰「滅絶師太,我不是你對手,更不想和你老人家動手,只盼你們雙方兩下罷鬥,杯酒言和。」他説到「雙方兩下罷鬥」六個字之時,辭意十分誠懇,原來他心中所想到的雙方,正是已去世的父母雙親,一邉是父親武當派的名門正派,一邉是母親白眉教的邪魔外道。

滅絶師太道︰「哈哈,憑你這臭小子一言,便要咱們罷鬥?你是武林至尊麼?」張無忌心念一動,道︰「是武林至尊便怎樣?」滅絶師太道︰「你便是有屠龍刀在手,也得先跟我倚天劍決一勝負。當眞成了武林中的至尊,那時再來發號施令不遲。」峨嵋群弟子聽師父出言譏刺張無忌,附和著都笑了起來。張無忌心中,便如電光般閃了幾閃︰「難道武林中人人想找我義父,想得到那柄屠龍刀,爲的是要成爲武林至尊?那時候眞的便『號令天下,莫敢不從?』但聽得衆人譏笑之聲,在耳邉響個不停。」

以張無忌藉藉無名的身份地位,説出兩下罷鬥的話來原是大大不配,他一聽得各人譏笑,登時面紅耳赤,突然一回頭,看到了站在峨嵋群弟子中間的周芷若。她臉上露出一副仰慕傾倒的神色,眼光中意示鼓勵,更是一望而知。張無忌已衝口説道︰「你爲什麼要殺死這許多人?每個人都有父母妻児,你殺死了他們,他們家中的孩児便要伶仃孤苦,受人欺辱。你是出家人,難道心中不會安麼?」他這幾句話説得情辭懇切,旁邉站的衆人都是心中一動。張無忌原本不擅辭令,但想到自己的身世,出言便即眞摯。周芷若胸口一熱,眼眶登自紅了。

滅絶師太臉色木然,始終不顯現七情六慾,祇是冷冰冰的道︰「好小子,我用得著你來教訓麼?你自負内力深厚,在這児胡吹大氣。好,你接得住我三掌,我便放了這些人走路。」張無忌道︰「我連你徒児的一掌都躱不開,何況是師太?我不敢跟你比武,只求你慈悲爲懷,體念上天好生之德。」吳勁草大聲叫道︰「曾相公,不用跟這老賊尼多説。咱們寧可個個死在老賊尼的手下,何必要她假作寬大。」滅絶師太斜眼瞧著張無忌,問道︰「你師父是誰?」張無忌心想︰「父親、義父雖都教過我武功,却都不是我師父。」説道︰「我没有師父。」此言一出,衆人均是太感奇怪,本來心想他在一招之間震跌靜虛,自是高人之徒,各人心中都還存著三分顧忌,那知他竟説没有師父。武林中人最尊師道,不肯吐露師父姓名,那是常事,但決不敢抹煞師父的存在,他説没有師父,那眞的没有師父了。

滅絶師太不再跟他多言,説道︰「接招吧!」右手一伸,隨隨便便的拍了出去。處此情勢,張無忌不接也是不行,當下不敢大意,雙掌並推,以兩隻手掌接她一掌。不料滅絶師太手掌一低,便像一尾滑溜無比、迅捷無倫的小魚一般,從他雙掌之下穿過,波的一響,拍在他的胸前。張無忌一驚之下,護體的九陽神功自然發出,和對方拍來的掌力一擋,就在這兩股勁力將觸將離的微妙狀態之下,滅絶師太的掌力忽然無影無蹤的消失。張無忌一呆,抬頭看她時,猛地裡胸口猶似受了鐵鎚的一擊。他立足不定,向後接連摔了兩個觔斗,哇的一聲,噴出一大口鮮血,委頓在地,便似一頓軟泥。滅絶師太的掌力如此忽吞忽吐,閃爍不定,引開敵人的内力,然後再加發力,實是内家武學中精奥之極的修爲,旁觀衆人中武功深湛之士識得這一掌的妙處,忍不住喝采。

蛛児搶到張無忌身旁,急忙扶他,説道︰「阿牛哥,你\dash{}你\dash{}」張無忌但覺胸口熱血翻湧,搖了搖手,道︰「死不了。」慢慢爬起身來,只聽滅絶師太對三名女弟子道︰「將一干妖人的右臂全都砍了。」那三名女弟子應道︰「是!」挺劍走向鋭金旗的衆人。張無忌忙道︰「滅絶師太,你\dash{}你説我受得你三掌,就放他們走路,我\dash{}我挨過你一掌,還有\dash{}還有兩掌。」

滅絶師太擊了他一掌,已試出他的内功正大渾厚,並非妖邪一路,甚至和自己所學,頗有相似之處,又見他雖然袒護魔教教衆,實則不是魔教中人,説道︰「少年人别多管閒事,正邪之分,該當清清楚楚,適纔這一掌,我只用了三分力道,你知道麼?」張無忌知她以一派掌門之尊,自是不會虛言,她説只用三分力道,那就是眞的只用三分,但不論餘下的兩掌如何難挨,總不能顧全自己性命,眼睜睜讓鋭金旗人衆受她宰割,便道︰「在下不自量力,再受\dash{}再受師太兩掌。」吳勁草大叫︰「曾相公,咱們深感你的大德!你英雄仗義,人人感佩。餘下兩掌千萬不可再挨。」

張無忌道︰「滅絶師太\dash{}」只叫了四個字,口一張,又是一大口血噴出來。蛛児大急,伸手待去扶他,那知自己腿膝一麻,便又摔倒。原來她雖得無忌解穴,但血脈未曾行開,眼見無忌受傷,焦急之下,便即奔出相救,猶似一個雙腿癱瘓之人,遇到火災等事變,却會突然發足狂奔。蛛児所以能彀移動,乃仗霎時間的精神支持,過得片刻,終於站立不定。滅絶師太嫌她礙事,左手袍袖一拂,已將她身子捲起,向後擲出。周芷若搶上一步接住,將她輕輕放在地下。蛛児急道︰「周姊姊,你快勸他别挨那兩掌,你的説話,他會聽的。」周芷若奇道︰「他怎會聽我的話?」蛛児道︰「他心中很喜歡你,難道你不知道麼?」周芷若滿臉通紅,啐道︰「那有此事?」

只聽滅絶師太朗聲道︰「你既要硬充英雄好漢,那是自己找死,須怪我不得?」右手一起,風聲獵獵,直襲張無忌的胸口。張無忌這一次不敢伸掌抵擋,身形一側,意欲避開她的掌力。滅絶師太右臂斜彎急轉,那手掌竟從決不可能的彎角裡橫了過來,拍的一聲,已擊中他背心。他身子便如一束稲草在空中平平的飛了出去,重重摔摔在地下,動也不動,似已斃命。滅絶師太這一招手法精妙無比,本來旁觀衆人都會忍不住喝采,但各人對張無忌的俠義心腸均是暗中欽佩,見他慘遇不幸,只有驚呼嘆息,竟没一人叫好。

蛛児道︰「周師姊,我求求你,快去瞧他傷得重不重。」周芷若一顆心突突跳動,聽蛛児求得懇切,原想過去瞧瞧,但衆目睽睽之下,以她一個十八九歳的少女,如何敢去看視一個青年的傷勢?何況傷他之人正是自己師父,這一過去,雖非公然反叛本門,究是對師父大大的不敬,是以一時之間猶疑不決,跨了一步,却又縮回。

只見張無忌背脊一動,掙扎著慢慢坐起,但手肘撐高尺許,突然支持不住,重新跌下。這時天已大明,陽光燦爛,人人見到他身下極大的一灘鮮血。張無忌已是昏昏沉沉,只盼一動也不動的躺著,但心中仍是記著尚有一掌未挨,救不得鋭金旗衆人的性命。他深深吸一口氣,以堅強無比的意志之力,硬生生坐起。但見他身子發顫,隨時都能再度跌下,各人屏住了呼吸注視,四周雖有數百人衆,但靜得連一針落地都能聽見。

就在這萬籟倶寂的一刹那間,張無忌突然間記起了九陽眞經中的幾句話︰「他強由他強,清風拂山崗,他橫任他橫,明月照大江。」他在幽谷中誦著這幾句經文時,始終不明其中之理,這時候猛地裡想起滅絶師太之強之惡,自己決非其敵,照著九陽眞經中要義,似乎不論敵人如何強猛如何兇惡,儘可當他是清風拂山、明月映江,雖能加於我身,却不能有絲毫損傷。然則怎樣方能不損我身?經文下面説道︰「他自狠來他自惡,我自一口眞氣足。」張無忌想到此處,心下豁然有悟,盤膝坐下,依照眞經中所示的法門,一調眞氣,發覺丹田中暖烘烘地、活潑潑地,眞氣流動,頃刻間便遍於四肢百骸,那九陽神功的大威力,這時方才顯現出來,他外傷雖重,嘔血成升,但内力眞氣,竟是半點也没損耗。

滅絶師太見他運氣療傷,心下也不禁暗自訝異,這少年果是有非常之能。須知她打張無忌的第一掌乃是「飄雪穿雲掌」中的一招,第二掌更加利害,是「鐵手九式」的第三式,這都是峨嵋派掌法中精華所在。第一掌她只出三分力,第二掌將力道加到了七成,料想便算不能將他一掌斃命於當場,至少也要叫他筋斷骨折,全身委癱,再也動彈不得,那知他俯伏半晌,便又坐起,實是大出她意料之外。

依照武林中的比武習慣,滅絶師太原可不必等候對方運息療傷,但她自重身份,自不會在此時乘人之危對一個後輩動手。丁敏君大聲叫道︰「喂,姓曾的,你若是不敢再接我師父第三掌,乘早給我滾滾得遠遠的。你在這児養一輩子傷,咱們也在這児等你一輩子嗎?」周芷若細聲細氣的道︰「丁師姊,讓他多休息一會,那也礙不了事。」丁敏君怒道︰「怎麼?你\dash{}你也來袒護外人,是不是瞧著這小子\dash{}」她本來想説︰「瞧著這小子英俊,對他有了意思啦」,但立即想到各大門派的許多知名之士都在一旁,這些粗俗的言語,實是不便出口,因此話到口邉,又縮了回去。但她言下之意,旁人怎不明白?下面這句話雖然不説,實則還是和説出口一般無異。

周芷若又羞又急,氣得臉都白了,却不分辯,淡淡的道︰「小妹只是顧念本門和師尊的威名,盼望别讓旁人説一句閒話。」這個大題目一提出,不但將丁敏君譏刺之言輕輕撇在一邉,而且顯得大是理直氣壯。丁敏君愕然道︰「什麼閒話?」周芷若道︰「本門武功天下揚名,師父更是當世數一數二的前輩高人,自不會跟這種後生小子一般見識。只不過見他大膽狂妄,這纔出手教訓於他,難道眞的會要了他的性命不成?本門俠義之名,已垂之百年,師尊仁俠寬厚,誰不欽仰。這年輕人螢燭之光,如何能與日月爭輝?便算他再去練一百年,也不能是咱們師尊的對手,多養一會児傷,又算得什麼?」這一番話説得人人暗中點頭。滅絶師太心下更喜,覺得這個小徒児識得大體,在各派的高人之前,替本門增添光彩。

張無忌體内眞氣一加流轉,登時精神煥發,把周芷若的話也是句句聽在耳裡,知道她是在極力迴護自己,又用言語先行扣住,使滅絶師太不便對自己痛下殺手。不由得心中大是感激,站起身來,説道︰「師太,晩輩捨命陪君子,再挨你一掌。」滅絶師太見他只這麼盤膝一坐,立時便精神奕奕,暗道︰「這小子的内力如此渾厚,當眞邪門。」説道︰「你只管出手擊我,誰叫你挨打,不還手?」張無忌苦笑道︰「晩輩這點児粗陋的功夫,連師太的衣角也碰不到半分,説什麼還手?」滅絶師太道︰「你既有自知之明,那便乘早走開。少年人有這等骨氣,也算難得。滅絶師太掌下素不肯饒人,今日對你破一破例。」張無忌躬身道︰「多謝前輩。這些鋭金旗的大哥們你也都饒了麼?」滅絶師太的長眉斜斜垂下,冷笑道︰「我的法號叫作什麼?」張無忌道︰「前輩的尊名是上『滅』下『絶』。」滅絶師太道︰「你知道就好了。妖魔邪徒,我是要滅之絶之,決不留情。難道『滅絶』兩字,是白叫的麼?」

張無忌道︰「既如此,請前輩發第三掌。」滅絶師太斜眼相睨,如這般強頑的少年,一生之中確是從未見過,她素來心冷,但突然之間,起了愛才之念,心想︰「我第三掌一出,他非死不可,這人究非妖人一流,年紀輕輕的如此送命,不免有些可惜!」微一沉吟,心意已決,第三掌要打在他丹田的要穴之上,運内力震傷他的丹田,使他立時閉氣暈厥,待誅盡魔教鋭金旗的妖人之後,再將他救醒。她左袖一拂,第三掌正要擊出,忽聽得一人叫道︰「滅絶師太,掌下留人!」這八個字的聲音有如針尖一般的鑽入各人耳中,人人覺得極不舒服。

只見西北角上一個白衫男子手搖摺扇,穿過人叢,走近身來。這人白衫的左襟之上,繡著一雙小小的血手,五指箕張,顏色殷紅,神態極是猛惡。這人行路足下塵沙不起,便如是在水面上飄浮一般。衆人一看,便知他是白眉教中的高手人物。

原來白眉教教衆的正式法服,和魔教一般,也是白袍,只是魔教教袍上繡著一個紅色火炬,白眉教則繡著一隻血手。那人走到離滅絶師太三丈開外,拱手笑道︰「師太請了,這第三掌嘛,由區區的代領如何?」滅絶師太道︰「你是誰?」那人道︰「在下姓殷,草字野王。」

他「殷野王」三字一出口,旁觀衆人登時起了鬨。要知殷野王的名聲,這二十年來在江湖上著實響亮,他父親白眉鷹王殷天正潛心鑽研武學,將白眉教的教務都交給了児子處理,殷野王名義上只是天微堂的香主,實則便是代理教主。滅絶師太見這人不過四十來歳年紀,但一雙眼睛猶如冷電,精光四射,氣勢懾人,倒也不能小覷於他,何況平時也頗聽到他名頭,當下冷冷的道︰「這小子是你什麼人,要你代接我這一掌?」張無忌心中激動︰「他是我舅舅,是我舅舅。難道他認出我來了?」殷野王哈哈一笑,道︰「我跟他素不相識,只是見他年紀輕輕,骨頭倒硬,頗不像武林中那些假仁假義,沽名釣譽之徒。心中一喜,便想領教一下師太的功力如何?」

最後一句話説得頗不客氣,意下似乎全没將滅絶師太放在眼裡。滅絶師太却也並不動怒,對張無忌道︰「小子,你倘若還想多活幾年,這時候便走,還來得及。」張無忌道︰「晩輩不敢貪生忘義。」滅絶師太點了點頭,向殷野王道︰「這小子還欠我一掌。咱們的帳一筆歸一筆,回頭不教閣下失望便是。」殷野王嘿嘿一笑,説道︰「滅絶師太,你有能便打死這個少年。這少年若是活不了,我教你們人人死無葬身之地。」一説完這幾句話,立時飄身而退,穿過人叢,喝道︰「現身!」

突然之間,沙中湧出無數人頭,每人身前支著一塊盾牌,各持強弓,一排排的利箭,對著衆人。原來白眉教的教衆在沙中挖掘地道,早將各派人衆團團圍住了。衆人注意著滅絶師太和張無忌對掌,全没分心,便是宋青書等有識之士,也祇防備白眉教突然奔前衝擊,那料得白眉教乘著沙土鬆軟,竟然挖掘地道,冷不防佔盡了周遭有利的地形。這麼一來,人人臉上變色,眼見利箭上的箭頭在日光下發出暗藍色的光芒來,顯是餵有劇毒。倘若殷野王一聲令下,各派除了武功最強的數人之外,其餘的祇怕都要性命難保。

當地五派之中,論到資望輩份,均以滅絶師太爲長,各人一齊望著她,聽她的號令。滅絶師太的性児最是固執不過,雖然眼見情勢惡劣,竟是絲毫不爲所動,對張無忌道︰「小子,你只好怨自己命苦。」突然間全身骨骼中發出辟辟拍拍的輕微爆裂之聲,炒豆般的響聲不絶,一掌已向張無忌胸口擊去。

這一掌,乃是峨嵋的絶學,叫做「佛光普照」。任何掌法劍法總是連綿成套,多則數百招,最少也有三五式,但不論三式或是五式,定然每一式中再藏變化,一式抵得數招乃至十餘招。可是這「佛光普照」的掌法,便只一招,而且這一招也無其他變化,一招拍出,擊向敵人胸口也好,背心也好,肩頭也好,面門也好,招式平平淡淡,一成不變,其威力之生,完全在於以峨嵋派九陽神功作爲根基。一掌既出,敵人擋無可擋,避無可避。當今峨嵋派中,除了滅絶師太一人之外,再無第二人會使。她本來只想擊中張無忌的丹田,將他擊暈便罷,但殷野王出來一加威嚇之後,要是她再手下留情,那便不是寬大,而是貪生怕死,向敵人屈膝投降了。因此這一招乃是用了全力,絲毫不留餘地。

張無忌見她手掌擊出,骨骼先響,也知這一掌非同小可,自己生死存亡,便決於這頃刻之間,那裡敢有些微怠忽?

\chapter{奇人怪事}

張無忌在這一瞬之間,只是記著「他自狠來他自惡,我只一口眞氣足」這兩句經文,決不想去如何出招抵禦,但把一股眞氣,匯聚胸腹。猛聽得砰然一聲大響,滅絶師太一掌已打在他胸口。旁觀衆人都是一聲驚呼,只道無忌定然全身骨骼粉碎,説不定竟被這排山倒海般的一擊將身子打成了兩截。那知一掌過去,張無忌臉露訝色,好端端的站著,滅絶師太却是臉如死灰,手掌微微發抖。

原來適纔滅絶師太這一招「佛光普照」,純以峨嵋九陽功爲基,偏生張無忌練的正是九陽神功。那峨嵋九陽功乃當年郭襄聽覺遠和尚背誦九陽眞經後,記憶得若干片段而化成,和原本九陽神功的威力相較,自是不可同日而語。但兩種内功威力有大有小之分,性質却是一致,那峨嵋九陽功一遇到九陽神功,猶如江河入海,又如水乳交融,登時無影無蹤。張無忌胸口輕輕一震,突然間全身舒適無比,精神大振,原來滅絶師太這一掌掌力中所含的内功修爲竟在不知不覺之間,已被張無忌的九陽神功吸去。這並非張無忌有意如此,乃是兩種内功本質相同,相互生出強烈感應,弱者投在強者之中,強者自然容納。滅絶師太擊他的第一掌乃是「飄雪穿雲掌」,第二掌是「截手九式」,均非九陽功所屬,是以擊在張無忌的身上,却能使他受傷嘔血。

這中間的道理,當時却無一人能彀理會得,要知武林人士,人人知道九陽眞經乃武學總訣,當南宋末葉,已經失傳,但九陽眞經却無一人見過。唯一見過的覺遠大師却又是個不會絲毫武功之人,至於一掌之交,内力便被對方吸去,更是誰都没聽見過。張無忌固然茫然無知,滅絶師太縱然見識廣博,也只道張無忌武功深湛,自己傷他不得而已。她内力渾厚,便是連擊百掌,掌力也不會耗竭,失了一掌之力,一時之間也未察覺。是以圏子内外的數百人,除了滅絶師太自己之外,個個均以爲她手下留情,有的以爲她愛惜張無忌的骨氣,有的以爲她顧全大體,不願五派在白眉教的毒箭下傷亡太重,更有的以爲她膽小害怕,屈服於殷野王的威嚇之下。

張無忌躬身一揖,説道︰「多謝前輩掌底留情。」滅絶師太哼了一聲,大是{\upstsl{尷}}尬,若説上前再打,自己明明説過只擊他三掌,倘若就此作罷,那更是向白眉教屈服的奇恥大辱。便在這微一遲疑之間,殷野王哈哈大笑,説道︰「識時務者爲俊傑,滅絶師太不愧爲當世高人。」喝令︰「撤去弓箭!」衆教徒聽了他的號令,陡然間翻翻滾滾,退了開去,一排盾牌,一排弓箭,排列得極是整齊,看來這殷野王以兵法部勒教衆,進退攻拒之際,頗具陣法。

滅絶師太臉上無光,却又如何能向衆人分辯,自己這一掌決非手下留情?各人明明見到她輕輕兩掌,便將張無忌打得重傷,但給殷野王一嚇之後,第三掌竟是徒具威勢,一點力道也没使上。她便是竭力申辯,各人也不會相信,何況她向來高傲慣了的,豈敢去求人相信?當下狠狠的向張無忌瞪了一眼,朗聲道︰「殷野王,你要領教我掌力,這就請過來。」殷野王道︰「今日承師太之情,不敢再行得罪,咱們後會有期。」滅絶師太左手一揮,不再言語,領了衆弟子向西奔去,崑崙、華山、崆峒各派人衆,及殷利亨、宋青書等跟隨而去。蛛児雙足尚自行走不得,急道︰「阿牛哥,快帶我走。」

張無忌却很想和殷野王説幾句話,道︰「等一會児。」迎著殷野王走了過去,説道︰「前輩救援的大德,晩輩決不敢忘。」殷野王拉著他的手,向他上上下下的打量了一會,道︰「你是姓曾?」張無忌眞想撲在他懷裡,叫出聲來︰「舅舅,舅舅!」但終於強行忍住,兩眼却不自禁的紅了。

有言道是︰「見舅如見娘」,張無忌父母雙亡,殷野王是他十年來第一次所見到的親人,如何不教他心情激動?殷野王見他眼色之中,顯得對自己十分親近,還道他感激自己救他性命,也不放在心下,眼光轉到躺在地下的蛛児時,淡淡的一笑,説道︰「阿離,不認得我了麼?」蛛児臉色大變,顫聲叫道︰「爹!」

這個「爹」字一出口,張無忌大吃一驚,但隨即明白了許多事情︰「原來蛛児是舅舅的女児,那便是我的表妹了。她殺了二娘,累死了自己母親,又説她爹爹一見到她便要殺她\dash{}哦,她用『千蛛絶戸手』戳死殷無祿,大槩這三個家人跟著主人,也對她母女不好。殷無福、殷無壽雖然恨她,却不能跟她動手,是以説了一句『原來是小姐』,便抱了殷無祿的屍身而去。」他回頭瞧著蛛児時,忽又想到︰「怪不得我總是覺得她舉動像我媽媽,那知道她和我有血肉之親,我媽是她嫡親的姑母。」

只聽殷野王冷笑道︰「你還知道叫我一聲爹,哼,我只道你跟了金花婆婆,便將白眉教不瞧在眼裡了。没出息的東西,跟你媽一模一樣,練什麼『千蛛絶戸手』,哼,你找面鏡子自己瞧瞧,成什麼樣子,我姓殷的家中有你這樣的醜八怪?」蛛児本來嚇得全身發顫,突然間抬起頭來,凝視著父親的臉,朗聲道︰「爹,你不提從前的事,我也不提,你既要説,我倒要問你,媽好好的嫁了你,爲什麼又要另娶二娘?」殷野王道︰「這\dash{}這\dash{}死丫頭,男子漢大丈夫那一個不有三妻四妾?你作逆不道,今日狡辯也是無用。什麼金花婆婆、銀葉先生,白眉教也没放在眼裡。」回手一揮,對殷無福、殷無壽兩人説道︰「帶了這丫頭走。」

張無忌雙手一攔,道︰「且慢!殷\dash{}殷前輩,你要拿她怎樣?」殷野王道︰「這丫頭是我的親生逆女,她毒死庶母,累死親母,如此禽獸不如之人,怎能留於世間?」張無忌道︰「那時殷姑娘年幼,見母親受人欺辱,一時不忿,做錯了事,還望前輩念在父女之情,從輕責罰。」殷野王仰天大笑,説道︰「好小子,你究竟是那一號的人物,連我殷家的事也要插手管了起來?你是『武林至尊』不是?」張無忌一時衝動,眞想便説︰「我是你外甥,可不是外人。」但話到口邉,還是忍住了。殷野王笑道︰「小子,你今天的性命是撿來的,再這般多管江湖上的閒事,再有十條小命,也不彀賠。」説著左手一擺,殷無福、殷無壽二人上前架起蛛児,拉到殷野王身後。

張無忌知道蛛児,落入她父親手中,性命多半無倖,情急之下,衝了上去便要搶人。殷野王眉頭一皺,左手陡地伸出,抓住張無忌的胸口,輕輕往外一揮。張無忌身不由主,便如騰雲駕霧般的直摔出去,砰的一聲,重重摔在黃沙之中。他有九陽神功護體,自是不致受傷,但陥身沙内,眼耳口鼻之中塞滿了沙子,難受之極。張無忌不肯干休,爬起來又搶上去。殷野王冷笑道︰「小子,第一下我手下留情,再一下可不客氣了。」張無忌懇求道︰「她\dash{}她是你親生女児啊,她小的時候你抱過她,親過她,你饒了她吧。」殷野王心念一動,瞧了蛛児一眼,但見到她浮腫的臉,不由得厭惡之情大增,喝道︰「走開!」張無忌反而走上一步,便想搶人。

蛛児叫道︰「阿牛哥,你别理我,我永遠記得你的好心。你快走開,你打不過我爹爹的。」便在此時,黃沙中突然間鑽出一個青袍人來,雙手一長,已抓住殷無福、殷無壽兩人的後領,跟著雙手一合,兩人額頭對額頭猛撞一下,登時暈去,那人抱起蛛児,疾馳而去。殷野王怒喝︰「青翼蝠王,你也來多管閒事?」

青翼蝠王韋一笑縱聲長笑,抱著蛛児向前急馳,他名叫「一笑」,這笑聲却是連綿綿不絶,何止百笑千笑?殷野王和張無忌一齊發足急追。這一次韋一笑不再大兜圏子,一逕向東南飄行。這人身法之快,實是匪夷所思,殷野王内力深厚,輕功了得,張無忌體内眞氣流動,更是越奔越快,但韋一笑快得更加厲害。眼見初時和他相距數丈,到後來變成十餘丈、二十餘丈、三十餘丈\dash{}終於人影不見。殷野王怒極而笑,見張無忌始終和自己並肩而馳,半步也没落後,心下暗自驚異,這時明知已無法追上韋一笑,却要考一考張無忌的脚力,足底加勁,身子如箭離弦,激射而出。但見張無忌不即不離,仍是和他並肩而行,忽聽張無忌道︰「殷前輩,這青翼蝠王奔跑雖快,未必長力也彀,咱們跟他死纏到底。」

殷野王吃了一驚,立時停步,自忖︰「我施展如此輕功,已是竭盡平生之力,别説開口説話,便是換錯了一口氣也是不成。這小子隨口説話,居然足下絲毫不慢,那是什麼邪門?」他陡然間停步,張無忌一竄已在十餘丈外,忙轉身回頭,退回到殷野王身旁,聽他示下。殷野王道︰「曾兄弟,你師父是誰?」張無忌忙道︰「不,不!你千萬不能叫我兄弟,叫我『阿牛』好了。我没有師父。」殷野王心念一動︰「這小子的武功如此怪異,留著大是禍胎,不如出其不意,一掌打死了他。」便在此時,忽聽得幾下極尖鋭的海螺之聲,傳了過來,正是白眉教有警的訊號。殷野王眉頭一皺,心想︰「定是洪水、烈火各旗怪我不救鋭金旗,又起了亂子。倘若一掌打不死這小子,這時候却没功夫與他纏鬥。不如借刀殺人,讓他去送命在韋一笑手裡。」便道︰「白眉教遇上了敵人,我須得趕回應付,你去找韋一笑吧。這人兇惡陰險,待得遇上了,你須先下手爲強。」

張無忌道︰「我本領低微,怎打得過他?你們有什麼敵人來攻?」殷野王側耳聽了一下號角,道︰「果然是魔教的洪水、烈火、厚土三旗都到了。」張無忌道︰「大家都是魔教一派,又何必自相殘殺?」殷野王臉一沉,道︰「小孩子懂得什麼?」轉身向來路奔回。

張無忌心想︰「蛛児落入了大惡魔韋一笑手中,倘若給他在咽喉上咬了一口,吸起血來,那裡還有性命在?」想到此處,更是著急,當即吸一口氣,發足便奔。好在韋一笑輕功雖佳,手上抱了一個人後。總不能踏沙無痕,沙漠之中還是留下了淡淡的一條足跡。張無忌打定了主意︰「他休息,我不休息,他睡覺我不睡覺,奔跑三日三夜,好丁也追上了他。」

可是在烈日之下,黃沙之中,奔跑三日三夜當眞是談何容易,他奔到傍晩,已是口乾唇燥,全身汗如雨下。但説也奇怪,脚下却毫不疲累,積蓄了數年的九陽神功一點一滴的發揮出來,越是使力,越是精神奕奕。他在一處泉水中飽飽的喝了一肚水,足不停步的奔跑。

奔到半夜,眼見月在中天,張無忌忽地恐懼起來,只怕突然之間,蛛児被吸乾了血的屍體在眼前出現。就在這時,隱隱聽得身後似有足步之聲,張無忌回頭一看,却没有人。他不敢耽擱,發足又跑,但背後的脚步聲立時跟著出現。張無忌大奇,回頭再看,仍是無人,仔細一看,沙漠中明明有三道足跡,一道是韋一笑的,一道是自己的,另一道却是誰的?再回過頭來時,身前只一道足跡。那麼有人在跟縱自己,定然無疑的了,怎麼總是瞧不見他,難道這人有隱身術不成?

他滿腹疑團,拔足又跑,身後的足步聲又再響起。張無忌叫道︰「是誰?」身後一個聲音道︰「是誰?」張無忌大吃一驚,喝道︰「你是人是鬼?」那聲音也道︰「你是人是鬼?」

張無忌急速轉過身來,這一次看到了身後那人留在地下的一點影子,才知那是個身法快的無與倫比之人,躱在自己背後。他叫道︰「你跟著我幹麼?」那人道︰「我跟著你幹麼?」張無忌笑道︰「我怎麼知道?所以要問你啊。」那人道︰「我怎麼知道?所以要問你啊。」張無忌見這人似乎並無多大惡意,否則他在自己身後跟了這麼久,隨便什麼時候一出手,都能致自己死命,便道︰「你叫什麼名字?」那人道︰「説不得。」張無忌道︰「爲什麼説不得?」那人道︰「説不得就是説不得,還有什麼道理好講。你叫什麼名字?」張無忌道︰「我\dash{}我叫曾阿牛。」

那人道︰「假的。」張無忌吃了一驚,心想︰「他怎麼知道?」問道︰「爲什麼是假的?」那人道︰「假的就是假的,眞眞假假,還不是一般。我問你,你半夜三更的狂奔亂跑,在幹什麼?」張無忌知道這是一位身懷絶技的異人,便道︰「我一個朋友給青翼蝠王捉了去,我要去救回來。」那人道︰「你救不回來的。」張無忌道︰「爲什麼?」那人道︰「青翼蝠王的武功比你強,你打他不過。」張無忌道︰「打他不過也要打。」那人道︰「很好,有志氣。你朋友是姑娘麼?」張無忌道︰「是的,你怎麼知道?」那人道︰「要不是姑娘,少年人怎會甘心拚命。很美吧?」張無忌道︰「醜得很?」那人道︰「你自己呢,醜不醜?」張無忌道︰「你到我面前,就看到了。」那人道︰「我不要看,那姑娘會武功麼?」張無忌道︰「會的,是白眉教殷野王前輩的女児,曾跟靈蛇島金花婆婆學武。」那人道︰「不用追了,韋一笑捉到了她,一定不肯放。」張無忌道︰「爲什麼?」

那人哼了一聲道︰「你是個傻瓜,不會用腦子,殷野王是殷天正的什麼人?」張無忌道︰「他們兩位是父子之親。」那人道︰「白眉鷹王和青翼蝠王的武功誰高?」張無忌道︰「我不知道。請問前輩,是誰高啊?」那人道︰「我也不知道。兩個人誰的勢力大些?」張無忌道︰「鷹王是白眉教教主,想必勢力大些。」那人道︰「不錯。因此韋一笑捉了殷天正的孫女,那是奇貨可居,不肯就還的,他想要挾殷天正就範。」張無忌搖頭道︰「只怕做不到,殷野王前輩一心一意想殺了他自己女児。」那人奇道︰「爲甚麼啊?」張無忌於是將蛛児毒死父親愛妾、累死親母之事簡略説了。

那人聽完後,嘖嘖讚道︰「了不起,了不起,當眞是美質良材。」張無忌奇道︰「什麼美質良材?」那人道︰「小小年紀,就會毒死庶母、害死親母,再加上靈蛇島金花婆婆的一番調教,當眞是我見猶憐。韋一笑要收她作個徒児。」張無忌吃了一驚,問道︰「你怎麼知道?」那人道︰「韋一笑是我好朋友,我自然知道他的心性。」

張無忌一呆之下,大叫一聲︰「糟糕!」發足便奔。那人仍是緊緊的跟在他背後。張無忌一面奔跑,一面問道︰「你怎麼又跟著我?」那人道︰「我好奇心起,要瞧瞧熱鬧。你還追韋一笑幹麼?」張無忌怒道︰「蛛児已經有些邪氣,我決不許她再拜韋一笑爲師。倘若也學成一個吸飲人血的惡魔,那怎生是好?」那人道︰「你很喜歡蛛児麼?爲什麼這般関心?」張無忌嘆了口氣,道︰「我不喜歡她,不過她\dash{}她有點児像我媽媽。」那人道︰「{\upstsl{嗯}},原來你媽媽也是個醜八怪,想來你也好看不了。」張無忌急道︰「我媽媽很是好看的,你别胡説八道。」那人道︰「可惜,可惜!」張無忌道︰「可惜什麼?」那人道︰「你這少年周身血性,著實不錯,可惜轉眼便是一具吸乾了血的僵屍。」

張無忌心念一動︰「他的話確也不錯,我就算追上了韋一笑,又怎能救得蛛児,也不過是白白饒上自己性命而已。」説道︰「前輩,你幫幫我,成不成?」那人道︰「不成。一來韋一笑是我好朋友,二來我也未必打得過他。」張無忌道︰「韋一笑既是你好朋友,你怎地不勸勸他?」那人長嘆一聲,道︰「勸有什麼用?韋一笑自己又不想吸飲人血,他是迫不得已,實是痛苦難當。」張無忌奇道︰「迫不得已?那有此事?」那人道︰「韋一笑練内功時走火,自此每次激引内力,必須飲一次人血,否則全身寒戰,立時凍死。」張無忌沉吟道︰「那是三陰脈絡受損麼?」

那人奇道︰「咦,你怎麼知道?」張無忌道︰「我只是猜測,不知對不對。」那人道︰「我曾三入長白山,想替他找一頭火蟾眼目,治療此病,但三次都是徒勞無功。第一次還見了火蟾,差著兩丈没捉到,第二次第三次連火蟾的影子也没見到。待眼前的難関過了之後,我總還得再去一次。」張無忌道︰「我同你一起去,好不好?」那人道︰「{\upstsl{嗯}}!你内力倒彀,就是輕功太差,那時再説吧。喂,我問你,幹麼你要去幫忙捉火蟾?」張無忌道︰「倘若捉到了,不但治好韋一笑的病,也救了很多人,那時候他不用再吸人血了。啊,前輩,他奔跑了這麼久,激引内力,是不是迫不得已,只好吸蛛児的血呢?」那人一呆,道︰「這倒説不定。他雖想收蛛児爲徒,但要是打起寒戰來,自己血液要凝結成冰,那時候啊,只怕便是自己的親生女児\dash{}」

張無忌越想越怕,捨命狂奔,那人忽道︰「咦,你後面是什麼?」張無忌回過頭來一看,突然間眼前一黑,全身已被一隻極大套子套住,跟著身子懸空,似乎是處身在一隻布袋之中,被那人揹在肩頭。張無忌伸手去撕那布袋,豈知那袋子非綢非革,堅韌異常,摸上去布紋宛然,顯是粗布所製,但撕上去紋絲不動。那人拍的一下,隔著袋子在無忌屁股上打了一記,笑道︰「小子,乖乖的在我乾坤袋中不要動,我帶你到一個好地方去。你開口説一句話,被人知覺了,我可救不得你。」張無忌道︰「你帶我到那裡去?」那人笑道︰「你已落入我乾坤袋中,我要取你小命,你逃得了麼?你只要不動不作聲,總有你的好處。」張無忌一想這話倒也不錯,當下便不掙扎。

那人提起袋子往地下一擲,哈哈大笑,説道︰「你能鑽出我的布袋,算你本事。」張無忌運起内力,雙手往外猛推,但那袋子軟軟的決不受力。他提起右脚,用力一脚踢出,波的一聲悶響,那袋子微微向外一凸,不論他如何拉推扯撕,翻滾頂撞,這隻布袋總是死樣活氣的不受力道。那人笑道︰「你服了麼?」張無忌道︰「服了!」那人道︰「你能鑽進我的布袋,是你的福緣。」提起布袋往肩頭上一掮,拔足便奔。

張無忌道︰「蛛児怎麼辦啊?」那人道︰「我怎麼知道?你再囉唆一聲,我把你從布袋裡抖了出來。」張無忌心想︰「你把我抖了出來,正是求之不得。」嘴裡却不敢答話,只覺那人脚下迅速之極,自己身子不輕,但他掮了自己,竟和空身走路無甚分别。

那人走了幾個時辰,張無忌在布袋中覺得漸漸熱了起來,知道已是白天,太陽曬在袋上,過了一會,只覺那人越走越高,似在上山。這一上山,又是上了兩個多時辰,張無忌這時身上已頗有寒意,心想︰「多半是到了極高的山上,峰頂積雪,所以這麼冷。」突然之間,身子飛了起來,他大吃一驚,忍不住叫出聲來。

他叫聲未絶,只覺身子一頓,那人已然著地,張無忌這纔明白,原來那人是帶了自己,正在縱躍,心想身處之地多半是極高的山峰上的危崖絶壁,那人揹負自己,如此跳躍,山岩積雪,甚是滑溜,倘若一個失足,豈不是兩人都一齊粉身碎骨?心中剛想到此處,那人又已躍起。

這人不斷的跳躍,忽高忽低,忽近忽遠,張無忌藏身在布袋之中,但也猜想得到當地的地勢險峻異常。當張無忌被那人帶著又一次高高躍起時,忽聽得遠處一個聲音叫道︰「説不得,怎麼到這時候纔來?」負著張無忌的那人道︰「路上遇到了一點児小事,韋一笑到了麼?」遠處那人道︰「没見啊,眞奇怪,連他也會遲到。説不得,你見到他没有?」遠處那人一面問,一面走近。張無忌暗自奇怪︰「原來這個人就叫『説不得』,無怪我問他叫什麼名字,他説是『説不得』,再問他爲什麼説不得,他説道『説不得就是説不得,那有什麼道理好講。』怎麼一個人會取這樣一個怪名?」又想︰「原來他和韋一笑是約好了在地相會的?却不知蛛児是否無恙?我落入了他的布袋之中,他又是韋一笑的好朋友,不知要如何對付我?」

祇聽那説不得道︰「鐵冠道児,咱們去找找韋兄去,我怕他出了什麼亂子。」那鐵冠道人道︰「青翼蝠王精警聰明,武功卓絶,那會有什麼亂子?」説不得道︰「我總覺得有些不對。」忽聽得一個聲音從底下山谷中傳了上來,叫道︰「説不得臭和尚,鐵冠老雜毛,快來幫忙,糟糕之極了。」説不得和鐵冠道人一齊驚道︰「是周顚,他什事情糟糕。」説不得又道︰「他好像受了傷,怎地説話時中氣如此衰弱?」他不等鐵冠道人答話,揹了張無忌便往下面躍去。鐵冠道人跟在後面,忽道︰「啊!周顚負著什麼人,是韋一笑!」

説不得叫道︰「周顚休慌,我們來助你了。」周顚笑道︰「慌你媽的屁,我慌什麼?吸血蝙蝠的老命要歸天!」説不得驚道︰「韋兄怎麼啦,受了什麼傷?」説著加快脚步。張無忌身在袋中,更如騰雲駕霧一般,忍不住低聲道︰「前輩,你暫且放下我,下去救人要緊。」説不得突然提起袋子,在空中轉了三個圏子,張無忌大吃一驚,倘若他一脱手,將布袋擲了出去,那後果當眞是不堪設想。祇聽説不得沉著嗓子道︰「小子,我跟你説,我是『布袋和尚説不得』,後面那人是鐵冠道人張中,下面説話的是周顚,咱們三個人,再加上冷面先生冷謙,彭瑩玉和尚,是魔教中的五散人。你知道魔教麼?」張無忌道︰「知道。原來大師也是魔教中人。」説不得道︰「我和周顚不大愛殺人,鐵冠道人、冷面先生、彭和尚他們,却是素來殺人不眨眼的。他們若是知道你藏在我這乾坤袋中,隨隨便便的給你一下子,你就筋碎骨裂,變成一團肉泥。」張無忌道︰「我又没有得罪貴教,爲什麼\dash{}」説不得道︰「鐵冠道人他們殺人,還要問得罪不得罪麼?從此之後,你若想活命,不得再在我袋中説出一個字來,知道麼?」張無忌點了點頭。説不得道︰「你怎麼不回答?」張無忌道︰「你不許我説出一個字來啊。」説不得微微一笑,道︰「你知道就好\dash{}啊,韋兄怎麼了?」

最後一句話,却是跟周顚説的,只聽周顚那啞嗓子説道︰「他\dash{}他\dash{}糟之透頂,糟之透頂。」説不得道︰「{\upstsl{嗯}},韋兄心口還有一絲暖氣,周顚,是你救他來的?」周顚道︰「廢話,難道是他救我來的?」鐵冠道人張中道︰「周顚,你受了什麼傷?」周顚道︰「我見吸血蝙蝠僵在路旁,凍得氣都快没有了,不合強盜發善心,運氣助他,那知吸血蝙蝠身上的陰毒當眞厲害,就是這麼一回事。」

説不得道︰「周顚,你這一次當眞是做了好事。」周顚道︰「什麼好事壞事,吸血蝙蝠此人又陰又古怪,我平素瞧著最不順眼,不過這一次他做的事很合周顚胃口,周顚便救他一會。想不到没救到吸血蝙蝠,陰寒入體,反而賠上周顚的老命。」鐵冠道人驚道︰「你傷得這般厲害?」周顚道︰「報應,報應。吸血蝙蝠和周顚生平不做好事,那知道一做好事便橫禍臨頭。」説不得問道︰「韋兄做了什麼好事?」周顚道︰「他激引内毒,陰寒發作,本來只須吸飲人血,便能抑制。可是他身旁明明有一個少女,他寧願自己送命,也不吸她的血,周顚一見之下,説道︰『啊喲不對,吸血蝙蝠倒行逆施,周顚也得胡作胡爲一下,周顚要救他一救。』」

張無忌身在布袋之中,聽得韋一笑没吸飲蛛児的血,眞是一喜非同小可。説不得反手在布袋外一拍,問道︰「那少女是誰?到那裡去了?」周顚道︰「我也這般問吸血蝙蝠,他説這是白眉老児的孫女,名叫殷離,吸血蝙蝠已收他爲徒,萬萬不能吸她的血。」説不得和鐵冠道人一齊鼓掌,説道︰「韋兄一念之善,或許便是我教中興的轉機,青蝠和白鷹兩王擕手,明教便聲勢大振了。」説不得説著。將韋一笑身子接了過來,驚道︰「他全身冰冷,那怎麼辦?」周顚道︰「所以我説你們快活得太早了些,吸血蝙蝠這條老命十成中已去了九成,一隻死蝙蝠和白眉鷹王擕手,於明教有什麼好處?」鐵冠道人道︰「你們在這児等一會,我下山去找個活人來,讓韋兄飽飲一頓人血。」説罷縱身便欲下山。周顚叫道︰「且慢!鐵冠雜毛,這児如此荒涼,等你找到了人,只怕韋一笑早就變成了韋不笑。説不得,你布袋中那個小子,拿出來給韋兄吃了罷。」張無忌一驚︰「原來他們早瞧出我藏身布袋之中。」説不得道︰「不成,這個人於本教有恩,韋兄若是吃了他,五行旗非跟韋兄拚命不可。」於是將張無忌如何身受滅絶師太三掌重擊,救活鋭金旗下數十名好手的事簡略説了,又道︰「當時我混在白眉教的隊伍之中,瞧得清清楚楚。這麼一來,五行旗還不死心塌地的服了這小子麼?」鐵冠道人問道︰「你把他裝在袋中,奇貨可居,想收服五行旗麼?」説不得道︰「説不得,説不得!總而言之,本教四分五裂,眼前大難臨頭,白眉教偏又跟五行旗打了個落花流水,咱們總得擕手一致,纔免覆滅。袋中這人有利於本教諸路人馬擕手,那是決然無疑的。」

他説到這裡,伸出右手,貼在韋一笑後心的「靈台穴」上,運起眞氣,助他抵禦寒毒。周顚嘆道︰「説不得,你爲朋友賣命,那是没得説的,可是你小心自己的老命。」鐵冠道人道︰「我來助你一臂之力。」伸右掌和説不得的左掌相接,兩股眞力,同時衝入韋一笑的體内。

過了一頓飯時分,韋一笑低低呻吟一聲,醒了過來,但牙関仍是不住相擊,顯然冷得厲害,顫聲道︰「周顚,鐵冠兄,多謝你兩位相救。」他對説不得却不言謝,須知兩人是過命的交情,口頭的道謝反而顯得多餘了。鐵冠道人功力深湛,但被韋一笑體内的陰毒逼了過來,奮力相抗,一時却説不出話來。説不得也是如此。

忽聽得東面山峰上飄下錚錚錚的幾下琴聲,中間挾著一聲清嘯。周顚道︰「冷面先生和彭和尚尋過來啦。」提高聲音叫道︰「冷面先生,彭和尚,有人受了傷,還是你們滾過來吧!」那邉琴聲錚的一響,示意已經聽到,彭和尚却問道︰「誰\dash{}受\dash{}了\dash{}傷\dash{}啦\dash{}」那聲音遠遠傳來,山谷鳴響。周顚低聲罵道︰「性急鬼,一會児也等不得。」

\chapter{義氣深重}

祇聽得彭和尚急急地道︰「到底是誰受了傷?説不得没事吧?鐵冠兄呢?周顚,你怎麼説話中氣不足?」他問一句,人便躍近數丈,待得問完,身子已到了近處,驚道︰「啊喲,是韋一笑受了傷。」周顚道︰「你慌慌張張,老是先天下之急而急。冷面兄,你來給想個法子。」最後那句話,却是向冷面先生冷謙説的。冷謙{\upstsl{嗯}}了一聲,却不答話,他和彭和尚定要細問端詳,自己大可省些精神。果然彭和尚一連串的問話,連珠價迸將出來,周顚説話偏又顚三倒四,待得説完經過,説不得和鐵冠道人也已運氣完畢。

彭和尚道︰「我從東北方來,得悉少林派掌門人空聞大師親率師弟空智大師,以及諸代弟子百餘人,正趕赴光明頂,參與圍攻我教。」冷謙道︰「正東,武當五俠!」他説話極是簡捷,便是殺了頭也不肯多説半句廢話,他説這六個字,意思是説︰「正東方有武當五俠來攻。」至於武當五俠是誰,反正大家都知到是宋遠橋、兪蓮舟、張松溪、殷利亨和莫聲谷,那也不必多費唇舌。

彭和尚道︰「六派分進合擊,漸漸合圍。五行旗接了數仗,情勢很是不利,眼前之計,咱們只有先到光明頂去。」周顚怒道︰「放你媽的狗臭屁!楊逍那小子不來求咱們,明教五散人便挨上門去嗎?」彭和尚道︰「周顚,眼前明教大禍臨頭,倘若六派攻破光明頂,滅了聖火,咱們還能做人嗎?楊逍得罪五散人固是他不對,但咱們助守光明頂,却不是爲了楊逍,而是爲了明教。」説不得也道︰「彭和尚的話不錯。楊逍雖然無禮,但護教事大,私怨事小。」周顚罵道︰「放屁,放屁!兩個禿驢一齊放屁,臭不可當。鐵冠道人,楊逍當年打碎你的左肩,你還記得麼?」鐵冠道人沉吟不答,過了半晌,纔道︰「護教禦敵,乃是大事。楊逍的帳,待退了外敵再算。那時咱們五散人聯手,不怕小子不低頭。」

周顚「哼」了一聲,道︰「冷謙,你怎麼説?」冷謙道︰「同去?」周顚道︰「你也向楊逍屈服?當時咱們立過誓,説明教之事,咱們五散人決計從此袖手不理。難道從前説過的話都是放屁的麼?」冷謙道︰「都是放屁!」周顚大怒,霍地站起,道︰「你們都放屁,我可説的是人話。」鐵冠道人道︰「事不宜遲,快上光明頂吧!」彭和尚勸周顚道︰「顚兄,當年大家爲了爭立教主之事,翻臉成仇,楊逍固然心胸狹窄,但細想起來,五散人也有不是之處\dash{}」周顚怒道︰「胡説八道,咱們五散人又不是想當教主,有什麼錯了?」説不得道︰「本教過去的是是非非,便是爭他一年半載,也是無法分辯明白。周顚,我問你,你是明尊天聖座下的弟子不是?」周顚道︰「是啊!」説不得道︰「今日本教大禍臨頭,咱們倘若袖手,死後見不得明尊。你要是害怕中原六大派,那就休去。咱們在光明頂上戰死殉教,你來收我們的骸骨吧!」

周顚跳起身來,一掌便往説不得臉上打去,罵道︰「放屁!」只聽得拍的一聲響,説不得已重重挨了一掌,他慢慢張口,吐出十幾枚被打落的牙齒,一言不發,但見他半邉臉頰由白變紅,再由紅變瘀,腫起老高。彭和尚等人大吃一驚,周顚更是呆了。要知説不得的武功和周顚乃是在伯仲之間,周顚隨手一掌,他或是招架,或是閃避,無論如何打他不中,那知他聽由挨打,這一掌竟受了重傷。周顚心中好過意不去,叫道︰「説不得你打還我啊,不打還我,你就不是人。」説不得淡淡一笑,道︰「我有氣力,留著去打敵人,打自己人幹麼?」周顚大怒,提起手掌,重重的在自己臉上打了一下,波的一聲,也吐出了十幾枚牙齒。

彭和尚驚道︰「周顚,你這是搗什麼鬼?」周顚怒道︰「我不該打了説不得,叫他打還,他又不打,我只好自己動手。」説不得道︰「周顚,你我情若兄弟,咱們四人便要去戰死在光明頂上,生死永期,你打我一掌,算得什麼?」周顚心中激動,放聲大哭,説道︰「我也去光明頂,楊逍的舊帳,暫且不跟他算了。」彭和尚大喜,説道︰「這纔是好兄弟呢。」

張無忌身在袋中,但對五人的話都聽得清清楚楚,心想︰「這五人武功極高,那是不必説了,難得的是大家義氣深重。明教之中高人輩出,難道個個都是邪魔外道麼?」心中正自{\upstsl{嘀}}怙,忽覺身子移動,想是説不得又負了自己,直上光明頂去。他得悉蛛児無恙之後,心下已無擔憂之事,所関懷者,只是中原武林六大門派圍剿明教,不知將來如何了局,又想上到了光明頂後,當可遇到幼時小友楊不悔,她長大之後,不知是否會認得自己。

説不得等五人負著兩人,行了一日一夜,到次日午後,張無忌忽覺那布袋是著地拖拉,初時不明其理,後來自己的腦袋稍稍一抬,額頭便在一塊岩石上重重碰了一下,好不疼痛,這纔明白,原來各人是在山腹的隧道中行走。這隧道中寒氣奇重,透氣也不大順暢,直行了大半個時辰,這纔鑽出山腹,又向上升。但上升不久,又鑽入了隧道,前後一共過了五個隧道,纔聽周顚叫道︰「楊逍,吸血蝙蝠和五散人來找你啦!」

過了半晌,聽得一個人聲音在前面説道︰「眞想不到蝠王和五散人大駕光臨,楊逍没能遠迎,還望恕罪。」周顚道︰「你假惺惺作甚?你肚中定在暗罵,五散人説話有如放屁,説過永遠不上光明頂,永遠不理明教之事,今日却又自己送上門來。」楊逍道︰「小弟正自憂愁,六大派四面圍攻,小弟孤掌難鳴,今得蝠王和五散人瞧在明尊天聖的面上,仗義相助,實是本教之福。」周顚道︰「你知道就好啦。」當下楊逍請五散人入内,僮児送上茶水。

突然之間,那僮児「啊」的一聲慘呼,張無忌身在袋内,却也毛骨悚然,不知是何緣故,過了好一會,却聽韋一笑説道︰「左使者,傷了你一個僮児,韋一笑以後當圖報答。」他説話時精神飽滿,和這些時來的氣息奄奄大不相同。張無忌心中一凜︰「他吸了這僮児的熱血,害死一條人命,自己的寒毒便抑制住了。」聽楊逍淡淡的道︰「咱們之間,還説什麼報答不報答?蝠王上得光明頂來,那便是瞧得起我。」

這七人個個是明教中頂児尖児的高手,雖然互有心病,但眼下大敵當前,七人一旦相聚,各人均是精神一振。食用一些點心後,便即商議禦敵之計。説不得將布袋放在脚邉,張無忌雖然又飢又渴,却記著説不得吩咐過的言語,只聽六個人分别估量敵方實力,只有冷謙靜靜聽著,一言不發。

七人商議了一會,彭和尚道︰「紫衫龍王和金毛獅王不知去向,光明右使存亡難卜,這三位是不必説了。眼前最不幸之事,是五行旗和白眉教的樑子越結越深,前幾日大鬥一場,雙方死傷均重。倘若他們也能到光明頂上,别説六大派圍攻,便是十二派、十八派,明教也有必勝把握。」説不得在布袋上輕輕踢了一脚,説道︰「袋中這個小子,和白眉教頗有淵源,最近又於五行旗有恩,將來或能著落在這小子身上,調處雙方嫌隙。」韋一笑冷冷的道︰「教主的位子一日不定,本教的紛爭一日不解,憑他有天大的本事,這嫌隙總是不能調處。左使者,在下要問你一句,退敵之後,你擁何人爲主?」楊逍淡淡的道︰「聖火令歸誰所有,我便擁誰爲教主,這是本教的祖規,你又問我作甚?」

韋一笑道︰「聖火令失落已達百年,難道聖火令一日不出,明教便一日没有教主?六大門派所以膽敢圍攻光明頂,没將本教瞧在眼裡,還不是因爲知道本教乏人統屬,内部四分五裂之故。」説不得道︰「韋兄這話是不錯的。我布袋和尚既非殷派,亦非韋派,是誰做教主都好,總之是要有個教主。就算没教主,有個副教主也好啊,號令不齊,如何抵禦外侮?」鐵冠道人道︰「説不得之言,正獲我心。」

楊逍變色道︰「各位上光明頂來,是助我禦敵呢,還是來跟我爲難?」周顚哈哈大笑,道︰「楊逍,你不願推選教主的用心,難道我周顚不知道麼?明教没有教主,便以你光明左使爲尊。哼哼,你職位雖然最高,旁人不聽你的號令,又有何用?你調得動五行旗麼?四大護教法王肯服你指揮麼?咱們五散人更是閒雲野鶴,没當你光明左使者是什麼東西!」楊逍霍地站起,冷冷的道︰「今日外敵相犯,楊逍無暇和各位作此口舌之爭,各位若是對明教存亡甘願袖手旁觀,便請下光明頂去吧!楊逍只要不死,日後再圖一一奉訪。」

彭和尚勸道︰「楊左使,你也不必動怒。六大派圍攻明教,凡是本教弟子,人人有関,又不是你一個人之事。」楊逍冷笑的道︰「只怕本教却有人希望楊逍給六大派宰了,好拔去了這口眼中釘。」周顚道︰「你説的是誰?」楊逍道︰「各人心中明白,何用多言?」周顚怒道︰「你是説我嗎?」楊逍眼望他處,不予理睬。

彭和尚見周顚眼中放出異光,似乎便欲起身和楊逍動手,忙勸道︰「古人道得好︰兄弟鬩於牆,外禦其侮。咱們便算吵得天翻地覆,説什麼也是教中自己兄弟。教主之議,暫且擱下不提,咱們且商量禦敵之計。」楊逍道︰「瑩玉大師識得大體,此言甚是。」周顚大聲道︰「好啊,彭賊禿識得大體,周顚便祇識小體?」他激發了牛性,什麼也不顧得,喝道︰「今日偏要議定這教主之位,周顚主張韋一笑出任明教的教主。吸血蝙蝠武功高強,機謀多端,本教之中誰也及不上他。」其實周顚平時和韋一笑並没什麼交情,相互間惡感還多於好感,但他存心氣惱楊逍,便推了韋一笑出來。

楊逍哈哈一笑,道︰「我瞧還是請周顚當教主的好,明教眼下已是四分五裂的局面,再請周大教主來顚而倒之,倒而顚之一番,那纔好看呢!」周顚大怒,喝道︰「放你媽的狗臭屁!」呼的一掌,便向楊逍頭頂拍落。

適纔他一掌打落説不得十多枚牙齒,乃因説不得不避不架,但楊逍豈是易與之輩?他在十餘年之前,便因立教主之事,與明教五散人起了重大爭執,當時五散人立誓不上光明頂,今日却又破誓重來,楊逍心下已暗自起疑,待見周顚突然出手,只道五散人約齊韋一笑,前來圖謀自己,驚怒之下,右掌揮出,往周顚的手掌上迎了過去。韋一笑站在旁邉,見楊逍掌心中隱隱有青氣流轉,知他已將「青竹手」練成,周顚傷後原氣未復,萬萬抵敵不住,立即手掌拍出,搶在頭裡,接了楊逍這一掌。兩人手掌相交,竟是無聲無息的黏一起。

原來楊逍雖和周顚有隙,但念在同教之誼,究不願一掌便傷他性命,因此這一掌「青竹手」未使全力,但韋一笑是何等樣人?一招「寒冰綿掌」拍到,楊逍右臂一震,登覺一股陰寒之氣,從肌膚中直透進來,忙運内力抵禦,兩人功力相若,登時相持不下。

周顚叫道︰「姓楊的,再吃我一掌!」剛纔一掌没有打到,這時第二掌擊向他的胸口。説不得叫道︰「周顚,不可胡鬧。」彭瑩玉也道︰「楊左使,韋蝠王,兩位快快罷手,不可傷了和氣!」

彭瑩玉伸手欲去擋開周顚那一掌,楊逍身形一側,左臂略長,左掌已和周顚的右掌黏住。説不得叫道︰「周顚,你以二攻一,算什麼好漢?」伸手往周顚肩頭抓落,要想將他拉開,那知手掌未落,突見周顚身子微微發顫,似乎已受内傷。説不得吃了一驚,他素知這位光明左使者功力通神,是本教絶頂的高手,只怕一掌之下,已將周顚傷了,眼見周顚的手掌仍和楊逍黏住,不肯撤掌,叫道︰「周顚,自己兄弟,拚什麼老命?」往他肩頭一扳,同時説道︰「楊左使,掌下留情。」生怕楊逍不撤掌力,順勢追擊。

不料一拉之下,周顚身子一晃,没能拉開,同時一股透骨冰冷的寒氣,從手掌心中直傳至胸口。説不得更是吃驚,暗想︰「這是韋兄妙絶天下的『寒冰綿掌』啊,怎地楊逍也練成了?此人再會了這種陰毒掌力,那更是如虎添翼,不可復制。」當下急運功力,與那寒氣相抗。但這股寒氣越來越是厲害,片刻之間,説不得牙関相擊,堪堪抵禦不住。鐵冠道人和彭瑩玉雙雙搶上,一護周顚,一護説不得。聚合四人之力,寒氣已不足爲患,然而只覺楊逍掌心中傳過來的力道一陣輕一陣重,時急時緩,瞬息萬變,四個人竟是不敢撤掌,生怕便在撤掌收力的一刹那間,楊逍突然發力,那麼四人不死也得重傷。説不得支持了一會,説道︰「楊左使,咱們對你\dash{}」只説得這幾個字,突然胸口一涼,似乎全身血液都要凍結成冰,原來他一開口説話,眞氣暫歇,便即抵擋不住楊逍手掌中傳過來的寒氣。

如此支持了一頓飯時分,但見韋一笑和四散人都是神色緊張,楊逍却悠然自若。冷面先生冷謙在旁冷眼旁觀,心下好生懷疑︰「楊逍武功雖高,但和韋一笑也不過在伯仲之間,未必便能勝得了他,再加上説不得等四個人,楊逍萬萬抵敵不住,何以他以一敵五,反而似操勝算,其中必有古怪?」他是個絶頂聰明之人,低頭沉思,一時却會不過意來。只聽周顚叫道︰「冷面鬼\dash{}打\dash{}打他背心\dash{}打\dash{}」冷謙未曾想明白其中原因,不肯便此出手,眼下五散人中只有自己一人閒著,解危脱困,全仗自己,倘若也和楊逍一起硬拚,多一人之力雖然好得多,却也未必定能制勝。然見周顚和彭瑩玉臉色發青,如再支持下去,陰毒入了内臟,那便是無窮之禍,當下伸手入懷,取出五枚爛銀打就的小筆,托在手中,説道︰「五枚銀筆,打你曲穴、巨骨、陽谿、五里、中都。」這五處穴道都是在手足之上,並非致命的要穴,他又先行説了出來,意思是通知楊逍,並非和你爲敵,乃是叫他撤掌罷鬥。

楊逍微微一笑,並不理會。冷謙叫道︰「得罪了!」左手一揚,右手一揮,五點銀光直向楊逍射去。楊逍待那五枚銀筆飛近,突然左臂橫劃,拉得周顚等四人擋在他的身前,但聽周顚和彭瑩玉齊聲悶哼,五枚小筆分别打在他二人身上,周顚中了兩枚,彭瑩玉中了三枚。好在冷謙原意不在傷人,出手甚輕,所中又不在穴道,雖然傷内見血,却無大礙。彭瑩玉低聲道︰「是乾坤大挪移!」冷謙聽到「乾坤大挪移五」字,登時省悟。原來這乾坤大挪移是明教中歷代相傳一種最厲害的武功,其根本道理,並不希奇,只不過是武學中「借力打力」「四兩撥千斤」的要質,但其中變化神奇,却是匪夷所思。數十年來,明教中從未聽説有人練會這種功夫,是以人人一時想不到這武功上去。如此看來,楊逍其實是毫不出力,祇不過是將韋一笑「寒冰綿掌」的掌力引著攻向四散人,而反過來又將四散人的掌力引去攻韋一笑,他居中悠閒而立,不過是隔山觀虎鬥而已。

冷謙道︰「恭喜!無惡意,請罷鬥。」他説話簡潔之極,「恭喜」兩字,是恭喜楊逍練成了明教中近百年來已然失傳的「乾坤大挪移」神功;「無惡意」是説咱們六個人這次上山,對你絶無惡意,原是誠心共抗外敵而來;請罷鬥是請雙方罷鬥,不可發生誤會。楊逍素知他的脾氣,他説話簡單明瞭,決不會多説一個字廢話,正因爲不肯多説一個字,自是從來不説假話。他既説「無惡意」,那是眞的没有惡意了,而且他適纔出手擲射的五枚銀筆,顯爲解圍,不在傷人,於是哈哈一笑,説道︰「韋兄,四散人,我説一、二、三大家同時撤去掌力,免有誤傷!」見韋一笑和周顚等都點了點頭,便緩緩叫道︰「一、二、三!」

那「三」字剛出口,楊逍便即收起「乾坤大挪移」神功,突然間背心一寒,一股鋭利的指力已戮中了他背上的「神道穴」。楊逍大吃一驚︰「蝠王好不陰毒,竟然乘勢偸襲。」待要回掌反擊,只見韋一笑身子一晃,已然跌倒,顯是也受人襲擊。楊逍一生之中不知見過多少陣仗,雖然這一下變起倉卒,心下並不慌張,身形向前一衝,先行脱却身後敵人的控制,回過身來,一瞥之下,只見周顚、彭瑩玉、鐵冠道人、説不得四人各已倒地,冷謙正向一個身穿灰色布袍之人拍出一掌。那人回手一格,冷謙「哼」了一聲,哼聲之中,微帶痛楚。楊逍吸一口氣,縱身上前,待欲相助冷謙,突覺一股寒冰般的冷氣從「神道穴」疾向上行,霎時之間自身柱、陶道、大椎、風府,遊遍了全身督脈諸穴。楊逍心知不妙,敵人武功既高,心又陰毒,抓正了自己與韋一笑、四散人六人一齊收功撤力的瞬息時機,閃電般猛施突襲,當下只得疾運眞氣和那寒氣相抗。這股寒氣和韋一笑所發的「寒冰綿掌」掌力全然不同,只見細絲般一縷冰線,但遊到何處穴道,何處便感酸麻,若是正掌對敵,楊逍有内力護體,決不致任這指力透體侵入,此刻既已受了暗算,只有先行強忍,助冷謙擊倒敵人再説。

那知他拔步上前,右掌揚起,剛要擊出,突然全身打個冷震,掌上勁力已消失得無影無縱。這時冷謙已和那人拆了二十餘招,眼見漸漸不敵。楊逍心中大急,只見冷謙一足踢出,被那人搶上一步,一指戮在臂上,冷謙身形一晃,向後便倒。楊逍又驚又怒,心想冷謙先生和這人拆得二十餘招,那麼此人武功雖強,也未必能在自己之上,只是一招未接,先受暗算,縱有天大本事,却是半點施展不出。

張無忌藏身在布袋之中,先聽得韋一笑、五散人和楊逍言語爭執,跟著動起手來,他心中焦急之極,既怕雙方有了損傷,又極想看看這明教七大高手情狀,可是布袋中一片漆黑,聲音聽得清清楚楚,却瞧不見半點袋外之物。過了一會,好容易冷謙以幾個字説得雙方罷鬥,那知突然又有強敵來襲,這人突然其來,張無忌事先没聽到半點聲音,韋一笑和四散人已被點倒,跟著冷謙在一番激鬥後倒地,楊逍雖然勉力站著,但無忌聽到他牙関相擊,呼吸凝重,顯然也已無力反抗。

半晌沉寂過後,脚步聲響,内堂一人奔了出來,叫道︰「爹,是誰來了?怎麼不讓我見見?」是個少女的聲音。張無忌心中一動︰「是不悔妹子。」只聽楊逍喘息著道︰「快走,快走,走得越遠\dash{}越\dash{}好\dash{}」楊不悔見到廳上的情形,驚呼道︰「爹,你\dash{}你受了傷麼?」回身瞧著那灰袍人,怒道︰「是你傷了我爹爹?」那人冷笑一聲,並不回答。楊逍急道︰「不悔,快聽爹的話,快走!」楊不悔本想撲上去掌擊那袍人,略一遲疑,伸手扶住了父親。

那灰袍人森然道︰「女娃児,出去!」楊不悔扶著楊逍,道︰「爹,你到外面去歇歇!」楊逍苦笑道︰「你先出去。」他自知爲敵人所制,豈能輕易脱身?楊不悔轉身向那灰袍人道︰「和尚,你何以暗害我爹爹?」那灰袍人冷笑道︰「好啊,你眼光鋭利,瞧出我是和尚,那便容你不得!」左手袖袍一拂,右手食指已在暗藏的袍袖之下,向楊不悔「秉風穴」上點去。楊逍眼見這指若是點中,女児非斃命當場不可,自己内力雖然未復,這情勢却不得不救,當即右肘橫伸,一個肘錘,向那灰袍人胸口撞到。

那灰袍人左指一彈,正中楊逍肘底的「小海穴」,但右指却偏了一偏,雖然仍是點中了楊不悔,已非致命之處,楊逍愛女心切,強忍住全身的冰冷酸麻,左足飛起,將女児踢出廳外,同時橫身擋在廳門之前,不讓那灰袍人追擊。那灰袍人冷笑道︰「這女娃児中了我的「一陰指」指力,不能活過三天三夜。」向楊逍凝望一眼,又道︰「光明使者名不虛傳,連中我兩指,居然仍能站立。」楊逍道︰「少林神僧空見大師慈悲厚德,門下出了你這種不肖弟子,你是『圓』字輩的了,叫作圓什麼?」

灰袍人暗吃一驚,讚道︰「了不起,了不起!竟給你瞧出了我的門戸來歷。貧僧圓眞!」張無忌當聽到楊逍説起少林神僧空見大師之時,已是全神貫注,待聽到那人自認是圓眞更是大吃一驚,心想︰「這人曾傳我少林九陽功,明知我身中玄冥神掌的陰毒,却又故意替我打通奇經八脈,叫我陰毒難除。看來此人武功奇高,又是陰險毒辣,實是六大門派中的第一厲害脚色。這次六派圍剿明教,這人突然掩上光明頂,楊逍和青翼蝠王等盡爲所制,這一次明教當眞是一敗塗地了。」

只聽楊逍説道︰「六大門派和我明教爲敵,眞刀眞槍,決一死戰,那纔是男子漢大丈夫的行逕,你少林派\dash{}」説到這裡,再也支持不住,雙膝一軟,坐倒在地。圓眞哈哈大笑,説道︰「出奇制勝,兵不厭詐,那是自古已然。我圓眞一人,打倒明教七大高手,難道你們輸得還不服氣麼?」楊逍道︰「你怎麼能偸入光明頂來?這祕道你如何得知?若蒙相示,楊逍死亦瞑目。」要知圓眞此次所以能偸襲成功,固是由於身負絶頂武功,但最主要的原因,却在於他知道偸上光明頂的祕道,越高明教教衆的十餘道哨線,神不知鬼不覺的突然出手,才能將明教七大高手點倒。圓眞笑道︰「你魔教光明頂七巓十三崖,自己當作天險,在我少林僧侶眼中,也不過是康莊大道而已,何足道哉?你們都中了我的一陰指,三人之内,各赴西天,那也不在話下。貧僧這便上坐忘峰去,埋下幾十斤火藥,再減了魔教的魔火,什麼白眉教啦,五行旗啦,急急忙忙上來相救,轟的一聲大響,地下埋著火藥炸將起來,煙飛火滅,不可一世的魔教從此無影無蹤。這叫做︰少林僧獨指滅明教,光明頂七魔歸西天。」楊逍等聽了這番話,心下均是大感驚懼,知他説得出做得到,自己送命不打緊,這傳了三十三世的明教,眞的要亡在這少林僧手下不成?只聽圓眞越説越是得意,又道︰「明教之中,高手如雲,你們倘若不是自相殘殺,四分五裂,何致有覆滅之禍?以今日之事而論,你們七人假若不是正在自拚掌力,貧僧便是悄悄的上得光明頂來,又焉能一擊成功?這叫做天作孼,猶可活,自作孼,不可活!哈哈,想不到當年威風赫赫的明教,楊破天一死,便落得如此下場。」楊逍、周顚等面臨身死教滅的大禍,聽了他這一番話,回想過去二十年來的往事,均是後悔不已,心想︰「這和尚的話可没説得錯。」

周顚大聲道︰「楊逍,我周顚大是該死!過去對你不起。你這人雖然不大好,但做了教主,也勝於没有教主而鬧得全軍覆没。」楊逍苦笑道︰「我何德何能,能做教主?大家都錯了,咱們弄得一團糟,九泉之下,也没面目回去見歷代明尊教主。」圓眞笑道︰「各位後悔,已然遲了。當年楊破天初任魔教頭子之時,氣燄是何等不可一世,只可惜他死得早了,没能親眼見到明教的慘敗。」周顚怒罵道︰「放你媽的狗臭屁,楊教主倘若在世,大夥児聽他號令,你這賊禿會偸襲得手麼?」

圓眞冷笑道︰「楊破天死也好,活也好,我總有法子令他身敗名裂\dash{}」突然之間拍的一響,跟著「啊」的一聲,圓眞的背上已中了韋一笑的一掌,便在同時,韋一笑也被圓眞反戮一指,正中胸口的「膻中穴」。兩人各退一步,同時摔倒,原來韋一笑極工心計,被圓眞一指點中後,雖然受傷極重,但他内力究竟高人一籌,並非登時全無反擊之力,只是裝作暈去,等到圓眞得意揚揚、決不防備之際,暴起襲擊。這一掌他是逼出全身勁力,爲了挽救明教的浩劫,意圖與敵同歸於盡。圓眞雖然厲害,但青翼蝠王是明教四大護教法王之一,向與殷天正、謝遜等人齊名,這奮力一擊,豈同小可?「寒冰綿掌」的掌力侵入體内,但覺胸口煩惡欲嘔,數番潛運内力欲圖穩住身子,總是天旋地轉,便欲摔倒,只得盤膝坐下,與那「寒冰綿掌」的掌力相抗。韋一笑連中兩下「一陰指」,更是氣息奄奄,動彈不得。

刹那之間,廳堂上寂靜無聲,八大高手一齊身受重傷誰都不能移動半步。楊不悔在大廳之外,她功力較淺,受傷更重。圓眞和明教的七大高手各運内力,企盼早一步能恢復行動,只要一方早得片刻,便能制死對方,自己得獲安全。各人心中都是緊張萬狀,要知明教的生死存亡,實繫於這一線之間。假若圓眞能先一步行動,他雖重傷未愈,却能提起寶劍將七人刺死,然後慢慢的將息養傷,要是明教七人中有任何一個能先動彈,那麼殺了圓眞,明教便此得救,本來七人這邉是佔了便宜,但五散人功力略淺,中了一招「一陰指」後便勁力全失,而内功深湛的楊逍和韋一笑却均連中兩指。「寒冰綿掌」和「一陰指」的勁力,原是不易分别高下,但韋一笑所拍出那一掌,已是在受傷之後,内力自不如圓眞在未受傷時所遞出的招數,看來對耗下去,倒是圓眞先能移動的局面居多。

楊逍等暗暗心焦,但這運氣引功之事,實是半分勉強不得,越是心煩氣躁,越易大出岔子,這些人個個都是内家高手,這中間的道理如何會不省得?冷謙等吐納數下,料知無法趕在圓眞的前頭,但盼光明頂上楊逍的下屬能有一人走進廳來,只須有明教的一名教衆入内,便是他不會絲毫武藝,這時只要提根木棍,輕輕一棍便能將圓眞打死。

可是等了良久,廳外那裡有半點聲息?這時已在午夜,光明頂上的教衆或分守哨防,或各自安臥,不得楊逍召喚,誰敢擅入議事廳堂?至於服侍楊逍的僮児,一人被韋一笑吸血而死,其餘的個個嚇得魂飛魄散,早已遠遠散開,别説楊逍按鈴叫人,就算叫到,只怕一時間也未必敢踏入廳堂,走到這吸血魔王的身前。

張無忌藏身在布袋之中,聽到身外一片寂靜,也知道寂靜之中,隱藏著極大的殺機,過了半晌,忽聽得説不得道︰「喂,布袋中的小朋友,你非救咱們一救不可。」張無忌道︰「怎麼救啊?」圓眞丹田中一口眞氣正在漸漸通暢,猛地裡聽得布袋中發出人聲,一驚非同小可,那眞氣立時逆運,全身劇烈的顫抖起來。

\chapter{生死成敗}

圓眞武功雖強,但自潛入議事堂後,一心在對付韋一笑、楊逍等諸高手,那有餘暇去察看地下一隻絶無異狀的布袋?突聞袋中有人説話,不禁倒抽了一口涼氣,暗叫︰「我命休唉!」只聽説不得道︰「這布袋的口子用『千纏百扣結』縛住,除我自己之外,旁人是萬萬解不開的,但你可站起身來。」張無忌道︰「是!」從布袋中站了起來。

説不得道︰「小兄弟,你捨身相救鋭金旗數十位兄弟的性命,義烈高風,人人欽佩。眼下咱們數人的性命,也全賴你相救。請你走將過去,一拳一掌,將那惡僧打死了吧。」張無忌心下沉吟,半晌不答。説不得道︰「這惡僧人乘人之危,忽施偸襲,這種卑鄙行逕,你是親耳聽到的了。你若不打死他,明教上下數萬人衆,都要被人一一誅滅。你去打死他,乃是大仁大勇的俠義行爲。」張無忌仍是躊躇不答,圓眞説道︰「我此刻半點動彈不得,你一拳打死我,豈不被天下好漢恥笑?」周顚怒道︰「臭賊禿,你少林派枉稱正大門派,却偸偸摸摸的上來暗襲,天下好漢不恥笑麼?」

張無忌向圓眞走了一步,便即停步,説道︰「説不得大師,貴教和六大門派之間的是非曲直,小可實不深知。小可極願爲各住援手,却不願傷了這位少林派的和尚。」彭瑩玉道︰「小兄弟你有所不知,你此時若不殺他,待這和尚功力一復,他非連你也害了不可。」圓眞笑道︰「我和這位小施主無怨無仇,怎能隨便傷人?何況這位小施主又非魔教中人,看來還是被布袋和尚不懷好意的擒上山來。你們魔教中人無惡不作,對他還有什麼好事做將出來。」這時雙方氣喘吁吁,説話都極艱難,但均是竭力提氣,意圖打動無忌之心。

張無忌甚感左右爲難,明知圓眞和尚居心險惡,但要上前一掌將他打死,却非本心所願,何況這一掌倘若打下,那便是永遠站在明教一面,公然和六大門派爲敵。太師父、武當六俠、周芷若等等,全成了自己的敵人。明教素被武林中人認爲是邪魔異端,如韋一笑吸食人血、義父謝遜濫殺無辜,確有許多不該之處,太師父當年諄諄告誡,千萬不可和魔教中人結交,以免終身受禍,自己父親因和魔教的母親成親,因而自刎武當山頭,殷鑑不遠,覆轍在前。又想到這圓眞是神僧空見的弟子,那空見大師甘受一十三掌「七傷拳」,只盼能感化我義父,結果却身死拳下,這等大仁大義的慈悲心懷,實是武林中千古罕有,我怎能再傷他弟子?再説,這位圓眞和尚曾傳我少林九陽功,也可説和我有幾分師徒之誼,雖然他打通我奇經八脈,蓄意加害,可是我却並没被他害死啊。

他生性只是記著旁人待他的好處,别人對他的欺壓侮辱,事後他總是替那人找出些理由來解釋一番,例如何太沖是爲悍妻所逼、朱長齡是愛刀成狂、朱九眞是對衛璧情有獨鍾等等,他心中早已一一原諒了他們。因之對於圓眞當年的暗算,他也絲毫没有記恨。只聽説不得又在催促勸説,便道︰「説不得大師,請你教我一個法子,不用傷害這位和尚,而他也傷你不得,小可定然照辦。」

説不得心想︰「眼下已是非拚個你死我活不可的局面。那裡還能雙方都可保全?不是圓眞死,便是咱們亡。」正自沉吟未答,彭瑩玉道︰「小兄弟仁人心懷,至堪欽佩。那便請小兄弟伸出手指在那圓眞胸口『玉堂穴』上輕輕一點。這一點對他並無損傷,只不過令他幾個時辰内不能運引内力。咱們派人送他下光明頂去,決不損他一根毫毛。」張無忌深明醫理,知道在「玉堂穴」上輕點一指,確能暫阻丹田中眞氣上行,但並不損傷身體。

却聽得圓説道︰「小施主千萬别上了他們的當。你點我穴道固然不打緊,但他們内力一復,立時便來殺我,你又如何阻止得了?」周顚罵道︰「放你的狗臭屁!咱們説過不傷你,自然不傷你,明教五散人説過的話,幾時不算數了?」張無忌心想楊逍和五散人都不是出爾反爾之輩,只有韋一笑一人可慮,便問︰「韋前輩,你説如何?」韋一笑顫聲道︰「我也暫不傷他便是,下次見面,大家再拚你死我活。」他説到你死我活這四個字時,已是聲音微弱異常,上氣不接下氣。張無忌道︰「這便是了,光明使者、青翼蝠王、五散人七位,個個是當世的英雄豪傑,豈能自毀諾言,失信於人?圓眞大師,晩輩可要得罪了。」説著走向圓眞身前。

那「玉堂穴」在人身胸口,位於「紫宮穴」下一寸六分、「膻中穴」上一寸六分,屬於任脈。這穴道並非致命的大穴,但位當氣脈必經的通道,若是一加阻塞,全身眞氣立受干撓。張無忌聽著圓眞的呼吸,待得離他二尺,説道︰「圓眞大師,晩輩是爲了周全雙方,你别見怪。」説著緩緩提起手來。圓眞苦笑道︰「此刻我全身動彈不得,只有任你小輩橫行。」自從「蝶谷醫仙」胡青牛一死,張無忌辨認穴道之技已是當世無匹,他與圓眞之間雖然隔著一隻布袋,但伸指出去便是點向「玉堂穴」,竟無厘毫之差。

猛聽得楊逍、冷謙、説不得齊聲叫道︰「啊喲!快縮手!」張無忌只覺右手食指一震,一股冷氣從手指尖直傳過來,有如閃電一般,登時全身皆冷。只聽周顚、鐵冠道人等一齊破口大罵︰「臭賊禿,膽敢如此使奸!」張無忌全身簌簌發抖,心裡已然明白,那圓眞雖然脚步不能移動,但勉力提起手指,放在自己「玉堂穴」之前。張無忌苦在隔著布袋,瞧不見他竟會使出這一步棋子,一指點去,兩根指尖相碰,圓眞的「一陰指」已隔著布袋直傳到他的體内。

張無忌雖然受損,但圓眞是將全身殘存的精神内力盡數逼出在手指之上,雙指一觸之後,他全身癱瘓,臉如白蠟,便如僵屍。廳堂上本來有八個人受傷後不能彀移動,這麼一來,又多了一個張無忌。周顚最是暴躁,破口大罵少林賊禿奸詐無恥。楊逍等人却想,這倒也怪圓眞不得,敵人要點他穴道,他伸手自衛,原無什麼不當。圓眞雖然一時之間疲累欲死,心中却自暗喜,心想這小子年紀不大,能有多少功夫,中了一陰指後,料他不到一日便即身死,自己散了的眞氣當可在一個時辰後慢慢凝聚,仍是任由自己爲所欲爲的局面。

當下廳堂之上,又回復了寂靜無聲,過了大半個時辰,四枝蠟燭逐一熄滅,廳中更是漆黑一片。楊逍等聽著圓眞的呼吸由斷斷續續而漸趨均勻,由粗重而逐步漫長,知他體内眞氣正自凝聚,但自己略一運功,那一陰指寒冰般的冷氣,便即侵入丹田,忍不住的發抖。各人越來越失望,心中難受之極。反盼圓眞早些回復功力,上來每人一拳,痛痛快快的將自己打死。勝於慘受這種無窮無盡的折磨。冷謙、周顚等人索性瞑目待死,倒也爽快,其中説不得和彭瑩玉兩人却甚是放心不下。原來五散人中,説不得和彭瑩玉都是出家和尚,但偏偏是這兩人最具雄心,最爲関心世人的疾苦,立志要大大做一番事業。

這時局勢已定,最後終於是非喪生在圓眞的手下不可,各人生平壯志,盡付流水。説不得道︰「彭和尚,咱們處心積慮,想要趕走蒙古韃子,那知到頭來還是一場空。唉,想是天下千千萬萬的百姓劫難未盡,還有得苦頭吃呢。」

張無忌守住心口一股熱氣,和那一陰指的寒氣相抗,但説不得的這幾句話却聽得清清楚楚,不禁奇怪︰「他説要趕走蒙古韃子?難道惡名遠播的魔教,還眞能爲天下百姓著想麼?」只聽彭瑩玉道︰「説不得,我早就説過,單憑咱們明教之力,蒙古韃子是趕不了的,總須聯絡普天下的英雄豪傑,一齊動手,才能成事。你師兄棒胡,我師弟周子旺,當年造反起事這等聲勢,終於一敗塗地,不是爲了没有外援麼?\dash{}」張無忌心道︰「周子旺?那不是周芷若姑娘的父親麼?」彭瑩玉以後幾句話,就没聽進耳裡。

忽聽周顚大聲道︰「死到臨頭,你們兩個賊禿還在爭不清楚,一個説要以明教爲主,一個説要聯絡正大門派。依我周顚看來,都是廢話,都是放屁。咱們明教自己四分五裂,六神無主,還主他媽個屁!彭和尚要聯絡正大門派,更是放屁之至,屁中之尤,六大門派正在圍剿咱們,咱們還跟他聯絡?」鐵冠道人忽然插口道︰「倘若楊教主在世,咱們將六大門派打得服服貼貼,何愁他們不聽明教號令。」周顚哈哈大笑,道︰「牛鼻子雜毛放的牛屁更是臭不可當,楊教主倘若在世,自然一切都好辦,這個誰不知道?要你多説\dash{}啊喲\dash{}啊喲\dash{}」他張口一笑,氣息散渙,一陰指寒氣直透到心肺之間,忍不住叫了出來。冷謙道︰「住嘴!」他這兩個字一出口,各人一齊靜了下來。

張無忌心中思潮起伏︰「看來明教這一教派,中間包藏著許多原委曲折,並不是單單專做壞事而已。」便道︰「説不得大師,貴教宗旨到底是什麼?可能見示否?」説不得道︰「哈,你還没死麼?小兄弟,你莫其妙的爲明教送了性命,咱們很是過意不去。反正你已没幾個時辰好活,本教的祕密就是跟你説了,也没干係。冷面先生,你説是麼?」冷謙道︰「説!」他説話當眞是簡潔之極,本該説「你對他説好了」,六個字却以一個「説」字來包括了。

説不得道︰「小兄弟,我明教源於大食國,唐時傳至中土。當時唐皇在各處勒建大雲光明寺,即是我明教的寺院。我教教義是衆生平等,若有金錢財物,須當救濟貧衆,不茹葷酒,崇拜明尊。祇因歷朝貪官汚吏欺壓我教,教中兄弟不忿,往往起事,自北宋方臘方教主以來,已是算不清有多少次了。」張無忌也聽到過方臘的名頭,知他是北宋宣和年間的四大寇之一,和宋江、田虎等人齊名,便道︰「原來方臘是貴教的教主?」説不得道︰「是啊。到了南宋建炎年間有王宗石教主在信州起事,紹興年間有余五婆教主在衢州起事,理宗紹定年間有張三槍教主在江西、廣東一帶起事。只因本教素來和朝廷官府作對。朝廷便説咱們是『魔教』,嚴加禁止。咱們爲了活命,行事不免隱隱祕詭怪,以避官府的耳目。正大門派和本派積怨成仇,更是勢成水火。當然,本教教衆之中,也不免偶有不自檢點,爲非作歹之徒,給正大門派抓住了把柄,於是本教之聲譽,便如江河之日下了\dash{}」

楊逍突然冷冷插口道︰「説不得,你是説我麼?」説不得道︰「我的名字叫做『説不得』,凡是説不得之事,我是不説的。各人做事,各人自己明白,這叫做啞子吃餛飩,肚裡有數。」楊逍哼了一聲,不再言語。張無忌猛地一驚︰「咦,怎地我身上不冷了?」原來他初中圓眞的一陰指時,寒冷難當,但隔了這些時候,寒氣已消失得無影無蹤。

須知張無忌在十歳那一年身中「玄冥神掌」的陰毒,直至十七歳那一年方才去淨,七年之間,日日夜夜圴在與體内寒毒抗,運氣禦寒已和呼吸、霎眼一般,不須意念,自然而成。何況他長期服食血蛙,練成九陽神功,體内陽氣充旺之極,過不多時,早已將陰毒驅除乾淨。

只聽説不得道︰「自從我大宋亡在蒙古韃子手中,明教更成朝廷死敵,歷代教主,均以聯絡江湖豪傑,驅除胡虜爲已任。只可惜近年來明教群龍無首,教中諸高手爲了爭奪教主之位,鬧得自相殘殺。終於有些潔身自好,翩然歸隱,有些另立支派,自任教主。教規一墮之後,與名門正派結的怨仇更深,纔有眼前之事。圓眞和尚,我説可没半句假話吧?」圓眞哼了一聲,道︰「不假,不假!你們死到臨頭,爲什麼要説假話?」他一面説,一面緩緩站了起來,向前跨了一步楊逍和五散人一齊「啊」的一聲,輕輕驚呼,各人雖明知他終於會比自己先復行動,却都没想到此人功力居然如此深厚,中了青翼蝠王韋一笑的「寒冰綿掌」,竟能如此迅速提氣運功。

只見他身形凝重,左足又向前跨了一步,身子却半點没有搖晃。楊逍冷笑道︰「空見神僧的高足,果然非同小可,可是你還没回答我先前的話啊。難道此中頗有曖昧説不出口嗎?」圓眞哈哈一笑,又向前邁了一步,説道︰「你若不知曉其中底細,當眞是死不暝目。你問我怎能知道光明頂的祕道,何以能神不知鬼不覺的上得山巓,好,我跟各位實説了,是貴教楊破天教主夫婦兩人,親自帶我上來的。」

楊逍一凜,暗道︰「以他身份,決不致會説謊話,但此事想來決不能彀!」只聽周顚已罵了起來︰「放你的狗屁!這祕道是光明頂的大祕密,是本教的莊嚴聖境。楊左使雖是光明使者,韋大哥是護教法王,也從來没有走過,自來只有教主一人,纔可行此祕道,楊教主怎會帶你一個外人行此祕道?」圓眞嘆了一口氣,出神半晌,幽幽的道︰「你既非査根問底不可,我便將二十五年前的一件隱事跟你説了。反正你們終不能活著下山,洩漏此事。唉!周顚,你説的不錯,這祕道是明教的莊嚴聖境,歷來只有教主一人,方能進入,否則,便是犯了十惡不赦的重罪。可是楊破天的夫人是進去過的,楊破天犯了教規,私帶楊夫人偸進祕道\dash{}
{\color{blue}周顚這時插口罵道︰「放屁!放屁!」彭瑩玉喝道︰「周顚,别吵!」}
\dash{}楊夫人又私自帶我走過祕道\dash{}
{\color{blue}周顚插口大罵︰「他媽的,{\upstsl{呸}},{\upstsl{呸}}!胡説八道。」}
\dash{}我不是明教中人,走進祕道也算不得犯了教規。唉,就算明教之徒,就算犯下重罪,我又怕什麼了?」他説起這段往事之時,聲音竟然甚是淒涼。鐵冠道人問道︰「楊夫人何以帶你走進祕道?」圓眞道︰「那是很久很久以前的事了,老衲今日已是七十餘歳的老人\dash{}少年時的事\dash{}好,一起跟你們説了。各位可知老衲是誰?楊破天是我師兄;楊夫人是我師妹,老衲出家之前的俗家姓氏,姓成名崑,外號『混元霹靂手』的便是!」

這幾句話一出口,楊逍等固然驚訝無比,布袋中的張無忌更是險些児驚呼出聲。冰火島上那日晩間謝遜所説的故事,清清楚楚的顯現在眼前;他師父成崑怎地殺了他父母妻子全家、怎地濫殺武林人士圖逼成崑出面、怎地拳傷空見神僧那成崑却不守諾言現身\dash{}無忌猛地想起︰「原來那時這惡賊成崑已拜空見神僧爲師,神僧爲要化解這場冤孼,纔甘心受我義父那一十三記七傷拳。豈知成崑竟連他自己師父也欺騙了,累得空見神僧飲恨而終。」

張無忌想到此處,立時又記起那天晩上自己對義父許下的諾言︰「義父,你眼睛看不見,等我大了,練好了武功,去替你報仇\dash{}義父,害你全家之人叫混元霹靂手成崑,無忌記在心中,將來一定替你報仇。」再想︰「義父所以時常狂性發作、濫殺無辜,各家各派所以齊上武當,逼死我爹爹媽媽,推究這一切事情的罪魁禍首,都是由於這成崑在從中作怪。」他心中憤怒無比,突然間全身燥熱,有如火焚。説不得這乾坤一氣袋密不通風。張無忌在袋中耽了這許多時候,本來早就氣悶之極,仗著内功深湛,以綿綿龜息之法呼吸,需氣極少,這纔支持了下來。此時猛地裡心神一亂,蘊蓄在丹田中的九陽眞氣失却主宰,茫然亂闖起來,霎時之間,便似身處洪爐,忍不住大聲呻吟。

周顚喝道︰「小兄弟,大家命在頃刻,誰都苦楚難當,是好漢子便莫示弱出聲。」張無忌應道︰「是!」以九陽眞經中運功之法鎭懾心神,調勻内息。平時只須依法施爲,立時便心如止水,神遊物外,這時却越是運功,四肢百骸越是難受,似乎每處大穴之中,同時有幾百枚燒紅了的小針在不住刺入。原來他修習九陽眞經數年,雖然得窺天下最上乘武學的祕奥,但以未經明師指點,只是自己一人暗中摸索,體内積蓄的九陽眞氣越儲越多,却不會導引運用。本來不加引發,倒也罷了,那圓眞的一陰指却是武林中最爲陰毒的功夫,一經加體,猶如在一桶火藥上點燃了藥引。偏生他又身處乾坤一氣袋中,激發了的九陽眞氣無處宣洩,反過來又向他身子衝激。在這短短的一段時間中,張無忌正經歷著修道練功之士一生最艱難最危險的関頭,生死成敗,懸於一線。周顚等那想到他竟會遲不遲、早不早,偏偏就在這時撞到三花聚頂、五氣朝元、龍虎交會的大関頭,只道他中了一陰指後垂死的呻喚。

張無忌竭力抵禦熱氣的煎熬,圓眞的話却仍是一句句清清楚楚的傳入耳中,聽他説道︰「我師妹和我兩家乃是世交,兩人從小便有婚姻之約,豈知楊破天暗中也在私戀我師妹,待他當上了明教教主,威震天下,我師妹的父母固是勢利之輩,我師妹也心志不堅,竟爾嫁了他。可是她婚後並不見得快活,有時和我相會,不免要找一個極隱祕的所在。楊破天對我這師妹事事依從,絶無半點違拗,她要他帶去看一看祕道,楊破天雖然極不願意,但經不起她軟求硬逼,終於帶了她進去,自此之後,這光明頂的祕道,明教數百年來最神聖莊嚴的聖地,便成爲我和教主夫人相會之地,哈哈,哈哈\dash{}我在這祕道中來來去去走過數十次,今日重上光明頂,還會費什麼力氣?」

周顚、楊逍等聽了他這番言語,人人啞口無言,周顚只罵了一個放字,下面這屁字便接不下去。每個人胸中憤怒如要炸裂,對於明教的侮辱,再没比這一件事更爲重大的了。而今日明教覆滅,更由這祕道而起。衆人雖然聽得眼中如欲噴出火來,却都知圓眞的話並非虛假。圓眞又道︰「你們氣惱什麼?我好好的姻緣被楊破天活生生拆散,明明是我愛妻,只因楊破天當上了魔教的大頭子,便將我愛妻佔了去。我和魔教此仇不共戴天。楊破天和我師妹成婚之日,我曾去道賀,喝著喜酒之時,我心中立下重誓︰成崑只教有一口氣在,定當殺了楊破天,定當覆滅魔教。我立下此誓已有五十餘年,今日方見大功告成,哈哈,我成崑心願已了,死亦暝目。」

楊逍冷冷的道︰「多謝你點破了我心中的一個大疑團,楊教主突然暴斃,死因不明,原來是你下的手。」

圓眞森然道︰「當年楊師兄武功高出我甚多,咱們同門學藝,誰的功夫如何,大家心中明白\dash{}」周顚接口道︰「因此你只有暗中加害楊教主了,不是下毒,便是如這一次般忽施偸襲。」圓眞嘆了口氣,搖頭道︰「不是。我師妹怕我偸下毒手,不斷的向告誡,倘若楊破天被我害死,她決計饒不過我。她説她和我暗中私會,已是萬分對不起丈夫,要再起下毒心,那是天理不容。楊師兄,唉,楊師兄,他\dash{}他是自己死的。」

楊逍、彭瑩玉等都「啊」了一聲。圓眞續道︰「假如楊破天眞是死在我掌底指下,我倒饒了你們明教啦\dash{}」他聲音漸轉低沉,回憶著數十年前的往事,緩緩的道︰「那一天晩間,我又和我師妹在祕道中相會,突然之間,聽到左首傳過來一陣極重濁的呼吸聲音。這是從來没有的事,這祕道構造隱祕之極,外人決計無法找到入口,而明教中人,却又誰也不敢進入。咱倆聽到這呼吸聲音,當時大吃一驚,便即悄悄過去察看,只見楊師兄坐在一間斗室之中,手裡執著一張羊皮,滿臉通紅。他已見到了我們,説道︰『你們兩個,很好很好,對得我住啊!』説了這幾句話,忽然間滿臉鐵青,但臉上這鐵青之色一顯即隱,立即又變成血紅之色,忽鐵忽紅,在瞬息之間,接連變換了三次。楊左使,你知道這種功夫吧?」

楊逍道︰「這是本教的『乾坤大挪移』神功。」周顚道︰「楊逍,你也已練會了,是不是?」楊逍道︰「『練會』兩字,如何能説?當年楊教主看得起我,曾傳過我一些這神功的粗淺入門功夫。我練了十多年,也只練到第二層而已。再練下去,便即全身眞氣如欲破腦而出,不論如何,總是無法克制。楊教主能於瞬息間變臉三次,那是練到第五層了。他曾説,本教歷代衆位教主之中,以第八代鍾教主武功最高,據説能將『乾坤大挪移』神功練到第六層,但便在練成的當天,走火入魔身亡,自此之後,從未有人練到第五層的。」周顚道︰「這麼難?」鐵冠道人道︰「倘若不這麼難,那能説得上是明教的護教神功?」

這些明教中的武學高手,對這「乾坤大挪移」神功都是耳聞已久,因此一經提及,雖然身處危境,仍是忍不住要談上幾句。彭瑩玉道︰「楊左使,楊教主將這神功練到第五層,何以要變換臉色?」原來彭瑩玉極工心計,這時詢問這種題外文章,却是另有深意,他知圓眞只要再走上幾步,各人便即一一喪生在他手底,好容易引得他談論往事,該當儘量拖延時間,只要本教七高手中有一人能回復行動,便可和他抵擋一陣,縱然不敵,事機或有變化,總勝於眼前這般束手待斃。

楊逍也是個極聰明之人,豈能不明白他的心意?説道︰「『乾坤大挪移』神功的主旨,乃在顚倒一剛一柔、一陰一陽的乾坤二氣,臉上出現青色紅色,便是體内血液沉降,眞氣變換之象。據説練至第六層時,全身都能忽紅忽青,但到第七層時,陰陽二氣轉換於不知不覺之間,外形上便半點也瞧不出表徵了。」彭瑩玉生怕圓眞不耐煩,便問他道︰「圓眞大師,我們楊教主到底是何因歸天?」

圓眞冷笑道︰「你們中了我一陰指後,當世只有四般人能彀解救。武當、少林、峨嵋三派的九陽神功,再加上當年一燈大師傳下雲南大理一派的一陽指。得有這四種神功之一相助,各位或能暫且恢復行動之力,若想拖延時候,自行運氣解救,老實跟位説,那是絶無用處。各位都是武學高手,便是受了再厲害的重傷,運了這麼久的内息,也該有些好轉了。却怎麼全身越來越僵呢?」

楊逍彭瑩玉等早已想到了這一層,但只教有一口氣在,總是不肯死心,只聽圓眞又道︰「那時我見楊師兄臉色變幻,心下也不免驚慌,我師妹知他武功極高,一出手便能致我們於死地,説道︰『大師哥,這一切都是我不好,你放成師哥下山,任何責罰,我甘心領受。』楊師兄聽了她的話,緩緩説道︰『我娶得你的人,却娶不得你的心。』只見他雙目瞪視,忽然眼中流下兩行鮮血,全身僵直,一動也不動了。我師妹大驚,叫道︰『大師哥,大師哥!破天,破天!你怎麼了?』」圓眞叫著這幾句話時,聲音雖然不響,但各人在靜夜之中聽來,又想到楊破天雙目流血的可怖形象,無不心中爲之一震。

只聽圓眞續道︰「她叫了好幾聲,楊師兄仍是毫不動彈,我叫師妺大著膽子去拉一拉他的手,早已僵硬,再探他鼻息,原來已是氣絶多時。我知她心下過意不去,安慰她道︰『看來大師哥是在練一種極難的武功,突然走火,眞氣逆胸,以致無法挽救。』我師妹道︰『不錯,他是在練明教的不世奇功『乾坤大挪移』,正在要緊関頭,陡然間發見了我和你私下相會的祕密。雖然不是我親手殺他,可是他却因我而死。』我正想説些什麼話來開導勸解,她忽然指著我身後,喝道︰『什麼人?』我急忙回頭,不見半個人影,再回過頭來時,只見她胸口插了一柄匕首,已是自殺身死。」

\qyh{}嘿嘿,楊破天説道︰『我娶得你的人,却娶不得你的心。』我得到了師妺的心,却終於得不到她的人。她是我生平至敬至愛之人,如果不是楊破天從中搗亂,我們的美滿姻緣,何至有如此悲慘下場?如果不是楊破天當上魔教的教主,我師妹也決計不會嫁給這個大上她二十多歳的師兄。楊破天是死了,我奈何他不得,但魔教還是在世上橫行。當時我指著師兄、師妹兩人的屍身,説道︰『我成崑立誓要竭盡所能,覆滅明教。大功告成之日,當來兩位之前自刎相謝。』哈哈,楊逍、韋一笑你們馬上便要死了,我成崑也已命不久長,只不過我是心願完成,欣然自刎,好於你們萬倍了。」

\qyh{}這些年來,我没一刻不是在籌思摧毀魔教。唉,我成崑一生不幸,愛妻爲人所奪,唯一的愛徒,却又視我若仇\dash{}」張無忌聽他提到謝遜,更是凝神注意,可是心志一集中,體内的九陽神功眞氣越加充沛,竟似四肢百骸,無一處不是脹得要爆裂開來,每一根頭髮都好像脹大了幾倍。只聽圓眞續道︰「我下了光明頂後,回到中原,去探訪我多年不見的愛徒謝遜。那知一談之下,他竟已是魔教中的四大護教法王之一,並且還竭力遊説,勸我也加入魔教,説什麼戮力同心,驅除胡虜。我這一氣之下,自是非同小可,但我轉念又想︰魔教源遠流長,根深蒂固,教中高手如雲,以我一人之力,那是決計毀它不了的。别説是我一人,便是天下武林豪傑聯手和它作對,也未必毀它得了。只有從中挑撥,叫它内部自相殘殺,那纔有毀了它的機會。」

楊逍等人聽到這裡,都不禁惕然心驚,這些年來,個個都如睡在墓裡,不知有大敵窺視在旁,處心積慮的要毀滅明教,各人偏生爲了爭奪教主之位,鬧得混亂不堪,圓眞這番話眞如暮鼓晨鐘,發人猛省。只聽他又道︰「當下我不動聲色,只説茲事體大,須得從長計議。過了幾天,我忽然假裝酒酔,意欲逼姦我徒児謝遜的妻子,乘機便殺了他父母妻児全家。我知道這麼一來,他恨我入骨,必定找我報仇。倘若找我不到,更會不顧一切的胡作非爲。哈哈,知子莫若父,知徒莫若師。謝遜這孩児什麼都好,便是易於憤激,不會細細思考一切前因後果\dash{}」

\chapter{挪移乾坤}

張無忌聽到此處,心中憤怒再也不可抑制,暗想︰「原來義父這一切不幸遭遇,全是成崑這老賊在暗中安排。」只聽圓眞得意洋洋的又道︰「謝遜濫殺江湖好漢,到處留下我的姓名,想要逼我出來,哈哈,我那會挺身而出?若要人不知,除非已莫爲,謝遜結下無數冤家,這些血仇自是盡數算在明教的帳上。外敵是樹得彀多了,再加上魔教教主之爭,你們内鬧不休,正好一一墮在我的計中。謝遜没殺了宋遠橋,雖是憾事,但他拳斃少林神僧空見,掌傷崆峒五老,五盤山上傷斃各家各派的好手不計其數,連白眉教的壇主也害了多人\dash{}好徒児啊好徒児\dash{}哈哈哈哈!」

他狂笑聲中,張無忌只覺耳中{\upstsl{嗡}}的一聲猛響,突然暈了過去,但片刻之間,立時又即醒轉。他一生受了無數欺凌屈辱,都能淡然置之,但想義父如此鐵錚錚的一條好漢子,竟在成崑的陰謀毒計之下弄得家破人亡、身敗名裂,盲了雙目孤零零在荒島上等死,這等深仇大恨,豈能不報?

他胸中怒氣一衝,佈滿周身的九陽眞氣更加鼓盪疾走,眞氣呼出不能外洩,那乾坤一氣袋漸漸脹起來,但楊逍等均在凝神傾聽圓眞的説話,誰也没留神這布袋已起了變化。只聽圓眞説道︰「楊逍、周顚、韋一笑,你們再没什麼話説了麼?」楊逍嘆了口氣,道︰「事已如此,還有什麼説的?圓眞大師,你能饒我女児一命麼?她母親是峨嵋派的紀曉芙,出身名門正派,並未屬我明教。」圓眞道︰「斬草除根,養虎貽患。」説著又走前一步,伸出手掌,緩緩往楊逍頭頂拍去。

張無忌在布袋中聽得事態緊迫,顧不得全身有如火焚,縱身一躍,擋在圓眞面前,左掌反手一撩,隔著布袋,將圓眞的一掌架了開去。圓眞被韋一笑打了一招「寒冰綿掌」後,本已身受重傷,這時勉能恢復行動,究竟元氣未復,被張無忌這麼一架,身子一晃,退了一步,喝道︰「好小子!你\dash{}你\dash{}」

張無忌口乾舌燥,全身眞氣越走越快。圓眞一定神,上前一掌向布袋上拍去,這一掌没拍到張無忌身子,被鼓起的布袋一彈,竟是退了一步。他大吃一驚,不明所以,那想到這布袋中的少年竟是身負九陽神功之人。這時張無忌體内的九陽眞氣已脹到即將爆裂,倘若乾坤一氣袋先行炸破,他便能全身脱困,否則駕御不了自己體内猛烈無比的眞氣,勢必肌膚寸裂,焚爲焦炭。

圓眞見布袋古怪,當下踏上兩步,又是一掌擊去,這一次他又退了一步,但那布袋却被他一掌推倒,像個大皮球般在地下打了幾個滾。張無忌人在袋中,立足不定,搖搖晃晃的便如大風浪中的一艘小舟,胸中氣悶,便不如將體内眞氣呼出。可是那布袋中這時也已脹足,要呼出一口氣,竟是越來越難。圓眞發出三拳,踢出兩脚,都被袋中眞氣反彈出來,張無忌在袋中渾然不覺。圓眞這幾下幸好只碰在袋上,要是眞的擊中張無忌身子,此時他體内眞氣充溢,圓眞手足非要受傷不可。

楊逍、彭瑩玉、説不得等見了這等奇景,也都驚得呆了。這乾坤一氣袋是説不得所有之物,他自己却也想不出如何會鼓脹成球,更想不通張無忌在這布球中是死是活。只見圓眞從腰間拔出一柄匕首,一刀向布袋上刺去,那布袋遇到尖刀時只是凹陥一下,却不穿破。要知這布袋的質料奇妙,非絲非革,乃是天地間的一件寶物,圓眞這柄匕首,又非寶刀,連刺數刀,却那裡奈何得了它?

圓眞見掌擊刀刺都是無效,心想︰「跟這小子糾纏什麼?」飛起一脚踢出,那大布袋骨溜溜的從廳門中直滾出去。

這時那布袋已膨脹成爲一個大圓球,在廳門上一撞,立即彈回,疾向圓眞衝去。圓眞見勢道來得猛烈,雙掌豎起,將那大球推開。只聽得砰的一聲大響,猶似晴天打了個霹靂,布片紛飛,這隻乾坤一氣袋已被張無忌的九陽眞氣脹破,炸成了碎片。圓眞、楊逍、周顚等人身前都被一股炙熱之極的氣流一衝,只見張無忌穩穩的站在當地,衣衫破爛,臉露迷茫之色,似對適纔的變故大爲不滿。

原來就在這頃刻之間,他的九陽神功已然大功告成,龍虎相會,天地交泰。要知大布袋内眞氣充沛,等於是數十位高手同時各出眞力,按摩擠逼他周身數百處穴道,這等機緣,自來無人遇到過,而這寶袋一碎,此後也再無人有此巧遇。内内外外的眞氣激盪,他身中數十處玄関一一衝破,這時只覺全身脈絡之中,有如一條條水銀在到處流轉,舒適無比。

圓眞是老奸巨猾、極工心計之輩,眼見張無忌神色不定,正是有機可乘,自己重傷之下,若不抓住良機,只要被對方佔了先手,那就危乎殆哉,當即搶上一步,右手食指伸出,直點他胸口「膻中穴」。張無忌揮掌一擋,這時他神功初成,招數却是平平,前時謝遜和父母所教的武功也尚未融會貫通,如何能和圓眞這種絶頂高手相抗?只一招之間,他手腕上「陽池穴」已被圓眞的一陰指點中,登時機伶伶的打個冷戰,退後了一步。可是他體内充沛欲溢的眞氣,便也在這瞬息間傳到了圓眞指上。這兩種力道一陰一陽,恰好互剋,但張無忌的内力來自九陽神功,遠爲渾厚。圓眞手指一熱,全身功勁如欲散去,再加重傷之餘,平時功力已剩不了一成,知道眼前情勢不利,一驚之下,轉身便走。

張無忌怒罵︰「成崑,你這大惡賊,留下命來!」拔足追出了廳門,只見圓眞背影一晃,已進了一扇側門。張無忌氣憤填膺,發足急追,這一發勁,砰的一響,額頭在門框上重重的撞了一下,原來他自己尚不知神功練成之後,一舉手一提足全比平時多了十餘倍勁力,一大步跨將出去,失了主宰,竟爾撞上門框。他一摸額頭。只覺隱隱有些疼痛,心想︰「怎地這等邪門,這一步跨得這麼遠?」忙從側門中進去,見是一座小廳。他一心一意要和義父復仇,也顧不得圓眞是否會在暗中伺伏襲擊,穿過廳堂,便追了下去。

廳後是一個院子,昏夜中暗香浮動,院子中的花卉送出異香,但見西廂房的窗子中透出火燈之光,張無忌縱身而前,推開房門,眼見灰影一閃,圓眞掀開一張繡帷,奔了進去。張無忌跟著掀帷而入,那圓眞却已不知去向,他凝神一看,不由得暗暗驚奇,原來置身所在竟似是一間大戸人家小姐的閨房。靠窗邉是一張梳妝台,台上紅燭高燒,照耀這房中花團錦簇,堂皇富麗,比之朱九眞家中,更有過之。另一邉是一張牙床,床上羅帳低垂,床前還放著一對女子的粉紅繡鞋,顯然是有人睡在床中。這閨房只有一扇進門,窗戸緊閉,他明明見到圓眞剛纔走進房來,怎地一刹那間變得無影無蹤,竟難道是有隱身法不成?又難道他不顧出家人的身份,居然躱上婦女的床中?

正自打不定主意要不要揭開羅帳搜敵,忽聽得步聲細碎,有人走來,張無忌身子一閃,躱在西壁的一塊掛氈之後,一個女子輕輕咳嗽,有兩個人進了房中。張無忌在掛氈後向外張望,只見兩個都是少女,一個約莫十六七歳,穿著淡黃綢衫,不住的咳嗽,左手扶在另一個少女肩上。那少女年紀更小,只是十四五歳,穿著青衣布衫,是個小鬟,説道︰「小姐你息一息,不要生氣著急!」

那小姐一陣劇烈咳嗽,反手便是一記巴掌,出手甚重,打在那小鬟臉上。那小鬟一個踉蹌,倒退了一步,可是那小姐一雙手搭在她的肩頭,她一倒退,小姐身子一晃,轉過臉來。張無忌在燭光下看得分明,這位小姐眼睛大大,眼球深黑,一張圓臉,正是他萬里迢迢從中原護送來到西域的楊不悔。此時相隔數年,她身材長得高大了,但神態絲毫不改,尤其嘴角邉使小性児時微微撇嘴的模樣,更加分明。只聽她喘著氣罵道︰「你叫我别著急,哼,你自己自然不著急,最好是我爹爹給人整死了,你再害死我,那便是你的天下了。」那小鬟不敢分辯,扶著她坐下。楊不悔道︰「快取我劍來!」

那小鬟走到壁前,摘下掛著的一柄長劍,張無忌見她雙脚之間繫著一根細細的鐵鍊,雙手的手腕上也鎖著一根鐵鍊,又見她左足跛行,背脊駝成弓形,待她摘了長劍回過身時,無忌更是一驚。但見她右目小、左目大,鼻子和嘴角也都扭曲,形狀極是怕人,心下不禁暗暗奇怪︰「這小姑娘相貌之醜,似乎尤在蛛児之上。不過蛛児是因中毒而面目浮腫,總能治愈,這小姑娘天生殘疾,却是醫不了的。」

只見楊不悔接過長劍,咳嗽了兩下,從懷中取出一個藥瓶,倒出兩顆藥丸吃了。張無忌心想︰「原來她藏得有靈丹妙藥,是以身中一陰指後尚能移動,想來定是至陽的熱藥。」果然楊不悔服藥之後,臉上不久便現出紅暈,額頭間冒出絲絲熱氣,她緩緩站起身來,説道︰「扶我去廳上瞧瞧。」那小鬟道︰「敵人恐怕未去,讓我先去探一探風色,再來扶小姐去。」她説話的聲音也是嘶啞難聽,像個粗魯的中年漢子。楊不悔道︰「誰要你假好心,扶著我。」那小鬟無奈,伸出右手來扶。她雙手鎖著,右手伸出,左手便跟著過來。楊不悔左手一翻,已扣住她右手脈門,手指按住她「會宗」「陽池」「外関」三穴,那小鬟全身酸麻,登時動彈不得,顫聲道︰「小姐,你\dash{}你\dash{}」

楊不悔冷笑道︰「我父女受了敵人暗算,命在旦夕之間,你這丫頭還不乘機報復的麼?咱父女豈能受你的折磨?今日先殺了你!」説著長劍翻過,便往那小鬟的頸中刺落。張無忌自見這小鬟周身殘廢,心下便十分可憐於她,突見楊不悔挺劍相刺,正在危急,不及細想,當即飛身而出,手指在劍刃上一彈。楊不悔拿劍不定,叮{\upstsl{噹}}一響,長劍登時落地。她雖在傷後,變招仍快,右手離劍後食中雙指直取張無忌的兩眼,那本來只是平平無奇的一招「雙龍搶珠」,但她一經父親數年調教,使將出來時大具威力。張無忌吃了一驚,向後躍開,衝口便道︰「不悔妹妹,是我!」楊不悔聽慣了他叫不悔妹妹四字,一怔之下,説道︰「是無忌哥哥嗎?」她只認出了「不悔妹妹」這四個字的聲音語調,却没認出張無忌的面貌。張無忌心微感懊悔,但已不能再行抵賴,只得説道︰「是我!不悔妹妹,這些年來你可好?」

楊不悔定神一看,見他衣衫破爛,面目汚穢,心下頗是怔忡不定,道︰「你\dash{}你\dash{}當眞是無忌哥哥麼?怎麼\dash{}怎麼會到這裡?」張無忌道︰「是説不得帶我上光明頂來的。那圓眞和尚到了這房中之後,突然不見,這裡另有出路麼?」楊不悔奇道︰「圓眞走了麼?」張無忌道︰「他被青翼蝠王擊了一掌,身受重傷,我追他到這裡,却不見了。他是我不共戴天的大仇人,非追到他不可。」楊不悔却牽掛著父親,道︰「這房中没另外出路。我瞧爹爹去。」説著順手一掌,往那小鬟的天靈蓋擊落,出手極是狠辣,張無忌驚叫︰「使不得!」伸手在她臂上一推,楊不悔這一掌便落了空。

楊不悔兩次要殺那小鬟,都受到張無忌的干預,心中大怒,厲聲道︰「無忌哥哥,你和這丫頭是一路的嗎?」張無忌奇道︰「她是你的丫鬟,我今日初次見面,怎會和她一路?」楊不悔道︰「你既然不明内情,那就别多管閒事。這丫頭是我家的大對頭,我爹爹用鐵鍊鎖住她手足,便是防她害我。此刻我父女中了敵人的一陰指,這丫頭自然要乘機報復,我父女落在她手裡,那是慘不可言了。」張無忌見這小鬟楚楚可憐,雖然形相奇特,却非兇惡之輩,説道︰「姑娘,你可有乘機報復之意麼?」那小鬟搖了搖頭,道︰「決計不會。」張無忌道︰「不悔妹妹,你聽,她説是不會的,還是饒了她吧!」

楊不悔道︰「好,既然是你講情,啊喲\dash{}」身子一側,搖搖晃晃的立足不定。張無忌知她中一陰指後受傷甚重,忙伸手相扶,突然間後腰「懸樞」「中樞」兩穴上一下劇痛,撲地跌倒。原來楊不悔嫌他礙手礙脚,賺得他近身,以套在中指上的打穴鐵環打了他兩處大穴。她打倒張無忌後,回過右手,便往那小鬟的右太陽穴上擊了下去。

這一下將落未落,楊不悔忽感丹田間寒冷徹骨,全身麻木,不由自主的放脱了那小鬟的手腕,雙膝一軟,坐在椅中。要知她受傷不輕,全仗至陽的妙藥抵擋得一陣,適纔使勁擊打張無忌的穴道,力氣已然用盡,再想打那小鬟,再也無能爲力。只見小鬟拾起地下的長劍,説道︰「小姐,你總是疑心我要害你,這時我要殺你,不費吹灰之力,可是我並無此意。」説著將長劍插入劍鞘,還掛壁間。張無忌站起身來,説道︰「你瞧,我没説錯吧!」原來他被點中穴道之後,片刻間便以眞氣衝解,立即回復行動。楊不悔眼睜睜的瞧著他,心下大爲駭異。

張無忌向那小鬟一揖,説道︰「姑娘,我要追那和尚,你可知此間另有通道麼?」那小鬟道︰「你當眞非追他不可?」張無忌道︰「這人傷天害理,作下了無數罪孼,我\dash{}我\dash{}便到天涯海角,也要追到他。」那小鬟咬著下唇,微一沉吟,點了點頭,一口吹滅了燭火,又取出一塊手帕,遮住楊不悔臉上,然後拉著張無忌的手,在黑暗中走去。

張無忌對任何人都信他不存惡意,這小鬟拉著他手,便即跟了她走,没行數步,已到了床前,那小鬟揭開羅帳,鑽進帳去,拉著張無忌的手却没放開。無忌吃了一驚,心想這小鬟雖然既醜且稚,總是女子,怎可和她同睡一床?何況此刻追敵要緊,當下縮手一掙。那小鬟低聲道︰「通道在床裡!」無忌聽了這個五個字,精神爲之一振,再也顧不得什麼男女之嫌,但覺那小鬟揭開錦被,橫臥在床,便也躺在她的身旁,也不知那小鬟扳動了何處機括,突然間床板一側,兩個人便摔了下去。

這一摔直跌了數丈,幸好地上鋪著極厚的軟草,絲毫不覺疼痛,只聽得頭頂輕輕一響,那床板已然回復原狀。張無忌心下暗讚︰「這機関佈置得妙極!誰能料到祕道的入口處,竟會是在小姐香閨的牙床之中。」拉著小鬟的手,向前急奔,跑出十餘丈,聽到那小鬟足上鐵鍊曳地之聲,猛地想起︰「這位小姑娘是跛子,足上又有鐵鍊,怎地跑得如此迅速?」那小鬟猜中了他的心意,笑道︰「我的跛脚是假的,騙騙老爺和小姐。」張無忌在黑暗中瞧不見她的形貌,心道︰「怪不得我媽媽説天下女子都愛騙人。今日不悔妹妹也來暗算我一下。」此刻忙於追敵,這念頭在心中一轉,隨即撇開,在甬道中曲曲折折的奔出數十丈,突然間到了盡頭,那圓眞却始終不見。

小鬟道︰「這甬道之中,我只到過這裡,相信前面尚有通路,可是我找不到開門的機括所在。」張無忌伸手細細摸索,但覺前面是凹凹凸凸的石壁,没一處縫隙,在凹凸處用力推擊,却是紋絲不動。那小鬟嘆道︰「我已試了幾十次,始終没能找到機括所在,眞是古怪之極。我曾帶了火把進來,細細察看,没發現半點可疑之處。」張無忌心念一動︰「她説没有機括,恐怕當眞没有機括。」提一口氣,運勁雙臂,在石壁左邉用力一推,毫無動靜,再向右邉推時,只覺石壁微微晃了一晃。

無忌大喜,再吸兩口眞氣,使勁推時,那石壁竟然緩緩退後,却是一堵極厚、極巨、極重、極實的大石門。原來光明頂這祕道構築精巧無比,有些地方使用隱祕的機括,但像這塊大石門,却又是全無機括,若非天生神力或是身負絶頂武功之人,萬萬推移不動。所以要如此構築,那是唯恐外人得知祕道的機密,這麼一來,像那小鬟一般雖能進入祕道,但武功不到,仍是只能半途而廢。張無忌這時九陽神功已然練成,這一推之力何等巨大,自能將石壁推開了。待那石壁移後三尺,他呼的拍出一掌,以防圓眞躱在石後偸襲,隨即閃身而入。

過了石壁,前面又是長長的一道甬道,兩人向前走去,只覺那甬道中都是斜坡,越走越下。約莫走了一百餘丈遠行,忽然前面分了幾道岔路。無忌逐一試步,發見岔路竟然共有七條之多,正没做理會處,忽聽得左首前面有人輕輕咳了一下,雖然立即抑止,但靜夜中聽來,已是十分清晰。無忌低聲道︰「走這邉!」搶步往最左一條岔道奔了下去。這條岔道忽高忽低,地下極是難行,無忌急於報仇,也顧不得危機四伏,鼓勇向前,聽得身後鐵鍊曳地的叮{\upstsl{噹}}之聲,響個不絶,便回頭道︰「敵人在前,情勢凶險,你還是慢些來的好。」那小鬟道︰「有難同當,怕什麼?」無忌心想︰「你也來騙我麼?」順著甬道不住左轉,走著螺旋形向下,那甬道越來越窄,到後來僅容一人,便似一口深井,突然之間,張無忌覺得頭頂一股極烈的巨風壓將下來,當下反手一把抱住那小鬟腰間,一縱而下,只聽得砰的一聲巨響,泥沙細石,落得滿頭滿臉。

張無忌定了定神,只聽那小鬟道︰「好險,那賊禿躱在旁邉,推大石來{\upstsl{砸}}壓咱們。」張無忌從斜坡回身走去,右手高高舉在頭頂,只走了幾步,手掌便已碰到粗糙的石面,只聽得圓眞的聲音隱隱從石後傳來︰「賊小子,今日葬了你在這裡,有個女孩児相伴,算你運氣。賊小子力氣再大,瞧你推得開這大石麼?一塊不彀,再加一塊。」只聽得鐵器撬石之聲,接著再是砰一聲巨響,又有一塊巨石被他撬了下來,壓在第一塊巨石之上。那甬道不過僅容一人可以轉身,張無忌伸手一摸,那巨石雖不能將甬道的口子嚴密封住,但最多伸得出一隻手去,身子萬萬不能鑽出。他吸了一口眞氣,雙手挺著巨石一搖,石旁許多沙石撲簌而下,那巨石却是半點不動,看來兩塊數萬斤的巨石疊在一起,當眞便有九牛二虎之力,只怕也拉曳不得。他雖練成九陽神功,究竟人力有時而窮,這等一座小丘般兩塊巨石,如何挪動得它半尺一寸?

只聽圓眞在巨石之外呼呼喘息,却是他重傷之後,使力撬這兩塊巨石,也是累得筋疲力盡,只聽他喘了幾口氣,問道︰「小子\dash{}你\dash{}叫\dash{}叫什麼\dash{}名\dash{}」説到這個「名」字,却又無力再説了。無忌心想︰「這時他便回心轉意,突然大發慈悲,要救我二人出去,也是決不能彀。不必跟他多費唇舌,自看甬道之下是否另有出路。」於是回身而下,順著甬道向前走去。

那小鬟道︰「我身邉倒有火摺,只是没蠟燭火把,生怕一點便完。」張無忌道︰「且不忙點火。」順著甬道只走了數十丈步,便已到了盡頭。兩人四下裡一摸索,張無忌摸到一隻木桶,喜道︰「有了!」手起一掌,將木桶劈散,只覺桶中散出許多粉末,也不知是石灰還是麵粉,他撿起一條木片,道︰「你點火吧!」

那小鬟取出火刀、火石、火線打燃了火,湊過去點那木片,突然間火亮耀眼,木片立時猛烈的燒將起來。兩人嚇了一大跳,鼻中聞到一股硝磺的臭氣。那小鬟道︰「是火藥!」把木片高高舉起,瞧那桶中的粉末時,果然都是黑色的火藥,她低聲笑道︰「要是適纔火星濺了開來,火藥爆炸,只怕連外邉那惡和尚也炸死了。」只見張無忌呆呆望著自己,臉上充滿了驚訝之色,神色極是古怪,便微微一笑,道︰「你怎麼啦?」

張無忌嘆了口氣,道︰「原來你\dash{}你這麼美?」那小鬟抿嘴一笑,道︰「我嚇得傻了,忘了裝怪臉!」説著挺直了身子。原來她既非駝背,更不是跛脚,雙目湛湛有神,修眉端鼻,頰邉微現梨渦,直是秀美無倫,只是年紀幼小,身材尚未長成,雖然容色絶麗,却掩不住容顏中的稚氣。張無忌道︰「爲什麼要裝那副怪樣子?」那小鬟笑道︰「小姐十分恨我,但見到我醜怪的模樣,心中就高興了。倘若我不裝怪樣,她早就殺了我啦。」無忌道︰「她爲什麼要殺你?」那小鬟道︰「她總疑心我要害死她和老爺。」張無忌搖搖頭,道︰「眞是多疑!適纔你長劍在手,她却已動彈不得,你並没害她。自今而後,她再不會疑心你了。」那小鬟笑道︰「我帶了你到這裡,小姐只有更加疑心了。咱們也不知是否能逃得出,她疑不疑心,也不去理她。」

她一面説,一面高舉木條,察看周遭情景。只見處身之地似是一間石室,堆滿了弓箭兵器,大都鐵銹斑斑,顯是明教昔人放在地道之内,以備抵禦外敵。再察看四周牆壁,却無半道縫隙,看來此處是這條岔道的盡頭,圓眞所以故意咳嗽,乃是故意引兩人走入死路。那小鬟道︰「公子爺,我叫小昭。我聽小姐叫你『無忌哥哥』,你大名是叫作『無忌』了?」無忌道︰「不錯,我姓張\dash{}」突然間心念一動,俯身拾起一枝長矛,拿著手中掂了一掂,覺得甚是沉重,似有四十來斤,説道︰「這許多火藥或能救咱們脱險,説不定便能將大石炸了。」小昭拍手道︰「好主意,好主意!」她拍手時腕上鐵鍊相擊,錚錚作聲。張無忌道︰「這鐵鍊礙手礙脚,把它弄斷了吧。」小昭驚道︰「不,不!老爺要大大生氣的。」無忌道︰「你説是我弄斷的,我纔不怕他呢。」説著雙手握住鐵鍊的兩端,用勁一崩。那鐵鍊不過筷子粗細,無忌這一崩少説也有兩三百斤力道,那知只聽得{\upstsl{嗡}}的一聲,震動作響,鐵鍊却是紋絲不動。

無忌「咦」的一聲,吸口眞氣,再加勁力,仍是奈何不得這鐵鍊半分。小昭道︰「這鍊子古怪得緊,便是寶刀利鑿,也傷它不了。鎖上的鑰匙在小姐手裡。」無忌點頭道︰「咱們若是出得去,我向她討來替你開鎖解鍊。」小昭道︰「只怕她不肯給。」無忌道︰「我和她交情非同尋常,她不會不肯的。」説著提起長矛,走到大石之下,側身靜片刻,聽不到圓眞的呼吸之聲,想已遠去。

小昭舉起火把,在旁相照。無忌道︰「一次炸不碎,看來要分開幾次。」當下勁運雙臂,在大石和甬通之間的縫隙中,用長矛慢慢刺了一條孔道,小昭遞過火藥,無忌便將火藥放入孔道之中,倒轉長矛,用矛柄打實,再舖設一條火藥,通到下面石室,作爲引子。

張無忌從小昭手裡接過火把,小昭便伸雙手掩住了耳朶,無忌擋在她的身前,俯身點燃了藥引,眼見一點火花沿著那火藥線向前燒去,猛地裡轟隆一聲巨響,一股猛烈的熱氣衝來,震得無忌向後退了兩步,小昭仰後便倒。無忌早有防備,伸手攬住了她腰,石室中煙霧瀰漫,那火把也被熱氣震熄。無忌道︰「小昭,你没事吧?」小昭咳嗽了幾下,道︰「我\dash{}我没事。」無忌聽她説話有些哽咽,微感奇怪,待得再點燃火把,只見她眼圏児紅了,問道︰「怎麼?你不舒服麼?」

小昭道︰「張公子,你\dash{}你和我素不相識,爲什麼\dash{}爲什麼待我這樣好?」無忌奇道︰「什麼呀?」小昭道︰「你爲什麼要擋在我身前?我是個低三下四的奴婢,你\dash{}你貴重的千金之軀,怎能遮擋在我身前?」無忌微微一笑,道︰「你是個小姑娘,我自是要護著你些児。」見石室中煙霧淡了些,便向斜坡上走去,只見那塊巨石安然無恙,巍巍如故,只炸去了極小的一角,無忌頗爲沮喪,道︰「祇怕要再炸七八次,咱們纔鑽得過去。可是所餘火藥,最多祇能再炸兩次。」提起長矛,又在石上鑽孔,鑽刺了幾下,一矛刺在甬関壁上,忽然一塊岩石滾了下來,露出一孔。無忌又驚又喜,伸手進去,扳住旁邉的巖石,搖了一搖,微微有些晃動,使勁一拉,又扳了一塊下來。他接連扳下三塊巨石,那孔穴已可容身而過。原來甬道的彼端另有通路,這一次爆炸没炸碎大石,却將甬道的石壁碎鬆了。

無忌手執火把先爬了進去,招呼小昭入來。那甬道仍是一路盤旋向下,張無忌這次學得乖了,左手挺著長矛,提防圓眞再加暗算,約莫走了七八十丈,到了一處石門,無忌將長矛和火把交給小昭,運勁推開石門,裡邉又是一間石室。這間石室極大,頂上垂下鐘乳,顯是天然的石洞,無忌走了幾步,突見地下倒著兩具骷髏。骷髏身上的衣服尚未爛盡,可以看得出是一男一女。

小昭似感害怕,挨到無忌身邉。張無忌高舉火把,在這石洞中巡視了一遍,道︰「這裡似乎又是盡頭之處,不知是否能再找到出路?」他伸出長矛,在洞壁上到處敲打,每一處都極沉實,找不到有聲音空洞的地方。他走近兩具骷髏,祇見那女子右手抓著一柄晶光閃亮的匕首,插在她自己胸口。無忌一怔之下,立時想起了圓眞的話來,他和楊夫人在祕道之下相會,給楊破天發見,楊破天憤激之下,走火身亡,楊夫人便以匕首自刎殉夫。「難道這兩人便是楊破天夫婦?」他再走到那男子身前,祇見他手旁攤著一張羊皮。張無忌拾起一看,祇見一面有毛,一面光滑,並無異狀。小昭接了過來,喜形於色,道︰「恭喜公子,這是明教武功的無上心法。」説著伸出左手食指,在楊夫人胸前的匕首上割破一條小小口子,將鮮血塗在羊皮之上,慢慢便顯現了字跡,第一行是「明教聖火心法︰乾坤大挪移」十一個字。

張無忌在無意中發見了明教的武功心法,却並不如何喜歡,心想︰「這祕道中無水無米,倘若走不出去,最多不過七八日,我和小昭便要餓死渴死。再高的武功學了也是無用。」向兩具骷髏瞧了幾眼,又想︰「那圓眞如何不將這乾坤大挪移的心法取了去?想是他做了這件大虧心事後,永遠不敢再來看一眼楊氏夫婦的屍體。當然,他決不知道這張羊皮上竟冩著武功心法,否則别説楊氏夫婦已死,便是活著,他也要來設法盜取了。」問小昭道︰「你怎地知道這羊皮上的祕密?」

小昭低頭道︰「老爺跟小姐説起時,我暗中偸聽到的。他們是明教徒,不敢違犯教規,到這祕道中來找尋。」

\chapter{祕道練功}

張無忌瞧著兩堆骷髏,頗生感慨,説道︰「把他們好好葬了吧。」兩人去搜了些炸下來的泥沙石塊,堆在一旁,再將楊破天夫婦的骸骨搬在一起。小昭忽在楊破天的骸骨中撿起一物,説道︰「張公子,這裡有封信。」無忌接過來一看,見封皮上冩著「夫人親啓」四個字。年深日久,封皮已頗霉爛,那四個字也已腐蝕得筆劃殘缺,但依稀仍可看得出筆致中的英挺之氣,那信牢牢封固,火漆印仍然完好。無忌道︰「楊夫人未及拆信,便已自殺。」將那信恭恭敬敬的放在骸骨之中,正要堆上沙石,小昭道︰「拆開來瞧瞧好不好?説不定楊教主有什麼遺命。」

無忌道︰「祇怕不敬。」小昭道︰「倘若楊教主有何未了心願,公子去轉告老爺小姐,也是好的。」無忌一想不錯,便輕輕拆開封皮,抽出一幅極薄的白綾來,祇見綾上冩道︰

\qyh{}夫人粧次︰自歸楊門,夫人日夕鬱鬱,余粗鄙寡德,無足爲歡,甚可歉咎,茲當永别,唯夫人諒之。三十二代周教主遺命,令余練成乾坤大挪移神功後,前赴丐幫總舵,迎歸第三十一代石教主遺物。今余神功第五層初成,即悉成師弟之事,血氣翻湧,不克自制,眞力將散,行當大歸。命也天也,復何如耶?」張無忌讀到此處,輕輕嘆了口氣,道︰「原來楊教主在冩這信之前,便已知道他夫人和成崑在祕道私會的事了。」見小昭想問又不敢問,於是將楊破天夫婦及成崑間的事簡略説了。小昭道︰「我説都是楊夫不好。她若是心中一直有著成崑這個人,原不該嫁楊教主。既是嫁了楊教主,便不該再和成崑私會。」無忌點了點頭,心想︰「她小小年紀,倒是頗有見識。」繼續讀了下去。

\qyh{}周教主神勇蓋世,智謀過人,仍不幸命喪丐幫四長老之手,石教主遺物不獲歸,本教聖火令始終未得下落。今余命在旦夕,有負周教主重託,實爲本教罪人。盼夫人持余此親筆遺書,召聚左右光明使者、四大護教法王、五行旗使、五散人,頒余遺命日︰『不論何人迎歸石教主遺物,重獲聖火令者,爲本教第三十四代教主。不服者殺無赦!令謝遜暫攝副教主之位,處分本教重務。』」張無忌心中一震,暗想︰「原來楊教主命我義父暫攝副教主之位。我義父文武全才,原是明教中的第一位人物。祇可惜楊夫人没看到這信,否則明教之中也不致如此自相殘殺,鬧得天翻地覆。」他想到楊破天對他義父如此看重,很是喜歡,却又不禁傷感,出神半晌,接讀下去︰

\qyh{}\dash{}乾坤大挪移心法,暫由謝遜接掌,日後轉奉新教主。光大我教,驅除胡虜,行善去惡,持正除奸,新教主其勉之。」無忌心想︰「照楊教主的遺命看來,明教的宗旨實在正大得緊啊,各大門派限於門戸之見,不斷和明教爲難,倒是不該了。」見那遺書上續道︰「余將以僅餘神功,掩石門而和成師弟共處,地老天荒,再不分離。夫人可依祕道全圖脱困。當世無第二人負乾坤挪移之功,即無第二人能推動此『無妄』位石門,待後世豪傑練成,余師兄弟骸骨杇矣。破天謹白。年月日。」

在書信之後,是一幅祕道全圖,註明各處岔道和門戸,無忌大喜,説道︰「楊教主本想將成崑関入祕道,兩人同歸於盡,那知他支持不到,死得早了,讓那成崑逍遙至今。幸好有此全圖,咱們能出去了。」當下細看全圖,找到了自己置身的所在,再一査察,登如一桶冰水從頭上淋將下來,原來唯一的脱困道路,正是被圓眞用大石塞阻了的那一條,雖得祕道全圖,却和不得一般無異。小昭道︰「公子且别心焦,説不定另有通路。」接過圖去,低頭細細査閲。

但那圖上冩得分明,除此之外,更無别處出路,張無忌見小昭臉上露出失望神色,苦笑道︰「楊教主的遺書上説道,倘若練成乾坤大挪移的神功,便可推動石門而出。當世似乎祇有楊逍先生練過一些,可是功力甚淺,就算他在這裡,也未必管用。再説,又不知『無妄位』在什麼地方,圖上也没註明,却到那裡找去?」小昭道︰「『無妄位』嗎?那是伏羲六十四卦的六位之一,乾盡午中,坤盡子中,其陽在南,其陰在北。『無妄』位在『明夷』位和『隨』位之間。」説著在石室中踏了踏方位,走到西北角上,道︰「該在此處了。」張無忌精神爲之一振,道︰「眞的麼?」奔到藏兵器的甬道之中,取過一柄大斧,將石壁上積附的沙土刮去,果然露出一道門戸的痕跡來。他心想︰「我雖不會乾坤大挪移之法,但九陽神功已成,這威力未必便遜於此法。」當下氣凝丹田,勁運雙臂,兩足擺成弓箭步,緩緩推將出去,那石門絶無動靜。不論他雙手如何移動方位,如何催運眞氣,直累得雙臂酸痛,全身骨骼格格作響,那石門仍是宛如生在石壁上般,連一分之微也没移動。

小昭勸道︰「張公子,不用試了,我去把剩下來的火藥拿來。」無忌道︰「好!我倒火藥忘了。」兩人將半桶火藥盡數裝在石門之中,點燃藥引,一炸之後,那石門雖然被炸得凹進七八尺去,甬道却不出現,看來這石門的厚度比寬度還大得多。無忌心中頗爲歉咎,拉著小昭的手,柔聲道︰「小昭,都是我不好,害得你不能出去。」小昭一雙明淨的眼珠望著無忌,説道︰「張公子,你該當怪我纔是,倘若不帶你來\dash{}那便不會\dash{}不會\dash{}」説到這裡,伸袖拭了拭眼泪,過了一會,忽然破涕爲笑,説道︰「咱們既然走不出去了,發愁也是没用。我唱個小曲児給你聽,好不好?」無忌實在没心緒聽什麼小曲,但也不忍拂她之意,微笑道︰「好啊!」

小昭坐在他的身邉,唱了起來︰「依山洞,結把茅,清風兩袖長舒嘯。問江邉老樵,訪山中故友,伴雲外孤鶴,他得志,笑閒人;他失志,閒人笑。」無忌起初兩句並無留意,待得聽到「他得志,笑閒人;他失志,閒人笑」那幾句時,心中驀地一驚,又聽她歌聲嬌柔清亮,圓轉自如,滿腹煩憂,不禁爲之一消,又聽她繼續唱道︰「詩情放,劍氣豪,英雄不把窮通較。江中斬蛟,雲間射鵰,塞外揮刀。他得志,笑閒人;他失志,閒人笑!」悠閒的曲聲之中,又充滿著豪邁之氣,便問︰「小昭,你唱得眞好聽,這曲児是誰做的。」小昭笑道︰「你騙我呢,有什麼好聽?我聽人唱,便把曲児記下了,也不知是誰做的。」無忌想著「英雄不把窮通較」這一句,順著小昭的調児哼了起來。小昭道︰「你是眞的愛聽呢,還是假的愛聽?」無忌笑道︰「怎麼愛聽不愛聽還有眞假之分嗎?自然是眞的。」小昭道︰「好,我再唱一段。要是有琵琶配著,唱起來便順口些。」左手的五根手指在石上輕輕按捺,唱了起來︰

\qyh{}世情推物理,人生貴適意。想人間造物搬興廢。吉藏凶,凶藏吉。」

\qyh{}富貴那能長富貴?日盈昃,月滿虧蝕。地下東南,天高西北,天地尚無完體。」

\qyh{}展放著愁眉,休爭閒氣。今日容顏。老於今日。古往今來,盡須如此,管他賢的愚的,貧的和富的。」

\qyh{}到頭這一身,難逃那一日。受用了一朝,一朝便宜,一朝便宜。百歳光陰,七十者稀。急急流年,滔滔逝水。」

曲中辭意豁達,顯是個飽經憂患、看破了世情之人的胸懷,和小昭的如花年華殊不相稱,自也是她聽旁人唱過,因而記下了,張無忌年紀雖然不大,十年來却是艱苦備嘗,今日困處山腹,眼見已無生理,咀嚼曲中「到頭這一身,難逃那一日」那兩句,不禁魂爲之消。所謂「那一日」,自是身死命喪的「那一日」,無忌以前面臨生死関頭,已不知凡幾,但從前或生或死,都不牽累旁人,這一次不但拉了一個小昭陪葬,而且明教的存毀、楊逍、楊不悔諸人的安危、義父謝遜和圓眞之間的深仇,都和他有関,實在是不想就此便死。

他站起身來,又去推那石門,只覺體内眞氣流動,似乎積蓄著無窮無盡的力氣,可是偏偏使不出來,就像有一條長堤攔住了滔滔洪水,水力被阻,無法宣洩。他試了三次,頹然而廢,只見小昭又已割破了手指,用血塗在那張羊皮之上,説道︰「張公子,你來練一練乾坤大挪移的神功,好不好?説不定你聰明過人,一下子便練會了。」張無忌笑道︰「明教的前任教主們窮終身之功,也没幾個練成的,他們既然當得教主,自是個個才智卓絶。我在旦夕之間,怎能勝越前賢?」小昭低聲道︰「受用了一朝,一朝便宜。便練一朝,也是好的。」無忌微微一笑,將羊皮接了過來,輕輕念誦,只見羊皮上所書,都是運氣導行、移宮使勁的法門,試一照行,竟是毫不費力的便做到了。那羊皮上冩道︰「此第一層神功,悟性特高者七年可成,其次者十四年可成。」無忌大奇︰「這有什麼難處?何以要練七年纔成?」

再接下去看第二層神功的法門,依法施爲,也是片刻間眞氣貫通,只覺十根手指之中,似乎有絲絲冷氣射出。但見心法中特别註明︰第二層神功悟性高者七年可成,次焉者十四年可成,如練至二十一年而無進展,則不可再練第三層,以防走火入魔,無可解救。

無忌又驚又喜,接著第三層練法。這時字跡已然隱晦,他正要取過匕首割自己手指,小昭在一旁瞧著,搶先用指血塗抹羊皮,無忌邉讀邉練,第三層、第四層神功勢如破竹般便練成了。小昭見他半邉臉孔脹得血紅,半邉臉頰却發鐵青,心中微覺害怕,但見他神完氣足,雙眼精光炯炯,料知無疑。待見他讀罷第五層神功心法續練時,臉上忽青忽紅,臉上青時身子微顫,如墮寒冰,臉上紅時額頭汗如雨下。

小昭取出手拍帕,伸到他頭上去替他抹汗,手帕剛碰到他額角,突然間手臂一震,身子一仰,險些児摔倒。無忌站起身來,伸衣袖抹去汗水,一時之間不明其理,却不知自己已然將這第五層神功練成了。

原來這乾坤大挪移神功,實則是運用力的一種極巧妙法門,根本的道理,在於發揮每個人本身所蓄有的潛力。須知每個人體内潛力原極龐大,只是平時使不出來,每逢火災等等緊急関頭,一個手無縛雞之力的弱者往往能負千斤。張無忌練就九陽神功後,本身所蓄的力量,已是當世無人能及,只是他未得高人指點,用不出來,這時一看到乾坤大挪移心法,體内潛力便如山洪突發,沛然莫之能禦。這乾坤大挪移神功所以難成,所以稍一不愼便致走火入魔,完全由於運勁的法門複雜巧妙無比,而練功者却無雄渾的内力與之相副。正好比要一個七八歳的小孩去舞百斤重的大鐵錐,錐法越是精微奥妙,越是會將他自己打得頭破血流,腦漿迸裂,但若舞錐者是個大力士,那便得其所哉了。以往練這神功之人,只因内力有限,勉強修習,變成心有餘而力不足。每個得到這聖火心法的武林健者,又有誰肯知難而退?

昔日的明教各位教主,大都也明白這其中的関鍵所在,但既得身任教主,個個是堅毅不拔,不肯服輸之人。大凡武學高手,都服膺「精誠所至,金石爲開」的話,於是孜孜兀兀,竭力修習,殊不知人力有時而窮,一心想要「人定勝天」,結果往往是飲恨而終。張無忌所以能在半日之間練成,而許多聰明才智武學修爲遠勝於他之人,竭數十年苦修而不能練成者,其間的分别,便在於一則内力有餘,一則内力不足而已。

張無忌練到第五層後,只覺全身説不出的力氣無不指揮如意,欲發即發,欲收即收,一切全憑心意所之,週身百骸,當眞是説不出的舒服受用。這時他已忘了去推那石門,跟著便練第六層的心法,一個多時辰後,已練到第七層。那第七層神功奥妙之處,又比第六層加深了數倍,一時之間不能盡解。好在他精通醫理穴道,遇到難明之處,拿來和醫理一加印證,便即豁然貫通。練到一大半之處,突然間見到一行文字,與張無忌依照試行,猛地裡氣血翻湧,心跳加劇。他定了定神,再從頭做起,仍是如此。他自練第一層神功以來,從未遇上過這等情形。

他跳過了第一句,再練下去時,又覺順利,但三句一過,重遇阻難,自此而下,阻難越來越多,直到篇末,共有一十三句未能照練。張無忌沉思半晌,將那羊皮供在石上,恭恭敬敬的躬身下拜,磕了幾個頭,祝道︰「弟子張無忌,無意中得窺明教神功心法,旨在脱困求生,並非存心窺竊貴教祕籍。弟子得脱險境之後,自當以此神功爲貴教盡力,不敢有負列代教主栽培救命之恩。」

小昭也跪下磕了幾個頭,低聲禱祝道︰「列代教宗在上,請你們保佑張公重整我教,光大列祖列宗的威名。」無忌站起身來,説道︰「我非明教教徒,奉我太師父的教訓,將來也絶不敢身屬明教。但我展讀楊教主的遺書後,知道明教的宗旨光明正大,自當竭盡所能,向各大門派解釋誤會,請雙方息爭。」小昭道︰「張公子,你説有一十三句句子尚未練成,何不休息一會,養足精神,把它都練成了?」張無忌道︰「我今日練成乾坤大挪移第七層神功,雖有一十三句跳過,未免略有缺陥,但正如你曲中所説︰『日盈昃,月滿虧蝕。天地尚無完體。』我何可人心不足,貪多務得?想我張無忌有何福澤功德,該受明教的神功心法?能留下一十三句練之不成,那纔是道理啊。」

小昭道︰「公子説得是。」接過羊皮,請無忌指出那未練的一十三句,暗暗念誦幾遍,記在心中。無忌笑道︰「你記著幹什麼?」小昭臉上一紅,道︰「不幹什麼?我想連公子也練不會,倒要瞧瞧那是怎樣的難法。」其實小昭記誦這一十三句,却是另有深意,她知張無忌坦誠無私,出洞之後,定要將羊皮交給楊逍,他這七層神功中少了一十三句,總是美中不足。因此她暗中記著,將來張無忌若要再練,即使羊皮已屬他人,她也可以背給他聽。

那知道張無忌事事不爲已甚,適可而止,正是應了「知足不辱」這一句話。原來當年創制乾坤大挪移心法的那位高人,内力雖強,却也未到九陽神功的地步,他所冩的第七層神功心法,自己也已無法練成,只不過是憑著聰明智慧,縱其想像,力求變化而已。張無忌練不通的那一十三句,正是那位高人單憑空想而想錯了的,似是而非,已然誤入歧途。要是張無忌存著求全之心,非練到盡善盡美不肯罷手,那麼到最後関頭便會走火入魔,不是瘋癲痴呆,便致全身癱瘓。

當下兩人搬過沙石,葬了楊破天夫婦的遺骸,走到石門之前。

這次張無忌單伸右手,按在石門邉上,依照適纔所練的乾坤大挪移神功心法,微一運勁,那石門便軋軋聲響,再加上一層力,石門緩緩的開了。小昭大喜,跳起身來,拍手叫好,只聽得她手足上鐵鍊相擊,叮叮{\upstsl{噹}}{\upstsl{噹}}的亂響。張無忌道︰「我再拉一拉你的鐵鍊。」小昭笑道︰「這一次定然成啦!」

那知張無忌拉住她雙手之間的鐵鍊,運勁向外拉動,那鐵鍊漸漸延長,却是不斷,小昭道︰「啊喲,不好!你越拉越長,我可更加不便啦。」無忌搖頭道︰「這鍊子當眞邪門,只怕便拉成十幾丈長,它還是不斷。」原來這鐵鍊乃是明教上代教主以一塊殞石鑄成,那殞石不知從那一個星星中落下,其中所含金屬的質地,與世間任何金鐵均是不同,當時適有巧匠,以烈火鑄成此鍊。無忌能將之拉得伸長,已是當世任何旁人力所不及,她見小昭垂頭喪氣,安慰她道︰「你放心,包在我身上,給你打開這鐵鍊。咱們困在這山腹之中,尚能出去,難道還奈何不了這兩根小小鐵鍊嗎?」小昭轉頭一笑,道︰「張公子,你答應我的,將來可不許賴。」無忌道︰「我去求不悔妹妹給你打開鐵鎖,她不會不肯。」張無忌要找圓眞報仇,返身再去推那塊萬斤巨石,可是他雖練成神功,究非無所不能,兩塊巨石被他推得微微撼動,却是決計不能掀開。他搖搖頭,便和小昭從另一邉的石門中走了出去。他回身推攏石門,見那石門那裡是門了,其實是一塊天然生成的大巖石,當年明教建造這地道之時,不知動用了多少人力,窮年累月,不知花了多少功夫,多少心血。

張無忌手持地道祕圖,循圖而行,地道中岔路雖多,但毫不費力的便走出了山洞。一出山洞來,強光閃耀,兩人一霎時間竟然睜不開眼來,過了一會,纔慢慢睜眼,只見遍地冰雪,太陽光照在冰雪之上,反射過來,倍覺光亮。小昭吹熄手中的木條,在雪地裡挖了一個小洞,將木條埋在洞裡,説道︰「木條啊木條,多謝你照亮張公子和我出洞,倘若没有你,咱們可就不籌莫展了。」張無忌哈哈大笑,胸襟爲之一爽,轉念又想︰「世人忘恩負義者多,這小姑娘對一根木條尚且如此,想來當是厚道重義之人。」側頭向她一笑,冰雪上反射過來的強光照在她的臉上,更顯得她膚色潔淨,柔美如玉,不禁讚嘆︰「小昭,你好看得很啊。」小昭喜道︰「張公子,你不騙我麼?」張無忌道︰「你别裝駝背跛脚的怪樣了,現在這樣子纔好看。」小昭道︰「你叫我不裝,我就不裝。小姐便是殺我,我也不裝。」張無忌道︰「瞎説!好端端的,她幹麼殺你?」

他走到崖邉,瞧了四下的地勢,原來是在一座山峰的中腰。他上光明頂時,是説不得將他藏在布袋中負上山來的,沿途地勢如何,一槩不知,此時也不知身在何處。他極目眺望,只見西北方山坡上有幾個人躺著,一動不動,似已死去,忙道︰「我去瞧瞧。」擕著小昭的手,一縱身,向那山坡疾馳而去。這時他體内九陽眞氣流轉如意,乾坤大挪移心法練到了第七層,一舉手,一抬足,在旁人看來,都非人力所能,雖然帶著小昭,仍是身輕如燕,有如兩頭大鳥般向山邉撲了上去。

到得近處,只見四個人死在雪地之中,白雪中濺著鮮血,四個人身上都有刀劍之傷。其中三人穿著明教徒的服色,另一人是個僧人,似是少林派子弟。無忌驚道︰「不好!咱們在山腹中耽了這許多時候,六大派的人攻了上去啦!」一摸四人心口,都已冰冷,顯已死去多時。急忙拉著小昭,跟著雪地裡的足跡向山上跟去。没走出數十丈,又見到有七個人死在地下,情形慘厲可怖。

張無忌大是焦急,説道︰「不知楊逍先生、不悔妹子等怎樣了?」他越走越快,幾乎是將小昭的身子提著飛行,轉了一個彎,只見五名明教徒的屍首,都是頭下脚上的倒懸在樹上,每個人臉上血肉模糊,似被什麼利爪抓過。小昭道︰「是華山派的虎爪手抓的。」無忌奇道︰「小昭,你年紀輕輕,見識却博,是誰教你的?」他這句話雖然問出了口,但記掛著光明頂上各人的安危,不等小昭回答,便即飛步上峰。一路上但見死屍狼藉,大多數是明教的教徒,但六大派的弟子也不在少數。想是張無忌在山腹中的一日一夜之間,六大派發動猛攻。明教因楊逍、韋一笑等重要首領盡數重傷,無人指揮,以致失利,但衆教徒雖在劣勢之下,兀自困鬥不屈,是以雙方死傷均重。

張無忌將到山頂,已聽見兵刃相交之聲,{\upstsl{乒}}{\upstsl{乒}}{\upstsl{乓}}{\upstsl{乓}}的打得極是激烈,他心下稍寬,暗想︰「戰鬥既然未息,六大派或許尚未攻入大廳。」快步往相鬥處奔去,突然間呼呼風響,兩枚鋼鏢向他擲來,跟著有人喝道︰「是誰?停步!」無忌脚下毫不停留,回手輕輕一揮,兩枚鋼鏢倒飛去,只聽得「啊」的一聲慘呼,跟著砰的一聲,有人摔倒在地。張無忌一怔,回過頭來,只見地下倒著一名灰袍僧人,兩枚鋼鏢穿過他肩頭,釘在地上。無忌更是一呆,他適纔這麼回手一揮,只不過想掠斜鋼鏢來勢,不致打到自己身上而已。那料到這輕輕一揮之勢,竟是如此大得異乎尋常。他急忙搶上前去,歉然道︰「在下誤傷大師,抱歉之至。」伸指一提,挾起鋼鏢。

那少林僧雙肩上登時血如泉湧,豈知這僧人極是驃悍,飛起一脚,砰的一聲,踢在張無忌小腹之上。無忌和他站得極近,没料到他竟會突施襲擊,一呆之下,那僧人已然倒飛去去,背脊撞在一棵樹上,右足折斷,口中狂噴鮮血。原來張無忌此時體内眞氣流轉,一遇外力,自然而然而生反擊。他見那僧人重傷,心下更是不安,上前扶起,連連道歉,那僧人惡狠狠的瞪著他,驚駭之心更甚於憤怒,雖然仍想出招擊敵,却已無能爲力了。忽聽得圍牆之内傳出接連三聲悶哼,張無忌無法再顧那僧人,拉著小昭,便從大門中搶了進去,穿過兩處廳堂,眼前是好大一片廣場。場上黑壓壓的站滿了人,西首人數較少,衣衫上鮮血淋漓,十之八九已然負傷,或坐或臥,是明教的一方。東首的人數一瞥之下,見楊逍、楊不悔、韋一笑、説不得諸人都坐在明教人衆之内,看情形仍是動彈不得。廣場之中,有兩個人正在拚鬥,各人凝神觀戰,無忌和小昭進來,誰也没加留心。張無忌慢慢走近,定神一看,只見相鬥的雙方都是空手,但掌風呼呼,威力遠不及數丈之外,顯然二個都是絶頂的高手。那兩人身形轉動,打得快極,突然間四掌相交,立時膠住不動,這般自奇速的躍動轉爲全然靜止,只在一瞬之間,旁觀衆人忍不住轟天價叫了一聲︰「好!」無忌看清楚兩人的面貌時。心下一震,原來那身材矮小、滿臉精悍之色的中年漢子,正是武當派的四俠張松溪。他的對手是個身材魁偉的禿頂老者,長眉勝雪,垂下眼角,鼻子鉤曲,有若鷹嘴。無忌心想︰「明教中還有這等高手,那是誰啊?」忽聽得華山派中有人叫道︰「白眉老児,快認輸吧,你那裡是武當派張四俠的對手。」張無忌聽到「白眉老児」四個字,心念一動︰「啊,原來他是我外公白眉鷹王!」胸中立時生出一股孺慕之意,便想撲上前去相認。

但見白眉鷹王殷天正和張松溪頭頂都冒出絲絲熱氣,兩人便在這片刻之間,竟已各出生平苦練的内家眞力,一決生死。一個是白眉教教主,明教的四大護教法王之一,一個是張三丰的得意弟子,身屬威震天下的武當七俠,眼看這一場比拚,不但是白眉教和武當派雙方威名所繫,而且高手以眞力決勝,敗的一方多半有性命之憂。只見兩個人猶似兩尊石像,連頭髮和衣角也無絲毫飄拂。殷天正神威凜凜,雙目炯炯,如電閃動,張松溪却是謹守武當心法中「以逸待勞、以靜制動」的要旨,嚴密守衛。他知殷天正比自己大了二十多歳,内力修爲是深了二十餘年,但自己正當壯年,長力充沛,對方年已衰邁,時間一久,便有取勝之機。

豈知殷天正實是當代武林中一位不世出的奇人,年紀雖大,精力絲毫不遜於少年,内力如潮,一個浪頭又是一浪頭般,從雙掌上向張松溪撞擊過去。

張無忌初見張松溪和殷天正時,心中一喜,但立即喜去憂來,一個是自己外公,和他有骨肉之親,一個是父親的師兄,待他有如親子,當年他身中玄冥神掌後,武當諸俠個個不惜損耗内功,盡心竭力的爲他療傷,倘若兩人之中有一人或傷或死,在他都是畢生大恨。張無忌微一沉吟,正想搶上去設法拆解,忽聽得殷天正和張松溪同時大喝一聲,四掌發力,一齊向後退出了六七步。

張松溪道︰「殷老前輩神功卓絶,佩服佩服!」殷天正聲若洪鐘,説道︰「張兄的内家修爲超凡入聖,老夫自愧不如。閣下是小婿同門師兄,難道今日定然非分勝負不可麼?」張無忌聽他言語中提到父親,眼眶登時紅了,心中不住叫著︰「别打了,别打了!」張松溪道︰「晩輩適纔多退一步,已輸半招。」兜頭一揖,神定氣閒的退了下去。突然武當派中搶出一個漢子,指著殷天正怒道︰「殷老児,你不提我張五哥,那也罷了!今日提起,叫人好生惱恨。我兪三哥、張五哥兩人,全是傷折在你白眉教手中,此仇不報,我莫聲谷枉居『武當七俠』之名。」嗆{\upstsl{啷}}{\upstsl{啷}}一聲,長劍出鞘,太陽照耀下劍光閃閃,擺了一招「萬嶽朝宗」的姿式。這是武當子弟和長輩動手過招時的起手式,莫聲谷雖然怒氣勃勃,但他此時早已是武林中極有身份的高手,在衆目睽睽之下,一舉一動,自不能失了禮數。

殷天正嘆了口氣,臉上閃過一陣黯然之色,緩緩的道︰「老夫自小女死後,不願再動刀劍。但若和武當諸俠空手過招,却又未免托大不敬。」指著一個手執鐵棍的教徒道︰「借你的鐵棍一用。」那明教徒雙手橫捧齊眉鑌鐵棍,恭恭敬敬的走到殷天正身前,躬身呈上。殷天正接過鐵棍,雙手一拗,拍的一聲,那鐵棍登時斷爲兩截。旁觀衆人「哦」的一聲,没想到這老児在久戰之後,仍具如此驚人神力。

莫聲谷知他不會先行發招,長劍一起,發一招「百鳥朝鳳」,但見到劍尖亂顫,霎時間便如化爲數十個劍尖,罩在敵人中盤,這一招雖然厲害,但仍是彬彬有禮的君子劍法。殷天正左手斷棍一封,説道︰「莫七俠不必客氣。」右手斷棍便斜{\upstsl{砸}}下來。數招一過,旁觀衆人群聳動,但見莫聲谷劍走輕靈,光閃如虹,吞吐開闔之際,又飄逸,又凝重,的是名家風範。殷天正的兩根斷棍本已笨重,招數更是呆滯,東打棍,西{\upstsl{砸}}一棍,當眞不成章法,但有識之士一看之下,便知他大智若愚,大巧若拙,實已臻武學中的極高的境界。他脚步移動也極緩慢,莫聲谷却縱高伏低、東奔西閃,只在一盞茶時分,已接連攻出六十餘招凌厲無倫的殺手。

\chapter{正邪決鬥}

再戰數十合後,莫聲谷的劍招愈來愈快,崑崙、峨嵋諸派均以劍法見長,這幾派的弟子見莫聲谷一柄長劍生出如許變化,心下都是暗暗欽服︰「武當劍法果然名不虛傳,今日裡大開眼界。」可是不論他如何騰挪劈刺,總是攻不進殷天正的兩根鐵棍之内。莫聲谷心想︰「這老児連敗華山、少林的三名高手,又和四哥對耗内力,我已是跟他相鬥的第五人,早就佔了不少便宜,若再不勝,師門的顏面何存?」猛地裡一聲清嘯,劍法忽變,那柄長劍竟似成了一條軟帶,輕柔曲折,飄忽不定,正是武當派的七十二招「繞指柔劍」。

旁觀衆人看到第十二三招時,忍不住齊聲叫起好來,這時殷天正已不能守拙待巧,身形遊走,也展開輕功,跟他以快打快,突然間莫聲谷長劍破空,刺向殷天正胸膛,劍到中途,劍尖微顫,竟是彎了過去,斜刺他的右肩。要知使這「繞指柔劍」,全仗以渾厚内力逼彎劍刃,使劍招閃爍無常,敵人便難以擋架。殷天正也從未見過這等劍法,急忙沉著相避,不料錚的一聲輕響,那劍反彈過來,直刺入他左手上臂。殷天正右臂一伸,不知如何,竟爾陡然間長了半尺,在莫聲谷手腕上一拂,挾手將他長劍奪過,左手便已按住他的「肩貞穴」。

白眉鷹王的鷹爪擒拿手乃是百餘年來武林中的一絶,當世無雙無對,莫聲谷肩頭落入他的掌心,他五爪只須輕輕一捏,莫聲谷的肩頭非碎成片片、終身殘廢不可。武當諸俠大吃一驚,待要搶出相助,只見殷天正嘆了口氣,道︰「一之甚爲,其可再乎?」放開了手,右手一縮,拔出長劍,左臂上傷口鮮血如泉湧出。他向長劍凝視半晌,説道︰「老夫縱橫半生,從未在招數上輸過一招半式。好張三丰,好張三丰!」他稱揚張三丰,那是欽佩他手創的七十二招「繞指柔劍」果然神妙難測,自己竟然擋架不了。

莫聲谷呆在當地。自己雖然先贏一招,但對方終究是故意的不下殺手,没損傷自己,怔了片刻,便道︰「老前輩手下留情。」殷天正一言不發,將長劍交還給他。莫聲谷精研劍法,但到頭來手中兵刃竟給對方奪去,心下羞愧難當,也不接劍,便即退下。

張無忌輕輕撕下衣襟,正想上去給外公裹傷,忽見武當派中步出一人,黑鬚垂胸,道家打扮,却是武當七俠之首的宋遠橋,説道︰「我替老前輩裹一裹傷。」從懷中取出金創藥,給他敷在傷口之上,隨即用帕子紮住。白眉教和明教的教衆見宋遠橋一臉正氣,料想他以武當七俠之首的身份,決不會公然下毒加害。殷天正説了聲︰「多謝!」更是坦然不疑。張無忌大喜。心道︰「宋師伯給我外公裹傷,想是感激他不傷莫七叔,兩家就此和好了。」那知宋遠橋裹好傷後,退開一步,長袖一擺,説道︰「宋某領教老前輩的高招!」

這一著大出張無忌意料之外,忍不住叫道︰「用車輪戰打他老人家,這不公平!」他一言出口,衆人的目光都射向這衣衫襤褸的少年,除了峨嵋諸人,以及宋青書,殷利亨、説不得等少數幾人外,誰都不知他的來歷。宋遠橋道︰「這位小友之言不錯,武當派和白眉教之間的私怨,今日暫且攔下不提。現下是六大派和明教一決生死存亡的関頭,武當派謹向明教討教。」

殷天正眼光緩緩移動,看到楊逍、韋一笑等人全身癱瘓,五行旗下的高手個個非死即傷,自己的児子殷野王伏地昏迷、生死未卜,明教和白眉教之中,除了自己一人之外,再無一個能抵擋得住宋遠橋的拳招劍法,可是自己連戰五個高手之餘,已是眞氣不純,何況左臂上這一劍受傷實是不輕。

殷天正微微一頓之間,崆峒派中一個矮小的老人大聲説道︰「魔教勢窮力絀,再不投降,還待怎的?空智大師,咱們這便去毀了魔教三十三代教主的神位吧!」原來少林寺方丈空聞大師坐鎭嵩山本院,這次圍剿魔教,少林小弟由空智率領,各派敬仰少林派在武林中的聲望地位,便舉他爲進攻光明頂的發號施令之人。空智尚未答言,只聽華山派中一人叫道︰「什麼投降不投降?魔教人衆,今日不能留下一個活口。要知除惡務盡,否則他日死灰復燃,又是爲害江湖,魔崽子們!見機的快快自刎,免得大爺們動手。」

殷天正暗暗運氣,但覺左臂上劍刺及骨,一陣陣作痛,素知宋遠橋追隨張三丰最久,已得這位不世出的武學大師眞傳,自己神完氣足之時和他相鬥,也是未知鹿死誰手,何況此刻?但明教衆高手或死或傷,只剩下自己一人支撐大局,只有拚掉這條老命了,自己死不足惜,所可惜者一世英名,竟在今日斷送。只聽宋遠橋道︰「殷老前輩,武當派和白眉教仇深似海,可是咱們却不願乘人之危,這場過節,儘可日後再行清算。咱們六大派這一次乃是衝著明教而來,白眉教已脱離明教,自立門戸,江湖上人人皆知,殷老前輩何必淌這場渾水?還請率領貴教人衆,下山去吧!」

武當派爲了兪岱岩之事,和白眉教結下極深的樑子,此事各派盡皆知聞,這時聽宋遠橋竟然替白眉教開脱,各人先是驚訝,隨即明白宋遠橋光磊落,不肯撿現成的便宜。殷天正哈哈一笑,説道︰「宋大俠的好意,老夫心領。老夫是明教四大護教法王之一,雖也自樹門戸,但本教有難,豈能置身事外?今日有死而已,宋大俠請進招吧!」説著踏上一步,立個旗鼓。兩條白眉微微顫動,凜然生威。

宋遠橋道︰「得罪了!」説罷左手一揚,右掌抵在掌心,一招「請心式」揮擊出去。這一招是武當派長拳中晩輩和長輩過招的招數。其時武當派創派未久,子弟不多,所有拳招劍法都是張三丰别出杼機的創始,並不依傍前人門戸,所以宋遠橋這一招出去,殷天正從未見過,不過見他彎腰弓背,微有下拜之態,料知是禮敬的起手,便道︰「不必客氣。」雙手一圏,封在當胸。依照拳理,宋遠橋必當搶步前,伸臂出擊,那知他伸臂出擊是一點不差,却没搶步上前,這一拳打出,竟和殷天正的身子相距一丈有餘。殷天正一驚︰「難道他武當拳術如此厲害,竟已練成了隔山打牛的神功?」當下不敢怠慢,右掌揮了出去,抵擋他的拳力。

不料這一掌揮出,前面空空蕩蕩,並未接到什麼勁力,不由得心中大奇,只聽宋遠橋道︰「久仰老前輩武學深湛,家師也常稱道。但此刻前輩已力戰數人,晩輩却是生力,過招之際太不公平。咱們只較量招數,不比膂力。」一面説,一面踢出一腿,這一腿又是虛踢,離對方身子仍有丈許之地,但這一腿的脚法極是精妙,方位奇特,當眞是匪夷所思,倘是近身攻擊,可就十分難防。殷天正讚道︰「好脚法!」以攻爲守,揮拳搶攻,宋遠橋側身閃避,還了一掌。

霎時之間,但見兩人拳來脚往,鬥得極是熱鬧,可是始終相隔丈許之地。雖然招不著身,一切全是虛打,但他二人何等身份,那一招失利、那一招佔先,各自心知,兩人全神貫注,絲毫不敢怠忽,便和貼身肉搏一模一樣。旁觀衆人不少是武學高手,只見宋遠橋走的是以柔克剛的路子,拳脚出手却是極快,殷天正大開大闔,招數以剛爲主。兩人見招拆招,忽守忽攻,似乎是分别練拳,各打各的,其實是鬥得激烈無比。

張無忌初看殷天正和張松溪、莫聲谷兩人相鬥時,関懷兩邉親人的安危,並没怎麼留神雙方出招,這時見殷天正和宋遠橋相隔得遠遠的相鬥,知道只有勝負之分,却無死傷之險,這纔潛心察看兩人的招數。看了半晌,但見兩人出招越來越快,他心下却是越來越加不解︰「我外公和宋大師伯都是武林中一流高手,但招數之中,何以竟存著這許多破綻?外公這一拳倘若偏左半尺,不就正打中宋師伯的胸口?宋師伯這一抓若是再遲出片刻,那不恰好拿中了我外公左臂?難道他二人故意相讓?可是瞧情形又不像啊。」

其實殷天正和宋遠橋雖然離身相鬥,招數上却是絲毫不讓,張無忌學會乾坤大挪移心法後,武學上的修爲已比他們均要勝了一籌,但説殷、宋二人的招數中頗有破綻,却又不然。原來張無忌這麼想,那是他是身負九陽神功之故,他所設想的招數雖比殷宋二人更精更妙,但常人却萬萬無法做到。正如飛禽見地下獅虎搏鬥,不免會想︰「何不高飛下撲,可制必勝?」殊不知獅虎在百獸之中,最爲兇猛厲害。但要他們高飛下撲,却是力所不能。張無忌見識未彀廣博,是以一時想不到其中的緣故。

忽見宋遠橋招數一變,雙掌飛舞,有若絮飄雪揚,軟綿綿不著力氣,正是武當「綿掌」。殷天正正呼喝一聲,打出一拳,兩人一以至柔,一以至剛,各逞絶技。鬥到分際,宋遠橋左掌拍出,右掌陡地裡後發先至,跟著左掌斜穿,又從後面搶了上來。殷天正見自己上三路全被他掌勢罩住,大吼一聲,雙拳「丁甲開山」,揮擊出去。兩人雙掌雙拳,便此膠在空中,四條手臂作著鬥力之狀,此時看來似乎古怪,但若是近身眞鬥,却是面臨最爲兇險的関頭。宋遠橋微微一笑,收掌後躍,説道︰「老前輩拳法精妙,佩服佩服!」殷天正也即收拳,説道︰「武當掌法,果然冠絶今古。」兩人説過不内力,鬥到此處,無法再行繼續,便以和局收場。

武當派中尚有兪蓮舟和殷利亨兩大高手未曾出場,只見殷天正臉頰脹紅。頭頂冒著熱氣,適纔這一場比試雖然不耗内力,但對手實在太強,却已是竭盡心智,眼見他已是強弩之末,兪殷二俠任何一人下場,立時便可將他打倒,穩享「打敗白眉鷹王」的美譽。兪蓮舟和殷利亨對望一眼,都搖了搖頭,心中均想︰「乘人之危,勝之不武。」

他武當二俠認爲乘人之危,勝之不武,旁人却未必都有君子之風,只見崆峒派中一個矮小老人一縱而出,正是那高叫焚燒明教歷代教主神位之人,輕飄飄的落在殷天正正面前,説道︰「我姓唐的跟你殷老児玩玩!」説話的口氣極是輕薄。殷天正向他橫了一眼,鼻中一哼,心道︰「若在平時,崆峒五老如何在殷某眼下?今日虎落平陽被犬欺,殷某一世英名,若是斷送在武當七俠手底,那也罷了,可萬萬不能讓你唐文亮豎子成名!」雖然全身骨節酸軟,只盼睡倒,就此長臥不起,但胸中豪氣一生,下垂的兩道白眉突然豎起,喝道︰「小子,進招吧!」

唐文亮瞧出他内力已耗了十之八九,只須跟他鬥得片刻,不用動手,他自己就會跌倒,當下雙掌一錯,搶到殷天正身後,一拳往他後心擊去。殷天正斜身一勾,唐文亮已然躍開,他脚下靈活之極,猶如一隻猿猴,不斷的跳躍。鬥了數合,殷天正眼前一黑,喉頭微甜,一口鮮血噴了出來,再也站立不定,一交坐倒。唐文亮大喜,喝道︰「殷天正,今日叫你死在我唐文亮拳下!」

張無忌只見唐文亮縱起身子,凌空下擊,正要飛身過去救助外公,却見殷天正右手斜翻,姿式妙到巓毫,正是對付敵人從上空進攻的一招殺手,眼看兩人處此方位之下,唐文亮已是無法自救。果然聽得喀喀兩響,唐文亮雙臂已被殷天正施展「鷹爪擒拿手」折斷,砰的一響,摔在殷天正身旁數尺之外。他四肢齊折,再也動彈不得。旁觀衆人見殷天正在重傷之餘,仍具如此神威,無不駭然。

崆峒五老中的第三老唐文亮如此大受挫敗,崆峒派人人臉上無光,眼見唐文亮躺在殷天正身畔,只因相距過近,竟然無人敢上前扶他回來。過了半晌,崆峒派中一個弓著背的高大老人重重踏步而出,右足踢起一塊石頭,直向殷天正飛去,口中喝道︰「白眉老児,我姓宗的跟你算算舊帳。」原來這人是崆峒五老的中第二老,名叫宗維俠。他説「算算舊帳」,想是曾吃過殷天正的虧。

這塊石頭飛去,禿的一聲,正中殷天正的額角,立時鮮血長流。這一下誰都大吃一驚,宗維俠踢這石頭過去,原也没想能擊中他,那知殷天正已是半昏半醒,没能避讓。當此情勢之下,宗維俠上前只須輕輕一指,便能致殷天正於死地,但見他踏上一步,武當派中走出一人,身穿土布長衫,神情質樸,却是兪蓮舟,身形微晃,便已攔在宗維俠身前,説道︰「宗兄,殷教主已身受重傷,勝之不武,不勞宗兄動手。殷教主跟敝派過節極深,這人交給小弟罷。」宗維俠道︰「什麼身受重傷?這人最會裝死,適纔若不是他故弄玄虛,唐三弟那會上他這個惡當。兪二俠,貴派和他有樑子,兄弟跟這老児也有過節,讓我先打他三拳出出氣。」兪蓮舟不願殷天正一世英雄,如此喪命,説道︰「宗兄的七傷拳天下聞名,殷教主眼下是這般模樣,那裡還禁得起宗兄的三拳?」宗維俠道︰「好!他折斷我唐三弟四肢,我也打斷他四肢便了。這叫做眼前報,還得快!」

他見兪蓮舟兀自猶豫,大聲説道︰「兪二俠,咱們六大派來西域之前,立過盟誓。今日你反而迴護魔教的頭子麼?」兪蓮舟嘆了口氣,説道︰「此刻任憑於你。回歸中原以後,我再領教宗大先生的七傷拳神功。」宗維俠心下一凜︰「這姓兪的何以一再維護於他?」他對武當派確實是頗有忌憚,但衆目睽睽之下,終不能示弱,當下冷笑道︰「天下事抬不過一個理字。你武當派再強,也不能恃勢橫行啊。」這幾句話寢寢然涉到了張三丰身上,宋遠橋便道︰「二弟,由他去吧!」兪蓮舟朗聲道︰「好英雄,好漢子!」便即退開。這「好英雄,好漢子」六個字,似乎是稱讚殷天正,又似是譏刺宗維俠的反話。但宗維俠不願和武當派惹下糾紛,假裝没有聽見,一見兪蓮舟走開,便向殷天正身前走去。只聽得少林派的空智大師發令道︰「華山派和崆峒派各位,請將場上的魔教餘孼一槩誅滅了。武當派從西往東搜索,峨嵋派從東往西搜索,别讓魔教有一人漏網。崑崙派預備火種,焚燒魔教巢穴。」他吩咐了五派後,雙手合什,説道︰「少林子弟各取法器,誦念往生經文,替六派殉難的英雄、魔教教衆超度,化除冤孼。」

衆人只待殷天正在宗維俠一拳之下喪命,則六派圍剿魔教的豪舉便即大功告成。只見明教的衆教徒一齊掙扎爬起,除了身受重傷無法動彈者外,各人盤膝而坐,雙手十指張開,舉在胸前,作火燄飛騰之狀,一齊跟著楊逍,唸誦明教的經文︰「焚我殘軀,熊熊聖火。生亦何歡,死亦何苦?爲善除惡,惟光明故。喜樂悲愁,皆歸塵土。憐我世人,憂患實多!憐我世人,憂患實多!」

明教衆高手自楊逍、韋一笑、説不得諸人以下,直至厨工伕役,個個神態莊嚴,絲毫不以身死教滅爲懼。空智大師合什道︰「善哉!善哉!」兪蓮舟心道︰「這幾句經文,想是他魔教教衆每當身死之前所要唸誦的了。他們不念自己身死,却在憐憫衆人多憂多患,那實在是大仁大勇的胸襟啊。當年創設明教之人,眞是個了不起的人物,只可惜傳到後世,反而變成了爲非作歹的淵藪。」

張無忌在六大門派高手之前,本來心存畏懼,遲遲不敢挺身而出,待見那宗維俠要上前擊死外公,空智又下了屠盡明教人衆的號令,當下顧不得人寡力薄,大踏步搶了出去,擋在宗維俠身前,説道︰「且慢動手!你如此對付一個身受重傷之人,也不怕人笑你懦弱無恥麼?」這幾句話聲音清朗,震憾全場,各派人衆本已奉了空智大師的號令,便要分别出手,突然聽到他這幾句話,一齊停步,回頭瞧著他。

宗維俠見説話的是個衣衫襤褸的少年,伸手一推,想要將他推在一旁,以便上前打死殷天正。張無忌練成第七層乾坤大挪移神功後,勁力收發由心,不論對方以任何精妙的拳招攻來,都能圓轉如意的應付。宗維俠一掌推到,張無忌隨手一掌拍出,砰的一響,宗維俠倒退三步,待要站定,豈知對方這一掌的掌力雄渾無比,仍是立足不定,幸好他下盤功夫扎得堅實,但覺到上身一直後仰,急忙右足在地下一點,縱身後躍,借勢縱開丈餘。落下地來時,這股掌勢仍未消解,又是踉踉蹌蹌的連退七八步,這纔站定。這麼一來,他和張無忌之間已相隔三丈以上。他心中驚怒莫名,旁觀衆人却是大惑不解,都想︰「宗維俠這老児在鬧什麼玄虛,怎地又退又躍,躍了又退,大搗其鬼?」便是張無忌自己,也想不透自己這麼輕輕一掌,竟有如此威力。

宗維俠呆了一呆,登時醒悟,向兪蓮舟怒目而視,喝道︰「大丈夫光明磊落,豈可暗箭傷人?」他料定是兪蓮舟在暗中相助,説不定還是武當諸俠一齊出手,否則單憑一人之力,未必能有這麼強的勁道。兪蓮舟給他説得莫名其妙,好在自己未曾對他偸襲,由得他胡説八道,反瞪他一眼,暗道︰「你裝模作樣,想幹什麼?」宗維俠大步上前,指著張無忌,喝道︰「小子,你是誰?」張無忌道︰「我叫曾阿牛。」一面説,一面伸掌貼在殷天正背心「靈台穴」上,一股熱力源源輸入他的體内。他的九陽眞氣何等渾厚,殷天正顫抖了幾下,便即睜開眼來,望著無忌,頗爲奇怪。無忌向他微微一笑,加緊輸送内力。殷天正正是武學大行家,如何不察覺到這股驚人的内力,宗維俠没走到身前,他胸口和丹田中閉塞之處已然暢通無阻,低聲道︰「多謝小友!」站起身來,傲然道︰「姓宗的,你崆峒派的七傷拳有什麼了不起,我便接你三拳!」宗維俠萬没想到霎時之間,他竟又神完氣足的站起身來,眼看這個現成的便宜是不易撿的了,心中忌憚著他「鷹爪擒拿功」的厲害,便道︰「崆峒派的七傷拳固然没什麼了不起,便請你接我三拳。」他盼殷天正和他拳相對,各比内力,那麼他自己以逸制勞,當可仗著七傷拳的神功取勝。

張無忌聽他一再提起「七傷拳」三字,想起在冰火島的那天晩上,義父叫醒自己,説及以七傷拳打死神僧空見的事,後來他叫自己背誦七傷拳的拳訣,還因一時不能記熟,挨了他好幾個耳光。這時那拳訣在心中流動,當即明白了其中的道理。要知九陽神功中包含了天下所有的内功,而乾坤大挪坤運勁使力的法門,又是集任何武功的大成,一法通,萬法通,任何武功在他面前都已無祕奥之可言。

只聽殷天正道︰「别説三拳,便接你三十拳却又怎地?」他回頭大聲向空智説道︰「空智大師,姓殷的還没死,還没認輸,你便出爾反爾,想要倚多取勝麼?」空智左手一揮,道︰「好!大夥児稍待片刻,又有何妨?」原來殷天正上得光明頂後,見楊逍等人盡皆重傷,己方勢力單薄,當下以言語擠住空智,不得仗著人多混戰。空智依著武林規矩,便約定逐一對戰,結果白眉教和五行旗的人衆,還是一個個的非死即傷,最後剩下的殷天正一人。但他既未認輸,便不能一擁上前的大加屠戮。

張無忌知道外公雖比先前好了些,却萬萬不能運勁使力,他所以要接宗維俠的拳招,只不過是護教力戰,死而後已,於是低聲道︰「殷老前輩,待我來替你先接,晩輩不成之時,老前輩再行出馬。」殷天正已瞧出他内力深厚無比,自己便他絶無傷勢之下,也是萬萬及不上他,但想自己爲教而死,理所當然,這少年不知有何干係,他本領再強,也決計敵不過對方敗了一個又來一個、源源不絶的人力,到頭來還不是和自己一樣,重傷力竭,任人宰割,如此少年英才,何必白白的斷送在光明頂上?當下問道︰「小友是那一位門下?似乎不是本教教徒,是嗎?」張無忌道︰「晩輩不屬明教,但對老輩心儀已久,今日和前輩並肩拒敵,乃是份所應當。」

殷天正大奇,正想再問,宗維俠又已踏上一步,大聲道︰「姓殷的,我第一拳來了。」張無忌道︰「殷前輩説不配跟他比拳,你先勝得過我,再跟他老人家動手不遲。」宗維俠大怒,喝道︰「你這小子是什麼東西?叫你知道崆峒派七傷拳的厲害。」張無忌心念一動︰「今日如要退敵,只有説明圓眞這惡賊的奸詐陰謀,然後才能設法使雙方罷手,若是當眞動手過招,我一人怎打得過六大門派這許多英雄?何況武當門下的衆師伯叔都在此地,我又怎能跟他們爲敵?」當下朗聲説道︰「崆峒派七傷拳的厲害,在下早是久仰的了。少林神僧空見大師,不就是喪生在貴派七傷拳之下麼?」

他此言一出,少林派群相聳動。他們得知空見大師是死在謝遜之手,這次少林派高手盡出,前來圍剿魔教,主要便是爲他復仇。但空見大師的屍首全身骨骼盡數震斷,外表一無傷痕,極似是中了崆峒派「七傷拳」的毒手。當時空聞、空智、空性三僧密議數日,認爲崆峒派眼下並無絶頂高手,能彀打死練就了「金剛不壞體」神功的空見大師,雖然空見的傷痕令人起疑,但料想非崆峒派所爲。後來空性大師也曾率領子弟,暗加訪査,得知空見大師在洛陽圓寂之日,崆峒五老均在西南一帶。既然非五老所爲,那麼崆峒派中更無其他好手能對空見神僧有絲毫損傷,因此便將對崆峒派所起的疑心擱下了。何況當時謝遜曾在洛陽客房外的牆上,咬破指頭,冩上成崑殺神僧空見於此牆下十一個大字,少林派後來査知冒名成崑做下無數血案的均是謝遜所爲,那是半點也没疑惑了。直到此時聽張無忌這句話,心下才各自一凜。

宗維俠怒道︰「空見大師爲謝遜惡賊所害,江湖上衆所週知,跟我崆峒派有什麼干係?」張無忌道︰「謝前輩打死神僧空見,是你親眼瞧見的麼?你是在一旁掠陣麼?是在旁相助麼?」宗維俠心想︰「這破衫小子居然跟我纏上了,多半是受了武當派的指使,要挑撥崆峒和少林兩派之間的不和。我倒要小心應付,不可入了人家圏套。」因此他雖没重視張無忌,還是正色答道︰「空見神僧喪身洛陽,其時崆峒五老都在雲南點蒼派柳大俠府上作客。我們怎能親眼見到當時情景?」

張無忌朗聲道︰「照啊!你當時既在雲南,怎能見到謝前輩害死空見大師?這位神僧是喪生於崆峒派七傷拳手下,人人皆知。謝前輩又不是你崆峒派的,你怎可嫁禍於人?」宗維俠道︰「{\upstsl{呸}}!{\upstsl{呸}}!空見神僧圓寂之處,牆上冩著『成崑殺僧空見於此牆下』十一個血字。謝遜冒著他師父之名,到處做下血案,那還有什麼可疑的?」張無忌心下一凜︰「我義父没説曾在牆上冩下這十一個字。他一十三拳打死神僧空見後,心中悲悔莫名,料來不會再冩此種示威嫁禍的字句。」當下仰天哈哈一笑,説道︰「這些字誰都會冩,謝前輩冩此十一個字,有誰見來?我偏要説這十一個字是崆峒派冩的。冩字容易,練七傷拳却難。」他轉頭向空智説道︰「空智大師,令師兄確是爲七傷拳拳力所害,是也不是?七傷拳是崆峒派不傳人的絶藝,是也不是?」

空智尚未回答,突然一名身披大紅金線袈裟而高大僧人閃身而出,手中金光閃閃的大禪杖在地下重重一頓,大聲喝道︰「小子,你是那家那派的門下?憑你也配跟我師父説話。」張無忌一看,這僧人肩頭拱起,説話中帶著三分氣喘,正是少林「十八羅漢」中的圓音,當年少林派上武當山興師問罪之師,便是他力證張翠山打死少林子弟,張無忌其時滿腔悲憤,將這一干人的形相牢記於心。此刻胸口熱血上衝,滿臉脹得通紅,身子也微微發抖,心中不住説道︰「無忌,無忌!今日的大事是要調解六大門派和明教的仇怨,千萬不可爲了一己私嫌,鬧得難以收拾。少林派的過節,日後再去算帳不遲。」雖然心中想得明白,但父母慘死的情狀,霎時間隨著圓音的出現而湧向眼前,不由得熱泪盈眶,幾乎難以自制。

圓音將禪杖重重在地下一頓,喝道︰「小子,你若是魔教的妖孼,快快引頸就戮,否則咱們出家人慈悲爲懷,也不來難爲於你,即速下山去吧!」他見張無忌的服飾打扮絶非明教中人,又誤以爲他竭力克制悲憤乃是心中害怕,是以有這幾句説話。

張無忌道︰「你便是圓音大師了?貴派有一位圓眞大師呢?請他出來,在下有幾句話請問?」圓音道︰「圓眞師兄不在此處,再有什麼事快説,咱們没空閒功夫跟你這野少年瞎耗。你到底是誰的門下?」他見張無忌適纔一掌將名列崆峒五老的宗維俠擊得連連倒退,料想他師父不是尋常人物,這纔一再盤問於他,否則此刻屠滅明教正當大功告成之際,那裡還耐煩跟這來歷不明的少年糾纏。

張無忌道︰「在下既非明教中人,亦非中原那一派門下,不過和明教以及武當、少林、峨、華山六大門派,都有一點干係。這次六大門派圍攻明教,實則是受了奸人的挑撥,中間存著極大的誤會。在下雖然年少,倒也得知其中的曲折原委,斗膽要請雙方罷鬥,査明眞情,誰是誰非,自可秉公判斷。」他語聲一停,六大派中登時爆發出哈哈、呵呵、嘩嘩、嘻嘻\dash{}各種各樣大笑之聲。數十人同聲指斥︰「這小子失心瘋啦,你聽他這麼胡説八道!」

\qyh{}他當自己是什麼人?是武當派張眞人麼?少林派空聞神僧麼?」

\qyh{}哈哈,哈哈!」

\qyh{}他發夢見到了屠龍寶刀,成爲武林至尊啦。」

\qyh{}他當咱們個個是三歳小孩児,呵呵,我肚子笑痛了!」

\qyh{}六大門派死傷了這許多人,魔教欠下了海樣深的血債,他想三言兩語,便將咱們打發回去\dash{}」

峨嵋派中,却只有周芷若眉頭緊蹙,黯然不語,她聽著各人的譏笑,心下暗暗難過。她自和張無忌在沙漠中一會,對少年便起了同情之心意,這時聽他這番不自量力的言語,不自禁的代他羞慚。

\chapter{居間調人}

六大派人衆的譏諷嘲笑,一句句的鑽進周芷若的耳中心中,她偸眼看張無忌時,却見他站在當場,昂然四顧,毫不爲意,只聽他朗聲道︰「只須少林派圓眞大師出來,跟在下對質幾句,他所安排下的奸謀便能大白於世。」這三句話一個字一個字的吐將出來,雖在數百人的鬧笑聲中,却是人人聽得清清楚楚。六大派衆高手心下都是一凜,登時便將對他輕視之心收起了幾分,均想︰「這小子年紀輕輕,内功怎麼如此了得?」

圓音待衆人笑聲停歇,氣喘喘吁吁的道︰「臭小子恁地奸猾,明知圓眞師兄不在此處,便要他出來對質。你何以不叫武當派的張翠山出來對質?」他最後一句話一出口,空智立時便喝︰「圓音,説話小心!」但華山、崑崙、崆峒諸派中已有許多人大聲笑了出來,只有武當派的人衆臉有慍色,默不作聲。原來圓音一隻右眼被殷素素在西子湖畔用暗器打瞎,始終以爲是張翠山下的毒手,一生耿耿於心,張無忌聽他辱及先父,怒不可遏,大聲喝道︰「張五俠的名諱,是你亂説得的麼?你\dash{}你\dash{}」圓音冷笑道︰「張翠山自甘下流,受魔教妖女迷惑,好色之報\dash{}」張無忌雖然心中一再説道︰「今日主旨是要使兩下言和鬥罷鬥,我萬萬不可出手傷人。」但一聽到這幾句話時,那裡還忍耐得住,一縱而前,左手一探,已抓住了圓音的後腰,提了起來,右手搶過他手中禪杖,橫過杖頭,便要往他頭頂擊了下去。圓音被他這麼一抓,有如雛雞落入鷹爪,竟無半分抵禦之力。少林僧隊中同時搶出兩人,兩根黃金禪杖分襲張無忌左右,那原是武學中救人的高明法門,所謂「圍魏救趙」,襲敵之所不得不救,便能解除陥入危境的夥伴。來救的兩僧正是圓心、圓業。張無忌左手抓著圓音,右手提著禪杖,一躍而起,雙足分點圓心圓業手中禪杖,只聽得嘿嘿兩聲,圓心和圓業同時仰天摔倒,幸好兩僧武功上均有不凡的造詣,臨危不亂,那兩條數十斤重的鍍金鑌鐵禪杖纔没反彈過來,打到自己身子。

衆人驚呼聲中,但見張無忌抓著圓音高大的身軀微一轉折,輕飄飄的落地,六大派中有七八個人叫了出來︰「武當派的『梯雲縱』!」原來張無忌自幼跟著父親及太師父、諸師伯叔,雖然没有正式按部就班的學練武當派武功,但見聞却多,這時練成「乾坤大挪移」神功,不論那一家那一派的武功,都能取爲我用。他對武當派的功夫耳濡目染,親炙最多,突然間不加思索的使用出來之時,自然而然的用上了這當世輕功中最著名的「梯雲縱」。兪蓮舟、莫聲谷等要像他這般縱起,再至空中輕輕迴旋數下,原亦不難,姿式之圓熟飄逸,尤有過之,但要一手抓著一個胖大和尚,一手提一根沉重的禪杖,仍要這般的身輕如燕,却是萬萬無法做到。

少林諸僧見圓音落入他的手中,但距已有七八丈遠,他只須禪杖一起,立時便能將圓音打得腦漿迸裂,要在這電光石火般的一瞬間衝上前去相救,決難辦到。唯一的法門是發射暗器,但張無忌只須舉起圓音的身子一擋,借刀殺人,反而害了他的性命。雖有空智、空性這等絶頂高手在側,但以變起倉卒,任誰也料不到這少年有如此身手,竟被張無忌攻了個措手不及,畢竟也是慢了一步。只見他咬牙切齒,滿臉仇恨復仇,舉起了禪杖,衆少林僧有的閉了眼睛,不忍再看,有的便待一擁而上,和圓音復仇。那知張無忌舉著禪杖的手,並不落下,似乎心中有什麼難以決定,但見他臉色漸轉慈和,慢慢的將圓音放下地來。

原來在這一瞬之間,張無忌已克制了胸中怒氣,心道︰「倘若我手下打死打傷了六大派中任誰一人,我便成爲六大派的敵人,再也不能成爲居間的調人。武林中這場兇殺,再也不能化解,那豈不是恰巧墮入成崑這奸賊的計中?不管他們如何辱我,我定當忍耐到底,這纔是眞正爲父母及義父復仇雪恨之道。」他想通了這節,便即放下圓音,緩緩的説道︰「你的眼睛不是張五俠打瞎的,不必如此記恨。何況張五俠自刎身死,什麼冤仇也該化解了。大師是出家人,四大皆空,何必對舊事如此念念不忘?」

圓音死裡逃生,呆呆的瞧著張無忌,見他將自己禪杖遞了過來,自然而然的伸手接過,低頭退開,心中是一股説不出的滋味。

崆峒派的宗維俠見張無忌露了一下身手,心下不禁暗暗驚異,但他既已身在場中,豈能就此示弱退下?大聲説道︰「姓曾的,你來強行出頭,到底是受了何人指使?」張無忌道︰「我只是秉公而言,盼望六大派和明教罷手言和,並無誰人指使在下。」宗維俠道︰「哼,要我跟魔教罷手言和,那是難上加難。這姓殷的老賊欠了我三記七傷拳,先讓我打了再説。」説著捋了捋衣袖。張無忌道︰「宗前輩開口七傷拳,閉口七傷拳,依晩輩之見,宗前輩的七傷拳還没練得到家。人身五行,心屬火、肺屬金、腎屬水、脾屬土、肝屬木,再加陰陽二氣,一練七傷,七者皆傷。這七傷拳的拳功每深一層,自身内臟便多受一層損害,實則是先傷己,再傷敵。幸好宗前輩拳功尚淺,尚有救藥。」宗維俠聽他這句話,的的確確是「七傷拳譜」的總綱,那拳譜中諄諄告誡,若非内功練到氣走諸穴、收發自如的境界,萬萬不可練這七傷拳。但宗維俠不自量力,一覺内功頗有成就,便即試練,一練之下,立時察覺到這路拳功威力無窮,既經陥溺,便難以自休,何況只見其利,未覺其害,早把拳譜總綱中的話抛諸腦後,這時張無忌説起,纔凜然一驚,説道︰「你怎麼又知道了?」

張無忌不答他的問話,却道︰「宗前輩試按你肩頭『雲門穴』,是否有輕微隱痛?雲門穴屬肺,那是肺脈傷了。你上臂『青靈穴』是否時時麻養難當?青靈穴屬心,那是心脈傷了。你腿上『五里穴』是否每逢陰雨,便即酸軟,五里穴屬肝,那是肝脈傷了。你越練下去,這種象徵越是厲害,再練得六七年便要全身癱瘓。」宗維俠凝神聽著他的説話,額頭上汗珠一滴滴的滲了出來。原來張無忌經謝遜傳授,已精通七傷拳的拳理,再加他深研醫術,明白損傷經脈後的徵狀,説將出來,竟是絲毫不錯。宗維俠這幾年早知自己身上有這些毛病,只是一來病況不重,二來心底暗自害怕,一味的諱疾忌醫,這時聽張無忌一一指出,不由得臉上變色,過了良久,纔道︰「你\dash{}你怎麼知道?」

張無忌淡淡一笑,説道︰「晩輩略明醫理,前輩若是信得過時,待此間事情一了,晩輩可設法給你驅除這些病症,只是七傷拳有害無益,不能再練。」宗維俠強道︰「七傷是我崆峒絶技,怎能説有害無益?當年我掌門師祖木靈子以七傷拳威震天下,名揚四海,壽至九十一歳,怎麼説會傷害自身?這不是胡説八道麼?」張無忌道︰「木靈子前輩想必内功深湛,自然能練,反而強壯臟腑。前輩一定要練,那自是由得你,其實依晩輩之見,你内功不到那個境界,練了也没用。」他所説的不無有理,但在各派高手之前,被這少年指摘本派的成名絶技無用,如何不惱?大聲喝道︰「憑你也配説我崆峒絶技有用無用。你説無用,那就試試。」

張無忌又是淡淡一笑,説道︰「七傷拳自是神妙精奥的絶技,我不是説七傷拳無用,而是説内功修爲倘若不到,那便練之無用。」周芷若躱在衆師妹身後,側身瞧著張無忌説話的神態,見他臉上尚帶少年人的稚氣,但勉強裝作見多識廣的老成模樣,這般侃侃而談,教訓崆峒五老之一的宗維俠,不免顯得有些可笑。

崆峒派中年輕性躁的子弟們見張無忌如此無禮,有的早已忍不住便要開口呼叱,但見宗維俠容色嚴肅,對這少年的言語凝神傾聽,絲毫不敢小視,又都把衝到口邉的叱罵聲縮了回去。只聽宗維俠道︰「依你説來,我的内功是還没到家了?」張無忌道︰「前輩的内功到家不到家,我也不知。不過前輩練這七傷拳時既然傷了身身,那麼不練也罷\dash{}」他剛説到這裡,忽聽得身後一人暴喝道︰「二哥跟這小子囉唆些什麼?他瞧不起咱們的七傷拳,便接我一招。」那人聲止拳到,呼呼風響,直擊張無忌後心。這一拳來得快捷異常,對準了無忌背上的靈台穴直擊而下。這靈台穴乃是人身的死穴,别説給凌厲無比的七傷拳擊中,便尋常一招,只要對準死穴,中招者非死也必重傷。

張無忌有心要以九陽神功懾服各派,明知身後有人來襲,却不轉身,對宗維俠道︰「宗前輩\dash{}」猛聽得鐵鍊嗆{\upstsl{啷}}聲響,搶出一人,嬌聲叱道︰「你暗施偸襲!」伸鍊往那人頭上套去,正是小昭。那人左手一翻,格開鐵鍊,砰的一拳,已結結實實的打在張無忌背上。那知這一拳雖然正中靈台穴,張無忌好似絶無知覺,既不搖晃,亦不使勁將偸襲之人震開,只是對小昭微笑道︰「小昭,不用擔心,這種七傷拳一點児也没有用處。」小昭吁了口氣,雪白的臉上轉爲暈紅,低聲道︰「我倒忘了你已練\dash{}」説到這裡,急忙住口,拖著鐵鍊退了開去。

張無忌轉過身來,只見突施偸襲之人是個大頭瘦身的老者,原來這人是崆峒五老中位居第四的常敬之。他一拳擊中無忌,見他渾如不覺,心下也自{\upstsl{嘀}}咕,衝口而出道︰「你\dash{}你練成『金剛不壞體』神功,那麼是少林派的了?」張無忌道︰「在下在少林派寺中學過一些功夫,不過不是少林派的弟子\dash{}」這常敬之知道凡是護身神功,全仗一口眞氣凝聚,一開口説話,眞氣即散,不等他住口,又一拳打了過去,砰的一聲,這一次是打在無忌的胸口。無忌笑道︰「少林派『金剛不壞體』神功練得深時,便在開口説話之時,也是諸邪不侵,你若不信,不妨再打一拳試試。」常敬之拳出如風,砰砰接連兩拳,這前後四拳,明明都打在對方身上,但張無忌笑嘻嘻的受了下來,竟似不関痛癢,四招開碑裂石的重手,在他便如清風拂體,柔絲撫身。

那常敬之的外號叫作「一拳斷嶽」,雖然誇大,但他拳力之強,那是老一輩的人一向知道的。這時衆人見他連擊張無忌四拳,全成了白費力氣,無不震驚。崑崙派和崆峒派素來不睦,這次雖然聯手圍攻明教,但雙方互有心病,崑崙派中有人冷冷的叫道︰「好一個『一拳斷嶽』啊!」又有人道︰「那麼四拳便斷什麼?」幸好常敬之一張臉膛本來黑黝黝地,雖然脹得滿臉通紅,倒也不太刺眼。少林派的諸人心中却各自懷疑︰「這人説曾來我寺中學過功夫,那是誰啊?『金剛不壞體』神功咱們決計不傳外人,何況除了昔年的空見大師,眼下本派無人具此功力。這少年這點點年紀,那能練成這門至少要有四十年火候的絶藝?」

宗維俠拱手道︰「曾兄神功,佩服佩服!能讓老朽領教三招麼?」他知道自己七傷拳的功力比常敬之深得多,老四不成,自己未必便損不了對方。

張無忌道︰「日後前輩眞正練成上乘的崆峒派絶技七傷拳,晩輩那便避之唯恐不及,眼下呢,那便勉力接你三拳,想也無妨。」言下之意是説,七傷拳本是好的,不過你還差得遠呢。宗維俠心下怒極,暗吸一口氣,跨上了一步,臂骨中格格作響,辟的一響,一拳打在張無忌胸口。拳面和他胸口相碰,只覺他身上似有一股極強的黏力,一時縮不回來,大驚之下,更覺有一股柔和的熱力從拳面一直傳到自己心脈,宗維俠運力一縮,但覺精神大{\upstsl{挀}},胸腹之間感到説不出的舒服。他呆了呆,又是一拳打去,這一次打中張無忌的小腹,只覺對方送回來的力道強極,他退了一步,這纔站定。

常敬之站在張無忌身側,見宗維俠臉上一陣紅一陣白,似已受了内傷,待他第三拳打出時,跟著也是一拳,變成了前後夾擊。宗維俠一拳打他胸前,常敬之一掌打他後背,雙拳夾攻,人人都可看出勁力凌厲非凡。那知兩人拳力到時,猶如打在空虛之處一般無二,兩道強勁的拳力被化解得無影無蹤。

常敬之明知以自己身份地位,第一次偸襲已是大爲不妥,但還可勉強説是爲了不忿對方出言侮辱崆峒絶技,怒氣無法抑制所致,這第二次偸襲却明明是下流卑鄙的行逕了。他本想合兩人七傷拳的威力,勢非一舉將他斃於拳下不可,只要將他打死,縱然旁人覺得不對,但他總是爲六大派除去一個礙手礙脚的麻煩,立下一場功勞。那知拳鋒甫著人體,勁力立時消於無形,當眞是令人大惑不解。

張無忌對宗維俠道︰「前輩覺得怎樣?」宗維俠怔了一怔,拱手道︰「多謝曾兄以内力替在下療養,神功驚人固不必説,而這番以德報怨的大仁大義,在下更是感激不盡。」

他此言一出,衆人都是大爲驚訝。原來張無忌在宗維俠連擊他三拳之際,運出九陽眞氣,送入他的體内,一瞬即過,但那九陽眞氣太過強勁,宗維俠已是受用不淺,他知道若非常敬之在張無忌身後偸襲,那麼第三拳上所受的好處將遠不止此。張無忌道︰「大仁大義四字,如何克當?宗前輩此刻奇經八脈都受劇震,最好是立即運氣調息,那麼練七傷拳時所積下來的毒害,當可在兩年内逐步除去。」宗維俠自己知道自己身上的毛病,拱手道︰「多謝,多謝。」張無忌俯下身來,接續唐文亮的斷骨,對常敬之道︰「拿些回陽五龍膏給我?」常敬之從身邉取了出來給他,張無忌道︰「你去向武當派討一服三黃寶臘丸,向華山派討一些玉眞散來。」常敬之依言討到,遞了給他,張無忌道︰「貴派的回陽五龍膏中,所用草烏是極好的;武當的三黃寶臘丸中,天竺黃雄黃籐黃三黃甚是有用,再加上玉眞散,唐前輩調養兩個月後,四肢當能完好如初。」説著續骨敷藥,片刻間整治完畢要知武林各派,均有傷科祕藥,各有各的靈效,這次圍攻明教,自是各有擕帶在身。

旁觀的人愈看愈奇,張無忌接骨手法之妙,非任何名醫可及,那是不必説了,何以各派擕有何種藥物,他也是一清二楚?這麼一來,崆峒派在勢已不能再跟他動手。常敬之抱起唐文亮,神色{\upstsl{尷}}尬的退了下去。唐文亮突然叫道︰「姓曾的,你治好我的斷骨,唐文亮十分感激,日後自當補報。可是崆峒派和魔教仇深似海,豈能憑你這一點小恩小惠,便此罷手?你要勸架,咱們是不聽的,你若説我忘恩負義,儘可將我四肢再折斷了。」衆人一聽,心道︰「同時崆峒耆宿,這唐文亮却比常之有骨氣得多。」張無忌道︰「依唐前輩説來,如何纔肯聽在下的解架?」唐文亮道︰「你露一手武功,倘若崆峒派及你不上,那纔無話可説。」

張無忌笑道︰「崆峒派高手如雲,要及晩輩不上,那是談何容易。不過晩輩這和事老是做定了,祇好捨命一試。」四下一望,見到廣場東首有一株高達三丈有餘的大松樹,枝椏四出,亭亭如蓋,便緩步走了過去,朗聲道︰「晩輩學過貴派的一些七傷拳法,若是練得不對,請崆峒派各位前輩莫見笑。」各派人衆聽了,更是驚訝︰「這小子原來連崆峒派的七傷拳也會,那是從何處學來啊?」祇聽他朗聲唸道︰「五行之氣調陰陽,損心傷肺摧肝腸,離藏精英恍惚,三焦齊逆兮魂魄飛揚!」

别派各人聽到,那也罷了,崆峒五老聽到他高吟這四句似歌非歌、似詩非詩的拳訣,無不凜然心驚。要知這正是七傷拳的總訣,那是崆峒派的不傳之祕,這少年如何知道?他們一時之間,那裡想得到乃是謝遜將七傷拳譜搶去後,轉而傳給他的。祇見張無忌高聲吟罷,走上前去,砰的一拳擊出,突然間眼前青翠晃動,那棵大松樹的上半截平平飛出,轟隆一響,摔在兩丈之外,地下祇留了四尺來長的半截樹幹,切斷之處,甚是平整。常敬之喃喃的道︰「這\dash{}這不是七傷拳啊!」要知七傷拳講究剛中有柔、柔中有剛,這種震斷大樹的拳法雖然威力驚人,但顯是純剛之力。但他走近一看,不由得張大了口,合不攏來,但見樹幹斷處脈盡皆震碎,却正是七傷拳練到最深時的功夫。原來張無忌存心威壓當場,倘若用七傷拳震碎樹脈,須至十天半月之後,松樹枯萎,纔顯功力,是以七傷拳力震樹之後,跟著以陽剛猛勁,斷樹飛枝。

祇聽得喝采驚呼之聲,各派中彼起此伏,良久不絶,唐文亮道︰「好!這是絶高的七傷拳法,唐文亮拜服!不過我要請教,曾小俠這路拳法從何處學來?」張無忌微笑不答,唐文亮厲聲道︰「金毛獅王謝遜現在何處?還請曾小俠告知。」張無忌心中一驚︰「啊喲不好,我炫示七傷拳功,却把義父帶了出來。我若是明言我跟義父之間的干係,擺明和六大派爲敵,這和事佬又作不成來。」當即朗聲道︰「你道貴派的七傷拳譜,是金毛獅王奪去的嗎?哈哈,錯了,錯了!奪譜之人,乃是當年的混元霹靂手成崑。那一晩崆峒山青陽觀中奪譜激鬥,有兩位中了混元功之傷。在下説的可不錯了?」

原來謝遜赴崆峒劫奪拳譜,成崑爲了存心擾亂武功,暗中相助,以混元功擊傷唐文亮、常敬之二老,當時謝遜不知,後來經神僧空見點破,這纔明白(請參閲全書)。這時張無忌心想成崑一生奸詐,禍嫁於人,我不免以其人之道,還治其人之身,何況這又不是説的假話。唐文亮和常敬之疑心了二十餘年,這時經張無忌一提,不由得對望了一眼,一時説不出話來。宗維俠道︰「那麼請問曾小俠,這成崑現下到了何處?」

張無忌道︰「混元霹靂手成崑一心挑撥六大派和明教不和,後來投入少林門下,法名圓眞。在下曾在少林之中跟他學過武功,此事千眞萬確,若有虛言,我是豬狗不如之輩,死後萬劫不得超生。」他這幾句話朗朗説來,衆人盡皆動容,祇有少林派僧衆一齊大嘩。須知圓眞是空見的入室弟子,佛學深湛,除了這次隨衆遠征明教之外,從來不出寺門一步,如何能是混元霹靂手成崑?

祇聽一人高喧佛號,緩步而出,身披灰色僧袍,貌相極是威嚴,左纔握了一串念珠,正是少林三大神僧之一的空性。他步入廣場,説道︰「曾施主,你如何胡言亂語,一再誣衊我少林門下?你幾時入過我少林寺學藝?當此天下英雄之前,少林清名豈能容你隨口汚辱?」張無忌躬身道︰「大師不必動怒,請圓眞出來跟晩輩對質,便知眞相。」

空性大師沉著臉道︰「曾施主一再提及敝師侄圓眞之名,你年紀輕輕,何以存心如此險惡?」張無忌道︰「在下是要請圓眞和尚出來,在天下英雄之前分辯是非黑白,怎地存心險惡了?」空性道︰「圓眞師侄爲我少林一派,苦戰妖孼,力盡圓寂,他死後清名,豈容你\dash{}」張無忌聽到「力盡圓寂」四字時,耳朶中{\upstsl{嗡}}的一聲響,臉色登時慘白,空性以後説什麼話,一句也没有聽見,喃喃的道︰「他\dash{}他當眞死了麼?決\dash{}決計不會。」空性指著西首的一堆僧侶屍首,大聲道︰「你自己瞧去吧!」

張無忌走到這一堆屍首,祇見屍體中有一具臉頰凹陥,雙目向上翻挺,果然便是投入少林後化名圓眞的混元霹靂手成崑。他俯身一探那屍首的鼻息,觸手之處,祇覺臉上肌肉冰涼,已然死去多時。張無忌又悲又喜,想不到害了義父一世的大仇人,終於惡貫盈滿,喪生於此,雖然不是死於自己手下,但義父的大仇,却是報了,胸中熱血上湧,仰天哈哈大笑,叫道︰「奸賊啊奸賊,你一生作惡多端,原來也有今日。」這幾下大笑聲震山谷,遠遠傳送出去,人人都是心頭一凜。

張無忌回過頭來,問道︰「這圓眞是誰打死的?」空性側目斜睨,臉上猶似罩著一層寒霜,並不答話。殷天正本已退在一旁,這時説道︰「他和小児野王比掌,結果一死一傷。」張無忌躬身道︰「是!」心道︰「想是圓眞中了青翼蝠王韋一笑的寒冰綿掌後,受傷大是不輕,我舅父的掌力也非同小可,這纔當場將他擊斃。舅父替我報了這場深仇,那眞是再好不過。」走到殷野王身旁,一搭他的脈息,知道性命無礙,便放寬了心,説道︰「多謝前輩!」空性在一旁瞧著,愈來愈怒,縱聲喝道︰「小子,走過來納命吧!」這幾個字轟轟入耳,聲若雷震。張無忌愕然回頭,道︰「怎麼?」

空性道︰「你明知圓眞師侄已死,却將一切罪過都推在他身上,如此惡毒,豈能饒你?老和尚今日要開殺戒。你是自裁呢,還是非要老和尚動手不可?」無忌心下躊躇︰「圓眞伏誅,罪魁禍首遭了應得之報,原是極大的喜事,可是從此無人對質,事情眞相反而不易明白,那便如何是好?」正自沉吟,空性踏上一步,右手便向他頭頂抓了下來。祇見他自腕至指,伸得筆直,勁道極是凌厲,殷天正喝道︰「是龍爪手,不可大意!」

張無忌身形一側,也不知他用何身法,輕飄飄的讓了開去。可是空性大師乃少林三大神僧之一,這「龍爪手」又是少林絶藝中的上乘功夫,他一抓不中,第二抓跟著發出,這一抓去勢更加迅捷剛猛。張無忌一側身,又向左側閃避,那知空性第三抓、第四抓、第五抓呼呼發出,瞬息之間,這個灰袍僧人便變成一條蒼龍一般,龍影飛空,龍爪舞動,將張無忌壓制得無處躱閃。猛聽得嗤的一聲響,張無忌平平飛出,右手衣袖已被空性抓在手中,光光的一條右臂上長長五條血痕,鮮血淋漓而下。少林僧衆喝采聲中,却夾雜著一個少女的驚呼。張無忌向聲音來處瞧去,祇見小昭一臉驚恐的神色,叫道︰「張公子,你\dash{}你要小心了。」

張無忌心中一動︰「這小姑娘對我倒眞的很好。」過來適纔空性龍爪手使動,威勢非凡,無忌從未見過,竭力閃過,不料對方越攻越快,不由得慌了手脚,空性的第三十三招從左下角斜翻上,跟著又從右上角斜翻而下,兩招混一,變幻難測,以致手臂被抓。空性的五根手指勝於五柄利錐,這一抓雖没傷到筋骨,但也已深入肌膚,大是疼痛。只見空性一招得手,縱身而起,又撲了過來。張無忌一時没想到抵禦之策,祇得倒退躍開,空性這一抓便即落空。

空性一抓不中,第二抓跟發出,張無忌又即縱身後退,兩人面對著面,一個撲擊,一個後躍,空性連抓了八九抓,盡皆落空。兩人始終相距兩尺有餘,雖然空性連連速攻,張無忌未有還手餘地,但兩人輕功上的造詣,却極明顯的分了高下。要知空性飛步上前,張無却倒退後躍,其間之難易,相去實不可以道里計,但空性始終趕他不上,脚下早已輸得一敗塗地。無忌祇須轉過身來,向前奔出數丈,立時便將空性遙遙的抛在後面了。

其實張無忌不須轉身,縱然倒退,也能擺脱對方的攻擊,但他所以一直和空性大師不接不離,始終相距在二三尺間,乃在察看他龍爪手招數中的祕奥,看到第三十七招時,祇見空性左手疾撲而前,用的又是第八招的「拏拿式」。他三十八招雙手自上而下同抓,方位雖變,姿式却和第十二招「搶珠式」相同。原來那龍爪手祇有三十六招,其要旨在於凌厲兇煞,不求變化繁多。空性生平極少和人動手,中年之時,雖逢大敵,但祇要使出這龍爪手來,無不立佔上風,總是在十二招以前,便即取勝,自第十三招起,祇是自己平時練習,從未在臨敵時用過,這一次直使到第三十六招,仍未能將敵人折服,那是生平從未有之事,到第三十七招時,迫得變化前招,力求克敵,心中思量︰「這小子祇不過仗著輕功高明,身形靈便,一味東躱西閃而已,倘若眞和我動手拆招,未必擋得了我十二招龍爪手。」

張無忌這時却已看通了龍爪手所以如此厲害的関鍵所在,這一路三十六式的抓法,本身原無破綻可尋,但他此時精通乾坤大挪移法之後,仗著本身神功,能在對方任何神拳招中造成破綻,但心中躊躇︰「此刻我便要取他性命,亦已不難,但少林派威名赫赫,這位空性大師又是少林寺中碩果僅存的三大耆宿之一,我若在天下英雄之前將他挫敗,少林派顏面何存?可是要想不動聲色的叫他知難而退,這人武功比崆峒諸老高明得太多,我却又難以辦到。」正感爲難之際,忽聽空性喝道︰「小子,你這是逃命,可不是比武!」

張無忌道︰「要比武\dash{}」空性乘他開口説話而眞氣不純之際,呼呼兩招攻出,不料張無忌縱身飄開,口中言語繼續接了下去︰「\dash{}也成,要是我贏得大師,那便如何?」這幾句話中間語氣没半分停頓,若是閉著眼聽他,便跟心平氣和的坐著説話一般無異,決不信他在説這三句話之間,已連續閃避了空性的五招快速進攻。空性道︰「你輕功固是極佳,但要在拳脚上贏得我,却也休想。」張無忌道︰「過招比武,誰又能逆料生死勝敗?晩輩比大師年輕得多,武藝雖低,氣力上可佔了便宜。」空性厲聲道︰「若是我在拳脚之上輸了給你,你要殺便殺,要剮便剮。」張無忌道︰「這個可不敢當!晩輩輸了,聽憑大師如何處分,不敢有半句異言,但若僥倖勝得一招半式,便請少林派退下光明頂。」空性道︰「少林派之事,由我師兄作主,我只管得自己。我不信這龍爪手拾奪不了你這小子。」張無忌心念一動,已有主意,説道︰「少林派龍爪手三十六招没半分破綻,乃是天下擒拿法中的無上絶藝,只不過大師練得不大對而已。」空性怒道︰「好吧!你要是破解得了我的龍爪手,我立即回去少林寺,終身不出寺門一步!」張無忌道︰「那也不必如此!」

兩人這一番對答之際,四周衆人采聲如雷,越來越是響亮。原來兩人一面對答,手脚身法却絲毫不停,只有愈鬥愈快,但説話的語調却和平時一模一樣,絶無半點停頓氣促。當空性説「你輕功固是絶佳」這句話時,呼呼連出兩招。

\chapter{以德服人}

當空性説︰「但要在拳脚上贏得我」那句話時,緩緩遞出一招,説到「却也休想」時,語音威猛,雙手顫動,疾拿三招。兩人邉鬥邉説,旁觀衆人的喝采聲却是始終掩蓋不了他二人的語音,待得張無忌説到「那也不必如此」時,陡然間身形拔起,在空中盤旋著連轉四個圏子,愈到愈高,又是一個轉折,直落在數丈之外,衆人只瞧得神眩目馳,那是生平從所未見的絶世輕功,若非今日親眼目睹,決不信世間能有人練到這個地步。青翼蝠王韋一笑自負輕功之佳,舉世無人能及,這時一見,也不禁駭然嘆服。

張無忌身子落地,空性也已搶到他的身前,却不乘虛進擊,大聲道︰「咱們這就比了嗎?」張無忌道︰「好,大師請發招。」空性道︰「你還是不住倒退麼?」張無忌微笑道︰「晩輩若再倒退半步,便算是輸了。」明教中楊逍、冷謙、周顚、説不得諸人身子不能動彈,眼睛耳朶却一無阻礙,聽得張無忌如此説法,都是暗吃一驚。他們個個見多識廣,眼見空性僧的龍爪手神威無儔,便是接他一招,也極不易,張無忌武功雖然了得,但就算能勝,總也在百餘招之後,攻守趨避,如何能不退半步?這句話説得未免過於托大。只聽空性道︰「那也不必如此,贏要贏得公平,輸要也輸得心服。」一言甫畢,喝道︰「接招!」左手虛探,右手挾著一股勁風,直拿張無忌左肩「缺盆穴」,正是一招「拏雲式」。

張無忌見他左手微動,便已知他要使此招,當下也是左手虛探,右手直拿對方「缺盆穴」。兩人所招式一模一樣,竟無半點分别,但張無忌後發先至,却在一刹那的相差之間佔了先著,空性的手指離他肩頭尚有兩寸,他的五根手指已抓到了空性的「缺盆穴」上。空性只覺穴道上一麻,右手力道全失。張無忌手指却不使勁,隨即縮回。

空性呆了一呆,雙手齊出,使一招「搶珠式」,拿向張無忌的左右太陽穴。張無忌仍是後發先至,那乾坤大挪移的手法實是神妙無方,隨意所之,他的兩手探出,又是搶先一步,拿到了空性的雙太陽穴,這太陽穴何等重要,在内家高手比武之際,觸手立斃,無挽救的餘地。張無忌的手指在他雙太陽穴上輕輕一拂,便即圏轉,變爲龍爪手中的第十七招「撈月式」,虛拿空性後腦的「風府穴」。空性被他拂中雙太陽穴時已是一呆,待見他使出「撈月式」,更是驚訝之極,立即向後躍開半丈,喝道︰「你\dash{}你怎地偸學到我少林派的龍爪手?」張無忌微笑道︰「天下武學殊途同歸,強分派别,乃是人爲,這龍爪手也未必是貴派獨有。」

空性低頭沉思,一時想不通其中的道理,説到這龍爪手上的造詣,便是師兄空聞、空智兩人,也是及自己不上,何以這少年接連兩招,都能後發先至,而且出招的手法勁力方向部位,更是穩迅兼備,便如有數十年苦練之功一般?他呆呆不語,廣場上數百人的眼光,一齊凝注在他臉上。適纔兩人動手過招,倏忽兩下,便即分開,除了第一流高手之外,餘人都没瞧出誰勝誰敗,只是眼見張無忌行若無事,空性却皺起眉頭苦苦思索,顯然優劣已判。

要知龍爪手經林派數百年來千鍊百錘,已成爲不敗的武功,若非張無忌也以龍爪手與之對攻,要在拳脚上取勝,確是不易。空性突然間大喝一聲,縱身過來,雙手猶如狂風驟雨,「捕風式」「捉影式」「撫琴式」「鼓瑟式」「批亢式」「擣虛式」「抱殘式」「守缺式」、八式連環,疾攻而至。張無忌神定氣閒,依式而爲,捕風捉影、撫琴瑟鼓、批亢擣虛、抱殘守缺,接連八招,招招後發先至。

空性神僧這八式連環的龍爪手,綿綿不絶,便如是一招中的八個變化一般,快捷無比,那知他快張無忌更快,每一招都佔了先手。空性每出一招,便逼得向後倒退一步,退到第七步時,「抱殘式」和「守缺式」穩凝如山般使將出來。這兩式正是龍爪手中最後的第三十五、三十六式的招數,一瞥之下,似乎其中破綻百出,施招者手忙脚亂,竭力招架,其實這兩招似守實攻,大巧若拙,每一處破綻中都隱伏著厲害無比的陥阱。龍爪手原來走的是剛猛路子,但到了最後兩式時,剛猛中暗藏陰柔,已到了返璞還眞、爐火純青的境界。張無忌一聲清嘯,踏步而上,抱殘守缺兩招虛式一帶,突然化作一招「拏雲式」,中宮直攻而入。

空性大喜,暗想︰「終於你著了我道児。」眼見他一條右臂已陥入重圍,再也不能全身而退,當下雙掌迴擊,陡然圏轉,呼的一響,往他臂彎上擊了下去。原來空性是有道高僧,見張無忌精通少林絶藝,生怕他和本門確有淵源,何況先前數招中他明明已抓到自己重穴,臨時有意相讓,因此這一招也没便下殺手,只求他右臂震斷便算。豈知雙掌掌緣剛和他右臂相觸,突覺一股柔和而厚重的勁力從他臂上發出,擋住了自己雙掌下擊,便在此時,張無忌的右手五指也已虛按在空性胸口「膻中穴」的周遭。

在這一瞬之間,空性心中登時萬念倶灰,只覺數十年來苦練武功、闖盪江湖,全成一場幻夢,點了點頭,緩緩説道︰「曾施主比老衲高明得多了。」左手抓住右手的五個手指,一施勁力,正要將之折斷,突覺左腕上一麻,勁道全然使不出來,正是張無忌的手指在他手腕穴道上輕輕拂過。只聽他朗聲道︰「晩輩以少林派的龍爪手勝了大師,於少林威名有何妨礙?晩輩若非以少林絶藝和大師對攻,天下再無第二門武功,能佔得大師半點上風。」

空性在一時憤激之中,原想自斷五指,終身不言武功,聽無忌如此説,但覺對方言語行事,處處對本門十分迴護,若非如此,少林派千百年來的威名,可説在自己手中損折殆盡,自己豈非成了少林一派的大罪人?言念及此,不由得對張無忌大是感激,眼中泪光瑩瑩,合什説道︰「曾施主仁義過人,老衲既感且佩。」張無忌深深一揖,説道︰「還須請大師恕晩輩犯上不敬之罪。」空性微微一笑,説道︰「這龍爪手到了曾施主手中,想不到竟有如此威力,老衲以前做夢也料想不到,日後有暇,還望駕臨敝寺,老衲要一盡地主之誼,多多請教。」本來武林中人説到「請教」兩字,往往含有挑戰之義,但空性説得十分誠懇,確是佩服對方武術,自愧不如。張無忌忙道︰「不敢,不敢。」

空性在少林派中身份極是崇高,雖然他因缺乏領袖和辦事的才幹,在寺中不任重要職司,但人品武功,素爲僧衆推服。少林派中自空智以下見他如此,都覺今日之事,本門是不便再出手向張無忌索戰的了。空智大師是這次六大派圍攻明教的首領,眼見情勢如此,心中十分{\upstsl{尷}}尬,魔教覆滅在即,却給這一個無名少年插手阻撓,倘若便此收手,豈不被天下豪傑笑掉了牙齒?一時拿示定主意,斜眼向華山派的掌門人神算子鮮于通使了個眼色。這鮮于通足智多謀,乃是這次圍攻明教的軍師,見空智大師使眼色向自己求教,當即摺扇一晃,緩步而出。

張無忌見來者是個四十餘歳的中年文士,眉目清秀,甚是俊雅瀟灑,心中先存了三分好感,拱手道︰「請了,不知這位前輩有何見教。」鮮于通尚未回答,殷天正道︰「這是華山派掌門鮮于通,武功平常,鬼計多端。」

張無忌一聽到鮮于通之名,暗想︰「這名字好熟,什麼時候聽見過啊?」只見鮮于通走到他身前一丈開外,立定脚步,拱手説道︰「曾少俠請了!」張無忌還禮道︰「鮮于掌門請了。」鮮于通道︰「曾少俠神功蓋世,連敗崆峒諸老,甚至少林神僧亦是甘拜下風,在下佩服之至。實不知是那一位前輩高人門下,調教出這等近世罕見的少年英俠出來?」張無忌一直在思索什麼時候聽人説起過鮮于通,對他的問話没有置答。鮮于通仰天哈哈一笑,朗聲説道︰「不知曾小俠何以對自己的師承來歷,也有這等難言之隱?古人言道︰『見賢思齊,見不賢\dash{}』」

張無忌聽到「見賢思齊」四字,猛地裡想起「見死不救」來,登時想起,五年前在蝴蝶谷中之時,胡青牛曾親口對他言道︰華山派有一個人名叫鮮于通,害死了他的妹子。當時張無忌小小的心靈之中曾想︰「這鮮于通如此可惡,日後倘若不遭報應,老天爺那裡還算有眼?」一凝神之際,將胡青牛的説話清清楚楚的記了起來︰「一個身上受了一十七處刀傷、非死不可的少年,我三日三夜不睡,耗盡心血救治了他,和他義結金蘭,情同手足,那知後來他却害死了我的親妹子。唉,我苦命的妹子,我兄妹倆自幼父母見背,相依爲命。」胡青牛説這番話時,那滿臉皺紋,泪光瑩瑩的哀傷情狀,曾令張無忌心中大是難過。後來胡青牛的妻子「毒仙」王難姑在鮮于通身上下了劇毒,胡青牛忍不住又給他治好,累得他夫妻反目,吃盡了無窮的苦楚,最後兩人死於非命,非始不是因此而起。

他想到此處,雙眉一挺,兩眼神光炯炯,向鮮于通直射過去,又想起鮮于通曾有個弟子薛公遠,被金花婆婆打傷後自己救了他的性命,那知後來反而要將自己煮來吃了,這兩師徒恩將仇報,均卑鄙無恥的奸惡之徒,薛公遠已死,眼前這鮮于通却非好好懲戒一番不可,當下微微一笑,説道︰「我身上又没受過一十七處刀傷,又没害死過我金蘭之交的妹子,那有什麼難言之隱?」

鮮于通聽了這句話,不由得全身一顫,背上冷汗直流。原來當年他得胡青牛救治性命後,和胡青牛之妹胡青羊相戀。胡青牛以身相許,竟致懷孕,那知鮮于通貪圖華山派掌門之位,棄了胡青羊不理,和當時華山派掌門的獨生愛女成親。胡青羊羞憤自盡,造成一屍兩命的慘事。這件事是胡家的家門之醜,胡青牛自然是不會跟人説起,鮮于通那是更加不會洩漏半句,不料事隔十餘年,突然被這少年當衆揭了出來,如何不令他驚惶失措,臉如土色?可是鮮于通是個極工心計之人,心念一動,已起毒念︰「這少年不知如何,竟知我的陰私,非下辣手除了他不可,否則給他説穿我的舊事,這一生就此身敗名裂了。」霎時間鎭定如恆,説道︰「曾少俠既不肯將師承見告,在下便以華山派的微末武藝。領教曾少俠的高招。想空性神僧尚非曾少俠的敵手,在下這點粗淺功夫,如何能在曾少俠眼中?咱們點到即止,還盼曾少俠手下留情。」説著右掌斜立,左掌便向張無忌肩頭劈了下來,朗聲道︰「曾少俠請!」竟是不讓張無忌再有説話的機會。

張無忌知他心意,隨手舉掌輕輕一格,説道︰「華山派的武藝高明得很,領不領教,都是一般。倒是鮮于掌門恩將仇報、忘恩負義的功夫,却是人所不及\dash{}」鮮于通不讓他説下去,施展生平本事,貼身疾攻,用的正是華山派絶技之一的七十二路「鷹蛇生死搏」。他將摺扇收攏,握在右掌之中,露出小半截尖利的扇柄,作蛇頭之形,左手五指使的則是鷹爪功路子;右手蛇頭點打刺戮,左手則是擒拿扭勾,雙手招數截然不同。

鮮于通所使這路「鷹蛇生死搏」,乃是華山派已傳百餘年的絶技,當年華山派大俠雲伯天,在伏牛山見到一場蒼鷹和毒蛇的生死搏鬥,因而有悟,創設這套武功。鷹蛇搏鬥並非奇事,歷來武學名家由此得到啓發的也在所多有,但華山派這套武功與衆不同之處,在於鷹式和蛇式同時施展,迅捷狠辣,兼而有之。可是力分則弱,這路武功用以對付常人,原能使人左支右絀,顧得東來顧不得西,張無忌只接數招,却已知對方招數雖精、力道不足,當下隨手拆接,説道︰「鮮于掌門,在下有一件不明之事請教?你當年身受一十七處刀傷,已是九死一生,人家拚著三日三夜不睡,竭盡心力的給你治好了,又和你義結金蘭、待你情若兄弟。爲什麼你這樣狠心,反而去害死了他的妹子?」

鮮于通無言可答,張口罵道︰「胡\dash{}」他本想罵「胡説八道」,跟對方來個強辯,須知鮮于通言辭便給,口齒伶俐,耳聽得張無忌在揭自己的瘡疤,便想捏造一番言語出來,不但遮掩自己的過錯,反而誣陥對方,待張無忌憤怒分神,便可乘機暗下毒手。那知剛説了一個「胡」字,突然間一股柔和而渾厚的掌力壓了過來,逼住他的胸口,鮮于通喉頭氣息一沉,下面那「\dash{}説八道」這三個字便咽回了肚中,一霎時之間,只覺肺中的氣息就要被對方掌力擠逼出來,急忙潛運内力,苦苦撐持,耳中却清清楚楚的聽得張無忌説道︰「不錯,不錯!你倒記得是姓『胡』的,爲什麼説了一個『胡』字,便不往下説呢?胡家小姐被你害得好慘,這些年來,你難道心中也不覺得慚愧麼?」

鮮于通正感呼吸便要斷絶,急急連攻三招,張無忌掌力一鬆,鮮于通只感胸口一輕,忙吸了一口長氣,喝道︰「你\dash{}」但只説了個「你」字,對方的壓力又逼到胸前,話聲立斷。張無忌道︰「大丈夫一身做事一身當,是即是,非即非,爲什麼支支吾吾、吞吞吐吐?蝶谷醫仙胡青牛先生當年救了你的性命,是不是?他的親妹妹是被你親手害死的,是不是?」張無忌並不知胡青牛之妹子如何被害,無法説得更加明白,但鮮于通却以爲自己一切醜史,對方全都了然於胸,又苦於言語無法出口,臉色更加白了。

旁觀衆人素知鮮于通口若懸河,最擅雄辯,此刻見他臉有愧色,聽到對方的嚴詞詰責竟是無話以對,對張無忌的説話不由得不信。原來張無忌以絶頂神功壓迫他的呼吸,除了鮮于通自己啞子吃黃蓮,有苦説不出之外,旁人但見張無忌雙掌揮舞,拆解鮮于通的攻勢,偶爾則反擊數掌,縱是各派一流高手,也瞧不破其中的祕奥。華山派中的諸名宿門人,眼見掌門人如此當衆出醜,被一個少年罵得狗血淋頭,却無一句辯解,人人均感羞愧無地。另有一干人知道鮮于通詭計多端,却以爲他暫且隱忍,暗中必有極厲害的報復之計。

只聽張無忌又嚴辭斥道︰「咱們武林中人,講究有恩報恩,有怨報怨,那蝶谷醫仙是明教中人,你身受明教的大恩,今日反而率領門人,前來攻擊明教。人家救你性命,你反而害死他的親人,如此禽獸不如之人,虧你也有臉來做一派的掌門!」他罵得痛快淋漓,心想胡先生今日若是在此,親耳聽到我如此爲他伸怨雪恨,當可一吐心中的積憤,眼下罵也罵得彀了,今日不傷他的性命,日後再我他算帳,當雙掌力一收,説道︰「你既自知羞愧,今日暫且寄下你頸上的人頭。」鮮于通突然間呼吸暢爽,喝道︰「小賊,一派胡言!」摺扇柄向著張無忌面門一點,向旁躍開。張無忌鼻中突然聞到一陣甜香,頭腦昏眩,脚下幾個踉蹌,但覺天旋地轉,眼前金星亂舞。

只聽鮮于通喝道︰「小賊,教你知道華山絶藝『鷹蛇生死搏』的厲害?」説著縱身上前,左手五指向張無忌右腋下的「淵腋穴」上了下去。他知道這一把抓下,張無忌絶無反抗之能,那知著手之處,便如抓到了一張滑溜溜的大魚皮,竟是便不出半點勁道,但聽得華山派門人弟子的采聲雷動︰「鷹蛇生死搏今日名揚天下!」

\qyh{}華山鮮于掌門神技驚人!」

\qyh{}教你這小賊見識見識貨眞價實的武功!」張無忌微微一笑,一口氣向鮮于通口鼻間吹了過去。鮮于通陡然聞到一股甜香,頭腦立時昏暈,這一下當眞是嚇得魂飛魄散,張口待欲呼喚,張無忌左手衣袖在他雙脚膝彎中一拂,鮮于通立足不定,撲地跪倒,伏在張無忌的面前,便似磕拜求饒一般。

這一個變故人人大出意料之外,明明張無忌已然身受重傷,搖搖欲倒,那知一刹那間,變成鮮于通跪在他的面前,難道他當眞是有妖法不成?只見他俯下身去,從鮮于通手中取過摺扇,哈哈長笑,朗聲説道︰「華山派自負名門正派,眞料不到還有一手放蠱下毒的絶藝,各位請看!」説著輕輕一揮,打開摺扇,只見扇上一面繪的是華山最高峰,千仞疊秀,有如削成,另一面冩著六句郭璞的「太華讚」︰「華山岳靈峻,削成四方。爰有神女,是挹玉漿。其誰遊之?龍駕雲裳。」圖文古雅,洵屬妙品。張無忌摺攏扇子,説道︰「誰知道在這把風雅的扇子之中,竟藏著一個卑鄙陰毒的機関。」一面説,一面走到一棵花樹之前,以扇柄對住花樹一指,片刻之間,花瓣紛紛萎謝,樹葉也變爲黃色。衆人看得清楚,無不駭然,均想︰「鮮于通在這把扇子中藏的不知是什麼毒藥,竟有這等厲害?」

只聽得鮮于通伏在地下,猶如殺豬般的慘叫,聲音淒厲,撼人心弦,「啊\dash{}啊\dash{}」的一聲聲長呼,猶如有人以利刃在一刀刀刺他的肌膚。本來以他這等武學高強之士,便是眞有利刃加身,也能強忍痛楚,決不致在衆人之前,如此大失身份的呼痛。他每呼一聲,便是削了華山派衆人的一層面皮。只聽他呼叫幾聲,大聲道︰「快\dash{}快殺了我\dash{}快打死我吧\dash{}」張無忌道︰「我倒有法子治你的痛楚,只不知你扇中所藏,是何毒物。不明毒源,難以解救。」鮮于通道︰「這\dash{}這是金蠶\dash{}金蠶蠱毒\dash{}快\dash{}快打死我\dash{}啊\dash{}啊\dash{}」

衆人聽到「金蠶蠱毒」四字,年輕的不知厲害,倒也罷了,各派耆宿却無不變色,有些正直的有德之士,已大聲的斥責起來。原來這「金蠶蠱毒」出於貴州苗疆,乃天下毒物之最,無形無色,中毒者有如千萬條蠶蟲同時在周身咬嚙,痛楚難當,無可形容。武林中人説及時無不切齒痛恨,須知這種蠱毒無跡可尋,憑你是神功無敵,也能被一個半點不會武功的婦女児童下了毒手,只是其物難得,各人均是只聞它的毒名,今日纔親眼見到鮮于通身受其毒的慘狀。張無忌又問︰「你將金蠶蠱毒藏在摺扇之中,怎麼會害到了自己?」鮮于通道︰「快\dash{}殺了我\dash{}我不知道,我不知道\dash{}」説到這裡,伸手在自己身上亂抓亂擊,滿地翻滾。張無忌道︰「你將扇中的金蠶蠱毒放出害我,却被我用内力逼出回來,你還有什麼話説。」鮮于通尖聲大叫︰「是我自己作孼\dash{}我自作孼\dash{}」伸出雙手扼在自己咽喉之中,想要自盡,但中了這金蠶蠱毒之後,全身已無半點力氣,拚命將額頭在地下碰撞,也是連面皮也撞不破半點。這毒物的厲害之處,就在這裡,叫中毒者眞的是求生不能,求死不得,偏偏又神智清楚,身上每一處的痛楚,加倍敏鋭的感到,因此比之中者立斃的毒藥,其可畏可怖,不可同日而語。

原來當年鮮于通害死胡青牛的妹子胡青羊,這姑娘明知他薄倖負義,但恩情不斷,臨死時反求兄長維護愛郎。胡青牛的妻子毒仙王難姑却心下不忿,在他身上下了金蠶蠱毒,胡青牛記著對妹子發過的誓言,終於救活了他。這鮮于通也眞工心計,乘著在胡青牛家中養傷之便,偸了王難姑的兩對金蠶,此後依法飼養,製成毒粉,藏在扇柄之中。扇柄上裝有機括,一加掀按,再以内力逼出,便能傷人於無形。他適纔一動手便被張無忌制住,呼吸一暢,内力使發不出,直到張無忌放手相讓,他即以「鷹蛇生死搏」中的一招「鷹揚蛇竄」,用扇柄虛指,將金蠶蠱毒射向敵人。幸得張無忌内力深厚無比,臨危之際屏息凝氣,反將毒氣噴回到鮮于通身上,只要他内力稍差,那麼眼前在地下輾轉呼號之人,便不是鮮于通而是他了。

張無忌熟讀王難姑的「毒經」,深知這金蠶蠱毒的厲害,暗中早已將一口眞氣運遍周身,察覺絶無異狀,這纔放心,眼見鮮于通如此痛苦,不禁起了惻隱之心,但想︰「我救是可以救他,却要他親口吐露自己當年的惡行。」於是朗聲道︰「這金蠶蠱毒救治之法,我倒也懂得,只是我問你什麼,你須老實回答,若有半句虛言,我便撒手不理,由你受罪七日七夜,到時肉腐見骨,滋味可不好受。」鮮于通身上雖痛,神志却極清醒,暗想︰「當年王難姑在我身上下了此毒之後,也説要我苦受折磨七日七夜之後,這纔肉腐見骨而死,怎地這小子説得一點也不錯?」可是心中仍不信他會有蝶谷醫仙胡青牛的神技,能解自己身上的劇毒,説道︰「你\dash{}救不了我的\dash{}」

張無忌微微一笑,倒過摺扇,在他腰眼中點了一點,説道︰「在此處開孔,傾入藥物後縫好,那便能驅走蠱毒。」鮮于通忙不迭的道︰「是,是!一點也\dash{}也\dash{}不錯。」張無忌道︰「那麼你説罷,你這一生之中,做過什麼虧心事。」鮮于通道︰「没\dash{}没有\dash{}」張無忌雙手一拱道︰「請了!你在這児躺七天七夜吧。」鮮于通忙道︰「我\dash{}我説\dash{}」可是要他當衆人之前,説出自己生平的虧心事來,那究是大大的爲難,他嚅嚅半晌,終於不説。突然之間,華山派中兩聲清嘯,同時躍出二人,手中長刀閃耀,縱身來到張無忌身前,一高一矮,年紀均已五旬有餘。那身矮老者尖聲説道︰「姓曾的,我華山派可殺不可辱,你如此對付我們鮮于掌門,非英雄好漢所爲。」

張無忌一抱拳,説道︰「兩位尊姓大名?」那矮小老者怒道︰「諒你也不配問我師兄弟的名號。」一俯身,左手便去抱鮮于通。張無忌拍出一掌,將他逼退一步,冷冷的道︰「他周身是毒,只須沾上一點,便和他一般無異,閣下還是小心些吧!」那矮小老者一怔之間,只聽鮮于通叫道︰「快救我\dash{}快救我\dash{}白垣師哥,是我用這金蠶蠱毒害死的,此外再也没有了,再也没虧心事了。」他此言一出,那高矮二老以及華山派人衆,一齊大驚。矮老者道︰「白垣是你害死的?此言可眞?你怎麼説他死於明教之手?」鮮于通叫道︰「白\dash{}白師哥\dash{}求求你,饒了我\dash{}」他一面説,一面不住的磕頭求告,説道︰「白師哥\dash{}你死得很慘,可是誰叫你當時這般逼迫於我\dash{}你要説出胡家小姐的事來,師父決不能饒我,我\dash{}我只好殺了你滅口啊。白師哥\dash{}你放了我\dash{}你饒了我\dash{}」雙掌用力扼迫自己的喉嚨,又道︰「我害了你,只好嫁禍於明教,可是\dash{}可是\dash{}我給你燒了多少紙錢,又給你做了多少法事,你怎麼還來索我的命?你的妻児老小,我也一直給你照顧得衣食無缺啊。」

此刻雖然日光普照,廣場上到處是人,但鮮于通這幾句哀求之言説得陰風慘慘,令人不寒而慄,似乎白垣的鬼魂眞的到了身前一般。華山派中識得白垣的衆人,更是暗自驚懼。張無忌聽他如此説,似也大出意料之外,本來只想要他自承以怨報德、害死胡青牛之妹的事,那知他反而招供害死了自己的師兄。原來胡青羊雖是因他而死,究竟是她自盡。白垣却是他親手加害。當時白垣身中金蠶蠱毒後輾轉翻滾的慘狀,今日鮮于通一一身受,腦海中想到只是「白垣」兩字,又驚又痛之下,便像自己見到白垣的鬼魂前來索命。

張無忌也不知那白垣是什麼人,但聽了鮮于通的口氣,知他將暗害白垣的罪行推在明教的身上,華山派所以參與光明頂之役,多半由此而起,於是朗聲説道︰「華山派各位聽了,白垣師父非明教所害,各位可錯怪了旁人。」那高大的老者突然快如閃電的手起一刀,往鮮于通頭上劈將下去。張無忌摺扇伸出,在他刀上一點,那柄長刀盪了開去,拍的一聲,砍在地下,直埋入土裡一尺有餘。那高老者怒道︰「這人是本派叛徒,人人得而誅之,你何必插手干預?」張無忌道︰「我已答應治好他身上的蠱毒,説過的話可不能不算。貴派門内紛爭,儘可待回歸華山之後,慢慢清理不遲。」那矮老者道︰「師哥,此人之言不錯。」飛起一脚,踢在鮮于通背心「大椎穴」上,這一脚既踢中了他的穴道,又將他身子踢得飛了起來,直摜出去,拍撻一聲,摔在華山派衆人的身前。鮮于通穴道上受踢,雖然全身痛楚不減,却已叫喊不出聲音,只是在地下掙扎扭動。他雖有親信門人弟子,但生怕沾到他身上的劇毒,誰也不敢上前救助。

那矮老者向著張無忌道︰「我兄弟倆,是鮮于通這傢伙的師叔,你幫我華山派弄明白了一件大事,令白垣師侄沉冤得雪,我謝謝你啦!」説著深深一揖,那高老者跟著也是一揖,張無忌急忙還禮,道︰「好説,好説。」那矮老者舉刀在手,虛砍一刀,厲聲道︰「可是我華山派的清名令譽,被你這小子當衆敗壞無遺,我兄弟倆跟你拚了這兩條老命!」那高大老者也道︰「我兄弟倆,跟你拚了這兩條老命。」敢情他身材雖然高大,却是唯那矮老者馬首是瞻,矮老者説什麼,他便跟什麼。張無忌道︰「華山派清者自清,濁者自濁,偶爾出一個敗類,不礙貴派威名。武林中不肖之徒,各大門派均是在所難免,兩位何必耿耿於懷?」那高老者道︰「依你説是不礙的?」張無忌道︰「不礙的。」高老者道︰「師哥,這小子説是不礙的,咱們就算了吧!」原來這高老者性子戇直,對張無忌又是暗存怯意,有些不敢和他動手。

那矮老者厲聲道︰「先除外侮,再清門戸。華山派今日若是勝不得這小子,咱們豈能再立足於武林之中。」那高老者道︰「好!喂,小子,咱們可要兩個打你一個。你要是覺得不公平,那便乘早認輸了事。」那矮老者眉頭一皺,喝道︰「師弟,你\dash{}」張無忌接口道︰「兩個打我一個,那是再好也没有,倘若你們輸了,可不能再跟明教爲難。」那高老者大喜,大聲道︰「咱們兩個打你一個,那你決計活不了。我師兄弟有一套兩儀刀法,變化莫測,聯刀攻敵,萬夫莫當。我就只擔心你定要單打獨鬥,一個對一個。你既肯一個對我們兩個,那是輸定了,説過的話,可不許反悔。」張無忌道︰「我決不反悔便是,老前輩刀下留情。」那高老者道︰「我刀是決不容情的,這路兩儀刀法一經施展,越來越是凌厲,那可没有什麼客氣。我瞧你這小子爲人也不壞,砍死了你,倒是怪可憐的\dash{}」那矮老者怒喝︰「師弟,少説一句成不成?」

\chapter{兩儀劍法}

那高老者道︰「少説一句,當然可以。不過我是警告他叫他留神,咱師兄弟這套兩儀刀法,乃是反兩儀,式式不依常規\dash{}」矮老者厲聲喝道︰「住口!」轉頭向張無忌道︰「請接招!」一刀便砍了過去。張無忌舉起鮮于通那柄摺扇,按在刀背上一引,那高老者大聲叫道︰「喂,喂!不成,不成!這個樣子,咱們寧可不比。」張無忌道︰「怎麼?」那高老者道︰「這把扇子中有毒,不小心濺了開來,那可不是玩的。」張無忌道︰「不錯,這種劇毒之物,留在世上只有害人。」右手食中兩根手指挾住扇柄,往下一擲,那扇子嗤的一聲直没入土中,地下僅餘一個小孔。這一手神功,廣場之上再無第二人能彀做到,衆人忍不住都大聲喝起采來。

那高老者將刀挾在腋下,雙手用力鼓掌,説道︰「你快去取一件兵刃來吧。」張無忌生性誠樸,本來不願當衆炫耀,不過今日的局面大異尋常,倘若不顯示神功,藝壓當場,要想六大派人衆就此罷手、回歸中原,那可是千難萬難了,便道︰「前輩看我用什麼兵刃的好?」那高老者伸出手去,在他肩頭拍了兩拍,笑道︰「你這娃児倒也有趣,你愛用什麼兵刃,居然問起我來了。」張無忌知他這麼拍幾下不過是老人家喜歡少年的表示,並無惡意,但旁觀衆人却都吃了一驚,心想這兩人對敵過招,一個人隨隨便便的伸手去拍對方肩膀,而張無忌居然也毫不疑忌,倘若那高老者手上使勁,或是乘機拍中他的穴道,豈不是不用比武,便分勝敗?其實張無忌有神功護身,那高老者倘若忽施暗算,也決計傷他不到。只聽那高老者笑道︰「我叫你用什麼兵刃,你便聽我的話麼?」

張無忌微笑道︰「那也不妨。」高老者笑道︰「你這娃児武藝很好,十八般兵刃,想是件件皆能的了。要你空手和我們兩個老人家過招,又是太説不過去。」張無忌笑道︰「空手也可以的。」那高老者遊目四顧,想要找一件最不稱手的兵刃給他,突然看到廣場左角放著幾塊大石,每塊總有二三百斤重,便道︰「我讓你也佔些便宜,用件極沉重的兵刃。」説著向那幾塊大石一指,仰天呵呵大笑。他原意是出個難題,開開玩笑,須知這些大石每一塊都極沉重,力氣小些的人連搬也搬不動,何況那些大石被人長期來當作凳坐,四周光溜溜的,無可著手之處,那能作爲兵刃?不料張無忌微微一笑,説道︰「這件兵刃倒也别緻,老前輩是考我的功夫來著。」説著緩步走了過去。那高老者連連搖手,叫道︰「我跟你説笑話的,不用當眞。還是借一把劍來,試試我師兄弟的刀法吧!」張無忌却不停步,走到石塊之前,左手一抄,已抄起了一塊最大的岩石,輕飄飄的托在手裡,回過身來,説道︰「兩位請!」話聲甫畢,連身帶石一躍而起,縱到了兩個老者的身前。

衆人只瞧得張大了,連喝采也忘記了,那高老者伸手猛拉自己的鬍子,叫道︰「這\dash{}這個可是奇哉怪也。」那矮老者却知今日實是遇上了生平從所未見的大敵,自己師兄弟的一世威名是否能保,全瞧這一戰了,當下穩步凝氣,目光注視著張無忌,説道︰「有僭了!」白光一閃,身隨刀進,直攻張無忌的右脅。那高老者道︰「師哥,眞打嗎?」矮老者道︰「還有假的?」刀鋒一掠,斜劈張無忌的肩頭。張無忌旁退一步讓開,只見斜刺裡青光閃耀,高老者的青鋼刀砍了過來。張無忌喝道︰「來得好!」橫過石頭一擋,{\upstsl{噹}}的一聲響,這一刀砍在石頭之下,火花四濺,石屑紛飛。張無忌提起石頭,順勢推了過去。高老者叫道︰「啊{\upstsl{唷}},這是『順水推舟』,你使大石頭也有招數麼?」

那矮老者大聲喝道︰「師弟,『混沌一破』!」一刀從背後反劃了個弧形,彎彎曲曲的斬向張無忌,那高老者接口道︰「太乙生萌,兩儀合德\dash{}」矮老者接口道︰「日月晦明。」兩人一面呼喝,刀招源源不絶的遞出,張無忌施展九陽神功,將那塊大石托在手裡運轉如意,宛似搬弄一個鐵彈。那高矮二老使開了兩儀刀法,招招狠辣,招招沉雄,那知張無忌手中這塊石頭實在太大,只須稍加轉側,便盡數擋住了二老砍劈過來的招數。

酣鬥良久,張無忌突然將大石往空中一抛,雙手揪住高矮二老的頭頸,面對面的一擠,二老被他抓住穴道,半步也動彈不得,張無忌身子向後彈出,那塊大石已向二老頭頂壓將下來。二老頸後的穴道已被封閉,這塊二百多斤的大石頭{\upstsl{砸}}了下來,焉有不粉身碎骨之理?衆人失聲驚呼聲中,張無忌左掌揚出一拍,將那大石推出丈餘,砰的一聲,落在地下,陥入泥中幾有尺餘,他微笑著伸手在二老肩頭輕輕拍了幾下,説道︰「兩位老人家休得驚慌,晩輩跟兩位開個玩笑。」他這麼一拍,高矮二老被封的穴道登時解了。那矮老者臉如死灰,長嘆道︰「罷了,罷了!」高老者却搖頭道︰「這個不算。」張無忌道︰「怎麼不算?」高老者道︰「你不過仗著力大,搬得起這塊大石頭,可不是在招數上勝了我哥児兩個。」張無忌道︰「那麼咱們再比過。」高老者道︰「再比也可以,不過得想個新鮮法児纔成,否則淨給你佔便宜,咱們輸了也不心服,你説是不是?」張無忌點頭道︰「是!」

小昭站在一旁,一直注視著場中的比拚,這時伸手刮著臉皮,叫道︰「羞啊,羞啊!鬍子一大把,自己老佔便宜,反説吃虧。」她手指上下移動,手腕上的鐵鍊便叮{\upstsl{噹}}作響,甚是清脆動聽。那高老者哈哈一笑,説道︰「常言説得好︰好虧就是便宜。我老人家吃過的鹽,還多過你吃的米。我走過的橋,長過你的路。小丫頭嘰嘰{\upstsl{喳}}{\upstsl{喳}}什麼?」他回過頭來,向張無忌道︰「要是你不服,那就不用比了。反正這一回較量你没有輸,我們也没有贏,雙方扯了個直。再過三十年,大家再比過也不遲\dash{}」那矮老者聽他越説越是胡混,自己師兄弟二人説什麼也是華山派的耆宿,怎能如此耍賴,當即喝道︰「姓曾的,我們認栽了,你要怎般處置,悉聽尊便。」張無忌道︰「兩位請便。在下只不過斗膽調處貴派和明教的過節,實是别無他意。」

那高老者大聲道︰「這個不成!這還没説出比武新鮮的主意,怎麼你就打退堂鼓了?這不是臨陣退縮、望風披靡麼?」矮老者皺著眉頭不語,他知這個師弟雖然一生瘋瘋癲癲,但靠了一張厚臉皮,往往説得對方頭昏腦脹,就此轉敗爲勝。今日在天下衆英雄之前施此技倆,原是没什麼光采,然而如果因此而勝得張無忌,至少功過可以相抵。只聽張無忌道︰「依前輩之意,該當如何?」高老者道︰「咱們華山派這套『反兩儀刀法』的絶藝神功,你是{\upstsl{嚐}}過味道了,殊不知崑崙派有一套『正兩儀劍法』,變化之精奇奥妙,和華山派的刀法可説是一時瑜亮,各擅勝場。倘若刀劍合璧,兩儀化四象,四象生八卦,陰陽相調,水火相濟,唉\dash{}」説到這裡,不住搖頭,緩緩嘆道︰「威力太強,威力太強!你是不敢抵擋的了!」

張無忌轉頭向著崑崙派,説道︰「崑崙派那位高人肯出來賜教?」那高老者搶著道︰「崑崙派中除了鐵琴先生夫婦之外,常人也不配和我師兄弟聯手。就不知何掌門有這個膽量没有?」衆人聽了,心中都是一樂︰「這老児説他傻,却又不傻,他是要激得崑崙派中的兩大高手下場。」

何太沖和班淑嫻對望了一眼,却不認得華山派中這高矮二老是什麼人,他們是掌門人鮮于通的師叔,班輩甚高,想必平時少在江湖上行走,自己又僻處西域,是以不知他二老的名頭,夫妻二人均想︰「這兩個老児鬥不過那姓曾的少年,便想拉我們趕這場混水。一起勝了,他們臉上也有光采,倘若輸了,哼哼,憑咱夫婦二人的兩儀劍法,難道會輸給這個少年,天下絶無是理!」只聽那高老者説道︰「崑崙派的何氏夫婦不敢和你動手,那也難怪。他們的正兩儀劍法雖然還不錯,但失之呆滯,比起華山派的反兩儀刀法來,原來稍遜籌兩籌。」

班淑嫻大怒,縱身入場,指著那高老者道︰「閣下尊姓大名?」那高老者道︰「我也姓何,何夫人請了。」這兩句話一出,旁邉衆人有許多笑了出來,顯然這高老者是撿了個現成便宜。班淑嫻是崑崙派的「太上掌門」,連她丈夫平日也忌她三分,數十年來在崑崙山下頤指氣使慣了,數百里方圓之内,儼然是個女王一般,如何肯受這等奚落取笑?突然間嗤的一聲響,一劍直向那高老者左肩刺了過去。這一下拔劍出招的手法迅捷無倫,在一瞬之前,還見她兩手空空,柳眉微豎,一瞬之後,已是長劍在手,劍尖離高老者的肩不及半尺。那高老者一驚之下,迴刀橫揮,{\upstsl{噹}}的一響,刀劍相交,在千鈞一髮「萬劫不復」,一正一反,均是施發了兩儀術數中的極致。莫看那高老者在張無忌手下縛手縛脚,似是功夫平庸,實則他刀法上的造詣確是不同凡響。

兩人刀劍相交,各自退開一步,不禁一怔,心下均是佩服對方這一招的精妙。兩人派别不同,武功大異,生平從未見過面,但一招之下,發覺自己這套武功和對方若合符節,配合得天衣無縫,猶似一個人一生寂寞,突然間遇到了知己般的喜歡。班淑嫻忍不住想︰「他華山派的反兩儀刀法果然了得,若和他聯手攻敵,當可發揮天下兵刃招數中的極致。」當下回頭向何太沖叫道︰「喂,你過來!」

何太沖雖對妻命不敢有違,但在衆目睽睽之下,仍要擺足掌門人的格子,「哼」的一聲,緩緩站起,四名小僮在前引路,一捧長劍,一捧鐵琴,另外兩個各持拂塵。走到廣場中心,四名小僮躬身退下,分站在何太沖身後。班淑嫻道︰「華山派的反兩儀刀法,招數上倒也不算含糊。」那高老者嬉皮笑臉的道︰「多蒙讚賞。」班淑嫻橫了他一眼,道︰「咱們四個人就拿這小娃児喂喂招,切磋一下崑崙、華山兩派的武功。」她一回頭,突然「咦」的一聲,瞪著張無忌道︰「你\dash{}你\dash{}」原來她和張無忌分手不過四年,雖然張無忌在這四年中自孩童成爲少年,身材高了。嘴唇上也生了淡淡的鬍鬚,但面目依稀還是相識。無忌道︰「咱們從前的事,要不要一切都説將出來?我是曾阿牛。」班淑嫻當即明白了他的用意,他不願以眞姓名示人,如果自己給他揭破,那麼他夫婦恩將仇報的種種不德事情,他也是毫不容情的當衆宣佈了,當下長劍一舉,説道︰「曾少俠武功大進,可喜可賀,還請指教。」言下顯然是説,咱們只比武藝,不涉舊事。張無忌微微一笑,道︰「久仰賢夫妻劍法通神,尚請手下留情。」

何太沖從身後小僮手中接過長劍,説道︰「曾小俠用什麼刀刃?」張無忌一見到他,便想起那對會吸毒的金冠銀冠小蛇。他摔入絶谷,這對小蛇因無食料,竟致生生餓死,此刻想來,不禁有些可惜,跟著又想起他夫婦在武當山上逼死自己父母、逼迫自己吞服毒酒、何太沖將自己打得青目鼻腫,一把將自己擲向山石。

倘若不是楊逍正好在旁,及時出手相救,自己這時屍骨早杇,還説什麼做魯仲連、做和事老?張無忌想到此處,胸頭怒氣上衝,心想︰「好何太沖,那一天你打得我何等厲害,今日我雖不能要了你的性命,至少也得狠狠打你一頓,出了當日這口惡氣。」只見何太沖夫婦和華山派的高矮二老分站四角,兩把彎刀和兩柄長劍在日光下閃爍不定,突然間雙臂一振,身子筆直躍起,到了空中,輕輕一個轉折,撲向西首一棵梅樹,左手一探,折了一枝梅花下來,這纔迴身落地。

衆人適纔已見過他施展輕功,此刻迴旋折梅,雖非躍得極高極遠,但神姿飄逸,輕若行雲,人人看了都是心頭説不出的舒適。只見他手持梅花,緩步走入四人之間,高舉梅枝,説道︰「在下便以這梅枝當兵刃,領教崑崙、華山兩派的高招。」那梅枝上疏疏落落的生著十來朶梅花,其中半數兀自含苞未放。衆人聽他如此説,都是一驚︰「這梅枝一碰即斷。那能和對方的寶劍利刃較量?」班淑嫻冷笑道︰「很好,你是絲毫没將華山、崑崙兩派的功夫放在眼下了?」張無忌道︰「我曾聽先父言道,當年崑崙派前輩何足道先生,琴劍棋三絶,世稱『崑崙三聖』。只可惜咱們生得太晩,没能瞻仰前輩的風範,實爲憾事。」這幾句話人人都聽得出來,張無忌讚崑崙派的前輩,却是將眼前的崑崙人物瞧得不堪一擊。猛聽得崑崙派中一人聲如破鑼的大聲喝道︰「小賊種,諒有多大能耐,竟敢對我師父無禮?」喝聲未畢,一個身穿杏黃道袍,滿腮虯髯的道人從人叢中竄了出來,身隨劍至,直向張無忌背心刺去。這道人身法極快,這一劍刺出,雖似事先已有警告,但劍招實在去得太快,實和偸襲絶無分别。

張無忌竟不轉身,待那劍尖將要觸到他背心衣服,左足向後翻出,一脚踏將下去,剛好將長劍踏在地下。那道人用力一抽,竟是紋絲不動。張無忌緩緩回過頭來,一看這個道人,原來是他初回中原之時,首次在海船中即行遇到的西華子,此人性子暴躁,一再對張無忌的母親殷素素口出無禮之言。張無忌心中一酸,説道︰「你是西華子道長?」西華子滿臉脹得通紅,並不答話,只是竭力抽劍,張無忌突然左脚一鬆,順勢用鞋底在劍刃上一點。西華子没料到他會陡然鬆脚,力道用得猛了,一個踉蹌,向後退了一步。憑著他的武功修爲,這一下雖然出其不意,但立時便可拿樁站定,不料剛使得個「千斤{\upstsl{墬}}」,猛地裡劍上一股極強的力道傳來,將他身子一推,登時一屁股坐倒,絶無抗禦的餘地,跟著聽得叮叮叮的幾聲清脆響聲,手中長劍已寸寸斷絶,掌中抓著的只餘一個劍柄。西華子驚愧難當,一瞥眼,只見師娘滿臉怒色,心知自己這一下丟了師門極大的臉面,事過之後必受重責,不禁更是惶恐,急忙一躍站起,喝道︰「小賊種\dash{}」張無忌本想就此放他回去,但聽他罵到「小賊種」三字,那是辱及了父母,手中梅枝在他身上一掠,已運勁點了他胸腹間的三處要穴,臉上裝作不知,對高矮二老和何氏夫婦説道︰「就請進招吧!」

班淑嫻對西華子低聲喝道︰「走開!丟的大人還不彀麼?」西華子道︰「是!」可是竟不移步。班淑嫻怒道︰「我叫你走開,聽見没有?」西華子道︰「是!是!師娘,是!」口中説得十分恭謹,却仍不動。班淑嫻怒極,心想這傢伙幹麼不聽起話來了?原來張無忌拂穴的手法快極,班淑嫻眼光雖然敏鋭,却萬萬想不到張無忌的内功出神入化,勁力可惜柔物而傳,梅枝的輕輕一拂,無殊以判官筆連點他數處穴道,當下伸手在西華子肩頭重重一推,喝道︰「站開些,别礙手礙脚!」

西華子的身子平平向旁移開數尺,手足姿式却是半點没有改變,就像是一尊石像被人推了一掌一般。這麼一來,班淑嫻和何太沖才知這徒児已在不知不覺間被張無忌點了穴道,心下暗自駭然,何太沖伸手便去西華子腰脅推拿數下,想替他解開穴道。那知勁力透入,西華子仍是一動不動。張無忌指著倚靠在楊逍旁的楊不悔道︰「這個小姑娘當年被你們封住了穴道,強灌毒酒,我無法給她解開,今日令徒也是一樣。貴我兩派的點穴手法截然不同,那也不足爲異。」

衆人聽他這麼説,眼光都射向楊不悔身上,見她現下也不過十五六歳年紀,説到「當年」,當然年紀更小,何太沖年婦以一派掌門之尊,竟然這般欺侮一個小姑娘,實在太失身份。班淑嫻見衆人眼色有異,心想多説舊事有何好處,一劍便往張無忌眉心挑來,便在同時,何太沖長劍指向張無忌後心,跟著華山派高矮二老的攻勢也即展開。張無忌身形一晃,從刀劍之間竄了開去,梅枝在何太沖臉上掠過。何太沖斜刺他腰脅,逼他以梅枝來格。張無忌左手食指彈向矮老者的單刀,梅枝便格向何太沖的長劍。何太沖長劍微轉,劍鋒對準梅枝削去,心想你武功再高,木質的樹枝終不能抵擋我劍鋒之一削,那知張無忌的梅枝跟著微轉,平平的搭在劍刃之上,一股柔和的勁力送出,何太沖的長劍直盪了開去,{\upstsl{噹}}的一響,剛好格開了高老者砍過來的一刀。

高老者叫道︰「啊哈,何太沖,你倒戈助敵麼?」何太沖臉上微微一紅,不能自認劍招被敵人内勁引開,只説︰「胡説八道!」狠狠一劍,疾向張無忌刺去。他出招攻敵,班淑嫻正好在張無忌的退路上伏好了後著,高矮二老跟著施展反兩儀刀法。這兩儀刀法和兩儀劍法雖然正反有别,但均是從八卦中化出,再回歸八卦,可説是殊途而同歸。四個人越使越是順手,刀劍配合得嚴密無比。張無忌本也料到他四人聯手,定然極不好鬥,却不知正反兩套武功聯在一起之後,陰陽相輔,竟是没絲毫破綻。他數次連遇險招,倘若手中是一件兵刃,當可運勁震斷對方刀劍,偏生一開始過於托大,只拿了一根梅枝,陡然間矮老者一刀著地捲到,張無忌閃身相避,班淑嫻一劍疾彈出來,喝一聲︰「著!」已刺中張無忌的大腿。

張無忌回指一點,何太沖的長劍又已遞到,高矮二老的單刀分取上盤下盤。張無忌一時難以抵禦,靈機一動,滑步搶到了西華子身後。班淑嫻跟上刺出一劍,招數之狠,勁力之猛,直是欲置張無忌於死地,那裡是比武較量的行逕?張無忌在西華子身後一縮,班淑嫻這一劍險險刺中徒児身子,硬生生的拆開,西華子却已「啊喲」一聲的叫了出來。待得何太沖從左首攻到,張無忌又是在西華子身側一避,他一時還捉摸不到這兩路正反兩儀武功的要旨,想不出破解的法子,只有繞著西華子東一轉,西一閃,暫且將他當作擋避刀劍的盾牌!心中暗叫︰「張無忌啊張無忌,你也未免太過小覷了天下英雄,今日纔遭此厄。『驕著必敗』這四個字,從今以後可得好好記在心中。焉知世上没有比『乾坤大挪移』更厲害的武功,没有比九陽神功更渾厚的内勁,常言道天外有天,人上有人,那是半點不錯的。」

他大腿上刺傷雖然不重,但鮮血點點滴滴的流下,瞧上去情勢頗爲狼狽。旁觀衆人之中,却有許多人忍不住失聲笑了出來。原來西華子猶似泥塑木彫般站在當地,張無忌在他身側鑽來躍去,每當何太沖等四人的刀劍從他身旁間不容髮的刺去劈過,西華子便大聲「咦!」

\qyh{}啊!」

\qyh{}唉喲!」的叫喊,偏又半點動彈不得,當眞是十二分的驚險,十二分的滑稽。班淑嫻怒氣上衝,眼見接連數次均可將張無忌傷於劍下,都是西華子橫擋其間,礙手礙脚,恨不得一劍將他劈爲兩段,只是究有師徒之情,下不得手。華山派的高老者叫道︰「何夫人,你不下手,我可要下手了。」班淑嫻恨恨的道︰「我管得你麼?」高老者一刀橫掃,逕往西華子腰間砍去。張無忌心想不妙,這一刀若是教他砍實了,不但自己少了個擋避兵刃的盾牌,而且西華子爲己而死,又生糾紛,當下左手衣袖一拂,一股勁風,將高老者的這一刀盪了開去。

那矮老者一聲不響,單刀斜劈,直攻張無忌,無忌身子一閃,讓在右首,矮老者這一刀却不變向,疾向西華子肩頭劈了下去,便似收不住勢,非歌住他身上不可,口中却叫道︰「西華道兄,小心!」原來這矮老者極工心計,知道若是劈死了西華子,勢須和崑崙派結怨成仇,這時裝作迫於無奈,咎非在己,以後便可推卸罪責。張無忌回身一掌推出,直拍矮老者胸膛。矮老者氣息一窒,左手一掌推出,手中單刀却仍是劈向西華子,驀地裡雙掌相交,矮老者踉蹌後退,險些跌倒。

西華子眼見張無忌兩番出手,相護自己,心下暗生感激之意,又想︰「今日若是逃得性命,決不能和華山派這高矮二賊善干罷休。」這時何太沖、班淑嫻夫婦見張無忌迴護西華子,兩人均是一般心意︰「這小子多了一層顧慮,交手時更加縛手縛脚。」竟不感謝他救徒之德,劍招上越發的凌厲狠辣。少林、武當、峨嵋各派中的高手見此情形,不禁暗暗搖頭,胸中微微感愧意,均覺若是在此局勢之下殺了張無忌,自己也不免内疚於心。

高矮二老的刀招也是決不放鬆,忽攻張無忌,忽砍西華子,這其中雖然張無忌是主,但二老知道若要傷他,極是爲難,但如攻擊西華子而引他來救,易於造成對方的破綻,因此反賓爲主,二老的兩柄彎刀,倒是往西華子身上招呼的爲多。無忌越鬥越是情勢不利,心想︰「我打他們不過,送了自己性命也就罷了,何必饒上這個道人?」當下已砍向西華子下盤。張無忌飛起一脚,踢他手腕,矮老者忙縮手時,不料西華子穴道已解,突然砰的一拳,結結實實打在矮老者鼻梁之上,登時鮮血長流。矮老者的武功原比西華子高得多,但那料得到他獃立了這麼久,居然忽能活動,變起倉卒,自是閃避不及。衆人一見,無不哈哈大笑。班淑嫻忍笑道︰「西華,快退下!」西華子道︰「是!那高的還欠我一拳!」伸掌想去打那高老者時,矮老者盤掃一腿,虛砍一刀,拍的一響,左手肘已撞在他的胸口。這三下連環三式,乃是華山派的絶技之一,西華子身子晃了幾晃,喉頭一甜,吐出了一口鮮血。何太沖左手手掌搭在他的腰後,掌力一吐,將他一個肥大的身軀送出數丈以外,向那矮老者道︰「好一招『長江三疊浪』!」手中長劍却是嗤的一聲,刺向張無忌,他掌底驅徒、口中譏刺、劍下刺敵,分别對付三人,竟然瀟灑自如,可見崑崙門人著實有非凡的修爲。

高矮二老不再答話,凝神向無忌進擊,此刻他四人雖然互有心病,但西華子這障礙一去,四個人刀法又配合得宛似天衣無縫一般。張無忌内力充沛,便是再鬥一日一晩,也不致疲乏,但對手四人的招數實在太過精妙,此攻彼援,你消我長,四個人合成了一個八手八足的極強高手,招數上反覆變化,層出不窮,所生出的壓力越來越是沉重,看來再鬥二三百招,張無忌不免命喪當場。

原來張無忌的九陽神功,學自天竺國(今印度)達摩老祖所傳的「九陽眞經」,而明教的「乾坤大挪移」,則淵源於波斯,兩者相合,已達人類智慧之巓峰。但華山、崑崙兩派的正反兩儀刀劍之術,却是以中國固有的河圖洛書、以及伏羲文王的八卦方位中推演而得,其奥妙精微之處,如果研到極致,比之西域的乾坤大挪移實是有過之而無不及,只是易理深邃,何太沖夫婦及高矮二老只不過學得二三成而已,否則早已將張無忌斃於刀劍之下,但饒是如此,張無忌空有一身驚世駭俗的渾厚内力,却也無法脱困於這正反兩儀的刀劍之外。

這一番劇鬥,人人看得怦然心動,只聽得何氏夫婦的劍上生出嗤嗤聲響,劍氣縱橫,步步進逼。張無忌連試數次,知道若求衝出包圍,原是毫不爲難,輕功一施,對方四人中無一追趕得上。但自己逃走雖易,要解明教之圍,却是談不上了,眼下之計,只是嚴密守護,累得對方力疲,再行俟機進攻。不料敵方四人都是内力悠長之輩,雙刀雙劍組成了一片光幕,將張無忌密密包圍,不知何時才顯疲累之象。張無忌無可奈何,只得苦苦支撐。何太沖等雖佔上風,但四人心下却都滿不是味児,以他們的身份名望,别説四人聯手,就是一對一的相鬥,給這麼一個後進少年支持到三四百合仍是是收拾他不下,也已大失面子。好在張無忌有挫敗神僧空性的戰績在先,無人敢小覷於他,否則眞是汗顏無地了。各人在兵刃上都感覺得到,張無忌反擊的招數漸少,但要進招傷他,總是給他在極無可能的局面下躱了開去。四人都是欠臨大敵,身經百戰,越鬥得久,越是不敢怠忽,竟是半點不見焦躁,沉住了氣,決不貪功冒進。旁觀各派中的長老名宿,便都指指點點,以此教訓本派的弟子。

峨嵋派的滅絶師太對衆弟子道︰「這少年的武功十分怪異,但崑崙、華山的四人,招數上已鉗制得他縛手縛脚。中原武功博大精深,豈是西域的旁門左道所能企及。兩儀化四象,四象化八卦,正變八八六十四招,奇變八八六十四招,正奇相合,六十四再以六十四倍之,共有四千零九十六種變化。天下武功變化之繁,可説無出其右了。」周芷若自張無忌下場以來,一直関心著他的勝負,她在峨嵋門下,頗獲滅絶師太的歡心,已得她易經原理的心傳,這時朗聲問道︰「師父,這正反兩儀,招數雖多,終究不脱於太極化爲陰陽兩儀的道理。陽分太陽、少陰,陰分少陽、太陰,是爲四象。太陽爲乾兌,少陰爲離震,少陽爲巽坎,太陰爲艮坤。弟子看這四位前輩招數果然精妙,最厲害的還是在脚下步法的方位。」她聲音清脆,一句句以丹田之氣媛緩的送了出來,許多人不禁都轉頭瞧著她。

張無忌雖在力戰之中,但耳目聰明,這幾句話聽得清清楚楚,一瞥之下,見説話的竟是周芷若,心中一動︰「她爲什麼這般大聲説話,難道是有意指點於我麼?」只聽減絶師太道︰「你的眼光倒也不錯,能瞧出前輩武功中的精要所在。」周芷若自言自語的道︰「乾南、坤北、離東、坎西、震東北、兌東南、巽西南、艮西北。自震到乾爲順,自巽到坤爲逆。」她提高聲音道︰「師父,是了,正如你所教︰天地定位,山澤道氣,雷風相薄,水火不相射,八卦相錯。數往者順,知來者逆。崑崙派的正兩儀劍法,那時自震位至乾位的順;華山派的反兩儀刀法,則是自巽位至坤位的道。師父,你説是不是啊?」滅絶師太聽徒児指了出來,心下甚喜,點頭道︰「你這孩子,倒也不虧了我平時的教誨。」她平時極少許可别人,這兩句話已是最大的讚譽了。

\chapter{獨戰高手}

滅絶師太欣悦之下,没留心到周芷若的話聲,實在太過響亮,兩人面對面的説話,何必中氣十足,將語音遠遠的傳送出去,但旁邉已有不少人覺察到異狀。周芷若見許多眼光射向自己,索性裝作天眞歡喜之狀,拍手叫道︰「師父,是啦是啦!咱們峨嵋派的四象掌圓中有方,陰陽相成,圓於外者爲陽?方於中者爲陰,圓而動者爲天,方而靜者爲地,天地陰陽,方圓動靜,似乎比這正反兩儀之學又勝一籌。」滅絶師太素來自負本派四象掌爲天下絶學,聽周芷若這麼説,正好對了她自高自大的心意,微微一笑,道︰「道理是這麼説,但也要瞧運用者的功力修爲。」

張無忌於八卦方位之學,小時候也曾聽父親詳細講過,須知易經的義理,原是張三丰、宋遠橋等人最得意的學問,張翠山所知雖淺。但武當派的内功以易經爲基本,那是非習不可的。這時他聽周芷若説及四象順逆的道理,心中一凜,察看何氏夫婦和高矮二老的步法招數,果是從四象八卦中變化而出,無怪自己的乾坤大挪移心法一點施展不上,原來是西域最精深的武功,遇上了中土最精深的心法,相形之下,還是中土功夫的義理更深,張無忌所以暫時不敗,只不過他已將西域武功練到了最高境界,而何氏夫婦,高矮二老的中土武功,所學尚淺而已。在這一瞬時之間,他腦海中如電閃般連轉了七八個念頭,立時想到七八種方法,每一種均可在舉手間將四人一一擊倒。

但他轉念又想︰「倘若我此時施展,只怕滅絶師太要怪上周姑娘,這老師太心狠手辣,什麼事做不出來?我可不能連累於她。」於是手上招式半點不改,凝神觀察四個人的招數,他既把握到敵手武功的總綱,看出去即是頭頭是道,再不似先前有如亂絲一團,分不清中間的糾葛披紛。這一番察看,教張無忌更領略到了中土武功的秘奥,雖非登堂入室,却已懂了個大略。

周芷若見他絶無好轉的徵象。心下暗自焦急,尋思︰「他在全力赴敵之際,自不能在片刻間悟到這種精微的道理。」眼見何氏夫婦越逼越緊,張無忌更是支持不住,突然間鼓起勇氣,仗劍飛身而出,叫道︰「崑崙、華山的四位前輩,你們既然拾奪不下這小子,讓我們峨嵋派來試試。」何太沖大怒,喝道︰「别來囉{\upstsl{嗦}}攪局,給我走開些。」班淑嫻柳眉倒豎,説道︰「這小子是你什麼人,要你一再迴護於他?你吃裡扒外,我崑崙派可不是好惹的。」周芷若被她説破心事。滿臉通紅。滅絶師太喝道︰「芷若,回來!他崑崙派不是好惹的,你没聽見嗎?」這兩句話的語氣,顯然是袒護徒児。

張無忌心中好生感激,暗想若再纏鬥下去,只怕這位周姑娘還要另生他法來相助自己,要是給滅絶師太瞧破了,那可於她有極大的危臉,於是哈哈大笑,説道︰「我是峨嵋派的手下敗將,會被滅絶師太擒獲,她們峨嵋派當然比你崑崙派高明些。」向左踏出兩步,右手梅枝一帶,一股勁風撲向矮老者的後心。這一招的方位時刻,拿捏得準確異常,矮老者身不由主,一刀往班淑嫻肩頭砍了下去。原來張無忌使的正是乾坤大挪移功夫,但依著八卦方位,倒反了矮者者刀招的去勢。班淑嫻一驚之下,急忙迴劍一檔,呼的一聲,高老者的彎刀又已砍至。

何太沖搶上袒護妻子,舉劍格開高老者的彎刀,張無忌一掌拍出,引得矮老者一刀刺向何太沖小腹。班淑嫻大怒,刷刷刷三劍,逼得矮老者手忙脚亂。矮老者叫道︰「别上了這小子的當!」何太沖登即省悟,倒反長劍,向張無忌刺去。張無忌挪移乾坤,何太沖這二劍在中途轉了方向嗤的一響,刺中了高老者的左臂。

高老者痛得哇哇大叫,一刀猛向何太沖當頭砍了下去。矮老者舉刀格開,喝道「師弟别亂,是那小子搗鬼,唉喲\dash{}」原來便在此時,張無忌迫使班淑嫻劍招轉向,一劍刺中了矮老者的肩後。頃刻之間,華山二老先後中劍受傷,旁觀衆人轟然大亂。只見張無忌梅枝輕拂、手掌斜引,以高老者之刀去攻班淑嫻左脅,以何太沖之劍去刺矮老者小腹。再鬥數回合,驀地裡何太沖夫婦雙劍相交,挺刃互刺。高矮二老者兵器碰撞,揮刀砍殺。到這時候,人人都已看出,乃是張無忌從中牽引,搞亂了四人兵刃的方向,至於他用的是什麼法子,却是無一能懂。只有光明使者楊逍學過乾坤大挪移之術,依稀瞧了些眉目出來,但也決不相信世間竟然有人將這功夫練到了如此精深的地步。

但見場中夫婦相鬥,師兄弟互斫,殺得好看煞人,班淑嫻不住的叫道︰「轉旡妄位,進蒙位,搶歸妹位\dash{}」可是張無忌乾坤大挪移功夫四面八方的罩住了,不論他們如何苦苦掙扎,刀劍使將出去,總是不由自主的招呼到了自己人身上。只聽那高老者叫道︰「師哥,你出手輕一些成不成?」矮老者道︰「我是砍這小賊,又不是砍你。」高老者叫道︰「師哥小心,這一刀只怕要轉彎\dash{}」果然不出所料,話聲未畢,那彎刀斜斜的斫向矮老者腰間。何太沖道︰「娘子。這小賊\dash{}」班淑嫻{\upstsl{噹}}的一聲,將長劍擲在地下,矮老者心想不錯,若以拳掌扭打,料想這小賊再不能用此邪法,跟著抛去單刀,一拳便向張無忌胸口打去,那知{\upstsl{嗖}}的一聲響,何太沖長劍迎面點至。矮老者手中没了兵刃,只得急忙低頭相避。班淑嫻叫道︰「兵刃撒手!」何太沖用力一甩,長劍遠遠擲出,高老者也跟著鬆手放刀,左手以擒拿手向張無忌後頸抓去。五指一緊,手掌中多了一件硬物,一怔之下,却是自己的彎刀,原來給張無忌搶過來遞回了他的手中。

高老者道︰「我不用兵刃!」使勁擲下,張無忌斜身抓住又已送在他的手裡,接連數次,高老者竟是無法將自己的兵刃抛擲脱手。高老者驚駭之餘,自己想想也覺古怪,哈哈大笑起來,説道︰「他媽的,臭小子當眞邪門。」這時矮老者和何氏夫婦拳脚齊施,分别向張無忌猛攻,華山、崑崙的拳掌之學,絲毫不弱於兵刃,一拳一脚,均具極大威力。但張無忌身子滑如游魚,每每在間不容髮之際避了開去,有時反擊一招半式,却又令三人極難擋架。到此地步,四人均已知萬難取勝,各自存了全身而退的打算。高老者突然叫道︰「臭小子,暗器來了!」一聲咳嗽,一口濃痰向張無忌吐去。無忌側身一讓,高老者已乘機將彎刀向背後抛出,笑道︰「你還能\dash{}啊喲\dash{}對不住\dash{}」原來張無忌左掌一引,將班淑嫻帶了過來,高老者這口濃痰正好打在她眉心之間。

班淑嫻怒極,決意與之同歸於盡,十指疾往張無忌抓去,矮老者雙手勾拿,正好擋著他的退路,高老者和何太沖眼見這是自纏鬥以來唯一的良機,一齊撲上,心想這一次將他擠在中間,抓住了厮打,雖然觀之不雅,却管教他再也無法取巧。張無忌一聲清嘯,拔身而起,雙手同時施展挪移乾坤的神功,在半空中輕輕一個轉折,飄然落在丈許之外,但見何太沖抱住了妻子的腰,班淑嫻抓住丈夫的肩頭,高矮二老也是互相緊緊摟住,四個人都摔在地下。何氏夫婦一發覺不對。急忙鬆手躍起,那高老者大叫︰「抓住了,這一次瞧你逃到那裡?啊喲不是\dash{}」矮老者怒道︰「快放手!」高老者道︰「你不先放手,我怎麼放得了?」矮老者道︰「少説二句成不成?」高老者怒道︰「你也比我高明不到那裡,還在擺師哥的架子。」

矮老者放開了雙臂,厲聲道︰「起來!」高老者對師哥究屬是心存畏懼。急忙縮手,雙雙躍起身來。高老者叫道︰「喂,臭小子,你這不是比武,專使邪法,算那門子的英雄?」矮老者知道越是糾纏下去,越是出醜,大大方方的向張無忌抱拳道︰「閣下神功蓋世,老朽生平從所未見,華山派認栽了。」張無忌還禮道︰「得罪!晩輩勝得僥倖,適纔若不是四位手下容情,晩輩已命喪正反兩儀的刀劍之下。」他這句話倒不是空泛的謙詞,當周芷若在未加指點之時,他確是險象環生,雖然終於獲勝,但對這四位對手的武功,實無絲毫小視之心。高老者得意洋洋的道︰「是麼?你自己也説勝得僥倖。」張無忌道︰「兩位尊姓大名?日後相見,也好有個稱呼。」高老者道︰「我師哥是『威震\dash{}』」矮老者喝道︰「住嘴!」向張無忌道︰「敗軍之將,羞愧無地,賤名何足掛齒。」説看回入華山派人叢之中。高老者拍手笑道,「勝敗乃兵家常事,老子是漫不在乎的。」抬起地下的兩柄彎刀,施施而歸。

張無忌走到鮮于通身邉,俯身又點了他兩處穴道,説道︰「此間大事一了,我即爲你療毒。此刻先阻住你毒氣入心。」便在此時,忽覺背後涼風襲體,微感刺痛,張無忌一驚之下,不及趨避,足尖使勁,身子如電般斜飛而上,只聽得{\upstsl{噗}}{\upstsl{噗}}兩聲輕響,跟著「啊」的一下長聲微呼,他在半空中轉過頭來,只見何太沖和班淑嫻的兩柄長劍並排插在鮮于通的胸口。原來何氏夫婦縱橫半生,却當衆敗在張無忌手底,實是心有未甘,兩人拾起長劍後對望一望,眼見張無忌正俯身在點解于通的穴道,突然使出一招「無聲無色」,疾向他背後刺了過去。

這「無聲無色」是崑崙派劍學中的絶招之一,必須兩人同使,兩人功力相若,内勁相同,劍招之出勁力可恰恰相反,於是兩柄長劍上所生的盪激之力、破空之聲一齊抵消。這種劍招本意是在黑暗之中應敵的極厲害手法,對方武功再強,也不能聞聲辨器,事先絶無半點徵兆,白刃已然加胸,但若用之背後偸襲,也是令人無法防備。不料張無忌心意不動,九陽神功自然護體。變招快極,但饒是如此,背上衣衫也已被劃破了兩條長縫。何氏夫婦收招不及,雙劍竟將華山派的掌門人釘死在地下。

張無忌落下地下,共聽得旁觀衆人嘩然大噪,何氏夫婦一不做,二不休,雙劍齊向張無忌攻去,心中均想︰「背後偸襲的險惡勾當既已當衆做了出來,今後顏面何存?若不將他刺死、自己夫婦也不能苟活於世。」是以劍劍是拚命的招數。

張無忌自避了數招,眼見何氏夫婦每一招都是同歸於盡的打法,突然心念一動。身子半蹲,左手在地下抓起一塊泥土,和著掌心中的汗水,捏成了兩粒小小丸藥,但見何太沖從左攻到,班淑嫻的長劍自右刺至,他發步一衝,搶到鮮于通的屍體之旁,假意在他懷裡一掏,轉過身來,雙掌分擊兩人。這一下用了六七成力,何氏夫婦只覺胸口窒悶。氣塞難當,不禁張口呼氣。張無忌手一揚,兩粒泥丸分别打進了兩人口中,乘看那股強烈的氣流,咽入喉中。何氏夫婦一齊咳嗽,可是已無法將丸藥吐出,心中大驚失色,眼見那物是從鮮于通身上掏將出來,想這鮮于通愛使毒藥,難道還有什麼好東西放在身上?

兩人霎時間面如土色,想起鮮于通適纔身受金蠶蠱毒的慘狀,班淑嫻幾乎便是暈倒。張無忌淡淡的道︰「這位鮮于掌門身上養有金蠶,裹在蠟丸之中,兩位均已吞了一粒。倘若急速吐出,乘看蠟丸未溶,或可有救。」到此地步,不由得何氏夫婦不驚,急運内力,嘔吐搜腸的要將「蠟丸」吐將出來。

何太仲和班淑嫻内功均佳,幾下嘔吐,便將胃中的泥丸吐了出來,這時早已成了一片混看胃液的泥沙,却那裡有蠟丸?華山派那高老者走近身來,指指點點的笑道︰「啊喲,這是金蠶糞,金蠶到了肚中,拉起屎來啦!」班淑嫻又驚又怒之下,一口氣正没處發洩,反手便是重重一掌。高老者低頭避過,逃了開去,大聲叫道︰「崑崙派的潑婦,你殺了本派掌門,華山派可跟你不能算完。」何氏夫婦聽他這麼一叫,心中更煩,暗想鮮于通雖然人品奸惡,終究是華山派掌門,自己夫婦失手將他殺了,已惹下武林中罕有的亂子,但金蠶蠱毒入肚,命在頃刻,這一切也已顧不了許多。眼前看來只有張無忌這小子能解此毒,但自己夫婦昔日如此待他,他怎肯伸手救命?

張無忌淡淡一笑,説道︰「兩位不須驚慌,金蠶雖然入肚,毒性要在六個時辰之後方始發作\dash{}晩輩了結此間大事。定當設法救你,只盼何夫人别再灌我毒酒,那就是了。」何氏夫婦大喜,雖給他輕輕的譏刺了幾句,也已不以爲意,只是道謝的言語却説不出口。訕訕的逃開。張無忌道︰「兩位去向崆峒派討四粒『玉洞黑石丹』服下,可使毒性不致立時攻心。」何太沖低聲道︰「多承指教。」即派大弟子去向崆峒派討來丹藥服下。張無忌暗暗好笑,那玉洞黑石丹固是解毒的藥物,但服後連續兩個時辰腹痛如絞,稍待片刻,何氏夫婦立即腹中大痛,只道是金蠶蠱毒發作,那料到已上了張無忌的當。不過張無忌只是小作懲戒,驚嚇他們一番而已,若説要報復前仇,豈能如此輕易?

這邉廂滅絶師太向宋遠橋叫道︰「宋大俠,六大派只剩下貴我兩派了,老尼姑女流之輩,全仗宋大俠主持全局。」宋遠橋道︰「小道已在殷教主手下輸過一陣,師太劍法通神,定能制服這個小輩。」滅絶師太冷笑一聲,拔出背上倚天劍。緩步走了出來,武當派中二俠兪蓮舟自始至終,一直注視著張無忌的動靜,對他武功之奇,深自駭異,暗想︰「滅絶師太劍法雖強,未必及得上崑崙、華山四大高手的聯手出戰,倘若她再失利,武當派又制服不了他,那麼六大派是栽到家了,我先得試一試他的虛實。」當下快步走入場中,緊了緊腰帶,説道︰「師太,讓咱們師兄弟五人先較量一下這少年的功夫,師太最後必可一戰而勝。」

他這幾句話意思説得十分明白,武當派向以内力悠長見稱,自宋遠橋以至莫聲谷,五個人一個個的跟張無忌輪流纏戰下去,縱然不勝,料想世間任何高手,也決不能連鬥武當五俠而不累得筋疲力竭,那時以強弩之末再來敵滅絶師太凌厲無倫的劍術,峨嵋派自非一戰而勝不可。滅絶師太明白他的用意,心想︰「我峨嵋派何必領你武當派這個人情?那時便算勝了,也是極不光采。難道峨嵋掌門能撿這種便宜,如此對付一個後生小輩?」她自來心高氣傲,目中無人,雖見張無忌武功了得,但想都是各派與鬥之人太過膿包所致,那日在雪地裡這小子何嘗不是給我手到擒來?後來我大舉屠戮魔教鋭金旗人衆,這小子出頭干預。又有什麼作爲?」當下衣袖一拂,説道︰「兪二俠請回!老尼倚天劍出手,不能平白無端的插回劍鞘!」兪蓮舟聽她如此説,抱拳道︰「是!」退了下去。

滅絶師太橫劍當胸,斜斜上指,走向張無忌身前,明教教衆喪生在她這倚天劍下的不計其數,這時未死的教衆見她出場,無不目眥欲裂,大聲鼓噪起來。滅絶師太冷笑,説道︰「吵什麼?待我料理了這小子,一個個來收拾你們,嫌死得不彀快麼?」殷天正知道她這柄倚天劍極是難當,本教許多高手部是未經一合,便即兵刃被她削斷,死於劍底,問道︰「曾少俠,你用什麼兵刃?」

張無忌道︰「我没兵刃。老爺子你説該當怎生對付她手中的寶劍纔好?」他親眼瞧見滅絶師太的倚天劍無堅不摧?心中可眞没有主意。殷天正緩緩抽出佩劍,説道︰「這柄白虹劍送了給你。這劍雖不如這賊尼的倚天劍有名,但也是江湖上罕見的利器。」説著伸指在劍刃上一彈,那劍陡地如軟帶般彎了過來,隨即彈直,{\upstsl{嗡}}{\upstsl{嗡}}作響,聲音甚是清越。張無忌恭恭敬敬的接了過來,説道︰「多謝老爺子。」殷天正笑道︰「這劍隨我數十年,已殺了不少奸詐小人。今日再見它飲老賊尼頸中鮮血,老夫死亦無恨。」張無忌道︰「晩輩盡力而爲。」

他提起白虹劍,轉過身來,劍尖向下,雙手抱看劍柄,向滅絶師太道︰「晩輩劍法絶非師太敵手,實是不敢和前輩放對。前輩曾對明教鋭金旗下衆位住手不殺,何不再高抬貴手?」滅絶師太的兩條長眉垂了下來,冷冷的道︰「鋭金旗的衆賊是你救的,滅絶師太手下決不饒人。你勝得我手中長劍,再來任性妄爲不遲。」明教鋭金、巨木、洪水、烈火、厚土五行旗下的教衆紛紛鼓噪,叫道︰「老賊尼,有本事就跟曾少俠肉掌過招。」

\qyh{}你劍法有什麼了不起,徒然仗著一把利劍而已。」

\qyh{}曾少俠的劍法比你高得多了,你去換一把平常長劍,若是在曾少俠手下走得了三招,算你峨嵋派高明。」

\qyh{}什麼三招?簡直一招半式也擋不住。」

滅絶師太神色木然,對這些相激的言語全然不理,朗聲道︰「進招吧!」張無忌没正式練過劍法,這時突然要他進手遞招,倒有點手足無措?想起適纔所見何太沖的兩儀劍法,招數頗爲精妙,當下斜斜刺出一劍。滅絶師太微覺詫異適︰「華山派的『峭壁斷雲』!」手中倚天劍一側,第一招便即搶攻,竟是不拆張無忌的來招,劍尖直刺他丹田要穴,出手之凌厲猛悍,世間少見。張無忌一驚,滑步相避,驀地裡滅絶師太長劍一閃,劍尖已指到了他的咽喉。張無忌大驚,百忙中臥倒打滾,待要站起,突覺後頸中涼風颯然,心知不妙,右足脚尖一撐,身子斜飛出去。這一招是從決不可能的局勢下,逃得性命,旁觀衆人待要開口喝采,却見滅絶師太飄身而上,半空中一劍上挑,不等張無忌落地,劍光已封住了他身周數尺之地。張無忌身在空中,無法避讓,在滅絶師太寶劍橫掃之下,只要身手再沉下尺許,那麼雙足齊斷,若是沉下三尺,則是齊腰斬爲兩截。

這當児眞是驚險萬分,張無忌所練成的乾坤大挪移法突生反應,不加思索的長劍一指,白虹劍的劍尖點在倚天劍劍尖之上,只見白虹劍一彎,咯的一聲爆響,劍身彈起,他身子已借力躍在空中。滅絶師太決不容情,躍步上前搶攻,{\upstsl{嗖}}{\upstsl{嗖}}{\upstsl{嗖}}連刺三劍,到第三劍上時張無忌身又下沉,只得揮劍一擋,叮的一聲,手中白虹劍只剩下半截。他順手一掌拍出,斜過來擊向滅絶師太的頭頂。滅絶師太揮劍一撩,削他手腕。張無忌瞧得奇準,伸指在倚天劍的刃面無鋒之處一彈,身子倒飛了出去。

滅絶師太手臂酸麻,虎口大震,長劍被他一彈之下幾欲脱手飛出,心頭一驚,只見張無忌落在兩丈之外,手持半截短劍,呆呆發怔。

這幾下交手,當眞是兔起鷸落,迅捷無倫,一刹那之間,滅絶師太連攻了八下快招,招招是致人死命的毒著,張無忌在劣勢之下一一化解。連續八次的死中求活、連續八次的死裡逃生。攻是攻得精巧無比,避也避得詭異之極。在這一瞬之中,人人的心都似要從胸腔中跳了出來。實不能相信適纔這幾下竟是人力所能及,攻如天神使法,閃似鬼魅變形,就像雷震電掣,雖然過去已久,兀自餘威迫人。

隔了良久良久,震天價的采聲纔不約而同的響了出來。適纔這八下快攻、八下急避,張無忌全是處於挨打的局面,手中長劍又被削斷,顯然已居下風。但滅絶師太的倚天劍被他手指一彈,登時半身酸麻,張無忌吃虧在少了對敵的經驗,若在此時乘虛反擊,已然勝了。滅絶師太心中自是有數,不由得暗自駭異,説道︰「你去換過一件兵刃,再來鬥過。」張無忌向手中斷劍望了一眼,心想︰「外公贈給我的一柄寶劍,給我一出手就毀了,實是對不起他老人家。還有什麼寶刀利刃,能擋得住倚天劍的一擊?」正自沉吟,周顚大聲道︰「我有一柄寶刀,你拿去跟老賊尼鬥一鬥。你來拿吧!」

張無忌道︰「倚天劍太過鋒鋭,只怕徒然又損了前輩的寶刀。」周顚道︰「損了便損了。你打她不過,咱們個個歸天,還保得這柄寶刀麼?」張無忌一想不錯上過去接了寶刀過來。楊逍低聲道︰「張公子,你須得跟她搶攻!可不能再挨打。」張無忌聽他叫自己爲「張公子」,微微一怔,隨即省悟,楊不悔已認出自己,當然是跟她爹爹説了,當下説道︰「多承前輩指教。」韋一笑低聲道︰「施展輕功,半步也不可停留。」張無忌大喜,又道︰「多承前輩指點。」須知光明使者楊逍、青翼蝠王韋一笑,武功之深厚,均可和滅絶師太一鬥,未必便輸於她,只恨受了圓眞的暗算,重傷之後,一身本事半點施不出來,但眼光尚在,兩個人各自指點了一個関鍵所在,正是對付滅絶師太的重要訣竅。

張無忌提刀在手,覺得這柄刀重約四十餘斤,烏沉沉的雖不起眼,但式樣古樸,顯是一件歷時已久的珍品,心想毀了白虹劍雖然可惜,終究是外公已經送了給我的兵刃,這把寶刀却是周顚之物,可不能再在自己手中給毀了,當下輕輕吸一口氣,説道︰「師太,晩輩進招了!」展開輕功,如一溜煙般繞到了滅絶師太身後,不待她回身,左一閃,右一趨,正轉一圏,反轉一圏,刷刷兩刀砍出。滅絶師太橫劍一封,正要遞劍出招,張無忌早已瞧得不知去向。他在未練乾坤大挪移法之時,輕功已比滅絶師太爲高,這時越奔越快,如風如火,似雷若電,連韋一笑這等素以輕功睥睨群雄之人,竟也自愧不如。但見他一個人影四下轉動,迫近身去便是一刀,招術未曾用老,已然避開。這一次攻守異勢,滅絶師太竟無反擊一劍的機會,只是張無忌礙於她倚天劍的鋒鋭,却也不敢過份逼近。他内力渾厚,奔到數十個圏子後,體内九陽神功發動,更似不點地的凌空飛行一般。峨嵋群弟子一看不對,如此纏鬥下去,師父是要吃虧。丁敏君叫道︰「今日的局面是剿滅魔教,可不是比武爭勝。衆位師姊妹,大家一齊上,攔住這小子,教他不得取巧,乖乖的跟師父較量眞實本領。」説看提劍躍出。峨嵋派男女弟子誰也不肯後人,手執兵刃,佔住了八面方位。周芷若站在西南角上,丁敏君冷笑道︰「周師妹?攔不攔在你?讓不讓也在你。」周芷若又氣又羞,道︰「你提我幹什麼?」

便在此時,張無忌已衝到了跟前,丁敏君嗤的一劍刺出。張無忌左手一伸,挾手將她長劍奪了過來,順手便向滅絶師太擲去。滅絶師太一劍將那長劍斬爲兩截,但張無忌一擲之力強勁之極,來劍雖斷,這勁力仍將滅絶師太的手腕震得隱隱發麻。張無忌身子更不停留,左手隨伸隨奪、隨奪隨擲,峨嵋群弟子此次來西域的無一不是派中高手,但一遇到張無忌伸手奪劍,竟是閃避的餘地也没有,給他手到拿來,數十柄長劍飛舞空中,白光閃閃,接二連三的向滅絶師太飛去。滅絶師太臉如嚴霜,將來劍一一削斷,削到後來。右臂大是酸痛,當即劍交左手。

滅稱師太左手使劍的本事,和右手使劍無甚分别,但見半空中斷劍飛舞,有的旁擊向外,兀自勁力奇大,圍觀的衆人紛紛後退。片刻之間,峨嵋群弟子個個空手,却只周芷若手中的長劍没有被奪。

在張無忌是報她適纔指點之德,豈知這麼一來,却把周芷若顯得十分突出。她心思周密,早知不妥,想要搶到張無忌身前攻擊數招,但張無忌身法實在太快,何況是故意避開了她,不近她身子五尺之内。周芷若雙頰暈紅,一時手足無措,丁敏君已冷笑道︰「周師妹,他果然待你與衆不同。」這時張無忌雖受峨嵋群弟子之阻,但穿來插去,將衆人視如無物,刀刀往滅絶師太要害招呼。滅絶師太變成了只有挨打,無法反擊的局面,心中暗暗焦急,丁敏君的言語却一聲聲傳入耳中︰「你眼看師父受這小子急攻。怎地不上前相助?你提看兵刃呆呆的站在一旁,只怕你心中,還是盼望這小子打勝師父呢。」滅絶師太心念一動︰「何以這小子偏偏留下芷若的兵刃不奪,莫非兩人當眞暗中勾結?我一試便知!」朗聲喝道︰「芷若,妳救欺師滅祖麼?」劍隨身出一劍向周芷若當胸刺了過去。

周芷若大驚,不敢舉劍擋架,叫道︰「師父,我\dash{}」但在這電光石火般的一瞬之間?那有時間分辯,她這「我」字剛出口,滅絶師太的長劍已刺到胸口。張無忌不知這一劍乃是試探是否眞有情弊,待得劍到胸膛,滅絶師太自會縮手,不致當眞刺下。他親眼目睹,見過滅絶師太殺死紀曉芙的狠辣,知道此人誅殺徒児,決不容情,當下不及細想,縱身躍上,一把抱起周芷若?飛出數尺,滅絶師太好容易反賓爲主,長劍顫動,直刺他的後心。張無忌旨在救人,脚下不免慢了一步,只得回刀一揮,{\upstsl{噹}}的一響,手中寶刀又是斷去半截。滅絶師太的長劍跟蹤刺到,無忌反手運勁,擲出半截寶刀,這一下用了九成力,滅絶師太登時氣息一窒,不敢舉劍撩削,伏地閃避。

那半截寶刀從她頭頂掠過,勁風刮得她隱隱生疼。無忌一見有機可乘,不及放下周芷若,隨即搶身而進,右手一探,狠狠一掌拍出。滅絶師太一膝跪地,舉劍削他手腕,張無忌變拍爲拿,反手勾處,已將椅天劍奪到了手中。這種化剛爲柔的急劇轉折,已屬乾坤大挪移第七層神功,滅絶師太武功雖高,却也萬萬料想不到。

張無忌雖然一招得勝,但對滅絶師太這類大敵,實是戒懼極深,絲毫不敢怠忽,以倚天劍指住她咽喉,生怕她又有奇招使出?慢慢的退開兩步。周芷若身子一掙,道︰「快放下我!」張無忌驚道︰「呀,是!」滿臉漲得通紅,忙將周芷若放下,鼻中聞到一陣淡淡幽香,只覺她頭上柔絲在自己左頰拂過,不禁斜望了她一眼,只見周芷若也是俏臉生暈,又羞又窘,臉上雖是充滿恐懼的神色,眼光中却不免流露出歡喜之意。

滅絶師太緩緩站直身子,一言不發,瞧瞧周芷若,又瞧瞧張無忌,臉色越來越是鐵青。張無忌倒轉劍柄,向周芷若道︰「周姑娘,貴派的寶劍,請你轉交尊師。」周芷若望向師父,只見她神色漠然,既非許可,亦非不准,一刹間心中轉過了無數念頭,知道今日的局面已是{\upstsl{尷}}尬無比,張無忌如此對待自己,師父必當我無私有弊,從此我便成了峨嵋派的棄徒,成爲武林中所不齒的叛逆。「難道我決心背叛峨嵋派麼?大地茫茫,教我到何處去覓歸宿之地?這少年對我不錯,可是他終究是妖邪一派,我却不是存心爲了他而背叛師門。」忽聽得滅絶師太厲聲喝道︰「周芷若,一劍將他殺了。」

\chapter{倚天寶劍}

當年周芷若跟張三丰赴武當山,張三丰以武當山上並無女子,一切諸多不便,當下揮函轉介,投入滅絶師太門下。她天資極是聰穎,又以自幼慘遭父母雙亡的大變,刻苦學藝,進步神速,深得師父的鍾愛,這八年之中,她没離開師父一步,滅絶師太的一言一動,於她便如是天經地義一般,心中從未生過半點違拗的念頭,這時聽到師父驀地一聲大喝,倉卒間無法細想,手起一劍,便向張無忌胸口刺了過去。

張無忌却決計不信她竟會向自己下手,絲毫没有閃避,一瞬之間,劍尖已到胸口,他一驚之下,待要躱讓,却已不及。周芷若手腕發抖,心想︰「難道我便刺死了他?」迷迷糊糊之中,長劍略偏,嗤的一聲輕響,倚天劍已從張無忌的右胸透入。

周芷若一聲驚叫,拔出長劍,只見劍尖殷紅一片,張無忌右胸處鮮血有如泉湧,四圍驚呼之聲大作。張無忌伸手按住傷口,身子搖晃,臉上神色極是古怪,似乎在問︰「你眞的要刺死我?」周芷若道︰「我\dash{}我\dash{}」想過去察看他的傷口,但終於不敢,掩面奔回。她這一劍竟然得手,誰都出於意料之外。小昭臉如土色,搶上前來,扶住了張無忌,道︰「張公子,你\dash{}你\dash{}」這一劍幸好稍偏,没刺中心臟,但已重傷右邉肺葉,張無忌對小昭道︰「你爲什麼要殺我\dash{}」説了這幾個字,肺中吸不進氣,彎腰劇烈咳嗽,他重傷之下,瞧出來分不清小昭和周芷若,鮮血{\upstsl{汩}}{\upstsl{汩}}流出,將小昭的上衣染得紅了半邉。

旁觀衆人不論是六大派或明教、白眉教的人衆,一時均是肅靜無聲。張無忌適纔連敗各派高手,武功高強,胸襟寬博,不論是友是敵,無不暗暗敬仰,這時見他無端端的被周芷若刺了一劍,心下均感不忿,眼見倚天劍透胸而入,傷勢極重,都関心這一劍是否致命。

小昭扶著他緩緩坐下,朗聲説道︰「那一位有最好的金創藥?」少林派中神僧空性快步而出,從懷中取出一包藥粉,説道︰「敝派玉靈散是傷科聖藥。」伸手撕開無忌胸前衣服,只見那傷口深及數寸,忙將玉靈散敷上去時,鮮血湧出,將藥粉都沖開了。空性大感束手無策,道︰「怎麼辦?怎麼辦?」崑崙派的何太沖夫婦更是焦急,他們只道已服下金蠶蠱毒,張無忌若是重傷而死,他們解毒無人,也是活不成了。何太沖搶到張無忌身前,急問︰「金蠶蠱毒怎生解救,快説︰快説啊。」

小昭哭道︰「走開!你忙什麼?張公子若是不活,大家左右是個死。」若在平時,何太沖是何等身份?怎能受一個青衣小婢的呼叱?但這時情急之下,仍是没口的急問︰「金蠶蠱毒怎生解救?」空性怒道︰「鐵琴先生,你再不走開,老納可要對你不客氣了。」便在此時,張無忌睜開眼來,微一凝神,伸左手食指在自己傷口周圍點了七處穴道,血流登時緩了。空性大喜,便即將玉靈散替他敷上。小昭撕下衣襟,替他裹好傷口,眼見張無忌臉白如紙,竟無半點血色,心中説不出的焦急害怕。

張無忌這時神智已略清醒,暗運内息流轉,只覺通到右胸便即阻塞,心中只想︰「我待教有一口氣息尚在,不能讓六大派將明教衆人盡數殺死!」當下將眞氣在左邉胸腹間運轉數次,緩緩站起身來,説道︰「峨嵋、武當兩派若有那一位不服在下的調處,可請出來較量。」他此言一出,衆人無不駭然,眼見周芷若這一劍刺得他如此厲害,竟然兀自挑戰。

滅絶師太冷冷的道︰「峨嵋派今日已然落敗,你若不死,日後再行算帳。咱們瞧武當派的吧!六大派此行的成敗,全仗武當派裁決。」這幾句話説得再也明白不過。

六大派圍攻光明頂,少林、崆峒、崑崙、華山、峨嵋五派的高手均已敗在張無忌手下,只餘下武當一派尚未跟他交過手。這時他身受劍傷,死多活少,别説一流高手,只須是幾個庸手跟他糾纏一番,他也是支持不住了,甚至無人和他對敵,只怕稍等片刻,他也會傷發而斃,武當五俠任誰一位上前,均可將他擊死,然後照原來策劃?誅滅明教。只是武當派自來極重「俠義」兩宇,要他們出手對付一個身負重傷的少年,未免於聲名大有損害,恐怕武當五俠誰都不願,但武當派若不出手,難道「六大派圍攻光明頂」這一件轟傳武林的大事,竟然鬧一個鎩羽而歸?此後六大派在江湖上臉面何存?其中的抉擇,可實在爲難之極了。滅絶師太説哪幾句話、意思説六大派今後是榮是辱,全憑武當派決定,且看武當派是否有人肯顧全大局,損及個人的名望。

宋遠橋、兪蓮舟、張松溪,殷利亨、莫聲谷五人面面相覷,誰都拿不出主意。宋遠橋的児子宋青書突然説道︰「爹,四位叔叔,讓孩児去料理了他。」武當五俠明白他的意思,他是武當派的晩輩,由他出手,勝於累及武當五俠的英名。兪蓮舟道︰「不成!咱們許你出手,跟咱們親自出手並無分别。」張松溪道︰「依小兄之見,大局爲重,我五兄弟的名聲爲輕。」莫聲谷道︰「名聲乃身外之物,只是如此對付一個重傷少年,良心難安。」一時議論難決,各人眼望宋遠橋,聽他示下。宋遠橋見殷利亨始終不發一言,可是臉上憤怒之色難平,心知他未婚妻紀曉芙失身於明教楊逍,以致身死,實是生平奇恥大恨,若不一鼓將明教誅滅,掃盡奸惡淫徒,這口氣如何嚥得下去,當下緩緩説道︰「魔教作惡多端,除惡務盡,乃是我輩俠義道的大節。兩者不能兼得之際,當取大者。青書,一切小心。」

宋青書躬身道︰「是!」走到張無忌身前。朗聲道,「曾小俠,你若非明教中人,儘可離開,自行下山養傷。六大派只誅魔教邪徒,與你無涉。」張無忌按住胸前傷口,輕輕的道︰「大丈夫急人之難,死而後已。多謝宋兄好意,可是在下與明教共存共亡!」明教和白眉教人衆紛紛高叫︰「曾少俠,你待咱們已然情至義盡,到此地步,不必再鬥。」殷天正脚步蹣跚的走近,説道︰「姓宋的,老夫再接你的高招!」那知一口氣提不上來,脚下一軟,又摔倒在地。

宋青書眼望張無忌,説道︰「曾兄,既然如此,小弟礙於大局,可要得罪了。」小昭擋在無忌身前,叫道︰「那你先殺了我再説。」無忌低聲道︰「小昭?你别擔心,這少年本領平常?我對付他綽綽有餘。」小昭急得道︰「張公子,你\dash{}身上有傷啊。」無忌微笑道︰「不怕!」宋青書聽他説「這少年本領平常,我對付他綽綽有餘」這兩句話,不禁大怒,厲聲道︰「好!在下本領平常,領教你的『綽綽有餘』!」張無忌柔聲向小昭道︰「小昭!你爲什麼待我這樣好?」小昭知道眼前的局面已無法挽回,霍地站起,淒然道︰「反正我獨個児也不會活著。」張無忌向她凝視半晌,眼光中充滿了無限的柔情蜜憶,心想︰「就算我此時死了,也有了一個眞正待我極好的知己。」

宋青書向小昭喝道︰「你走開些!」張無忌道︰「你對這位小姑娘粗聲大氣,忒也無禮!」宋青書在小昭肩頭一推,將他推開數步,説道︰「妖女邪男,那有什麼好東西!快站起來,接招吧!」張無忌道︰「你父親乃是謙謙君子,閣下年少氣盛,却是這等粗暴!跟你動手,還用得著我站起來麼?」他嘴裡這麼説,實則内勁提不上來,自知決計無力站起。

張無忌重傷後虛弱無力的情形,許多人都瞧了出來,宋青書如何會不知道,只聽兪蓮舟朗聲説道︰「青書,點了他的穴道,令他動彈不得,也就是了,不必傷他性命。」宋青書道︰「是!」左手虛引,右手倏出一招,往張無忌肩頭點來。張無忌動也不動,待他手指點到肩上的「肩貞穴」時,勁力一卸,宋青書這一指之力猶似戳入了水中,更無半點著力之處,只因出其不意,身子向前一衝,險險撞到張無忌身上,急忙站定,却已不免有點狼狽。

宋青書定了定神,飛起一脚,逕往他的胸口踢去,這一脚已使了六七成力。兪蓮舟雖叫他不可傷了張無忌的性命,但不知怎的,宋青書心中對無忌隱隱蓄著極深的恨意,這倒不是無忌説他武功平常之故,却是因見周芷若瞧看張無忌的眼光之中,一直含情脈脈,極是関懷。最後雖是奉了師父之命而刺他一劍,但顯而易見,她心中難受異常。宋青書自見到周芷若後,一雙眼光,極難有片刻離開她身上,雖然常自強制自己,不可多看,以免被人認爲是輕薄之徒,但周芷若的一舉一動、一顰一笑,無不被他瞧得清清楚楚,只是她刺劍之後,更是黛眉緊蹙,哀傷欲絶。宋青書隱隱知道︰「這一劍刺了之後,不論張無忌這小子是死也好,活也好?再也不能從她心上抹去了。」他明知自己倘若擊死了張無忌,周芷若必定深責自己,可是心頭妬火中燒,却又不肯放過這唯一制他死命的良機。宋青書本來文武雙全,乃是武當派第三代弟子中出類拔萃的人物,人品也是端方正直,但一遇到這「情」之一関,竟是方寸大亂。

衆人眼見宋青書這一腿踢去,張無忌若非躍起相避,只有出掌硬接,那知足尖將要踢到他的胸口,張無忌左手五指輕輕一拂,宋青書這一腿之力竟然轉向,從他身側斜了過去,相距不過三寸。但就是差了這麼三寸,這一腿全然踢了個空,宋青書在勢已無法收腿,跟著跨了一步,左足足跟後撞,直攻張無忌的背心,這一招既快且狠,人所難料,原是極高明的招數,但張無忌手指一拂,又已將他的撞擊卸開。

三招一過,旁觀衆人無不大奇。宋遠橋叫道︰「青書,他本身已無半點勁力,這是四兩撥千斤之法。」究竟宋遠橋眼光老辣,瞧出張無忌此時勁力全失,所用的功夫雖然叫做「乾坤大挪移」,其基本道理,却與中原武學中「四兩撥千斤」的「借力打力」並無二致。宋青書被父親一語提醒,招數忽變,雙掌輕飄飄地,若有若無的拍擊而出,正是武當絶學之一的「綿掌」。要知「借力打力」,原是武當派武功的根本,所謂「四兩撥千斤」,須得對方出力千斤,方能借勁運勁,這時他所使的「綿掌」,本身的勁力就是在若有若無之間,叫張無忌想借力也無從借起。不料他綿掌一招招的打出,張無忌的「乾坤大挪移」神功已練到至高無上的第七層境界,别説綿掌雖輕,終究是有形之物,便是傷人於無形的毒氣怪聲,他也能隨意化解,但見他閉目微笑,左手五指猶如撫琴鼓瑟,忽挑忽撚,忽彈忽撥,上身半點不動,片刻間將宋青的三十六招綿掌掌力盡數卸了。

宋青書心中大駭,偶一回頭,突然和周芷若的目光相接,只見她滿臉関懷之色,不禁心中又酸又怒,知道她関懷的決不是自己,深深吸一口氣,左手一掌猛擊張無忌右頰,右手一指便點他在肩後的「魄戸穴」,這一招叫做「花開並蒂」,名稱好聽,招數却是厲害,雙手遞招之後,跟著右手一掌擊他左頰,左手食指却疾點他左肩後的「風戸穴」。這兩招「花開並蒂」併成一招,連續四式,便如暴風驟雨般使出,勢道之猛,手法之快,眞是非同小可。衆人見了這等聲勢,「啊」的齊聲驚呼,不約而同的跨上一步。只聽得拍拍兩下清脆的響聲,宋青書左手一掌打在自己左頰之上,右手一掌打在自己右頰之上,同時一指點中了自己「魄戸穴」,另一指點中了自己「風戸穴」。他這招「花開並蒂」四式齊中,却給張無忌以「乾坤大挪移」中最神妙的功夫,挪移到了他自己身上。倘若宋青書出招稍慢,那麼自己點中了「魄戸穴」後,以後兩式便即無力使出,偏生他四式連環,迅捷無倫,「魄戸穴」雖被點中,手臂尚未麻木,直到使了第二套「花開並蒂」之後,這纔手足酸軟,砰的一聲,仰天摔倒,掙扎了幾下,却再也站不起來了。

宋遠橋快步搶出,左手推拿幾下,已解開了児子的穴道,但見他兩邉面頰高高腫起,每一邉留下五個烏青的指印,知他受傷雖輕,但這児子心高氣傲,今日當衆受此大辱,直比殺了他還要難受,當下一言不發,擕了他手回歸本派。

這時四周喝采之聲,此起彼落,議論讚美的言語,嘈雜盈耳,突然間張無忌口一張,噴出幾口鮮血,按住傷口,又咳嗽起來。衆人凝視看他,極爲関懷,人人均想︰他重傷下抵禦宋青書的急攻,雖然得勝,但内力損耗必大。有的人看看張無忌,又望望武當派衆人,不知他們就此認輸呢,還是另行派人出鬥。

宋遠僑道︰「今日之事,武當派已然盡力,想是魔教氣數未盡,上天生下這個奇怪少年來。若再纏鬥不休?名門正派和魔教又有什麼分别?」兪蓮舟道︰「大哥説得是。咱們即日回山,請師父指點。日後武當派捲土重來,待這少年傷愈之後。再決勝負。」他這幾句話説得光明磊落,豪氣逼入,今日雖然認輸,但不信武當派終究會技不如人。張松溪和莫聲谷齊道︰「正該如此!」忽聽得刷的一聲,殷利亨長劍出鞘,雙眼泪光瑩瑩,大踏步走出去,劍尖對看張無忌,説道︰「姓曾的,我和你無冤無仇,此刻再來傷你,我殷利亨枉稱這『俠義』兩字。可是那楊逍和我仇深似海,我是非殺他不可,你讓開吧!」張無忌搖頭道︰「但教我有一口氣在,不容你們殺明教一人。」殷利亨道︰「那我可先得殺了你!」張無忌噴出一口鮮血,神智昏迷,心情激盪,輕輕的道︰「殷六叔,你動手吧!」

殷利亨聽到「殷六叔」三字,只覺語氣極爲熟悉,心念一動︰「無忌幼小之時,常常這樣叫我,這少年\dash{}」凝視他的面容,竟是越看越像,雖然分别了八年,張無忌已自一個小小孩童成長爲壯健少年,加之鬍鬚不剃。長髮未理,相貌已是大異。但殷利亨心中先存下「難道他竟是無忌」這個念頭,細看之下,記憶中的面貌一點點地顯現出來,不禁顫聲道,「你\dash{}你是無忌麼?」張無忌全身再無半點力氣,自知去死不遠,再也不必隱瞞。叫道︰「殷六叔,我\dash{}我常常在想念你。」殷利亨自來是情感極爲充沛之人,雙目流泪,{\upstsl{噹}}的一聲抛下長劍,俯身將他抱了起來,叫道︰「你是無忌,你是無忌,你是我五哥的児子張無忌!」宋遠橋、兪蓮舟、張松溪、莫聲谷四人一齊圍攏,各人又驚又喜,心頭均是説不出的滋味。殷利亨這麼一叫,旁邉衆人無不驚訝,那想到這個捨命力護明教的少年,竟是武當派張翠山的児子。

殷利亨見無忌昏暈了過去,忙摸出一粒「天王護心丹」,塞入他的口中,將他交給兪蓮舟抱著,抬起長劍,衝到楊逍身前,戟指罵道︰「姓楊的,你這豬狗不如的淫徒,我\dash{}我\dash{}」喉頭便住,再也罵不下去,一劍遞出,便要往楊逍心口刺去。楊逍絲毫不能動彈,微微一笑,閉目待斃。突然斜刺裡奔過來一個少女,擋在楊逍身前,叫道︰「休傷我爹爹!」

殷利亨凝劍不前,定睛一看,不禁「啊」的一聲,全身冰冷,只見這少女長挑身材、秀眉大眼,一模一樣是當年紀曉芙的形貌。他自和紀曉芙定親之後,每當練武有暇,心頭甜甜的,總是想看未婚妻的俏麗倩影,及後得知紀蹺芙爲明教光明使者楊逍擄去,失身於他,更且因而斃命,心中之憤恨,自是難以言宣。此刻突然見到紀曉芙重新出現,身子一晃,失聲叫道︰「曉芙妹子,你\dash{}你\dash{}」那少女却是楊不悔,説道︰「我姓楊,紀曉芙是我媽媽,她早已死了。」

殷利亨呆了一呆,這纔明白,喃喃的道︰「啊?是了,我眞胡塗!你讓開,我今日要替你媽報仇雪恨。」楊不悔指著滅絶師太道︰「好!殷叔叔,你去殺了這個老賊尼。」殷利亨道︰「爲\dash{}爲什麼?」楊不悔道︰「我媽是給這名賊尼一掌打死的。」殷利亨道︰「胡説八道,你小孩子家懂得什麼?」楊不悔冷冷的道︰「那是在蝴蝶谷中,老賊尼叫我媽來刺死我爹爹,我媽不肯,老賊尼就一掌將我媽打死了。我親眼瞧見的,張無忌哥哥也是親眼瞧見的。你再不信,不妨問問那老賊尼自己。」當紀蹺芙身死之時,楊不悔年幼,什麼也不懂得,但後來年紀大了,慢慢回想,自然明白了當年的經過。

殷利亨回過頭去,望著滅絶師太,眼中露出疑問之色,道︰「師太\dash{}她説\dash{}紀姑娘是\dash{}」滅絶師太嘶啞著嗓子,説道︰「不錯,這等不知廉恥的孼徒,留在世上又有何用?她和楊逍是兩廂情願,寧肯背叛師門,不願遵奉師命,去刺殺這個淫徒惡賊。殷六俠,爲了顧全你的顏面,我始終隱忍不言。哼,這等無恥的女子,你何必念念不忘於她?」殷利亨鐵青著臉,大聲道,「我不信,我不信!」滅絶師太道︰「你問問這女孩子,他叫什麼名字?」殷利亨的目光轉到楊不悔臉上,泪眼模糊之中,瞧出來活脱便是一個紀曉芙,耳中却聽她清清楚楚的,説道︰「我叫楊不悔。媽媽説︰這件事她永遠也不後悔。」{\upstsl{噹}}的一聲,殷利亨擲下長劍,回過身來,雙手掩面,疾衝下山。宋遠橋和兪蓮舟大叫︰「六弟,六弟」但殷利亨既不答應,亦不回頭,提氣急奔,突然間失足摔了一交,但爬起身來,片刻間奔得不見了蹤影。他和紀曉芙之事江湖上多有知聞,眼見事隔十餘年,他仍是如此傷心不由得都替他難過。要知以武當六俠殷利亨的武功,奔跑之際如何會失足摔跌?那自是心神大亂、魂不守舍之故了。

這時宋遠橋、兪蓮舟、張松溪、莫聲谷四人分坐四角,各出一掌,抵在張無忌胸、腹、背、腰四處大穴之上,齊運内力?給他療傷。四人内力甫施,立時覺得無忌體内有一股極強的吸力,源源不絶的將四人内力吸引過去。四人一齊大驚,暗想如此不住吸去,只須一兩個時辰,自己内力便致耗竭無存,但無忌生死未卜,那便如何是好?正没做理會處,張無忌緩緩睜開眼睛,「啊」了一聲。宋遠橋等心頭一震猛,覺得手掌心有一股極暖和的熱力,反傳過來,竟是無忌的九陽神功起了應和,轉將内力反輸到四人體内,宋遠橋叫道︰「使不得!你自己靜養要緊。」四個人急忙撤掌而起,但覺似有一片滾水周流四肢百骸,舒適無比,顯是無忌不但將吸去的内力還了四人,而且他體内九陽眞氣充盈鼓盪,反而幫助四人增強了内功的修爲。宋遠橋等四人面面相覷,暗自震駭,眼見他重傷垂死,那知内力竟是如此強勁渾厚,沛不可當。

此刻張無忌外傷尚重,内息却已運轉自如,慢慢站起身來,説道︰「宋大伯、兪二伯、張四伯。莫七叔,恕侄児無禮!太師父他老人家福體安康。」

宋逮橋道︰「師父他老人家安好!無忌,你\dash{}你長得這麼大了\dash{}」説了這句話,心頭雖有千言萬語,却再也説不下去了。白眉鷹王殷天正見這個救了自己性命的少年竟是自己外孫,高興得呵呵大笑,却終究站不起身。

滅絶師太鐵青著臉,將手一揮,峨嵋群弟子跟著她向山下走去。周芷若低著頭走了幾步,終於忍不住向張無忌一望,張無忌却也正目送著她離去。兩人目光相接,周芷若蒼白的臉頰上飛上一陣紅暈,眼光中似乎是説︰「我刺得你如此重傷?眞是萬分的過意不去,你可要好好保重。」張無忌好像明白了她的意思,微微點了點頭。周芷若秀眉上揚,心中十分喜歡,隨即回過頭去,加快脚步,遠遠去了。這一切全瞧在宋青書的眼中,他目光中閃耀著幾星刻毒的恨意,但一瞬即過,誰也没有見到。

武當派和張無忌相認,再加峨嵋派這一去,六大派圍剿魔教之舉登時風流雲散,崆峒和華山兩派跟著作别。何太沖走近身來,説道︰「小兄弟,恭喜你們親人相認啊\dash{}」張無忌不等他接著説下去,從懷中摸出兩枚避瘴氣、去穢惡的尋常藥丸,遞了給他,説道︰「請賢夫婦各服一丸,金蠶蠱毒便可消解。」何太沖拿著這兩粒藥丸,但見黑黝黝的毫不起眼,不信便能消解得那天下至毒的金蠶蠱毒。張無忌道︰「在下既説消解得,便是消解得。」他説話聲音雖然微弱,但光明頂這一戰鎭懾六大門派氣度之中,自然而然生出一種威嚴,不由得何太沖不信。他心想︰「即使他騙我?所給的藥不能消解蠱毒,當著武當四俠之前,我也不能強逼他給眞藥於我。何況少林派的那幾個賊禿似乎也有迴護這小賊之意。今日只好認命罷喇。」當下苦笑著説聲︰「多謝」微一稽首,和班淑嫻分别將藥丸服下,指揮衆弟子收拾本派死者的屍首,告辭下山。

兪蓮舟道︰「無忌,你傷重不能行走下山,只好在此調養,咱們可又不能在此陪你,盼你痊癒之後,來武當一行,也好讓師父見了你喜歡。」張無忌含泪點頭。各人有許多事想問、有許多話想説,但見無忌神情委頓,均知多説一句話便是加重他一分傷勢,只有忍住不言。猛聽得少林派中一人大聲叫了起來︰「圓眞師兄的屍首呢?」另一人道︰「咦,怎麼不見了圓眞師伯的法體?」莫聲谷好奇心起,搶步過去一看,只見七八名少林僧在收拾本門戰死者的遺屍,可是單單少了圓眞的一具屍體。

圓音指著明教教衆,大聲喝遭︰「快把我圓眞師兄的法體交出來,莫惹得和尚無名火起,一把火燒得你們個個屍骨成灰。」周顚笑道︰「哈哈,哈哈!眞是笑話奇談!你這活賊咱們也不要,要這死和尚幹麼?拿他當豬當羊,宰來吃他的瘦骨頭麼?」少林人衆一想倒也不錯,當下十餘個僧人四出搜索,却那裡有圓眞的屍身。衆人雖覺奇怪,但想多半是華山、崆峒各派收取本門死者的屍身之時,誤將圓眞的屍身收了去。當下少林、武當兩派人衆連袂下山,張無忌上前幾步,躬身相送。宋遠橋道︰「無忌孩児,今日一戰,你是名揚天下,對明教更是恩重如山。盼你以後多所規勸引導,總當使明教改邪歸正,少作些壞事。」無忌道︰「孩児遵奉師伯教誨,自當盡力而爲。」張松溪道︰「你一切小心在意,事事提防奸惡小人!」無忌又應道︰「是!」他和宋遠橋等久别重逢,又即分離,十分的依依不捨。

楊逍和殷天正待六大派人衆走後,兩人對望一眼,齊聲説道︰「明教和白眉教全體教衆,叩謝張大俠護教救命的大恩!」頃刻之間,黑壓壓的人衆跪滿了一地。

張無忌見人人行此大禮,不由得慌了手脚,何況其中尚有外公、舅舅諸人在内,急忙跪下還禮,那知他這一急跪,胸口劍傷破裂,幾口鮮血噴出,登時暈了過去。小昭搶上扶起,明教中兩個没有受傷的頭目抬過一張軟床,扶他睡在床上。楊逍皺了眉頭,説道︰「快扶張大俠到我房中靜養,這幾天中,誰也不能去驚動於他。」那兩名頭目躬身答應,將張無忌抬入楊逍房中。小昭跟隨在後。經過楊不悔身前時,楊不悔冷冷的道︰「小昭!你裝得眞像,我早知你必有古怪,只是没料到這麼一個醜八怪竟是一位千嬌百媚的小美人児。」小昭低頭不語,叮叮{\upstsl{噹}}{\upstsl{噹}}的拖著鐵鏈,緊跟在張無忌身後。

這幾天中,明教教衆救死扶傷,忙碌不堪。經過這場從地獄邉逃回來的大戰,各人心中都明白了以往自相殘殺、以致召來外侮的不該。人人関懷著張無忌的傷勢,誰也不提舊怨,安安靜靜的耽在光明頂上養傷。張無忌九陽神功已成,周芷若刺他這一劍雖然厲害。但只因劍尖透入時偏了數寸,只傷及肺葉,未中心臟,因此靜養了七八天,傷口漸漸愈合。殷天正、楊逍、韋一笑、説不得等人躺在軟床之中,每天由人抬進房來探視,見他一天好似一天,都是極爲欣慰。

到第八天上。張無忌已可坐起。那天晩上,楊逍和韋一笑又來房中探病,張無忌道︰「兩位身中一陰指後,這幾天覺得怎樣?」楊韋二人每日都要苦熬刺骨之寒的折磨,傷勢只有越來越重,但怕無忌掛懷,都道︰「好得多了!」張無忌見二人臉上黑氣籠罩,説話也是有氣無力,説道︰「我内力已回復了六七成,便替兩位治一治看。」楊逍忙道︰「不,不!張大俠何必忙在一時?待你貴體痊癒?再替咱們醫治不遲。此刻用力早了,傷勢若有反覆,咱們心中何安?」韋一笑道︰「早醫晩醫,那也不爭在這幾日。張大俠靜養貴體要緊。」張無忌道︰「我義父當年和兩位平輩論交,兩位都是我的長輩,再稱『大俠』什麼,侄児可實在不敢答應。」楊逍微笑道,「將來咱們都是你的屬下,在你跟前,連坐也不敢坐,還説什麼長輩平輩?」張無忌一怔,問道︰「楊伯伯你説什麼?」韋一笑道︰「張大俠,這明教教主的重任,除了你來承當之外,那裡還有旁人?」

張無忌雙手亂搖,道︰「此事萬萬不可!萬萬不可!」便在此時,忽聽得東面遠遠傳來一陣陣尖利的哨子之聲,正是光明頂山下有警的訊號。楊逍和韋一笑微微一怔,心中均想︰「難道六大派輸得不服,去而復返麼?」但兩人都是第一流的高手,臉上絲毫不動聲色。楊逍又道︰「昨天吃的人參還好麼?小昭,你再到藥室去取些,給張大俠煎湯喝。」只聽西面、南面同時哨子聲大作。張無忌道︰「是有外敵來攻麼?」韋一笑道︰「本教和白眉教不乏好手,張大俠不必掛心,諒小小幾個毛賊,何足道哉!」

可是片刻之間,哨子聲已在半山間響起。那敵人來得好快,顯然不是小小毛賊。楊逍笑道︰「我出去安排一下,韋兄便在這裡陪著張大俠。嘿嘿,明教難道一蹶不振?變成人人可欺的膿包了。」他雖傷得動彈不得,但言語中仍是充滿著豪氣。張無忌暗自尋思︰「少林、武當這些名門正派,決不會不顧信義,重來尋仇。來者只怕多半是殘忍奸惡之輩。光明頂上所有高手人人重傷,這七八天中,没一人能將傷勢養好,不論外敵是強是弱,咱們都無法抵擋。倘若強自出戰,只有人人送命。」突然間門外脚步聲急,一個人闖了進來,滿臉血汚,胸口插看一柄短刀。

\chapter{秘道避禍}

那人衝進室來,叫道︰「敵人從三面\dash{}攻上山來\dash{}弟兄們抵敵\dash{}不住\dash{}韋一笑問道︰「什麼敵人?」那人手指室外,想要説話,但身子向前一俯,就此死去。但聽得傳警呼援的哨聲,此起彼落,顯是情勢極爲急迫。突然又有兩個人奔進室來,楊逍認得當先一人是洪水旗的掌旗副使,只見他一條右臂齊肩斬斷,臉色猶如鬼魅,後面那人也是全身浴血。那掌旗副使雖然身受重傷,仍是十分鎭定,微微躬身,稟道︰「張大俠、楊左使、韋法王,山下來攻的是巨鯨幫、海沙派、神拳門各路人物。」

楊逍長眉一軒,「哼」的一聲道︰「這些么魔小醜,也欺上門來了嗎?」那掌旗副使道︰「領頭的是個西域番僧,武功甚強,他持著倚天寶劍\dash{}」張無忌等三人聽到「倚天寶劍」四字,一齊「啊」了一聲。楊逍道︰「眞是倚天寶劍,你没瞧錯麼?」那副使道︰「這位王兄弟在我身旁執著火把,我是瞧得清清楚楚的。那番僧將我的鬼頭刀和右臂一劍削斷,我還看到劍刃上的『倚天』兩字,決計錯不了。」

他説到這裡,冷謙、鐵冠道人張沖、彭瑩玉、説不得、周顚等五散人分别由人抬了進來。只見周顚氣呼呼的大叫︰「好丐幫,勾結了三江幫、巫山幫來乘火打劫,我周顚只要有一口氣在,跟他們永世没完\dash{}」他話猶未了,殷天正,殷野王父子撐著木杖,走進室來。殷天正道︰「無忌孩児,你睡著别動,他媽的『五鳳刀』和『斷魂槍』這兩個小小門派,諒他們能把咱們怎樣?」

楊逍一聽,心想︰「這次來攻光明頂的,大大小小的幫會門派,著實不在少數。恨只恨咱們個個動彈不得。」這些人中,楊逍在明教中位望最尊、殷天正是白眉教的教主、彭瑩玉最富智計,這三人生平不知遇到過多少大風大浪,每每能當機立斷,轉危爲安,但眼前的局勢實是已陥絶境,人人重傷之下,敵人大舉來攻,眼看著只有束手待斃的份児。這時每個人隱然都已將張無忌當作教主,不約而同的望的著他,盼他突出奇計,解此困境。張無忌在這頃刻之間,心中轉過了無數念頭。他自知武功雖較楊逍、韋一笑諸人爲高,但説到見識計謀,只怕這些高手人人都勝他甚多,他們既然一無良策,自己那裡有什麼更高明的法子。

正沉吟間,突然想起一事,衝口而出叫道︰「咱們快到秘道中暫且躱避,敵人未必能彀發覺。就算發覺了,一時也不易攻入。」

他想到此法,自覺是眼前最佳的方策,語音之中,甚是興奮,不料衆人面面相覷,竟無一人附和,似乎個個認爲此法決不可行。張無忌道︰「大丈夫能屈能伸,咱們暫且避過,待傷愈之後,和敵人一決雌雄,那也不算是墜了威風。」楊逍道︰「張大俠此法誠然極妙。」他轉頭向小昭道︰「小昭,你扶張大俠到秘道去。」張無忌道︰「大夥児一齊去啊!」楊逍道︰「你先去,咱們隨後便來。」張無忌聽他語氣,知道這些人決不會來,不過是要自己躱避而已,當下朗聲説道︰「各位前輩,我張無忌雖非貴教中人,但和貴教共過一場患難,總該算得是生死之交。難道我就貪生怕死,能撇下各位,自行前去避難?」

楊逍道︰「張大俠有所不知,明教歷代傳下嚴規,這光明頂上的秘道,除了教主之外,本教教衆,誰也不許闖入,擅進者死。你和小昭不屬本教,不必守此規矩。」這時只聽得隱隱喊殺之聲,四面八方的傳來。只是光明頂上道路崎嶇,地勢峻險,一處處的関隘,均有鐵閘石門,明教雖無猛烈抵抗,來攻者却不易迅速奄至。加之明教名頭素響,來襲敵人心懷顧忌,未敢貿然深入,但聽這厮殺之聲,却總是在一步步的逼近。

偶然遠處傳來一兩聲臨死時的號呼之聲,顯是明教弟子竭力禦敵,以致慘遭屠戮。張無忌心想︰「再不走避,一個時辰之内,明教上下人衆,無一得免。」當下説道︰「這不可進入秘道的規矩,難道決計變更不得麼?」楊逍神色黯然,搖了搖頭。彭瑩玉忽道︰「各位聽我一言︰張大俠武功蓋世,義薄雲天,於本教有存亡續絶的大恩。咱們擁立張大俠爲本教第三十四代教主。倘若教主有命,號令衆人進入秘道。大夥児遵從教主之令,那便不是壞了規矩。」楊逍、殷天正、韋一笑心中本有此意,一聽彭和尚之言,人人叫好。

張無忌急忙搖手道︰「小子年輕識淺、無德無能,如何敢當此重任?加之我太師父與張眞人當年諄諄告誡,命我不可身入明教,小子應承在先。彭大師之言,萬萬不可。」殷天正道︰「我是你親外公,叫你入了明教。就算外公親不過你太師父,大家半斤八兩,我和張三丰的話就相互抵消了吧,只當誰也没有説過。入不入明教,憑你自決。」殷野王也道︰「再加一個舅父,那總彀斤兩了吧?常言道︰見舅如見娘。你娘既已不在,我就如同是你親娘一般。」張無忌聽外公和舅父如此説,甚覺淒然,又道︰「當年楊教主會有一通遺書,我從秘道中帶將出來,原擬大家傷愈之後傳觀。楊教主的遺命是要我義父金毛獅王暫攝教主之位。」説著從懷中取出那封遺書。交給楊逍。彭瑩玉道︰「張大俠,大丈夫身當大變,不可拘泥小節,謝獅王既不在此,便請你依據楊教主遺言,暫攝教主尊位。」衆人齊道︰「此言最是。」

張無忌心想︰「此刻救人重於一切,其餘儘可緩商。」於是朗聲道︰「各位既然如此見愛,小子若再不允,反成明教的大罪人了。小子張無忌,暫攝明教教主職位,渡過今日難関之後,務請各位另擇賢能。」衆人齊聲歡呼,雖然大敵逼近,禍及燃眉,但人人喜悦之情,見於顏色。須知明教自楊破天暴斃之後,統率無人,一個威震江湖的大教,鬧得自相殘殺、四分五裂,置身事外者有之,自立門戸者有之,爲非作歹者亦有之,從此一蹶不振,危機百出,今日重立教主,中興可期,如何不令人大爲振奮?能彀行動的便即拜倒行禮,殷天正、殷野王雖是尊親,亦無例外。

張無忌忙道︰「各位請起。楊左使,請你傳下號令︰本教上下人等,一齊退入秘道。命烈火旗縱火阻敵,將光明頂上的房舍盡數燒了。」楊逍道︰「是!謹遵教主令諭。」當即傳出令去,命洪水、烈火二旗斷後,其餘各人,退入秘道。明教是主,白眉教是客,當下命白眉教教衆先退,跟著是鋭金、巨木、厚土三旗,光明頂上諸般職事人員,五散人和韋一笑等先後退入。待張無忌和楊逍退入時,洪水旗諸人分别進來,東西兩面已是火光燭天。這場火越燒越旺,烈火旗人衆手執噴筒,不斷噴射西域特産的石油。那石油近火即燃,最是厲害不過,來攻的各門派人數雖多,却畏火不敢逼近,只是四面團團圍住,不令明教人衆漏網。烈火旗人衆進入秘道後関上閘門,不久房舍倒塌,將那秘道的入口掩在火燄之下。

這場大火,直燒了兩日兩夜,兀自未熄。光明頂是明教總壇所在,百餘年的經營,數百間美輪美奐的廳堂宇,盡成焦土。來攻敵人待火勢略熄、到火場中翻尋時,見到不少明教徒戰死者的屍首,皆已燒成焦炭,面目不可辨認,只道明教教衆寧死不降,人人自焚而死,楊逍、韋一笑等都已命喪火場之中。

小昭持著秘道的地圖,將衆人分别領入一間間石室安置。此時已然深入地底,上面雖然烈火熊熊,在秘道中却聽不到半點聲音,也絲毫不覺炎熱。

衆人進入秘道時,帶足了糧食清水,便是一兩個月不出去,也不會餓死。明教和白眉教人衆各旗歸旗,各壇歸壇,肅靜無聲,衆人均知這秘道是向來不許擅入的聖地,承蒙教主天大的恩典,纔得入内,因此誰也不敢多走一步。楊逍等首腦人物都聚在楊破天的遺骸之旁,聽張無忌述説如何見到楊前教主的遺書、如何練成明教聖火心法的乾坤大挪移神功。張無忌述説已畢,將那張記述武功心法的羊皮交給楊逍。楊逍不接,躬身説道︰「楊前教主的遺書上冩得明白︰『乾坤大挪移心法,暫由謝遜接掌,日後轉奉新教主。』這份心法自當由教主掌管。」

當下衆人傳閲楊破天的遺書,盡皆慨嘆︰「那料到楊教主一世神勇睿智,竟因夫婦之情而致走火歸天。咱們若得早日見此遺書,何致有今日的一敗塗地。」各人想到死難同伴之慘,自己狼狽逃命之辱,無不咬牙切齒的痛罵成崑,楊逍道︰「這成崑雖是楊教主的師弟、是金毛獅王的師父,可是咱們都未能見他一面,可見此人心計之工。原來數十年前,他便處心積慮的要摧毀本教。」周顚道︰「楊左使、韋蝠王,你們都墮入了他的道児而不覺,也可算得無能。」他本想扯上殷天正,只是礙於教主的情面,將「白眉老児」四個字嚥入了肚裡。楊逍臉上一紅,説道︰「總算天網恢恢,疏而不漏,這成崑惡賊終究命喪野王兄的掌底。」烈火旗的掌旗使辛然恨恨的道︰「成崑這惡賊作了這麼大的孼,倒給他死得太便宜了。」衆人議論了一會,當下分别靜坐用功,療養傷勢。

在這秘道中過了七八日,張無忌的劍創已好了九成,結了個寸許長的疤。他這一復元,便即替受了外傷的弟兄們治療,雖然藥物多缺,但針炙推拿,當眞是著手成春,衆人初時只道這位少年教主武功深不可測,豈知他醫道之精,幾乎已可和當年的「蝶谷醫仙」胡青牛並駕齊驅。再過數日,張無忌劍傷痊癒,當即運起九陽神功,給楊逍、韋一笑、楊不悔及五散人逼出體内一陰指的寒毒,三日之間,衆大高手内傷盡去,無不意氣風發,便要衝出秘道,盡殲來攻的敵人。張無忌道︰「各位傷勢已愈,内力未純,既已忍耐多日,索性便再等幾天。」

這數日中,人人加緊磨練,武功較淺的磨刀礪劍,武功深的則練氣運勁,自從六大派圍攻光明頂以來,明教始終挨打受辱,這口怨氣可實在{\upstsl{彆}}得狠了。這天晩間,楊逍坐在張無忌身旁,將教中歷來的規矩、明教在各地支壇的勢力、教中重要人物的才能性格,一一詳細稟告。只聽得鐵鍊叮{\upstsl{噹}}響,小昭托了一茶盤,送上兩碗茶來。張無忌想起一事,説道︰「楊左使,這個小姑娘近來無甚過犯,請你打開鐵鎖,放了她吧!」楊逍道︰「教主有令,敢不遵從。」當下叫楊不悔進來,説道︰「不悔,教主替小昭説情,你給她開了鎖吧。」楊不悔道︰「那鑰匙放在我房裡的抽屜之中,没帶下來。」張無忌道︰「那也不妨,這鑰匙想來也燒不爛。」

楊逍等女児和小昭退出,對張無忌道︰「教主,小昭這小丫頭年紀雖小,却是極爲古怪,對她不可不加提防。」張無忌道︰「這小姑娘來歷如何?」楊逍道︰「半年之前,我和不悔下山遊玩,見到她一人在沙漠之中,撫著兩具屍首哭泣。我們上前一問,她説死的二人是她爹娘,她爹爹在中原得罪了官府,一家三口,全被充軍來到西域,前幾日因不堪蒙古官兵的凌辱,逃了出來,終於她爹娘傷發力竭,雙雙斃命。我見她小小一個女孩,孤苦伶仃,雖然容貌奇醜,説語倒也不蠢,於是給她葬了父母,收留了她,叫她服侍不悔。」

張無忌點了頭,心想︰「原來小昭父母雙亡,身世極是可憐,跟我竟是一般。」楊逍又道︰「我們帶了小昭回到光明頂上之後,有一日我教不悔武藝,小昭在旁聽著,那也罷了,怎知我解釋到六十四卦方位之時,不悔尚未領悟,小昭的眼光已射到了正確的方位之上。」張無忌道︰「想是她天資聰穎,悟性比不悔妹子快了一點。」楊逍道︰「初時我也這麼想,倒很高興,但轉念一想,起了疑心。故意説了幾句極難的口訣,那是我從未教過不悔的,其時日光西照,地火明夷,火水未濟,故意説錯了方位,只見她眉頭微蹙,竟然發覺了我的錯處。從此我便留上了心,知道這小姑娘曾得高人傳授,身懷上乘武功,到光明頂上非比尋常,乃是有所爲而來。」

張無忌道︰「或者她父親精通易理,那是家傳之學,亦未可知。」楊逍道︰「教主明鑒,文士所學的易經,和武功中的易理頗有不同。倘若小昭所學竟是她父母所傳,那麼她父母當是武林中的一流高手了。我其時不動聲色,過了幾日,纔閒閒問起她父母的姓名身世。她推得乾乾淨淨,竟是不露絲毫痕跡。當時我也不發作,只叮矚不悔暗中留神,那一日我説個笑話,不悔哈哈大笑,小昭在旁聽著,忍不住也笑了起來。其時她站在我和不悔的背後,只道我父女瞧不見她的笑容,豈知不悔手中正在把玩一柄匕首,那匕首明淨如鏡。將她的笑容清清楚楚的映了出來。她却那裡是個醜丫頭?容顏之美,比之不悔只有過之而無不及。待我轉過頭來,她立時又變成了擠眼歪咀的怪相。」張無忌微笑道︰「整日價裝這怪樣,當眞是著實不易。」心想︰「楊左使是何等厲害的人物,小昭這小丫頭到他面前去耍槍花,自然要露出破綻來了。」

楊逍又道︰「當下我仍是隱忍不言,這日晩間,夜靜人定之後,我悄悄到女児房中,來窺探小昭動靜。只見這丫頭正從不悔房中出來。她逕往東邉房舍,不知找尋什麼,每一間房間?每一處隱僻之所,無不細細尋到。我再也忍不住了,現身而出,問她找尋什麼,是誰派她到光明頂來臥底。她倒也鎭靜,竟是毫不驚慌,説無人派她,只是喜歡到處玩玩,乃是好奇之心所致。我諸般恐嚇勸誘,她始於不露半句口風,我関著她餓了七天七夜,餓得她奄奄一息,她仍是不説。於是我造了這副玄鐵{\upstsl{銬}}鐐來,將她{\upstsl{銬}}住,令她行動之時,發出叮{\upstsl{噹}}聲響,那便不能暗中加害不悔。教主,這小丫頭是敵人派來臥底,那是決計無疑的,以她精通八卦方位這一節看來,只怕不是武當,是峨嵋派的了。只是諒這小小丫頭,礙得甚事?念她服侍教主一場,教主慈悲饒她,那也是她的造化。」

張無忌站起身來,笑道︰「咱們在地牢中関了這麼多日,也該出去散散心了?」楊逍大喜,問道︰「這就出去?」張無忌道︰「傷勢未愈的,無論如何不可動手。洪水、巨木兩位掌旗使暫且在旁觀戰,便要立功,也不忙在一時,其餘的便都出去吧。」楊逍出去一傳號令,秘道中登時歡聲雷動。張無忌推開阻門巨石,當先出去、待衆人走盡,又將巨石推上。那厚土旗的掌旗使顏垣是明教中第一的神力之士,他試著運勁一推那塊小山般的巨石,竟如蜻蜒撼石柱,紋風不動,不禁伸出了舌頭,縮不回去,心中對這位青年教主更是佩服無已。

衆人出得秘道,生怕驚動了敵人,連咳嗽之聲也是半點全無。張無忌站在一塊大石之上,天上月光瀉將下來,只是白眉教人聚排在西首賓位,天微、紫微、天吊三堂、神蛇、青龍、白虎、朱雀、玄武五壇,各有統率,整整齊齊的排著。

東首是明教五旗︰鋭金、巨木、洪水、烈火、厚土,各旗正副掌旗使率領本旗弟兄,分五行方位站定。中間是楊逍屬下天、地、風、雷四門門主所統的光明頂教衆。那天字門所屬的是中原男性教衆,地宇門所屬乃女子教衆。由楊不悔擔任門主;風字門乃釋家道家等出家人;雷字門則是西域番邦人氏的教衆,雖然連日激戰,各旗四門無不傷殘甚衆,但此刻人人精神振奮。青翼蝠王韋一笑及冷謙等五散人站在張無忌身後,衛護教主,人人肅靜,只候張無忌令下。

張無忌緩緩説道︰「敵人來攻本教重地,咱們雖然善罷,亦已不得。但本人若非迫不得已,不願多所殺傷,務希各位體念此意,白眉教各位由殷教主率領,自西攻擊。五行旗由巨木旗掌旗使聞蒼松總領,自東攻擊。楊左使率領天、地、風、雷四門,自北攻擊。五散人自南攻擊。韋蝠王與本人居中策應。」衆人一齊躬身應命,却無半點聲響發出。張無忌左手一揮,低聲道︰「去吧!」各人分成四隊,分從東南西北四方包圍光明頂。張無忌向韋一笑道︰「蝠王!咱兩個從秘道中出去,攻他們一個措手不及。」韋一笑大喜,説道︰「此計妙極!」兩人重行回入秘道,從楊不悔閨房的入口處鑽了出來。其時上面堆滿了瓦爍,一走出來,滿鼻便是焦臭之氣。敵人之中却也不乏好手,其時明教人衆距離尚遠,但光明頂上留著的敵人已然發覺,大呼小叫,相互警告。張無忌和韋一笑相視一笑,心中均想︰「這批傢伙大驚小怪,不必相鬥,勝敗已分。」兩人隱身在倒塌了的半堵磚牆之後,月光下但見黑影來回奔走。片刻之間,説不得和周顚兩人並肩先至,已從南方攻到,衝入人群之中,砍瓜切菜殺般殺了起來。

跟著殷天正,楊逍、五行旗人衆齊到,衆人勇氣百倍,大呼酣鬥,猶似虎入羊群一般。奪得光明頂的本有丐幫、三江幫、巫山幫、海沙派等十餘個大小幫會,但眼見光明頂燒成一片白地,明教人衆没一個漏網,只道已然大獲全勝,丐幫、巨鯨幫等一大半幫會這幾日都已紛紛下山,光明頂上只剩下神拳門、三江幫、巫山幫、五鳳刀四個幫會門派。明教教衆突然間殺將出來,這四個門派中雖然也擁有若干好手,却那裡是楊逍、殷天正這些一流名家的對手,不到一頓飯功夫,已是死傷大半。

張無忌現身而出,朗聲説道︰「明教高手此刻聚會光明頂。諸大幫會門派聽了,再鬥無益,一齊抛下兵刃投降,饒你們不死,好好送你們下山。」突然間一個身材極爲矮小的番僧越衆而出,説著一口清脆的中原口音,喝道︰「你這小賊是誰?」楊逍喝道︰「番僧無禮!這位是本教新任教主張教主!」那番僧叫道︰「什麼張教主,李教主,吃我一劍!」猛地裡手腕一翻,這一下來得好快,寒光閃閃,一柄長劍刺到張無忌身前,月光下看得分明!正是峨嵋派的倚天寶劍。

張無忌側身避過,説道︰「此劍乃峨嵋派之物,何以到了大師手中?」那番僧刷刷刷連攻三劍,劍光閃爍,招數極盡變幻。張無忌知道寶劍厲害,連連閃避,突然間左手一長,倏地一拿,已抓住那番僧的右腕。那番僧手臂酸麻,{\upstsl{噹}}的一聲,倚天劍跌在地下。豈知這番僧的武功眞了得,左手猶似閃電般擊出,一拳打在張無忌胸口。張無忌神功護體,這一拳的勁力反彈出去。那番僧身子一晃,撲地跌倒,跟著幾個翻身,便像一個大冬瓜般滾了開去。他躍起身來時,又已將那倚天劍抓在手中。彭瑩玉揮劍攔阻,那僧番長劍一閃,彭和尚手中劍只剩下了一個劍柄。那番僧更不戀戰,急衝下山去了。

張無忌心中掛念著周芷若,不知她手中的倚天劍何以會給那番僧奪去,決意要將那番僧擒住。問個明白,當下縱身躍起,疾追而下,忽聽得左首山坳中「啊」一聲尖叫,似是楊不悔的聲音,跟著一柄明晃晃的長劍直飛而上,顯是被人擊落了兵刃。張無忌救人要緊,急向聲音來處奔去。那山坳中長滿了濃密的小樹,無法瞧得見樹枝樹葉後面的情形,張無忌更不理會,一躍而入,猛地裡勁風撲面,一柄鋼刀迎面破將過來。張無忌身形略側,抓住那人手臀,將那人摔出數丈之外。只聽得樹叢中有喝罵鬥毆之聲,當即搶步進去,但見一條大漢直上直下的揮著兩柄板斧,風聲呼呼,砍得枝葉紛飛。楊不悔空著雙手,只有閃躍而避。

張無忌身形一晃,已站在這條大漢身前,喝道︰「住手!」那大漢給他威嚴所懾,爲之一怔,隨即雙斧猛劈過去。張無忌左手一拂,使出乾坤大挪移的心法,將他斧頭的去勢拂得偏了,只聽得{\upstsl{噹}}的一聲巨響,火光飛濺,雙斧一齊斫在山石之上,石屑崩舞,斧頭的刃口都捲了起來。那大漢雙臂酸麻,無力再行舉斧。楊不悔一拳打中他的太陽穴,那大漢眼珠突出,登時斃命。張無忌道︰「不悔妹子,没受傷麼?」楊不悔道︰「没有。多謝你來救我。」張無忌微微一笑,道︰「我們回去吧!」

這麼一耽擱,料知那番僧已無法追上,兩人剛回到山頂,忽聽得遠處一個陰慘慘的聲音尖厲異常的叫道︰「有誰貪生怕死,下手決不容情!有誰貪生怕死,下手決不容情。」巫山幫等人衆本已一敗塗地,正自逃竄躱避,聽到這鬼哭一般的聲音一叫,突然間精神大振,轉身死鬥,頃刻之間,倒將明教教衆殺傷了多人。但一來技不如人,二來寡不敵衆,縱然拚死惡鬥,總是一個個的倒斃下去。

張無忌朗聲説道︰「再鬥更有何益,一齊投降吧!」諸幫衆紅了眼睛,竟不罷手,月光下瞧他們的臉色時,却個個有恐懼之情,便似每個人身後跟著一個惡鬼,督促他們非戰鬥至死不可。張無忌起了不忍之心,身形晃動,猶如一陣風般轉過每個幫衆身邉,手指連伸,人人都給點中了穴道,一一抛去兵刃,摔倒在地。只有三名高手及時避開,不能一招點中,但片刻之間,已給楊逍、韋一笑、殷野王三人分别擊死。

明教這一仗大獲全勝,敵人中除了極少數人逃走之外,三百餘人非死即擒,光明頂上登時燒起熊熊大火,感謝明尊佑護。這十餘日中,巫山幫等人衆已在山頂搭了若干茅棚,暫行棲身,當下五行旗下教衆又再砍伐樹木,搭蓋茅舍,地字門下的女教衆則忙著燒水煮飯。衆人大勝之餘。雖然一夜不睡,也不疲累,只見白眉鷹王殷天正站起身來,大聲説道︰「白眉教教下各人聽了︰本教和明教同氣連枝,本是一脈。二十餘年之前,本人和明教的夥伴們不和,這纔遠赴東南,自立門戸,眼下明教由張大俠出任教主,人人捐棄舊怨,群策擊力。白眉教這個名字,從今日起再也没有了,大夥児都是明教的教衆,咱們人人聽張教主的分派號令。要是那個不服,快快給我滾下山去吧!」

白眉教教衆歡聲雷動,都道︰「咱們大夥児都入明教,那是何等的美事。殷教主和張教主是至親的家人,聽那一位教主的號令都是一樣。」殷天正大聲道︰「從今日起只有張教主,那個再叫我一聲『殷教主』,那是大大的犯上叛逆。」張無忌拱手道︰「白眉教和明教反而復合,眞是天大的喜事。只是在下迫於情勢,暫攝教主之位。此刻大敵已除,咱們正該重推教主。明教和白眉教中有這許多英雄豪傑,小子年輕識淺,何敢居長?」

周顚單刀直入,爽爽快快的道︰「張教主,你倒代咱們想一想,咱們爲了這教主之位,鬧得四分五裂,好容易個個都服了你。你若再推辭、那麼你另派一個人出來當教主吧。哼哼!不論是誰,我周顚首先不服。若是要我周顚當吧,别個児可又不服。」彭瑩玉也站起身來説道︰「張教主,倘若你不肯擔此重任,明教又回到了自相殘殺、大起内鬨的老路上,難道到那時又來求你搭救。」張無忌心想︰「這干人説的話也是實情,當此情形之下,我實不能袖手不顧。」於是朗聲説道︰「各位既是垂愛,小子不敢推辭,暫攝教主重任,只是有三件事要請各位允可,否則小子寧死不肯坦當。」

衆人紛紛説道︰「教主有令,莫説三事,便是三十件也允得。不知是那三件,便請教主示下。」張無忌道︰「本教被人視爲邪魔外道,雖説是教外之人心地偏狹,不明本教眞相。但本教之中人數多了,難免良莠不齊,亦有不肖之徒行爲放縱,殘害無辜。這第一件事,是自今而後,從本人以下,人人須得嚴守教規,爲善去惡、行俠仗義。本人請冷謙冷先生擔任刑堂執法香主,凡是違犯教規,和本教弟兄爭鬥修怨者,一律處以重刑,即令是本人的外公、舅父等尊長,亦無例外。」衆人躬身説道︰「正該如此。」彭瑩玉道︰「當年楊教主在世之時,本教教規何等嚴峻,近年來逐漸敗壞。張教主和冷兄好好整頓一番,乃是本教第一件要緊是。」冷謙跨上一步,説道︰「奉令!」他不喜多話,簡簡單單約兩個字,等於是答應自當竭盡所能,奉行張無忌所吩咐的命令。

張無忌道︰「這第二件事,説來比較爲難。本教和中原六大派結怨已深,雙方門人弟子、親戚知友,都是互有殺傷。咱們既往不咎,前衍盡釋,不再去和六大派尋仇。」衆人聽了,心頭都是氣忿不平,良久無人答話。周顚道︰「倘苦六大派再來惹事生非呢?」張無忌道︰「那時隨機應變,倘若對方一意進逼,咱們自也不能束手待斃。」鐵冠道人道︰「好吧!反正咱們的命都是教主救的,教主要咱們怎樣,那便怎樣。」彭瑩玉大聲道︰「各位兄弟,六大派殺了咱們不少人,咱們也殺了六大派不少人,如果雙方仇怨糾纏,循環報復,只有越死越多。教主命令咱們不再尋仇,也正是爲咱們好。」衆人心想這話原也不錯,終於都答應了。

張無忌抱拳道︰「各位寬洪大量,實是武林之福,蒼生之幸。這第三件事,乃是依據楊前教主的遺命而來。楊前教主遺書中説道︰凡是覓回聖火令、前赴丐幫請回第三十一代教主遺物者,接任第三十四代教主之位,在他逝世以後,教主之位由金毛獅王謝法王暫攝,咱們即當前赴海外,迎歸謝法王,由他攝行教主尊位,然後設法尋覓聖火令和前代教主的遺物。那時小子退位讓賢,各位不得再有異議。」

衆人聽了這番話,不由得面面相覷,心想︰「群龍無首數十年,好容易得了位智勇雙全、才德並備的教主。日後倘若是本教一個碌碌無能之徒無意中拾得聖火令,難道竟由他出任教主?」楊逍道︰「楊前教主的遺言冩於二十餘年之前,其時世局,與今大不相同。金毛獅王自是要去迎接的,聖火令自是要尋覓的,但若由旁人擔任教主,實難令大衆心服。」張無忌堅執楊前教主的遺命決不可違,衆人拗他不過,只得依了,均想︰「金毛獅王只怕早已死了,聖火令失落將近百年,那裡還找得著?且聽他的,將來若是有變,再作道理。」

當下張無忌命人燃起聖火,宰殺牛羊,和衆人歃血爲盟,不可違了這三件約言。行誓已畢,天色已然大亮,忽聽得樹林中一人大聲呼叫起來,聲音極是驚惶。

\chapter{沙漠埋屍}

鐵冠道人喝道︰「什麼人無事驚惶?」只見林中兩個人急奔而出,正是洪水旗下的弟兄,奔到洪水旗掌旗使唐洋跟前,稟報了幾句。唐洋奇道︰「有這等事?」指指點點發了幾句號令,洪水下三十餘名弟兄,四面八方的搜了下去,餘下的各佔方位,佈成防敵的陣勢,唐洋親自率領數人,到樹林中去査察。洪水旗苦戰之餘,剩下的已不足百人,但唐洋指劃分派,凜然若不可犯,單是洪水旗一旗,便足與江湖一般幫會門派分庭抗禮,張無忌瞧在眼中,暗想明教中人才濟濟,前途不可限量,心下甚是欣慰。

過不多時,唐洋從林中快步出來,向張無忌躬身行禮,臉上頗有惶愧之色,説道︰「啓稟教主,屬下唐洋領罪。」張無忌道︰「唐旗使何事?」唐洋道︰「屬下派人看管俘虜,不料衆俘虜突起發難,搶了看管人員的刀刃,人人自殺而死,看守者阻止不及,大虧職守。」張無忌道︰「此事甚奇。」與衆人同到林中,只見巫山幫、五鳳刀各被俘虜人衆,一齊屍橫地下。洪水旗下的奉命看管的八名教衆,倒有六人受了傷,跪在地下領罪。張無忌道︰「這些人確是自殺,並非爲人所害?」領頭的看守者稟道︰「啓稟教主,這些俘虜忽地一聲不息的跳將起來,擊倒了屬下,搶去刀劍。人人自殺,自始至終没出一句聲。」張無忌點點頭道︰「事出意外,並非你們過失,起來吧!」那人道︰「謝教主恕罪之恩!」

張無忌一看衆俘虜的傷痕,確是個個自殺斃命,只見屍堆中一人的手臂微微一動,尚未斷氣,當即俯身伸掌貼住那人的靈台要穴,一股九陽眞氣送了過去。那人睜開眼來,神色茫然。張無忌道︰「你爲什麼自殺?」那人斷斷續續的道︰「有誰貪生\dash{}怕死\dash{}下手\dash{}下手決不\dash{}容情\dash{}」張無忌一怔,記起適纔激戰之時。山腰間有人如此呼喝,對方立即拚死惡鬥,知道其中定是大有蹊蹺,又問︰「是誰下手決不容情?」那人道︰「我一家\dash{}一家老少\dash{}妻子幼児\dash{}都在人家手中\dash{}」張無忌道︰「在誰的手中?咱們給你去救將出來。」那人搖了搖頭,唇角邉露出一絲苦笑,頭一低,就此氣絶。

楊逍等聽了那人之言,都是面面相覷,猜不透其中含意。張無忌命洪水旗將衆屍體搬到山腰裡掩埋了,和殷天正,楊逍、韋一笑回入茅棚,商議此事。彭瑩玉道︰「這些人的家屬落入旁人手中,受人挾制,若不死戰,只怕妻児老小個個難以活命。江湖上有誰有這等威力權勢,能驅策這許多幫會門派的豪傑?能將他們的家小扣以爲質?」這些人除了張無忌之外,個個熟知江湖間情事,即均想不起有這麼一個了不起的人物。周顚道︰「那番僧手持倚天寶劍,定與峨嵋派暗有勾結,看來是六大門派在背後主持其事。滅絶老賊尼陰險狠毒,她鬥不過咱們教主,便指使一般嘍囉來跟本教爲難。」冷謙道︰「不是。」周顚道︰「爲什麼不是?」冷謙不答。周顚又問︰「爲什麼不是?」冷謙仍是不答。

説不得道︰「我想扣押諸幫會家小之事,在中原有預謀。六大派圍攻本教,期在必勝,滅絶老尼這些人自負得緊,決不會想到一個『敗』字,不致事先伏下這著棋子。」衆人點頭稱是。周顚道︰「就算你的話有理,那麼暗中跟咱們爲難的人是誰?」説不得道︰「倘若成崑這惡賊未死,咱們定説是他。現下可就難猜得很了。」衆人商議了半日,不得要領。張無忌道︰「此事且擱在一旁。本教眼前第一大事,是去海外迎歸金毛獅王謝法王。此行非本人親去不可,有那一位願與本人同去」衆人一齊站起身來,説道︰「願追隨教主,同赴海外。」

張無忌道︰「前往海外的人手也不必太多,何況此外尚有許多大事需人料理。這樣吧,楊左使率領天地風雷四門,留鎭光明頂,重建總壇。金木水火土五旗分赴各地,招集明教分散了人衆,傳諭本人所約三事。外公和舅父率同舊部,探聽究是那一些厲害的敵人暗中在跟本教爲難,再尋訪光明右使和紫衫龍王的兩位下落。韋蝙王請分别前往六大派掌門人居處,説明本教止戰休好之意,不能化敵爲友,也當止息干戈。這件事甚不易辦,但韋蝠王大才,定能克建殊功。至於赴海外迎接謝法王之事,則由本人和五散人同去。」

此時他是教主之尊,每一句話即是不可違抗的嚴令,衆人一一接令,無不凜遵。楊不悔却道︰「爹,我想到海外去瞧瞧滿海冰山的風光。」楊逍微笑道︰「那你向教主求去,我可作不了主。」楊不悔掀起了小嘴,却不作聲。張無忌微微一笑,想起數年前護送楊不悔來西域時,一路上她纏著要説故事,自己曾將冰火島上各種奇景、以及白熊、火猴、海豹、怪魚,各種珍異動物説給她聽,這當児她便想親自去看看了,當下説道︰「不悔妹子,海行甚多兇險,你若不怕,楊左使又放心你去,那麼楊左使和你一起都到海外去吧。」楊不悔拍手道︰「我怕什麼?爹,咱倆都跟無忌\dash{}不,跟教主去!」楊逍望著張無忌不答,聽他示下。張無忌道︰「既是如此,偏勞冷先生留鎭光明頂,天地風雷四門,暫歸冷先生統率。」冷謙道︰「是!」周顚拍手頓足,大叫︰「妙極,妙極!」説不得道︰「周兄,妙什麼?」周顚道︰「教主如此倚重冷謙,那是咱五散人的面子。再説,大海茫茫,不知要坐幾日幾夜的海船,多了楊左使父女,談談説説,何等快活,倘若同著冷謙,那只不過多一塊不開口的木頭罷了。」衆人一齊大笑,冷謙却既不生氣,也不發笑,只常没有聽見。

當日衆人飽餐歡聚,分别休息。張無忌要楊不悔替小昭開了玄鐵{\upstsl{銬}}鐐,但那鑰匙失落在火場的焦木瓦礫之中,再也尋找不著。小昭淡淡的道︰「我帶了這叮叮{\upstsl{噹}}{\upstsl{噹}}的鐵鍊,走起路來反而好聽,還是戴著的好。」張無忌安慰她道︰「小昭,你安心在光明頂上住著,我接了謝法王回來,借他的屠龍寶刀給你斬脱{\upstsl{銬}}鐐。」小昭搖了搖頭,並不答應。

次日清晨,張無忌率領衆人,和冷謙分别。冷謙道︰「教主,你身繫本教的安危存亡,務請保重。」張無忌道︰「冷先生坐鎭總壇,多多辛苦。」冷謙向周顚道︰「小心,怪魚,吃你!」周顚握著他手,心中頗爲感動,五散人情若兄弟,冷謙今日破例多説了這六個字,那的確是十分耽心大海中的怪魚將衆兄弟吃了。冷謙和天地風雷四門首領直送下光明頂來,這纔不捨而别。

無忌等行了百餘里,在沙漠中就地歇宿。無忌睡到中夜,忽聽得西首隱隱傳來叮{\upstsl{噹}}、叮{\upstsl{噹}}、叮{\upstsl{噹}},清脆的金屬撞擊之聲。他練就九陽神功之後,耳目比常人靈敏十倍,側耳傾聽,心中一動,當即悄悄起來,向聲音來處急速迎了上去。奔出數里,只見小小一個人影,正在黑暗中移動,他搶步上去,叫道︰「小昭,怎麼你也來?」那人影正是小昭,她突然見到無忌,哇的一聲,哭了出來,撲在他的懷裡,抽抽噎噎的只是哭泣,却不説話。無忌輕拍她的肩頭,説道︰「好孩子,别哭,别哭!」小昭似乎受盡了委曲,終於得到發洩,哭得更加響了,説道︰「你到那裡,我\dash{}我也跟到那裡。」無忌心想︰「這小姑娘父母雙亡,又見疑於楊左使父女,原是十分可憐。想是我對她和言悦色,是以對我十分依戀。」

張無忌於是説道︰「好,别哭啦,我也帶你一起到海外去便了。」小昭大喜,抬起頭來。只見她清麗秀美的小小臉龐,在銀波如水的月光照映下,當眞是出塵脱俗,晶瑩的泪水尚未擦去,却已笑得極是歡暢,猶似一朶帶著曉露的水仙。張無忌微笑道︰「小昭,你將來大了,一定美得不得了。」小昭笑道︰「你怎知道?」張無忌尚未回答,忽聽得東北角上蹄得雜沓,有大隊人馬自西而東,奔馳而過,但聽那蹄聲漸漸遠去,至少也有一百餘人。

過不多時,韋一笑和楊逍先後奔到。説道︰「教主,深夜之中,大隊人馬奔馳,説不定又是本教之敵。」張無忌命小昭去和彭瑩玉等人會合,自行帶同楊韋二人,奔向蹄聲傳來之處査察。到得近處,果見沙漠中留下一排馬蹄印痕。韋一笑俯身察看,忽然抓起一把沙子,説道︰「有血跡。」張無忌將沙子湊近鼻端,登時聞到一陣新鮮的血腥之氣。三個人循著蹄印追出數里,楊逍忽見左首沙中掉著半截單刀,抬起一看,見刀柄上刻著「馮人豪」三字,微一沉吟,説道︰「這是崆峒派中的人物。教主,想是崆峒派在此預備下馬匹,回歸中原。韋一笑道︰「從光明頂下來,已然事隔半月有餘,他們尚在這裡,不知搗什麼鬼?」

三人既然査知是崆峒派,便不放在心上,回歸原地安睡。行到第五日上,前面草原上來了一行人衆,張無忌視物及遠,已看清楚大部份是身穿緇衣的尼姑,夾雜看七八個男子。雙方行到相距十餘丈處,一名尼姑尖聲叫道︰「是魔教的惡賊!」衆人紛紛拔出兵刃,散了開來。張無忌瞧這情勢,對方準是峨媚派的人衆,不知何以去而復回,而那些人也是從未見過的,當下朗聲説道︰「衆位師太是峨嵋門下嗎?」一名身材瘦小的中年尼姑越衆而出,厲聲道︰「魔教的惡賊,多問什麼?上來領死吧。」張無忌道︰「師太上下如何稱呼?何以如此動怒?」那尼姑喝道︰「邪惡奸賊,憑你也配問我名號!你是誰?」

韋一笑惱她對教主無禮,一衝而前,身形如同鬼魅,穿入衆人之中,已點了兩名男弟子的穴道,抓住兩人後領,猛地發脚,遠遠奔了出去,將兩人摔在地下,隨即又奔回原處。這幾下兔起鵠落,快速無倫,峨嵋衆人一怔之間,那兩名男弟子已被他就像騰雲駕霧搬運到了數十丈以外,橫臥就地,一動不動。只聽韋一笑冷笑一聲,説道︰「這位是當世武功第一,天下肝膽無雙的奇男子,統率左右光明使、四大護教法王、五散人,五行旗、天地風雷四門的明教張教主,趕過峨嵋派下山,奪過滅絶師太手中倚天寶劍,像這樣的人物,也配來問一聲師太的法名麼?」

他這番話一口氣的説將出來,峨嵋群弟子盡皆駭然,眼見韋一笑適纔露了這麼一手匪夷所思的武功,無人再會懷疑他的説話。那中年尼姑定了定神,纔道︰「閣下是誰?」韋一笑道︰「在下姓韋,外號叫做青翼蝠王。」峨嵋派中幾個人不約而同的驚呼一聲。便有四個人奔去救護那兩個被他搬到了遠處的同門。韋一笑道︰「奉張教主令︰明教和六大派上息干戈,釋怨修好。貴同門週身無傷,蝙蝠王這次没吸他的血。」原來韋一笑自經張無忌以九陽神功療傷之後,不但驅除了所中的一陰指寒毒,連從前積下的陰寒之氣也消了大半,不必每次行動,便須吸食人血以抗寒毒。

那四個人抬了那兩名被點中穴道的同門回來,正待設法給他解治,只聽得嗤嗤兩響,兩粒黃沙被以強勁之極的指力彈了過來。帶著破空之聲,直射那二人的穴道,登時替他解了。

原來那是楊逍以「彈指神通」的奇功,反運「擲石點穴」的功夫,將那兩名峨嵋弟子的穴道解了。那中年尼姑眼見對方人數雖然不多,但個個武功高得出奇,何況明教教主之尊也親身在此,若是動起手來,只怕立時便吃大虧,便道︰「貧尼法名靜空,不敢請問這位施展彈指神通、擲石點穴絶技的施主是誰?」楊逍尚未回答,周顚已哈哈笑道︰「他是本教光明使者,可跟你是一家人啦!」靜空退了一步,雙眉倒豎,喝道︰「原來你便是害死我紀師妹的惡賊楊逍!」手中長劍一振,忍不住便要撲前跟他拚命。

張無忌道︰「此中情由,靜空師太一問尊師便知,不必在此多生糾葛。」靜空道︰「我師父呢?」張無忌道︰「尊師從光明頂下來,已半月有餘,預計此時已進玉門関。各位東來,難道中間錯過了麼?」靜空身後一個三十來歳的女子説道︰「師姊别聽他胡説。咱們分三路接應,有信號火箭聯絡,怎會錯過不見?」周顚聽她説話無禮,正要教訓她幾句,張無忌低聲道︰「周先生不必跟她一般見識。她們尋不著師父,自然著急。」靜空滿臉懷疑之色,説道︰「家師和衆位師姐妹是不是落入了明教之手?大丈夫光明磊落,何必隱瞞?」周顚笑道︰「老實跟你們説,峨嵋派不自量力,來攻光明頂乃自滅絶師太以下,個個被擒,現下正関在水牢之中,教她們思過待罪,関他個十年八年,放不放那時再説。」彭瑩玉忙道︰「各位莫聽這位周兄説笑,滅絶師太神功蓋世,門下弟子個個武藝高強,怎能失陥於明教之手?此刻貴我雙方已然止息干戈,各位回去峨嵋,自然見到。」靜空將信將疑。猶豫不決。

韋一笑道︰「這位周兄愛説笑話。難道本教教主堂堂之尊,也會騙你們小輩不成?」那中年女子道︰「魔教向來詭計多端,奸詐狡獪,説話如何能信?」洪水旗掌旗使唐洋左手一揮,突然之間,巨木在東、烈火在南、鋭金在西、洪水在北,厚土居中,五行旗旗下教衆兵刃出手,將一干峨嵋弟子團團圍在中間。白眉鷹王殷天正大聲説道︰「老夫是白眉鷹王,只須我一人出手,就將你們一干小輩都拿下了。明教今日手下留情,年青人以後説話可得檢點些。」這幾句話轟轟發發,震得峨嵋群弟子耳朶中{\upstsl{嗡}}{\upstsl{嗡}}作響,心神動盪,難以自制。眼見殷天王白鬚白眉,神威凜凜,衆入無不駭然。

張無忌一拱手,説道︰「多多拜上尊師,便説明教張無忌問她老人家好。」當先向東便去。唐洋待韋一笑、殷天正等一一走過,這纔揮手召回五行旗。峨嵋弟子瞧了這等聲勢,暗暗心驚,眼送張無忌等遠去,個個目瞪口呆,説不出話來。

彭瑩玉道︰「教主,我瞧這事確是其中另有蹺蹊。滅絶師太諸人東還,不該和這干門下弟子錯失道路。各門各派沿途均有聯絡記號,那有影蹤不見之理?」衆人邉走邉談,都覺峨嵋派這許多人突然在大漠中消失,其理難明,何況那倚天寶劍落入了那番僧之手,更是兆頭不好。張無忌掛念周芷若的安危,却又不便和旁人談商。

這日行到傍晩,説不得忽道︰「這裡有些古怪!」奔向左前方的一排矮樹之間察看,從一名教衆的手裡接過一把鐵鏟,在地下挖掘起來,過不多時,赫然露出一旦屍首。這屍首已然腐爛,面目殊不可辨,但從身上衣著看來,顯然是崑崙派的弟子。幾名教衆一齊助手挖掘,不久掘出一個大坑,坑中橫七豎八的堆著十六七具屍體,盡是崑崙弟子。倘若是本派掩埋,決不致如此草草,顯是敵人所爲。再査那些屍體,人人身上有傷。説不得命手下教衆將各具屍體好好分開,一具具的妥爲妥葬。

衆人你瞧瞧我,我瞧瞧你,心頭的疑問都是一樣︰「誰幹的?」大家怔了一陣,彭瑩玉纔道︰「此事倘不査個水落石出,這筆爛帳定然冩在明教頭上。」各人都是見多識廣之輩,心照不宣,均知前面等著一批武功高強、行事毒辣的勁敵。只是這群敵人詭秘陰險,更顯得難以對付。説不得朗聲説道︰「大家聽了,若是明刀明槍的交戰,大夥児在教主率領之下,雖不敢説天下無敵,也決不致輸於旁人。只是暗箭難防,此後飲水食飯、行路住宿,處處要防敵人下毒暗算。」教衆們齊聲答應︰「是!」

又行一陣。眼見夕陽似血,天色一陣陣的黑了下來,衆人正要覓地休息,只見東北角天邉三四頭兀鷹不住在天空盤旋。突然間一頭兀鷹俯衝下去,立即又急飛而上,羽毛紛紛掉落,啾啾哀鳴,顯是給下面什麼東西擊中了,吃了一個大虧。

鋭金旗的掌旗使莊錚死在倚天劍下之後,副旗使吳勁草承張無忌之命,升任了正旗使,這時見幾頭兀鷹古怪,道︰「我去瞧瞧。」帶了兩名弟兄,急奔過去,過了一會,一名教衆先行奔回,向張無忌稟道︰「稟告教主,武當派殷六俠摔在山谷之中。」張無忌大吃一驚,道︰「是殷六俠?受了傷麼?」那人道︰「似乎是受了重傷,吳旗使一見是殷六俠,命屬下急速稟報教主。吳旗使已下谷救援去了\dash{}」張無忌心急如焚,不等他説完,快步奔去,楊逍、殷天正等隨後跟來。到得近處,只見那裡是一個峭壁,下臨深谷,崖旁生滿了長草小樹,吳勁草左手抱著殷利亨,正在十分吃力的攀援上來。張無忌掛懷殷利亨的生死安危,沿著山壁搶了下去,一手抓住吳勁草右臂,另一手便去探殷利亨的鼻息。只覺他呼吸細微,張無忌便放寬了心,接過殷利亨的身子,幾個縱躍,便上了峭壁,將他橫放在地下,定神一看,不禁又是驚怒,又是難過。但見他膝、肘、踝、腕、足趾、手指,所有四肢的関節,全都被人折斷了,氣息奄奄,動彈不得,對方下手之毒,實是駭人聽聞。他神智尚未迷糊,一見到無忌,臉上微露喜色,吐出了口中的兩顆石子。原來他受傷後被人推下山谷,仗著内力精純,一時却不致死,兀鷹想來吃他,被他側頭咬起地下的石子,噴氣射擊,如此苦苦撐持,已有數日。

楊逍見那四頭兀鷹尚自盤旋未去,似想等衆人抛下殷利亨後,便飛下來啄食他的屍體,心下惱怒,從地下拾起四粒小石,嗤嗤連彈,四頭兀鷹應聲落地,每一個的腦袋都是被小石打得粉碎。殷利亨點了點頭,多謝楊逍替他出了這口氣。

張無忌先給他服下止痛護心的藥丸,然後設法替他接續斷骨,但一加査察,便即皺起了眉頭,但見他四肢共有二十來處斷折,每一處斷骨均是被重手指力捏成粉碎,再也無法接續。殷利亨道︰「跟三哥一樣,是少林派金剛指力\dash{}指力所傷\dash{}」張無忌登時想起當年父親所説三師伯兪岱岩受傷的經過來,他也是被少林派的金剛指力捏得骨節粉碎,臥床已達二十餘年。其時自己父母尚未相識,不料事隔這許多年月,又有一位師叔傷在少林金剛指之下。

他定了定神,説道︰「六叔不須煩心,這件事交給了侄児,定教奸人難逃公道。那是少林派中何人所爲,六叔可知道麼?」殷利亨搖了搖頭,他數日來苦苦掙命,早已筋疲力盡,此刻心頭一鬆,再也支持不住,便此昏暈了過去。張無忌想起自己身世,父母所以自刎而死,最主要的是爲了對不起三師伯。今日六師叔又遭此難,再不勒逼少林派交出這罪魁禍首,如何對得起兪殷二位?又如何對得起死去的父母?

張無忌見殷利亨雖然昏暈,性命已是無礙,只是斷肢難續,多半也要和兪岱岩同一命運。他負著雙手,遠遠走了開去,要安安靜靜的細細思量一下。他走上一個小丘,坐了下來,心中兩種念頭不住交戰︰「要不要到少林寺去,找到那罪魁禍首,跟爹爹、媽媽、六叔報此大仇?若是少林派肯坦率承認,交出行兇之人,事端就不致擴大,否則豈非明教要和武當派聯合,共同對付少林?我已和衆兄弟歃血盟誓,決不再向六大派尋仇生事,此刻事情鬧到了自己頭上,就將誓言抛諸腦後,那如何能彀服衆?禍端一開,此後怨怨相報,只怕又要世世代代的流血不止,不知要傷殘多少英雄好漢的性命?」

這時天已全黑,明教衆人點起燈火,埋鍋造飯,張無忌兀自坐在小丘之上,眼見月亮升起,仍是拿不定主意,一直想到半夜,纔這麼決定︰「咱們且到少林寺去求見掌門空聞神僧,説明前因後果,要他給一個公道。」轉念又想︰「但若把話説僵了,非動手不可,那便如何?」他嘆了一口氣,站起身來。要知他年紀輕輕,初當大任,所遭逢的却是江湖上最辣手的難題,身處恩怨仇殺之際,便是老成練達、識見超卓之士,也未必能有善策,何況他武功雖佳,處事的經驗却是淺鮮之極,一心想要止戰息爭,但兇殺血仇,對一件件迫人而來。張無忌機緣巧合,當了明教教主的重任,推不掉、甩不脱,此後的煩惱艱困,實是無窮無盡呢!

他回到燈火之旁,衆人雖然肚餓,却誰都没有動筷吃飯,恭敬肅穆的站著等候。張無忌好生過意不去,忙道︰「各位以後自管用飯,不必等我。」去看殷利亨時,只見楊不悔已用熱水替他洗淨創口,餵他飲湯。殷利亨神智仍是迷糊,突然間眼睛定定的瞧著楊不悔,大聲道︰「曉芙妹妹,我想得你好苦,你知道麼?」楊不悔滿臉通紅,神色極是{\upstsl{尷}}尬,右手拿著匙羹,低聲道︰「你再喝幾口湯。」殷利亨道︰「你答應我,永遠不離開我。」楊不悔道︰「好啦,好啦!你先喝了這湯再説。」殷利亨似乎甚爲喜悦,張口把湯喝了。

次日張無忌傳下號令,各人暫且不要分散,一齊到嵩山少林寺去,問明打傷殷利亨的原委再説。韋一笑、周顚等個個是俠義之士,眼見殷利亨如此重傷,均是心中不平,聽教主説要到少林問罪,齊聲喝采,楊逍爲了紀曉芙之事,一面對殷利亨極是抱憾,口中雖然不言,心裡却立定了主意,決意竭全力,爲他報仇,更命女児好好照顧服侍,稍補自己的前過。

一路無話,這日衆人進了玉門関。分别買了牲口代步。殷利亨時昏時醒,張無忌問起他如何受傷的情形,殷利亨茫然難言,只是説︰「少林派的和尚,五個人圍攻我一個。是少林派的武功,決計錯不了。」

衆人生怕招搖,惹人耳目,都買了商販的衣服換了,有的更推著獨輪木車,裝了皮貨藥材之物。這日清晨動身,在甘涼大路上趕道、驕陽如火,天氣漸漸熱了起來,行了兩個時辰,眼見前面一排二十來棵大柳樹。衆人心中甚喜,催趕坐騎,奔到柳樹之下休息。到得近處,只見柳樹下已有九個人坐著。八個大漢均作獵戸打扮,腰跨佩刀,背負弓箭,還帶著五六頭獵鷹,墨羽利爪,模樣極是神駿。另一人却是個年輕公子,身穿寶藍綢衫,輕搖摺扇,掩不住一副雍容華貴之氣。

張無忌翻身下馬,突然和那公子的目光一觸,只見他雙日炯炯有神,紫電般的閃了一閃,目光隨即隱没,轉過頭來時,却變成了一副文弱儒雅的神態。這年輕公子美得出奇,手中摺扇白玉爲柄,但握著扇柄的手,白得和扇柄竟無分别。

但在一瞥之間,衆人目光不約而同的都瞧向少年公子腰間,只見黃金爲鉤、寶帶爲束,懸著一柄長劍,劍柄上赫然鏤刻著「倚天」兩個篆文。這劍的形狀長短,正和滅絶師太持以大屠明教教衆、周芷若用以刺得張無忌重傷幾死的倚天劍一模一樣。明教衆人大爲愕然,周顚第一個忍不住要開口相詢,便在此時,只總得東邉大路上馬蹄雜沓,一群人亂糟糟的乘馬奔馳而來。

衆人凝目一瞧,却是一隊元兵,約莫有五六十人,另有一百多名婦女,被元兵用繩縛著曳之而行。這些婦女大都小脚伶仃,如何跟得上馬匹,有的跌倒在地下,便被繩子掛著,隨地拖行。所有婦女都是漢人,顯是這群元兵擄掠來的良民百姓,其中半數都已衣衫被撕得稀爛,有的更是裸露了大半身,哭哭啼啼,極是悽慘。那些元兵有的手持酒瓶,喝得半酔,有的則用鞭子抽打衆女。這些蒙古人一生長於馬背,鞭術精奇,一鞭抽去,便捲下了女子身上一大片衣衫,餘人歡呼喝采,引以笑樂。

蒙古人侵入中國,將近百年,素來瞧得漢人比牲口也還不如,只是這般在光天化日之下,淫虐欺辱,却也是極少見之事,明教衆人見了,無不目眥欲裂,只待張無忌一聲令下,使即衝上救人。

忽聽有那少年公子説道︰「六破,你去叫他們放了這干婦女,如此胡鬧,成什麼樣子。」他説話也聲音清脆無比,又嬌又嫩,竟然似個女子。一名大漢應道︰「是!」解下繫在柳樹上的一匹黃馬,翻身上了馬背,大聲説道︰「喂,大白天這般胡鬧,你們也没官去管束麼?快快把衆婦女放了!」元兵中一名軍官裝束之人有騎馬乘衆而出,在臂彎中摟著一個少女,斜著酔眼,哈哈大笑,説道︰「你這死囚活得不耐煩了,來管老爺的閒事!」那大漢冷冷的道︰「天下盜賊四起,都是你們這班不恤百姓的官兵鬧出來的,乘早給我規矩些吧。」那軍官打量柳蔭下的衆人,心下微感詫異,暗想平常老百姓一見官兵,遠遠躱開尚自不及,怎麼這群人吃了豹子膽、老虎心,竟敢管起官軍的事來?一眼掠過,見到那少年帽子頭巾上兩粒龍眼般大的明珠,瑩然生光,貪心登起,大笑道︰「兔児相公,跟了老爺去吧!有得你享福的!」説著雙腿一挾,催馬向那少年公子衝來。那公子本來和顏悦色,瞧著衆元兵的暴行似乎也不生氣,待聽得這軍官如此無禮,秀眉微微一蹙,説道︰「别留一個活口。」

他這「口」字剛説出,颼的一聲響,一支羽箭射出,將那軍官射得洞胸而過,乃是他身旁一個獵戸所發。此人發箭手法之快,勁力之強,幾乎已是武林中的一流好手,那裡是尋常獵戸的身手。那軍官一聲不出,抱著懷中少女,一齊倒衝下馬來。只聽得颼颼颼連珠箭發,八名獵戸一齊放箭,當眞是百步穿楊,箭無虛發,每一箭便射死一名元兵。衆元兵雖然變起倉卒,大吃一驚,但個個是弓馬嫻熟的戰士,各人連聲呼哨,便即還箭。那八名獵戸躍上馬背,衝了過去,一箭一個,一箭一個,頃刻之間,射死了三十餘名元兵。餘下的見情勢不對,一聲忽哨,丟下衆婦女回馬便走。那八名獵戸跨下的都是駿馬,風馳電掣般追將上去,八枝箭射出,便有八名元兵倒下,追出不到一里,蒙古官兵盡數就殲。

那少年公子牽過坐騎,縱馬而去,更不回頭再望一眼,他在瞬息間屠滅五十餘名蒙古官兵,便似家常便飯一般,竟是絲毫不以爲意。周顚叫道︰「喂,喂!慢走!我有話問你啊!」那公子更不理會,在八名獵戸擁衛之下,遠遠的去了。

\chapter{女扮男裝}

張無忌、韋一笑等若是施展輕功追趕,原也可以追及奔馬,向那少年公子問個明白,但群豪見那八名獵戸神箭殲敵,俠義爲懷,心下均存了敬佩之意,不便貿然冒犯。衆人紛紛議論,却都猜不出這九個人的來歷。楊逍道︰「那少年公子明明是女扮男裝,這八個獵戸打扮的高手却對她恭謹異常。這八人箭法如此神妙,不似是中原的那一個門派的人物。」這時楊不悔和厚土旗下衆人過去慰撫一衆被擄的女子,問起情由,知道均是附近村鎭中的百姓,遺下從元兵的屍體王搜檢出金銀財物,分發衆女,命她們各自從小路歸家。

此後數日之間,群豪總是談論著那箭殲元兵的九人,這些人心中都起了惺惺相惜之意,所謂英雄重英雄,恨不得能與之訂交爲友,把臂談心。周顚對楊逍道︰「楊兄,令愛本來也算得是絶色的美女,可是和那位男裝打扮的小姐一比,相形之上,那就比下去啦。」楊逍道︰「不錯,不錯。他們若肯加入本教,那八個獵戸的排名就該在『五散人』之上。」周顚怒道︰「放你娘的臭屁!騎射功夫有什麼了不起?你叫他們跟周顚比劃比劃。」楊逍沉吟道︰「比之周兄自是稍有不如,但以武功而論,看來比冷謙兄要略勝一籌。」要知明教五散人中,武功以冷謙爲冠,這是衆所周知之事,楊逍和周顚素來不睦,雖然不再明爭,但周顚一有機會,便是和楊逍鬥幾句口。這時周顚聽他説八獵戸的武功高於冷謙,那顯是把五散人壓了下去,心頭愈怒,正待反唇相譏,彭瑩玉笑道︰「周兄又上了楊左使的當,他是有意激你生氣呢!」周顚哈哈大笑,説道︰「我偏不生氣,你奈何得我?」但過不多時,又指摘起楊逍騎術不佳來。群豪相顧莞爾,知道他瘋瘋癲癲,説話行事,均是顚三倒四,每次和楊逍鬥口,總是敗下陣來。

這時殷利亨每日在張無忌醫療之下,神智已然清醒,説起那日從光明頂下來,心神激盪,竟在大漠中迷失了道路,越走越遠,在黃沙莽莽的戈壁中摸了八九日。待得覓回舊路,已和武當派師兄弟們失去了聯絡,這日突然遇到了一批少林僧人,那些人一言不發,便即上前挑戰,殷利亨雖然打倒了四人,但寡不敵衆,終於身受重傷。他説這批僧人的武功是少林一派,確然無疑,只是並未在光明頂上會過,想來是後援的人衆,到底何以對他忽下毒手,實是猜想不透。一路之上,楊不悔對他服侍得十分周到,她知自己母親從前負他良多,又見他情形如此悽慘,不禁憐惜之心大起。

這天黃昏,群豪過了永登,各人加緊催馬,要到江城子投宿。正行之間,忽聽得蹄聲響處,大路上兩騎馬並肩馳來,奔到數十丈外,即便躍下馬背,牽馬候在道旁,神態甚是恭敬。群豪一看,那二人獵戸打扮,正是箭殲元兵的八雄中人物。群豪大喜,紛紛下馬,迎了上去。那兩人走到張無忌跟前,躬身行禮,其中一人朗聲説道︰「敝上仰慕明教張教主仁厚重義,群俠英雄了得,命小人邀請各位赴敝莊歇馬,以表欽敬之忱。」張無忌還禮道︰「豈敢豈敢!不知貴上名諱如何稱呼?」那人道︰「敝上姓趙。閨名不敢擅稱。」衆人聽他直認那少年公子是女扮男裝,足見相待之誠,心中均喜。張無忌道︰「自見諸位弓箭神技,每日裡讚不絶口,得蒙不棄下交,幸如何之。只是叨擾不便。」那人道︰「各位均是當世英雄,敝上心儀已久,今日路過敝地,豈可不奉三杯水酒,聊盡地主之誼。」張無忌一來願盼結識這幾位英雄人物,二來要打聽倚天劍的來龍去脈,便道︰「既是如此,咱們自當造訪寶莊。」那二人大喜,上馬先行,在前領路。行不出一里,又有二人馳來。

那二人遠遠的便下馬相候,又是神箭八雄中的人物,再行里許,神箭八雄的其餘四人也並騎來迎。明教群豪見他們禮教如此周到,盡皆喜慰。順著青石板鋪的大路,來到一所大莊院前,莊子周圍一條小河環繞,河邉滿是綠柳,在甘涼一帶竟能見到這等江南風景,群豪都是精神爲之一爽,只見莊門大開,放下吊橋,那位小姐仍是穿著男裝,站在門口迎接。那小姐一見衆人來到,搶上前來,躬身行禮,朗聲道︰「明教諸位豪俠今日駕臨綠柳山莊,當眞是蓬蓽生輝。張教主請,楊使者請,殷老前輩請!韋蝠王請\dash{}」她對明教群豪竟是個個相識,不須引見,便一一道出名號,而且教中地位誰高下,也是順著次序説得中無錯誤。

衆人一怔之下,周顚忍不住便問︰「大小姐,你怎地知道咱們的賤名?難道你竟有未卜先知的本領麼?」趙小姐微笑道︰「明教群俠名滿江湖,誰不知聞?近日光明頂一戰,張教主以絶世神功威懾六大派,更是傳遍武林。各位東赴中原,一路上不知將有多少武林朋友仰慕接待,豈獨小女以爲然?」衆人一想不錯,但口中咱是連連謙遜,問起那神箭八雄的姓名師承時,一個身材高大的漢子道︰「在下是趙一傷,這是錢二敗,這是孫三毀,這是李四摧。」再指著另外四人道︰「這是周五輸,這是吳六破,這是鄭七滅,這是王八衰。」

明教群豪聽了,無不啞然,心想這八人的姓氏依著「百家姓」上「趙錢孫李、周吳鄭王」排列,已是十分奇詭,所用的名字更是個個不吉,至於「王八衰」云云,那直是匪夷所思了,知道這定然不是眞名、但江湖中人避禍避仇,隨便取一個假名,也是尋常得緊,當下不再多問?趙小姐親自領路,將衆人讓進大廳。群豪一看,大廳上中間懸著一幅趙孟頫繪的「八駿圖」,八駒姿態各自不同,匹匹神駿風發。左邉壁上懸著一幅大字,文曰︰

\begin{quotation}
白虹座上飛\hskip8pt青蛇匣中吼\

殺殺霜在鋒\hskip8pt團團月臨紐

劍決天外雲\hskip8pt劍衝日中斗

衝破妖人腹\hskip8pt劍拂佞臣首

潛月辟魑魅\hskip8pt勿但驚妾婦\

留斬泓下蛟\hskip8pt莫試街中狗
\end{quotation}

詩末題了一行小字道︰「夜試倚天寶劍,洵神物也,書『説劍』詩以讚之。汴梁趙明。」

筆致英挺,有如騰蛟起鳳,直欲從壁上飛出。張無忌家學淵源,對書法的品評頗有眼光,見這一幅字雖然英氣勃勃,却有撫媚之致,顯是出自女子的手筆,知是這位趙小姐所書。他雖讀書不多,但詩句含意並不晦澀,一誦即明,心道︰「這柄倚天寶劍果然起在她手中。詩中説道『劍破妖人腹,劍拂佞臣首』,足見俠義正直,又説『留斬泓下蛟,莫試街中狗』,却又自負得緊。她落款『汴梁趙明』,原來是汴梁人氏中單名一個『明』字。」便道︰「趙姑娘文武全才,佩服佩服。原來姑娘是中州舊京世家。」那小姐微微一笑,道︰「張教主的尊大人號稱『銀鉤鐵劃』,自是第一流的書家。張教主家傳的書藝,小女子待會尚要求懇一幅法書。」

張無忌一聽此言,臉上登時紅了,他十歳喪父。並未好好跟父親習練書法,此後學醫學武,於文字一道,實是淺薄之至,便道︰「姑娘要我冩字,那可要了我的命啦。先先父見背太早,在下未克繼承先父之學,大是慚愧。」説話之間,莊丁已獻上茶來,只見雨過天青的瓷杯之中,飄浮著嫩綠的龍井茶葉,清香撲鼻。群豪暗暗奇怪?此處和江南相距數千里之遙,如何能有新鮮的龍井茶葉?這位姑娘,實是處處透著奇怪。只見趙明端起茶杯,先喝了一口,意示無他,等群豪用過茶後,説道︰「各位遠道光降,敝莊諸多簡慢,尚請恕罪。各位旅途勞頓,想必餓了,請這邉先用些酒飯。」説著站起身來,引著群豪穿廊過院,到了二座大花園中。

那花園佔地極大,山石古拙,溪池清澈,花卉不多,却極是雅緻。張無忌不能領略這座園子的勝妙之處,楊逍却已暗暗點頭,心想這花園的主人實非庸夫俗流,胸中大有丘壑,只見一個水閣之中,已安排了兩桌酒席。趙明請張無忌入座,趙一傷、錢二敗等神箭八雄,則在邉廳裡陪伴明教的其餘教聚入席。殷利亨無法起身,由楊不悔在廂房裡餵他飲食。

趙明斟了一大杯酒,一口乾了,説道︰「是紹興的女貞陳酒,已有一十八年的功力,各位請嘗嘗酒味如何?」楊逍、韋一笑、殷天正等雖已深信這位趙小姐仍是俠義之輩,但仍是處處小心,細看酒壺、酒杯均無異狀,趙小姐已喝了第一杯酒,這纔去了疑忌之心、放懷飲食。明教的教規本來是所謂「食菜事魔」,禁酒忌葷,但到了石教主手中,已革除了這種飲食上的禁忌,蓋明教的總壇遷到崑崙山中之後,當地氣候嚴寒,倘若不食牛羊油脂,内力稍差者便抵受不住。

水閣四周的池中種著七八株水仙一般的花卉,似水仙而大,花作白色,香氣幽幽。群豪臨水而飲,清風送香,極是暢快。那趙小姐談吐甚健,説起中原各派的武林軼事,竟有許多連殷天正和殷野王也不知道。她於少林、峨嵋、崑崙諸派武功頗少許可,但對張三丰和武當七俠抑是推祟備至,每一句評讚又是洞中竅要。群豪聽得津津有味,心下好生佩服,但問到她自己的武功師承時,趙明却是笑而不答、往往將話題岔了開去。

酒過數巡,趙明酒到杯乾,極是豪邁,每一道菜上來,她總是搶先挾一筷吃了,眼見她臉泛紅霞,微帶酒暈,容光更增麗色。自來美人,不是溫雅嬌美,便是艷媚婉轉,這位趙小姐却是十分美麗之中。更帶著三分英氣,三分豪態,同時雍容華貴,自有一副端嚴之致,令人肅然起敬,不敢逼視。張無忌道︰「趙姑娘,承蒙厚待,敝教上下無不威激。在下有一句言語想要動問,只是不敢出口。」趙明道︰「張教主何必見外?我輩行走江湖,所謂『四海之内,皆兄弟也』,各位若是不棄,便交交小妹這個朋友。有何吩咐垂詢,小妹自當竭誠奉告。」張無忌道︰「既是如此,在下想要請問,趙姑娘這柄倚天寶劍是從何處得來?」

趙明微微一笑,解下腰間倚天劍,放在桌上,説道︰「小妹自和各位相遇,各位目光灼灼,不離此劍,不知是何緣故,可否先行見告?」張無忌道︰「實不相瞞,此劍原爲峨嵋派滅絶師太所有,敝教弟兄,喪身在此劍之下者實不在少。在下自己,也會被此劍穿胸而過,險喪性命,是以人人関注。」趙明道︰「張教主神功無敵,聽説曾以乾坤大挪移法。從滅絶師太手中奪得此劍,何以反爲此劍所傷?又聽説劍傷張教主者,乃是峨嵋派中一個青年弟子,武功也只平平,小妹對此殊爲不解。」説話時盈盈妙目,凝視張無忌臉上,決不稍瞬,口角之間,似笑非笑。

張無忌臉上一紅,心道︰「她怎麼知道得這般清楚?」便道︰「對方來得過於突兀,在下未及留神,至有失手。」趙明微笑道︰「那位周芷若周姊姊,大槩是太美麗了,是不是?」張無忌更是滿臉通紅,道︰「姑娘取笑了。」端起酒杯,想要飲一口掩飾窘態,那知左手微顚,竟潑出了幾滴酒來,濺在衣襟之上。趙明微笑道︰「小妹不勝酒力,再飲恐有失儀,現下説話已是不知輕重了。我進去換一件衣服,片刻即回,諸位請各自便,不必客氣。」説著站起身來,團團一揖,走出水閣,穿花拂柳的去了。那柄倚天劍仍是平放在桌上,並不取出。侍候的家丁們不斷送上菜肴。

群豪相互對視了一眼,這些菜肴便不再食,等了良久。却不見趙明回轉。周顚道︰「她把寶劍留在這裡,倒放心咱們。」説著便拿起劍來,托在手中,突然「噫」的一聲,説道︰「怎地這般輕?」抓住劍柄,抽了出來。劍一出稍,群豪一齊站起身,無不驚愕。這那裡是斷金切玉、鋒鋭絶倫的倚天寶劍,竟是一把木製的長劍,各人鼻端同時聞到一股淡淡的香氣,但見劍刃色作淡黃,竟是檀香木所製。

周顚一時不知所措,將木劍又還入劍鞘,喃喃的道︰「楊\dash{}楊左使,這\dash{}這是什麼玩意児?」他雖和楊逍成日鬥口,但心中實是佩服他見識卓超、此時遇上了疑難,不自禁脱口便向他詢問。楊逍的臉色極是鄭重,低聲道︰「教主,這趙小姐十九不懷好意。此刻咱們身處危境,急速離開爲是。」周顚道︰「怕她何來?她敢有舉動,憑著咱們這許多人,還不殺他個落花流水?」楊逍道︰「自進這綠柳莊來,只覺處處透著詭異,似正非正,似邪非邪,難在捉摸不到這綠柳莊到底是何門道。咱們何必留在此地,事事爲人所制?」張無忌點頭道︰「楊左使所言不錯。咱們已用過酒菜,如此告辭便去。」説著便即離坐。鐵冠道人道︰「那眞倚天劍的下落,教主便不尋訪了麼?」彭瑩玉道︰「依屬下之見,這趙小姐故佈疑陣,必是有所爲而來。咱們便是不去尋她,她自會再找上咱們。」張無忌道︰「不錯,咱們後發制人,以逸待勞。」

當下各人一齊出了水閣,回到大廳,命家丁通報小姐,説明教衆人多謝盛宴,便此作别。趙明匆匆出來,身上已換了一件淡黃的綢衫,更顯得瀟灑飄逸,容光照人,説道︰「纔得相會,如何便去?莫是嫌小女子接待太過簡慢麼?」張無忌道︰「多謝姑娘厚賜,怎説得上『簡慢』二字。咱們俗務纏身,未克多時。日後相會,當再討教。」趙明嘴角邉似笑非笑,直送出莊來。神箭八雄恭恭敬敬的站在道旁,躬身送客。

群豪抱拳而别,一言不發的縱馬疾馳,眼見離綠柳莊已遠,四下裡一片平野,更無旁人。周顚大聲説道︰「這位趙大小姐未必安著什麼壞心眼児,她拿一柄木劍跟教主開個玩笑,那是女孩児家胡鬧,當得什麼眞?楊逍,這一次你可走了眼啦!」楊逍沉吟道︰「到底是什麼道理,我一時也説不上來,只是覺得不對勁。」周顚笑道︰「大名鼎鼎的楊左使在光明頂上一戰之後,變成了驚弓之\dash{}啊喲。」身子一晃,倒撞下馬來。説不得和他相距最近,急忙躍下馬背,搶上扶起,説道︰「周兄,怎麼啦?」周顚笑道︰「没\dash{}没什麼,想是多喝了幾杯,有些児頭暈。」他一説起「頭暈」兩字,群豪相顧失色,原來自離綠柳莊後,一陣奔馳,各人都微微有些頭暈,只是以爲酒意發作,誰也没有在意,但以周顚武功之強,酒量之宏,喝這幾杯酒怎能倒撞下馬?其中定有蹊蹺。

張無忌抑起了頭,思索王難姑所載「毒經」中所載,有那一種無色、無味、無臭的毒藥,能使人服後頭暈,但從頭至尾想了一遍,各種毒藥都不相符,而且自己飲酒食菜,與群豪絶無分别,何以却絲毫不覺有異?突然之間,腦海中猶如電光般一閃,猛地裡想起一事,不由得大吃一驚,叫道︰「在水閣中飲酒的各位一齊下馬,盤膝坐下,千萬不可運氣調息,一任自然。」又下令道︰「五行旗和白眉旗下弟兄,分佈四方、嚴密保護諸位首領,不論有誰走近,一槩格殺!」白眉教歸併明教後,去了這個「教」字,稱爲「白眉旗」,旗下教衆和五行旗下衆人一聽教主頒下嚴令,轟然答應,立時抽出兵刃,分佈散開。張無忌叫道︰「不等我回來,不得離散。」

群豪一時不明所以,只感微微頭暈,絶無其他異狀,何以教主如此驚慌,張無忌又再叮囑一句︰「不論心頭如何煩惡難受,總之是不可調運内息,否則毒發無救。」群豪吃了一驚︰「怎地中了毒啦?」只見張無忌身形一晃,已竄出十餘丈外,他嫌騎馬太慢,竟是施展絶頂輕功,一溜煙般直撲綠柳莊去。

他心中焦急異常,知道這次楊逍、殷天正等人所中劇毒,一發作起來只不過一時三刻之命,決不似中了「一陰指」後那麼可以遷延時日,倘若不及時搶到解藥,衆人那就性命休矣。這二十餘里途程,片刻即至,到得莊前,一個起落,身子已如一枝箭般射了進去。守在莊門前的衆莊丁只是眼睛一花,似乎有個影子在身邉閃過,竟没看清有人闖進莊門。張無忌直衝後園,搶到水閣,只見一個身穿嫩綠綢衫的少女左手持杯,右手執書,坐著飲茶看書,正是趙明。這時她已換了女裝。

她聽見張無忌脚步之聲,回過頭來,微微一笑。張無忌道︰「趙姑娘,在下向你討幾棵花草。」也不等趙明答話,左足一點,從池塘岸躍向水閣,身手平平飛渡,猶如點水蜻蜒一般,雙手已將水中那些像水仙一般花草,七八棵株盡數拔起。正要踏上水閣,只聽得嗤嗤幾聲響處,幾枚細微的暗器直向他面門射來,張無忌右手袍袖一拂,將那些暗器捲在衣袖之中,左手衣袖挪出,攻向趙明。趙明斜身相避,只聽得呼呼風響,桌上的茶壺、茶杯、果碟等物,一齊被那袖風帶出,越過池塘,摔在花木之中,片片粉碎。張無忌身子站定,一看手中花草,只見每一棵花的根部都掛著雞蛋大小的球莖,殷紅如血。他心中大喜,知道解藥已得,當即揣在懷内,説道︰「多謝解藥,告辭!」趙明笑道︰「來時容易去時難!」擲去書巻。雙手順勢從書中抽出兩柄薄如紙。白如霜的短劍,面搶上來。

張無忌掛念殷天正衆人的傷勢,不願和她戀戰,右袖拂出,釘在袖上的十多枚金針反向趙明射了過去。趙明斜身閃出水閣,右足在台階上一點,重行回入,就這麼一出一進,十餘枚金針都落入了池塘之中,張無忌讚道︰「好身法。」眼見她左手前,右手後在兩柄短劍斜刺而至,心想︰「這丫頭心腸如此毒辣,倘若我不是練過九陽神功,讀過王雖姑的『毒經』,今日明教已不明不白的傾覆在她手中。」雙手一探,挾手便去奪她短劍。趙明的武功也甚了得,皓腕倏翻,雙劍便如閃電般削他手指。張無忌這一奪竟然無功,心下暗奇,但他神功變幻,何等奥妙,雖然奪不下她的手刃,手指拂處,已拂中了趙明雙腕穴道。她再也拿捏不住,乘勢擲出,張無忌頭一側,登登兩響,兩柄短劍都釘在水閣的木柱之上,餘勁不衰,兀自顫動。張無忌心頭微驚,倒不是驚訝她武功了得,以武功而論,她還遠不到楊逍、韋一笑、殷天正等人的地步,但心思靈機,變招既快且狠,雙劍雖然把捏不住,仍要脱手傷人,若以爲她兵刃非出手不可,已不足爲患,躱避遲了一瞬。那便命喪劍底。可見臨敵時的機變,往往能補功力之不足,弱能勝強,便由於此。

趙明雙劍出手,右腕翻處,已抓了那柄套著倚天劍劍鞘的木劍在手,她却不敢拔劍出鞘,伸鞘往張無忌腰間{\upstsl{砸}}來。張無忌左手食中兩指疾點她左肩「肩貞穴」,待她側身相避,右手一探,這是乾坤大挪移法豈能再度無功,早已將木劍挾手奪了過來。趙明站定脚步,笑吟吟的道︰「張公子,你這是什麼功夫?難道便是乾坤大挪移神功麼?我瞧那也平平無奇。」張無忌左掌攤開,掌中一朶珠花輕輕顫動,正是趙明插在鬢邉之物。

趙明心中大吃了一驚,暗想︰「他摘去我鬢邉珠花,我竟是絲毫不覺,倘若他有意傷我性命,只須當摘下珠花之時,手指乘勢在我左邉太陽穴上一戳,我此刻早已命赴黃泉了。」臉上却是絲毫不動聲色,淡然一笑,説道︰「你既喜歡我這朶珠花,送了給你便是,也不須動手強搶。」張無忌倒給她説得有些不好意思,左手一揚,將珠花擲了過去,説道︰「還你!」轉身便出水閣。

趙明伸手接住珠花,叫道︰「且慢!」張無忌轉過身來,只聽她笑道︰「你何以偸了我珠花上兩粒最大的珍珠?」張無忌道︰「胡説八道,我没功夫跟你説笑。」趙明將那朶珠花高高舉起,正色道︰「你瞧,這裡不是少了兩粒珍珠麼?」張無忌一瞥眼之下,果見珠花中有兩根金絲的頂上没了珍珠,料知地是故意摘去,想引得自己走近身去,又施什麼詭計,只哼了一聲,不去理會。趙明手按桌邉厲聲説道︰「張無忌,你有種就走到我身前三步之地。」那知無忌最沉得住氣,偏不受她激,説道︰「你説我膽小怕死,也由得你。」説著又跨下了兩步台階。

趙明見激將之計無效,花容變色,慘然道︰「罷啦,罷啦,今日我栽到了家,有何面目去見旁人?」反手拔下柱上的一柄短劍,叫道︰「張無忌,多謝你成全!」無忌回過頭來,只見白光一閃,趙明已將短劍往自己胸口插了下去。無忌冷笑道︰「我纔不上你\dash{}」下面「當」字還没説出,只見那短劍當眞往她胸口插入,趙明慘呼一聲,嬌軀倒在桌邉。張無忌這一驚著實不小,那料到她居然會如此烈性,數招不勝,便即揮劍自戕,心想這一劍若非正中心臟,或有可救,當即轉身,回來著她傷勢。

他走到離桌三步之處。正要伸手去扳她肩頭。突然間脚底一軟,登時空了,身子直墜下去。張無忌暗叫不好,雙手袍袖運氣下拂,身子在空中微微一停,一掌便往東邉擊去,這一掌只要擊中了,便能借力躍起,不致落入脚底的陥阱。那知趙明自殺是假,這著也早已料到,右掌運勁揮出,不讓他手掌碰到桌子。這一切兔起鵠落,全是瞬息之間的事,雙掌一交,張無忌的身子早已落下了半截,百忙之中,他手腕一翻,抓住了趙明右手的四根手指,她手指又滑又膩,立時便要滑脱,但無忌只須有半分可資著力之處,便有騰挪餘地,手臂暴長,已抓住了趙明的上臂,只是他身子重而趙明身輕,一拉之下,兩人一齊落入了陥阱。眼前一團漆黑,身子不住下墜,但聽得拍的一響,頭頂翻板已然合上。這一跌下,直有十餘丈深,張無忌雙足一著地,立即躍起施展「壁虎游牆功」到陥阱頂上。伸手去推翻板。觸手之處,堅硬冰涼,竟是一塊巨大的鐵板,被機括扣牢牢地。他雖具乾坤大挪移神功,但身懸半空。不似站在地下那樣,可將力道挪來移去,一推之下,鐵板紋絲不動,身子已落了下來。趙明格格笑道︰「八根粗鋼條扣住了,你人在下面,怎能離得開?」好忌惱她狡猾奸詐,不去理她,在陥阱四壁摸索,尋找脱身之計。這陥阱的周壁摸上去都是冷冰冰的,十分光滑,堅硬異常。趙明道︰「張公子,你的『壁虎游牆功』當眞了得。這陥阱是純鋼所鑄,打磨得滑不留手,連細縫也没一條,你居然游得上去,嘻嘻,嘿嘿!」

無忌怒道︰「你也陪我陥身在這裡,有什麼好笑?」突然想起︰「這丫頭奸滑得緊,這陥阱中必有出路,别要讓她獨自逃了出去。」當下上前兩步,抓住了她的手腕,趙明驚道︰「你幹什麼?」無忌道︰「你别想獨個児出去,你要活命,乘早開了翻板。」趙明笑道︰「你慌什麼?咱們總不致餓死在這裡。待會他們尋我不見,自會放咱們出去。最耽心的是,我手下人若以爲我出莊去了,那就糟糕。」張無忌道︰「這陥阱之中,没有出路的機括麼?」趙明道︰「瞧你生就一張聰明面孔,怎地問出這等笨話來?這陥阱又不是造來自己住著好玩的,那是用以捕捉敵人的東西,難道故意在裡面留下開啓的機括,好讓敵人脱身而出麼?」無忌一想倒也不錯,説道︰「翻板一動,有人落下,外面豈能不知?你快叫人來開啓翻板。」趙明道︰「我的手下,人都派出去啦,明天這時候,他們便回來了。你不用心急,好好休息一會有剛纔吃過喝過,也不會就餓了。」

無忌大怒,心想︰「我多待一會児不要緊,可是外公他們還有救麼?」五指一緊,用上了二成力,喝道︰「你不立即放我出去,我先殺了你再説。」趙明笑道︰「你殺了我,那你是永遠别想出這鋼牢了。喂,男女授受不親,你握著我手幹麼?」無忌被她一説,不自禁的放脱了她手腕,退後兩步,靠壁坐下。只是這鋼牢方圓不過數尺,兩人最遠也只能相距兩步,張無忌又是憂急,又是氣惱!鼻中却聞到趙明身上的少女氣息,加上懷中的花香,不禁心神爲之一蕩,當下站了起來,怒道︰「我明教和你素不相識,無怨無仇,你何故處心積慮,要置我個個於死地?」趙明道︰「你不明白的事情太多,既然問起,待我從頭説來。你可知我是誰?」

無忌一想不對,雖然頗想知道這少女的來歷和用意,但若等她從頭至尾的慢慢説來,殷天正等人已然毒發斃命,何況怎知她説的是眞是假,若是她捏造一套謊話來胡説八道一番,更是徒耗時刻,眼前更無别法,只有逼她叫人開啓翻板,便道︰「我不知道你是誰,這時候也没功夫聽你説。你到底叫不叫人來放我?」趙明道︰「我無人可叫。再説,在這裡大喊大叫之上面也聽不見。」張無忌怒極,和身撲上。趙明驚叫一聲,出手撐拒,早被張無忌點中了脅下穴道,動彈不得。無忌左手叉住她咽喉,道︰「我只須輕輕使力,你這條性命便没了。」這時兩人相距極近,只覺她呼吸急促,吐氣如蘭,無忌枕頭仰起,和她離開得遠些。趙明突然嗚嗚咽咽的哭了起來,泣道︰「你欺侮我,你欺侮我!」

這一著又是大出張無忌意料之外,一愕之下,放開了扼在她喉嚨中的手,説道︰「我又不想欺侮你,是要你放我出去。」趙明哭道︰「我又不是不肯,好,我叫人啦!」提高嗓子,叫道︰「喂,喂,來人哪!把翻板開了,我落在鋼牢中啦。」她不斷的叫喊,外面却毫無助靜。趙明笑道︰「你瞧,有什麼用?」無忌氣惱之極,説道︰「也不羞,又哭又笑的,成什麼樣子。」趙明道︰「你自己纔不羞,一個大男人,却來欺侮弱女子?」張無忌道︰「你是弱女子麼?你詭計多端,比十個男子漢還要厲害。」趙明笑道︰「多承張大教主誇讚,小女子愧不敢當。」

張無忌一咬牙,心想事勢緊急,倘若不施辣手,明教便要全軍覆没,伸過手去,嗤的一聲。將趙明的裙子撕下手掌大的一片。趙明以爲他忽起歹念,這纔具的驚惶起來,叫道︰「你\dash{}你做什麼?」張無忌道︰「你若決定要放我出去,那便點頭。」趙明道︰「爲什麼?」無忌不去理她,吐些唾液,將那片綢手浸濕了,説道︰「得罪了,我這是無法可施。」當下將那濕綢封住了她的口鼻。趙明立時呼吸不得,片刻之間,胸口氣息窒塞,説不出的難過。她却也眞硬氣,竟是不肯點頭,熬到後來,眼前金星亂舞,竟暈了過去。

\chapter{興師問罪}

張無忌一搭趙明的手腕,只覺脈息極是微弱,當下揭開封住她口鼻的濕綢布。過了半晌,趙明悠悠醒轉,睜開眼來,狠狠地瞪了無忌一眼。無惡道︰「這滋味不大好受吧?你放不放我出去?」趙明恨恨的道︰「我便再昏暈一百次,也是不放,要麼你就乾脆殺了我。」張無忌見她如此硬挺,一時倒是束手無策,咬一咬牙,説道︰「我爲了救衆人性命,只好動粗了,無禮莫怪。」抓起她的左脚,扯脱了她的鞋襪。趙明又驚又怒,叫道︰「臭小子,你幹什麼?」無忌不答,又扯脱了她右足的鞋襪,伸雙手食指點住她兩足脚底心的「湧泉穴」上,運起九陽神功,一股暖氣便即從「湧泉穴」上來回游走。

那「湧泉穴」在足心陥中,乃「足少陰腎經」的起端,感覺最是敏鋭,張無忌精通醫理,自是明曉。平時児童嬉戲,以手搔爬遊伴足底,即令對方周身酸麻,此刻無忌以九陽神功的暖氣擦動她「湧泉穴」,那是比之羽毛絲髮更加難當百倍。只擦動數下,趙明忍不住格格嬌笑,想要縮脚閃避,苦於穴道被點,那裡動彈得半分?這分難受,遠甚於刀割鞭打,便如幾千萬隻跳蚤,在五臟六腑、骨髓血管中爬動咬囓,只笑了數聲,便難過得哭了出來。無忌忍心不理,繼續施爲,趙明一顆心幾欲從胸腔中跳了出來,週身毛髮,癢得幾欲根根脱落,罵道︰「臭小子\dash{}賊\dash{}小子,總有一天,我\dash{}我將你千刀\dash{}千刀萬剮\dash{}好啦,好啦,饒\dash{}饒了我吧\dash{}張\dash{}張公子\dash{}張教\dash{}教主\dash{}嗚嗚\dash{}嗚嗚\dash{}」張無忌道︰「你放不放我?」趙明哭道︰「我\dash{}放\dash{}快\dash{}停手\dash{}」無忌這纔放心,説道︰「得罪了!」在她背上推拿數下,解開了她的穴道。

趙明喘了一口長氣,罵道︰「賊小子,替我著好鞋襪!」無忌拿起羅襪,一手便握住她左足,剛纔一心脱困,意無别念,這時一碰到她溫膩柔軟的足踝,心中不禁又是一蕩。趙明將脚一縮,羞得滿面通紅,幸好黑暗中無忌也没瞧見,她一聲不響的穿好鞋襪,在這一霎時之間,心中起了異樣的感覺,似乎只想張無忌再來摸一摸自己的脚。却聽無忌厲聲喝道︰「快些,快些!快放我出去。」趙明一言不發,伸手摸到鋼壁上刻著的一個圓圏,倒轉短劍劍柄,在圓圏中忽快忽慢,忽長忽短的敲擊七八下,敲擊之聲甫停,豁喇一響,一道亮光從頭頂照射下來,那翻板登時開了。原來這鋼壁的圓圏之處有細管和外邉相連,趙明以約定的訊號敲擊,管機関的人不敢怠慢,立即打開翻板。

張無忌没料到説開便開,竟是如此直捷了當,不由得一愕,説道︰「咱們走吧!」趙明低下了頭,站在一邉,默不作聲。張無忌想起她是一個女孩児家,自己一再折磨於她,好生過意不去,躬身一揖,説道︰「趙姑娘,適纔在下實是迫於無奈,這裡跟你謝罪了。」趙明索性將頭轉了過去,向著牆壁,肩頭微微聳動,似在哭泣。她奸詐毒辣之時。張無忌跟她鬥智鬥力,殊無雜念,這時内愧於心,又見她背影姻娜苗條,後頸中皮色瑩白勝玉,秀髮蓬鬆,不由得微起憐惜之意,説道︰「趙姑娘,我走了,張某多多得罪。」趙明的背脊微微扭了一下,仍是不肯回過頭來。

無忌不敢再行耽擱,又即施展「壁虎遊牆功」一路遊上,待到離那陥阱之口尚有丈餘,右足在鋼壁上一點,沖天竄出,袍袖一拂,護住頭臉,生怕有人伏在阱口突加偸襲。身子尚未落下,遊目一望,水閣中不見有人。他不願多生事端,越過圍牆,抄小徑奔回明教群豪歇息之處。眼見夕陽在山,剛纔在陥阱中已耽了將近一個時辰,不知殷天正等性命如何,心中憂急,脚下奔得更快,片刻間已到了原處,舉目一望,吃了一驚。

只見大隊蒙古騎兵奔馳來去,將明教群豪圍在中間,衆元兵彎弓搭箭,一箭箭向人圏中射去。張無忌心想︰「本教的首領人物一齊中毒,無人領頭,如何抵擋得住大隊敵兵的圍攻?」脚下加快,搶上前去。剛奔到鄰近,只聽得人叢中一個清脆的女子聲音叫道︰「鋭金旗攻東北方,洪水旗至西南方包抄。」那正是小昭的聲音,她呼喝之聲甫歇,明教中一隊白旗教果從東北方衝殺出來,一隊黑旗教衆兜至西南包抄,元兵分隊抵敵,突然間黃旗的厚土旗、青旗的巨木旗教衆從中間並肩殺出,猶似一條黃龍,一條青龍捲將出來。元兵陣脚被衝,一陣大亂,當即退後。

張無忌幾個起落,已奔到教衆身前,衆人見教主回轉,齊聲吶喊,精神大振。無忌見殷天正、楊逍等團團坐在地下,小昭却手執一面小旗,站在一個土丘之上,指揮教衆禦敵。五行旗、白眉旗各路教衆都是武藝高強之士,一經小昭以奇門八卦之術佈置方位,元兵竟是久攻不進。小昭叫道︰「張公子,你來指揮。」張無忌道︰「還是你指揮得好。待我出去擒住領兵的將軍,脅他退兵。」只聽得颼颼數聲,幾枝箭向他射了過來,無忌從教衆手裡接過一枝長矛,一一撥落,手臂一振。那長矛便如一枝箭般飛了出去,將一名元兵百夫長穿胸而過,釘在地下。衆元兵大聲叫喊,又退出了數十步。

突聽得號角嗚嗚響動,接著十餘騎馬奔馳而至,無忌眼尖,早看到是趙明手下的「神箭八雄」,不禁眉頭微蹙,暗想︰「這八人箭法太強,若任得他們發箭,只怕衆弟兄損傷非小。須得先下手爲強!」但見那「神箭八雄」中爲首的趙一傷手中搖動一根金色的龍頭短杖,大聲叫道︰「主人有令,立即收兵。」帶兵的元兵千夫長大叫了幾句蒙古話,衆元兵撥轉馬頭,翻翻滾滾的去了。

那錢二敗手中端著一隻托盤,下馬走到張無忌身前,躬身説道︰「我家主人請教主收下留念。」無忌一看,只見托盤中鋪著一塊黃色錦緞,緞上放著一隻黃金盒子,鏤刻得極是精緻。無忌也不怕他弄什麼鬼,伸手拿了,錢二敗躬身行禮,倒退三步,這纔轉身上馬而去。無忌將黃金盒子順手交給了小昭,他掛念著衆人中毒的病勢,也無暇去看盒中是何物事,從懷中取出那些水仙模樣的花來,命人取過清水,捏碎那血紅的球莖,調在清水之中,分别給殷天正、楊逍等人服下。這一役中,凡是赴水閣飲宴之人,除了張無忌因有九陽神功護體,諸毒不侵之外,所有明教首腦,無不中毒,只是楊不悔陪著殷利亨在外,小昭及諸教衆在廂廳中飲食,各人遵從教主號令,各物沾口之前均悄悄以銀針試過,倒是没有中毒。

那解毒物甚是對症,不到半個時辰。群豪體内毒性消解,不再頭暈眼花,只是周身乏力而已,問起中毒和取得解藥的原因,張無忌嘆道︰「咱們已然處處提防,酒水食物之中有無毒藥,我當可瞧得出來,那知那女子下毒的心機直是匪夷所思,這種水仙模樣的花叫作『酔仙靈芙』,雖然極是難得,本身却無毒性。這柄假倚天劍乃是用海底的『奇鯪香木』所製,本身也是無毒,可是這兩種香氣混在一起,便成劇毒之物了。」周顚拍腿叫道︰「都是我不好,誰叫我手癢,去拔出這倚天劍來瞧他媽的勞什子。」無忌道︰「她既處心積慮的設法陥害,周兄便是不去動劍,她也會差人前來拔劍下毒,那是防不了的。」周顚道︰「走!咱們一把火去把那綠柳莊燒了!」

他剛説了那句話,只見來路上黑煙衝天而起,紅燄閃動,正是那綠柳莊著了火,群豪面面相覷,説不出話來。

群豪心中同時轉著一個念頭︰「這個趙姑娘事事料敵機先,早就算到咱們毒解之後,定會前去燒莊,她反而先行放火將這綠柳莊燒了。此人年紀雖輕,又是個女流之輩。却實是勁敵。」周顚拍腿叫道︰「她燒了莊子便怎地?咱們還是趕去,追殺她個落花流水。」楊逍道︰「她既連莊子都燒,自是事事有備,料想未必能追趕得上。」周顚道︰「楊兄,你的武功也還罷了,講到計謀,總算比周顚稍勝半籌。」楊逍笑道︰「豈敢、豈敢?周兄神機妙算,小弟如何能及?」張無忌笑道︰「兩位不必太謙。咱們這次受有多大損傷,只没十三個位弟兄受了箭傷,也算是是天幸,這就趕路吧。」

群豪在道上問起無忌,如何能想到各人中毒的原因,無忌道︰「我記得『毒經』中有一條説道︰『奇鯪香木』如與芙蓉一類花香相遇,往往能使人沉酔數日不解。毒氣若入臟俯,大損心肺。是以我叫各位不可運息用功。内息一作,花香侵入各處經脈,爲害就是難以估計了。」韋一笑道︰「想不到小昭這小丫頭居然建此奇功,倘若不是她在危急之際挺身而出,本教死傷必重。」楊逍初時認定小昭乃是受強敵指使,前來明教臥底,但今日一役之中,她反而成了明教的功臣,却是令他大出意料之外,一時也想不透其中的原由。

當晩衆人一早就投店歇宿了,小昭倒了臉水,端到無忌房中,無忌説道︰「小昭,你今日建此奇功,以後小用再做這些丫頭的賤役了。」小昭嫣然一笑,道︰「我服侍你很高興,那又是什麼賤役不賤役了?」待無忌盥洗已畢,將那隻黃金盒子取了出來,道︰「不知盒中有没有毒蟲毒藥,毒箭暗器之類藏著?」無忌道︰「不錯,咱們該當小心纔是。」將盒子放在桌上,拉著小昭走得遠遠地,取出一枚銅錢,揮手擲出,叮的一聲響,正打在金盒子的邉緣,那盒蓋彈了開來,並無異狀。無忌走近一看,只見盒中却是一朶珠花,兀自微微顫動,正是無忌曾從趙明鬢邉摘下的那朶珠花,趙明所除去的兩粒珍珠,却已重新穿在金絲之上。無忌一看,不由得呆了,一時想不出趙明此舉是何用意。

小昭笑道︰「張公子,這位趙姑娘可對你好得很啊,巴巴的派人來送你這麼貴重的一朶珠花。」無忌道︰「我是男子漢,要這種女孩子的首飾何用?小昭,你拿去戴吧。」小昭連連搖手,笑道︰「那怎麼成?人家對你一片情意,我怎麼敢收?」無忌左手三指拿著珠花,笑道︰「著!」將那珠花擲出,手勢不輕不重,剛好插在小昭的頭髮上,却又没傷到她的皮膚。小昭伸手想去摘了下來,無忌搖手道︰「乖孩子,難道我送你一點玩物也不成麼?」小昭雙頰紅暈,低聲道︰「那我可多謝啦。就怕小姐見了生氣。」無忌道︰「今日你幹了這番大事,楊左使父女那能對你再存什麼疑心?」小昭滿心甚歡,説道︰「我見你去了很久不回來,心中急得什麼似的,又見韃子來攻,不知怎樣,忽然大著膽子呼喝起來。現在這時候自己想想,當眞害怕。張公子,你跟五行旗和白眉旗的各位爺們説説,小昭大膽妄爲,請他們不可見怪。」無忌微笑道︰「他們多謝你遠來不及,那裡會見怪了。」

自此一路無話,衆人沿途談論趙明的來歷?誰都摸不著端倪。張無忌將雙雙跌入陥阱,自己搔她脚底脱困等情隱去了不説,雖然自己心中無愧,但當衆談論,總覺難以啓齒。不一日來到河南境内,其時天下大亂,四方群雄並起,蒙古官兵的魁査更加嚴緊。明教大隊人馬,成群結隊的行走不便,分批到嵩山脚下會齊,這纔同上少室山,由吳勁草持了張無忌等人的名帖,投向少林寺去。

張無忌知道此次來少林問罪,雖然不欲再動干戈,但結果如何,殊難逆料,倘若少林僧人竟是蠻不講理的動武,明教不得不起而應戰,當下傳了號令,命五行旗和白眉旗下各路教衆,裝作觀賞風景,散在寺周四方,若聽得自己三聲清嘯,便即攻入接應。諸教衆接令,分頭而去。

過不多時,寺中一名老年的知客僧隨同吳勁草迎下山來,説道︰「本寺方丈和諸長老閉関靜修,恕不見客中。」群豪一聽,盡皆變色。周顚怒道︰「這位是明教教主,親自來少林拜山,老和尚們居然不見,未免忒也托大。」那知客僧低首垂眉,滿臉愁苦之色,説道︰「不見!」周顚大怒,伸手便是去抓他胸口衣服,説不得舉左手一擋,説道︰「周兄不可莽撞。」彭瑩玉道︰「方丈既是坐関,那麼咱們見見空智、空性兩位神僧。也是一樣。」那知客僧雙手合什,冷冰冰的道︰「不見。」彭瑩王道︰「那麼達摩堂首座呢,羅漢堂首座呢?」那知客僧仍是愛理不理的道︰「不見!」

殷天正猶如霹靂般一聲大喝︰「到底見是不見?」雙掌排山倒海般推出,轟隆一聲,將道旁的一株大松樹推爲兩截,上半截連枝帶葉,再帶著三個烏鴉巢,垮喇喇的倒將下來。那知客僧至此臉上才有懼色,説道︰「各位遠道來此,本來原當禮接,只是諸位長老盡坐関,各位下次再來吧!」説著合什躬身,轉身去了。韋一笑身形一晃,已攔在他的身前,説道︰「大師上下如何稱呼?」那知客僧道︰「不敢,小僧法名慧賢。」明教群豪一聽,無不氣惱,想那「慧」字輩的僧人,是當今少林派中的第三代子弟,連「圓」宇輩的第二代子弟都不派一個下山見客,那實是欺人太甚,此若能忍,孰不可忍?韋一笑伸手在他肩頭輕輕拍了兩下,笑道︰「很好,很好,大師擅説『不見』兩字,不知閻羅王招請佛駕,大師見是不見?」慧賢被他這麼一拍,一般冷氣從肩頭直傳到心口?全身立時寒戰,牙齒互擊,格格作響。他強自忍耐,側身從韋一笑身旁走過,一路不停的抖索,一蹌踉上山。

張無忌道︰「韋蝠王拍了他這兩掌,他師父師叔伯焉能置之不理?咱們逕自上山,瞧這群和尚是否當眞不見?」衆人料想一場惡鬥已是難免,少林派素來是武林中泰山北斗,千年來江湖上號稱「長勝不敗的門派」,今日這一場大戰,且看明教和少林派到底是誰強誰弱,各人精神百倍,快步上山,想到少林寺中高手如雲,眼前這一場大戰,激烈處自是非同小可。

不到一盞茶時分,已到了寺前的石亭。張無忌想起昔年隨太師父上山,在這亭中和少林派三大神僧相見,今日重來,雖然前後不過數年,但昔年是個瘦骨伶仃的病童,今日却是明教教主之尊,緬懷舊事,當眞是恍然隔世。群豪在石亭中稍候半刻,料想寺中必有大批高手出來,決當先禮後兵,責問殷利亨如此痛下毒手,是何原由,衆僧若是蠻不講理,那時再行動武不遲。豈知等了半天,寺中竟是靜悄悄的絶無動靜。

張無忌道︰「進寺去!」當下楊逍、韋一笑往左,殷天正、殷野王在右,鐵冠道人、彭瑩玉、周顚、説不得四散人在後,擁著張無忌進了寺門。來到大雄寶殿,但見佛像莊嚴,殿上一塵不染,佛像前香煙繚繞,琉璃燈中火光瑩然,就是不見一人。張無忌朗聲説道︰「明教張無忌,會同座下楊逍、殷天正,韋一笑諸人前來拜山,求見方丈大師。」他説的聲音雖然並不甚響,但中氣充沛,内力渾厚,一兩里内都能清清楚楚的聽到,殿旁高懸的銅鐘大鼓受這聲音激盪,{\upstsl{嗡}}{\upstsl{嗡}}{\upstsl{嗡}}的響了起來。

楊逍、韋一笑等相互對望一眼,心中均想︰「教主内力之深,實是聞所未聞,當年楊教主在世,也是遠有不及。看來今日之戰,本教可操必勝。」張無忌這幾句話,少林寺前院後院,到處都可聽見,但等了半晌,寺中竟無一人出來。周顚大聲喝道︰「醜媳婦終須見翁姑,少林堂堂門派,難道這般藏頭縮尾,便能躱一輩子麼?」他的話聲可比張無忌説的響得多了,但殿上鐘鼓却無應聲。

群豪又等片刻,仍是不見有人出來。殷天正道︰「管他們安排下什麼詭計,咱們且闖進去!」群豪轟然道好。殷天正大踏步當先,走進後院,只見闖處靜悄悄地,不見有一個僧人的影子。群豪越來越是驚詫,均知以少林派如此一個久享盛名的偌大門派,寺中武功卓絶的僧人固然極多,而智謀之士亦復不少,今日佈了這個「空寺計」,定然伏下極厲害的陰謀,各人心中的戒備也是每走一步,便提高了一層。待得走到伽藍殿口仍是不見有人,韋一笑向布袋和尚道︰「説不得,你我二人上高掠陣!」説不得一點頭,縱身而起,待得雙足落在屋簷,只見韋一笑已在屋頂的三丈以外,心下暗自嘆服︰「韋蝠王輕功之精,我布袋和尚永遠趕他不上。」只聽周顚在底下大呼︰「喂,少林寺的和老兄,這般躱起來成什麼樣子?扮新娘子嗎?」

張無忌和群豪一殿一院的搜尋下去。始終不見有一名僧人的蹤跡,而任何異狀亦未發見。到得羅漢堂中,那是少林派高手精研武技的所在,這時見到壁上留著刀槍劍戟等兵刃長年懸掛過的痕跡,兵刃却已盡數取去。明教群豪不再説話,快步走入達摩堂,只見地下整整齊齊的放著九個蒲團,都已坐得半爛,堂中再無别物。楊逍道︰「向聞達摩堂中所居者,乃是少林派的前輩耆宿,有的十年不出堂門一步,怎能不經一戰,便見本教而遠避?」彭瑩玉道︰「我心中忽有異感、只覺這寺中陰氣沉沉,大大不祥。」周顚笑道︰「和尚進廟,得其所哉,有什麼異感?」張無忌想起昔日跟圓眞學練「少林九陽功」的情景,道︰「咱們到那邀去瞧瞧。」領著群豪,逕到圓眞當年靜修之處,但見牆壁上宛然留著圓眞用手掌壓破的那個掌印,只是人亡室空,四壁肅然。

周顚突道︰「滿寺和尚逃得清光,想必光明頂一戰。教主威名遠揚,少林派掛了免戰牌啦!」楊逍道︰「咱們到藏經閣瞧瞧!」

到得山後藏經閣,但見一排排的都是空木架,數千數萬巻佛經已不知去向。群豪相顧茫然,猜不透其中源由。若説少林派避禍逃遁,難道竟甘心捨棄這經營千餘年的基業?再説,就算首腦人物走了,留下若干火工、沙彌守寺打掃,明教群豪到來之時,也決不會跟這些人爲難,妄加殺戮。難道是生怕留下活口,被明教逼問之下,洩漏秘密麼?

衆人回到大雄寶殿,韋一笑和説不得也分别回來,説道四下搜尋,未有發見,連適纔那知客僧也是不知去向,竟似突然間土遁而去一般。楊逍轉身出殿,召了厚土旗掌旗使顏垣進來,命他率領旗下教衆,四下搜集有無地窖、地道之類秘密藏身之所。顏垣應命而去,過了兩個時辰,回殿稟報,説道到處都已詳加插査,並無秘密藏身的所在,有幾處坐関靜修的密室,築於極隱僻之處,但室中空空,並無人居。那顏垣精於土木構築之學,旗下教衆有不少是高手匠人,經厚土旗嚴密査過,少林寺自是一所空寺無疑了。楊逍、殷天正、彭瑩玉等都是見多識廣、足智多謀之士,此刻見了這等異像,却誰也猜不透少林派在鬧什麼玄虛、安排下什麼惡辣詭計。

衆人正自群疑滿腹、面面相觀之際,猛聽得西邉喇喇一聲響,數十丈外的一株松樹倒了下來。、群豪吃了一驚,同時躍起身來來。奔到斷樹之處,只見那株松樹生於一個大院子之旁,院子中並無人跡,却不知如何,偌大一株松樹,竟會無風自倒,壓塌了半堵圍牆。衆人走近斷截處一看,只見脈絡交錯斷裂,顯是被人以重手法震碎,只是樹脈斷裂處略現乾枯,並非適纔所爲。群豪仔細觀察周遭,只聽得「咦,不對!」

\qyh{}啊,這裡動過手。」各種聲音此起彼落。原來這大院之中,到處都有激烈戰鬥過的痕跡,地下青石板上,旁邉樹枝樹幹上、圍牆石壁上,留下不少兵刃砍斬、舉掌劈擊的印記。這些記印尚甚新鮮,不過是兩三日内之事,但顯而易見,動手過招的都是第一流高手,石板上還有許多淺淺的脚印,乃是高手此拚内力時所留下。

韋一笑伏地聞嗅氣息,更發現了許多所在有血腥之氣,只是昨日剛下過一場大雨,因之洗得乾乾淨淨。這大院子空空曠曠,適才明教群豪已見院中無人,並不再加細擦,倘若不是那株松樹因受掌風撞擊而於此時倒下,誰也不致到這院子中來詳加査看。

彭瑩玉道︰「楊左使,你説如何?」楊逍道︰「三四日之前,少林寺中必定經過一場慘烈非常的激鬥。那是絶無可疑的。難道少林派全軍覆没,竟被殺戮得一個不存?」彭瑩玉道︰「我意正和楊左使相同。依這事勢推斷,必當如此,可是少林派的對頭之中,又那裡有這樣厲害的一個幫會門派?莫非是丐幫?」周顚道︰「丐幫勢力雖大,高手雖多,總也不能一舉便把少林寺中的衆光頭殺得一個不剩。除非是咱們明教,纔有這等本事,可是本教明明没有幹這件事啊?」鐵冠道人道中︰「周顚你少説廢話成不成,本教有没有幹這事,難道咱們自己不知道?」

不料周顚這句話聽來似是廢話,却提醒了楊逍一件事,他「啊」的一聲叫,説道︰「教主,咱們再到達摩院中瞧一下。」張無忌知他既説此話,必有原曲,點頭道,「好!」群豪快步來到達摩院中,只見院中地下仍是放著那九個破爛蒲團,一尊達摩祖師的石像,高高供在神座之上,背脊向外,臉面朝壁,那是紀念達摩祖師當年面壁九年,因而豁然貫通、參悟武學精要,這典故武林中人個個皆知,誰也不以爲奇。周顚道︰「咱們適纔來看,就是這副模樣,那有什麼希奇?」楊逍向殷野王道︰「殷世兄,你助我一臂之力,將那達摩石像扳轉身來看看。」殷天正道︰「這個不妥!」須知達摩祖師是少林寺的創建之人,乃禪宗傳來中土的初祖,不但少林派奉若神聖,而天下武林人物,也是人人不敢冒犯,楊逍道︰「鷹王放心,萬事由小弟一人承當!」説著縱身一躍,上了神座,伸手便去扳那石像。只是那石像太過沉重,一時扳之不動。殷天正道︰「野王,你去助楊左使一臂之力。」殷野王應聲躍上,兩人一齊使力,將那具二千餘斤重的大石像扳了過來。

群豪一見,臉上盡皆變色,只見那具佛像顏面已削成一塊平板,五官全然不見,上面却刻著四行大字︰「先誅少林、再滅武當,唯我明教,武林稱王!」這十六個字顯然是以指力刻劃,深入石理。殷天正、鐵冠道人、周顚等不約而同的一齊叫了出︰「這是遺禍江東的毒計!」楊逍和殷野王躍下神座,周顚道︰「鐵冠牛鼻,倘若不是我那句話,楊左使怎能想得到敵人的移禍之計。」鐵冠道人憂心忡忡,那有心情跟他鬥口。問楊逍道︰「楊左使,你怎地想得到石像中會有古怪?」楊逍道︰「適纔我來達摩院時,已看到這石像曾有移動的痕跡,可是那裡想得到其中竟藏著這麼一個天大的陰謀。」

彭瑩玉道︰「小僧尚有一事不明,要請左使指教。用手指刻下這十六字之人,既是存心教禍本教,使本教承坦毀滅少林派的大罪名,好讓天下武林群起而攻,然則他何以又使達摩佛像面向牆壁?倘若不是楊左使細心,那不是誰也没發現石像上會有這一十六個字麼?」楊逍臉色凝重,説道︰「這石像是另外有人給轉過去的,暗中有一位武功高強之士,在相助本教,咱們已領了人家極大的情,直到此刻方知。」群豪齊聲道︰「此人是誰?楊左使從何得知?」楊逍嘆道︰「這其中的原委曲折,我也猜想不透\dash{}」他這句話尚未説完,張無忌突然「啊」的一聲,大叫起來,説道︰「佛像上的字跡説道︰『先誅少林、再滅武當』,料想武當即當遭難。」

韋一笑道︰「咱們義不容辭,立即赴援,且看誣衊本教的到底是那一批狗奴才。殷天正也道︰「事不宜遲,大夥立即出發。看來這批奸賊已先走了數日。」張無忌想起武當山自太師父以下,個個對自己恩重如山,又不知宋遠橋等是否已從西域回歸本山,這一路上始終不聽到他們的音訊,倘若途中有什麼耽擱變故,那麼留守本山的只有太師父和若干第三代弟子,三師伯兪岱岩殘廢在床,強敵突然來攻,却如何抵敵?想到此處,不由得憂心如焚,朗聲道︰「各位前輩、兄長、武當派乃先父出身之所,今當大難,若有失閃,本座日後難以爲人。救兵如救火,早到一刻好一刻,現請韋蝠王陪同本座,先行赴援,各位陸續分批趕來,一切請楊左使和外公指揮安排。」説著雙手一拱,閃身出門。韋一笑展開輕功,和他並肩而行,群豪答應之聲未出,兩人已到了少林寺外的石亭之中。這兩人輕功之佳、奔馳之速,當世再無第三人能修及上。

到得嵩山脚下,天色漸黑,兩人那要敢有片刻耽擱,足不停步的急奔,直走了一夜,已奔出數百里之遙。韋一笑初時毫不落後,但時候一長,内力漸漸不繼。張無忌心想︰「要到武當山上,至少還得一日一夜的急馳,血肉之軀,究竟不能無窮無盡的奔跑不息,何況強敵在前,尚須留下精力大戰。」於是對韋一笑道︰「韋蝠王,咱們到前面。市鎭上去買兩匹坐騎,歇一歇力。」韋一笑早有此意,只是不便出口,便道︰「教主,買賣坐騎,太耗辰光。」過不多時,迎面便有五六乘馬馳來。韋一笑縱身而起,早將兩個乘者提起,輕輕放在地下,叫道︰「教主,上吧。」張無忌微一遲疑,覺得如此攔路劫馬,豈非和強盜無異?韋一笑叫道︰「處大者事者不拘小節,那顧得這許多?」呼喝聲中又將兩名乘者提下馬來。那幾人倒也會一點武功,紛紛喝罵,抽出兵刃便欲和韋一笑動手,韋一笑雙手勒住四匹馬匹,將那些人的兵刃踢得亂飛。只聽一人喝道︰「逞兇行劫的是那一路好漢,快留下萬児來!」張無忌心想糾纏下去,只有更加多得罪人,縱身躍上馬背,和韋一笑手中各牽一馬,絶塵而去。那些人破口大罵,却是不敢來追。

張無忌道︰「咱們雖然迫於無奈,但焉知人家不是身有急事,此舉究屬於心不安。」韋一笑笑道︰「教主,這些小事,何足道哉,昔年明教行事,那纔稱得上『肆無忌憚、橫行不法』呢!」説著哈哈大笑。張無忌心想︰「明教被視人爲邪魔異端,自有來由。可是到底何者爲是,何者爲邪,却也不易下個確論。」想起身負教主重任,但見識膚淺,很多事拿不定主意,雖然武功極強,可是天下事豈能一切盡以武力解決,他騎在馬背之上,心下茫然,只盼早日接得謝遜歸來,便可卸却自己難以勝任的擔子。便在此時,突見人影一晃,兩個人攔在當路。

\chapter{武當報訊}

韋一笑和張無忌將馬一勒,只見攔在馬前的是兩名乞丐,每人手中均執一杖,背負布袋,却是丐幫中人,韋一笑喝道︰「讓開!」馬鞭攔腰捲去,縱馬便衝。一丐舉杖擋開馬鞭,另一名乞丐忽哨一聲,左手一揚,韋一笑的坐騎受驚,人立起來。便在此時,樹叢中又竄出四名乞丐,看這諸人身法,竟是丐幫中的硬手。韋一笑叫道︰「教主只管趕路,待屬下跟鼠輩糾纏。」張無忌見這些丐幫人物意在阻住武當派的救兵,用心極是惡毒,由此可知,武當的處境實是極爲危險,心知韋一笑的輕功武技並臻佳妙,與這一干人周旋,縱然不勝,至少足以自保,當下雙腿一挾,催馬前衝。兩名丐幫的幫衆橫過鋼杖,攔住馬頭,張無忌俯身向外,挾手便將兩根鋼杖奪了過來,順手一擲、只聽得啊啊兩聲慘呼,這兩名丐幫子弟已被鋼杖各自打斷了大腿骨,倒在地下。無忌本無傷人之意,只是見纏住韋一笑的那四人武功著實不弱,只怕自己走後,韋一笑更增強敵,於是幫他料理了兩個。

嵩山和武當山雖然分處豫郡兩省,但一在河南西部,一在湖北北隅,相距並不甚遠。一過馬山口後,自一路都是平野,馬匹奔跑,更是迅速,中午時分,過了内鄕。張無忌腹中飢餓,便在一處市集上買些麵餅充飢,忽聽得背後牽著的坐騎一聲悲嘶。無忌回過頭來,只見那馬肚腹上已插了一柄明晃晃的尖刀,一個人影在街口一晃,立即隱去,正是敵人向他坐騎下了毒手。無忌飛身過去,一把抓起那人,只見又是一名丐幫子弟,前襟上兀自濺滿了馬血。張無忌怒從心起,一伸手閉了牠的「大椎穴」,叫他周身酸痛難當,挨苦三日三夜方罷。

此時坐騎已死,身邉又無銀兩,一搜那乞丐身邉,倒有一百多兩銀子,無忌説道︰「你殺我坐騎,以此賠還。」回身在市集中牽了一匹雄駿的大馬,抛下銀兩,也不理馬主肯是不肯,縱馬使行。一口氣奔到漢水塘上的三官殿,幸好江邉泊著一艘大渡船。無忌牽馬上船,命梢公搖到對岸。船至中流,無忌望著滔滔江水,不禁想起那日太師父擕同自己在少林寺求醫而歸,在漢水上遇到常遇春、又救了一個周姓女孩的事來。正自出神,突覺船身搖晃,船艙中{\upstsl{汩}}{\upstsl{汩}}的湧進水來。張無忌生長冰火島上,精通水性,原也不怕沉船,只是這麼一來,又是多所耽擱,一轉首,只見船尾那梢公滿臉獰笑,湧身正要躍入江中。無忌身法快極,那梢公身子甫行躍起,它被他長臂在半空中抓住,伸指在他脅下一戳,那梢公殺豬價叫喊起來。張無忌喝道︰「快搖櫓,若再弄鬼,戳瞎你的雙目!」那梢公不敢不依,只得使勁搖嚕。

無忌剝下那梢公的衣褲,塞在船艙中漏水之處。勉強掙到南岸,無忌抓住梢公胸口,問道︰「是誰命你行此毒計,快快説來。」那梢公道︰「小人是丐幫\dash{}」無忌不等他説完,縱馬向南。此時天色早黑,望出來一片朦朧,再行得一個時辰,更是星月無光,那坐騎疲累已極,再也無法支持,跪倒在地下。無忌拍拍馬背,説道︰「馬児。馬児,你在這児歇歇,自行去吧!」他展開輕功,疾向南馳。

行到四更時分,忽聽得前面隱隱有馬蹄之聲,顯是有大幫人衆,無忌加快脚步,從這群人身旁掠過。他身法太快太輕,又在黑夜之中,竟是無人知覺。瞧這群人的行向,正是往武當山而去,二十餘人不發一言,無法探知是什麼來頭,但隱約可見這些人均是擕帶兵刃,此去乃是和武當派爲敵,絶無可疑。無忌一見,心中反寬︰「我畢竟將他們追上了,那麼武當派當是尚未遭難。」再走不到半個時辰,前面又有一群人往武當山而去,如此先先後後,一共遇見了五批人。

待看到第五批上武當山去的武林中人之後,張無忌心下忽又憂急︰「却不知已有幾批人上了山去?是否已有人和本派中人動手?」張無忌雖然並非武當派弟子,但他因父親的淵源,向來將武當派視作自己的門派。這麼一想,奔跑得更加快了,將到半山,忽聽得前面有幾人追逐呼叱之聲。無忌從山側繞了過去,只見前面共有四條黑影,其一在前,三個白衣人往後追趕。追趕的三人之中,一個人長聲喝道︰「兀那和尚,你上武當山來幹什麼?」又一個道︰「你便是報了訊息給武當派知道,又有什麼用處?」只聽得嗤嗤聲響,有人用暗器向前面那人射了過去,聽那暗器之聲甚勁,發射者的手力大是不弱。

無忌心想︰「原來前面那人是向本派報訊的朋友,後面追趕的三人乃是敵人,企圖攔截。」他搶到頭裡,隱在山側的樹叢之中,片刻間前邉奔跑之人已縱到身側,只見他光頭大袖,果然是個僧人。他手執一柄黑黝黝的戒刀,將激射而至的暗器一一抽打在地,左足一跛一拐,顯已受傷,接著後面三人追趕而至,無忌從樹叢中望將出來,看得分明,追趕的三個人手執鋼杖,明明是丐幫人物,穿著明教的白袍。無忌心下暗暗惱怒︰「爹爹曾説,丐幫當年行俠仗義,在老幫主九指神丐洪七公率領之下,威震江湖,乃是中原第一個大幫會。豈知傅到今日,幫中弟子幹的盡是不肖之事。」眼見前面那僧人脚步蹣跚,奔跑得漸漸慢了下來。一名丐幫子弟喝道︰「你少林派已然全軍覆没,諒你這漏網之魚、斧底游魂又成得什麼氣候,快快束手投降,我明教尚可留一條生路給你。」無忌聽他冒認明教之名,心下更怒。

前面那僧人似乎料知逃跑不脱,回轉身來,橫刀而立,喝道︰「罪惡滔天的明教邪魔,且看你們橫行得到幾時,佛爺今日跟你們拚了。」三名丐幫的部衆揮杖而上,登時將他圍在垓心,一招一式,盡往他身上要害招呼。無忌見那少林僧的武功甚是了得,以一敵三,絲毫不落下風,鬥到酣處,喝一聲︰「著!」一刀將一名丐幫弟子的右臂砍了下來,乘著餘下二人一愕之際,反手一刀,又砍中另一人的肩頭,剩下一人駭然敗退,不敢再追。無忌見這少林僧刀法精奇狠辣,不禁暗讚一聲︰「好!」

那少林僧回過身來,更不喘息,提氣便向山上踏步走去。無忌心道︰「明教和少林、武當派嫌隙未解,何況又有人從中冒名爲惡,自己倘若貿然出面,只怕更增糾紛。此刻時機緊迫,不能多耗無謂的辰光,我且暗中跟隨在後,相機援助。」只見那少林僧一路上山,快到山頂,只聽得一人喝道︰「是那一路的朋友,光降武當?」喊聲甫畢,山石後閃出四個人來,兩道兩俗,當是武當派的第三四代弟子了。那少林僧合什説道︰「少林僧人空相,有急事求見武當張眞人。」無忌聽這人自己報名爲「空相」,心下微微一怔︰「原來他是『空』字輩的,和空聞方丈、空智、空性三大神僧是師兄弟輩,怪不得武功如此卓絶。雖然比之空性神僧似乎稍有不及,但也是極爲難得的高手了。」轉念又想︰「若不是空字輩的一流好手,怎能在少林派傾覆之際,獨脱大難?他不辭艱辛的上武當來報訊,確是盡了俠義之道。」

武當派的一名道人説道︰「大師遠來辛苦,請移步觀中奉茶。」説著在前引路,空相將戒刀交給了另一名道人,以示不敢擕帶兵刃進觀。張無忌在山上住過數年,到處山石樹木,無不熟悉、見那道人將空相引人三清殿,便蹲在長窗之外,只聽空相大聲説道︰「道長快請稟報張眞人,事在緊急,片刻延緩不得!」

那道人稽首道︰「大師來得不巧,敝師祖自去歳坐関,至今一年有餘,本派弟子亦已久不見他老人家慈範。」空相皺起眉頭道︰「如此則請通報宋大俠。」那道人道︰「大師伯率同家師及諸位師叔,和貴派聯盟,遠征明教未返。」張無忌聽得「遠征明教未返」六字,心頭暗自吃驚,知道宋遠橋等在歸途之中,也必遇到了阻難,只聽空相長嘆一聲道︰「如此説來,武當派也和我少林一般,今日難逃此劫了。」那道人不明他這一嘆的奇意,説道︰「敝派事務,現由洞玄子師兄主持,小道即替大師通報。」空相道︰「洞玄道長是那一位的弟子?」那道人道︰「是兪三師叔門下。」空相長眉一軒,道︰「兪三俠身子雖然殘廢,心中可是明白,老僧這幾句跟兪三俠説了罷。」那道人道︰「是,謹遵大師的吩咐。」轉身入内。

那空相在廳上踱來踱去,心下極是不耐,時時側耳傾聽,是否敵人已攻到了山上,過不多時,那道人快步出來,躬身説道︰「兪三師叔有請。兪三師叔言道,請大師恕他不能出迎之罪。」這時那道人的神態舉止之中,比先前越加恭謹,想是兪岱岩一聽得有「空」字輩的少林僧駕臨,已囑咐他必須禮貌十分周到。空相點了點頭,隨著他走向兪岱岩的臥房。張無忌站在窗外,尋思︰「三師伯四肢殘廢,耳目加倍靈敏,我若到他窗外竊聽,只怕被他發覺。」走到離兪岱岩臥房數丈之處,便停住了脚步。

過了約莫一盞荼時分,那道人匆匆從兪岱岩房中出來,低聲叫道︰「清風、明月!到這邉來。」便有兩個道僮走到他的身前,叫了聲︰「師叔!」那道人道︰「預備軟椅,三師叔要出來。」兩名道僮答應了。張無忌在武當山上住過數年,那知客的道人是兪蓮舟新收的弟子,他不相識,却識得清風、明月兩個道僮,知道兪岱岩有時出來,便坐在軟椅之中,由道僮抬著行走。見二僮走向放軟椅的廂房,當下悄悄跟隨在後,一等二僮進房,突然叫道︰「清風,明月,認得我麼?」

二僮嚇了一跳,凝目瞧無忌時,依稀有些面熟,一時却認不出來。無忌笑道︰「我是無忌小師叔啊,你們忘了麼?」二僮登時憶起舊事,心中大喜,叫道︰「啊,小師叔,你回來啦!你的病好了?」三個人年紀相若,同在山上之時,常在一處玩耍,翻觔斗、豁虎跳、鬥蟋蟀、捉田雞,無所不爲。無忌道︰「清風,讓我來假扮你,去抬三師伯,瞧他知不知道。」清風躊躇道︰「這個\dash{}不大好吧!」無忌道︰「三師伯見我病愈歸來,自是喜出望外,高興還來不及,那裡會責罵於你。」二僮素知自張三丰祖師以下,武當六俠個個對這位小師叔極其寵愛,他病愈歸山。那是天大的喜事,他要開這個小小的玩笑,逗兪岱岩病中一樂,自無傷大雅。明月道︰「小師叔怎麼説,就怎麼辦吧!」清風當下笑嘻嘻的脱下道袍、鞋襪,給無忌換上了,明月則替他挽起一個道髻,片刻之間,一個翩翩公子,變成了眉清目秀的小道僮。

明月道︰「你要冒充清風,相貌不像,就説是觀中新收的小道僮,清風跌跛了腿,由你去替他。那你叫什麼名字?」無忌想了一想,笑道︰「清風一吹,樹葉便落,我叫掃葉。」清風拍手道︰「這名字倒好\dash{}」三人正説得高興,那道人在房外喝罵︰「兩個小傢伙,嘻嘻哈哈的搗什麼鬼,半天不見人出來。」無忌和明月伸了伸舌頭,抬起軟椅,逕往兪岱岩房中。

無忌和明月扶起兪岱岩,放在軟椅之中,只見兪岱岩臉色極是鄭重,也没留神抬他的道僮是誰,只聽他説道︰「到後山小院,見祖師爺爺去。」

清風應道︰「是!」轉過身去,抬著軟椅前端,無忌抬了後端,兪岱岩只瞧見清風的背影,便瞧不見無忌。空相隨在軟椅之側,同到後山,那知客道人不得兪岱岩召喚,便不敢同去。張三丰閉関靜修的小院在後山竹林深處,修篁森森,綠蔭遍地,除了偶聞鳥語之外,竟是半點聲息也無。

清風和無忌抬著兪岱岩來到小院之前,停下軟椅,兪岱岩正要開聲求見,忽聽得隔門傳出張三丰蒼老的聲音道︰「少林派那一位神僧光臨寒居,老道未克遠迎,還請恕罪。」呀的一聲,竹門推開,張三丰緩步而出。空相心頭微微一驚︰「他怎麼知道來訪的是少林僧人?」但隨即想起︰「想必那知客道人早已命人前來稟報,張三丰老道故弄玄虛。」兪岱岩却知師父近年來武功越來越是博大精深,從空相的脚步聲中,已可測知他的武學門派、修爲深淺。但猜他是空聞、空智,空性少林三大神僧中的一位,却是猜錯了,想是空相少出寺門,外界均不知少林派中有這樣一位武學高手。張無忌的内功此時已在空相之上,由實返虛,自眞歸樸,不論舉止、眼光、脚步、語聲,處處深藏不露,張三丰反聽不出來。他見太師父雖然紅光滿面,但白鬚白髮,此之八九年前分手之時,著實已蒼老了幾分,心中又是歡喜,又是悲傷,忍不住眼泪便要奪眶而出,急忙轉過頭去。

只聽空相合什説道︰「小僧少林空相,參見武當前輩張眞人。」張三丰稽首還禮,道︰「不敢,大師不必多禮,請進説話。」五個人一起進了小院,但見室内板桌上一把茶壺,二隻茶杯,地下一個蒲團,壁上掛著一柄木劍,此外一無所有。空相道︰「張眞人,少林派慘遭千年未遇之浩劫,魔教突施偸襲,本派自方丈空聞師兄以下,或殉寺戰死,或力屈被擒,僅小僧一人拚死逃脱。魔教大隊人衆,正向武當而來,今日中原武林存亡榮辱?全繫於張眞人一人之手。」説著放聲大哭。

饒他張三丰百年修爲,猛地裡聽到這個噩耗,也是大吃了一驚,半晌説不出話來,定了定神,纔道︰「魔教竟然如此猖狂,少林寺中高手如雲,不知如何竟會著了魔教的毒手?」空相道︰「空智、空性兩位師兄率同門下弟子,和中原五大派結盟西征,圍攻光明頂,不知如何失手,盡遭擒獲\dash{}」張無忌暗暗心驚,尋思︰「敵人究竟是誰?怎地這等厲害?」只聽空相續道︰「這日山下報道,遠征人衆大勝而歸,方丈空聞師兄得訊大喜,率同合寺弟子,迎出山門,果見空智、空性兩位師兄帶領西征弟子,回進寺來,另外還押著數百名俘虜。來人到得大院之中,方丈問起得勝情由,空智師兄唯唯否否,空性師兄忽地叫道︰「師兄留神,我等落入人手。衆俘虜盡是敵人\dash{}」方丈驚愕之間,衆俘虜抽出兵刃,突然動手。本派人衆一來措手不及,二來無一人擕帶兵刃,赤手空拳的禦敵,大院子的前後出路均已被敵人堵死,一場激鬥,終於落了個一敗塗地,空性師兄當場殉難\dash{}」説到這裡。已是泣不成聲。張三丰心下黯然,説道︰「這魔教如此歹毒,行此惡計,又有誰能彀提防?」

只見空相伸手解下背上的黃布包袱,打開包袱,裡面是一層油布,再打開油布,赫然露出一顆首級,環眼圓睜,臉露憤怒之色,正是少林三大神僧之一的空性大師。張三丰、兪岱岩、張無忌都識得空性面目,一見之下,不禁「啊」的一聲,一齊叫了出來。空相泣道︰「我捨命搶得空性師兄的法體,張眞人,你説這大仇如何得報?」説著將空性的首級恭恭敬敬放在桌上。伏地拜倒。張三丰淒然躬身,稽首行禮。

張無忌想起光明頂上此武較量之際,這位空性神僧慷慨磊落,豪氣過人,實不愧爲堂堂少林派的一代宗師,不意慘遭奸人戕害,落得身首分離,心下甚是難過。他性情溫和,轉過了頭,不敢多看空性的首級。張三丰見空相伏地久久不起,哭泣甚哀,便伸手扶他,説道︰「空相師兄,少林武當本是一家,此仇非報不可\dash{}」他剛説到這個「可」字,只聽得砰的一聲響,空相雙掌一齊擊在他小腹之上。

這一下變故突如其來,張三丰武功之深,雖已到了從心所欲、無不如意的最高境界,但那能料到這位身負血仇、遠來報訊的少林高僧,竟會對自己忽施襲擊?在一瞬之間,他還道空相悲傷過度心智迷糊,昏亂之中將自己當作了敵人,但隨即知道不對,小腹上中的掌力,竟是少林派外門神功「金剛般若掌」,但覺空相竭盡全身之勁,將掌力不絶的催送過來,臉白如紙,嘴角却帶獰笑。

張無忌、兪岱岩、明月三人驀地見此變故,也都驚得呆了。兪岱岩苦在身子殘廢,不能上前相助師父一臂之力,張無忌年輕識淺,在這一刹那間,還没領會到空相竟是意欲立斃太師父於掌底,兩人只驚呼了一聲,便見張三丰左掌揮出,拍的一聲輕響,擊在空相的天靈蓋上。這一掌其軟如綿,其堅勝鐵,空相登時腦骨粉碎,如一堆濕泥般癱了下來,一聲也没哼出,便即斃命。須知張三丰過百年的修爲,功力通神,空相雖是當代武林中一流高手,却也經不起他這輕輕一掌。

兪岱岩忙道︰「師父,你\dash{}」只説了一個字,較即住口,只見張三丰閉目坐下,片刻之間,頭頂冒出絲絲白氣,知他正以上乘内功療傷。猛地裡張三丰口一張,噴出幾口鮮血,無忌在旁見著,心下大驚,知道太師父受傷著實不輕,倘若他吐出的是紫黑瘀血,那麼憑他深厚無此的内功,三數日即可平復,但他所吐的却是鮮血,況是狂噴而出,那麼臟腑已受重傷。在這霎時之間,張無忌心中轉過了無數念頭︰「是否立即表明身份,相救太師父?」便在此時,只聽得門外脚步聲響,那知客道人到了門外,聽他步聲急促,顯是心情十分慌亂,却是不敢貿然進來,也不敢出聲。兪岱岩道︰「是藏玄麼?什麼事?」那知客道人藏玄説道︰「稟報三師叔得知,大批敵人到了觀外,都是穿著魔教的服色,要見祖師爺爺,口出汚穢言語,説要踏平武當派\dash{}」兪岱岩喝道︰「住口!」他生怕張三丰聽到了之後分心,激動傷勢。

張三丰緩緩睜開眼來,説道︰「少林派金剛般若掌的威力果是非同小可,看來非得靜養三月。傷勢難愈。」無忌心道︰「原來太師父所受之傷,比我所料的更重。」只聽張三丰又道︰「明教大舉上山,乃是有備而來。唉,不知遠橋、蓮舟他們平安否?岱岩,你説該當如何?」兪岱岩默然不語,心知武當山上除了師父和自己之外,其餘三四代弟子的武功都不足道,出面禦敵,只有徒然送死,今日之事,只有自己捨却一命,和敵人敷衍周旋,經師父避地養傷,日後再復大仇,於朗聲説道︰「藏玄,你去跟那些人説,我便出來相見,請他們在三清殿上小坐片刻。」藏玄答應著去了。

張三丰和愈岱岩師徒相處日久,心意相通,聽兪岱岩這麼説,已知通他的用意。説道︰「岱岩,生死榮辱,無足介懷,武當派的絶學却不可因此中斷。我坐関十八月,得悟武學精要,一套太極拳和太極劍,此刻便傳了你吧。」兪岱岩一呆,心想自己殘廢已久,那裡還能學什麼拳法劍術?何況此時強敵已經入觀,那裡還有餘暇傳授武功,只叫了聲︰「師父!」便説不下去了。

張三丰淡淡一笑,説道︰「我武當開派以來,行俠江湖,多行仁義之事,以大數而言,決不該自此而絶。我這套太極拳、太極劍,與古來武學之道全然不同,講究以靜制動、後發制人。你師父年過百齡,縱使不遇強敵,又能有幾年好活?所喜者能於垂暮之年,創制這套武功出來。遠橋、蓮舟、松溪、利亨、聲谷都不在身邉,第三四代子弟之中,除青書外並無傑出人材,何況他也不在山上。岱岩,你身負傳我生平絶藝的重任。武當派一日的榮辱,有何足道?只須這套太極拳能傳至後代,我武當派大名必能傳之千古。」説到這裡,神采飛揚,豪氣彌增,竟是没將壓境的強敵放在心上。

兪岱岩唯唯答應,知道師父要自己忍辱負重,以接傳本派絶技爲第一要義。張三丰至胸前左臂半環,掌與面對成陰掌,右掌翻過成陽掌,説道︰「這是太極拳的起手式。」緩緩站起身來,雙手下垂,手背向外,手指微舒,兩足分開平行,接著兩臂慢慢提起,跟著一招一式的演了下去,口中叫著招式的名稱︰攬雀尾、單鞭、提手上勢、白鶴亮翅、摟膝勾步、手揮琵琶、進步搬攔錘、如封似閉、抱虎歸手、十字手\dash{}張無忌目不轉晴的凝神觀看,初時還道太師父故意將姿式演得特明緩慢,使兪岱岩可以著得清楚,但看到第七招「手揮琵琶勢」之時,只見他左掌陽、右掌陰,目光凝視左手手臂,雙雙慢慢推出,竟是凝重如山,却又輕靈似羽,張無忌突然之間省悟︰「這是以慢打快,以靜制動的上乘武學,想不到世間竟會有如此奥妙的功夫。」他武功本是極高,一經領會,登時越看越是入神,但見張三丰雙手圓轉,每一招都含著太極式的陰陽變化。這是從中國固有哲理中變化出來的武學,與來自天竺達摩祖師的武功大異其趣,雖然未必便能勝過,但精微之處,却是決不遜色。

約莫一頓飯時分,張三丰使到上步高探馬,上步攬雀尾,單鞭而合太極,神定氣閒的站在當地,雖在重傷之後,但一拳術練完,精神反見健旺。他雙手抱了個太極式的圓圏,説道︰「這套拳術的訣竅是『虛靈頂勁、涵胸拔背、鬆腰垂臀、沉肩墜肘』十六個字,純以意行,最忌用力,形神合一,那便是舉術的最高境界。」當下細細的解釋了一遍。兪岱岩一言不發的傾聽,知道時勢緊迫,無暇發問,雖然中間不明白之處極多,但只有硬生生的記住,倘若師父有甚不測,這些口訣招式總是由自己傳了下去,日後再當由聰明才智之士領悟文中的精奥。張無忌所領略的可就多了,須知「乾坤大挪移法」根本之主旨實與太極拳有異曲同工之妙,都是借力打力,雖然法門大異,却是殊途同歸。張三丰的每一句口訣、每一記招式,他心中事先隱隱約約都是已然想到,一説出來,立時便有大獲我心之感。

張三丰見兪岱岩臉有迷惘之色,問道︰「你懂了幾成?」兪岱岩道︰「弟子愚魯,只懂得三四成,但招式和口訣都記住了。」張三丰道︰「那也難爲你了。倘若遠橋在此,當能懂得五成。唉,你五師弟悟性最高,可惜不幸早亡,我若有五年功夫,好好點撥於他,當可傳我這門絶技。張無忌聽他提到自己父親,心中不禁一酸。張三丰道︰「這拳勁首要似鬆非鬆,將展未展,勁斷意不斷\dash{}」正要往下解釋,只聽有前面三清殿上傳來一個蒼勁悠長的聲音,喝道︰「張三丰老道既然縮頭不出,咱們把他徒子徒孫先行宰了。」另一個粗豪的聲音道︰「好啊!先一把火燒了這道觀再説。」又有一個尖鋭的聲音道︰「燒死老道,那是便宜他。咱們擒住了他,綁到各處門派中遊行示衆,讓大家瞧瞧這武學泰斗的模樣。」

後山外院和前殿相距里許之遙,但這幾個人的語聲都清楚傳至,足見敵人有意炫示功力,而功力確亦不凡。兪岱岩聽到這等侮辱師尊的言語,心下大怒,眼中如要噴出火來。張三丰道︰「岱岩,我叮囑過你的言語,怎麼轉眼便即忘了?不能忍辱,豈能負重?」兪岱岩道︰「是,謹奉師傅教誨。」張三丰道︰「你全身殘廢,敵人不會對你如何提防,千萬戒急戒躁。倘若我苦心創製的絶藝不能傳之後世,那你便是我武當的罪人了。」兪岱岩只聽得全身出了一陣冷汗,知道師父此言的用意,不論敵人對他師徒如何凌辱欺侮,總之是要苟免求生,忍辱傳藝。

張三丰從身邉摸出一對鐵鑄的羅漢來,給兪岱岩道︰「這空相説道少林派已經滅絶,也不知是眞是假,此人是少林派中高手,連他也投降敵人,前來暗算於我,那麼少林派必遭大難無疑。這對鐵羅漢是百年前郭襄郭女俠贈送於我。你日後那還少林傳人。就盼從這對鐵羅漢身上,傳留少林派的一項絶藝!」説著大袖一揮,走出門去。兪岱岩道︰「抬我跟著師父。」明月和無忌二人抬起兪岱岩,跟在張三丰的後面。

四個人到得三清殿上,只見殿中坐著的、站著的,黑壓壓的都是人頭,總有三四百人之多。張三丰居中一站,打個稽首,却不説話。兪岱岩大聲道︰「這位是我師尊張眞人。各位群上武當,不知有何見教?」

張三丰的大名威震武林,人人的目光都集於他的身上,但見他一襲灰布道袍,白髮如銀,除了身材十分高大之外,也無特殊異狀。張無忌看這干人時,只見半數穿著明教教衆的服色,爲首的十餘人却各穿本服,想是自高身份,不願冒充旁人,高矮僧俗,數百人擁在殿中,看得眼都花了。便在此時,忽聽得門外有人傳報道︰「教主到!」殿中衆人一聽,立時肅靜無聲,爲首的十多人搶先出殿迎接,餘人也跟著快步出殿,霎時之間,大殿中數百人走了個乾乾淨淨。只聽得十餘人的脚步聲自遠而近,走到殿外停住,張無忌從殿門中望去,不禁吃了一驚,只見八個人抬著一座黃緞大轎,另有七八人前後擁衛,停在門口,那抬轎的八個轎夫不是旁人,正是綠柳莊的那「神箭八雄」。張無忌心中一動,雙手在地下一抹,抹了雙掌灰土,跟著便滿滿的塗在臉上。明月見他塗成這等鬼臉,又是好笑,又是驚惶,只道他眼見大敵到來,是以扮成這副模樣,一時心中無主,也便依樣葫蘆,灰土抹臉,兩個小道僮登時變成了灶君菩薩一般再也瞧不出本來面目。

轎門掀起,從驕子中走出一個少年公子,一身白袍,袍上繡著一道血紅的火燄,輕搖摺扇,正是女扮男裝的趙明。也走進殿中,神箭八雄等在外侍候,只十餘個首領人物跟進了殿來。一個身材魁梧的漢子踏上一步,恭恭敬敬的躬身説道︰「啓稟教主,這個就是武當派的張三丰老道,那個殘廢人想必是他的第三弟子兪岱岩。」趙明點點頭,上前幾步,收攏摺扇,一揖到地,説道︰「晩生執掌明教張無忌,得見武林中山斗之望,幸也何如!」張無忌又是一驚,心中罵道︰「這賊丫頭冒充明教教主,那也罷了,居然還來冒用我的名字,當面欺騙太師父。」張三丰聽到「張無忌」三字,也覺奇怪︰「怎麼魔教教主是如此年輕俊美的一個少女,名字偏又和我那無忌孩児相同,」當下稽首還禮,説道︰「不知教主大駕光臨,未克遠迎,還請恕罪,」趙明道︰「好説,好説!」

知客道人藏玄率領火工道僮。獻上茶來。趙明一人坐在椅中,她手下衆人遠遠的垂手站在其後,不敢走近她身旁五尺之内,似乎生怕不敬,冒瀆於她。

\chapter{狹路相逢}

張三丰百載的修爲,謙沖恬退,早已萬事不縈於懷,但師徒情深,對宋遠橋等人的生死安危,却是十分牽掛,當即説道︰「老道的幾個徒児不自量力,曾赴貴教討教高招,迄今未歸,不知彼等下落如何,還請張教主明示。」趙明嘻嘻一笑,説道︰「宋大俠、兪二俠、張四俠、莫七俠四位,目下是在本教手中。每個人受了點傷,性命却是無礙。」張三丰道︰「受了點傷?多半是中了點毒。」趙明笑道︰「張眞人對武當絶學,可也自負得緊。你既説他們中毒,那就算是中毒吧。」須知張三丰深知幾個徒児的武功,個個已是當世一流好手,就算衆寡不敵,總能有幾人脱身回報,此刻既是一鼓成擒,定是中了敵人無影無蹤,難以防避的毒藥無疑。趙明見他猜中,也不隱瞞,隨即坦然相告。張三丰又問︰「我那姓殷的小徒呢?」趙明嘆道︰「殷六俠中了少林派的埋伏,便和這位兪三俠一模一樣,四肢爲大力金剛指折斷。死是死不了,可就動也動不得!」張三丰聽了這句話,鑒貌辨色,情知趙明之言非虛,心頭一痛,哇的一聲,噴了一口鮮血出來。

趙明背後衆人相顧色喜,知道空相已然偸襲得手,這位武當高人已受重傷,強敵既去,那更是無所忌憚了。

趙明道︰「我有一句良言相勸,不知張眞人肯俯聽否?」張三丰道︰「教主請説。」趙明道︰「普天之下,莫非王土,率土之濱、莫非王臣。」我蒙古皇帝威加四海。張眞人若能投效皇室,皇上立頒殊封,武當派自當大蒙榮寵,宋大俠等人人無悉,更趕不在話下。」張三丰抬頭望著屋樑,冷冷的道︰「明教雖然多行不義,胡作非爲,却是向來和元人作對,是幾時投效了皇室啦?老道倒是孤陋寡聞得緊。」趙明道︰「棄暗投明,自來識時務者爲俊傑。少林派自空聞、空智神僧以下,個個投效,盡忠朝廷。本教也不過見大勢所趨,追隨天下賢豪之後而已,何足奇哉?」張三丰雙目如電,望到趙明臉上,説道︰「元人殘暴,多害百姓,如今天下群雄並起,正是要驅逐胡虜,還我河山。凡我黃帝子孫,無不有著個驅除韃子之心,這纔是大勢所趨,老道雖是方外的出家人,却也知大義所在,空聞、空智乃當世神僧,豈爲勢力所屈?這位姑娘何以説話顚三倒四如此?」

趙明身後突然閃出一個身材魁梧的漢子,大聲喝道︰「兀那老道,言語不如輕重!武當派毀滅就在眼前,你不怕死,難道這山上百餘名遭人弟子,個個都不怕死麼?」這人説話中氣充沛,身高膀闊,形相極是威武。張三丰長聲吟道︰「人生自古誰無死,留取丹心照汗青!」這是文天祥的兩句詩句,文天祥慷慨就義之時,張三丰正當年青,對這位英雄丞相極是欽仰,常常自嘆其時武功未成,否則必當捨命去救他出難,此刻面臨生死関頭,自自然然的吟出文天祥這兩句詩來。他頓了一頓,又道︰「其實文丞相也是不免有所拘執?但求我自丹心一片,管他日後史書如何書冩!」望了兪岱岩一眼,心道︰「我却盼這套太極拳能彀流傳後世,何嘗不是和文丞相一般,顧全身後之名?唉!管他能傳不能傳,武當派能存不能存!」

趙明白玉般的左手輕輕一揮,那大漢躬身退到了她的身後。她微微一笑,説道︰「張眞人既是如此固執,暫且不必説了。就請各位一起跟我走吧!」説著站起身來,她身後四個人身形晃動,團團將張三丰圍住。這四人一個便是那魁梧大漠,一個穿著乞児的鶉衣,當是丐幫中的高手,一個是身形瘦削的和尚,另一個却是中年女子。張無忌見這四人的身法或凝重、或飄逸,個個非同小可,心頭一驚︰「這位趙姑娘怎地手下竟有如許高手?」

眼見張三丰若是不跟隨而去。那四個人便要出手抓他,張無忌心想︰「敵方高手甚衆,這一班人又盡是奸詐無恥、不顧信義之輩、非圍攻光明頂的六大派可比。我就算是擊敗了其中數人,他們也決計不肯服輸,一擁而上,總之難以保護太師父和三師伯的平安。但事已至此,只有竭力一拚,别無善策了。」他正要挺身而出,喝阻那四人,忽聽得門外陰惻惻的一聲長笑,一個青色人影閃了進來,這人身法如鬼如魅,如風如電,倏忽間欺身到那魁梧漢子的身後,一掌拍出。那大漢武功甚是了得,知道有敵來襲,更不轉身,反手便是一掌,意欲和他互拚硬功。豈知那人不待此招打老,左手已拍到那中年婦人的肩頭。那婦人閃身躱避,裙底飛出一腿,踢他小腹,那人早已攻向那名僧人。瞬息之間,他連出四掌,攻擊了四大高手,雖然每一掌都没打中,但手法之高,眞是匪夷所思。這四人情知到了勁敵,各自躍開數步,凝神接戰。

那青衣人站在張三丰身旁,並不理會敵人,却是躬身向張三丰拜了下去,説道︰「明教張教主座下晩輩韋一笑,參見張眞人!」原來這人正是韋一笑。他擺脱了途中敵人的糾纏,兼程趕至。

張三丰聽他説自稱是「明教張教主座下」,還道他也是趙明一黨,伸手擊退四人,多半另有陰謀,當下冷冷的道︰「韋先生不必多禮,久仰青翼蝠王輕功絶頂,世所罕有,今日一見,果是名不虛傳。」韋一笑大喜,他少到中原,素來聲名不響,豈知這位張三丰居然知道自己輕功了得的名頭,躬身説道︰「張眞人武林山斗,晩輩得眞人稱讚一句,當眞是榮於華袞。」他轉過身來,指著趙明道︰「趙姑娘,你鬼鬼祟祟的冒充明教,敗壞本教的聲名,到底是何用意?是男子漢大丈夫,何必如此陰險毒辣?」趙明格格一笑,説道︰「我本來不是男子漢大丈夫,陰險毒辣了,你便怎樣?」韋一笑第一句便説錯了,被她駁得無言可對,一怔之下,説道︰「各位到底是何來歷?先攻少林,再擾武當,倘若各位和少林武當有怨有仇,明教原有不該管閒事,但各位冒我明教之名,喬扮本教教衆,我韋一笑不能不管!」

張三丰原本不信百年來和朝廷作死敵的明教,竟會投降蒙古,聽了韋一笑這幾句話,這纔明白,心想︰「魔教雖然聲名不佳,遇上這等大事,究竟毫不含糊。」

趙明向那魁梧大漢説道︰「你聽他吹這等大氣,你去試試,瞧他有什麼眞才實學。」那大漢躬身道︰「是!」收了收腰間的鸞帶,穩步走到大殿中間,説道︰「韋蝠王,在下領教你的寒冰綿掌功夫!」韋一笑不禁心頭一驚︰「這人怎地知道我的寒冰綿掌?他明知我有此技,仍是上前挑戰,倒是不可輕敵。」雙掌一拍,説道︰「請教閣下的萬児?」那人道︰「咱們既是冒充明教而來,難道還能以眞名示人?蝠王這一問,未免太笨。」趙明身後的十餘人一齊大笑起來。韋一笑冷冷的道︰「不錯,是我問得笨了。閣下甘作朝廷鷹犬,做異族奴才,還是不説姓名的好,没的辱没了祖宗。」那大漢臉上一紅,怒氣上升,呼的一掌,便往韋一笑胸口拍去,竟是中宮直進,逕取要害。

韋一笑脚步錯動,早已避過,身形閃處,一指戳向他嘴心,他不先出寒冰綿掌,要先探一探這大漢的深淺虛實。那大漢左臂後揮,守中含攻。數招一過,那大漢掌勢漸快,韋一笑只覺他掌風之中,隱含一股熱氣,往他手掌一看,只見他雙掌掌心已變得血也似紅,心中一動︰「莫非這是硃砂七煞掌的功夫?這種功夫聽説早已失傳,這漢子是什麼來歷,居然會使這種奇異的掌法?」

眼見對方的掌力越來越是厲害,韋一笑知道自己内傷雖經張無忌治好,不必像從前那樣,運功一久,便即飲人血抑制體内陰毒,但傷癒未久,即逢強敵,實是絲毫不敢怠慢,雙掌一錯,將寒冰綿掌的功夫使了出來。兩人掌勢漸緩,逐步到了互較内力的境地,突然間呼的一聲,大門中擲進一團黑越越的大物,猛向那大漢身上衝去。這一團物事比一大袋米還大,天下居然有這等巨大的暗器。當眞奇了。那大漢運勁拍出一掌,將這一團物事擊出丈許,著手之處,只覺軟綿綿地也不如是什麼東西。但聽得「啊」的一聲慘呼,原來有人藏在袋中,身中那大漢勁力凌厲無儔的硃砂七煞掌,焉有不筋折骨斷之理?

那大漢一愕之下,全身一震,早已被韋一笑無聲無息的欺到身後,在他背心「大椎穴」上拍了一招「寒冰綿掌」。那大漢又驚又怒,急轉身軀,奮力一掌往韋一笑頭頂擊去。韋一笑藝高人膽大,哈哈一笑,竟然不避不讓,那大漢掌到中途。手臂已然酸軟無力,一掌雖然擊在韋一笑的天靈蓋上,那裡有半點勁力,只不過是如同替他輕輕一抹一般。原來韋一笑的寒冰綿掌一經著身,對方勁力立卸,但高手對戰,竟敢任由強敵掌擊腦門,膽氣之豪,實是從所未聞,旁觀衆人看在眼裡,無不驟然。倘若那大漢竟有抵禦寒冰綿掌之術,勁力一時不去,這一掌打在他的頭頂豈不腦漿迸裂?但韋一笑一生行事古古怪怪,愈是旁人不敢爲,不肯爲、不屑爲之事,他愈是幹得興高睬烈,津津有味。他乘那大漢分心之際出擊偸襲,本來已有點不彀光明正大,可是他跟著便以腦門坦然受他一掌,却又是光明正大過了火,到了膽大妄爲、視生死如児戲的地步。

在這一瞬之間,那丐幫高手已然扯開布袋,拉出一個人來,只見他滿臉血紅,早在硃砂七煞掌掌力的一擊之下斃命,此人衣衫破爛,正是丐幫子弟,不知如何,却被人裝在布袋中擲了進來。那丐幫高手大怒,喝道︰「是誰鬼鬼祟祟\dash{}」一語未畢,一隻白茫茫的袋子已兜頭罩到,他提氣後躍,避開了這一罩。只見一個胖大和尚笑嘻嘻的站在身前,正是布袋和尚説不得到了。他的乾坤一氣袋被張無忌在光明頂上迸破,没了趁手的兵器,只得胡亂做幾隻布袋應用,究竟没原來那隻刀劍不破的乾坤袋厲害。他輕功雖然稍有不及韋一笑處,但造詣也是極高,加之中途没受阻撓,前脚後脚的便趕到了。

説不得也回過身向張三丰行禮,説道︰「明教張教主座下、遊行散人布袋和尚説不得,參見武當掌教祖師張眞人。」張三丰還禮道︰「大師遠來辛苦。」説不得道︰「敝教光明使者,白眉旗白眉鷹王、以及四散人、五旗使,各路人馬,都已上了武當。張眞人你且袖手旁觀,瞧明教上下,和這批冒名作惡的無恥之徒一較高低。」其實他這番話只是虛張聲勢,明教大批人衆未能這麼快便全體趕到。但趙明聽在耳裡,秀眉微蹙,心想︰「他們居然來得這麼快,是誰洩漏了機密?」忍不住問道︰「你們張教主呢?叫他來見我。」説著向韋一笑望了一眼,目光中有疑問之色,顯是問他教主到了何處。張無忌一直隱身在明月之後,知道連韋一笑和説不得迄未認出自己,眼見到了這兩個得力的幫手,心中極是喜慰。

趙明冷笑道︰「一隻毒蝙蝠,一個臭和尚,成得什麼氣候?」一言甫畢,忽聽得東邉屋角一人長笑問道︰「説不得,楊左使到了没有?」這人聲者響亮,蒼勁豪邁,正是白眉鷹王殷天正到了。説不得尚未回答,楊逍的笑聲已在西邉角上響起。

只聽楊逍笑道︰「鷹王,究竟是你老當益壯,先到了一步。」殷天正笑道︰「楊左使不必客氣,咱二人同時到達,仍是分不了高下。只怕你還是瞧在張教主份上,讓了我三分。」楊逍道︰「當仁不讓!晩輩已竭盡全力,仍是不能快得鷹王一步。」原來他二人途中較勁,比賽脚力,殷天正内力較深,楊逍步履輕快,竟是並肩出發,平頭齊到。長笑聲中,兩人一齊從屋角縱落。張三丰久聞殷天正的名頭,何況他又是張翠山的岳父,當下走上三步,拱手道︰「張三丰恭迎殷兄、楊兄的大駕。」心中却頗爲不解︰「殷天正明明是白眉教的教主,又説什麼『瞧在張教主份上』?」

殷楊二人躬身行禮,齊聲道︰「久仰張眞人清名,無緣拜見,今日得睹芝顏,三生有幸。」張三丰道︰「兩位均是一代宗師,大駕同臨,洵是盛會。」趙明心中愈益惱怒,眼見明教的高手越來越多,張無忌雖然尚未現身,只怕説不得所言不虛,確是在暗中策劃,佈置下什麼厲害的陣勢,自己安排得妥妥貼貼的計謀,看來今日且難以成功,但好容易將張三丰打得重傷,這是千載難逢、絶無第二次的良機,今日若不乘此機會收拾了武當派,日後待他養好了傷,那便棘手之極了,一雙漆黑溜圓的眼珠轉了兩轉,冷笑道︰「江湖上傳言武當乃是正大門派,豈知耳聞不如目見,暗中和魔教勾勾搭搭,全仗魔教撐腰,本門武功可説不値一哂。」説不得道︰「趙姑娘,你這是婦人之見,小児之識,張眞人威震武林之時,只怕你祖父都未出世,小孩児懂得什麼?」

趙明身後的十餘人一齊踏上一步,向説不得怒目而視。説不得揚揚自若,笑道︰「你們説我這句話説不得麼?我名字叫作『説不得』,説話却向來是説得又説得,諒你們也奈何我不得。」趙明手下的瘦削僧人怒道︰「主人,待屬下將這多嘴多舌的和尚料理了!」説不得叫道︰「妙極,妙極,你是野和尚,我也是野和尚,咱們來比拚比拚,請武當宗師張眞人指點一下不到之處,勝過咱們苦練十年。」説著雙手一揮,從懷中又抖了一隻布袋出來,旁人見他布袋一隻又是一隻,取之不盡,不知他僧袍底下,到底還有多少隻布袋。

趙明微微搖頭,道︰「今日咱們是來討教武當絶學,不論是武當派那一位下場,咱們都樂於奉陪,武當派究竟是有眞才實學,還是浪得虛名,今日一戰便可天下盡知。至於明教和咱們的過節,日後再慢慢算賬不遲。張無忌那小鬼奸詐狡猾,我不抽他筋、剝他皮,難消心頭之恨,可也不忙在一時。」張三丰聽到「張無忌那小鬼」六個字時心中大奇︰「明教的教主難道眞的也叫做張無忌?怎地又是「小鬼』?」説不得笑嘻嘻地道︰「本教張教主少年英雄,你趙姑娘只怕比咱們張數主還小著幾歳,不如嫁了咱們教主,我和尚看來倒也相配\dash{}」他話未説完,趙明身後衆人已轟雷般怒喝起來︰「胡説八道!」

\qyh{}住咀!」

\qyh{}野和尚放狗屁!」趙明紅暈雙頰,容顏嬌艷無倫,神色之中只有三分薄怒,倒有七分靦腆,一個呼叱群豪的首領,霎時之間成一怔怔作態的小姑娘。但這神氣也只是頃刻間有事,她微一凝神,臉上便如罩了一層寒霜,向張三丰道︰「張眞人,你不肯露一手,便留一句話,只説武當派乃是欺世盜名之輩,咱們大夥児拍手便走。便是將宋遠橋、兪蓮舟這批殺之汚刀的鼠輩放還給你,又有何妨?」

便在此時,鐵冠道人和殷野王先後趕到,不久周顚和彭瑩玉也到了山上,明教這邉又增了四個好手。趙明估量形勢,情知雙方決戰,未必能操勝算,最擔心的還是怕張無忌在暗中作什麼手脚。

趙明的眼光在明教諸人的臉上掃了一轉,心想︰「張三丰所以成爲朝廷心腹之患,乃是由於他近百年來盛名不衰,被人奉爲武林中的泰山北斗。若憑他這等風燭殘年,還能活得多少時候?咱們也不須取他性命,只是折辱武當派一番,此行便是大功告成了。」於是冷冷的道︰「咱們造訪武當,原是想領教張眞人的武功到底是眞是假,若是和明教群毆,難道咱們不認得光明頂的道路麼?這樣吧,是眞的假不了,是假的也眞不得!這裡有三個跟隨我多年的家人,一個學過幾招三脚貓的拳脚,一個會得一點粗淺内功,再有一個練過幾天殺豬屠狗的劍法。阿大、阿二、阿三,你們站出來,張眞人只須將我這三個不中用的家人打發了,咱們佩服武當派的武功,確是名下無虛。要不然嘛,江湖上自有公論,那也不用我多説。」説著雙手一拍,她身後緩步走出三個人來。

只見那阿大是個精乾枯瘦的老者,雙手捧著一柄長劍,赫然便是那柄倚天寶劍,這人滿臉皺紋,又矮又小,愁眉苦臉,似乎剛纔給人痛毆了一頓,要不然便是新死了妻子児女,旁人只要瞧他臉上神情,幾乎便要代他傷心落泪。那阿二同樣的枯瘦,身材却高得多,頭頂心滑油油的,禿得不剩半根頭髮,兩邉太陽穴凹了進去,深陥半寸。那阿三却是精壯結實,虎虎有威,臉上、手上、項頸之中,凡是可見到肌肉處口盡皆盤根紮結,似乎周身都是精力,脹得要爆炸出來,張三丰、殷天正、楊逍等人一看,心下都是一驚,周顚説道︰「趙姑娘,這三位都是武林中頂児尖児的高手,我周顚便一個也鬥不過,怎地不識羞的喬裝了家人,來開張眞人的玩笑麼?」趙明道︰「他們是武林中頂児尖児的高手?我倒也不知道。他們叫什麼名字啊?」周顚被她一問,登時語塞,但隨即打個哈哈,道︰「這位是『一劍震天下』矮神君,{\upstsl{嗯}},這位是『丹氣霸八方』禿頭天王。至於這一位嘛,天下無人不如,那個不曉,嘿嘿,嘿嘿,乃是\dash{}那個『神拳無敵』猛尊者。」

趙明聽他瞎説八道的胡謅,不禁{\upstsl{噗}}{\upstsl{哧}}一笑,説道︰「我家裡三個煮飯烹茶、抹桌掃地的家人,什麼神君、天王的!張眞人,你先跟我家的阿三比比拳脚吧。」那阿三踏上一步,抱拳道︰「張眞人請!」左足一蹬,喀喇一聲響,蹬碎了地下三塊方磚。著脚之處的青磚被他蹬碎並不希奇,難在鄰近的兩塊方磚竟被這一脚之力震得粉碎。楊逍和韋一笑對望一眼,心中都道︰「好傢伙!」那阿大阿二兩人緩緩退開中低下了頭,向衆人瞧都不瞧。這三人自進殿後,一直跟在趙明身後,只是始終垂目低頭,神情猥瑣,誰也没加留神,不料就這麼向前一站,登如淵停嶽峙,儼然大宗匠的氣派,但退了回去時,却又是一副畏畏縮縮、傭僕厮養的模樣。殷天正心想︰「張眞人身受重傷,就算不傷,他這麼大的年紀,怎能和這等人拚鬥拳脚?瞧此人武功,純是剛猛一路,讓我來接他一接。」當下朗聲説道︰「張眞人何等身份,豈能和低三下四之輩動手過招?這不是天大的笑話?别説是張眞人,就算我姓殷的,哼哼,諒這些奴才也不配受我一拳一脚。」他明知阿大、阿二、阿三,決非庸流,但偏要將他們説得十分不堪,好將事情攬到自己身上。

趙明道︰「阿三,你從前叫什麼名字,自己還記得麼?説給他們聽聽,且看配本配和武當高人動手過招。」她言語之中,始終緊緊的扣住了「武當」二字。那阿三道︰「小人自投靠主人之後,從前名字早就不用了。既是主人有命,小人不敢不説。我從前複姓宇文,單名一個『策』字。」這宇文策三字一出口,衆人心中都是一凜。

殷天正大聲道︰「宇文策,宇文策!二十年前,長安城中薛氏五雄是你所殺的麼?那天晩上紅衣蒙面、自稱『八臂神魔宇文策』,壽筵中連殺十三高手,是你幹的好事麼?」宇文策冷冰冰的道︰「你老人家倒好記心,我自己都忘記了,你倒原原本本的記著。」衆人一聽他直認不諱,無不大怒。須知長安城薛氏五雄武功既高,爲人又是慷慨仗義,突然間一晩中被一個蒙面的紅衣怪人盡數擊斃,成爲武林中轟傳的大事。那一晩喪於這自己報名爲宇文策的紅衣怪客手下者,除薛氏五雄外,另有華山、峨嵋派中的幾位高手,大家査不到宇文策的來歷,便把這筆帳算在明教名下者有之,算在白眉教名下者亦有之。殷天正等雖和薛氏五雄没有交情,却爲此事很吃了個啞巴苦頭,被人打了悶棍喊不出冤。想不到事隔二十年,眞正的兇手這纔出頭。

這宇文策雖然除了這件事外,在中原武林中迄未第二次露面,但單只這件事,已足以使人聞名而懼,憑著「八臂神魔字文策」七個字,不但足可和張三丰一較高下,而且即使他不向武當掌門人挑戰,張三丰既如他現身,自非主持武林公道不可。他説出「宇文策」三個字來,可説已是逼得張三丰更無置之不理的餘地。殷天正大聲道︰「好!你既是八臂神魔,讓姓殷的來鬥上一鬥,倒是一件快事。」説著搶上兩步,拉開了架子。宇文策道︰「殷天正,你是邪魔外道,我宇文策是外道邪魔。咱倆一鼻孔出氣,自己人不打自己人。你要打,咱們另揀日手來比過。今日主人有命,只令小人試試武當派功夫的虛實。」他轉頭向張三手道︰「張眞人,你要是不想下場,只説一句話便可交代,咱們決計不敢動蠻硬逼。」

張三丰微微一笑,心想自己雖然身受重傷,但若施以自己新創太極拳中「以虛擊實」的上乘武學法門,未必較輸於他,所難對付者,倒是擊敗阿三之後,那阿二便要上前比拚内力,這却絲毫取巧不得,自己傷後運不得内力,這一関決計無法過去,但火燒眉毛,且顧眼下,只有打發了這宇文策再説。當下緩步走到殿心,向殷天正道︰「殷兄美意,貧道心領。貧道近年來創了一套拳術,叫作『太極拳』,自覺和一般武學頗有不同處。這位宇文施主定要印證武當派功夫,殷兄若是將他打敗,諒他心有不甘。貧道就以太極拳中的招數,和這位宇文施主拆幾手,且看貧道的多年心血,是否不値方家一哂。」殷天正聽他如此説,又是喜歡,又是擔憂,心想他言語之中,對這套「太極拳」頗具自信,張三丰是何等樣人,既出此言,自有把握,否則豈能輕墜一世的威名,但他重傷嘔血,人人親見,只怕拳技雖精,終究是内力難支,當下也不便多言,只得抱拳道︰「晩輩恭睹張眞人神技。」

宇文策見張三丰居然飄然下場,心下倒起了三分怯意,但轉念又道︰「今日我便是和這老道拚個兩敗倶傷,那也是聳動武林的盛舉了。」當下屏息凝神,雙目{\upstsl{盯}}住在張三丰臉上,内息暗暗轉動,周身骨格便霹霹拍拍,發出輕微的爆響之聲。衆人又是相顧一愕,知道這是外門硬功練到了最上乘境地之象,只聽説少林派的三大神僧有此造詣,不料這八臂神魔居然也具此内勁。何況這種功勁是佛門正宗武功,自外而内,不帶半分邪氣,乃是金剛伏魔的神通,張三丰見到這等神情?也悚然一驚︰「此人來歷不小啊!不知我這太極拳是否對付得了?」當下雙手緩緩舉起,正要讓那八臂神魔進招,忽見兪岱岩身後走出一個蓬頭垢面的小道僮來,説道︰「太師父,這位施主要見識我武當派的拳技,又何必勞動太師父大駕?待弟子演幾招給他瞧瞧,也就彀了。」

這個滿臉塵垢的小道僮,正是張無忌。殷天正、楊逍等人和他分手不久,雖然他此刻衣服形貌全都改變,但一聽聲音,立即認了出來。明教群衆見教主早已在此,心中均是大喜。張三丰和兪岱岩却那裡能彀想到?張三丰一時瞧不清他的面目,只道便是清風,説道︰「這位宇文施主身具金剛伏魔的外門神通,想必是西域少林一支的高手。你小孩児一招之間便被他打得筋折骨裂,豈同児戲?」張無忌左手牽住張三丰的衣角,右手拉著他的左手,輕輕搖晃,説道︰「太師父,你教我的太極拳法從未用過,也不知成是不成。難得這位宇文施主是外家高手,讓弟子試試以柔克剛、運虛打實的法門,那不是很好麼?」説話之間,將一股極渾厚、極柔和的九陽神功,從手掌上向張三丰體内傳了過去。

張三丰在刹那之間,只覺掌心中傳來這股力道雄偉無比,雖然遠不及自己内力的精純,但{\upstsl{汩}}{\upstsl{汩}}然、綿綿然,直是無止無歇、無窮無盡,一驚之下,定晴往無忌臉上瞧去,只見他目光中不露芒華,却隱隱然有一層溫潤晶瑩之意,那是内功已到絶頂之境的現象。生平所遇人物,只有本師覺遠大師、大俠郭靖等寥寥數人,似乎纔有這等修爲,至於當世高人,除了自己之外,想不起再有第二人能臻此境界。霎息之間,他心中轉過了無數疑端,然而無忌的内力沛然而至,顯是在助自己療傷,此人絶無歹意,乃可一定,於是微笑道︰「我衰邁昏庸,能有什麼好功夫教你?不過你要領教宇文施主的絶頂外家功夫,那也是好的,務須小心在意。」他決不知這個小道童就是張無忌,總道是那一派的高手少年,突然趕到赴援,因此言話中極是沖謙客氣。張無忌道︰「太師父,你待孩児恩重如山,孩児便是粉身碎骨,也不足以報太師父和衆位師叔的大恩。我武當派功夫雖不敢説天下無敵,但也不致輸於西城少林的手下。太師父儘管放心。」他這幾句話説得懇摯無比,幾句「太師父」純出自然,決計做作不來,連張三丰也是大是奇怪︰「難道他竟是本門弟子,暗中潛心修爲,就如昔年本師覺遠大師一般?」至於趙明、宇文策等人,更無絲毫疑心。張三丰放下張無忌的手,退了回去,坐在椅中,還目去瞧兪岱岩時,只見他也是一臉迷惘之色。

宇文策見張三丰居然遣這小道僮出戰,對自己之輕蔑藐視,可説已到了極處。可是他是個深沉陰鷙之人,臉上不動聲色,心想我一拳先將這小道童打死,激得老道心浮氣粗,則和他正經動起手來,我當便有制勝把握,當下也不多言,只説︰「小孩児,發招吧!」張無忌道︰「我新學這套拳術,乃我太師父張眞人多年心血所創,叫做『太極拳」。晩輩初學乍練,未必即能領悟拳中精要,三十招之内,恐怕不能將你擊倒。但那是我學藝未精,不是這套拳術不行,這一節你須得明白。」宇文策不怒反笑,轉頭向阿大、阿二道︰「大哥、二哥,天下竟有這等狂妄的小子。」阿二縱聲大笑,阿大目光鋭利却已瞧出無忌不是易與之輩,説道︰「三弟,不可輕敵。」

宇文策踏上一步,呼的一拳,便往張無忌胸口打到,這一招神速如電的拳招遞出,不料拳到中途,他左手拳更快的搶上,後發先至,撞擊張無忌面門,招術上之詭異,實是罕見。張無忌自聽張三丰演説「太極拳」之後,一個多時辰中,始終在默想這套拳術的拳理,見宇文策左拳擊到,使出太極拳中一招「攬雀尾」,右脚實,左脚虛,運個「棚」字訣,粘連粘隨,右掌已搭住敵人左腕,橫勁撥出,宇文策身不由主的向前一衝,跨出兩步,方始站定。旁觀衆人見此情景,齊聲驚噫。

\chapter{廿載恩仇}

這一招「攬雀尾」,乃是天地間有自太極拳以來,第一次和人過招動手。張無忌身具九陽神功,精擅乾坤大挪移之術,突然使出太極拳中「粘」的功夫,雖然所學還不到兩個時辰。却已如畢生研習一般。宇文策給他這麼一棚,這一拳中千百斤的力氣,猶似打入了汪洋大海,無影無蹤,無聲無息,反而自己身子被自己的拳力帶得斜跌兩步。他一驚之下,怒氣填膺,快拳連攻,只見臂影晃動,便似數十條手臂、數十個拳頭同時擊向張無忌一般。三清殿上除了明月一個小道僮外,餘下個個都是高手,見了他這等狂風驟雨般的攻勢,心中無不驚歎︰「八臂神魔,名不虛傳,無怪當年長安城中十三高手,都命喪於他拳底。」

張無忌有意要顯揚武當派的威名,自己本身武功一槩不用,招招都用張三丰所創太極拳的拳招。單鞭、提手上勢、白鶴亮翅。摟膝拗步,待使到一招「手揮琵琶」時,右捺左收,刹時間悟到了太極拳旨中的精微奥妙之處,這一招使得猶如行雲流水,瀟灑無比。宇文策只覺上盤各路全處在他雙掌的籠罩之下,無可閃避,無可抵禦,只有運勁於背,硬接他這一掌,同時右拳猛揮,只盼兩人各受一招,成兩敗傷之局。不料張無忌雙手一圏,如抱太極,一股雄渾無比的力道組成了一個急旋渦,只帶得宇文策在原地急轉七八下,如轉陀螺,如旋紡錘,好容易使出「千斤墜」之力定住身形,已是滿臉脹得通紅,甚是狼狽。

明教群豪大聲喝采。楊逍叫道︰「武當派太極拳功夫如此神妙,眞是今人大開眼界。」周顚笑道︰「宇文老兄,我勸你改個外號,叫做「八臂陀螺』!」殷野王道︰「多轉幾個圏児也不算丟臉,古人不是説『三十六著,轉爲上著』麼?」説不得道︰「當年梁山泊好漢中有個黑旋風,那旋風嘛,原是要轉的!」衆人你一言我一語,宇文策只氣得臉色自紅轉青,怒吼一聲,縱身撲上,左手或拳或掌,變幻莫測,右手却純是手指的功夫,拿抓點戳、勾挖拂挑,五根手指如判官筆,如點穴蹶,如刀如劍,如槍如戟,攻勢凌厲之極,張撫忌太極拳招未熟,登時手忙脚亂,應付不來,突然間嗤的一聲,衣袖被宇文策撕下了一截,只得展開輕功,急奔閃避,暫躱這從所未見的五指功夫。宇文策{\upstsl{吆}}喝追趕,但那裡及得上張無忌輕功的飄逸,接連十餘抓,盡數落空。

無忌一面躱閃,心下轉念︰「我只逃不鬥,豈不是輸了?這太極拳我還不大會使,且以挪移乾坤的功夫,跟他鬥上一鬥。」一個迴身,雙手擺一招太極拳中「野馬分鬃」的架式,左手却已使出乾坤大挪移的手法,{\upstsl{噗}}的一聲響,宇文策右手一指戳向張無忌肩頭,却不知如何被他一帶、竟戳到了自己左手上臂,只痛得眼前金星直冒,一條左臂幾乎提不起來。楊逍瞧出這不是太極拳功夫,却搶先叫道︰「太極拳當眞了得。」宇文策又痛又怒,喝道︰「這是妖法邪術,什麼太極拳了?」刷剛剛連攻三指。無忌縱身避開,眼見宇文策又是長臂疾伸,雙指戳到,他再使挪移乾坤心法,一牽一引,托的一響,宇文策的兩根手指直插進了三清殿中的一根大木柱之中,深至指根。衆人又是吃驚,又是好笑。

衆人轟笑聲中,兪岱岩厲聲喝道︰「且住!宇文策,你這是少林派金剛指力!」張無忌縱身躍開,一聽到「少林派金剛指力」七個字,立時想起,兪岱岩和殷利亨兩人都是爲少林派金剛指力所傷,二十年來,武當派上下都是爲此深怨少林,看來眞兇却是眼前此人。只聽宇文策冷冷的道︰「是金剛指力便怎樣?誰教你硬充好漢,不肯説出屠龍刀的所在?這二十年殘廢的滋味可好受麼?」

兪岱岩厲聲道︰「宇文策,多謝你今日言明眞相,原來我一身殘廢,是你西域少林派下的毒手。只可惜\dash{}只可惜了我的五弟。」要知當年張翠山自刎而死,乃是爲了兪岱岩傷於殷素素的金針之下,無顏面對師兄之故。其實兪岱岩中了金針之後,殷索素託龍門鏢局運回武當,醫治數月,自會痊癒,他所以四肢被人折斷,實出於大力金剛指的毒手,倘若當日找到了這個罪魁禍首,張翠山夫婦也不致慘死了。兪岱岩既悲師弟無辜喪命,又恨自己成爲廢人,滿腔怨毒,眼中如要噴出火來。張無忌聽了兩人之言,立即明白了一切前因後果,他幼時會聽父親説過,少林寺有一火工頭陀偸學武藝,擊死少林寺達摩堂首座苦智禪師,少林派中各高手大起爭執,以致苦慧禪師遠走西域,開創了西域少林一派。

張三丰道︰「宇文施主心腸忒也歹毒,咱們可没想到當年苦慧禪師的傳人之中,竟有施主這等人物。」宇文策獰笑道︰「苦慧!哼,苦慧是什麼東西?」張三丰一聽,恍然大悟。那一年兪岱岩爲大力金剛指所傷後,武當派遣人前往質問少林,少林派掌門方丈堅決不認,便疑心到西域少林一派,但多年打聽,得知西城少林已是式微之極,所傳子弟只是精研佛學,不通武功,此刻聽了宇文策這句「苦慧是什麼東西」,心知他若是西域少林傳人,絶無辱罵開派祖師之理,登時朗聲説道︰「怪不得,怪不得!施主是火工頭陀的傳人,不但學了他的武功,也盡數傳了他狠戾陰毒的性児!怪不得少林派會毀在施主手上。那個空相什麼的,是施主的師兄弟吧」宇文策道︰「不錯,他是我的師兄,他可不叫空相,法名剛相。張眞人,我「金剛門」的般若金剛掌,跟你武當派的掌法比起來怎樣?」兪岱岩厲聲道︰「遠遠不如,他頭頂挨了我師一掌,早已腦漿迸裂。班門弄斧,死有餘辜。」

宇文策大吼一聲,撲將上來。張無忌一招太極拳「如封似閉」,將他擋住,説道︰「宇文策,拿『黑玉斷續膏』來!」説著伸出了右掌。宇文策心中一驚︰「本門的續骨妙藥,秘密之極,連本門的尋常子弟也不知其名,這小道童却從何處聽來?」他那知蝶谷醫仙胡青牛的「醫經」之中,會説到這種藥名。醫經中説道西域有一路外家武功,疑是少林旁支,手法極是怪異,斷人肢骨,無藥可治,僅其本門秘藥「黑玉斷續膏」可救,然此膏如何配製,却是其方不傳。無忌想到此節,順口説了出來,原來也只試他一試之意,待見他臉色陡變,即知所料無誤,只聽宇文策道︰「你這小小道僮如何得知本門秘藥之名?」張無忌道︰「拿來!」他想起了父母之仇,恨不得立時置之於死地,不願跟他多説一句。

宇文策適纔和他交過了手,雖然吃了一點小虧,但見自己的大力金剛指使將出來之時,他只有躱閃逃避,並無還手之力,只要留神他古裡古怪的牽引手法,鬥下去可操必勝,當下踏上一步,喝道︰「小傢伙,你跪下來磕三個響頭,那就饒你,否則這個姓兪的便是榜樣。」張無忌決意要取他的「黑玉斷續膏」,然而如何對付他的金剛指,一時却無善策,乾坤大挪移之法雖可傷他。却不能逼得他取出藥來。心中正自沉吟,張三丰道︰「孩子,你過來!」張無忌道︰「是!太師父。」走到他的身前。

張三丰道︰「用意不用力,太極圓轉,無使斷續。當得機得勢,令對手其根自斷。一招一式,務須節節貫串,如長江大河,滔滔不絶。」他適纔見張無忌臨敵使招,已頗得太極三昧,只是他原來武功太強,拳招之中,稜角分明,未能體會「太極拳」那「圓轉不斷」之意。

張無忌武功已高,関鍵處一點便透,聽了張三丰這幾句話,登時便有領悟,心中虛想著那太極圖圓轉不斷,陰陽變化之意。宇文策冷笑道︰「臨陣學武,未免遲了吧?」張無忌雙眉揚處,説道︰「剛來得及,正好叫閣下試招。」説著轉過身來,右手圓轉向前,朝宇文策面門探去,正是太極拳中一招「高探馬」。宇文策右手中指成刀形砍落,張無忌「雙風貫耳」,連消帶打,雙手成圓形擊出,這一下變招,果然體會了太師父所教「圓轉不斷」四字的精義,左圏右圏,一個圏圏跟著一個圏圏、大圏、小圏、平圏、立圏、正圏、斜圏、一個個太極圖形,只套得宇文策跌跌撞撞,身不由主的立足不穩,猶如中酒昏迷。突然之間,他五指猛力戳出,張無忌使出一招「雲手」,左手高,右手低,一個圓圏已將他手臂套住,九陽神功的剛勁使出,喀喇一聲,宇文策的右臂上下臂骨齊斷。這九陽神功的剛勁好不厲害,宇文策一條手臂的臂骨斷成了六七截,骨骼碎裂,登時不成模樣。以這份勁力而論,却非以柔勁爲主的太極拳所能及。

張無忌恨他歹毒,那「雲手」使出時連綿不斷,有如白雲行空,一個圏圏未完,第二個圏圏已生,又是喀喇一響,宇文策的左臂亦斷,跟著喀喀喀幾聲,他左腿右腿也被一一絞斷。張無忌生平和人動手,從未下過如此辣手,但此人是害死父母、害苦三師伯六師叔的大兇手,若非要著落在他身上取到「黑玉斷續膏」,早已取了他的性命。

宇文策一聲悶哼,已然摔倒。趙明手下早有一人搶出,將他抱起退開。那禿頭的阿二閃身而出,一掌疾向無忌胸口劈來,掌尖未至。無忌已感到一陣令人窒息的壓力,當下一招「斜飛勢」,將他掌力引偏。這禿頭老者一聲不出,下盤穩得如牢釘在地,專心致志,一掌一掌的劈出,内力雄渾無比。無忌見他掌路和宇文策乃是一派,看年紀是宇文策的師兄,武功輕捷不及,却是遠爲沉穩,他試用太極拳中粘、引、棚、按等口訣,想將他身子帶歪,不料這人内力太強,反而粘得自己跌出了一步。張無忌雄心陡起,心想︰「我倒跟你比拚比拚,瞧是你的少林内功厲害,還是我的九陽神功厲害。」見他一掌劈到,便也一掌劈出,那是硬碰硬的蠻打,絲毫没有取巧的餘地,雙掌相交,砰的一聲巨響,兩人身子都晃了一晃。

張三丰「噫」的一聲,心中叫道︰「不好!這等蠻打,力強者勝。正和太極拳的拳理全然相反。這禿頭老者有力渾厚,武林中甚是罕見,只怕這一掌之下,小孩児便受重傷。」便在此時,兩人第二掌再度相交,砰的一聲,那阿二身子一晃,退了一步,張無忌却是神定氣閒的站在當地。

九陽神功和少林派内功都出自達摩祖師。兩者殊途同歸,練到最高境界,可説是不分高下。但「金剛門」的創派師祖火工頭陀乃是從少林寺中偸學的武藝,拳脚兵刃固可偸學、那内功一道講究的是體内氣息運行,你便是眼睜睜的瞧著他打坐靜修,却怎知他内息如何調勻、周天如何搬運?因此外功可偸學,内功却是偸學不來的,這一門的武藝外功極強,不輸於少林正宗,内功却遠遠不及了。這阿二是「金剛門」中的異人,天生神力,由外而内,居然另闢蹊徑,練成了一身極強的内功,其造詣早已超過了當年的師祖火工頭陀,可説乃是天授。他雙掌之下,極少有人接得住三招,此時蠻打硬拚,却被無息的掌力震得退出了一步。不由得又驚又怒,深深吸一口氣?雙掌齊出,同時向無忌劈來,只聽得張無忌叫道︰「殷六叔,你瞧我給你出這口惡氣!」

原來這時殷利亨已在楊不悔、小昭等人陪同之下,由兩名明教教衆用軟兜抬著,到了武當山上,五行旗下諸好手,也是先後趕到。張無忌一聲喝處,右拳揮出,砰的一聲大響,那禿頭阿二連退三步,雙目鼓起,胸口氣血翻湧。無忌説道︰「殷六叔,圍攻你的衆人之中,可有這禿頭在内麼?」殷利亨道︰「不錯!此人正是首惡。」張無忌問得明白,不必再留餘地,只聽那禿頭阿二周身骨節霹霹拍拍的發出響聲,正自運勁。兪岱岩叫道︰「渡河未濟、盤其中流!」意思是叫無忌不等阿二運功完成,就上前攻他一個措手不及。要知兪岱岩見識甚高,知道這阿二内力強猛,這一運功勁,掌力非同小可,實是難擋。

張無忌應道︰「是!」踏上一步,却不出擊,阿二雙臂一振,一股力道排山倒海般推了過來。無忌吸一口氣,體内眞氣流轉,一掌揮出,一拒一迎,將對方掌力盡行碰了回去。那阿二大叫一聲,身子猶似發石機射出的一塊大石,喀喇喇一聲響,撞破牆壁,衝了出去。衆人駭然失色之際,牆壁的破洞中閃進一個人來,提著阿二的身子,放在地下。只見此人矮矮胖胖,圓如石鼓,模樣甚是可笑,身法却極靈活,正是明教厚土旗掌旗使顏垣。那禿頂阿二雙臂臂骨、胸前肋骨、肩頭鎖骨,已盡數被他自己剛猛雄渾的掌力震斷。顏垣放下阿二,向無忌一躬身,又從牆洞中鑽了出去。倏來倏去,便如是一頭肥胖的土鼠。

趙明見這小道童連敗自己手下兩個一流高手,心中早已起疑,一見顏垣向無忌行禮。妙目顧盼,立時認了出來,心中暗罵自己。「該死,該死!我先入爲主,一心以爲這小鬼在外佈置,没想到他竟假裝道僮,在此搗鬼,壞我大事。」當下細聲細氣的道︰「張教主,怎地如此没出息,假扮起小道僮來?滿口太師父長、太師父短,也不害羞。」張無忌見她認出了自己,便朗聲道︰「先父翠山公正是太師父座下的第五弟子,我叫一聲『太師父』有甚麼害羞不害羞?」説著轉身向張三丰,磕頭道︰「孩児張無忌。叩見太師父和三師伯。事出倉卒,未及稟明,還請恕孩児欺瞞之罪。」張三丰和兪岱岩又驚又喜,説什麼也想不到這個力敗「金剛門」二大高手的少年,竟是當年那個面黃肌瘦、病得死去活來的孩童。張三丰呵呵大笑,伸手扶起,説道︰「好孩子,你没有死,翠山可有後了。」轉頭向殷天正道︰「殷兄,恭喜你生了這麼一個好外孫。」殷天正笑道︰「張眞人,恭喜你教出來這麼一位好徒孫。」

趙明罵道︰「什麼好外孫、好徒孫!兩個老不死,養了一個奸詐狡獪的小鬼出來。阿太,你去試試他的劍法。」那滿臉愁苦之色的阿大應道︰「是!」刷的一聲,拔出倚天劍來,各人眼前青光閃閃,寒氣侵人,端的是好一口寶劍。張無忌道︰「此劍是峨嵋派所有。何以到了你的手中?」趙明啐道︰「小鬼,你懂得什麼?滅絶老尼從我家中盜得此劍。此刻物歸原主,倚天劍跟峨嵋派有什麼干係?」張無忌原不知倚天劍的來歷,給她反口一問,竟是答不上來,當下岔開話題,説道︰「趙姑娘,你取『黑玉斷續膏』給我,治好了我二師伯、六師叔的斷肢,咱們既往不咎。」趙明冷笑道︰「既往不咎?説説倒是容易。你知道少林派空聞、空智,武當派的宋遠橋、兪蓮舟他們,此刻都在何處?」張無忌搖頭道︰「我不知道,還請姑娘見示。」趙明冷笑道︰「説得稀鬆平常,我幹麼要跟你説?不將你碎屍萬段,難抵當日綠柳莊鐵牢中,對我輕薄羞辱之罪!」説到「輕薄羞辱」四字,想起當日情景,不由得滿臉飛紅,又惱又羞。

張無忌聽到她説及「輕薄羞辱」四字,臉上也是一紅,心想那日爲了解救明教群豪身上所中之毒,纔不得不出此下策,用手搔她脚底,其實並無絲毫輕薄之意,不過男女授受不親,雖説從權,究是大大的越禮,此事並未和旁人説過,倘若衆人當眞以爲自己調戲人家少女,那可糟了,眼下無可辯白,只得説道︰「趙姑娘,這『黑玉斷續膏』你到底給是不給?」趙明俏目一轉,笑吟吟的道︰「你要黑玉斷續膏,那也不難。只須你依我三件事,我便雙手奉上。」張無忌道︰「那三件事?」趙明道︰「眼下我可還没想起。日後待我想到了,我説一件,你便跟著做一件。」

張無忌道︰「那怎麼成,難道你要我自己殺了自己,要我做豬做狗,我也依你?」趙明笑道︰「我決不會要你殺了自己,更不會叫你做豬做狗,嘻嘻,就是你肯做,也做不來呢。」張無忌道︰「你先説將出來,如果是不違武林中俠義之道,而我又做得到的,那麼依你自也不妨。」趙明正待接口,一轉眼,看到小昭鬢邉插一朶珠花,正是自己送給張無忌的那朶,不禁大惱,又見小昭明眸皓齒,桃笑李妍,年紀雖稚,却出落得猶如曉露芙蓉,甚是惹人憐愛,心下更恨,一咬牙,對阿大道,「把這姓張的小子兩條臂膀斬了下來!」阿大應道︰「是!」一振倚天劍,走上了一步,説道︰「張教主,主人有命,叫我斬下你的兩條臂膀。」周顚心中已蹙了很久,這時再也忍不住了,破口罵道︰「放你娘的狗臭屁!你不如自己斬下自己的雙臂。」阿大滿臉愁容,苦口苦面的道︰「那也説得有理。」周顚這下子可就樂了,大聲道︰「那你快斬啊。」阿大道︰「也不必忙。」

張無忌站在一旁,心中頗爲發愁,這口倚天寶劍鋒鋭無匹,任何兵刃一碰即斷,唯一的迎敵之策,只有以乾坤大挪移法空手奪他兵刃,然而伸手到這等鋒利的寶劍之旁去搶奪,只要對方的劍招稍奇,變化略有不測,自己一條手臂自指尖以至肩頭,不論那一處給劍鋒一帶,立時削斷,如何對敵,倒是頗費躊躇。忽聽張三丰道︰「無忌,我創的太極拳,你已學會了,另有一套太極劍,不妨現下傳了你,和這位施主過過招。」無忌喜道︰「多謝太師父。」轉頭向阿大道︰「這位前輩,我劍術不精,請太師父指點一番,再來跟你過招。」那阿大對張無忌的武功原本暗自忌憚,自己雖有寶劍在手,佔了便宜,究是勝負難知,聽説他要新學劍招,那是再好不過,須知新學的劍招不論如何精妙,總是生疏難熟。劍術之道,講究的是輕翔靈動,至少也得練上一二十年,臨敵時方能得心應手,熟極而流。他點了點頭,説道︰「你去學招吧,我在這裡等你!學兩個時辰彀了吧?」

張三丰道︰「不用到旁的地方,我在這児教,無忌在這児學,即炒即賣,新鮮熱辣。不用半個時辰,一套太極劍法便能教完。」他此言一出,除了張無忌外,人人驚駭,幾乎不相信自己耳朶,均想︰就算武當派的太極劍法再奥妙神奇,但在這裡公然教招,敵人瞧得明明白白。還有什麼秘奥可言?阿大道︰「那也好。我迴避到殿外等候便是。」他竟是不欲佔這個便宜,以傭僕身份,却行武林宗師之事。張三丰道︰「那也不必。我這套劍法初創,也不知管不管用。閣下是劍術名家,正要請閣下瞧瞧,指出其中的缺陥破綻。」楊逍心念一動,突然想起一事,朗聲道︰「閣下原來是「玉面神劍」方長老,閣下以堂堂丐幫長老之尊,何以甘爲旁人厮僕?」明教群豪一聽,都是吃了一驚。周顚道︰「你不是死了麼!怎麼又活轉了,這\dash{}這怎麼可以?」

那阿大悠悠嘆了口氣,低頭説道︰「老朽百死餘生,過去的事説他作甚?我早不是丐幫的長老了。」老一輩的人,都知玉面神劍方東白是丐幫四大長老之首,劍術之精,名動江湖,而且英俊瀟灑,是武林中出名的美男子,十多年前身染重病身亡,當時人人都感惋惜,不意他竟是尚在人世,面目却已大異。

張三丰道︰「老道這路太極劍法能得玉面神劍指點幾招,榮寵無量。無忌,你有佩劍麼?」小昭上前幾步,呈上無忌從趙明處擕來的那柄木製假倚天劍。張三丰接在手裡,笑道︰「是木劍?老道這不是用來畫符捏訣、作法驅邪?」當下站起身來,左手持劍,右手捏個劍訣,雙手成環,緩緩抬起。這起式一展,跟著三環套月、大魁星、燕子抄水、左攔掃、右攔掃\dash{}一招招的演了下來,使到第五十三式「指南針」,雙手同時劃圓,復成第五十四式「持劍歸原」。張無忌小記招式,只是細看他劍招中「神在劍先、綿綿不絶」之意。張三丰一路劍法使完,竟無一人喝采,各人心中均是大感詫異︰「這等慢吞吞、軟綿綿的劍法,如何能用來和人對敵過招?」轉念又想︰「料來張眞人有意放慢了招數,好讓他瞧瞧明白。」

只聽張三丰道︰「孩児,你看清楚了没有?」張無忌道︰「看清楚了。」張三丰道︰「都記得了没有?」張無忌道︰「已忘記了一小半。」張三丰道︰「好,那也難爲了你。你自己去想想吧。」張無忌低頭默想。過了一會,張三丰問道︰「現下怎樣了?」張無忌道︰「已忘記了一大半。」

周顚失聲叫道︰「糟糕!越來越忘記得多了。張眞人,你這路劍法很是深奥,看一遍怎能記得?請你再使一遍給咱教主瞧瞧吧。」張三丰微笑道︰「好,我再使一遍。」、提劍出招,演將起來。衆人只看了數招,心下大奇,原來第二次所使,和第一次使的竟是没一招相同。周顚叫道︰「糟糕,糟糕,這可更加叫人胡塗啦。」張三丰畫劍成圏,問道︰「孩児,怎樣啦?」張無忌道︰「還有三招没忘記。」張三丰收劍歸座,只見張無忌在殿上緩緩踱了一個圏子,抬起頭來,滿臉喜色,叫道︰「這我可全忘了,忘得乾乾淨淨的了。」張三丰道︰「不壞,不壞!忘得眞快,你這就請玉面神劍指教吧!」説著將手中木劍遞了給他。張無忌躬身接過,轉身向方東白道︰「方前輩請。」

周顚爬耳搔頭,滿心擔憂,只見方東白揉身進劍,説道︰「有僭了!」一劍刺到,青光閃處,發出嗤嗤聲響,只見内力之強,實不下於那個禿頭阿二。衆人凜然而驚,心想他手中所持莫説是砍金斷玉的倚天寶劍,便是一根廢銅爛鐵,在這等内力運使之下,也是威不可當,「玉面」雖已迥非昔時,「神劍」兩字却是名不虛傳。張無忌左手劍訣一引,木劍橫過,畫個半圓,平搭在倚天劍的劍脊之上,勁力傳出,倚天劍登時一沉。方東白讚道︰「好劍法!」一抖腕翻劍,劍尖向無忌左脅刺到。無忌迴劍圏轉,拍的一聲,雙劍相交,各自飛身而起。方東白手中的倚天寶劍被這麼一震,顫動不絶,發出{\upstsl{嗡}}{\upstsl{嗡}}之聲,甚是清越。

這兩把兵刃一是寶劍,一是木劍,但平面相交,寶劍和木劍變成了毫無分别,張無忌這一招乃是以己之鈍,擋敵之無鋒,實已得了太極劍法的精奥。要知張三丰所轉給他的乃是「劍意」,而不是「劍招」,要他將所見到的劍招忘得半點不勝,才能得其神髓,臨敵時以意馭劍,千變萬化,無窮無盡。倘若心中尚有一兩招劍法記著忘不乾淨,那麼心有拘囿,劍法便不能純。這意思楊逍、殷天正等高手已隱約懂得,周顚却終於遜了一籌,這纔空白憂急了半天。

這時只聽得殿中嗤嗤之聲大盛,方東白的劍招凌厲狠辣,以極渾厚内力,使極鋒鋭利劍、出極精妙招術,青光盪漾,劍氣瀰漫,殿上衆人便覺有一個大雪團在身前轉動,發出蝕骨寒氣。張無忌的一柄木劍在這團寒光中劃著一個個圓圏,每一招均是以弧形刺出,以弧形收回,他心中竟無半點渣滓,以意運劍,那木劍每發一招,便似放出一條細絲,要去纏在倚天寶劍之上,這些細絲越積越多,像是積成了一團團絲綿,將倚天劍裹了起來。兩人拆到二百餘招之後,方東白的劍招漸見澀滯,手中這柄寶劍好像不斷的在增加重量,五斤、六斤、七斤\dash{}十斤、二十斤\dash{}偶爾一劍刺出,眞力運得不足,便被對方木劍帶著運轉幾個圏子。

方東白越鬥越是害怕,眼看二百招已過,始終無法削斷無忌手中的木劍,激鬥三百招雙方居然劍鋒不交,那是他生平使劍以來從所未遇之事。對方如同已撒出了一張大網,一步步的在向中央收緊。方東白連換了六七套劍術,旁觀衆人瞧得眼都花了,張無忌却總是持劍畫圓,旁人除了張三丰外,没有一個瞧得出他每一招到底是攻是守。他這路太極劍法只是大大小小,正反斜直各種各樣的圓圏、要説招數,可説只有一招,然而變化無絶,應付不窮。猛聽得方東白朗聲長嘯,鬚眉皆豎,倚天劍中宮疾進,那是竭盡身家性命的孤注一擲、乾坤一擊!

張無忌見他來勢猛惡,迴劍擋招,方東白手腕微轉,倚天劍側了過來,擦的一聲輕響,木劍的劍頭已削斷六寸,倚天劍不受絲毫阻撓,直刺到張無忌胸口而來。無忌一驚,左手翻轉,本來捏著劍訣的食中兩指一張,已挾住倚天劍的劍身,右手半截木劍向方東白的右臂斫落。劍雖木製但在他九陽神功運使之下無殊鋼刃。方東白右手運力一奪,那倚天劍被對方兩根手指挾住了,猶如鐵鑄,竟是不動分毫,當此情景之下,他除了撒手鬆劍,向後躍開,再無他途可循。只聽張無忌喝道︰「快撒手!」方東自一咬牙,竟不鬆手,便在這電光石火的一瞬之間,拍的一聲響,他一條手臂已被木劍打落,便和用利劍削斷一般無異。方東白不肯鬆手,原已存了捨臂護劍之心,左手伸出,不等斷臂落地,已搶著抓住,斷臂雖已離身,五根手指仍是牢牢的握著倚天劍。張無忌見他如此勇悍,既感驚懼,且復歉仄,竟没有再去跟他爭劍。方東白左手抓過倚天劍的劍柄,走到趙明身前,躬身説道︰「主人,小人無能力,甘領罪責。」趙明冷冷的道︰「我叫你去砍下這小子雙臂啊。」方東白臉上早已血色全無,聽了這話,應道︰「是!」左手迴劍一揮,倚天劍霜刃到處,竟將他的左臂又削了下來。衆人一見,比之見無忌以木劍斷他右臂更是驚訝,不約而同「哦」的一聲叫了出來。

張無忌大怒,指著趙明罵道︰「趙姑娘你這人忒也狠毒。方先生已竭盡全力,何以你仍是放他不過。」趙明冷冷的道︰「是你砍斷他的手臂,又不是我砍斷他的,到底是你狠毒還是我狠毒?」張無忌道︰「你\dash{}你\dash{}你\dash{}」説了這三個「你」字,怒氣塞喉,竟是説不下去了。趙明嫣然一笑,道︰「是我的奴僕,也用得著你來操心。」她眼光轉向張三丰,説道︰「今日瞧在明教張教主的臉上,放過了武當派。」左手一揮,道︰「走吧!」手下人抱起方東白、禿頭阿二,宇文策的身子,向殿外便走。

張無忌叫道︰「且慢!不留下黑玉斷績膏,休想走下武當山。」縱身而上,伸手往趙明肩頭抓去。手掌離她肩頭約有尺許,突覺兩股無聲無息的拳風分從左右襲到,這兩股拳風之來,事先没半點徵兆,張無忌一驚之下,翻掌抵敵。

\chapter{千金之諾}

張無忌雙掌翻出,右手接了從右邉擊來的一掌,左手接了從左邉來的一掌,四雙手掌同時碰到,只覺對方勁力奇強,掌力中竟夾著一股陰冷無比的寒氣,這股寒氣自己熟悉之至,正是幼時纏得他死去活來的「玄冥神掌」的掌力。張無忌一驚之下,九陽神功隨念而生,陡然間左脅右脅之上同時被兩個敵人拍上一掌。張無忌一盤悶哼,向後摔出,但見襲擊自己的乃是兩個身形高瘦的老者。這兩個老者各出一掌和張無忌雙掌比拚,餘下一掌無影無蹤的拍到了他的身上。

楊逍和韋一笑齊聲怒喝,撲上前去。那兩個老者又是揮出一掌,砰砰兩聲,楊逍和韋一笑騰騰騰退出數步、只感胸口氣血翻湧,寒冷徹骨。那兩個老者身子晃了兩晃,右邉那人冷笑道︰「明教好大的名頭,却也不過如此!」轉過身子,護著趙明走了。衆人生恐張無忌受傷,顧不得追趕,紛紛圍攏著他。只見殷天正抱著無忌,坐在地下,滿臉憂急。張無忌微微一笑,右手輕輕擺了一下,意示並不妨事。他體内九陽神功發動,將玄冥神掌的陰寒之氣逼了出來。他身旁功力稍弱之人竟是抵受不住,有的竟是牙関格格相擊,但掛念教主安危,誰也不肯退開。張無忌道︰「外公,衆位先生,我不妨事,請大家退開些。」衆人見他開口説話,這纔放心,依言走開數步,只見無忌頭頂便如蒸籠,不絶有絲絲白氣冒出。他解開上衣,兩脅之上宛然各有一個深深的黑色手掌印。這兩個掌印在九陽神功運轉之下,自黑轉紫,自紫而灰,終於消失不見。前後不到半個時辰,昔日數年不能驅退玄冥掌毒,頃刻間便被他消除淨盡。無忌站起身來,笑道︰「這一下雖然好險,可是終究讓咱們認出了對頭的面目。」楊逍、韋一笑和那兩個老者對掌之時了各出全力,因之玄冥陰毒及腕而止,不能深入體内,但兩人兀自打坐運氣,過了半天才驅盡陰毒。

這時鋭金旗掌旗使吳勁草進來稟報,來犯敵人已盡數下山。兪岱岩命知客道人安排素席,宴請明教諸人。筵席之上,張無忌才向張三丰及兪岱岩稟告别來情由,衆人聽聞之下,盡皆驚嘆。張三丰道︰「那一年也是在這三清殿上,我和這老人對過一掌,只是當年他假扮蒙古軍官,不知到底二老中的那一老。説來慚愧,直到今日,咱們還是摸不清對頭的底細。」楊逍道︰「那姓趙的少女不知是什麼來歷,連玄冥二老如此高手,竟也甘心供她驅使。」張無忌道︰「眼下有兩件大事。第一件是去搶奪黑玉斷續膏,好治癒兪三伯和殷六叔的傷。第二件是打聽宋大師伯他們的下落。這兩件大事,都要著落在那姓趙的姑娘身上。」兪岱岩苦笑道︰「我殘廢了二十年,便眞有仙丹神藥,那也是治不好的了,倒是救大哥、六弟他們要緊。」張無忌道︰「事不宜遲、請楊左使、韋蝠王、説不得大師三位,和我一同下山追蹤敵人。五行旗各派一位掌旗副使,分赴峨嵋、華山、崑崙、崆峒、及福建南少林五處,和各派聯絡,打探消息。請外公和舅舅前赴江南,整頓白眉旗下教衆。鐵冠道長、周先生、彭大師以及五行旗掌旗使暫駐武當,稟承我太師父張眞人之命,居中策應。」他在席上隨口吩咐,殷天正、楊逍、韋一笑等逐一躬身接令。張三丰初時還疑心他小小年紀,如何能統率群豪,此刻見他發號施令,殷天正等武林大豪居然一一凜遵,心下甚喜,暗想︰「他能學到我的太極拳、太極劍,只不過是内功底子好、悟性強,雖屬難能,還不算是如何可貴。但他能管束明教、白眉教這些大魔頭,引得他們走上正途,那纔是了不起的大事呢。嘿,翠山有後,翠山有後。」想到這裡,忍不住捋鬚微笑。

張無忌和楊逍、韋一笑、説不得等四人草草一飽,便即辭别張三丰,下山去探聽趙明的行蹤。殷天正等送到山前作别,楊不悔却依依不捨的跟著父親,送出里許。楊逍道︰「不悔,你回去吧,好好照著著殷六叔。」楊不悔應道︰「是。」眼望著張無忌,突然臉上一紅,低聲道︰「無忌哥哥,我有幾句話要跟你説。」楊逍和韋一笑等三人心下暗笑︰「他二人是青梅竹馬之交,少不得有幾句體己的話児要説。」當下加快脚步,遠遠的去了。楊不悔道︰「無忌哥哥,你到這裡來。」牽著他的手,到山邉的一塊大石上坐下。

無忌心中疑惑不定︰「我和她從小相識,交情非比尋常,但這次久别重逢,她一直對我冷冷的愛理不理。此刻不知有何話説?」只見不悔未開言臉上先紅,低下頭半晌不語,過了良久,纔道︰「無忌哥哥,我媽去世之時,託你照顧我,是不是?」無忌道︰「是啊。」不悔道︰「你將我萬里迢迢,從淮河之畔送到西域我爹爹手裡,這中間出死入生,經盡千辛萬苦。大恩不言謝,此番恩德,我只深深記在心裡,從來没跟你提過一句。」無忌道︰「那有什麼好提的?倘若我不是陪你到西域,我自己也就没有這番遇合,只怕此刻早已毒發而死。」不悔道︰「不,不!你仁俠厚道,自能事事逢凶化吉。無忌哥哥,我從小没了媽媽,爹爹雖親,可是有些話我不敢對他説。你是咱們教主,但在我心裡,我仍是當你親哥哥一般。那日在光明頂上,我乍見你無恙歸來,心中眞是説不出的歡喜,只是我不好意思當面跟你説,你不怪我吧?」無忌道︰「不怪!當然不怪。」不悔又道︰「我待小昭很兇,很殘忍,或許你瞧著不順眼。可是我媽媽死得這麼慘,對於惡人,我從此便心腸很硬。後來見小昭待你好,我便不恨她了。」無忌微笑道︰「小昭這小丫頭很有點児古怪,不過我看她不是壞人。」

其實紅日西斜,秋風拂面。微有涼意。楊不悔臉上柔情無限,眼波盈盈,低聲道︰「無忌哥哥,你説我爹爹和媽媽是不是對不起殷\dash{}殷\dash{}六叔?」無忌道︰「這些過去的事,那也不用説了。」不悔道︰「不,在旁人看來,那是很久以前的事啦,連我都十七歳了。不過殷六叔始終没忘記媽媽。這次他身受重傷,日夜昏迷,時時拉著我的手,不斷的叫我︰「曉芙!曉芙!」他説︰「曉芙!你别離開我。我手足都斷了,成了廢人,求求你,别離開我。可别抛下我不理。」她説到這裡,泪水盈眶,甚是激動。無忌道︰「那是六叔神智胡塗中的言語,作不得準。」不悔道︰「不是的,你不知道,我可知道的。他後來清醒了,眼睛瞧著我的時候,那神氣一模一樣,是在求我别離開他。只是他不説出口而已。」

無忌嘆了一口氣,深知這位六師叔武功雖強,感情却極軟弱,自己幼時便曾見他往往爲了一件小事而哭泣一場,紀曉芙之死對他打擊尤大,眼下更是四肢斷折,也難怪他惶懼不安,於是道︰「我當竭盡全力,設法去奪得黑玉斷續膏來,醫治三師伯和六師叔之傷。」不悔道︰「殷六叔這麼瞧著我,我越想越覺爹爹和媽媽對他不起,越想越覺得他可憐。無忌哥哥,我已親口答應了殷\dash{}殷六叔,他手足痊癒也好,終身殘廢也好,我總是陪他一輩子,永遠不離開他了。」説到這裡,眼泪流了下來,可是臉上神采飛揚,又是害羞,又是得意。

無忌吃了一驚,那料到楊不悔竟會向殷利亨付託終身,一時説不出話來,只道︰「你\dash{}你\dash{}」不悔道︰「我斬釘截鐵的跟他説了,這輩子跟定了他。他如果一生一世動彈不得,我就一生一世陪在他的床邉,侍奉他的飲食,跟他説笑話児解悶。」

張無忌道︰「可是你\dash{}」楊不悔搶著道︰「我不是驀地動念,便答應了他,我一路想了很久很久。不但他離不開我,我也離不開他,要是他傷勢不治,我也活不成了。跟他在一起的時候,他這麼怔怔的瞧著我,我比什麼都喜歡,無忌哥哥,我小時候什麼事都跟你説,我要吃個燒餅,便跟你説,在路上見到個糖人児好玩,也跟你説。那時候咱個没錢買不起,你半夜裡去偸了來給我,你還記得麼?」無忌想起當日和她擕手西行的情景,兩小相依爲命,不禁有些難過,低聲道︰「我記得。」

不悔按著他的手背,道︰「你給了我那個糖人児,我捨不得吃,可是拿在手裡走路,太陽曬著曬著,糖人児熔啦,我傷心得什麼似的,哭著不肯停。你説再給我找一個,可是從此再也找不到那樣的糖人児了。你雖然後來買了更大更好看的糖人児給我,我也不要了,反而惹得我又大哭了一場。那時你很著惱,罵我不聽話,是不是?」無忌微笑道︰「我罵了你麼,我可記不得了。」不悔道︰「我的脾氣很執拗,殷六叔是我第一個喜歡的糖人児,我再也不喜歡第二個了。無忌哥哥,有時我自己一個児想想,你待我這麼好,幾次救了我的性命,我\dash{}我該當侍奉你一世纔是,然而我總當你是我的親哥哥一樣,我心底裡親你敬你,可是對他啊,我是説不出的可憐,説不出的喜歡。他年紀大了我一倍,又是我的長輩,説不定人家會笑話我,爹爹又是他的死對頭,我\dash{}我知道不成的\dash{}不管怎樣,我總是跟你説了。」她説到這裡,再也不敢向無忌多望一眼,站起身來,飛奔而去。

無忌望著她的背影在山坳邉消失,心中悵悵的,也不知道什麼滋味、悄立良久,纔追上韋一笑等三人。説不得和韋一笑見他眼邉隱隱猶有泪痕,不禁向著楊逍一笑,意思是説︰「恭喜你啦,不久楊左使便是教主的岳丈大人了。」

四個人下得武當山來,楊逍道︰「這趙姑娘前後擁衛,看樣子不會單身行走,要査她的蹤跡並不爲難。咱們分從東南西北四方搜尋,明日正午在穀城會齊。教主尊意若何?」張無忌道︰「甚好,便是如此。我査西方一路罷。」原來穀城在武當山之東,他向西搜査,那是比旁人多走些路,又囑咐道︰「玄冥二老武功極是厲害,三位若是遇上了,能避則避,不必孤身與之動手。」三人答應了,當即行禮作别,分赴東南北三方査察。

且説張無忌向西都是山路,他展開輕功,行走好不迅速,只一個多時辰,已到了十偃鎭。他在鎭上麵店裡要了一碗麵,向店伴問起是否有一乘黃緞軟轎經過。那店伴道︰「有啊!還有三個重病之人,睡在軟兜裡抬著,往西朝黃龍鎭去了,走了還不到一個時辰。」張無忌大喜,心想這些人行走不快,不如等到天黑再追趕不遲,以免洩露了自己行藏。當下行到僻靜之處,找一塊大石,睡了一覺,待到初更時分,這纔向黃龍鎭來。

到得鎭上,未交二鼓天時,他閃身牆角之後,見街上靜悄悄的並無人聲,一間大客店中却是燈燭輝煌。無忌一意要査清楚趙明的來歷,顧不得孤身犯險,一縱身,輕輕上了屋頂,幾個起伏,已到了那客店旁一座小屋的屋頂,黑暗中凝目前望,只見鎭甸外的河邉空地上,豎著一座氈帳,帳前帳後人影綽綽,守衛得極是嚴密,心想︰「趙姑娘莫非是住在這氈帳裡面?她相貌説話都和漢人一模一樣,起居飲食却帶著幾分蒙古之風。」但其時元人佔治中土已久,漢人的豪紳大賈以競學蒙古風尚爲榮,那也不足爲異。他正自籌思如何走近帳蓬,忽聽得客店的一扇窗中,傳出幾下呻吟之聲。無忌心念一動,輕輕縱下地來,走到窗下,向屋裡一張。

只見房中三張床上躺著三人,其餘兩人瞧不見面貌,對窗那人正是八臂神魔宇文策,他低聲哼著,顯是傷處十分痛楚,雙臂雙腿上都是纏著白布。張無忌猛地想起︰「他四肢被我震碎,定用他本門靈藥黑玉斷續膏敷治。此刻不搶,更待何時?」一推窗子,縱身而進,房中站著的一人驚呼一聲,一拳打來。張無忌左手抓住他拳頭,右手一指便點了他軟麻穴,回頭一看,只見躺著的其餘二人正是禿頂阿二和玉面神劍方東白,被他點倒的那人身穿青布長袍,手中兀自拿著兩枚金針,想是在給三人針炙止痛。桌上放著一個黑色瓶子,瓶旁則是幾塊艾絨。

無忌拿起黑瓶,拔開瓶塞一聞,只覺一股辛辣之氣,十分觸鼻。宇文策叫道︰「來人哪,搶藥\dash{}」張無忌運指如風,連點躺著三人的啞穴,撕開字文策手臂的繃帶一看,果見他一條手臂全成黑色,薄薄的敷著一層膏藥。他生怕趙明詭計多端,故意在黑瓶中放了假藥,引誘自己上當,當下在宇文策及禿頂阿二的傷處刮下藥膏,包在繃帶之中,心想瓶中縱是假藥,從他們傷處刮下的決計不假。此時外面守護之人早已聽見聲音,有人踢開房門,搶了進來。無忌望也不望,抬腿一一踢出,霎時間客店中人聲鼎沸,亂成一片。無忌接連踢出六人,已刮盡了宇文策和禿頂阿二傷處的藥膏,心想若再耽擱。惹得玄冥二老趕到,那可大大不妙,當即將那瓶和刮下的藥膏在懷中一揣,提起那個醫生,向窗外擲了出去。只聽得砰的一聲響,那醫生重重中了一掌,摔在地上,不出所料,窗外正是有高手埋伏襲擊。無忌乘著這一空隙,飛身而出,黑暗中白光閃動,兩柄利刃疾刺而至。張無忌左手牽,右手引,乾坤大挪移心法牛刀小試,左邉一劍刺中了右邉那人,右邉一槍戳中了左邉那人,混亂中聲,無忌早已去得遠了。

他一路上好不喜歡,心想此行雖然査不到趙明的眞相,但奪得了黑玉斷續膏,那可比什麼都強。此時等不及到穀城去和楊逍等人會面,逕回武當,命洪水旗遣人前赴穀城,通知楊逍等回山。張三丰等聽説獲得黑玉斷續膏,無不大喜。張無忌細著看宇文策傷處刮下來的藥膏,再從黑瓶中挑了些藥膏來詳加比較,確是一般無異。那黑瓶乃是一塊大玉彫成,深黑如漆,觸手生溫,盎有古意,單是這個瓶子,便是一件極珍貴的寶物。張無忌再無懷疑,命人將殷利亨抬到兪岱岩房中,兩床並列放好。楊不悔跟了進來,她不敢和無忌的眼光相對,臉上却是容光煥發,心中感激無量,顯然張無忌送她到西域,在崑崙派代她喝毒酒這許多恩情,都遠比不上治好殷利亨這麼要緊。

張無忌道︰「三師伯,你的舊傷都已癒合,此刻醫治,侄児須將你手脚骨骼重行折斷,再加接續,望你忍得一時之痛。」兪岱岩實不信自己二十年的殘廢能重行痊癒,但想最壞也不過是治療無效,二十年來,早已什麼都不在乎,心中只想︰「無忌是盡心竭力,要補父母之過,否則他是終身不安。我一時之痛,又算得什麼?」他是骨氣奇硬的好漢子,也不多説,只微微一笑,道︰「你放膽幹去便是。」無忌命楊不悔出房,解去兪岱岩全身衣服,將他斷骨處盡數摸得清楚,然後點了他的昏睡穴,勁奔十指,喀喀喀響聲不絶,將他斷骨已合之處,重行一一折斷。兪岱岩雖然穴道被點,仍是痛得醒了過來。張無忌手法如風,大骨小骨一加折斷,立即拚到準確部位,敷上黑玉斷續膏,纏了繃帶,再夾上木板。醫治殷利亨那便容易得多,斷骨部位早就在西域時已予扶正,這時只須敷上黑玉斷續膏便成。

等到治完殷利亨,張無忌也已忙得汗流挾背,當下派五行旗正副掌旗使輪流守衛,以防敵人前來擾亂。當日下午,無忌用過午膳,正在雲房中小睡,以復夜來一晩奔波的疲勞,睡夢中,忽聽得脚步輕響,有人在房門口一張,小昭守在門外,低聲問︰「什麼事?教主睡著啦。」厚土旗掌旗使顏垣輕聲道︰「殷六俠痛得已暈去三次,不知教主\dash{}」張無忌不等他話説完,翻身奔出,快步來到兪岱岩房中,只見殷利亨雙眼翻白,又已暈了過去,楊不悔急得滿臉都是眼泪,不知如何是好,那邉兪岱岩咬得牙齒格格直響,顯是在硬忍痛楚,只是他性子堅強,不肯發出一下呻吟之聲。

無忌見了這等情景,大是驚異,在殷利亨「承泣」「太陽」「膻中」等穴上推拿數下,將他救醒過來,問兪岱岩道︰「三師伯,是斷骨處痛得厲害?」兪岱岩道︰「斷骨處疼痛,那也罷了,只覺腸胃心肺、五臟六腑,實是麻癢難當\dash{}好像,好像千萬條小蟲在亂鑽亂爬。」無忌這一驚非同小可,聽兪岱岩所説,那明明是身中劇毒之象,又問殷利亨道︰「六叔,你覺得怎樣?」殷利亨道︰「紅的、紫的、青的、綠的、黃的、白的、藍的\dash{}鮮艷得緊,許許多多小球児在飛舞,轉來轉去,眞是好看,眞是好看\dash{}你瞧,你瞧\dash{}」無忌「啊」的一聲大叫,險險自己也暈了過去,他心中所想到的,只是王難姑所遺「毒經」中的一段話︰「七蟲七花膏,以毒蟲七種、毒花七種,搗亂煎熬而成,中毒者先感内臟麻癢,如七蟲咬囓,然後眼前現斑爛彩色,奇麗變幻,如七花飛散。七蟲七花膏所用七蟲七花,依人而異,大凡最具靈驗神效者,共四十九種配法,變化異方復六十三種。須施毒者自解。」

無忌額頭汗涔涔而下,知道終於是上了趙明的惡當,她在黑玉瓶中所盛的固是七蟲七花膏,而在宇文策和禿頂阿二身上所敷的,竟也是這劇毒的藥物,不惜捨却兩名高手的性命,要引得自己入殼,這等毒辣心腸,當眞是匪夷所思。此刻他行動如風,迅即拆除兩人身上的夾板繃帶,用燒酒洗淨兩人四肢所敷的劇毒藥膏。楊不悔見了無忌鄭重的臉色,心知事不妙,再也顧不得男女之嫌,幫著用酒洗滌殷利亨四肢。但見黑色透入肌理,洗之不去,如染匠漆匠,手上所染顏色非一旦可除。

張無忌不敢亂用藥物,只取了些鎭痛安神的丹藥給二人服下,走到外室,又是驚懼,又是慚愧,心力交瘁。不由得雙膝一軟,驀然倒下,伏在地上便哭了起來。楊不悔大驚,只叫︰「無忌哥哥,無忌哥哥!」無忌嗚咽道︰「是我殺了三伯六叔。」他心中只想︰「這七蟲七花膏至少也有一百多種配製之法,誰又知道她用的是那七種毒蟲,那七種毒花?化解此種劇毒,全仗以毒攻毒之法,只要看不準一種毒蟲毒花,用藥稍誤,立時便送了三伯六叔的性命。」突然之間,他清清楚楚的明白了父親自刎時的心情,大錯已然鑄成,除了自刎以謝之外,確是再無别的道路。他緩緩站起身來,楊不悔問道︰「當眞是無藥可救了麼?連勉強一試也不成麼?」無忌搖了搖頭。楊不悔應道︰「傲!」居然神色泰然,並不如何驚慌,無忌心中又是一動,想起她所説的那句話來︰「他要是死了,我也不能活著。」心想︰「那麼我害死的不止是兩個,而是三個。」

心中正自一片茫然,只見吳勁草走到門外稟道︰「教主,那個趙姑娘在觀外求見。」張無忌一聽,悲憤不能自已,叫道︰「我正要找她!」從楊不悔腰間拔出長劍,執在手中,大踏步走出。小昭取下鬢邉的珠花,交給無忌,道︰「公子,你去還了給趙姑娘。」

張無忌向小昭望了一眼,心想︰「你倒懂得我的意思。我和這姓趙的姑娘仇深如海,我們身上不能留下她任何物事。」當下一手杖劍,一手持花,走到觀門之外,只見趙明一人站在當地,臉帶微笑,其時夕陽如血,斜映雙頰,艷麗不可方物,她身後十餘丈處,站著玄冥二老,兩個人牽著三匹駿馬,眼光却瞧著别處。張無忌身形一晃,早已欺到趙明身前,左手一探,已抓住了她雙手手腕,右手長劍的劍尖抵住她胸口,喝道︰「快,取解藥來!」

趙明微笑道︰「你脅迫過我一次,這次又想來脅迫我麼?我上門來看你,這樣兇霸霸的,難道是待客之道麼?」張無忌道︰「我要解藥!你若是不給,我是不想活,你也不用想活了。」趙明微微一紅,輕聲啐道︰「{\upstsl{呸}}!臭美麼?你死你的,関我什麼事,要我陪你一塊児死?」張無忌正色道︰「誰跟你説笑話,你不給解藥,今日便是你我同時畢命之日。」趙明雙手被他握住,只覺得他全身顫抖,激動已極,又覺得他掌心中有一件堅硬之物,問道︰「你手裡拿著什麼?」張無忌道︰「你的珠花,還你!」左手一抬,已將珠花插在她的鬢上,隨即又垂手抓住她的手腕,這兩下一放一握,手法快如閃電。趙明道︰「那是我送你的,你爲什麼不要?」張無忌恨恨的道︰「你作弄得我好苦!我不要你的東西。」趙明道︰「你不要我的東西?這句話是眞是假?爲什麼你一開口就問我討解藥?」張無忌每次跟她鬥口,總有落於下風,一時語塞,想起兪岱岩、殷利亨不久人世,心中一痛,眼圏児不禁紅了,幾乎便要流下泪了,忍不住想出口哀告,但想起趙明的種種惡毒之處,却又不肯在她面前示弱。

這時殷天正等都已得知訊息,擁出觀門,見趙明已被張無忌擒住,玄冥二老却站在遠處,似乎漠不関心,又似是有恃無恐,各人也便站在一旁,靜以觀變。

趙明微笑道︰「你是明教教主,武功之強,震動天下,怎麼遇到一點児難題,便像小孩子一樣,哇哇哭泣,剛纔你已哭過了,是不是?眞是好不害羞。我跟你説,你中了我玄冥二老的兩掌玄冥神掌,我這次是來瞧瞧你傷得怎樣。不料你一見人家的面,就是死啊活啊的纏個不清。你到底放不放手?」張無忌心想,她若想乘機逃走,那是萬萬不能,只要她脚步一動,自己立時便又可抓住她,於是放開了她的手腕。趙明伸手摸了摸鬢邉的珠花,嫣然一笑,道︰「怎麼?你自己倒像是没受什麼傷。」張無忌冷冷的道︰「區區玄冥神掌,未必使傷得了人。」趙明道︰「那麼大力金剛指呢?七蟲七花膏呢?」這兩句話便似兩個大鐵錘,重重錘在無忌胸口,他恨恨的道︰「果眞就是七蟲七花膏。」趙明正色道︰「張教主,你要黑玉斷續膏,我可以給你。你要七蟲七花膏的解藥,我也可以給你。只是你須得答應我三件事,那我便心甘情願的奉上。倘若你用強威逼,那麼你殺我容易,要得解藥,那是難上加難。你再對我濫施惡刑,我給你的也是假藥毒藥。」張無忌心頭一喜,道︰「那三件事?快説快説。」趙明微笑道︰「我不早跟你説過麼?我一時想不起來,什麼時候想到了,我隨時跟你説,只須你金口一諾,決不違約,那便成了。我不會要你去捉天上的月亮,也不會叫你去做違背武林俠義之道的惡事,更不會叫你去死。」張無忌聽她説「不會叫你去做違背武林俠義之道的惡事」,登時便放下了心,尋思︰「只要不背俠義之道,那麼不論多大的難題,我也當竭力以赴。」當下慨然道︰「趙姑娘,倘若你惠賜靈藥,治好了我兪三伯和殷六叔,但教你有所命,張無忌決不敢辭。赴湯𨂻火,唯君所驅。」

趙明伸出手掌,道︰「好,咱們擊掌爲誓。我給解藥於你,治好了你三師伯和六師叔之傷,日後我求你做三件事,只須不違俠義之道,你務當竭力以赴,決不推辭。」張無忌道︰「謹如尊言。」和她手掌輕輕相擊三下。趙明取下鬢邉珠花,道︰「現下你肯要我的物事吧?」張無忌生怕牠不給解藥,不敢拂逆其意,將珠花接了過來。趙明道︰「我可不許你再去送給那個俏丫鬟。」張無忌道︰「是。」趙明笑著退開三步,説道︰「解藥立時送到,張教主請了!」長袖一拂,轉身便去。玄冥二老牽過馬來,侍候她上馬先行,三乘馬蹄聲得得,下山去了。

趙明等三人剛轉過山坡,左首大樹後閃出一條漢子,正是神箭八雄中的錢二敗,挽強弓,搭長箭,朗聲説道︰「我家主人拜上張教主,書信一封,敬請收閲。」説著颼的一聲,將箭射了過來。張無忌左手一抄,將箭接在手中,只見那箭並無箭鏃,箭桿上却綁著一封信。張無忌解下一看,信封上冩的是「張教主親啓」,拆開信來,一張素箋上冩著幾行簪花小楷,文曰︰「金盒夾層,靈膏久藏。珠花中空,内有藥方。二物早呈君子左右,何勞憂之深也?唯以微物不足一顧,賜之婢僕,委諸塵土,豈賤妾之所望耶?」無忌將這張素箋連讀了三遍,又驚又喜,又是慚愧,忙著那朶珠花,逐顆珍珠試行旋轉,果有一顆珍珠能彀轉動,當下將珠子旋下,金鑄花幹中空,藏著一捲白色之物。無忌從懷中取出針炙穴道所用的金針,將那捲物事挑了出來,乃是一張薄紙,上面冩著七蟲爲那七種毒蟲,七花是那七種毒花,中毒後如何解救,一一冩得明白。其實無忌只須得知七蟲七花之名,如何解毒,却是不須旁人指點。牠一看解法,全無錯誤,心知並非趙明弄鬼,大喜之下,奔進内院,依法配藥救治。果然只一個多時辰,兪殷二人毒勢便即大爲減輕,内臟麻痺漸止,眼前彩暈漸消。張無忌再去取出趙明盛珠花送他的那隻金盒,仔細用心察看,終於發見了夾層所在,其中滿滿的裝了黑色藥膏,氣息却是芬芳清涼。這一次無忌不敢再魯莽了,找了一隻狗來,折了他一條後腿,挑些藥膏敷在傷處,等到第二日早晨,那狗精神奕奕,絶無中毒象徵,傷處更是大見好轉。

過了三日,兪殷二人體内毒性盡去,於是張無忌將眞正的黑玉斷續膏再在兩人四肢上敷塗。這一次全無意外,那黑玉斷續臂果是功效如神,兩個多月後,殷利亨雙手已能活動,只是兪岱岩殘廢已久,要説盡復舊觀,勢所難能,但瞧他傷勢復元的情勢,半載之後,當可在腋下撐兩根拐杖。以杖代足,緩緩行走,雖然仍是殘廢,却不復是絲毫動彈不得的廢人了。

張無忌在武當山上這麼一耽擱,派出去的五行旗人衆先後回山,帶回來的訊息却是令人大爲驚訝。峨嵋、華山,崆峒、崑崙各派遠征光明頂的人衆,竟無一個回轉本派,江湖上沸沸揚揚,都説魔教勢大,將六大派前赴西城的衆高手一鼓聚殲,然後再分頭消滅各派。少林寺僧衆突然失蹤之事,在武林中已引起了空前未有的波動。五行旗各掌旗使此去,幸好均持張三丰所傳的武當派信符,自己又不洩漏身份,否則早已和各派打得流花落水。各掌旗使言道,此刻江湖上衆門派、衆幫會,以及鏢行、山寨、船幫、碼頭、無不嚴密戒備,生怕明教大舉來襲。

過了數日,殷天正和殷野王父子也回到武當,説道白眉旗已重行改編,盡數隸屬明教,只是東南群雄並起,反元義師此起彼伏,天下已然大亂。

\chapter{明教大會}

這時元軍仍是極強,義師旋踵便被撲滅,無一得成大事,況且起事者各自爲戰,互相並無呼應聯絡,終被元朝官兵一一殲除。

當日晩間,張三丰在後殿擺設素筵,替殷天正父子接風。席間殷天正説起各地舉義失敗的情由,每一處起義,明教和白眉教下的弟子均有參與,被元兵或擒或殺,殉難者極衆。群豪聽了,盡皆扼腕慨歎。

楊逍道︰「天下百姓痛苦,人心思變,正是驅除韃子,還我河山的良機。昔年楊教主在世,日夜以興復爲念,只是本數向來行事偏激,百年來和中原武林多派怨仇相纏,難以擕手抗敵。天幸張教主主理教務,和各派怒仇漸解,咱們正好同心協力,共抗胡虜。」周顚道︰「楊左使,你的話聽來是不錯。可惜都是廢話,近乎放屁一類。」楊逍聽了也不生氣,道,「還請周兄指教。」周顚道︰「江湖上都説咱明教殺光了六大派的高手,一聽到『明教』兩字,人人恨之入骨,什麼『同心協力、共抗胡虜』云云,説來好聽,却是如何做起?」楊逍道︰「咱們雖然蒙此惡名,但眞相總有大白之日,何況張眞人可爲明證。」周顚笑道︰「倘若是咱們殺了宋遠橋、滅絶老尼、何太沖他們,張眞人還不是蒙在鼓裡,如何作得準?」鐵冠道人喝道︰「周顚,張眞人和教主之前,不可瘋瘋顚顚!」周顚伸了伸舌頭,却不言語了。彭瑩玉道︰「周兄之言,倒也不是全無道理。依貧道之見,咱們當大會明教各路首領,頒示張教主和武林各派修好之意。同時人多眼寬,到底宋大俠,滅絶師太他們到了何處,在大會中也可有個査究。」周顚道︰「要査宋大俠他們的下落,那是容易得緊,可説不費吹灰之力。」衆人齊道︰「怎麼樣?你何不早説?」周顚洋洋得意,喝了一杯酒,説道︰「只須教主去問一聲趙姑娘,少説也就明白了九成。我説哪,這些人不是給趙姑娘殺了,便是給她擒了。」

這兩個多月來韋一笑、楊逍、彭瑩玉、説不得等人,曾分頭下山探聽趙姑娘的來歷和蹤跡,但自從那日觀前現身、和張無忌擊掌爲誓之後,此人便不知去向,連她手下所有的人衆,也是個個無影無蹤,找不著半點痕跡。群豪諸多猜測,均料想她和朝廷有関,但除此之外,再也尋不著什麼線索了。此時聽周顚如此説,衆人都道︰「你這纔是廢話!要是尋得著那姓趙的女子,咱們不會著落在她身上打聽嗎?」周顚笑道︰「你們自然尋不著,教主却不用尋找,自會見著。教主還欠著她三件事没辦,難道這位如此厲害的小姐,就此罷了不成?嘿!這位姑娘花容月貌,可以我一想到她便渾身汗毛直豎,害怕得發抖。」衆人聽著都笑了起來,但想想也確是實情。

張無忌嘆道︰「我只盼她快些出三個難題,我盡力辦了,就此了結此事,否則終日掛在心上,不知她會出什麼古怪花樣。彭大師適纔建議,本教召集各路首領一會,此事倒是可行,各位意下如何?」群豪均道︰「甚是!在武當山上空等,終究不是辦法。」楊逍道︰「教主,你説在何處聚會最好?」張無忌略一沉吟,説道︰「本人今日忝代教主,常自想起本教兩位人物的恩情。一是蝶谷醫仙胡青中先生,他老人家已死於金花婆婆之手。另一位是常遇春大哥,不知他此刻身在何處。我想,本教這次大會,便在淮北蝴蝶谷中舉行。」周顚拍手道︰「甚好,甚好!這個『見死不救』,昔年我每日跟他鬥口,人倒是挺不錯的。他見死不救,自己死時也無人救他,正是報應。我周顚倒要去他墓上磕幾個響頭。」當下群豪各無異議,言明三個多月後的八月中秋,明教各路首領,齊集淮北蝴蝶谷胡青牛故居聚會。

次日清晨,五行旗和白眉教下各掌職信使,分頭自武當出發,傳下教主號令,諸路教衆,凡香主以上者,除留下副手於當地主理教務外,一槩於八月中秋前趕到淮北蝴蝶谷,參見新教主。

其時距中秋尚有三月,張無忌見兪岱岩和殷利亨尚未痊可,深恐傷勢有甚反覆,以致功虧一貫,因此暫留武當,照料兪殷二人,暇時則向張三丰請教太極拳劍的武學。韋一笑、彭瑩玉,説不得諸人,仍是各處遊行,探聽趙明一干人的下落。楊逍奉教主之命,勉強留在武當,但爲紀曉芙之事,對殷利亨深感慚愧,平日只有閉門讀書,輕易不離室門一步。這日午後。張無忌來到楊逍房中,商量來日蝴蝶谷大會,有那幾件大事要向教衆交代。他以年輕識淺,忽當重任,常自有戰戰兢兢之意,唯懼不克負荷,誤了大事。楊逍深通教務,因此無忌請他留在身邉,隨時向他諮詢商量。

兩人談了一會,無忌順手取過楊逍案頭的書來,見封面上冩著「明教流傳中土記」七個字的題簽,下面註著「弟子光明左使楊逍恭撰」一行小字。無忌嘆道︰「楊左使,你文武全才,眞乃本教的棟樑之士。」楊逍謝道︰「多謝教主嘉獎。」無忌翻開書來,但見小楷恭錄,事事旁徵博引,書中載得明白,明教於唐武后延載元年傳入中土,其時波斯人拂多誕持明教「三宗經」來朝,中國人始習此教經典。唐大曆三年六月二十九日,長安洛陽建明教寺院「大雲光明寺」,此後太原、荊州、揚州、洪州、越州等重鎭,均有大雲光明寺。至會昌三年,朝廷下令殺明教徒,明教勢力大衰,自此之後,明教便成爲犯禁的秘密教會,歷朝均受官府摧殘。明教爲圖生存,行事不免詭秘,終於摩尼教這個「摩」字,被人改爲「魔」字,世人遂稱之爲魔教。

張無忌讀到此處,不禁長嘆一聲,説道︰「楊左使,本教教旨原是去惡行善,和釋道並無大異,何以自唐代以來,歷朝均受慘酷屠戳?」楊逍道︰「釋家雖説普渡衆生,但僧衆出家,各持清修,不理世務。道家亦然。本教則聚集鄕民,不論是誰有甚危難困苦,諸教衆一齊出力相助。官府欺壓官民,什麼時候能少了?什麼地方能少了?一遇到有人被官府冤屈欺壓,本教勢必和官府相抗。」張無忌點了點頭,説道︰「只有朝廷官府不去欺壓良民,豪紳土豪不敢橫行不法,到那時候,本教方能眞正的興旺。」楊逍拍案而起,大聲道︰「教主之言,正説出了本教教旨的関鍵所在。」張無忌道︰「楊左使,你説當眞能有這麼一日麼?」楊逍沉吟半響,説道︰「但盼眞道有這麼一天。宋朝本教方臘方教主起事,也只不過是爲了想叫官府不敢欺壓良民。」他翻開那本書來,指到明教大教主方臘在浙東起事、震動天下的記載,張無忌看得悠然神往,掩巻説道︰「大丈夫固當如是。雖然方教主殉難身死,却終是轟轟烈烈的幹了一番事業。」兩人四目相投,心意相通,不禁血熱如沸。

楊逍又道︰「本教歷代均遭嚴禁,但始終屹立不倒。南宋紹興四年,有個官員叫做王居正,對皇帝上了一道奏章,説到本教之事,教主可以一觀。」説著翻到書中一處,抄錄著王居正那道奏章。張無忌見那奏章中冩道︰「伏見兩浙州縣有吃菜事魔之俗。方臘以前,法禁尚寬,而事魔之俗猶未至於甚熾。方臘之後,法禁愈嚴,而事魔愈不可勝禁。\dash{}臣聞事魔者,每鄕每村有一二桀黠,謂之魔頭,奉錄其鄕村姓氏名字,相與組盟爲魔之黨。凡事魔者不肉食。而一家有事,同黨之人皆出力以相賑卹。蓋不肉食則費省,費省故易足。同黨則相親,相親故親卹而事易濟\dash{}」

張無忌讀到這裡,説道︰「那王居正雖然仇視本教,却也知本教教衆節儉樸實,相親相愛。」他接下去又看那奏章︰「\dash{}臣以爲此先王導其民使相親相友相助之意。而甘淡薄,教節儉,有古淳樸之風。今民之師帥,既不能以是爲政,乃爲魔頭者竊取以瞽惑其黨,使皆歸德於其魔,於是從而附益之以邪僻害教之説。民愚無知,謂吾從魔之言︰事魔之道,而食易足、事易濟也,故以魔頭之説爲皆可信,而爭趨歸之。此所以法禁愈嚴,而愈不可勝禁。」他轉頭向楊逍道︰「楊左使,『法禁愈嚴,而愈不可勝禁』這句話,正是本教深得民心的明証。這部書可否借我一閲,也好讓我多知本教往聖先賢的業績遺訓?」楊逍道︰「正要請教主指教。」

無忌將書收起,説道︰「兪三伯和殷六叔傷勢大好了,我們明日便首途赴蝴蝶谷去。我另有一事要和左使相商,那是関於不悔妹子的。」楊逍只道他要開口求婚,心下甚喜,説道︰「不悔的性命全出教主所賜,屬下父子感恩圖報,非只一日。教主但有所命。無不樂從。」張無忌於是將楊不悔那日如何向自己吐露心事的情由,一一説了。楊逍一聽之下,錯愕萬分,怔怔的竟然説不出話來,隔了半晌,纔道︰「小女蒙殷六俠垂青,原是楊門之幸,只是他二人年紀懸殊,輩份又異,這個\dash{}」説了「這個」兩字,却又接不下去。張無忌道︰「殷六叔未滿四十,方當壯盛。不悔妹子雖叫他一聲叔叔,也不是眞有什麼血緣泛親,師門之誼。他二人情投意合,倘若成了這頭姻緣,上代的仇嫌盡數化解,正是大大的美事。」楊逍原是個十分豁達之人,又爲紀曉芙之事,每次見到殷利亨,總是抱愧於心,暗想不悔既然傾心於他,倒是了贖自己的前愆,從此明教和武當派再也不存芥蒂,於是長揖説道︰「教主玉成此事,足見関懷。屬下先此謝謝。」

當晩張無忌傳出這個喜訊,群豪紛紛向殷利亨道喜。楊不悔害羞,躱在房中不肯出來。張三丰和兪岱岩得知此事時,起初也頗驚奇,但隨即便爲殷利亨喜歡。説到婚期,殷利亨道︰「待大師哥他們回山,衆兄弟完聚,那時再辦喜事不遲。」

次日張無忌偕同楊逍、殷天正、殷野王、鐵冠道人、周顚、小昭等人,辭别張三丰師徒,首途前往淮北。楊不悔留在武當,服侍殷利亨,當時男女之防雖嚴,但他們武林中人,也不理會這些小節。

明教一行人曉行夜宿,向東北方行去,一路上見田地荒蕪,民有飢色。沿海諸路本著殷實富庶之區,但眼前餓殍遍野,生民之困,已到極處。群豪慨嘆百姓慘遭劫難,又知蒙古人如此暴虐,霸居中土之期必不久長,正是天下英雄揭竿起事的良機。這一日來到界牌集,離蝴蝶谷已然不遠,正行之間,忽聽得前面喊殺之聲大震,有兩支人馬正在交兵。群豪縱馬上前,穿過一座森林,只見千餘名蒙古兵分列左右,在進攻一座山寨。寨上飄出一面繪著紅色火燄的大旗,正是明教的旗幟。寨中人數較少,己有漸漸不支之勢,但兀自健鬥不屈。蒙古兵矢發如雨,大叫︰「魔教的叛賊,快快投降!」

周顚道︰「教主,咱們上嗎?」張無忌道︰「好!先去殺了帶兵的軍官。」楊逍、殷天正、殷野王、鐵冠道人、周顚五人應命而出,衝入敵陣,長劍揮動,兩名元兵的百夫長首先落馬,跟著統兵的千夫長也被殷野王一刀砍死。元兵群龍無首,登時大亂。山寨中人見來了外援,大聲歡呼。寨門開處,一條黑衣大漢手挺長矛,當先衝出,元兵當者辟易,無人敢攖其鋒。

只見那大漢長矛一閃,便有一名元軍被刺,倒撞下馬。衆元兵驚呼連連,四下奔逃。楊逍等見這大漢威風凜凜,有若天神,無不讚嘆︰「好一位英雄將軍。」此時張無忌早已看清楚那大漢的面貌,正是常自想念的常遇春常大哥,只是劇鬥方酣,不即上前相見。明教人衆前後夾攻,元軍死傷了五六百人,餘下的不敢戀戰,分頭落荒而走。常遇春橫矛大笑,叫道︰「是那一路的兄弟前來相助?常某感激不盡。」張無忌叫道︰「常大哥,想煞小弟也。」縱身而前,緊緊握住了他的手掌。

常遇春躬身下拜,説道︰「教主兄弟,我又是你大哥,又是你屬下,眞是歡喜得不知如何纔好。」原來常遇春歸五行旗中巨木旗下該管,張無忌接任教主等等情由,已得掌旗使聞蒼松示知,這幾天中他率領本教兄弟,日夜等候張無忌到來,不料元軍却來攻打。常遇春見己寡敵衆,本擬故意示弱,將元軍誘入寨中,一鼓而殲,但張無忌等突然趕到應援,他便乘勢開寨殺出。他在明教中職位不高,當下向楊逍,殷天正等一一參見,恭執下屬之禮。群豪以他是教主的結義兄弟,都不敢以長上自居,執手問好,相待盡禮。

常遇春邀群豪入寨,殺牛宰羊,大擺酒筵,説起别來情由。這幾年來淮南淮北水旱相繼,百姓苦不堪言。常遇春無以爲生,便嘯聚一班兄弟,做那打家劫舍的綠林好漢勾當,倒也逍遙快活,山寨中有糧食金銀多了,便去賑濟貧民。元軍幾次攻打,都奈何他不得。

衆人在山寨中歇了一晩,次日和常遇春一齊北行,料得元軍新敗,兩三月内決計不敢再來。數日後到了蝴蝶谷外前。先到的教衆,得知教主到來,列成長隊,迎出谷來。其時巨木旗下執事人等,早已在蝶谷中搭造了許多茅舍木屋,以供典會的各路好漢居住。韋一笑、彭瑩玉、説不得等均已先此到達。張無忌接見諸路教衆後,備了祭品,分别到胡青牛夫婦及紀曉芙墓前致祭。想起當日離谷時何等淒惶狼狽,今日歸來却是雲荼燦爛,風光無限,眞是如同隔世。

再過三日便是八月中秋,蝴蝶谷中築了高壇,壇前燒起熊熊大火。張無忌登壇宣示和中原諸門派盡釋前愆、反元抗胡之意,又頒下教規,重申行善去惡、除暴安良的教旨。衆香主歡聲雷動,一齊凜遵,各人身前點起香束,立誓對教主令旨,決不敢違。是日壇前火光燭天,香播四野,明教之盟,至此爲極。年老的教衆眼見這片興旺氣象,想起十餘年來本教四分五裂、幾致覆滅的情景,忍不住喜極而泣。

張無忌又宣示道︰「本教歷代相傳,不茹葷酒。但眼下處處災荒,只能有什麼便吃什麼,何況咱們今日第一件大事,乃是驅除韃子,衆兄弟不食葷腥,精神不旺,難以力戰。自今而後,廢了不茹葷酒這條教規。咱們立身處世,以大節爲重,飲食禁忌,只是餘事。」當晩蝴蝶谷中月明如晝,數千教衆暢懷盡歡,至曉方罷。

次日衆人睡至午間,這纔起身。張無忌剛梳洗罷,屬下教衆報道︰「洪水旗下弟子朱元璋、徐達諸人求見。」張無忌大喜,親自迎出門去。朱元璋、徐達率同湯和、鄧愈、花雲、吳良、吳禎諸人恭恭敬敬的站在門外,一見無忌出來,一齊躬身行禮,説道︰「參見教主!」無忌常常念著那日徐達救命之恩,一見衆人,喜之不盡,當即還禮,左手擕著朱元璋,右手挽著徐達,同進室内,命衆人坐下。衆人告了罪,纔行就座。這時朱元璋已然還俗,不再作僧人打扮,説道︰「屬下等奉教主令旨,趕來蝴蝶谷,本應早到候駕,但途中遇上了一件十分蹊蹺之事,屬下等跟蹤追査,以致誤了會期,還請教主恕罪。」

張無忌道︰「却不知是遇上了何等蹺蹊之事?」朱元璋道︰「六月上旬。咱們便得到教主的令旨,大夥児好生喜歡,咱兄弟們商議,該當備什麼禮物,慶賀教主纔是,准北是苦地方,没什麼好東西的,幸得會期尚遠,大夥児便一起上山東去闖闖。咱們生怕給官府認了出來,因此扮作了趕脚的騾車夫,屬下算是個車夫頭児。這天來到河南的歸德府,接了幾個老西客人,往山東荷澤。正行之間,忽然有一夥人趕了上來,輪刀使槍,模樣十分兇狼,將咱們車中的客人都趕了下去,叫咱們去載别的客人。那時花兄弟性子暴,便要跟他們放對,徐兄弟向他使個眼色,叫他瞧清楚情由,再動手不遲。那夥人將咱們九輛大車有趕到一處山坳之中,那裡另外還有十多輛大車候著,只見地下坐著的都是和尚。」

張無忌道︰「都是和尚?」朱元璋道︰「不錯。那些和尚個個垂頭喪氣,萎靡不振,但其中好些人模樣極是不凡,有的太陽穴高高凸起,有的魁梧奇偉,徐兄弟悄悄跟我説,這些和尚都是身負高強武功之人。那夥兇人叫衆和尚坐在車裡,押著咱們一路向北。屬下料想其中必有古怪,暗地裡叫衆兄弟著意提防,千萬不可露出形跡。一路上咱們留神那夥兇人的説話,可是這群人詭秘得緊,在咱們面前一句話也不説,後來吳良兄弟大著膽子,半夜裡到他們窗下去偸聽。連聽了四五夜,這纔探得了一些端倪。原來這些和尚竟然是河南嵩山少林寺的高僧。」張無忌本已料到了幾分,但還是「啊」的一聲。朱元璋接著道︰「吳良兄弟又聽到一個人説︰『主人當眞神機妙算,令人拜服。少林、武當六派高手。盡入掌中,自古以來,還有誰能做得到這一步的?』又有一人説︰『這還不算希奇。一箭雙鵰,却把魔教的衆魔頭也牽連在内。』咱們七個人假裝出恭,在茅廁裡悄悄商量,都説此事既然牽連本教在内,碰巧落在咱們手上,總須査個水落石出,也好稟報教主知曉。」張無忌道︰「各位説得甚是。」朱元璋道︰「大夥見一路北行,越發裝得獃頭獃腦,湯和兄弟和鄧愈兄弟又假裝爭五錢銀子,笨手笨脚的打了一場架,顯得半點不會武功。那夥兇人拍手呵呵大笑,對咱們再不在意,咱們又老爺長、老爺短的,對他們恭敬得厲害。吳禎兄弟曾想弄些麻煩來,半途上麻翻了這夥兇人,救出少林群僧。可是咱們細想,這件事來龍去脈半點不知,眼著這夥兇人又是精明幹練,武功了得,没的一個失手,打草驚蛇,反而誤了大事,是以始終没敢下手。到得河間府,遇上了六輛大車,也是有人押解,車中坐的却是些俗家人。吃飯之時,我聽得一個少林僧跟一個新來的客人招呼,説道︰『宋大俠,你也來啦!』」

無忌站起身來忙問︰「他説是宋大俠?那人怎生模樣?」朱元璋道︰「那人瘦長身材,五六十歳年紀,三絡長鬚,相貌甚是清雅。」無忌一聽,正是宋遠橋的形相,不禁又驚又喜,再問其餘諸人的容貌身形,果然兪蓮舟、張松溪、莫聲谷三人也都在内,又問︰「他們都受了傷嗎?還是戴了{\upstsl{銬}}鐐?」朱元璋道︰「没有{\upstsl{銬}}鐐。也瞧不出受什麼傷,説話飲食,都和常人無異,只是精神不振,走起路來有點虛虛晃晃。那宋大俠聽少林僧這麼説,只苦笑了一下,没有答話。那少林僧再想説什麼,押解的兇人便過來拉開了他。此後兩批人前後相隔十餘里,再不同食同宿,屬下從此也没再見到宋大俠他們。七月初三,咱們載著少林群僧到了大都。」張無忌道︰「啊,到了大都,果然是朝廷下的毒手,後來怎樣?」朱元璋道︰「那夥兇人領著咱們,將少林群僧送到西城一座大寺院中,叫咱們也睡在廟裡\dash{}」

張無忌道︰「那是什麼廟?」朱元璋道︰「屬下進寺之時,曾抬頭瞧了瞧廟前的扁額,見是叫做『萬法寺』,但便因這麼一瞧,吃了一個兇人的一下馬鞭。當晩咱們兄弟們悄悄商量,這些兇人定然放不過咱們,勢必要殺了衆人滅口,天一黑,咱們便偸著走了。」張無忌道︰「事情確是兇險,幸好這類兇人,倒也没有追趕。」湯和微笑道︰「朱大哥也料到了這著,事先便安排下手脚。咱們到鄰近的騾馬行中去抓七個騾馬販子來,跟他們換了衣服,然後將這七人砍死在廟中,臉上斬得血打模糊,好讓那些兇人認不出來。又將跟咱們同來的大車車夫都殺了,銀子散得滿地,裝成是兩夥人爭錢銀兇殺一般。待那夥兇人回廟,再也不會起疑。」張無忌心中一驚,只見徐達臉上有小怒之色,鄧愈顯得頗是{\upstsl{尷}}尬,湯和説來得意洋洋、只有朱元璋却是絲毫不動聲色,恍若没事人一般。張無忌暗想︰「這人下手好辣,實是個厲害脚色。」説道︰「朱大哥此計雖妙,但從今而後,咱們決不可再行濫殺無辜。」這是教主的訓諭,朱元璋等一齊起立,躬身説道︰「謹遵教主令旨。」

張無忌道︰「朱大哥七位探聽到少林、武當兩派高手的下落,此功確是不小。待安排了抗元起義的大事之後,咱們便同赴大都,相救兩派高手。」他説過公事,再和徐達等相敘私誼。説起那日偸宰張員外耕牛之事,一齊拊掌大笑。

當晩張無忌,大會教衆,焚火燒香,宣告諸足並起,共抗元朝,諸路教衆相互呼應,累得元軍疲於奔命。那便大事可成。是時定下方策,教主張無忌率同光明左使楊逍、青翼蝠王韋一笑執掌總壇,爲全教總帥。白眉鷹王殷天正,率同白眉旗下教衆,在江南起事。朱元璋、徐達、湯和、鄧愈、花雲、吳良、吳禎,會同常遇春寨中人馬,和郭子興、孫德崖等,在准北濠州起兵。布袋和尚説不得率領韓山童、劉福通、杜遵道、羅文素、盛文郁、王顯忠、韓皎児等人,在河南穎川一帶起事。彭瑩玉率領徐壽輝、鄒普旺、明五等,在江西贛、饒、袁、信諸州起事。鐵冠道人率領布三王、孟海馬等,在湘楚荊襄一帶起事。周顚率領芝麻李,趙君用等在徐宿豐沛一帶起事。冷謙會同西域教衆,截斷西域開赴中原的蒙古救兵。五行旗歸總壇調遣,何方吃緊,便向何方應援。

這等安排方策。十九出於楊逍的計謀,張無忌宣示出來,教衆歡聲雷動,意氣風發。張無忌又道︰「單憑本教一教之力,難以撼動元朝近百年的基業,須當聯絡天下英雄豪傑,群策群力,大功方成。眼下中原武林人物,半數爲朝廷所擒,總壇即當設法營救。明日衆兄弟散處四方,一遇機會,便即殺韃子動手,總壇也即前赴大都救人。今日在此盡歡,此後相見,未知何月。衆兄弟須當義氣爲重,大事爲先,決不可爭權奪利,互逞殘殺,若有此等不義情由,總壇決不寬饒。」衆人齊聲答應︰「教主令旨,却不敢違!」呼喊聲山谷鳴響。

當下衆人歃血爲盟,焚香爲誓,決死不負大義。次日清晨,諸路人衆紛紛向張無忌告别。衆人雖均是意氣慷慨的豪傑,但想到此後血戰四野,不知誰存誰亡,大事縱成,今日蝴蝶谷大會中的群豪只怕活不到一半。是時蝴蝶谷前聖火高燒,也不知是誰忽然朗聲喝了起來︰「焚我殘軀,熊熊聖火。生亦何歡,死亦何苦?」各人一個個的齊聲而唱,相和之聲越來越響︰「焚我殘軀,熊熊聖火。生亦何歡?死亦何苦?爲善除惡,唯光明故。喜樂悲愁,皆歸塵土。憐我無人,憂患實多!憐我無人,憂患實多!」

那「憐我世人,憂患實多!憐我世人,憂息實多!」的歌聲,飄揚在蝴蝶谷中,群豪白衣如雪,一個個走到張無忌面前,躬身行禮,昂首而出,再不回顧。張無忌想起如許大好男児,此後一二十年之中,行將鮮血灑遍中原大地,忍不住熱泪盈眶。但聽歌聲漸遠,壯士離散,熱鬧了數日的蝴蝶谷重歸沉寂,只剩下楊逍、韋一笑以及朱元璋等寥寥數人。張無忌詳細詢明萬法寺坐落的所在,以及那干兇人形貌,説道︰「朱大哥,此間濠泗一帶,方當大亂,不可錯過了起事之機。你們不必陪我上大都去,咱們就此别過。」朱元璋、徐達、常遇春等齊道︰「但盼教主馬到成功,屬下等靜候好音。」拜别了張無忌,出谷自去舉事。

張無忌道︰「咱們也要動身了。小昭,你身有{\upstsl{銬}}鐐,行動不便,就在這裡等我吧。」小昭委委曲曲的答應了,可是她一直送出谷來,送了三里,又送三里,終是戀戀不捨的不肯分别。無忌道︰「小昭,你越送越遠,回去時路也要不認識啦。」小昭道︰「張公子,你到了大都,會見到那個趙姑娘不會?」無忌道︰「説不定會得見到。」小昭道︰「你要是見到她,代我求她一件事成不成?」無忌奇道︰「你有什麼事求她?」小昭雙臂一伸,道︰「向趙姑娘借倚天劍一用,把這鐵鍊児割斷了,否則我終身不得自由。」無忌見她神情楚楚?説道極是可憐,心中有些不忍,便道︰「只怕她不肯將寶劍借給我,何況要一直借到這裡。」小昭道︰「那麼\dash{}那麼,你將我帶到她的跟前,請她寶劍一揮,不就成了。」無忌笑道︰「説來説去,你還是要跟我上大都去。楊左使,你説咱們能帶她嗎?」楊逍心知張無忌既如此説,已有擕她同去之意,便道︰「帶她同去,那也不妨,教主衣著茶水,也有個人服侍,只是造鍊聲叮叮{\upstsl{噹}}{\upstsl{噹}},容易引人注目。這樣吧,叫她裝作生病,坐在大車之中,平時不可出來。」小昭大喜,忙道︰「多謝公子,多謝楊左使。」向韋一笑看了一眼,又加上一句︰「多謝韋法王。」韋一笑笑道︰「多謝我幹什麼?你小心我發起病來,吸你的血。」説著露出滿口森森白牙,裝個怪樣。小昭明知他是開玩笑,却也不禁有些害怕,退了三步,道︰「你\dash{}你别嚇我。」

這日午後,三騎一車,逕向北行。一路無話,不一日已到元朝的京城大都(即今北平城)。其時蒙古人鐵騎所至,直至數萬里外,歷來帝國幅員之廣,無一能及。大都是帝皇之居,各小國各部族的使臣貢員,不計其數。張無忌等一進城門,便見街上來來往往,許多都是黃髮碧眼之徒。四人到得西城,找了一家客店投宿。楊逍出手闊綽,裝作是富商大賈模樣,要了三間上房,那店小二奔走趨奉,服待得極是殷勤。楊逍問起大都城裡的風景古蹟,談了一會,漫不經意的問起有什麼古廟寺院。那店小二説了幾所,便説到西城的萬法寺來,説道︰「這萬法寺眞是好大一座叢林,寺裡的三座大銅佛,便是走遍天下,也找不出第四座來,原該去見識見識。但客官們來得不巧,這半年來,寺中住了西番的佛爺,平常人就不敢去了。」楊逍道︰「住了番僧,去瞧瞧也不礙事啊。」那店小二伸了伸舌頭,四下裡一張,低聲道︰「客官們初來京城,不是小的多嘴,説話還得留神些。那些西番的佛爺們見了人愛打便打,愛殺便殺,見了標緻的娘児們更是一把便抓進寺去。這是皇上聖旨,金口許下的,有誰敢老虎頭上拍蒼蠅,走到西番佛爺的跟前?」西域番僧倚仗蒙古人的勢力,橫行不法,欺壓漢人,楊逍等知之已久,只是没料到京城之中竟亦這般肆無忌憚,當下也不跟那店小二多説。晩飯後各自合眼養神,等到二更時分,三人從窗中躍出,向西尋去。

\chapter{用心險惡}

那萬法寺高達四層,寺後的一座九級寶塔,更是老遠便可望見。張無忌、楊逍、韋一笑三人展開輕功,片刻間便已到了寺前。三人一打手勢,繞到寺院左側,想登上寶塔,居高臨下的察看寺中情勢,不料離塔三十餘丈,便見塔上人影綽綽,每一層寶塔上都有人來回巡査,塔下更有二三十人守著。三人一見之下,心中又驚又喜,情知此塔守衛如此嚴密,少林、武當各派人衆必是囚禁在内,倒是省了一番探訪的功夫。只是敵方戒備森嚴,救人必是極不容易。何況空聞方丈、空智神僧、宋遠橋、兪蓮舟、張松溪等,那一個不是武功卓絶,竟然盡數落在他們手中,則對方能人之多、手段之厲害,自是不言可喩。三人來萬法寺之前,已然商議定當,決計不可鹵莽從事。當下悄悄退開。

突然之間,第六層寶塔上亮起火光,有八九人手執火把緩緩移動,那火把從第六層亮到第五層,又從第五層亮到第四層,一路下來,到了底層後,從寶塔正門出來,走向寺去。楊逍揮了揮手,從側面慢慢欺近身去。那萬法寺後院一株株都是參天古樹,三人躱在樹後以爲掩蔽,一聽見風聲響動,便即奔上數丈。要知萬法寺中高手如雲,實是絲毫不敢托大,三人的輕功造詣雖然已到了爐火純青的地步,却也唯恐被人察覺,須得乘著風動落葉之聲,纔敢移步。如此走上二十多丈,火把照耀之下,已看清楚十餘名黃袍男子,手中各執兵刃,押著一個寬袖大袍的老者。那人偶一轉頭,無忌看得明白,正是崑崙派掌門人鐵琴先生何太沖。他不禁心中凜︰「果然連何先生也在此處。」

眼見一干人進了萬法寺的後門,三人等了一會,見四下確實無人,這纔從後門中閃身而入。那寺院房舍衆多,規模之大,幾乎可和少林寺相彷彿,好在中間一座大殿的長窗中燈火明亮,料得何太沖是被押到了該處。三人閃身而前,到了殿外,張無忌伏在地下,從長窗的縫隙中向殿内張望,楊逍和韋一笑分列左右把風守衛,防人偸襲。他二人雖然藝高人膽大,但此刻深入龍潭虎穴,心下也不禁惴惴。

那長窗的縫隙甚細,無忌只見到何太沖的下半身,殿中尚有何人,却無法瞧見。只聽何太沖氣沖沖的説道︰「我既墮奸計,落入你們手中,要殺要剮,一言而決。你們逼我做朝廷鷹犬,那是萬萬不能,便再説上三年五載,也是白費唇舌。」張無忌暗暗點頭,心想︰「這何先生雖不能説是什麼正人君子,但大関頭極是把持得定,不失爲一派掌門之尊。」只聽一個男子口音,冷冰冰的道︰「你既是固執不化,主人也不勉強,這裡的規矩你是知道的了?」何太沖道︰「我便是十根手指一齊斬斷,也不投降。」那人道︰「好,我再一遍,你如勝得了咱們這裡三人,立時放你出去,如若敗了,便斬斷一根手指,囚禁一月,再問你降也不降。」何太沖道︰「我已斷了兩根手指,再斷一根,又有何妨?拿劍來!」那人冷笑道︰「等你十指齊斷之後,再來投降,咱倒也不要你這廢物了。拿劍給他!摩訶巴思,你跟他走走!」另一個粗壯的聲音應道︰「是!」

無忌暗運神功,輕輕將那縫隙挖大了一點,只見何太沖手持一柄木劍,劍頭包著布,又軟又鈍,不能傷人,對面則是一個高大的番僧,手中拿著的却是一對青光閃閃的純鋼戒刀。兩人的兵刃利鈍懸殊,幾乎不用比試,強弱便判。何太沖毫不氣餒,木劍一晃,説道︰「請!」刷的便是一劍,去勢極是凌厲,崑崙劍法,當眞别有獨到之祕。那番僧摩訶巴思身裁雖然壯大,行動却甚敏捷,一對戒刀使將開來,刀刀斬向何太沖的要害。無忌只看了數招,便即暗驚︰「怎地何先生脚步虛浮,氣急敗壞,竟似内力全然失却?」

張無忌自習得九陽神功及乾坤大挪移心法之後,於天下武學之變,盡羅胸中,這幾個月來在武當山上日夕向張三丰請益,更是精進了一層,此刻見何太沖和那番僧動手,越看越覺其中必有蹺蹊。何太沖劍法雖精,内力却和常人相去不遠,劍招上的凌厲威力,全然施展不出,只是那番僧的武功實是遜他兩籌,好幾次猛攻而前,眼見便可將他斃於刀下,但總是被何太沖以精妙招術反得先機。拆到五十餘招後,何太沖喝一聲︰「著!」一劍東劈西轉,斜迴而前,托的一聲輕響,已戳在那番僧腋下。倘若他手中持的平常利劍,又或内力不失,劍鋒早已透肌而入,那番僧性命不保,但他所用木劍劍頭包布,那番僧只是微微一痛而已。

只聽那冷冷的聲音説道︰「摩訶巴思退!溫臥児上!」張無忌向聲音來處一看,只看説話之人臉上如同罩著一層黑煙,一部稀稀朗朗的花白鬍子,正是玄冥二老之一。他負手而立,雙目閉住,似乎對眼前事漠不関心。再向前看,只見一張舖著錦緞的矮几之上,有一雙脚踏著。脚上穿一對鵞黃鞋子,鞋頭上各綴一顆明珠。張無忌心中一動,眼見這對脚脚掌纖美,踝骨渾圓,依稀認得正是在綠柳莊中,自己曾經捉過的趙明的一對脚。他在武當山見她,全以敵人相待,但此時不知如何,看到了一對踏在錦凳上的纖足,忍不住面紅耳赤,心跳加劇。

但見趙明的右足輕輕點動,料想她是全神貫注的在看何太沖和溫臥児比武,約莫一盞茶時分,何太沖叫聲︰「著!」趙明的右足在錦凳上一登,溫臥児又敗下陣來。只聽那黑臉的玄冥老人説道︰「溫臥児退,黑林缽夫上!」張無忌聽到何太沖氣息粗重,想必他連戰二人,已是十分吃力。片刻間劇鬥又再展開。那黑林缽夫用的似是銅棍鐵杵一類的粗重兵刃,使將開來,滿殿都是風聲,殿旁的燭火被風勢激得忽明忽暗,燭影猶似天上浮雲,一片片的在趙明脚上掠過。驀地裡眼前一暗,殿上紅燭熄了半邉,喀的一響,木劍斷折,何太沖一聲長嘆,抛劍在地,這場比拚終於是輸了。

玄冥老人道︰「鐵琴先生,你降是不降?」何太沖昂然道︰「我既不降,也不服。我内力若在,這番僧焉是我的對手!」玄冥老人冷冷的道︰「斬下他左手的無名指,送回塔去。」無忌回過頭來,楊逍向他搖了搖手,意思顯然是説︰「倘若此刻衝進殿去救人,不免誤了大事。」但聽得殿中斷指、敷藥、止血、裹傷,何太沖甚是硬氣,竟是一哼也没哼。那群黃衣人手執火把,將他送回寶塔囚禁。張無忌等縮身在牆角之後,火光下見何太沖臉如白紙,咬牙切齒,神色極是憤怒。

一行人走遠後,忽聽得一個嬌柔清脆的聲音在殿内響起,説道︰「鹿杖客,崑崙派的劍法果眞了得,他刺到摩訶巴思那一招,左邉這麼一劈,右邉這麼一轉\dash{}」説話的正是趙明,她一邉説,一邉走到殿中,手裡提著一柄木劍,照著何太沖的劍法使了起來。番僧摩訶巴思手舞雙刀,跟她喂招。那黑臉的玄冥老人便是鹿杖客,讚道︰「主人眞是聰明無比,這一招使得分毫不錯。」她練了一次又練一次,每次都是將劍尖戮到摩訶巴思腋下,雖然劍是木劍,但重重一戮,每一次又都戮在同一部位,料必相當疼痛。摩訶巴思却聚精會神的跟她喂招,竟無半點怨避之意。

她練熟了這幾招,又叫溫臥児出來,再試何太沖如何擊敗他的劍法。張無忌看到此時,心頭早已明明白白,原來趙明將各派高手囚禁在此,使藥物抑住各人的内力,逼迫他們投降朝廷。衆人自然不降,那便命人逐一與之相鬥,她在旁察看,得以偸學各門各派的精妙招數,用心之毒,計謀之惡,實是令人髮指。

這時趙明已在和黑林缽夫喂招,最後數招有些遲疑不決,問道︰「鹿杖客,是這樣的麼?」鹿杖客怔了一怔,轉頭道︰「鶴兄弟,你瞧清了没有?」左首角落裡一個聲音答道︰「苦頭陀一定記得更加清楚。」趙明笑道︰「苦頭陀,勞你的駕,請你指點一下。」只見右首走過來一個白髮披肩的頭陀,駝背跛足,滿面橫七豎八的都是刀疤,原來相貌已是全然不可辨認。這頭陀身材魁偉,雖然駝背,仍和鹿杖客差不多高矮。他一言不發,接過趙明手中木劍,刷刷刷刷數劍,便向黑林缽夫攻去,用的竟是崑崙派劍法,劍招之純,便似從小練熟了的一般。此時張無忌方始看見,那黑林缽夫是個西域武士,所使的是一根長達八尺的鐵杖。

苦頭陀顯是依樣葫蘆,模仿何太沖的劍招,因此也是絲毫不用内力,那黑林缽夫却是全力施爲,鬥到酣處,他揮杖橫掃,熄後點亮了的紅燭,突又黑了一半。何太沖乃是在這一招上無可閃避,迫得以木劍硬擋鐵杖,這纔折劍落敗,但那苦頭陀劍輕飄飄的削出,猶似輕燕掠過水面、貼著鐵杖削了上去。黑林缽夫握杖的手指被木劍劍身一削,虎口處穴道酸麻,登時拿捏不住,{\upstsl{噹}}的一聲,鐵杖落在地下,撞得青磚磚屑紛飛。黑林缽夫滿臉通紅,心知這木劍若是換了利劍,自己十根手指早已削斷,躬身道︰「拜服,拜服!」俯身拾起鐵杖。苦頭陀雙手托著木劍,交給趙明。趙明笑道︰「苦師傅,最後一招精妙絶倫,也是崑崙派的劍法麼?」苦頭陀點了點頭。趙明又道︰「那何太沖不會麼?」苦頭陀又點了點頭。趙明笑道︰「苦師傅,你教教我。」苦頭陀空手比劍,趙明持劍照做。練到第三次時,苦頭陀行動如電,雖然駝背跛足,却是快得不可思議,趙明便顯然跟不上了。但她聰穎過人,劍招雖然慢了,仍是依模依樣,絲毫不爽。苦頭陀翻過身來,雙手向前一送,停著就此不動。

趙明一怔,張無忌心中暗暗喝一聲采︰「好,大是高明!」趙明側頭看著苦頭陀的姿勢,想了一想,登時領悟,説道︰「啊,苦師傅,你手中若有兵刃,一杖已擊在我的臂上。這一招如何化解?」苦頭陀反手做個姿勢,抓住鐵杖,左足飛出,頭一抬,顯是已奪過敵人鐵杖,同時將人踢飛。這幾下似拙實巧,看來已不是崑崙派的家數。趙明笑道︰「好師傅,你快教教我。」神情又嬌又媚,張無忌心中怦的一跳,心想︰「你内力不彀,這一招學不來的。可是她這麼求人,實是教人難以推却。」苦頭陀做了兩個手勢,正是示意道︰「你内力不彀,没法子學。」一跛一拐的走開,不再理她。無忌尋思︰「苦頭陀武功之強,只怕和玄冥二老不相上下,雖不知内力如何,但招數神妙,大是勁敵。他只打手勢不説話,難道是個啞巴?可是他耳朶又不聾,決不能啞。趙姑娘對他頗見禮遇,定是個大有來頭的人物。」

趙明見苦頭陀不肯再教,微微一笑,也不生氣,説道︰「叫崆峒派的唐文亮來。」過不多時,唐文亮被押著進殿,鹿杖客又派了三個人和他過招。唐文亮不肯在兵刃上吃虧,空手心掌,先勝兩場,到第三場上,對手催動内力,唐文毫無可與抗,亦被斬斷了一根手指。這一次趙明練招,由鹿杖客在旁指點。無忌此時已瞧出端倪,趙明顯是内力不足,情知非數載之間可以速成,是以想盡學各家各派之所長,俾成一代高手,這條路子,原亦可行,招數精到極處之時,大可補功力之不足。

趙明練過掌法,説道︰「叫滅絶老尼來!」一名黃衣人稟道︰「滅絶老尼已絶食五日,今天仍是倔強如昔。」趙明笑道︰「餓死了她吧!唔,叫峨嵋派那個小姑娘周芷若來。」手下人答應了,轉身出殿。

張無忌重回武當,數月之間,早已將别來經過一一向張三丰稟明,得知峨嵋派的周芷若,便是當日在漢水船中所遇的那個少女,雖然其時年幼,説不上什麼男女之情,但於她殷殷照料之意,常懷感激。光明頂上周芷若刺他一劍,那也是奉了師父的嚴命,張無忌心中不存仇怨,這時聽趙明帶她前來,不禁心頭一震。

過了片刻,一群黃衣人押著周芷若進殿。張無忌望了她幾眼,但見她比之在光明頂上時,略覺憔悴,但容顏清麗如昔,雖是身處敵人掌中,却泰然自若。鹿杖客正要派人和她比劍,趙明忽道︰「周姑娘,你這麼年輕,已是峨嵋派的及門高弟,著實令人生羨。聽説你是滅絶師太的得意弟子,深得她老人家劍招絶學,是也不是?」周芷若道︰「家師武功博大精深,説到傳她老人家劍招絶學,那是談何容易?」趙明笑道︰「這裡的規矩,只要誰能勝得咱們這裡三人,那便平平安安的送他出門,再無絲毫留難。尊師何以這般涯岸向高,不屑跟咱們切磋一下武學?」周芷若道︰「家師是寧死不辱。堂堂峨嵋掌門豈肯在你們手下苟且求生?你説得不錯,家師確是瞧不起卑鄙陰毒的小人,不屑跟你動手過招。」趙明竟不生氣,笑道︰「那周姑娘你呢?」周芷若道︰「我小小女子,有什麼主張?師父怎麼説,我便怎麼做。」趙明道︰「尊師叫你也不要跟咱們動手,是不是?那爲了什麼?」周芷若道︰「峨嵋派的劍法,雖不能説是什麼了不起的絶學,終究是中原正大門派的武功,不能讓無恥之徒偸學了去。」

趙明一怔,没料到自己的用心,居然會給滅絶師太猜到了,聽她左一句「陰毒小人」,右一句「無恥之徒」,忍不住心頭有氣,嗤的一聲輕響,倚天劍已執在手中,説道︰「你師父罵咱們是無恥之徒。好!我倒要請教,這口倚天劍明明是我家家傳的寶物,怎地會給你峨嵋派偸盜了去?」周芷若淡淡的道︰「故老相傳,倚天劍和屠龍刀,乃是中原武林中的兩大利器,却從没聽説和番邦女子有什麼干係?」趙明臉上一紅,怒道︰「哈!瞧不出你口齒伶俐得緊。你今日是決意不肯出手的了?」周芷若搖了搖頭。趙明道︰「旁人比武輸了,或是不肯動手,我都截下他們一根指頭。你這小{\upstsl{妞}}児想必自負花容月貌,以致這般驕傲,我也不截你的指頭。」説著伸手向苦頭陀一指,道︰「我叫你跟這位師父一樣,臉上劃你二三十道劍痕,瞧你還驕傲不驕傲?」

她左手一揮,兩個黃衣人搶上前來,執住了周芷若的雙臂。趙明微笑道︰「要劃得你的俏臉蛋變成一個蜜蜂窩,那也不必會什麼峨嵋派的精妙劍法。你以爲我三脚貓的把式,就不能叫你變成個醜八怪麼?」周芷若珠泪盈眶,身子發顫,眼見那倚天劍的劍尖離開自己臉頰不過數寸,只要這惡魔手腕一送,自己轉眼便和那那個醜陋可怖的頭陀一模一樣。趙明笑道︰「你怕不怕?」周芷若再也硬不起來,點了點頭,低聲道︰「怕。」趙明道︰「好啊!那麼你是降順了?」周芷若道︰「我不降!你把我殺了吧!」趙明笑道︰「我從來不殺人的。我只劃破你一點児皮肉。」

寒光一閃。趙明手中長劍便往周芷若臉上劃去,突然間{\upstsl{噹}}的一響,殿外擲進一件物事,將倚天劍撞了開去。在此同時,殿上長窗震破,一人飛身而入,那兩名握住周芷若的黃衣人身上不由自主的向外跌飛,一人迴左臂獲住了周芷若,伸右掌和鹿杖客砰掌一掌相交,各自退開了兩步。衆人看那人時,正是明教教主張無忌。

張無忌這一下闖入救人,當眞是如同飛將軍從天而降,誰都大吃一驚,即令是玄冥二老這種一等一的高手,事先竟没絲毫警覺。鹿杖客聽得長窗破裂,即便搶在趙明身前相護,和張無忌一掌相交,竟然立足不定,退開兩步,待要提氣再上,刹時間全身燥熱難當,宛似突然間跌入了一座熔爐之中。原來雙掌相交,張無忌的九陽眞氣逼進了他的體内,鹿杖客練的是至陰至寒的内功,一遇純陽之氣侵襲,難以寧靜。玄冥二老的另一個鶴筆翁在旁瞧見,一怔之下,急忙搶到他的身後,握住了他的左手,合兩人之力,這纔將九陽眞氣消淨。

周芷若眼見大禍臨頭,萬難避過,不料竟會有人突然出手相救。她被張無忌左手摟在胸前,碰到他寬廣堅實的胸膛,鼻中只聞到一股濃烈的男子氣息,又驚又喜,一刹那間身子軟軟的幾欲暈去。要知張無忌以又九陽神功和鹿杖客的玄冥神掌相抗,全身眞氣鼓盪而出,周芷若情竇初開,自知人事以來,從未和男子如此肌膚相親,氣息相聞,何況這男子又是她日夜思念的夢中之伴、意中之人?心中只覺得無比的喜歡,四周敵人如在此刻千刀萬劍同時斬下,她也無憂無懼。

楊逍和韋一笑一見教主衝入救人,跟著便閃身而入,分站在他身後左右。趙明手下的衆高手以變起倉卒,初時微見慌亂,但隨即瞧出闖進殿來的敵人只有三人,殿内殿外的守衛武士呼哨相應,知道外邉再無敵人,當下衆人立即堵死了各處門戸。靜候趙明發落。趙明既不驚惶,也不生氣,只是怔怔的向張無忌望了一陣,眼光轉到殿角兩塊金光燦爛之物,原來她伸倚天劍去劃周芷若的臉時,張無忌擲進一物,撞開她的劍鋒,所用之物正是她贈給無忌的黃金盒子。倚天劍鋒鋭無倫,一碰之下,早將金盒子剖成兩半。她向兩半金盒凝視半晌,向張無忌道︰「這隻盒子,你如此厭惡,非要它破損不可麼?」無忌見到她眼光中充滿了幽怨之意,並非憤怒責怪,竟是淒然欲絶,一怔之下,甚感歉咎,柔聲道︰「我身上没帶暗器,匆忙之際,隨手在懷中一探,摸了盒子出來,實非有意,望趙姑娘莫怪。」趙明眼中光芒一閃,問道︰「這盒子是你隨身帶著麼?」張無忌道︰「是。」見趙明妙目凝望自己,而自己一隻手還摟著周芷若,臉上微微一紅,便把周芷若放開了。趙明嘆了口氣,道︰「我不知周姑娘是你\dash{}是你的好朋友,否則也不會這般對她。原來你們\dash{}」説到這裡,將頭轉了開去。張無忌道︰「周姑娘和我\dash{}也没什麼\dash{}只是\dash{}只是\dash{}」説了兩個「只是」,却接不下去。趙明又轉頭對那兩半截金盒望了一眼,没説一句話,可是眼光神色之中,却是説了千言萬語。

周芷若心頭一驚︰「這個女魔頭對他顯是十分鍾情,難道世上竟有此事?」張無忌的性格心情,却不似這兩個少女細膩周至,趙明的神色他只模模糊糊的懂了一些児,全没體會到其中深意。他只覺得趙明贈他珠花金盒,治好了兪岱岩和殷利亨的殘疾,此時他却將金盒毀成兩截,未免對人家不起,於是走向殿角,俯身拾起兩半截金盒,説道︰「我去請高手匠人,將金盒重行鑲好。」趙明喜道︰「當眞麼?」張無忌點了點頭,暗覺奇怪,心想你我都統率無數英雄豪傑,怎麼去重視這些無関緊要的金銀玩物?何況這隻黃金盒子雖然做得甚是精緻,也不能算是什麼珍異寶物。盒中所藏的黑玉斷續膏已經取出,盒子便無多大用處,破了不必掛懷,再鑲好它,也是小事一樁。眼前有多少大事待決,你却儘跟我説這隻盒子,想必是年輕姑娘婆婆媽媽,對這種身邉瑣事特别関心,眞是女流之見,當下將兩截金盒子揣在懷中。

趙明道︰「那你去吧!」張無忌心想宋大師伯等尚未救出,怎能就此便去,但敵方高手如雲,己方只有三人,説到救人,眞是談何容易,問道︰「趙姑娘,你擒獲我大師伯等人,究竟意欲何爲?」趙明笑道︰「我是一番好意,要勸他們爲朝廷出力,各享榮華富貴。那知他們固執不聽,我迫於無奈,只得慢慢勸説。」張無忌哼了一聲,轉身回到周芷若的身旁,他在敵方衆高手環伺之下,俯身拾盒,坦然而回,竟是來去自如,旁若無人。他冷冷的向衆人掃視一眼,説道︰「既是如此,咱們便告辭了!」説著擕住周芷若的手,轉身欲出。

趙明道︰「你自己要去,我也不留。但你把周姑娘帶去,居然不來問我一聲,你當我是什麼人了?」張無忌道︰「這倒是在下少禮了。趙姑娘,請你放了周姑娘,隨我同去。」趙明不答,向玄冥二老使個眼色。鶴筆翁踏上一步,説道︰「張教主,你説來便來,説去便去,要救人便救人,教咱們這夥人的老臉往那裡擱去?你不留下一手絶技,兄弟們難以心服。」張無忌聽了鶴筆翁的聲音,怒氣上沖,喝道︰「當年我幼小之時,被你擒住,性命幾乎不保。今日你還有臉來跟我説話?接招!」呼的一掌,便向鶴筆翁拍了過去。鹿杖客適纔吃過他的苦頭,知道單憑鶴筆翁一人之力,不是他的敵手,搶上前來,向他擊出一掌。張無忌右掌仍是擊向鶴筆翁,左掌從右掌下穿過,還了鹿杖客一掌。這是眞力對眞力相碰,中間實無閃避取巧的餘地。三個人四掌相交,身子各是一晃。

當日在武當山上,玄冥二老以雙掌和張無忌對掌,另出雙掌擊在他的身上,此刻重施故技,又是兩掌拍了過來。張無忌那日吃了此虧之後,早已想到破解之法,焉能重𨂻覆轍?手肘微沉,乾坤大挪移心法展開,拍的一聲大響,鶴筆翁的左掌擊在鹿杖客的右掌之上。兩人掌法相同,功力相若,都是震得雙臂酸麻,至於何以竟會弄得自己人和自己人比拚掌力,他二人武功雖高,竟然也不明白其中的訣竅。兩人又驚又怒之下,張無忌雙掌又已擊到。玄冥二老仍是各出雙掌,一守一攻,所用掌法已和適纔全然不同,但被張無忌一引一帶,仍是鹿杖客的左掌擊到了鶴筆翁的右掌之上,這挪移乾坤的手法之巧,計算之準,實已到了匪夷所思的地步。

玄冥二老駭然失色,眼見張無忌第三次舉掌擊來,竟然不約而同的各出單掌。三個人眞力相交,玄冥二老只覺對方掌力中的那股純陽之氣,激盪得自身氣血翻湧,極是難耐。張無忌一招快似一招,想起幼時被鶴筆翁打了一招玄冥神掌,數年之間不知吃了多少苦頭,他縱是十分寬宏大量,此刻想起來也不免心頭有氣,因此擊向鹿杖客的一掌尚留餘地,對鶴筆翁却是毫不放鬆。

二十餘掌一過,鶴筆翁一張青臉已脹得通紅,眼見張無忌又是一掌擊到,他左掌虛引,意欲化解,右掌却斜刺裡重重擊出。只聽得拍拍兩響,鶴筆翁這一掌狠狠的打在鹿杖客肩頭,而張無忌那一掌却終究無法化開,被他正好打中在胸口。總算張無忌不欲傷他性命,這一掌眞力只用了三成,鶴筆翁哇的一聲,吐出一口鮮血,臉色更是紅得發紫,身子搖搖晃晃,倘若張無忌乘勢再補上一掌,那是非斃命當場不可。鹿杖客肩頭吃了這一掌,也是痛得臉色大變,嘴唇都咬出血來。

玄冥二老是趙明手下數一數二能人,豈知不出三十招,便各受傷。趙明手下衆人固然盡皆駭然,便是楊逍和韋一笑,也是大爲駭異,須知那日玄冥二老在武當山出手,張無忌的武功遠没今日之強,不意數月之間,進展神速若是。

原來張無忌留居武當數月,一面替兪岱岩、殷利亨治傷,一面便向張三丰請教武學中精微深奥的難題。九陽神功、乾坤大挪移、再加上武當絶學的太極拳劍,三者漸漸融成一體,幾乎已到了武學中最高的境界。楊逍和韋一笑略一思索,即明其理,不禁讚嘆張三丰學究天人,那纔眞的是稱得上「深不可測」四字。

玄冥二老比掌敗陣,齊聲呼嘯,同時撤出了兵刃。只見鹿杖客手中拿著一根短杖,杖頭分叉,作鹿角之形,通體黝黑,不知是何物鑄成;鶴筆翁手持雙筆,筆端鋭如鶴嘴,却是晶光閃亮。他二人追隨趙明已非一日,也從未見過他二人使用兵刃。這三件兵刃使展開來,只見一團黑氣,兩道白光,霎時間便將張無忌困在垓心。張無忌身邉不帶兵器,赤手空拳,情勢頗見不利,但他絲毫不懼,存心要試試自己武功,在這兩大高手圍攻之下,是否能空手抵敵。

玄冥二老自恃内力深厚,玄冥神掌是天下絶學,是以一上陣便和他對掌,比拚内力,豈知張無忌的九陽神功,却非任何内功所能及,因此數十掌一過便即落敗,但一用到兵刃,那是以招數詭異取勝。他二人的名號便是從所用兵刃上而得,鹿角短杖和鶴嘴雙掌,每一招都是凌厲狠辣,世所罕見。張無忌聚精會神,在三件兵刃之間穿來插去,攻守自如,只是一時瞧不通二人兵刃招數中的路子,想要取勝,却也不易。

趙明手掌輕擊三下,大殿中白刃耀眼,三個人攻向楊逍,四個人攻向韋一笑,另有兩人出兵刃制住了周芷若。但敵人數實在太多,每打倒二人,立時更有二人擁上,張無忌被玄冥二老纏住了,始終分身不出相援。他和楊韋二人若要全身而退,勉強或能辦到,要救周芷若却是萬萬不能,心下正自焦急,忽聽趙明説道︰「大家住手!」這四個字聲音並不響亮,她手下衆人却一齊凜遵,立即躍開。楊逍刷的一聲,將長劍還入劍鞘。韋一笑右手握著從敵人手裡奪來的一口單刀,順手一揮,擲還給了原主,哈哈大笑。趙明手下許多高手如苦頭陀等一直没有出手,倘若群相上前,張無忌等人自然寡不敵衆。但楊逍和韋一笑身處虎狼之域,竟是泰然自若,衆人心下不禁暗暗佩服。倒是張無忌看到一名漢子手執匕首,抵住周芷若後心,臉上頗有憂色。

周芷若黯然道︰「張公子,三位請即自便。三位一番心意,小女子感激不盡。」趙明笑道︰「張公子,這般花容月貌的人児,我見猶憐。定是你的意中人了?」張無忌臉上一紅,説道︰「周姑娘和我從小相識。在下幼時中了這位\dash{}」説著向鶴筆翁一指,「\dash{}的玄冥神掌,陰毒入體,周身難以動彈,多虧周姑娘服侍我食飯喝水,此番恩德,不敢有忘。」趙明道︰「如此説來,你們倒是青梅竹馬之交了。你是想娶她爲魔教的教主夫人,是不是?」張無忌臉上又是一紅,説道︰「匈奴未滅,何以爲家!」趙明臉一沉,道︰「你是定要跟我作對到底,非滅了我不可,是也不是?」張無忌搖了搖頭,説道︰「我至今不知姑娘的來歷,雖然有過數次爭執,但每次均是姑娘找上我張無忌,不是我張無忌來找姑娘尋事生非。只要姑娘放了我衆位師伯叔以及各派武林人士,在下感激不盡。決不敷衍推搪。」趙明笑道︰「嘿,總算你還没忘記。」轉頭向周芷若瞧了一眼,對張無忌道︰「這位周姑娘既非你意中人,也不是什麼師兄師妹、未婚夫妻,那麼我要毀了她的容貌,跟你絲毫没有干係\dash{}」

\chapter{光明右使}

她眼角一動,鹿杖客和鶴筆翁各挺兵刃,攔在周芷若之前,另一名漢子手執利刃,對準周芷若的臉頰。張無忌若要衝過來救人,玄冥二老這一関便不易闖過。趙明冷冷的道︰「張公子,你還是跟我説實話的好。」韋一笑忽然伸出手掌,在掌心吐了數口唾味沫,伸手在鞋底擦了幾下,哈哈大笑,衆人正不知他搗甚麼鬼,突然間青影一晃一閃。趙明只覺自己左頰右頰上被一隻手掌摸了一下,看韋一笑時,却已站在原地,只是手中多了兩柄短刀,不知是從何人腰間掏來的。趙明心念一動,知道不好,不敢伸手去摸自己臉頰,忙取手帕在臉上一擦,果見帕上黑黑的沾了不少汚泥,顯是韋一笑鞋底的汚穢再混著唾沫,思之幾欲作嘔。

只聽韋一笑説道︰「趙姑娘,你要毀了周姑娘的容貌,那也由得你。我張教主名揚四海,英俊瀟灑,要娶幾個美貌女子爲室,便是三妻四妾,又有何難?他壓根児就没將這位周姑娘放在心上。只是你心狠手辣,我姓韋的却放不過你。你今日在周姑娘臉上劃一道傷痕,姓韋的加倍奉還,劃傷兩道。你劃她兩道,我劃你四道。你斷她一根手指,我斷你兩根。姓韋的説得出,做得到,青翼蝠王言出必踐,生平没説過一句空話。你防得我一年半載,却防不得我十年八年。你想派人殺我,未必追得我上。告辭了!」這「了」字出口,早已人影不見。身法之快,衆人無不駭然。他這幾句話説來平平靜靜,但人人均知決非空言恫嚇,眼看趙明白裡泛紅、嫩若凝脂的粉頰之上,被韋一笑的汚手抹上了幾道黑印,倘若他手中先拿著短刀,趙明的臉頰早就損毀。這般來去如電、似鬼似魅的身法,確是再強的高手也防他不了,即令張無忌,必也是自愧不如。若是長途競走,無忌當可以内力取勝,但在庭除廊廡之間,如此趨退若神,當眞是天下只此一人而已。

張無忌躬身一揖,説道︰「趙姑娘,咱們就此告辭。」説著擕了楊逍之手,轉身出殿,心知在韋一笑十分有力的威嚇之下,趙明不敢再對周芷若如何。趙明瞧著他的背影,又羞又怒,却不下令攔截。

張無忌和楊逍回到客店,韋一笑已在店中相候。無忌笑道︰「韋蝠王,你今日給了他們一個下馬威,好叫他們得知明教可不是好惹的。」韋一笑道︰「嚇嚇小姑娘,倒也不是難事。她裝得兇神惡煞一般,可是聽我説要毀她的容貌,擔保她三天三晩睡不著覺。」楊逍笑道︰「她睡不著覺,那可不好,咱們前去救人更加難了。」張無忌道︰「楊左使,説到救人,你有什麼妙計?」楊逍躊躇道︰「咱們這裡只有三人,何況形跡已露,這件事當眞辣手。」張無忌歉然道︰「我見周姑娘危急,忍不住出手,終於壞了大事。」楊逍道︰「事勢如此,那是誰都忍不住的。教主獨力打敗玄冥二老,大殺敵人的威風,那也很好。」三人商談半晌,不得要領,當即分别就寢。

次日無忌醒來,一睜開眼,便見窗子打開,一張臉向著他他凝望。無忌吃了一驚,揭帳一看,只見那臉上疤痕累累,醜陋可怖,正是那個苦頭陀。無忌一驚更甚,從床中一躍而起,只見苦頭陀的臉仍是呆呆望著自己,却無出手相害之意。無忌心中一涼︰「怎地睡得如此大意?敵人早就到了窗外,居然並不驚覺?」叫道︰「楊左使!韋蝠王!」楊逍二人在鄰室齊聲相應。無忌心中一寬,却見苦頭陀的臉已從窗邉隱去。無忌縱身出窗,見苦頭陀從大門中匆匆出去,這時楊韋二人也已趕到,見此外並無敵人,三人發足向苦頭陀追去。那苦頭陀等在街角,一見三人走來,立即轉身,向北行去,脚步邁得甚大,却非奔跑。三人打個手勢,當即跟隨其後。

這苦頭陀一足雖跛,但邁開大步,行走甚是迅速。此時天方黎明,街上行人稀少,不多時便出了北門。苦頭陀繼續前行,折向小路,又走了七八里,來到一處亂石崗上,這纔停步轉身,向楊逍和韋一笑擺了擺手,要他二人退開,隨即抱拳向張無忌行禮。無忌還了一禮,心下尋思︰「這頭陀帶咱們來到此處,不知有何用意?這裡四下無人,若是動武,那是以一敵三,顯是落了下風,瞧他情狀,似乎不含敵意。」盤算未定,苦頭陀荷荷一聲,雙爪齊到,撲了上來。他左手處打,右手龍爪,十指成鉤,攻勢極是猛惡。

張無忌左掌揮出,化開了一招,説道︰「上人意欲如何?請先表明尊意,再行動手不遲?」苦頭陀毫不理會,竟似没聽見他説話一般,只見他左手自虎爪變成鷹爪,右手却自龍爪變成虎爪,一攻左肩,一取右腹,出手狠辣之至。張無忌道︰「當眞非打不可嗎?」苦頭陀鷹爪變獅掌,虎爪變鶴嘴,一擊一啄,招式又變,三招之間,雙手變了六種姿勢。張無忌不敢怠慢,施展太極拳法,身形猶如行雲流水,便在亂石崗上跟他鬥了起來。但覺這苦頭陀的招數甚是繁複,有時大開大闔,門戸正大,但倏然之間,又是詭祕古怪,全是邪派武功,顯是正邪兼修,淵博無比。張無忌只是用太極拳跟他拆招。鬥到七八十招時,苦頭陀呼的一拳,中宮直攻。張無忌一招「如封似閉」,將他拳力封住,跟著一招「單鞭」,右掌已拍在他的駝背之上。只是這一掌没發内力,手掌一沾即離。

苦頭陀知他手下留情,向後躍開,斜眼向張無忌望了半晌,突然向楊逍做個手勢,要借他長劍一用。楊逍解下劍鞘,連著劍鞘雙手托住,送到苦頭陀面前。張無忌暗暗稱奇︰「怎地楊左使將兵刃借了給敵人?」苦頭陀拔劍出鞘,打個手勢,叫張無忌向韋一笑借劍。張無忌搖搖頭,接過他左手拿著的劍鞘,使招「請手」,便以劍鞘當劍,左手捏了劍訣,劍鞘橫在身前。苦頭陀刷的一劍,斜刺而至。張無忌見他教導趙明學劍,知他劍術極是高明,絲毫不敢輕忽,施展這數月中在武當山上精研的太極劍法,凝神接戰。但見對手的劍招忽快忽慢,處處藏著機鋒,但張無忌一加拆解,他當即撤回,另使新招,幾乎没一招是使得到底了的。張無忌心下讚嘆︰「若是半年之前遇到此人,劍法上我未必能是他的敵手。比之那玉面神劍方東白,這苦頭陀又高一籌了。」

他心中一起愛才之念,不願在招數上明著勝他,眼見苦頭陀長劍揮舞,使出「亂披風」勢來,白刃映著日光,有如萬道金蛇,在空中亂鑽亂竄。張無忌看得分明,驀地裡倒過劍鞘,刷的一聲響,劍鞘已套在劍刃之上,雙手環抱一搭,輕輕扣住苦頭陀雙手手腕,微微一笑,縱身後躍。他手上只須略加使勁,便已將長劍奪了過來。這一招奪劍之法,險是險到了極處,巧也巧到了極處。

那知他縱身後躍,身子尚未落地,苦頭陀已抛下長劍,呼的一掌拍到。張無忌聽到風聲,知道這一掌中眞力充沛,非同小可。他有意試一試苦頭陀的内力到了何等地步,右掌迴轉,硬碰硬的接了他這掌,左足這纔著地。霎時之間,苦頭陀掌上眞力源源催至。張無忌運起乾坤大挪移心法中最高深的功夫,將對方掌力漸漸積蓄,突然間大喝一聲,反彈出去,便如一個大水庫在山洪爆發時儲滿了洪水,猛地裡開閘放水,將苦頭陀送來的掌力盡數送回。那等於是將苦頭陀一二十掌的掌力歸無成爲一掌拍出,世上原無如此大力。這一掌苦頭陀倘若受實了,勢須立時腕骨、臂骨、肩骨、肋骨一齊折斷,連血也噴不出來,當場成一團血肉模糊,死得慘不可言。

此時雙掌相黏,苦頭陀萬難閃避,張無忌左手抓住他的胸口往上一抛,苦頭陀一個龐大的身軀向上飛起,只聽得砰的一聲巨響,亂石橫飛,這一掌威力無儔的開山之掌,盡數打在亂石堆裡。楊逍和韋一笑在旁看到這等聲勢,齊聲驚呼出來,他二人只道苦頭陀和教主比拚内力,至少也得一盞茶時分,方能分别高下,那料到片刻之間,便到了決生死的関頭。二人心中雖有話説,却已不及言講,待見苦頭陀平安無恙的躍下地來,手心中都已捏了一把冷汗。

苦頭陀雙足一著地,登時雙手作火燄飛騰之狀,放在胸口,躬身向張無忌拜了下去,説道︰「小人光明右使范遙,參見教主。敬謝教主不殺之恩。小人無禮冒犯,還請恕罪。」張無忌大吃一驚,這啞巴苦頭陀,不但開了口,而且更是本教的光明右使,這一著實非始料所及,急忙伸扶起,説道︰「原來是本教范右使,自家人不須多禮。」楊逍和韋一笑跟他到亂石崗來之時,早已料到了三分,只是范遙的身形面貌變化實在太大,不敢便説,待得見他施展的武功,更是猜到了七分,這時聽他自報姓名,兩人搶上前來,緊緊握住了他的手。楊逍向他臉上凝望半晌,不禁潸泪下,説道︰「范兄弟,做哥哥的想得你好苦。」范遙抱住楊逍身子,説道︰「大哥,多謝聖神佑護,賜下教主這等能人,你我兄弟終有重會之日。」楊逍道︰「兄弟怎地變成這等模樣?」范遙道︰「我若非自毀容貌,自殘肢體,怎瞞得過混元霹靂手成崑那奸賊?」三人一聽,都知他是故意毀容,混入敵人身邉臥底,以便偵査當年楊破天教主的死因。楊逍更是傷感,説道︰「兄弟,這可苦了你了。」要知楊逍、范遙當年江湖上人稱「逍遙二仙」,都是英俊瀟灑的美男子,此刻范遙竟然變得醜陋不堪,其苦心孤詣,實非常人所能爲。韋一笑性情古怪,向來和范遙不睦,但這時也不由得深爲所感,拜了下去,説道︰「范右使,韋一笑到今日纔算服了你。」范遙跪下還拜,笑道︰「韋蝠王輕功獨步天下,越老越妙,苦頭陀昨晩大開眼界。」

楊逍四下一望,説道︰「此處離城不遠,敵人耳目衆多,咱們到前面山坳中説話。」四人邁開脚步,奔出十餘里,到了一個小崗之後,該處一望數里,不愁有人隱伏偸聽,但從遠處却瞧不見崗後的情景。四人坐地,説起别來情由。

原來那日楊破天突然間不知所蹤,明教衆高手爲爭教主之位,互不相下,以致四分五裂。范遙却深信教主並未逝世,獨行江湖,尋訪楊教主的下落,忽忽數年,没發現絲毫蹤跡,後來想到或許是爲丐幫所害,暗中捉了好些丐幫的重要人物拷打逼問,仍是問不出半點端倪,倒是無辜害了不少丐幫的幫衆。此時聽到明教諸人紛爭,鬧得更加厲害,有人正在到處尋他,蓋范遙在明教中地位極高,倘若以他號召,自然立時聲勢大盛。范遙心灰意懶之下,竟去出家做了個帶髮頭陀。

也是事有湊巧,這日他在太行山脚下經過,爲避大雨,在一座破廟中躱雨,無意中偸聽到了兩個人的談話,其中之一便是成崑。另一個却是個和尚,後來纔得知,那是少林寺四大神僧之首的空見大師。范遙在光明頂上曾見過成崑,知道他是楊教主的師弟,本想待他二人説過正事,便即出來相見,那知只聽得幾句,便驚得呆了。只聽見成崑跪在地下,向空見神僧深自懺悔,説他如何酒酔之下,逼姦弟子謝遜的妻子,又不合殺了他全家老少,以致謝遜到處找他尋仇。他始終避不見面,謝遜便殺害武林中成名的英雄好漢,留下了成崑的姓名。

張無忌等早知成崑和謝遜結仇的經過,但此時聽苦頭陀范遙説起,仍是心有餘憤。范遙接著説,那日在破廟中只聽成崑痛哭流涕,苦求空見大師收錄他爲弟子,以佛家大慈悲力,化解他的罪孼。空見大師説道︰「善哉善哉!苦海無邉,回頭是岸。放下屠刀,立地成佛。們既眞心懺悔,佛門廣大,決不拒你於門外。」當下便給他剃度,收他弟子,還答應助他了結謝遜這一作冤孼。

范遙説到這裡,張無忌簡述謝遜擊斃空見神僧的經過,這位神僧甘受謝遜開山破石般拳力的打擊,全是在盼望化解武林中一樁大血仇,那知成崑竟然欺騙了師父,在他臨死時隱身不出,不和謝遜相見。接著楊逍又説到成崑如何偸襲光明頂,明教遭受大難,但這奸賊終於和殷天正、殷野王父子比拚掌力,力盡身死。范遙雙手合什,説道︰「阿彌陀佛,善哉善哉!」楊逍見這個當年的風流人物,今日竟成爲如此模樣,不禁黯然神傷。

范遙説道︰「金毛獅王和我素來交好,他全家遭難之事,我也略有所聞,那料到竟是他的業師所爲。大雨停後,他二人出廟而去,我便悄悄跟在後面。我知他二人武功了得,只是遠遠躡著,那知空見大師居然還是知道了,在前高宣佛號,説道佛家子弟,須當慈悲爲懷。我便不敢再跟。過了一年,忽聽得空見神僧的死訊,那心中疑竇大起,料想必和成崑有関,於是暗暗到少林寺偵査。我不敢逕到寺中,只在嵩山附近察看,果然皇天不負苦心人,聽到了成崑和朝廷密使的説話。那朝廷密使不是旁人,便是昨晩敗於教主手下的鹿杖客,只是他二人武功太高,我一人決計不是對手,由於離得太遠,隱隱約約的只聽到三言兩語,但『須當毀了光明頂』這七個字,我却是聽得清清楚楚的。屬下既知本教有難,不敢置身事外,一路跟隨鹿杖客來到京師。鹿杖客我是不敢惹他,其餘次一流的人物那就没有什麼,我終於打聽明白,這一干武林人物,都是汝陽王察罕特穆爾的手下。」

那汝陽王察罕特穆爾乃朝廷宗室,官居太尉,執掌天下兵馬大權,智勇雙全,是朝廷中的第一位能人,江淮義軍起事,均被他遣兵撲滅。義軍屢起屢敗,皆因察罕特穆爾統兵有方之故。張無忌等久聞其名,這時聽到鹿杖客等乃是他的手下,雖不驚訝,却也不禁爲之一怔。楊逍問道︰「那麼那個趙姑娘是誰?」

范遙道︰「大哥不妨猜上一猜。」楊逍道︰「莫非是察罕特穆爾的女児?」范遙拍手道︰「不錯,一猜便中。這位汝陽王生有一子一女,児子叫做庫庫特穆爾,女児便是這位姑娘了,她的蒙古名叫什麼紹明郡主。這兩個孩子都生性好武,倒學了一身好武功。兩人又喜歡作漢人打扮,説漢人的話,各自取了一個漢名,男的叫做王保保,女的便叫做趙明了。『趙明』二字,是從她的封號『紹明郡主』而來。」韋一笑道︰「這兄妹二人倒也古怪,一個姓王,一個姓趙,倘若是咱們漢人,那可笑煞人了。」范遙道︰「其實他們姓特穆爾,却把名字放在前面,這是番邦蠻俗。」
\footnote{\footnotefon{}東方姓氏普遍前置,名字後置。西方姓氏普遍族姓後置,名字前置。

上古中國姓氏,一般取族姓。上古八大姓︰姬、姜、姚、嬴、姒、{\upstsl{妘}}、嬀、姞,皆從女,係皆出自母系氏族。姓爲族姓。氏爲父氏。由於分封、遷徙、放逐等政治活動,分家一般都要改氏,以示同宗家的區别。春秋時期父系氏族崛起,人們稱氏不稱姓。如陳完(前706年 - ?)嬀姓,陳氏,爲陳厲公之子,陳國内亂入齊國後,又改爲田完(陳與田古音相近),歸附齊國,爲大夫,諡敬仲,是田齊的始祖及田姓得姓始祖。其後裔中有中國古代著名軍事家田穰苴,孟嘗君田文,吳孫子孫武,齊孫子孫臏,新朝皇帝王莽,吳大帝孫權等。

也有同時稱姓與氏的,如秦王嬴趙政(亦可稱秦政、嬴政、趙政;始皇嬴政)。秦漢時期姓氏合流,氏從此取代姓。上古姓氏作爲宗家的宗家,今天已十分罕見了。六次人口普査,人口前十分别是李、王、張、劉、陳,楊、趙、黃、周、吳。

日本的姓多爲天皇賜予,類似印度的種姓(如臣、連、君、直、相臣、伴造、百八十部、國造、縣主、眞人、朝臣、宿禰、忌寸、道師、稲置等約三十個姓,wiki);氏名則類似上古的族姓;苗字則類似上古的氏名;苗字用於區别宗家和分家。日本的通字完全不同于我國慣用的輩分排行字號。(如織田信長,其完整姓名爲「織田彈正忠平朝臣信長」。其中織田爲「家名・苗字」,「彈正忠」爲通稱(官名),「平」爲「氏(うじ)」,「朝臣」爲「姓(かばね)」,「信長」爲「諱(いみな)」。 又如,德川家康的全名是「德川次郎三郎源朝臣家康」,其中「德川」是「家名・苗字」,「次郎三郎」是「通称」,「源」是「氏(うじ)」,而「朝臣」是「姓(かばね)」,「家康」是「諱(いみな)」。wiki)

也有不采用族姓,子輩使用其父名爲姓,孫輩仍使用其父名爲姓。如某些印度姓氏,及某些少數民族姓氏。槩因地區發展不均衡,無法形成統一的宗族觀念所致。

亦有全盤采納漢姓,如黨項族的李姓;鮮卑族的元氏(拓跋氏);葉赫那拉的葉氏、那氏;佟佳氏的佟氏;納西族的木氏、和氏。

人們改姓有出於賜姓目的的(如明朝的鄭和、鄭成功等),有出於避禍的(如馮、同,皆出自司馬遷),有因過繼從而改姓的,這種最普遍。也有因少數民族漢化而改姓的。
總之,改姓是一件非常複雜且麻煩的事。}
楊逍道︰「瞧這位趙姑娘的容貌身裁,活脱是漢人的一個美女,可是只須見她一行事,那番邦女子的兇蠻野性,立時便顯露了出來。」

張無忌直到此刻,方知趙明的來歷,雖知她必是朝廷貴人,却没料到竟是天下兵馬大元帥汝陽王的郡主。和她交手數次,每次都是多多少少的落了下風,雖然她武功遠遠不及自己,但心思機敏、奇變百出,自己却又遠不是她的敵手。

范遙接著説道︰「屬下暗中繼續探聽,得知汝陽王決意剿滅江湖上的教派會。他顯是採納了成崑的計謀,第一步便想要除滅本教。我仔細思量,本教内部紛爭不休,外敵却如此之強,滅亡的大禍已迫在眉捷,要圖挽救,只有混入王府,査知汝陽王的謀劃,那時再相機解救。除此之外,實在别無其他良策。只是我曾和成崑朝過相,要使我所圖謀不致洩露,只有想法子殺了此人。」韋一笑拍手道︰「正該如此。」范遙道︰「可是此人實在狡猾,武功又強,我接連暗算了他三次,每一次都没成功。第三次雖然刺中了他一劍,我却也被他劈了一掌,好容易纔得脱逃,不致露了形跡,但却已身受重傷,養了年餘纔好。這時汝陽王府中圖謀更急,我一咬牙,便毀了自己容貌,打折了腿,假裝駝背啞巴,投到了西域花剌子模國去。」

韋一笑道︰「到花子剌子模?萬里迢迢的,跟這事有什麼相干?」范遙一笑,正待回答,楊逍拍手道︰「兄弟,此計太妙。韋兄,范兄弟到了花剌子模,找個機緣一顯身手,那邉的蒙古王公必定收錄。汝陽王正在招聘四方武士,花剌子模的王公爲了討好汝陽王,定然會送他到王府效力。這麼一來,范兄弟成了西域花剌子模國進獻的武士,他容貌已變,又不開口,成崑便有天大的本事,也認他不出了。」韋一笑長聲一嘆,説道︰「楊教主派逍遙二仙排名在四大法王之上,確是目光如炬。這等計謀,什麼鷹王、蝠王,都是想不出來的。」范遙道︰「韋兄,你讚得我也彀了。教主,有一件事屬下須得向你領罪。」張無忌道︰「范右使何必過謙?」范遙站起身來,恭恭敬敬的説道︰「屬下犯了殘殺本教兄弟的重罪。屬下果如楊左使所料,在花剌子模殺獅斃虎,頗立威名,當地王公便送屬下到汝陽王府中。屬下爲了堅王爺之信,在大都鬧市之中,親手格斃了本教三名香主,顯得本人和明教早就結下深仇。」張無忌沉默半晌,心想︰「殘殺本教兄弟,原是本教五大禁忌之一,因此楊左使、四法王、五行旗等雖然爭奪教主之位,儘管相鬥甚烈,却從來不傷本教兄弟的性命,范右使此罪實是不輕,但他主旨是爲了護教,非因私仇,按理又不能加罪於他。」於是説道︰「范右使出於護教苦心,本人不便深責。」范遙躬身道︰「謝教主恕罪。」張無忌暗想︰「這位范右使行事之辣手,世所罕有。他能在自己臉上砍上十七八刀,能將自己好好一條大腿打折,那麼殺幾個教中無辜的香主,自也不在他的意下。明教被人稱作邪教魔教,其來有自,不知將來如何方得改了這些邪氣魔氣?」

范遙見張無忌口中雖然説「不便深責」,臉上却有不豫之色,一伸手,拔出楊逍腰間長劍,左手一揮,已割下了右手二根手指。張無忌大吃一驚,挾手搶過他的長劍,説道︰「范右使,你\dash{}你\dash{}這是爲何?」范遙道︰「殘殺本教無辜兄弟,乃是重罪。范遙大事未了,不能自盡。先斷二指,日後再斷項上這顆人頭。」張無忌道︰「本人已恕了范右使的過失,何苦如此?身當大事之際,唯須從權,范右使,此事不必再提。」忙取出金創藥,替他敷了傷處,撕下自己衣襟,給他包紮好了,恐怕日後眞的會自刎謝罪,想到他爲本教受了這等重大的折磨,心中大是感動,突然雙膝跪倒,説道︰「范右使,你有大功於本教,受我一拜。你再殘害自身,那便説我無德無能,不配當此教主大任。你再自刺一劍,我便自刺兩劍。我年幼識淺,不明事理,原是分不出好歹。」范遙、楊逍、韋一笑見教主跪倒,急忙一起拜伏在地。

楊逍垂泪道︰「范兄弟,你休得再是如此。本教興衰,全繫教主一人,教主令旨,你可千萬不能違背。」范遙拜道︰「屬下今日比拳試掌,對教主已是死心塌地的拜服。苦頭陀性情乖張,還請教主原宥。」張無忌雙手扶他起身,經此一事,他和范遙相互知心,再無隔門闋。范遙當下再敘投入汝陽府後的所見所聞。那汝陽王察罕特穆爾實有經國的大才,雖握兵權,朝政却被奸相把持,加之當今皇帝昏庸無道,弄得天下大亂,民心沸騰,全仗汝陽王東征西討,擊潰義軍無數。可是此滅彼起,歳無寧日,汝陽王忙於調兵遣將,將撲滅江湖上教派幫會之事,暫且擱在一邉。數年之後,他一子一女長大,世子庫庫特穆爾(王保保)隨父帶兵,女児明明特穆爾(趙明)竟然統率蒙漢西域的武士番僧,向教派幫會大舉進擊。成崑暗中助她策劃,乘著六大派圍攻光明頂之際,由趙明帶同大批高手,企圖乘機收漁人之利,將明教和六大派一鼓剿滅。綠柳莊中下毒等等情由,便是因此而起。只是近年來范遙奉命在海外搜尋謝遜下落,西域之行没能參與,直到後來方始得知,趙明以西域番僧所獻出毒藥「十香軟筋散」,暗中下在從光明頂歸來的六大派高手的飲食之中。那「十香軟筋散」味鹹如鹽,清香似菜,想那鹽粒青菜一般的滋味,混在菜餚之中,有誰能來辨得出?這毒藥的藥性一發作,全身筋骨日漸酸軟,雖能行動如常,内力却已半點發揮不出,因此六大派遠征明教的衆高手在一月之内,一一分别就擒。只是在對華山派下毒時機會不巧,被人撞破,但眞刀眞槍的動起手來,華山派還是不敵玄冥二老、神箭八雄、以及阿大、阿二、阿三等人的身手,死了十多人後,餘人盡數被囚。

擒獲少林群僧,用的仍是這個法子。但少林寺平常防衛嚴密異常,要想混入寺中下毒,那可是大大的不易,不比行旅之間,必須在市鎭客店中借宿打尖,那麼下毒輕而易舉。范遙説道︰「我本來只道是成崑幹的好事,他以空見神僧之弟子的身份,要在寺中下毒自是不費吹灰之力,可是他既已喪命光明頂上,這件事就十分奇怪了。我剛從海外歸來,正好趕上了圍擒少林群僧之役,只是我向來不開口,不便向人探詢下毒的情形,何況少林派向來對本教無禮,讓他們多吃些苦頭,正是人心大快。就算將少林派的臭和尚們一起都殺光了,苦頭陀也不皺一皺眉頭。教主,你又要不以爲然了,哈哈!」楊逍插口道︰「兄弟,那尊達摩像,是你做的手脚了?」范遙笑道︰「我見郡主叫人在達摩石像的臉上刻下了那幾個字,意圖嫁禍本教,我後來便又悄悄回去,將達摩像推轉,叫他仍是面壁參禪。大哥,你們倒眞心細,這件事還是叫你們瞧了出來。那時候你可想得到是兄弟麼?」楊逍道︰「咱們推敲起來,對頭之中,似有一位高手在暗中維護本教,可那能想得到竟是我的老搭擋好兄弟!」他説到這裡,四人一齊大笑。

於是楊逍向范遙簡略説明,明教決和六大派捐棄前嫌,共抗蒙古,因此定須將衆高手救了出來。范遙道︰「敵衆我寡,單憑我們四人,難以辦成此事,須當尋得十香軟筋散的解藥,給那一干臭和尚、臭尼姑、牛鼻子們服了,待他們回復内力,一鬨衝出,攻韃子們一個措手不及然後一齊逃出大都。」他十多年不開口,説起話來,聲調已然很不自然,加之明教向來和少林、武當等名門正派是對頭冤家,是以言語之中,對衆高手竟是毫不客氣。楊逍向他連使眼色,范遙決不理會。張無忌對這小節却全不爲逆,拍手説道︰「范右使之言不錯,只不知如何取得十香軟筋散的解藥?」

范遙道︰「我從不開口,因此郡主雖對我頗加禮敬,却向來不跟我商量什麼要緊事。只有她一個人自言自語,對方却不答一句話,那豈不掃興?加之我來自西域小國,她亦不能將我當作心腹,所以那十香軟筋散的解藥是什麼,我却無法知道。不過我知此事牽涉重大,暗中早就留上了心,如我所料不錯,那麼這毒藥和解藥是由玄冥二老分掌,一個管毒藥,一個管解藥,而且經常輪流掌管。」楊逍嘆道︰「這位郡主娘娘心計之工,一般鬚眉男子也及她不上。難道她對玄冥二老也不放心麼?」范遙道︰「一來當是不放心,二來也是更加穩當。好比咱們此刻想奪想偸解藥,就不知是找鹿杖客好呢,還是找鶴筆翁好。而且,毒藥和解藥氣味顏色完全一般無異,若非掌藥之人知曉,旁人去偸藥,説不定反而偸了毒藥。須知那十香軟筋散另有一般厲害處,一個人中了此毒後,筋萎骨軟,自是不在話下,倘若第二次再服毒藥,就算只有一點児粉末,也是立時血逆氣絶,無藥可救。」韋一笑伸了伸舌頭,道︰「如此説來,解藥是萬萬不能偸錯的。」范遙道︰「話是如此説,咱們只管把玄冥二老身上的藥偸來,找一個華山派、崆峒派的小脚色來試一試,那一種藥整死了他,便是毒藥了,這還不方便麼?」張無忌知他邪氣未脱,不把别人的性命放在眼裡,只笑了笑,説道︰「那可不好。説不定咱們辛辛苦苦偸來的兩種都是毒藥。」楊逍一拍大腿道︰「教主此言有理。咱們昨晩這麼一鬧,或許把郡主嚇怕了,竟把解藥放在自己身邉。依我説,咱們須得先行査明解藥由何人掌管,然後再計議行事。」也沉吟片刻,説道︰「范兄弟,那玄冥二老生平最喜歡的是什麼調調児?」范遙道︰「鹿好色,鶴好酒,那還有什麼好東西了。」

楊逍問張無忌道︰「教主,可有什麼藥物,能使人筋骨酸軟,好似中了十香軟筋散一般?」張無忌想了一想,笑道︰「要使人全身乏力,昏昏欲睡,那並不難,只是用在高手身上,不到半個時辰,藥力便消,要像十香軟筋散那麼厲害,可没有法子。」楊逍笑道︰「有半個時辰,那也彀了。屬下倒有一計在此,只不知是否管用,要請教主斟酌。雖説是計,説穿了也是不値一笑。范兄弟設法去邀鶴筆翁喝酒,酒中下了教主所調配的藥物,范兄弟先行鬧將起來,説是中了鶴筆翁的十香軟筋散,那時解藥在何人身上,當可査知,乘機便即奪藥救人。」

張無忌道︰「此計是否可行,要瞧那鶴筆翁的性子如何而定,范右使你看怎樣?」范遙將此事從頭至尾擬假思想一遍,覺得這計策雖然簡單,倒也没有破綻,説道︰「我想楊大哥之計可行。鶴筆翁性子狠辣,又不及鹿杖客陰毒多智,只須解藥在鶴筆翁身上,我武功雖不及他,當能對付得了。」楊逍道︰「但若是在鹿杖客身上呢?」范遙皺眉道︰「那便棘手得多。」他站起身來,在山崗旁走來走去,隔了良久,雙手一拍,道︰「只有這樣。那鹿杖客精明過人,若要騙他,多半會被他識破機関,只有抓住了他虧心之事,硬碰硬的威嚇,他權衡輕重,就此屈從也未可知。當然,這樣蠻幹説不定會{\upstsl{砸}}鍋,冒險不小,可是除此之外,似乎别無善策。」

楊逍道︰「這老児有什麼虧心事?他人老心不老,有什麼把柄落在兄弟的手上麼?」范遙道︰「今年春天,汝陽王納妾,邀咱們幾個人在花廳便宴。汝陽王誇耀他新妾美貌,命新娘娘出來敬酒,我見鹿杖客一雙賊眼骨溜溜的亂轉,大爲心動。」韋一笑道︰「後來怎樣?」范遙道︰「後來也没怎樣,那是王爺的愛妾,他便有天大的膽子,也不敢打什麼歹主意。」韋一笑道︰「眼烏珠轉幾轉,可不能説是什麼虧心事啊?」

\chapter{妙盜解藥}

范遙道︰「不是虧心事,可以將它做成虧心事。此事要偏勞韋兄了,你施展高來高去的輕功本領,將汝陽王的愛姬劫來,放在鹿杖客的床上。這老児十之七八,定會按捺不住,胡天胡帝。就算他眞是識得大體,能彀臨崖勒馬,我也會闖進他房去,教他百口莫辯,水洗不得乾淨,只好乖乖的將解藥雙手奉上。」楊逍和韋一笑同時拍手笑道︰「這個栽贓的法児,大是高。憑他鹿杖客奸似鬼,也要鬧個灰頭土臉。」張無忌又是好氣,又是好笑,心想自己所領的這批邪魔外道,行事之奸詐陰毒,和趙明手下那批人物並無什麼不同,只是一者爲善,一者爲惡,這中間就大有區别,以陰毒的法児去對付陰毒之人,可説以毒攻毒。他想到這裡,當下便即釋然,微笑道︰「只可惜累了汝陽王的愛姬。」范遙笑道︰「我早些闖進房去,不讓鹿杖客佔了便宜,也就是了。」

當下四人詳細商議,奪得解藥之後,由范遙送入高塔,分給少林、武當各派高手服下。張無忌和韋一笑則在外接應,一見范遙在萬法寺中放起煙火,便即寺外四處民房放火,群俠便可乘亂逃出。楊逍事先買定馬匹、備就車輛,候在西城門外,群俠出城後分乘車馬,到昌平會合。張無忌對焚燒民房一節,覺得未免累及無辜。楊逍道︰「教主,世事往往難以兩全。咱們救出群俠,日後如能驅走韃子,那是爲天下千萬蒼生造福,今日害得幾百家人家,那也説不得了。」

四人計議已定,分頭入城幹事。楊逍自去騾馬市收買坐騎。張無忌則配了一服麻藥,命韋一笑拿去交給范遙,爲了掩飾藥性,他特地加了三種香料,和在酒中之後,上口時更是醇美馥郁。韋一笑却在市上買了一個大布袋,只等天黑,便去汝陽王府夜劫王姬。

玄冥二老、范遙等爲了看守六大派高手,都就近住在萬法寺中,趙明則仍住王府,只有晩間要學練武藝,纔乘車來寺。范遙回到自己居室,想起二十餘年來明教四分五裂,今日中興有望,也不枉了自己吃了這許多苦頭,心下甚是欣慰。他住在西廂,玄冥二老却住在後院的寶相精舍。他平時爲了忌憚二人精明了得,生恐露出馬脚,極少和他二人接交,因此居室也是離得遠遠地,這時想邀鶴筆翁飲酒,如何不著形跡,倒不是一件易事。眼望後院,只見夕陽西斜,那七歳寶塔下半截已照不到太陽,塔頂玻璃瓦上的日光也漸漸淡了下去。他一時不得主意,負著雙手,慢慢踱步到後院中去,突然之間,鼻中聞到一股肉香。這肉香從寶相精舍對面的一間廂房中透出,那是神箭八雄中孫三毀和李四摧四人所住。范遙心念一動,走到廂房之前,伸手推開房門,那肉香更是撲鼻衝到,只見李四摧蹲在地下,對著一個紅泥火爐不住{\upstsl{搧}}火,火爐上放著一隻大瓦罐,炭火燒得正旺,肉香陣陣從瓦罐中噴出。孫三毀則在擺設碗筷,顯然哥児倆要大快朶頤。

兩人見苦頭陀推門進來,微微一怔,見他神色木然,不禁暗暗叫苦。原來兩人適纔在街上打了一頭大黃狗,割了四條狗腿,悄悄在房中烹煮,那萬法寺是和尚廟,在廟中烹狗而食,實在不妙,旁人見到那也罷了,這苦頭陀却是佛門子弟,莫要惹得他生起氣來,打上一頓,苦頭陀武功甚高,哥児倆萬萬不是他的對手,何況是自己做錯了事,給他打了也是活該。心下正自惴惴,只見苦頭陀走到火爐邉,揭開罐蓋,瞧了一瞧,深深吸一口氣,似乎説︰「好香,好香!」突然間伸手入罐,也不理湯水煮得正滾,撈起一塊狗腿肉,張口便咬,大爵起來,片刻間將一塊狗肉吃得乾乾淨淨,舐唇嗒舌,只覺美味無窮。孫李二人大喜,忙道︰「苦大師請坐,請坐!難得你老人家愛吃狗肉。」

苦頭陀却不就坐,又從瓦罐中抓起一塊狗肉,蹲在火爐邉便大嚼起來。孫三毀要討好他,篩了一碗酒,送到他的面前。苦頭陀端起酒碗,喝了一口,突然都吐在地上,左手在自己鼻子下聞了幾聞,意思説「此酒太劣,難以入口。」大踏步出房,回到自己房中,提了一個大酒葫蘆進來。孫李二人初時見他氣憤憤的出去,又擔心起來,待見他自擕美酒,登時大喜,説道︰「對!對!咱們的酒原非上品,苦大師既有美酒,那是再好不過了。」兩人端凳擺碗,恭請苦頭陀坐在上首,將狗肉滿滿的盛了一盤,放在他面前。要知苦頭陀武功極高,在趙明手下實是第一流的人物,平時神箭八雄是萬萬巴結不上的,今日能請他吃一頓狗肉,説不定他老人家心裡一喜歡,傳授一兩手絶招,那就終身受用不盡了。

苦頭陀拔開葫蘆上的木塞,倒了三碗酒。那酒色作金黃,稠稠的猶如稀蜜一般,一倒出來便是清香撲鼻。李四摧大聲喝采︰「好酒,好酒!」苦頭陀范遙尋思︰「不知玄冥二者在不在家,倘若是外出未歸,這番做作可都白耗了。」他拿起酒碗,放在火爐上的小罐中燙熱,其時狗肉湯煮得正滾,命二人燙熱了再飲。三個人輪流燙酒,那酒香直送出去,鶴筆翁不在廟中便罷,否則便是隔著數進院子,也會聞香趕到。

果然對面寶相精舍板門呀的一聲打開,只聽鶴筆翁叫道︰「好酒,好酒,嘿嘿!」他老實不客氣,跨過天井,推門便進,只見苦頭陀和孫李二人飲酒吃肉,興會淋漓。鶴筆翁一怔,笑道︰「苦大師,你也愛這個調調児啊,想不到咱們倒是同道中人。」孫李二人忙站起身來,説道︰「鶴公公,快請來喝幾碗,這是苦大師的美酒,等閒難以喝到。」鶴筆翁坐在苦頭陀對面,兩人喧賓奪主,大吃大喝起來,將孫李二人倒成了端肉斟酒的厮役一般。四個人吃了半晌,都已有六七分酒意,范遙心想︰「可以下手了。」自己滿滿斟了一杯酒後,順手將葫蘆橫放了。原來張無忌所配的麻藥,范遙拿來輾成粉末,挖空了酒葫蘆的木塞,將藥粉藏在其中,木塞外包了一層布。葫蘆直置之時,藥粉不致落下,四個人喝的都是尋常美酒,這葫蘆一打橫,那酒透過布層,浸顯藥末,一葫蘆的酒都成了毒酒。葫蘆之底本圓,橫放直置,誰也不會留意,何況四人飲了好半天,除了醺醺微酔,十分舒暢之外,更無半點異狀。范遙見鶴筆翁將面前的一碗酒喝乾了,便拔下木塞,將酒葫蘆遞了給他。鶴筆翁自己斟了一碗,順手替孫李兩人都加滿了,見苦頭陀碗中酒滿將溢,便没給他斟。四個人舉碗齊口,骨都骨都的都喝了下去。

除了范遙之外,三個人喝的都是毒酒。孫李二人内力不深,都毒酒一入肚,片刻間便覺手酸脚軟,混身不得勁児。孫三毀低聲道︰「四弟,我肚中有點不對。」李四摧也道︰「我\dash{}我\dash{}像是中了毒。」此時鶴筆翁也覺到了,一運内勁,那口氣竟是提不上來。不由得面色大變。范遙站起身來,滿臉怒氣,一把抓住他的胸口,口中荷荷而呼,只是説不出話。孫三毀驚道︰「苦大師,怎麼啦?」范遙手指醮了點酒,在桌上冩了「十香軟筋散」五個字。

孫李二人均知十香軟筋散是由玄冥二老掌管,眼前情形,確是苦頭陀和哥児倆都中了此藥之毒。兩人相互使個眼色,躬身向鶴筆翁道︰「鶴公公,咱兄弟可没敢冒犯你老人家,請你老人家高抬貴手。」他二人料定鶴筆翁所要對付的只是苦頭陀,他們二人只不過適逢其會,遭受池魚之殃而已。

鶴筆翁詫異萬分,那十香軟筋散這個月由自己掌管,明明是藏在左手所使的一枝鶴嘴筆中,這兩件長刃貼身擕帶,從不離身一步,要説有人從自己身邉偸了毒藥出去,那是決計不能,可是稍一運氣,却是半點使不出力道,確是中了十香軟筋散之毒無疑。其實張無忌所調製的麻藥雖然藥力頗強,比之十香軟筋散却大大不如,服食後所覺異狀也是截然不相同,但鶴筆翁平素只聽慣了十香軟筋散使人眞力渙散,使不出力道的話,到底不曾親自服過這種毒藥,因此兩種藥物雖然差異甚大,他終究無法辨明,這時見苦頭陀又是慌張,又是惱怒,孫李二人更在旁不住口的哀告,那裡還有半點疑惑,説道︰「苦大師不須惱怒,咱們是相好兄弟,在下豈能有加害之意?在下也中了此毒,渾身不得勁児,只不知是何人在暗中搗鬼,當眞奇了。」

苦頭陀又醮酒水,在桌上冩了「快取解藥」四字。鶴筆翁點點頭,道︰「不錯。咱們先服解藥,再去跟那暗中搗鬼的奸賊算帳。解藥在鹿兄身邉,苦大師請和我同去。」苦頭陀心下暗喜,想不到楊逍這計策甚是使得,輕輕易易便將解藥的所在探了出來,他伸左手握住鶴筆翁的右腕,故意裝得脚步蹣跚,跨過院子,一齊走向寶相精舍。鶴筆翁見了他這等支持不住的神態,心中一喜︰「這苦頭陀武功的底子是極高的,只是一直没機會跟咱兄弟倆較量一個高下,瞧他中毒後這等慌張失措,只怕内力是遠遠不如咱們。」

兩人走到精舍門前,靠南一間廂房是鶴筆翁所在,鹿杖客則住在靠北的廂房中,只見北廂房房門牢牢緊閉,不知鹿杖客是否在内。鶴筆翁叫道︰「鹿兄在家嗎?」只聽得鹿杖客在房内應了一聲。鶴筆翁伸手推門,那門却在裡邉閂著。他再道︰「鹿兄,快開門,有要緊事。」鹿杖客道︰「什麼要緊事?我正在練功,你别來打擾成不成?」鶴筆翁的武功和鹿杖客出自一師所授,原是不分軒輊,但鹿杖客一來是師兄居長,二來智謀遠勝,因此鶴筆翁對他向來尊敬,聽他口氣中頗有不悦之意,便不敢再叫。苦頭陀心想這當口不能多所耽擱,倘若那麻藥的藥力消失了,把戲立時拆穿,當下不理三七二十一,右肩在門上一撞,門閂斷折,板門飛開。只聽得一個女子聲音尖聲叫了出來。

鹿杖客正站在床前,一聽門聲,當即回過頭來,一臉孔驚惶和{\upstsl{尷}}尬之色。只見床上橫臥著一個女子。那女子全身被裹在一張薄被之中,只露出了一個頭。那薄被外有繩索綁著,猶如一個舖蓋捲児相似。那女子一頭長髮披在被外,皮膚白膩,容貌極是艷麗,見苦頭陀和鶴筆翁過來,張著圓圓的大眼,顯得十分害怕。苦頭陀認得這正是汝陽王新納的愛姬韓氏,暗道︰「韋蝠王果然好本事,孤身出入王府,將這韓姬手到擒來。」實則汝陽王府雖然警衛森嚴,但衆武士所衛護的也只是王爺、世子、和郡主三人,汝陽王姬妾甚衆,誰也没想到有人會去行刺或是綁架他的一名姬人,何況韋一笑來去如電,機警靈變,一出手便神不知鬼不覺的將韓姬架了來。倒是如何放在鹿杖客房中,反是爲難得多,他候了半日,好容易等到鹿杖客出房如廁,這纔閃身入房,將韓姬放在他的床上,隨即悄然遠去。

鹿杖客回到房中,一眼便見到一個女子橫臥在床,他一縱身便上了屋頂,四下察看有無敵蹤,其時韋一笑早已去得遠了,除了孫李二人的房中傳出陣陣轟飲歡笑之聲,更無他異。鹿杖客情知此事不妙,當下不動色聲的回到房中,一看那個女子,更是嚇得呆了。那日王室納姬,設便宴款待郡主手下十數名有體面的高手,那韓姬敬酒時盈盈一笑,鹿杖客年事雖高,竟是不禁色授魂與。

鹿杖客好色貪淫,一生之中,所摧殘的良家婦女已是不計甚數。那日他見了韓姬的美色,歸來後深自歎息,如何不早日見此麗人,倘若在王爺娶爲姬妾之前落入他的眼中,自是逃不過他的手掌,後來想念了幾次,不久另有新歡,也便將她漸漸淡忘了。不意此刻這韓姬竟會從天而降,在他床上出現,他驚喜交集,略一思索,便猜到定是他大弟子游龍子猜到了爲師的心意,偸偸去將韓姬劫了出來。只見那韓姬被裹在一張薄被之中,頭頸中肌膚勝雪,隱約可見到赤裸的肩膀,似乎身上未穿衣服,他怦然心動,悄聲問她如何能來此,連問數聲,韓姬始終不答,鹿杖客這纔想到,原來她已被人點中了穴道。

正要伸手去解她穴道,突然鶴筆翁等到了門外,跟著房門又被苦頭陀撞開,這一下變生不意,鹿杖客自是狼狽萬分,要待掩隱,已是不及。他心念一動,料定是王爺發覺愛姬被劫,派苦頭陀來捉拿自己,事已至此,只有走爲上著,右手刷的一聲,抽了鹿角杖在手,左臂已將韓姬抱起,便要破窗而去。鶴筆翁驚道︰「鹿師哥,快取解藥來。」鹿杖客道︰「什麼?」鶴筆翁道︰「小弟和苦大師,不知如何竟中了十香軟筋散之毒。」鹿杖客道︰「你説什麼?」鶴筆翁又説了一遍。鹿杖客奇道︰「十香軟筋散不是歸你掌管麼?」鶴筆翁道︰「小弟便是莫名其妙,咱們四個人好端端的喝酒吃肉,突然之間,一齊都中了毒。鹿師哥,快取解藥給咱們服下要緊。」鹿杖客聽到這裡,驚魂始定,將韓姬放回床中,令她臉朝裡床。鶴筆翁素知這位師兄風流成性,在他房中出現女子,那是司空見慣,絲毫不以爲奇,何況鶴筆翁中毒之後驚惶詫異,絲毫没留神去瞧那女子是誰,即在平時,他也未必認得出來,蓋在王爺宴會席上韓姬出來敬酒時一拜即退,鶴筆翁全神貫注的只是喝酒,那去管她這個珠環翠繞的女子是美是醜?

鹿杖客放下韓姬,説道︰「苦大師請到鶴兄弟房中稍息,左下即取解藥過來。」一面説,一面伸手將兩人輕輕推出房去。這一推之下,鶴筆翁身子一晃,險險摔倒。苦頭陀十分機警,也是一個踉蹌,裝作内力全失的模樣,豈料他内力深厚,受到外力時自然而然的生出反應抗禦。鹿杖客一推之下,立時發覺師弟確是内力已失,苦頭陀却是假裝。他深恐有誤,再是用力一推,鶴筆翁和苦頭陀又都向外一跌,但同是一跌,一個下盤虛浮,另一個却是既隱且實。鹿杖客不動聲色,笑道︰「苦大師,當眞得罪了。」一面説,一面伸手去扶,著手之處,却是苦頭陀手腕的「會宗」和「湯池」兩穴。苦頭陀何等機警,一見他如此出手,已是機関敗露,左手一揮,登時使重手法打中了鶴筆翁後心的「魂門穴」,使他三個時辰之内,不論如何救治,都是全身軟癱,動彈不得。兩大高手中去了一個,單打獨鬥,他便不懼鹿杖客一人,當即嘿嘿冷笑,説道︰「你要命不要,連王爺的愛姬也敢偸?」

他這一開口講話,玄冥二老登時驚得呆了,他們和苦頭陀相識已有十五六年,從未聽他説過一言半語,只道他是天生的啞巴。鹿杖客雖已知他不懷好意,却也絶未想到此人居然能彀説話,心想他既如此處心積慮的作偽,則自己處境之險,更無可疑,當下説道︰「原來苦大師並非眞啞,十餘年來苦心相瞞,意欲何爲?」苦頭陀道︰「王爺知你心謀不軌,命我裝作啞巴,就近監視察看。」這句話中其實破綻甚多,但此時韓姬在床,鹿杖客心懷鬼胎,不由得不信,兼之汝陽王對臣下善弄手腕,他也向來知道。苦頭陀此言一出,鹿杖客登時軟了,説道︰「王爺命你來拿我麼?嘿嘿,諒你苦大師武藝雖高,未必能叫我鹿杖客束手就擒。」説著一擺鹿杖,便待動手。

苦頭陀笑了笑,説道︰「鹿先生,苦頭陀的武功就算不及你,也差不了太多。你要打敗我,只怕不是一兩百招之内能彀辦到。你勝我三招兩式不難,但想既挾韓姬,又救師弟,你鹿杖客未必能有這個能耐。」鹿杖客向師弟了瞥了一眼,知道苦頭陀之言倒非虛語。他師兄弟二人自幼同門學藝,從壯到老,數十年中没分離過一天。兩人都無妻子児女,可説是相依爲命,要他撇下師弟,孤身逃走,終究是硬不起這個心腸。苦頭陀見他意動,喝命孫李二人進房,関上房門。説道︰「鹿先生,此事尚未揭破,大可看落在苦頭陀身上,給你遮掩過去。」鹿杖客奇道︰「如何遮掩得了?」苦頭陀頭也不回,反手便點了孫李二人的啞穴和軟麻穴,手法之快,認穴之準,鹿杖客也是暗暗嘆服。只聽苦頭陀道︰「你自己是不會宣揚的了,令師弟想來也不致故意跟你爲難,苦頭陀是啞巴,以後仍是啞巴,不會説話。這兩位兄弟呢,苦頭陀替你點上他們死穴滅口,也不打緊。」

孫李二人大驚失色,心想此事跟自己半點也不相干,那想到吃狗肉竟吃出這等飛來橫禍,要想出言哀求,却苦於開不得口。苦頭陀指著韓姬道︰「至於這位姬人呢,老衲倒有兩個法児。第一個方法乾手淨脚,將她和孫李二人一併帶到冷僻之處,一刀殺了,報知王爺,説她和李四摧這小白臉戀奸情熱,私奔出走,被苦頭陀見到,惱怒之下,將奸夫淫婦當場格殺却,還饒上孫三毀一條性命。第二條路是由你將她帶走,好好隱藏,以後是否洩露機密,瞧你自己的本事。」鹿杖客不禁轉頭,向韓姬瞧了眼。只見她眼光之中,滿是求懇,顯要他接納第二個法児,鹿杖客見到她這等麗質天生,心想倘若一刀殺了,豈非可惜,不由得心中大動,説道︰「多謝你爲我設身處地,想得這般週到。你却要我爲你幹什麼事?」他明知苦頭陀必有所求,否則決不能如此善罷。

苦頭陀道︰「此事容易之至。峨嵋派掌門滅絶師太和我交情很深,那個姓周的年輕姑娘,是我跟老尼姑生的私生女児。求你賜予解藥,好救這兩人出去,郡主面前,由老衲一力承當,倘若牽連於你,教苦頭陀和滅絶老尼一家男盜女娼,死於非命,永世不得超生。」

原來苦頭陀深知鹿杖客生性風流,若從男女之事上頭著手,易於取信,他聽楊逍説起明教許多兄弟喪命於滅絶師太的劍下,因此捏造一段和尚尼姑的謊話。要知范遙此人邪性未脱,説話行事,決不依正人君子的常道,至於罰下「男盜女娼」的重誓云云,更不在他的意下。

鹿杖客聽了這幾句話,一怔之下,隨即微笑,心想你這頭陀幹這等事來脅迫於我,原來是爲救你的老情人和親生女児,那倒也是人情之常,此事雖然擔些風險,但換到個絶色佳人,確也値得。也見苦頭陀有求於己,登時便放寬了,笑道︰「那麼將王爺的愛姬劫到此處,也是出於苦大師的手筆了?」苦頭陀道︰「投我以解藥,報之以韓姬,匪報也,永以爲好也。」鹿杖客大喜,只是深恐室外有人,不敢縱聲大笑,突然間一轉念,又問︰「然則我師弟何以會中十香軟筋散之毒?這毒藥你從何處得來?」苦頭陀道︰「那還不容易?這毒藥由令師弟看管,他是好酒貪杯之人,飲到酣處,苦頭陀難道會偸他不到手麼?」

鹿杖客再無疑惑,説道︰「好!苦大師,兄弟結交了你這個朋友,我決不賣你,盼你别再令我上這種惡當。」苦頭陀指著韓姬笑道︰「下次如再有這種香艷的惡當,請鹿杖先生也安排個圏套,給苦頭陀鑽鑽,老衲欣然領受。」兩人相對一笑,心中却各自想著别的主意。鹿杖客在暗暗盤算,如何安置好韓姬之後,要出其不意的弄死這個惡頭陀。

苦頭陀心知鹿杖客雖是暫受自己脅迫,但玄冥二老是何等的身份,吃了這個大虧豈肯就此罷休,只要他一安頓韓姬,解開鶴筆翁的穴道,立時便會找自己動手,但那時六派高手已經救出,自己早拍拍屁股走路了。范遙見鹿杖客遲遲不將解藥取出,心想我若催他,他反爲刁難,於是慢吞吞的坐了下來,説道︰「鹿兄何不解開韓姬的穴道,大家一起來喝幾杯?燈下看來人,這等艷福幾生才修得到啊!」鹿杖客情知這萬法寺中人來人往,韓姬在此多耽一刻,便多一分危險,當下取過鹿角杖,旋下了其中一根鹿角,取過一隻杯子,倒了些粉末在杯,説道︰「苦大師,你神機妙算,兄弟甘拜下風,解藥在此,便請取去。」苦頭陀搖頭道︰「這麼一點児藥末,管得什麼用。」鹿杖客道︰「别説要救兩人,便是六七個人也足彀了。」苦頭陀道︰「你何必小氣,便多賜一些又何妨?老實説,閣下足智多謀,老頭陀深怕上了你的當。」鹿杖客見他多要解藥,突然心中起疑,説道︰「苦大師,你要相救的,莫非不單是滅絶師太和令愛兩人?」

苦頭陀正要飾詞解釋,忽聽得院子中脚步聲響,有七八人奔了進來,只聽一人説道︰「脚印到了此處,難道韓姬竟到了萬法寺中?」鹿杖客臉上變色,一把將盛著解藥的杯子揣在懷中,只道苦頭陀在外伏下人手,一等取到解藥,便即出賣自己。苦頭陀搖了搖首,叫他且莫驚慌,取過一條單被,罩在韓姬身上,連頭蒙住,又放下帳子,只聽得院子中一人説道︰「鹿先生在家麼?」苦頭陀指了指自己的嘴巴,意思説自己是啞子,叫鹿杖客出聲答應。鹿杖客朗聲道︰「什麼事?」那人道︰「王府裡有一位姬人被歹徒所劫,瞧那歹徒的足印,却是到萬法寺來。」鹿杖客向苦頭陀怒視一眼,意思是説︰若非你故意栽贓,依你的身手,豈能留下足跡?苦頭陀裂嘴一笑,做個手勢,叫他打發那人,心中却想︰「韋蝠王栽贓栽得十分到家,把足印從王府引到了這裡。」

鹿杖客冷笑道︰「你們還不分頭去找,在這裡嚷嚷的幹什麼?」他武功地位,人人對之極是忌憚,那人唯唯答應,不敢再説什麼,立時分派人手,在附近搜査。鹿杖客知道這一來,萬法寺四下都有人嚴加追索,雖然料想他們還不敢査到自己房裡來,但要帶韓姬出去藏在别處,却是無法辦到了,不由得皺起眉頭,狠狠的瞪著苦頭陀。范遙心念一動,低聲道︰「鹿兄,萬法寺中有個好去處,大可暫且收藏你這位愛寵,過得一天半日,外面査得鬆了,再帶出去不遲。」鹿杖客怒道︰「除非藏在你的房裡。」范遙笑道︰「這等美人藏在我的房中,老頭陀未必不動心,鹿兄不呷醋麼?」鹿杖客問道︰「那麼你説是什麼地方?」范遙一指窗外的塔尖,微微一笑。

鹿杖客聰明機警,一點便透,大拇指一翹,説道︰「好主意!」要知那寶塔是監禁六大派高手的所在,看守的總管,便是鹿杖客的大弟子游龍子。旁人什麼地方都可疑心,決不會疑心王爺愛姬竟會劫到最是戒備森嚴的重獄之中。苦頭陀低聲道︰「此刻院子中没人,事不宜遲,立即動身。」將床上那單被四角提起,便將韓姬裹在其中,成爲一個大包袱,右手提著,交給鹿杖客。

鹿杖客心想你别要又讓我上當,我肯負韓姬出去,你聲張起來,那時人贓並獲,還有什麼可説的,不禁臉色微變,竟不伸手去接。苦頭陀知道他的心意,説道︰「爲人做到底,送佛送上天,苦頭陀再替你做一次護花使者,又有何妨?誰叫我有事求你呢。」説著負起包袱,推門而出,低聲道︰「你先走把風,有人阻攔査問,殺了便是。」

鹿杖客斜身閃出,却不將背脊對正苦頭陀,生怕他在後偸襲。苦頭陀反手掩上了門,佝僂著身子,負了韓姬,逕往寶塔。此時已是戍末,除了塔外的守衛武士,再無旁人走動。那些武士見到鹿杖客,一齊躬身行禮,恭恭敬敬的站在一旁。未到塔前,游龍子得屬下報知,遠遠已迎了出來,説道︰「師父,你老人家今日興緻好,到塔上坐坐麼?」鹿杖客點了點頭,和苦頭陀正要邁步進塔去,忽然塔門開處,走出一個人來,却是趙明。鹿杖客作賊心虛,大吃一驚,没料到郡主竟在塔内,三人一齊上前參見。趙明向游龍子笑道︰「你師父眞收得個好徒児,只管去迎接師父,就不顧得來接我了。」游龍子躬身道︰「小人不知郡主駕到,請恕失迎之罪。」趙明笑道︰「你安排得很是週到,明教想來救人,只怕没麼容易。」原來昨晩張無忌這麼一鬧,趙明却不知明教只來了三人,只怕他們大舉來襲,因此親到寶塔上巡視一周。見塔上戒備周密,每一層均有兩位高手把守,很是放心。她向苦頭陀微微一笑,説道︰「苦大師,我正在找你。」苦頭陀點了點頭。絲毫不動聲色。趙明道︰「請你陪我到一個地方去一下。」苦頭陀心中暗暗叫苦︰「好容易將鹿杖客騙進了寶塔,只待下手奪到他的解藥,大功便即告成,那知這小丫頭却在這時候來叫我。」要想找什麼藉口不去,倉卒之間無善策,何況他是假啞巴,想要推托,苦於無法説話,情急智生,心想︰「且由這鹿杖客去想法子。」當下提起手中包袱,向鹿杖客晃了一晃。

鹿杖客大吃一驚,肚裡暗罵苦頭陀害人不淺。趙明道︰「鹿先生,苦大師這包裹袋著什麼?」鹿杖客道︰「{\upstsl{嗯}},{\upstsl{嗯}},是苦大師的舖蓋。」趙明奇道︰「舖蓋?苦大師揹著舖蓋幹什麼?」她{\upstsl{噗}}{\upstsl{哧}}一笑,説道︰「苦大師嫌我太蠢,不肯收這個弟子,自己捲舖蓋不幹了麼?」苦頭陀搖了搖頭,右手伸起來亂打了幾個手勢,心想︰「一切由鹿杖客去想法子撒謊,我做啞巴自有做啞巴的好處。」趙明看不懂他的手勢,只有眼望鹿杖客,等他解釋。

鹿杖客靈機一動,已有了主意,説道︰「是這樣的,昨晩魔教的幾個魔頭這麼來一鬧,屬下生怕他們其志不小\dash{}這個,説不定要到高塔中來救人。因此屬下和苦大師決定住到高塔中來,親自把守,以免誤了郡主的大事。這舖蓋是苦大師的棉被。」趙明大悦,笑道︰「我原想請鹿先生和鶴先生來親自鎭守,只是覺得過於勞動大駕,不好意思開口。難得鹿鶴兩位肯分我之憂,那是再好没有了。苦大師,有鹿先生在這裡把守,諒那些魔頭也討不了好去,你跟我去吧。」説著伸手握住了苦頭陀的手掌。苦頭陀無可奈何,心想此刻若是揭破鹿杖客的瘡疤,一來於事無補,二來韓姬明明負在自己背上,未必能使趙明相信,只得將那個大包袱交了給鹿杖客。鹿杖客伸手接過,道︰「苦大師,我在塔上等你。」游龍子道︰「師傅,讓弟子來拿舖蓋吧。」鹿杖客笑道︰「不用!是苦大師的東西,爲師的要討好他,親自給他揹舖蓋捲児。」苦頭陀心中暗罵,伸手在包袱外一拍,正好打在韓姬的屁股上,好在她已被點中了穴道,這一聲驚呼没能叫出聲來,但鹿杖客已是嚇得臉上變色,不敢再多逗留,向趙明一躬身,便即負了韓姬入塔。他心早已打定主意,一進塔内,立時便將一條棉被換入包袱之中,倘若苦頭陀向趙明告密,他便來個死不認帳。苦頭陀被趙明牽著手,一直走出萬法寺,心中又是焦急,又是奇怪,不知她要帶自己到那裡去。趙明拉上斗蓬上的風帽,罩住了一頭秀髮,悄聲道︰「苦大師,咱們去瞧張無忌那小子去。」

\chapter{勇救各派}

苦頭陀又是一驚,斜眼看她,只見她眼波流轉,粉頰暈紅,却是七分嬌羞,三分喜悦,決不是識破了他機関的模樣。苦頭陀心中大安,回憶昨晩在萬法寺中她和張無忌相見的情景,那裡是兩個生死冤家的樣子。他一想到「冤家」兩字,突然心念一動︰「冤家?莫非郡主對我教主暗中已生情意?」轉念再想︰「她爲什麼要我跟去,却不叫她更親信的玄冥二老?是了,只因我是啞巴,不會洩漏她的祕密。」當下點了點頭,古古怪怪的一笑。趙明嗔道︰「你笑什麼?」苦頭陀心想這個玩笑不能開,於是指手劃脚的做了幾個手勢,意思説苦頭陀自當盡力維護郡主周全,便是龍潭虎穴,也和郡主同去一闖。

趙明不再多説,當先引路,不久便到了張無忌留宿的客店門外。苦頭陀暗暗驚訝︰「郡主也眞神通廣大,立時便査到了教主駐足的所在。」隨著趙明走進客店。趙明問掌櫃的道︰「咱們找姓曾的客官。」原來張無忌住店之時,又用了「曾阿牛」的假名。店小二進去通報。張無忌正在床上打坐養神,只待萬法寺中煙花射起,便去接應,忽聽有人來訪,甚是奇怪,迎到客堂,一見訪客竟是趙明和苦頭陀,心中一動,暗叫︰「不好,定是趙姑娘揭破了苦大師的身份,特此來跟我理論。」只得上前一揖,説道︰「不知趙姑娘駕到,有失迎迓。」趙明道︰「此處非説話之所,咱們到那邉的小酒家去小酌三杯如何?」張無忌道︰「甚好。」

趙明仍是尚先引路,離那客店五間舖面,便是一家小小的酒家,内堂疏疏的擺著幾張板桌,桌上插著一筒長長的木筷。此刻天時已晩,店中一個客人也無。趙明和張無忌相對而坐,苦頭陀識趣,打個手勢,説自己到外面喝酒。趙明點了點頭,叫店二拿一隻鍋子,切三斤生羊肉,打兩斤白酒。張無忌滿腹疑團,心想她是金枝玉葉的郡主之尊,却和自己到這家汚穢的小酒家來吃涮羊肉,不知安什排著什麼詭計。趙明斟了兩杯酒,拿過無忌的酒杯,喝了一口,笑道︰「這酒裡没安毒藥,你儘管放心飲用便是。」張無忌道︰「姑娘召我來此,不知有何見教?」趙明道︰「喝酒三杯,再説正事。我先乾爲敬。」説著舉杯一飲而盡。無忌拿起酒杯,燈光下只見杯邉留著淡淡的胭脂唇印,鼻中聞到一陣清幽的香氣,也不知這香氣是從杯上的唇印而來,還是從她身上而來,不禁心中一蕩,便把一杯酒喝了。趙明道︰「再喝兩杯,我知道你對我終是不放心,每一杯我都先{\upstsl{嚐}}一口。」

無忌知她詭計多端,確是事事提防,難得她肯先行{\upstsl{嚐}}酒,免了自己多冒一層危險,可是接連喝了三杯她飲過的殘酒,心神不禁有些異樣,一抬頭,只見趙明淺笑盈盈,酒氣將她粉頰一蒸,更是嬌艷萬狀,不可方物。無忌那敢多看,忙將頭轉了開去。趙明低聲道︰「張公子,你可知道我是誰?」無忌搖了搖頭。趙明道︰「我今日跟你説了,我爹爹便是當朝執掌兵馬大權的汝陽王。我是蒙古女子,眞名叫作明明特穆爾,『趙明』兩字,乃是我自己取的漢名,皇上封我爲昭明郡主。」倘若不是苦頭陀早晨已經説過,張無忌此刻原不免大吃一驚,但聽她居然將自己身份毫不隱瞞的相告,亦頗出意料之外,只是他不善作偽,並不假裝大爲驚訝之色。趙明奇道︰「怎麼?你早知道了?」無忌道︰「不,我怎會知道。不過我見你以一個年輕姑娘,却能叫這許多武林高手聽你號令,身份自是非同尋常。」趙明撫弄著酒杯,半晌不語,提起酒壼又替兩人斟了酒,緩緩説道︰「張公子,我問你一句話,請你從實告我。要是我將你那位周姑娘殺了,你待怎樣?」

張無忌奇道︰「周姑娘又没有得罪於你,好端端的如何要殺她?」趙明道︰「有些人我不喜歡,我便殺了,難道一定要是得罪了我,我纔殺她?有些人不斷得罪我,我却偏偏不殺,比如是你,你得罪我還不彀多麼?」她説到這裡,眼光中孕著的全是笑意。無忌嘆了口氣,道︰「趙姑娘,我得罪你,實在迫於無奈。不過你贈藥救了我的三師伯、六師伯,我總是很感激你。」趙明笑道︰「你這人當有三分傻氣。兪岱岩和殷利亨之傷,都是我部屬所下的手,你不怪我,反來謝我?」無忌笑道︰「我三師伯受傷已二十年,那時候你都未出世呢。」趙明道︰「這些人是我爹爹的部屬,也就是我的部屬,那有什麼分别?你别將話岔開去,我問你︰要是我殺你的周姑娘,你對我怎樣?是不是要殺了我替她報仇?」

張無忌沉吟半晌,説道︰「我不知道。」趙明道︰「怎麼會不知道?你不肯説,是不是?」無忌道︰「爹爹媽媽是給人逼死的。那日我在武當山上,我向著爹爹媽媽的屍體立誓,日後我長大成人,定要替他們復仇。我把少林派、峨嵋派、崆峒派這些人的面貌,牢牢記在心中。當時我年紀小,心裡充滿了仇恨。可是後來年紀大了,事情懂得多了,我仇恨之心一點點的淡了下來。我實在不清楚,到底是誰害死了我的爹爹媽媽。不應該説是空智大師、鐵琴先生這些人,也不應該説是我的外公舅父,甚至於,也不該是你手下的阿大、阿二、玄冥二老之類的人物。趙姑娘,我這幾天心裡只是想,倘若大家不殺人,和和氣氣、親親愛愛的都做朋友,豈不是好?」這一番話,他在心頭已想了很久,可是没有對楊逍説,没有對張三丰説,也没有對殷利亨説,突然在這小酒家中對趙明説了出來,這番言語一出口,自己覺得有些奇怪。

趙明聽他説得誠懇,想了一想道︰「那是你心地仁厚,倘若是我,那可辦不到。要是誰害死了我的爹爹哥哥,我不但殺他滿門,連他親戚朋友,凡是他所相識的人,我個個要殺得乾乾淨淨。」張無忌︰「那我一定要阻攔你。」趙明道︰「爲什麼?你幫助我的仇人麼?」無忌道︰「我想你殺一個人,對你自己便多一份害處,多一分罪業。趙姑娘,你殺過人没有?」趙明笑道︰「現下還没有,將來我年紀大了,要殺很多人。我的祖先成吉斯汗大帝,是拖雷、拔都、忽必烈這些英雄。我只恨自己是女子,要是男人啊,嘿嘿,可眞要轟轟烈烈幹一番大事業呢。」她斟一杯酒,自己喝了,笑道︰「你還是没回答我的話。」

張無忌道︰「你要是殺了周姑娘,殺了我手下任何一個親近的兄弟,我便不再當你是朋友,我永遠不跟你見面,便見了面也永不説話。」趙明笑道︰「那你現下當我是朋友麼?」無忌道︰「假如我心中恨你,也不跟你在一塊児喝酒了。唉!我只覺得要恨一個人眞難。我生平最恨的是那個混元霹靂掌成崑,可是他現下死了,我又有些可憐他,似乎倒盼望别死似的。」趙明道︰「要是我明天死了,你心裡怎樣想?你心中一定説︰謝天謝地,我這個刁鑽兇惡的大對頭死了,免了我多少煩惱。」

無忌大聲道︰「不,不!我不盼望你死,一點也不。韋蝠王這樣威嚇你,要在你臉上劃幾條刀痕,我後來想想,很是擔心。趙姑娘,你要不再跟咱們爲難了,把六大派高手都放了出來,大家快快活活的做朋友,豈不是好?」趙明喜道︰「好啊,我本來就盼望這樣。你是明教教主,一言九鼎,你去跟他們説,要大家歸順朝廷。待我爹爹奏明皇上,每個人都有封賞。」張無忌緩緩搖頭︰「我們漢人大家都有個心願,要你們蒙古人退出漢人的地方。」

趙明霍地站起身來,説道︰「怎麼?你竟説這種犯上作亂的言語,那不是公然發叛麼?」張無忌道︰「我本來就是反叛,難道你到此刻方知?」趙明向他凝望良久,臉上的憤怒和驚詫慢慢消退,顯得又是溫柔,又是失望,終於又坐了下來,説道︰「我早就知道了,不過要聽你親口説了,我纔相信那是千眞萬確,無可挽回。」説到這裡聲調中竟是十分的淒苦和傷心。張無忌的心腸本軟,這時更加抵受不住她如此的難過,幾乎便欲衝口而出︰「我聽你的話便是。」但這念頭一瞬即逝,立即把持住心神,可是也想不出什麼話來安慰她。

兩人默默對坐了好一會,無忌道︰「趙姑娘,夜色已深,我送你回去吧。」趙明道︰「你連陪我多坐一會児也不願麼?」無忌忙道︰「不!你愛在這裡飲酒説話,我便陪你。」趙明微微一笑,道︰「有時候我獨個想,倘若我不是蒙古人,又不是什麼郡主,只不過是像周姑娘那樣,是個平常的漢人姑娘,那你或許會對我好些。張公子,你説是我美呢,還是周姑娘美?」無忌没料到她竟會問到這一句話,究竟是番邦女子性情直率,口没遮欄,燈光掩映之下,但見她嬌美無限,不禁脱口而出︰「自然是你美。」趙明伸出右手,按在他的手背之上,眼光中全是喜色,道︰「張公子,你喜不喜歡常常見我,倘若我時時邀你到這児來喝酒,你來不來?」無忌的手背碰到她柔滑的手掌心,一顆心怦怦而動,定了定神,纔道︰「我在這児不能多耽,過不幾天,便要南下。」趙明道︰「你到南方去幹什麼?」無忌嘆了口氣,道︰「我不説你也猜得到,若是説了出來,我惹得你生氣\dash{}」

趙明眼望窗外的一輪皓月,忽道︰「你答應過我,要給我做三件事,總没忘了吧?」無忌道︰「自然没忘。便請姑娘即行示下,我盡力去做。」趙明轉過頭來,直視著他的臉,説道︰「現下我只想到了第一件事。我要你伴我去取那柄屠龍刀。」

張無忌原也猜到她要自己所做的這件事,定然極不好辦,却萬萬没想到第一件事便是個天大的難題。趙明見他大有難色,問道︰「怎麼?你不肯麼?這件事可並不違背俠義之道。」無忌心想︰「屠龍刀在我義父手上,此事江湖上衆所週知,那也不用瞞她。」便道︰「屠龍刀是我義父金毛獅王謝大俠之物,我豈能背叛義父,取刀給你。」趙明道︰「我不要你去偸去搶、去拐去騙。我也不是眞的要了這把刀。我只要你去向你義父借來,給我把玩一個時辰,立刻便還給謝大俠。你們是義父義子之情,難道向他借一個時辰,他也不肯?借一把刀瞧瞧,又不是吞没他的,又不是用來謀財害命,也是違背俠義之道了?」無忌道︰「這把刀雖然名聞武林,其實也没什麼看頭,只不過是特别沉重些,鋒利些而已。」趙明道︰「説什麼『武林至尊,寶刀屠龍,號令天下,莫敢不從。倚天不出,誰與爭鋒?』倚天劍是在我手中,我定要瞧瞧那屠龍刀是什麼模樣。你若是不放心,我看刀之時,你儘可站在一旁。憑著你的本領,我決不能強佔不還。」

張無忌尋思︰「救出了六大派高手之後,我本是要立即動身去迎歸義父,請他老人家擔任教主大位。趙姑娘言明借刀看一個時辰,雖然難保她不有什麼詭計,可是我全神提防,諒她也不能將刀奪了去。只是義父曾説,屠龍刀之中,藏著一件武功絶學的大祕密。以義父的聰明才智,雙眼未盲之時已得寶刀,始終參詳不出,這趙姑娘在短短一個時辰之中,豈能有何作爲?何況我和義父一别十年,説不定他在孤島之上,已參透了寶刀中的祕密。」

趙明見他沉吟不答,笑道︰「你不肯,那也由得,我可要另外叫你做一件事,那却難得多了。」張無忌知道這女又刁又毒,倘若另外出個難題,自己決計辦不了,忙道︰「好,我答應去給你借屠龍刀,但咱們言明在先,你只能借看一個時辰,倘若意圖強佔,我可決不干休。」趙明笑道︰「是了。我又不會使刀,重甸甸的要來幹麼?你便恭恭敬敬的送給我,我也不希罕呢。你什麼時候動身去取?」無忌道︰「這幾天就去。」趙明道︰「那再好也没有了。我去收拾收拾,你什麼時候動身,來約我便是。」無忌又是一驚,道︰「你也同去?」趙明道︰「當然啦。聽説你義父是在海外孤島之上,要是他不肯歸來,難道要你萬里迢迢的借了刀來,給我瞧上一個時辰,再萬里迢迢的送去,又萬里迢迢的歸來?天下也没有這個道理。」

張無忌想起北海中波濤的險惡,茫茫大洋之中,自己能否找得到冰火島,已是十分渺茫,若要來來去去的走上三次不出岔子,那可是半點把握也没有,這位姑娘説得不錯,義父在冰火島上一住二十年,未必肯以垂暮之年,重歸中土,便道︰「大海中風波無情,你何必親自去冒這個險?」趙明道︰「你冒得險,我爲什麼便不成?」無忌躊躇道︰「你爹爹肯放你去嗎?」趙明道︰「爹爹叫我統率江湖群豪,這幾年來我往東到西,爹爹從來就没管我。」無忌聽到「爹爹叫我統率江湖群豪」這句話,心中一動︰「我到冰火島去迎接義父,不知何年何月方歸。倘若那是她的調虎離山之計,乘我不在,便大舉對付本教,倒是不可不妨,若是和她同住,她手下人有所顧忌,便可免了我的後顧之憂。」於是點頭道︰「好,我出發之時,便來約你\dash{}」

一句話没説完,突然間窗外紅光一亮,跟著喧嘩之聲大作,從不遠處傳了過來。趙明走到窗邉一望,驚道︰「啊喲,萬法寺的寶塔起火!苦大師,苦大師,快來。」連叫數聲,苦頭陀竟不現身,她走到外堂,不見苦頭陀的蹤影,問那掌櫃時,却説那位大師一到便走,從没停留,早已就去了兩個時辰。趙明大是詫異,却還没想到苦頭陀竟會背叛自己。張無忌見火頭越燒越旺,深怕大師伯等功力尚未恢復,竟被燒死在高塔之中,説道︰「趙姑娘,少陪了!」一語甫畢,已是穿窗而出。趙明叫道︰「且慢!我和你同去。」待他竄出窗子,張無忌已絶塵而去。

且説鹿杖客見苦頭陀被郡主叫去,心中大定,當即負著韓姬,來到游龍子室中。游龍子是高塔的總管,居於最高的第七層中央,便於眺望四周,控制全局。鹿杖客進房後,對游龍子道︰「你在門外瞧著,别放人進來。」游龍子一出門,他當即掩上房門,解開包袱,放了韓姬出來。只見她駭得花容黯淡,眼光中滿是哀懇之色,鹿杖客悄聲道︰「你到了這裡,那便不用害怕,我自會好好待你。」眼下還不能解開她的穴,以防她聲張出來壞事,於是將她放在游龍子床上,拉過被來蓋在她身上,另取一條棉被,裹在包中,放在一旁。鹿杖客此人極工心計,知道韓姬所在之處,便是是非之地,不敢多所逗留,匆匆出房,囑咐游龍子不可進房,也不可放别人進去,他知這個大弟子對己既敬且畏,決不敢稍有違背。

他心中略加盤算︰「此事若要苦頭陀守住祕密,非賣他一個人情不可,只得先去放了他的老情人和女児。恰好昨晩魔教的教主這麼一鬧,事情正是那姓周姑娘身上而起,只須説是那教主將滅絶老尼和周姑娘救去,當眞是天衣無縫,郡主再也没半點疑心。這小魔頭武功如此高強,郡主也不能怪咱們失察之罪。」

峨嵋派一干女弟子都囚在第四層上,滅絶師太因是掌門之尊,單獨囚在一間小屋中。鹿杖客命看守者入門,只見滅絶師太盤膝坐在地下,閉目靜修。她已絶食數日,容顏雖然憔悴,但反而更顯桀傲強悍。鹿杖客説道︰「滅絶師太,你好!」滅絶師太緩緩睜開眼來,道︰「在這裡便是不好,有什麼好?」鹿杖客道︰「你如此倔強,主人説留著也是無用,命我來送你歸天。」滅絶師太死志早決,説道︰「好極,只是不勞閣下動手,請借一柄短劍,由我自己了斷便是。還請閣下叫我徒児周芷若來,我有幾句話屬咐於她。」鹿杖客轉身出房,命人帶周芷若,心想︰「她母女之情,果然與衆不同,否則爲甚麼不叫别的大徒児,單單叫她。」

不久周芷若來到師父房中,滅絶師太道︰「鹿先生,請你在房外稍候,我只説幾句話便成。」周芷若待鹿杖客出房,反手掩上了門,撲在師父懷裡,嗚咽出聲。滅絶師太一生心腸雖硬,當此死别之際,却也不禁傷感,輕輕撫摸她的頭髮。周芷若知道跟師父説話的機會無多,便即將昨晩張無忌前來相救之事説了。滅絶師太皺起眉頭,沉吟不晌,道︰「他爲什麼單是救你,不救旁人?那日你在光明頂上刺他一劍,爲什麼他反來救你?」周芷若紅暈雙頰,輕聲道︰「我不知道。」滅絶師太怒道︰「哼,這小子太過陰險。他是魔教的大魔頭,能有什麼好心。他是安排下圏套,要你乖乖的上鉤。」周芷若奇道︰「他\dash{}他安排下圏套?」滅絶師太道︰「咱們是魔教的死對頭,在我倚天劍下,不知殺了多少魔教的邪惡奸徒,魔教自是恨峨嵋派入骨,焉有反來出手相救之理?這姓張的魔頭定然看上了你,要你墮入他的殼中。他叫你將咱們擒來,然後故意賣好,將你救了出去。」

周芷若柔聲道︰「師父,我瞧他\dash{}他倒不是假意。」滅絶師太大怒,喝道︰「你定是和你那個不成器的紀曉芙一般,瞧中了魔教的淫徒。倘若我功力尚在,一掌便劈死了你。」周芷若嚇得全身發抖,説道︰「徒児不敢。」滅絶師太道︰「你眞的不敢,還是花言巧語,欺騙師父?」周芷若垂泪道︰「徒児決不敢有違恩師的教訓。」滅絶師太道︰「你跪在地下,罰個重誓。」周芷若依言跪下,不知怎樣説纔好。滅絶師太道︰「你這樣説︰小女周芷若對天盟誓,日後我若對魔教教主張無忌這淫徒心存愛慕,若是和他結成夫婦,我親身父母死在地下,屍骨不得安穩;我師父滅絶師太必成厲鬼,令我一生日夜不安;我若和他生下児女,男子代代爲奴,女子世世爲娼。」

周芷若大吃一驚,她天性柔和溫順,從没想到所發的誓言之中,竟能會如此毒辣,不但詛咒死去的父母,也詛咒到没出世的児女,但見師父兩道如電一般的目光,狠狠{\upstsl{盯}}在自己臉上,不由得目眩頭暈,便依著師父所説,照樣念了一遍。

滅絶師太聽她罰了這個毒誓,容色便霽,溫言道︰「好了,你起來吧。」周芷若已是哭得泪人一般,委委屈屈的站起身來。滅絶師太臉一沉,道︰「芷若,我不是故意逼你,這全爲你好。你一個年紀輕輕女孩子,以後師父不能再照看你,倘若你重𨂻你紀師姊的覆轍,師父身在九泉之下,也不得安心。何況師父要你負起興復本派的重任,更是半點大意不得。」説著除下左手食指上的鐵指環,站起身來,説道︰「峨嵋派女弟子周芷若跪下聽諭。」周芷若一怔,當即跪下。滅絶師太將鐵指環高舉過頂,説道︰「峨嵋派第三代掌門女尼滅絶,謹以本門之位,傳於第四代女弟子周芷若。」

周芷若被師父逼著發了那個毒誓之後,頭腦中已是一片混亂,突然聽到要自己接任本派的掌門,更是茫然失措,驚得呆了。滅絶師太一個字一個字的緩緩説道︰「周芷若,奉接本門掌門鐵指環,伸出左手。」周芷若恍恍惚惚的依言舉起左手,滅絶師太便將鐵指環套在她的食指之上。周芷若顫聲道︰「師父,弟子年輕,入門未久,如何能當此重任?你老人家必能脱困,别這麼説!弟子實在不能\dash{}」説到這裡,抱著師父雙腿,哭出聲來。

鹿杖客在外面早已等得很不耐煩,聽到獸聲,打門道︰「喂,你們話説完了嗎?以後説話的日子長著呢。」滅絶師太喝道︰「你囉唆什麼?」對周芷若道︰「師尊之命,你也敢違背麼?」當下將本門掌門人的戒規申述一遍,要她記在心中。周芷若見師父言語之中,儼然囑咐後事的神態,更是驚懼,説道︰「弟子做不來,弟子不能\dash{}」滅絶師太厲聲道︰「你不聽我言,便是欺師滅祖之人。」她見周芷若苦苦楚楚可憐,想到自己即將大去,要這個性格柔順的弱女子挑這副如此沉重的擔子,只怕她當眞不堪負荷,不過峨嵋派群弟子之中,只有她悟性最高,要修習最高深武功,光大本門,除她之外,更無第二個弟子合適,想到此後長長的日子之中,這小弟子勢必經歷無數艱辛危難,不禁心中一酸,將她扶了起來,摟在懷裡?柔聲説道︰「芷若,我所以叫你做掌門,不傳給你的衆位師姊,那也不是我偏心,只因峨嵋派掌門人必須武功卓絶,始能與别派較一日之短長。」周芷若道︰「弟子的武功那能及得上衆位師姊?」滅絶師太微微一笑,道︰「她們成就有限,到了現下的境界,很難再有多大進展,那是天資所関,非人力所能強求。你此刻雖然不及衆位師姊,日後却是不可限量。{\upstsl{嗯}},不可限量,不可限量,那便是這四個字。」周芷若神色迷茫,瞧著師父,不知其意何在。

滅絶師太將口唇附在她的耳邉,低聲道︰「你已是本門掌門,得將本門的一件大祕密説與你知。本派的創派祖師郭女俠,乃是當年大俠郭靖的小女児。郭大俠在元兵攻破襄陽之時,惡戰殉難,他臨死時曾將一個祕密,説與本派祖師郭女俠知悉。郭大俠當年名震天下,生平有兩項絶藝,其一是行軍打仗的兵法,其二便是武功。郭大俠的夫人黃蓉女俠,最是聰明機智,她眼見元兵勢大,襄陽終不可守,他夫婦二人決意以死報國,那是知其不可爲而爲之的赤心精忠,但郭大俠的絶藝就此失傳,豈不可惜?何況她料想蒙古人縱然一時佔得了中國,我漢人終究不干爲韃子奴隸,日後中原血戰,那兵法和武功兩項,將有極大的用處。因此她聘得高手匠人,將楊過楊大俠的一柄玄鐵重劍,再加以西方精金,鑄成了一柄屠龍刀,一柄倚天劍。」周芷若對屠龍刀和倚天劍之名,習聞已久,却從來不知這一對刀劍,竟是本派祖師郭襄女俠的母親所鑄。

滅絶師太又道︰「黃女俠在鑄刀鑄劍之前,和郭大俠兩人窮一月心力,繕冩了兵法和武功的精要,分别藏在刀劍之中。屠龍刀中藏的乃是兵法,此刀名爲『屠龍』,意爲日後有人得到刀中兵書,當驅除韃子,殺了韃子皇帝。倚天劍中藏的則是武學祕笈,其中最爲寶貴的,乃是一部『九陰眞經』,一部『降龍十八掌掌法精義』,盼望後人習得劍中武功,替天行道,爲民除害。」周芷若睜著眼睛,愈聽愈奇,只聽師父又道︰「黃女俠鑄成一刀一劍之後,將寶刀授給児子郭公破虜,寶劍授給本派郭祖師。當然,郭祖師曾得父母傳授武功,郭公破虜也得傳授兵法。但郭公破虜和父母同時殉難,郭祖師的性子和父親的武功不合,因此本派的武學,和當年郭大俠並非一路。」

周芷若曾聽師姊們説過,江湖上各幫各派如何爭奪屠龍刀,以致群俠同上武當,逼死了張無忌的父母,今日聽師父説起,才知此刀此劍原來和本派有著偌大的関連,只聽滅絶師太又道︰「一百年來,武林中風波迭起,這對刀劍換了好幾次主人。後人只知屠龍寶刀乃是武林至尊,唯倚天劍可與匹敵,但到底何以至尊,那就誰都不知道了。郭公破虜青年殉國,没有傳人,是以刀劍中祕密,只有本派郭祖師傳了下來。她老人家生前曾竭盡心力,尋訪屠龍寶刀,始終没有成功,逝世之時,將這祕密傳給了恩師一清師太。我恩師爲人太過慈和,心腸太軟,收了我那不成材的師姊,累得屠龍刀固然没有找到,連本門的倚天劍也給我師姊盜了出去,拿去獻給朝廷。我恩師飲恨而終,遺命要我尋到屠龍刀,奪回倚天劍。」周芷若道︰「啊,原來我有這樣的一位師伯。」

滅絶師太臉上突然籠罩了一股煞氣,説道︰「這等欺師滅祖的本門叛徒,你也叫她師伯麼?」周芷若低下了頭,不敢言語。滅絶師太道︰「後來這個叛徒終於給我找到,此人心術不正,武功難以學到上乘,嘿嘿,爲師總算不負你師祖的遺志,清理了門戸。」周芷若驚道︰「清理了門戸?」滅絶師太臉上閃過一絲又驕傲又殘酷的神色,昂然道︰「不錯,在長沙嶽麓山脚下,我追到了那個叛徒。我用一招『非花非煙』,刺入她的心窩。這招『非花非煙』,正是她從前教過我的,一直譏笑我使得不對,説我永遠學不會。那一晩嶽麓山月下鬥劍,我本在二百招内便可取她性命,但我偏偏要用這一招『非花非煙』殺她,以致多鬥了一百多招。嘿嘿,那是二十多年前的事了。」周芷若聽到此處,不由得打了個寒噤,不知如何,對那個背叛本門的師伯,心中竟是隱隱生出了幾分同情和憐憫之意。

只聽得鹿杖客又伸手打門,説道︰「完了没有?我可不能再等了。」滅絶師太道︰「不用性急,片刻之間,便説完了。」她悄聲對周芷若道︰「時刻無多,咱們不能多説廢話。總而言之,這柄倚天劍後來是韃子皇帝賜給了汝陽王,我到汝陽王府中劫了回來,這一次不幸誤中奸計,落入了魔教手中。」周芷若道︰「不是啊,是那個趙姑娘奪了去的。」滅絶師太眼睛一瞪道︰「這姓趙的女子,明明和那魔教教主是一路,難道你到此刻,仍是不信爲師的言語?」周芷若心中實在難以相信,但不敢和師父爭辯。滅絶師太道︰「爲師要你接任掌門,實有深意。爲師此番落入奸徒手中,一世英名,付與流水,實也不願再生出此塔。那姓張的淫徒對你心存歹意,決不致害你性命,你可和他虛與委蛇,乘機奪到倚天劍。那屠龍刀是在他義父惡賊謝遜手中。這小子無論如何不肯吐露謝遜的所在,但天下却有一人能叫他去取得此刀。」周芷若明知師父説的乃是自己,又驚又羞,又喜又怕。滅絶師太道︰「這個人,那就是你了。我要你以美色相誘,取得寶刀寶劍,原非俠義之人份所當爲,但成大事者,不顧小節,你且試想,倚天劍在姓趙女子手中,屠龍刀在謝遜惡賊手中,他這一干人同流合汚,刀劍相逢,取得郭大俠的兵法武功。自此荼毒倉生,天下不知將有多少人無辜喪生,妻離子散,而驅除韃子的大業,更是難上加難。芷若,我明知此事太難,實是不忍要你擔當,可是我輩一生學武,所爲何事?芷若,我是爲天下的百姓求你。」説到這裡,突然間站起身來,雙膝跪下,向周芷若拜了下去。

周芷若這一驚實是非同小可,急忙也跪了下去,叫道︰「師父。」滅絶師太道︰「悄聲,别讓外邉的惡賊聽見,你答不答應?你不答應,我不能起來。」

\chapter{天龍五刀}

周芷若心亂如麻,在這短短的時刻之中,師父連續要叫自己做三件難事,先是立下毒誓,不許對張無忌傾心,再要自己接任本派掌門,然後又要自己以美色對張無忌相誘,取得屠龍刀和倚天劍。這三件事便是在十年之中分别要她答應,以她柔和溫婉的性格,也是無抵擋不住,何況是在這片刻之間?周芷若神智一亂,登時便暈了過去,什麼也不知道了。

突然間只覺上唇間一陣劇烈疼痛,她睜開眼來,只見師父仍是直挺挺的跪在自己面前。周芷若哭道︰「師父,你老人家快些請起。」滅絶師太道︰「那你是答應我的所求了?」周芷若流著泪點了點頭,險險又欲暈去。滅絶師太抓住她的手腕,低聲道︰「你取到屠龍刀和倚天劍後,找個隱祕的所在,一手執刀,一手持劍,運起内力,以刀劍互砍,寶刀寶劍便即斷折,即可取出藏在刀身和劍刃中的祕笈。這是取出祕笈的唯一法子,那寶刀寶劍也從此毀了。你記住了麼?」她説話聲音雖低,語氣却是嚴峻。周芷若點頭答應,滅絶師太又道︰「這法子是本派最大的祕密,自從當年黃女俠傳於本派郭師祖,此後只有本派掌門,始能獲知這個祕密。想那屠龍刀和倚天劍都是鋒鋭絶倫的利器,無堅不摧,無物不破,就算有人同時得到寶刀寶劍,有誰敢冒以刀劍互砍,無端端的同時毀了這兩件寶刃?你取得兵法之後,擇一個心地仁善,赤誠爲國的志士,將兵書傳授於他,要他立誓驅除胡虜。那武功祕笈便由你自練。爲師生平有兩大願望,第一是逐走韃子,光我漢室河山;第二是峨嵋派武功領袖群倫,蓋過少林、武當,成爲中原武林中的第一門派。這兩件事説來甚難,但眼前擺著一條明路,你只須遵師囑,未始不能一一成就,那時爲師在九泉之下,也要對你感激涕零。」

她説到這裡,只聽得鹿杖客又在打門。滅絶師太道︰「進來吧!」板門一開,進來的不是鹿杖客而是苦頭陀。滅絶師太也不以爲異,心想這些人都是一丘之貉,不論是誰來都是一樣,便道︰「你把這孩子領出去吧。」她不願在周芷若的面前自刎,以免她心神過於激盪,只怕抵受不住。

那知苦頭陀走近身來,低聲道︰「這是解你體内毒性的解藥,請快服了。待會聽得外面叫聲,大家拚力殺出。」滅絶師太奇道︰「閣下是誰?何以給解藥於我?」苦頭陀道︰「在下是明教的光明右使范遙,特來相救師太。」滅絶師太怒道︰「魔教的奸賊,到此刻尚來戲弄於我。」苦頭陀笑道︰「好吧!就算是我戲弄你,這是毒上加毒的毒藥,你有没膽子服了下去?藥一入肚,一個時辰後肚腸寸寸斷裂,死得慘不可言。」滅絶師太一言不發,接過他手中的藥粉,張口便服入肚内。周芷若驚叫︰「師父\dash{}師父\dash{}」苦頭陀伸出另一隻手掌,喝道︰「不許作聲,你也服了這毒藥。」周芷若一驚,已被苦頭陀捏住她的臉頰,將藥粉倒入她的口中,跟著一瓶清水灌了下去,藥粉盡數落喉。滅絶師太大驚,心想周芷若一死,自己全盤策劃付諸東流,當下奮不顧身的撲上,一掌向苦頭陀打去。可是她此時功力全失,這一掌能有什麼力道,被苦頭陀輕輕一推,便撞到了牆上。苦頭陀笑道︰「少林群僧、武當諸俠都已服了我這毒藥。我明教教徒是好人還是歹徒,你片刻便知。」説著哈哈一笑,轉身出房,反手帶上了門。

原來苦頭陀護送趙明去和張無忌相會,心中只是掛著奪取解藥之事。趙明命他在小酒家的外堂中相候,他立即出店,飛奔回到萬法寺,進了高塔,逕登最高一層,走到游龍子房外。游龍子正站在門外。見了他便恭恭敬敬的叫聲︰「苦大師。」

苦頭陀點了點頭,心中暗笑︰「好啊,鹿老児爲師不尊,自己躱在房中,和王爺的愛姬風流快活,却叫徒児在門外把風。乘著這老児正在胡天胡帝之時,掩將過去,正好奪了他的解藥。」當下佝僂著身子,從游龍子身旁走過,突然間反手一指,點中了他小腹上的穴道。别説游龍子絲毫没有提防,便是全神戒備,也未必躱得過苦頭陀這一指,他要穴一被點中,立時呆呆的不能動彈,心下大爲奇怪,不知什麼地方得罪了這個啞巴頭陀,難道剛纔這一聲「苦大師」叫得不彀恭敬麼?

苦頭陀一推房門,快如閃電的撲向床上,雙脚尚未落地,一掌已擊向床上之人。他深知鹿杖客武功了得,這一掌若是不能將他擊得重傷,那便是一場不易分得勝敗的生死搏鬥,是以這一掌用上了十成的勁力。只聽得拍的一聲響,這一掌只擊得棉被破裂,棉絮紛飛,揭開棉被一看,只見韓姬口鼻流血,已被他一掌打得香消玉碎,却不見鹿杖客的影子。苦頭陀心念一動,回身出房,將游龍子拉了進來,塞在床底,剛掩上門,只聽得鹿杖客在門外怒聲叫道︰「龍児,龍児,你怎敢擅自走開?」

原來鹿杖客在滅絶師太室外等了好一陣,暗想她母女二人婆婆媽媽的不知説到幾時方罷,心中掛念著韓姬,便即回到游龍子房來,見這一向聽話的大弟子居然没在房外守衛,心下好生惱怒,推開房門,幸好並無異狀,韓姬仍是面向裡床,身上蓋了棉被。鹿杖客拿起門閂,先將門上閂,轉身笑道︰「美人児,我來給你解開穴道,可是你不許出聲説話。」一面説,一面便伸手到被窩中去,手指剛碰到韓姬的背脊,突然間手腕上一緊,五根鐵鉗般的手指已將他脈門牢牢扣住。這一下全身勁力登失,半點力道也使不出來,只見棉被掀開,一個長髮頭陀鑽了出來,正是苦頭陀。

范遙右手扣住鹿杖客的脈門,左手運指如風,連點了他周身一十九處大穴。鹿杖客登時軟癱在地,再也動彈不得,眼光中滿是怒色,苦頭陀指著他説道︰「老夫行不改姓,坐不改名,明教光明右使,姓范名遙便是。今日你遭我暗算,枉你自負機智絶倫,其實是昏庸無用之極。此刻我若殺了你,非英雄好漢之所爲,且留下你一條性命,你若有種,日後只管來找我范遙報仇。」生怕他運氣衝穴,此人内力深厚,却是不可不防,當下抓著他的四肢,喀喇喀喇數聲,將他手足的骨骼都折斷了。范遙此人邪氣極重,此時興猶未足,伸手脱去鹿杖客全身依服,將他剝得赤條條地,和韓姬的屍身並頭而臥,再拉過棉被,蓋在這一死一活的二人身上。這纔取過他的鹿角雙杖,旋開鹿角,倒出解藥,然後逐一到各間囚室之中,分給空聞大師、宋遠橋、兪蓮舟等各人服下。待得一個個送畢解藥,耗時已然不少,中間又不免費些唇舌,解釋幾句。最後來到滅絶師太室中,見她不信此是解藥,索性嚇她一嚇,説是毒藥,要知范遙恨她傷殘本教衆多兄弟,能彀陰損她幾句,也覺快意。

他分送解藥已畢,正自得意,忽聽塔下人聲喧嘩,其中鶴筆翁的聲音最是響亮︰「這苦頭陀是奸細,快拿他下來,快拿他下來!」苦頭陀暗暗叫苦︰「糟了,糟了,是誰去救了這傢伙出來?」探頭向塔下一望,只見鶴筆翁率領了大批武士,已將高塔團團圍住。苦頭陀這一探頭,孫三毀和李四摧雙箭齊發,大罵︰「惡賊頭陀,害得人好慘!」原來鶴筆翁等三人穴道被點,本非一時所能脱困,他三人藏在鹿杖客房中,旁人也不敢貿然進去。豈知汝陽王府中派出來的武士在萬法寺中到處搜査,不見王爺愛姬的影蹤,便有人想起了鹿杖客生平好色貪花的性子來。

可是衆武士對鹿杖客向來忌憚,雖然大家心中起疑,王爺愛姬的失蹤只怕和他有関,却有誰敢去太歳頭上動土?挨到後來,各人心想倘眞搜査不到蹤跡,王爺必定罪責,那武士總管哈禮赤花心生一計,命一個猥猥瑣瑣的小武士去敲鹿杖客的房門,諒那鹿杖客身份極高,就算動怒,也不能對這無足輕重的小武士怎麼樣。這小武士硬起了頭皮,提心吊膽的去打門,不料打了數下,房中無人答應。哈禮赤花一咬牙,命他只管推門進去瞧瞧。這一瞧,便瞧見鶴筆翁和孫三毀、李四摧倒在地下。其時鶴筆翁運氣衝穴,已衝開了三四成,哈禮赤花給他一解穴,登時便行動自如。他怒氣沖天,査問鹿杖客和苦頭陀的去向,知道到了高塔之中,便率領衆武士圍住高塔,大聲呼喊,叫苦頭陀下來決一死戰。

苦頭陀暗罵︰「決一死戰便決一死戰,難道我姓范的還怕了你不成?只是那些臭和尚、老尼姑服藥未久,一時三刻之間功力不能恢復。這鶴筆翁已聽到我和鹿杖客的説話,就算我去將鹿杖客殺了,也已不能滅口,這便如何是好?」一時徬徨無計,只聽得鶴筆翁叫道︰「死頭陀,你不下來,我便上來了!」苦頭陀返身入游龍子的房中,將鹿杖客和韓姬一起裹在被窩之中,回到塔邉,將兩人高高舉起,叫道︰「鶴老児,你只要走近塔門一步,我便將這頭淫鹿摔了下來。」衆武士手中高舉火把,照耀得四下裡白晝相似,只是那寶塔太高,火光照不上去,但影影綽綽的,仍是可看到鹿杖客和韓姬的面貌。鶴筆翁大驚,叫道︰「師哥,師哥,你没事麼?」連叫數聲,不聽見鹿杖客答話,只道已被苦頭陀弄死,不禁心下氣苦,叫道︰「賊頭陀,你害死我師哥,我跟你誓不兩立。」苦頭陀手肘一撞,解開了鹿杖客的啞穴。鹿杖客立時破口大罵︰「賊頭陀,你這裡應外合的奸細,千刀萬剮的殺了你\dash{}」苦頭陀容他罵得幾句,又點上了他的啞穴。鶴筆翁見師兄未死,心下稍安,只怕苦頭陀眞的將師兄摔了下來,不敢走向塔門。

這樣僵持了良久,鶴筆翁始終不敢上來相救師兄。苦頭陀只盼儘量拖延時光,多拖得一刻便好一刻,他站在欄干之旁,哈哈大笑,叫道︰「鶴老児,你師兄膽大包天,竟將王爺的愛姬偸盜出來。是我捉姦捉雙,將他二人當場擒獲,你敢包庇師兄麼?哈禮赤花總管,你還不快快將這老児拿下?他師兄弟二人反上作亂,罪不容誅。你拿下了他,王爺定然重重有賞。」哈禮赤花斜目睨視鶴筆翁,要想動手,却又不敢。他見苦頭陀突然開口説話,雖覺奇怪,但清清楚楚的瞧見鹿杖客是和韓姬裹在一條棉被之中,早已信了九成,他高聲叫道︰「苦大師,請你下來,咱們同到王爺跟前分辯是非。你們三位都是前輩高人,小人誰也不敢冒犯。」苦頭陀膽大包天,心想回到王府之中去見王爺,待得分明白是非黑白,塔上諸俠體内毒性已解,當即叫道︰「妙極,妙極!我正要向王爺領賞。哈總管,你看住這個鶴老児,别讓他乘機逃了。」

正在此時,忽聽得馬蹄聲響,一乘急馬急奔進寺,直衝到高塔之前,衆武士一齊躬身行禮,叫道︰「小王爺!」苦頭陀從塔上望將下來,只見此人頭上束髮金冠閃閃生光,跨著一匹神駿白馬,身穿錦袍,正是汝陽王的世子庫庫特穆爾,漢名王保保的便是。他厲聲問道︰「韓姬呢?父王大發雷霆,要我親來査看。」哈禮赤花上前稟告,言語之中,竟説是鹿杖客將韓姬盜了來,現被苦頭陀拿住。鶴筆翁急道︰「小王爺,莫聽他胡説八道。這頭陀乃是奸細,他陥害我師哥\dash{}」王保保雙眉一軒,叫道︰「一起下來説話!」

苦頭陀在王府中日久,知道這位小王爺王保保精明能幹,猶勝乃父,自己的詭計瞞得過旁人,須瞞不過小王爺,自己一下高塔,倘若小王爺三言兩語之際便識穿破綻,下令衆武士圍攻,單是一個鶴筆翁便不好鬥,自己脱身或不爲難,塔中諸俠那就救不出來了,事到如今,只有破臉,於是高聲説道︰「小王爺,我拿住了鹿杖客,他師弟恨我入骨,我只要一下來,他立刻便會殺了我。」王保保道︰「你快下來,鶴先生殺不了你。」苦頭陀搖頭道︰「我還是在塔上平安些。小王爺,我苦頭陀一生不説話,今日事出無奈,被迫開口,那全是我報答王爺的一片赤膽忠心。你若不信,我苦頭陀只好跳下高塔,一頭撞死給你看了。」

王保保聽他言語之中,已有七八成是在胡説八道,顯然是有意拖延時間,低聲問哈禮赤花道︰「哈總管,他有何圖謀,要故意延擱,是在等候什麼人到來麼?」哈總管道︰「小人不知\dash{}」鶴筆翁搶著道︰「小王爺,這賊頭陀搶了我師哥的解藥,要解救高塔中囚禁著的叛逆。」王保保一聽,登時省悟,叫道︰「苦大師,我知道你的功勞,你快下來,我重重有賞。」苦頭陀道︰「我被鹿杖客踢了兩脚,腿骨都快斷了,這會児動彈不得。小王爺,請你稍待片刻,我運氣療傷,當即下來。」王保保喝道︰「哈總管,你派人上去,扶苦大師下塔。」苦頭陀大叫︰「使不得,使不得,誰一移動我的身子,我兩條腿子就廢了。」

王保保此時更無懷疑,眼見韓姬和鹿杖客雙雙裹在一條棉被之中,就算兩人並無苟且之事,父王也不能再要這個姬人,低聲道︰「哈總管,舉火,焚了這座塔子。派人用強弓射住,有人從塔上跳下,一槩格殺。」哈禮赤花答應了,傳下令去,登時弓箭手彎弓搭箭,團團圍住高塔,有些武士便去取火種柴草。鶴筆翁大驚道︰「小王爺,我師哥在上面啊。」王保保冷冷的道︰「這頭陀不能在上面等一輩子,塔下一舉火,他自會下來。」鶴筆翁叫道︰「他若是將我師哥摔將下來,那可怎麼辦?小王爺,這火不能放。」王保保哼了一聲,不去理他。

片刻之間,衆武士已取過柴草火種,在塔下點起火來。鶴筆翁是武林中大有身份之人,受汝陽王禮聘入府,向來甚受敬重,不料今日連中苦頭陀的奸計不算,連小王爺也不以禮貌相待,眼見師兄的性命危在頃刻,這時也不理什麼小王爺不小王爺,提起鶴筆雙筆,縱身而上,挑向兩名正在點火的武士,巴巴兩響,兩名武士遠遠摔開。王保保大怒,喝道︰「鶴先生,你竟要犯上作亂麼?」鶴筆翁道︰「你别叫人放火,我自不會來跟你搗亂。」王保保不去理他,喝道︰「點火!」左手一揮,突然他身後竄出五名紅衣番僧,從衆武士手中接過火把,向塔下的柴草擲了過去。柴草一遇火燄,登時便燃起熊熊烈火。鶴筆翁大急,從一名武士手中搶過一根長矛,撲打著火的柴草。王保保喝道︰「拿下了!」那五名紅衣番僧各持戒刀,登時將鶴筆翁圍住。鶴筆翁怒極,一抛長矛,伸手便來拿左首一名番僧手中的兵刃。不料這番僧絶非庸手,戒刀一翻,反{\upstsl{{\upstsl{刴}}}}他的肩頭,鶴筆翁待得避開,身後金刃劈風之聲,又有兩柄戒刀同時砍到。

原來這五名番僧乃是王保保手下的親信,屬於「天龍十八」之部内。王保保之出府時,喜歡單騎獨行,但十八番僧總是遠遠相隨衛護。這天龍十八部共分五刀五劍、四杖四鈸,這五人乃是「五刀神」,每個人各有傑出的技藝。若是單打獨鬥,誰也不是鶴筆翁的對手,但五刀神聯手,攻守相助,鶴筆翁武功雖高,一時却有些手忙脚亂,何況眼見火勢上騰,師兄的處境極是危險,不免沉不住氣。

鶴筆翁一被「天龍五刀」纏住,王保保手下衆武士加柴點火,將那高塔燒得更加旺了。這寶塔有磚有木,在這大火災燒之下,底下數層便必必剝剝的燒了起來。苦頭陀抛下鹿杖客,衝到囚禁武當諸俠的室中,叫道︰「韃子在燒塔了,各位内力是否已復?」只見宋遠橋、兪蓮舟等人各自盤坐用功,凝神專志,誰也没有答話,顯然到了回復功力的緊要関頭。這時看守諸俠的武士有幾名搶過干預,都被苦頭陀抓將起來,一個個擲出塔外,活活的摔死,其餘的冒火突煙,逃了下去,也有幾名被燒斷了去路,無法出塔,只有反而逃了上來。

過不多時,火燄已燒到了第三層,囚禁在這一層中的華山派諸人,不及等功力恢復,十分狼狽的逃到了第四層。火燄毫不停留的上騰,跟著第四層中的崆峒派諸俠也逃了上去,有的奔走稍慢,連衣服鬚髮都燒著了。

苦頭陀正束手無策之際,忽聽得一人叫道︰「范右使,接住了!」正是韋一笑的聲音。苦頭陀大喜,往聲音來處瞧去,只見韋一笑站在萬法寺後殿的殿頂,雙手一抖,將一條長繩抛了過來,苦頭陀伸手接住。韋一笑叫道︰「你縛在欄干上,當是一道繩橋。」苦頭陀剛將繩子縛好,神箭八雄中的趙一傷颼的一箭,便將繩子從射斷。苦頭陀和韋一笑同時破口大罵,可是知道這神箭八雄箭法厲害,若要搭繩橋須得先除去八人再説。韋一笑罵道︰「射你個奶奶,那一個不抛下弓箭,老子先宰了他。」一面罵,一面抽出兵刃,縱身下地。他所用的乃一對虎頭雙鉤,若非今日事態緊急,那是輕易不動兵刃。他雙足剛著地,五名青袍番僧立時仗劍圍了上來,却是天龍十八部中的五劍僧,五個人手中兵刃青光閃爍,劍招極是詭異,和韋一笑鬥在一起。

鶴筆翁揮動鶴筆苦戰,高聲叫道︰「小王爺,你再不下令救火,我可對你要不客氣了。」王保保那去理他。四名手執禪杖的高大番僧分立小王爺的四周,生怕有人偸襲。鶴筆翁焦躁起來,雙筆突然使一招「橫掃千軍」,將身前的三名番僧逼開兩步,提氣一衝,已衝到了高塔之旁。五名番僧一齊追到,鶴筆翁雙臂一展,正如大鳥般上了高塔第一層的屋簷。那五名番僧見火勢燒得正旺,便不追上。

鶴筆翁一層層的上躍,待得登上第四層屋簷時,苦頭陀從第七層探頭出來,高舉鹿杖客的身子大笑叫道︰「鶴老児,給我停步!你再動一步,我便將鹿老児摔成一團肉泥。」鶴筆翁果然不敢再動,叫道︰「苦大師,我師兄弟跟你往日無冤,近日無仇,你何苦如此跟咱們爲難,你要救你的老情人滅絶師太,要救你女児周姑娘,儘管去救便是,我決計不來阻攔。」

滅絶師太服了苦頭陀給她的解藥後,只道眞是毒藥,自分必死,只是周芷若竟被他也灌了毒藥,自己全盤計劃,盡數化爲泡影,心中如何不苦?正自傷心,忽聽得塔下喧嘩之聲大作,跟著苦頭陀和鶴筆翁鬥口、王保保下令縱火等等情形,一一聽得清楚。她心下奇怪︰「莫非這鬼模樣的頭陀當眞是救我來著?」試一運氣,立時便覺丹田中一股暖意升將上來,和自中毒以來的情形大不相同。原來滅絶師太不肯聽趙明之令,到大殿上比武,已自行絶食了六七日,胃中早是空空如也,解藥一入肚中,迅速化入血液,藥力行開,比誰都快,加之她内力深厚,猶在宋遠橋、兪蓮舟、何太沖諸人之上,僅比少林派掌門空聞神僧稍遜,那十香軟筋散的毒性,遇到解藥漸漸消退,被滅絶師太用力一逼,内力登時生出,不到半個時辰,内力已復了五六成,她正在加緊催動内功,忽聽得鶴筆翁在外面高聲大叫。

鶴筆翁幾句高聲大叫,字字如利箭般鑽入滅絶師太的耳中︰「\dash{}你要救你的老情人滅絶師太,要救你女児周姑娘,儘管去救便是,我決計不來阻攔。」滅絶師太自幼嚴守清規,少年之時,連男子的面孔也不見,什麼「老情人」云云,叫她如何不怒?她大踏步走到欄干之旁,怒聲喝道︰「你滿嘴胡説八道,不清不白的説些什麼?」鶴筆翁求道︰「老師太,你快勸勸你老\dash{}老朋友,先放我師兄下來。我擔保你一家三口,平安離開。玄冥二老説一是一,説二是二,決不致言而無信。」滅絶師太怒道︰「什麼一家三口?」苦頭陀雖然身處危境,還是呵呵大笑,很是得意,説道︰「老師太,這老児説我是你的舊情人,那位周姑娘嘛,是我跟你生的私生女児。」滅絶師太怒容滿面,在時明時暗的火光照耀之下,看來極是可怖,沉聲喝道︰「鶴老児,你上來,我跟你拚上一百掌再説。」若在平時,鶴筆翁説上來便上來,何懼於一個峨嵋掌門,但此刻師兄落在别人手中,不敢蠻來,叫道︰「苦頭陀,那是你自己説的,可不是我信口開河。」滅絶師太雙目瞪著苦頭陀,厲聲問道︰「這是你説的麼?」苦頭陀哈哈一笑,正要乘機挖苦她幾句,忽聽塔下喊聲大作,往下一望,只見火光中一條人影如穿花蝴蝶般迅速飛舞,在人群中穿插來去,嗆{\upstsl{啷}}{\upstsl{啷}},嗆{\upstsl{啷}}{\upstsl{啷}}之聲不絶,衆番僧、衆武士手中兵刃紛紛落地,原來是教主張無忌到了。

張無忌這一出手,圍攻韋一笑的五名持劍番僧五齊飛。韋一笑大喜,一閃身,搶到他的身旁,低聲道︰「我到汝陽府去放火。」張無忌點了點頭,已明白他的用意。須知自己這裡只有寥寥數人,要是急切間救不出人,對方湧來的應援人手定然越來越多,這青翼蝠王到汝陽王府去一放火,衆武士保護王爺要緊,乃是個絶妙的調虎離山,斧底抽薪之計。只見韋一笑一條青色人影一晃,已自掠過高牆。

張無忌一看周遭情勢,朗聲問道︰「范右使,怎麼了?」苦頭陀叫道︰「糟糕之極!燒斷了出路,一個也没能逃得出。」此時天龍十八部的番僧,倒有十四人攻到了無忌身畔。無忌心想擒賊先擒王,只須擒住了那頭戴金冠的韃子王公,便能要脅他下令救火放人,當下身形一側,從衆番僧間竄了過去,猶似游魚破水,直欺到王保保身前。驀地裡左首一劍刺到,寒氣逼人,劍尖直指胸口。張無忌急退一步,只聽得一個女子聲音説道︰「張公子,這是家兄王保保,你莫傷他。」但見她手中長劍顫動,婀娜而立,刃寒勝水,劍是倚天劍,貌美如花,人是趙明。她急跟張無忌而來,只不過遲了片刻。

張無忌道︰「你快下令救火放人,否則我可要對不起兩位了。」趙明叫道︰「天龍十八部,此人武功了得,結天龍陣擋住了。」那十八番僧適纔吃過無忌的苦頭,不須郡主言語點明,已知道他的厲害,只聽得{\upstsl{噹}}的一聲大響,「四鈸神」手中的八面銅鈸齊聲敲擊,十八番僧來回遊走,擋住王保保和趙明的身前,將無忌隔開了。無忌一瞥之下,見這十八名番僧盤旋遊走,步法極是詭異,十八個人阻成一道人牆,看來其中還蘊藏著不少變化。他心念一動,忍不住想憑著一身武功,衝一衝這座天龍陣,但便在此時,砰的一聲大響,高塔上倒了一條大柱下來。無忌一回頭,只見火燄已燒到第六層上。火舌繚繞之中,兩個人拳掌交加,鬥得極是激烈,正是滅絶師太和鶴筆翁。最高一層的欄干之旁,倚滿了人,都是少林、武當各派人物,這一干人武功尚未全復,何況那高塔離地數十丈,縱有絶頂輕功,内力絲毫未失,跳下來縱不活活摔死,也必筋折骨斷。

張無忌一個念頭在腦海中飛快的轉了幾轉︰「要破此天龍陣,非片刻間所能奏效,何況擊敗衆番僧,又有别的好手上來,想擒那韃子王公,大也不易。滅絶師太和這鶴筆翁鬥了這些時,始終未曾落敗,看來她功力已復,那麼我大師伯等内力當也已經恢復,只是寶塔太高,無法躍將下來而已。」一動念間,突然滿場遊走,雙手忽打忽拿、忽拍忽奪,將神箭八雄盡數擊倒,此外衆武士中,凡是手持弓箭的,都被他或斷弓箭,或點穴道,眼看高塔近旁已無彎弓搭箭的手,便縱聲叫道︰「塔上各位前輩,請逐一跳將下來,在下在這裡接著。」

塔上諸俠一聽,都是一怔,心想此處相距地面數十丈,若是跳了下去,力道何等巨大,你便是有千斤之力,也無法接住。崆峒、崑崙各派的人中,便有人嚷道︰「千萬跳不得,莫上這小子的當!他要騙咱們摔得粉身碎骨。」無忌眼看煙火濔漫,已燒到了第七層,衆人若再不跳,勢必葬身火窟,提聲叫道︰「莫七叔,你等我恩重如山,難道小侄會存心相害麼?你先跳吧!」莫聲谷原是個極爲大膽之人,心想與其活活燒死,還不如活活摔死,便叫道︰「好!我跳下來啦!」縱身一躍,從高塔上跳了下來。張無忌看得分明,待莫聲谷身子離地約有四尺之時,一掌輕輕拍出,擊在他的腰裡。這一掌中所運,正是「乾坤大挪移」的絶頂神妙武功,吞吐控縱之間,已將他自上向下的一股巨力,撥爲自左至右。莫聲谷的身子向橫裡直飛出去,一摔數丈,此時他功力已恢復了七八成,一個迴旋,已然穩穩站在地下,順手一掌,將一名蒙古武士打得口噴鮮血。他大聲叫道︰「大師哥、二師哥、四師哥!你們都跳下來吧!」

塔上衆人見莫聲谷居然安好無恙,一齊大聲歡呼起來。宋遠橋愛子情深,要他先脱險地,説道︰「青書,你跳下去!」宋青書自出囚室後,一直站在周芷若身旁,説道︰「周姑娘,你快跳。」周芷若功力未復,不能去相助師父,却不肯自行逃生,聽宋青書這麼説,搖了搖頭道︰「我等師父!」

這時何太沖班淑嫻等已先後跳下,都由張無忌施展乾坤大挪移神功,自直墜改爲橫摔,一一脱離險境。這一干人功力雖未全復,但只須恢復得五六成,已是衆番僧、衆武士所難以抵擋。莫聲谷等頃刻間奪得兵刃,護在張無忌身周。王保保和趙明的手下意圖殺上阻撓,均被莫聲谷、何太沖、班淑嫻等擋住。塔上每躍下一人,張無忌便多了一個幫手。那些人自被趙明囚入高塔之後,人人受盡屈辱,也不知有多少人被割去了手指,此時重出生天,個個含憤拚命,霎時間已有十餘武士屍橫就地。

王保保見情勢不佳,傳令︰「調我的飛弩親兵隊來!」哈總管正要去傳小王爺號令,突然間東南角上火光衝天。哈總管一驚,叫道︰「小王爺,王府走了火啦,咱們快去保護王爺要緊。」王保保関懷父親安危,顧不得擒殺叛賊,忙道︰「妹子,我先回府,你諸多小心!」不等趙明答應,掉轉馬頭,直衝出來。王保保這一走,天龍十八部的衆番僧及王府武士倒去了一大半,餘下衆武士見王府失火,誰也没想到只是韋一笑一個人搗鬼,只道大批叛徒進攻王府,無不驚惶。

其時宋青書、宋遠橋、兪蓮舟、張松溪等都已躍下高塔,雙方強弱之勢登然逆轉,待得空聞大師、空智大師,以及少林派達摩堂、羅漢堂、藏經閣衆高僧一一躍下時,趙明手下的武士已無可抗禦。趙明心想此時若再不走,反而自己要成爲他的俘虜,當即下令︰「各人退出萬法寺。」轉頭向張無忌道︰「明日黃昏,我再請你飲酒,務請駕臨。」

\chapter{以德報怨}

張無忌一怔之間,尚未答應,趙明已是一笑嫣然,退入了萬法寺的後殿,只聽得苦頭陀在塔頂大聲叫道︰「周姑娘,快跳下,火燒眉毛啦,你再不跳,難道想做焦炭美人麼?」周芷若道︰「我陪著師父!」滅絶師太和鶴筆翁鬥得正酣,她功力尚未全復,但此時早已將生死置之度外,掌法中只攻不守。鶴筆翁却一來掛念著師兄,心有二用,二來適纔中了麻藥之後,手脚究也不十分靈便,是以兩人竟鬥了個不分上下。滅絶師太聽到徒児的説話,叫道︰「芷若,你快跳下去,别來管我!這賊老児辱我太甚,豈能容他活命?」鶴筆翁心中暗暗叫苦︰「這老尼全是拚命的打法,我救師兄要緊,難道跟她在這火窟中同歸於盡不成?」當下大聲説道︰「滅絶師太,這話是苦頭陀説的,跟我可不相干。」滅絶師太撤掌迴身,問苦頭陀道︰「兀那頭陀,這等瘋話可是你説的?」苦頭陀嘻皮笑臉的道︰「什麼瘋話?」這一句話,明擺著要滅絶師太親口重複一遍︰「他説我是你的老情人,周芷若是我跟你生的私生女児。」這兩句,她如何能説得出口。但就是苦頭陀這句話,滅絶師太已知鶴筆翁之言不假,只氣得全身發顫。

鶴筆翁見滅絶師太背向自己,突然一陣黑煙捲到,正是偸襲的良機,煙霧之中,{\upstsl{噗}}的一掌,擊向滅絶師太的背心。周芷若和苦頭陀看得分明,齊聲叫道︰「師父小心!」「老尼小心!」但滅絶師太迴掌反擊,已擋不了鶴筆翁的陰陽雙掌,左掌和他的左掌相抵,鶴筆翁的右手所發的玄冥神掌,終於擊在他的背心。那玄冥神掌何等厲害,當年在武當山上,甚至和張三丰都對得一掌,此刻一掌擊在滅絶師太的背心,滅絶師太身子一晃,險險摔倒。周芷若大驚,搶上扶住了師父。苦頭陀却心中大怒,喝道︰「陰毒卑鄙的小人,留你作甚?」提起裹著鹿杖客和韓姬的被窩捲児,抛了下來。鶴筆翁爲人雖然狠毒,却是同門情深,危急之際不及細想,撲出來便想抓住鹿杖客。但那被窩捲児離塔太遠,鶴筆翁只抓到被窩一角,却跟著一起摔將下來。

張無忌站在塔下,煙霧瀰漫之中,瞧不清塔上這幾人的糾葛,眼見一大綑物事和一個人摔下,那綑事物不知是什麼東西,隱約間只看到其中包得有人,但那人却看清楚是鶴筆翁。他生性仁善,明知鶴筆翁曾累得自己不知吃過多少苦頭,甚至自己父母之死,也和他有莫大関連,可是終究不忍袖手不顧,任由他跌得粉身碎骨,立即縱身上前,雙掌分别拍出,將那被窩和鶴筆翁分向左右,擊出三丈。

鶴筆翁一個迴旋,已然站定,心中暗叫一聲︰「好險!」他萬没想到張無忌以德報怨,竟會救了自己一命,轉身去看師兄時,却又吃了一驚。原來張無忌雙掌齊使乾坤大挪移之去,同時化解兩邉自上向下急墜的來勢,究屬不易,何況那被窩中裹著鹿杖客和韓姬兩人,下墜之力更強,他一掌拍出,無法再顧得那被窩捲摔向何處。豈知這一拍之下,被窩散開,滾出兩個赤裸裸的人來,正好摔入火堆之中。鹿杖客穴道未解,動彈不得,鬚髮登時著火。鶴筆翁大叫︰「師哥!」搶入火堆之中,抱起了鹿杖客。他躍出火堆,立足未定,兪蓮舟叫道︰「吃我一掌!」一掌擊向他肩頭。鶴筆翁不敢抵敵,沉肩相避,豈知兪蓮舟這一掌,雖然似已用老,他肩頭下沉,兪蓮舟這一掌仍能跟著下擊,拍的一聲,只痛得他額頭冷汗直冒,此刻救師兄要緊,一咬牙,抱著鹿杖客身子,飛身躍出了高牆。便在此時,塔中又是一根燃燒著的大木柱倒將下來,壓著韓姬的屍身,片刻間全身是火。只聽得塔下衆人齊聲大叫︰「快跳下來,快跳下來!」

苦頭陀在塔頂東竄西躍,躱避火勢。那寶塔樑柱燒毀後,磚石紛紛跌落,塔頂已微微晃動,隨時都能塌將下來。滅絶師太厲聲道︰「芷若,你跳下去!」周芷若道︰「師父,你先跳了,我再跳!」滅絶師太突然縱身而起,一掌向苦頭陀的左肩劈下,喝道︰「魔教的賊子,實是容你不得!」苦頭陀在塔頂再也不能逗留,一聲長笑,縱身躍下。張無忌一掌擊出,將他輕輕送開,讚道︰「范右使,大功告成,當眞難能!」苦頭陀站定脚步,説道︰「若非教主神功蓋世,大夥児人人成了高塔上的烤豬。范遙行事不當,何功之有?」

滅絶師太見苦頭陀躍下,長嘆一聲,伸臂抱住了周芷若,踴身往塔下一跳,待離地面約有丈許,雙臂一推一托,反將周芷若托高了數尺,然後落下。這麼一來,周芷若變成只是從丈許高的空中落下,絲毫無礙,滅絶師太的下墜之勢却反而加強。張無忌搶步上前,運起乾坤大挪移神功,往她腰後拍去。豈知滅絶師太一來死志已決,二來決不肯受明教半分恩怨,見張無忌手掌拍到,拚起全身殘餘的力氣,反手一掌擊出。雙掌相交,砰的一聲大響,無忌那挪乾坤的掌力被她這一掌轉移了方向,但聽得喀喇一響,滅絶師太重重摔在地下,登時脊骨斷成數截,無忌却也被她挾著下墜之勢的這一掌打得胸口血氣翻湧,連連退了幾步,心下大惑不解,滅絶師太這一掌,明明便是自殺。

周芷若撲到師父身上,哭叫︰「師父,師父!」其餘峨嵋派的衆男女弟子,一齊圍在師父身旁,亂成一團。滅絶師太道︰「芷若,從今日起,你便是本派掌門,我要你做的事,你都不會違背麼?」周芷若哭道︰「是,師父,弟子不敢忘記。」滅絶師太微微一笑,道︰「如此,我死也瞑目\dash{}」這時只見張無忌走上前來,伸手要搭他脈博,看看是否尚有挽救之方,滅絶師太右手驀地裡一翻,緊緊抓住張無忌的手腕,厲聲道︰「魔教的淫徒,你若是玷汚了我愛徒的清白,我做鬼也不饒過\dash{}」最後一個「你」字没説出口,已然氣絶身亡,但手指竟是絲毫不鬆,五根指甲,將無忌手腕上的血也搯了出來。

苦頭陀叫道︰「大夥児一齊跟我來,到西門外會齊。若再耽擱,奸王可要派大隊人馬來啦。」張無忌抱起滅絶師太的屍身,低聲道︰「咱們走吧!」周芷若將師父的手指輕輕扳離無忌的手腕,接過屍身,向無忌一眼也不瞧,便向寺外走去。這時崑崙、崆峒、華山諸派高手早已蜂湧而出,只有少林派空聞、空智兩位神僧不失前輩風範,過來合什向張無忌道謝,和宋遠橋、兪蓮舟等相互謙讓一番,這纔先後出門。

張無忌以乾坤大挪移神功,相援六派高手下塔,内力幾已耗盡,最後和滅絶師太所對那一掌,更是大傷元氣。莫聲谷將他一把抱起,負在背後,無忌默運九陽神功,内力這纔漸漸增強。其時天已黎明,群雄來到西門,驅散把守城門的官兵,出城數里,楊逍已率領騾馬大車來接,向衆人賀喜道旁。空聞大師道︰「今番若不是明教張教主和各位相救,我中原六大派氣運難言。大恩不言謝,爲今之計,咱們該當如何,便請張教主示下。」張無忌道︰「在下識淺,有什麼主意,還是請少林方丈發號施令。」空聞大師堅執不肯。張松溪道︰「此處離城不遠,咱們今日在韃子的京城中鬧了這樣一個天翻地覆,那奸王豈能罷休?待得王府中火勢救滅,定必派遣兵馬來追。咱們還是先離此處,再定行止。」何太沖道︰「奸王派人來追,那是最好不過,咱便殺他一個落花流水,出出這幾日胸中的烏氣。」張松溪道︰「大夥児功力未曾全復,要殺韃子也不忙在一時,還是先避一避的爲是。」

空聞大師道︰「張四俠説的是,今日便是殺得多少韃子,大夥児也必傷折不小,咱們還是暫且退避。」少林掌門人説出來的話究竟聲勢又是不同,旁人再無異議。空聞大師又問︰「張四俠,依你高見,咱們該向何處暫避?」張松溪道︰「韃子料得咱們不是向南,便向東南,咱們偏偏反其道而行之,逕向蒙古,諸位以爲如何?」衆人都是一怔,心想蒙古是韃子的根本之地,如何反而深入敵境。楊逍却拍手説道︰「張四俠的見地高極。蒙古地廣人稀,莽莽荒漠之中,隨便找一處荒山,儘可躱得一時,韃子定道咱們回歸中原,萬萬想不到咱們竟會前往蒙古。」衆越想越覺張松溪此計大妙,當下撥轉馬匹,逕向北行。

行出五十餘里,群俠在一處山谷中打尖休息。楊逍早已購齊各物,乾糧酒肉,無一缺或。衆人談起脱困的經過,都説全仗張無忌和范遙兩人相救。這邉廂周芷若和峨嵋派衆人在地下掘了一坑,埋葬滅絶師太。空聞、空智、宋遠橋、張無忌等一一過去行禮致祭。滅絶師太一代大俠,雖然性情怪僻,但平素行俠仗義,正氣凜然,武林中人所共敬。峨嵋派弟子放聲大哭,餘人也各淒然。

空聞大師朗聲説道︰「人死不能復生,峨嵋諸俠只須繼承師太遺志,師太雖死猶生。這一次奸人下毒,誰都吃了大虧,本派空性師弟也爲韃子所害,此仇是非報不可,如何報仇,却須從長計議。」空智大師道︰「中原六大派原先與明教爲敵,但張教主以德報怨,反而出手相救,雙方仇嫌,自是一筆勾銷。今日乘大夥児都在間,老衲舉明教張教主爲中原武林盟主,此後只須張教主號令到處,中原各門各派一齊凜遵,同心協力,驅除胡虜。」他説一句,群豪便喝一聲采,待他説完,衆人更是歡聲雷動,只有周芷若默默無言,心中翻來覆去,儘想著師父囑咐自己的事。

張無忌連連搖手,請道︰「各位且慢,此事萬萬使不得。武林各派,向以少林爲尊。説到德高望重,則要算我太師父張眞人。武當諸俠都是我的師伯師叔,小子何敢僭越?」宋遠橋道︰「無忌,大夥児推你爲武林盟主,固然有一半是爲了今日感你相救之德,可是衆人也是爲天下蒼生請命。只盼各門各派從此齊心,再不自相殘殺,一致對付胡虜。中原武林中,若無一位發號施令的總盟主,只怕驅除韃子的大業,著實不易成功呢。」張松溪也道︰「少林派兩位神僧的推舉之意,極是誠懇。你太師父這麼高的年紀,難道還能請他老人家擔當這等劇繁重任?」衆人一再相勸,張無忌心下惴惴不安,無論如何不肯答應,説道︰「小子年輕識淺,若説稍有所長,也不過武功上略略有些成就。天下武林盟主一席,責任非輕,只有少林方丈神僧,或是宋大師伯,那纔合適。」楊逍道︰「教主,時機一失,不可再來。難得今日群雄聚會,大衆歸心。這武林盟主你若不當,别無群雄齊心歸服之人,大夥児一旦散向三岳五嶽,再要聚集,那可難了。當日你在光明頂上,囑咐咱們要和六大派化解仇怨,齊心合力,難道你便忘了。」

張無忌凜然心驚,默默無言。范遙大聲道︰「教主,做這武林盟主可不是做皇帝,大夥児不是要你作威作福,乃是要你任天下之大勞,負天下之大怨。你是不是男子漢大丈夫?這等天下的大勞大怨,你竟推三阻四的不肯擔當麼?范遙當你是英雄,甘心追隨於你,事到臨頭,你竟畏首畏尾麼?」張無忌向他恭恭敬敬的作了一揖,説道︰「范右使責備得是,無忌謹受教益。男児生於天地之間,原當不避艱危。」他抱拳向群俠説道︰「諸位推愛,小子不敢再辭,但願大業克成,不負平生之志。」

群雄聽張無忌這麼一説,登時歡聲雷動。楊逍取過一皮袋酒來,刺破手指,將向滴在酒裡,各人依次滴過,再每人喝了一口血酒,立誓自今而後,同心同德,以驅除胡虜、還我大漢山河爲志。張無忌又是興奮,又是惶恐,但想到范遙那幾句話,爲武林盟主者,當任天下之大勞,負天下之大怨,唯有鞠躬盡瘁,以報託付之重而已,至於成與不成,誰也不能逆料。他想到此處,心下反而坦然。這幾個月來,他經歷了不少風浪,增長了不少見識,此時出任武林盟主,反比當初接任明教教主之時,内心要鎭定得多。同時對驅除韃子一事,認爲義所當爲,不似於明教的正邪善惡,心中有許多不安之情,猶豫之意。

待各人歃血爲盟已畢,張無忌道︰「方今天下紛擾,我明教教衆已分處四方,機緣一到,立即舉義抗元。盼各派尊長知照本門本派的弟子,就近投效義軍,不得爭權奪利,自相吞併。一切是非爭執,只可向本派掌門投告,由本人會同各派掌門長老,秉公評斷。」衆人齊聲答應,説道︰「原該如此。」張無忌道︰「此間大事已了,我有些私人俗務,尚須回大都一轉,謹與各位作别。今後數年之間,當與各位並肩馳驅疆場,與韃子決一死戰。」群豪呼聲震天,山谷鳴響,一齊送到谷口。楊逍道︰「教主,你是天下英雄之望,一切多多保重。」無忌道︰「兄弟理會得。」馬鞭一響,胯下坐騎向南馳去。

將近大都之時,無忌心想昨晩萬法寺中這一戰,汝陽王手下的許多武士已識得自己面目,倘若撞上了,只怕諸多不便,於是到一家農家去買了一套莊稼漢子的舊衣服換了,頭上戴個斗笠,用煤灰泥巴將手臉塗得黑黑地,這纔進城。

他回到西城的客店外,四下一打量,前後左右,並無異狀,當即閃身入内,進了自己的住房。小昭正坐在窗邉,手中做著針線,見他進房,一怔之下,這纔認得了他出來,滿臉歡容,如春花之初綻,笑道︰「公子爺,我還道是那一個莊稼漢闖錯了屋子呢,眞没想到是你。」無忌笑道︰「你在做什麼?一個児悶不悶?」小昭臉上一紅,將手中縫著的衣衫藏到了背後,忸怩道︰「我胡亂做些活計。」忙將衣衫藏在枕頭底下,斟茶給無忌喝,笑道︰「你洗不洗臉?」無忌道︰「不洗了。」拿著茶杯,心下沉吟︰「趙姑娘要我陪她去借屠龍刀。一來大丈夫千金一諾,不能失信於人。二來我本要去接應義父他老人家回歸中土。義父本來擔心中原仇家太多,他眼盲之後,應付不了。此時武林群豪同心對抗胡虜,私人的仇怨,什麼都該化解了。只須我陪他老人家在一起,諒旁人也不能動他一根毫毛。大海中風濤險惡,小昭這孩子是不能一齊去的。{\upstsl{嗯}},有了,我要趙姑娘安頓她在王府之中,那倒比别的處所平安得多。」

小昭見他忽然微笑,問道︰「公子,你在想什麼?」無忌道︰「我要到一個很遠很遠的地方去,帶著你很是不便。我想到了一處所在,可以送你去寄居。」小昭臉上突然變色,道︰「公子爺,你到什麼地方去,我跟你到什麼地方,小昭要天天這樣服侍你,不願到陌生的所在去寄居。」無忌勸道︰「我是爲你好。我要去的地方很遠,很是危險,不知道什麼時候才能彀回來。」小昭道︰「公子爺,在光明頂上那個山洞之中,小昭已打定了主意,不論你到那裡,我都跟你到那裡。除非你把我殺了,才能撇下我。你是不是見了我討厭,不願意我陪著你麼?」無忌道︰「不,不!你知道我是很喜歡你的,我不願意你去冒無謂的危險。我一回來,立刻就會找你。」小昭搖頭道︰「你撇不下我的。只要在你身邉,什麼危險我都不在乎。公子爺,你帶我去吧!」

張無忌握著小昭的手,道︰「小昭,我也不須瞞你,我是答應了趙姑娘,要陪她往海外一行。大海之中,波濤連天,我是不得不去,但你去冒此奇險,殊是無益。」小昭脹紅了臉,道︰「你陪趙姑娘一起去我更加要跟著你。」説了這兩句話,急得雙眼中已是泪水盈盈。無忌道︰「爲什麼更加要跟著我?」小昭道︰「那趙姑娘心地歹毒,誰也料不得她會對你怎樣。我跟著你,也好照著你些児。」無忌心中一動︰「莫非這小姑娘對我暗中已生情意?」聽她這辭中忱忱之誠,心下不禁感激,笑道︰「好,我帶便帶了你去,大海中暈起船來,可不許叫苦。」小昭大喜,連連答應,道︰「我若是惹得你麻煩了,你把我抛下大海去餵魚吧!」無忌笑道︰「我怎麼捨得?」

他二人雖然相處日久,有時旅途之際客舍不便,便同臥一室,但小昭自居婢僕,無忌又是性格端方之人,從來不説戲謔調笑的言語。這時無忌衝口而出説了一句「我怎麼捨得」,自知失言,不由得臉上一紅,轉過了頭望著窗外。小昭却嘆了口,自去坐在一邉。無忌道︰「你爲什麼嘆氣?」小昭道︰「你眞正捨不得的人多著呢,峨嵋派的周姑娘,汝陽王府的郡主姑娘,將來不知道還有多少,你心中那會掛念著我這個小丫頭。」無忌走到她的面前,説道︰「小昭,你一直待我很好,難道我不知道麼?難道我是個忘恩負義、不知好歹的人嗎?」他説著這兩句話時,聲音極是誠懇。小昭大是害羞,又是喜歡,低下了頭道︰「我又没要你對我怎樣,只須你許我永遠服侍你,做你的小丫頭,我就心滿意足了。你一晩没睡,一定倦了,快上床休息一會吧。」説著掀開被窩,服侍無忌安睡,自去坐在窗下,拈著針線縫衣。無忌聽著她手上的鐵鍊偶而發出輕微的錚錚之聲,只覺心中十分的平掙滿足,過不多時,便合上眼睡著了。

這睡直到傍晩始醒,無忌吃了碗麵,道︰「小昭,我帶你去見趙姑娘,借她的倚天劍斬斷你手脚上的{\upstsl{銬}}鐐。」兩人走到街上,但見蒙古兵卒騎了馬來回奔馳,戒備甚嚴,想是昨晩汝陽王府失火、萬法寺大亂之故。無忌和小昭一聽到馬蹄聲音,便縮身在屋角後面,不讓邏兵見到,不多時便到了那家小酒店中。無忌帶著小昭推門入内,只見趙明已坐在昨晩飲酒的座頭上,笑哈哈的站了起來,説道︰「張公子眞乃信人。」無忌見她神色如常,絲毫不以昨晩之事爲忤,暗想︰「這位姑娘城府眞深,按理説我派人殺了她父親的愛姬,將她費盡心血捉來的六派高手一齊放了,她必定惱怒異常,不料她一如平時,且看她待會如何發作。」只見桌上已擺設了兩副杯筷,無忌欠一欠身,便即就坐,小昭遠遠站著伺候。

無忌抱拳説道︰「趙姑娘,昨晩之事,在下諸多得罪,還祈見諒。」趙明笑道︰「爹爹那韓姬妖妖嬈嬈的,我見了就討厭,多謝你叫人殺了她,我媽媽儘誇讚你聰明呢。」張無忌一怔,説不出話來。趙明又道︰「那些人你救了去也好,反正他們不肯歸降,我留著也是無用。你救了他們,大家一定感激你得緊,當今中原武林,聲望之隆,自是無人再及得上你了。張公子,我敬你一杯!」説著笑盈盈的舉起酒杯。

便在此時,門口人影一晃,走進一個人來,却是苦頭陀。他先向張無忌行了一禮,再恭恭敬敬的向趙明拜了下去,説道︰「郡主,苦頭陀前來向你告辭。」趙明並不還禮,冷冷的道︰「苦大師,你瞞得我好苦。你郡主這個觔斗栽得可不小啊。」苦頭陀站起身來,昂然説道︰「苦頭陀姓范名遙,乃是明教光明右使。朝廷與明教爲敵,本人混入汝陽王府,自是有所而來,多承郡主禮敬有加,今日特來作别。」

趙明仍是冷冷的道︰「你要去便去,又何必如此多禮?」苦頭陀道︰「大丈夫行事光明磊落,自今而後,在下即與郡主爲敵,倘若不明白相告,有負郡主平日相待之意。」趙明向無忌看了一眼,道︰「你到底有什麼本事,能使手下個個對你這般死心塌地?」無忌道︰「咱們是爲國爲民、爲仁俠、爲義氣,范右使和我素不相識,可是一見如故,肝膽相照,只是不枉了兄弟間這個「義」字。」苦頭陀哈哈大笑,道︰「教主這幾句言語,正説出了屬下的心事。教主,你多多保重,這位郡主娘娘心狠手辣,大非尋常,你千萬提防了。」無忌道︰「是,我自是不敢大意。」趙明笑道︰「多謝苦大師稱讚。」

苦頭陀轉身出店,經過小昭身邉時,突然一怔,臉上神色驚愕異常,似乎突然見到什麼可怕之極的鬼魅一般,失聲叫道︰「你\dash{}你\dash{}」小昭奇道︰「怎麼啦?」苦頭陀向她呆望了半晌,搖頭道︰「不是的\dash{}不是的\dash{}我看錯人了。」推門走了出去,一面口中喃喃的道︰「眞像,眞像。」趙明與無忌對望一眼,都不知他説小昭像誰。忽聽得遠處傳來幾下忽哨之聲,三長兩短,聲音極是尖鋭。張無忌一怔,記得這是峨嵋派招聚同門的訊號,當日在西域遇到滅絶師太等一干人時,曾數次聽到她們以此訊號相互聯絡,抵禦明教教衆的來攻,心下甚奇︰「怎地峨嵋派又回到了大都?莫非又遇上了什麼敵人麼?」忽聽趙明道︰「那是峨嵋派門下,似乎遇到了什麼急事,咱們去瞧瞧,好不好?」無忌奇道︰「你怎麼知道?」趙明笑道︰「我在西域率人跟了她們四日四夜,俟機拿人,怎麼會不知道。」無忌道︰「好,咱們便去瞧瞧。趙姑娘,我先求你一件事,要借你的倚天劍一用。」趙明笑道︰「你未借屠龍刀,先向我借倚天劍,算盤倒是精明。」解下腰間繫著的寶劍,遞了過去。

無忌拿在手裡,拔劍出鞘,道︰「小昭,你過來。」小昭走到他的身前,無忌揮動長劍,嗤嗤嗤幾下輕響,小昭手上脚上的{\upstsl{銬}}鍊一齊削斷,嗆{\upstsl{啷}}{\upstsl{啷}}的跌在地下。小昭拜道︰「多謝公子、多謝郡主。」無忌還劍入鞘,交給趙明,只聽得峨嵋派的哨聲更是淒厲,直往東北方去,便道︰「咱行去吧。」趙明摸出一小綻黃金,抛在桌上,閃身便出店門。無忌生怕小昭輕功太淺,跟隨不上,右手拉住她手,左手托在她腰間,不即不離的跟在趙明身後。只奔出十餘丈,便覺小昭的身子輕飄飄的,始終不見落後。雖然無忌此刻並未施展極上乘的輕功,但脚下已是極快,小昭居然能彀跟上,那麼她武功顯然不弱。轉眼之間,趙明已越過幾條僻靜小路,來到一堵半塌的圍牆之外。無忌聽到牆内隱隱有女子爭執的聲音,知道峨嵋派便在其内,拉著小昭的手,越牆而入,黑暗中落地無聲。圍牆内遍地長草,原來是個廢園。趙明跟著進來,三個人便伏在長草之中。

廢園的北隅有個破敗的涼亭,亭中影影綽綽,聚會著十來個人,只聽得一個女子的聲音説道︰「你是本門最年輕的弟子,論資望,説武功,那一門都輪不到你來做本派的掌門\dash{}」無忌一聽這聲音好熟,却是丁敏君的話聲,當下蛇行鼠伏,從長草中低身而前,走到離涼亭數丈之處,這纔停住。此時星光黯淡,瞧出來矇朧一片,但無忌眼光鋭敏,已隱約看清楚亭中有男有女,都是峨嵋派的弟子,除丁敏君外,其餘滅絶師太座下的大弟子均在其中,左首一人身形修長,青衫曳地,正是周芷若。只聽丁敏君的話聲極是嚴峻,不住口的道︰「你説,你説\dash{}」

只聽周芷若緩緩的道︰「丁師姊説的是,小妹是本門最年輕的弟子,不論資歷、武功、才幹、品德,那一項都彀不上做本派的掌門。先師命小妹當此大任,小妹原曾一再苦苦推辭,但先師厲言重責,要小妹發下毒誓,不得有負先師的囑咐。」只聽一個作尼姑裝束的女子道︰「先師西去之時,確有遺言要周師姊繼任本派掌門,這幾句話咱們人人聽到,不但是本派同門,便是少林、武當、崑崙、崆峒諸派英俠,也均可作證。」又有一個中年漢子道︰「先師英明果決,既要用師妹繼任掌門,必有深意。咱們同受先師栽培的大恩,自當遵奉先師遺志,同心輔佐周師妹,以光本派武德。」

丁敏君冷笑道︰「馮師哥説先師必有深意,這『必有深意』四個字,果然是説得好。咱們在高塔之上、高塔之下,不是親耳聽到苦頭陀和鶴筆翁大聲叫嚷麼?周師妹父母是誰,先師爲何對她另眼相看,這還不明白不過麼?」苦頭陀昨日對鹿杖客説,滅絶師太是他的老情人,周芷若是他二人的私生女児,只不過是他邪魔外道的古怪脾氣一時發作,隨口開句玩笑,但鶴筆翁這一公然叫嚷出來,旁人聽在耳裡,雖然未必相信,總不免有幾分疑心,何況這等男女之私人們總是寧信其有,不信其無,而滅絶師太對周芷若如此另眼相自,旁人均是不明所以,「私生女児」這四個字,正是最好的解釋。各人聽了丁敏君這幾句話,一齊默然不語。

周芷若顫聲道︰「丁師姊,你若是不服小妹接任掌門,儘可明白言講。你胡言亂語,敗壞先師畢生清譽,該當何罪?小妹先父姓周,名諱上子下旺,先母薛氏。小妹蒙武當派張眞人之薦,引入先師門下,在此之前,從未見過先師一面。你受先師大恩,今日先師屍骨未寒,便來説這等言語,這\dash{}這\dash{}」説到這裡,聲音已是哽,眼珠滾滾而下,再也説不下去了。丁敏君冷笑道︰「你想在本派掌門,尚未得同門公認,自己身份未明,便想作威作福,分派我的不是,什麼敗壞先師清譽,什麼該當何罪,你想來治我的罪,是不是?我倒要請問︰你既受先師之囑,繼承掌門,便該即日回歸峨嵋,掌管門戸,何以突然不聲不響又回大都?先師逝世,本派事務千頭萬緒,所在均要掌門人分理,你孤身一人回到大都,却是爲何?」

周芷若道︰「先師有一副極重的擔子,交在小妹身上,要小妹務必辦到,是以小妹非回大都不可。」丁敏君道︰「那是甚麼事?此處除了本派同門,並無外人,你儘可明白言講。」周芷若道︰「這是本派最大的機密,除了本派掌門人之外,不得説與旁人得知。」丁敏君冷笑道︰「哼,哼!你什麼都往『掌門人』三個字上一推,須騙我不倒。我來問你︰本派和魔教仇深似海,本派同門,不少喪於魔教之手,魔教教衆死於先師倚天劍下的,更是不計其數。先師所以逝世,便因不肯受那魔教教主一托之故。然則先師屍骨未寒,何以你便悄悄的來尋魔教姓張的小淫賊、那個當教主的大魔頭?」

張無忌躱在長草之中,聽到最後這幾句話,身子不禁一震,便在此時,只覺一根柔膩的手指伸到自己左頰之上,輕輕括了兩下,正是身旁的趙明,用手指替他括羞。無忌滿臉通紅,心想︰「難道周姑娘眞的來找我麼?」只聽周芷若囁囁嚅嚅的道︰「你\dash{}你又胡説八道了\dash{}」丁敏君十分得意,大聲道︰「到這時候你還想抵賴?你叫大夥児先回峨嵋,咱們問你回大都有什麼事,你偏又吞吞吐吐的不肯説。衆同門情知不對,這纔攝在你的後面。你向你父親苦頭陀探問小淫賊的所在,當咱們不知道麼?你去客店找那小淫賊,當咱們不知道麼?」

\chapter{捷立不屈}

她左一句「小淫賊」,右一句「小淫賊」,張無忌脾氣再好,却也不禁著惱,突覺頭頸中有人呵了一口氣,不問可知,那是趙明又在取笑了。只聽丁敏君又道︰「你愛找誰説話,愛跟誰相好,旁人原是管不著。但姓張的小淫賊是本派的生死對頭,昨晩衆人推他爲武林盟主,你既算是本派掌門,何以不出言反對?就算彼衆我寡,反對不了,至少也得聲明一句,我峨嵋派不服,不當他是武林盟主,却爲何你一言不發,一般的歃血爲盟,我瞧你啊,正是打從心中歡喜出來呢。那日在光明頂上,先師叫你刺他一劍,他居然不閃不避,對你眉花眼笑,而你也對他擠眉弄眼,不痛不養的輕輕刺了他一下,這中間若無私弊,有誰相信?」

周芷若哭了出來,説道︰「誰擠眉弄眼了?你儘説些難聽的言語來誣賴人。」

丁敏君冷笑一聲,道︰「我這話難聽,你自己所作所爲,便不怕人説難看了,你的話便好聽了。哼,剛纔你怎麼問那客房中的掌櫃來著?」

\qyh{}勞你的駕,這裡可有一位姓張的客官嗎?」

\qyh{}{\upstsl{嗯}},二十來歳年紀,身材高高的,或者,他不説姓張,另外開個姓氏。」她尖著嗓子,學著周芷若慢吞吞的聲調,説得别特别的妖媚宛轉,靜夜聽來,當眞令人毛骨悚然。

張無忌心下惱怒,暗想這丁敏君乃是峨嵋派中最爲刁鑽刻薄之人,周芷若柔弱仁懦,萬萬不是她的對手,但若自己挺身而出,爲周芷若撐腰,則一來這是峨嵋派本門事務,外人不便置喙,二來只有使周芷若處境更爲不利,眼見周芷若被丁敏君擠逼得絶無分辯餘地,自己却是束手無策。

峨嵋派中本有若干同門,遵從滅絶師太的遺命,奉周芷若爲掌門人,但丁敏君辭鋒咄咄,説得入情入理,各人心中均想︰「先師和魔教結怨太深,周師妹和魔教教主果是干係非同尋常,倘若她將本派賣給了魔教,那便如何是好?」

只聽丁敏君又道︰「周師妹,你是武當派張眞人引入先師門下,那魔教的小淫賊是武當派張五俠之子。這中間到底有什麼古怪陰謀,誰也不知底細。」她大聲説道︰「衆位師兄師姊,先師雖有遺言,命周師妹接任掌門,可是她老人家萬萬料想不到,她圓寂之後,本派的掌門人立即便去尋那魔教教主,相敘私情。此事和本派存亡興衰,関係太大,先師若知今晩之事,她老人家必定另選掌門。先師的遺志,乃是要本派光大發揚,決不是要本派覆滅在魔教之手。依小妹之見,咱們須得繼承先師遺言,請周師妹交出掌門鐵指環,咱們另推一位德才兼備,資望武功足爲同門表率的師姊,出任本派掌門。」她説了這幾句話,同門中已有五六人出言附和。

周芷若道︰「我受先師之命,接任本派掌門,這鐵指環決不能交。我實在不想當這掌門,可是我曾對先師立下重誓,決不能\dash{}決不能有負她老人家的託付。」這幾句話説來半點力量也無,有些本來不作左右袒的同門,聽了也不禁暗暗搖頭。

丁敏君厲聲,道︰「這掌門鐵環,你不交也得交!本派第一條門規,嚴戒欺師滅祖。第二條門規,嚴戒淫邪無恥。你犯了第一、第二兩條大戒,還能掌理峨嵋門戸麼?」

趙明將嘴唇湊到張無忌耳邉,低聲道︰「你的周姑娘要糟啦!你叫我一聲好姊姊,我便出頭去替她解圍。」無忌心中一動,知道這位姑娘足智多謀,必有妙策使周芷若脱困,但她年紀比自己小得多,這一聲「好姊姊」叫起來未免肉麻,實在叫不出口,正自猶豫,趙明又道︰「你不叫也由得你,我可要走啦。」無忌無奈,只得在她耳邉低聲叫道︰「好姊姊!」趙明{\upstsl{噗}}{\upstsl{哧}}一笑,正要長身而起,亭中諸人已然驚覺。丁敏君喝道︰「是誰鬼鬼祟祟的在這児偸聽。」

突然間牆外傳來幾聲咳嗽,一個蒼老的女子聲音,説道︰「黑夜之中,你峨嵋派在這裡鬼鬼祟祟的幹什麼?」幾陣衣襟帶風之聲掠過空際,涼亭外已多了兩個人。這二人對著月光而立,張無忌看得分明,一個是體態龍鍾的老婦,手持拐杖,正是金花婆婆,另一個是身形婀娜的少女,容貌奇醜,却是殷野王之女,無忌的表妹蛛児阿離。

那日韋一笑將蛛児擒去,上光明頂時隨手在山邉一放,轉身再尋時便已不知去向。張無忌自和她分别以來,常自想念,不料此刻忽爾出現,而且又和金花婆婆在一起,無忌心喜之下,幾欲出聲招呼。

只聽得丁敏君已冷冷的道︰「金花婆婆,你來幹什麼?」金花婆婆道︰「你師父在那裡?」丁敏君道︰「先師已於昨日圓寂,你在園外聽了這麼久,却來明知故問。」金花婆婆失聲道︰「啊,滅絶師太已圓寂了!是怎麼死的?爲什麼不等著再見我一面?唉,唉,可惜,可惜\dash{}」一句話没再説得下去,彎了腰不住的咳嗽。蛛児輕輕拍著她背,一面向丁敏君冷笑道︰「誰耐煩來偸聽你們説話?我和婆婆經過這裡,但聽你幾哩咕嚕的説個不停,我認得你的聲音,這纔進來瞧瞧。我婆婆問你,你没聽見麼?你師父是怎樣死的?」丁敏君怒道︰「這干你什麼事?我爲什麼要跟你説?」

金花婆婆舒了口長氣,緩緩的道︰「我生平和人動手,只在你師父手下輸過一次,可是那並不是武功招數不及,只是敵不過倚天劍的鋒利。這幾年來我發願要找一口利刃,再與滅絶師太一較高下。老婆子走遍了天涯海角,總算不枉了這番苦心,一位故人答應借寶刀於我一用,打聽得峨嵋派人衆被朝廷囚禁在萬法寺中,有心要救你師父出來,和她較量一下眞實本領,豈知萬法寺已成一片瓦爍。唉!命中注定,金花婆婆畢生不能再雪此敗之辱,滅絶啊滅絶,你便不能遲死一天半日嗎?」

丁敏君道︰「我師父此刻若是尚在人世,你也不過遭一次挫敗,叫你輸得死心塌\dash{}」突然間拍拍拍拍,四下清脆響聲過去,丁敏君目眩頭暈,幾欲摔倒,臉上已被金花婆婆左右開弓,連擊了四掌。别看這老婆婆病骨支離,咳嗽連連,豈知出手竟是迅捷無倫,又是手法怪異,這四掌打得丁敏君竟無絲毫抗拒躱閃的餘地。

丁敏君驚怒交集,刷的一聲拔出了長劍,指著金花婆婆道︰「你這老乞婆,當眞活得不耐煩了?」金花婆婆似乎根本没聽見她的辱罵,對她手中閃閃發光的利劍也似視而不見,只緩緩的道︰「你師父到底是怎樣死的?」她説話的語音極其蕭索,顯得十分的心灰意懶。丁敏君手中長劍的劍尖雖然距她胸口不過兩尺,終究是不敢便刺了出來,但仍是倔強異常的罵道︰「老乞婆,我爲什麼要跟你説?」金花婆婆長嘆一聲,自言自語的道︰「滅絶師太,你一世英雄,可算是武林中出類拔萃的人物,豈知一旦身故,門下弟子竟是如此不肖,竟無一個像樣的人出來接掌門戸嗎?」

一個身材高大的中年女尼走上一步,合掌説道︰「貧尼靜住,參見婆婆。先師圓逝之時,遺命由周芷若周師妹接任掌門。只是本派之中,尚有若干同門未服。先師既已圓寂,今婆婆難償心願,大數如此,夫復何言?本派掌門未定,不能和婆婆定什麼約會,但峨嵋派乃武林大派,決不能墮了先師的威名。婆婆有什麼吩咐,便請示下,日後本派掌門,自當憑武林規矩,和你作一了斷。但若婆婆自恃前輩,逞強欺人,峨嵋派雖然今遭喪師大難,也唯有和你周旋到底,血濺荒園,有死而已。」這一番話侃侃道來,不亢不卑,連伏在長草中的張無忌和趙明也是聽得爲之暗暗叫好。金花婆婆眼中亮光一閃,説道︰「尊師圓寂之時,已然傳下遺命,派下了繼任的掌門人,那好極了。是那一位?便請一見。」她言語之中,顯然已比對丁敏君説話時客氣得多了。

周芷若上前施了一禮,説道︰「婆婆萬福!峨嵋派第四代掌門人周芷若,問婆婆安好。」丁敏君大聲道︰「也不害羞,便自封爲本派第四代掌門人了。」

蛛児冷笑道︰「這位周姊姊爲人很好,我在西域之時,多承周姊姊的照料。她不配做掌門人,難道你反配麼?你再在我婆婆面前放肆,瞧我不再賞你幾個嘴巴!」

丁敏君大怒,刷的一劍,便向蛛児分心刺來。蛛児一斜身,伸掌便往丁敏君臉上擊去。她這身法手法和金花婆婆一模一樣,但動作之迅捷,却是輸了一籌。丁敏君立即低頭,便躱了開去,但她那一劍却也没能刺中蛛児。

金花婆婆笑道︰「小妮子,我教了多少次,這麼容易一招還没是没學會。瞧仔細了!」右手揮去,順手在丁敏君左頰上一掌,反手在她右頰上一掌,跟著又是順手擊左頰,反手擊右頰,這四掌段落分明,人人都瞧得清清楚楚,但丁敏君只覺全身在一股大力的籠罩之下,四肢竟是動彈不得,給她連打四掌,絶無招架之能。

蛛児笑道︰「婆婆,你這手法我是會的,就没這股内勁。我來試試。」丁敏君身子仍是被金花婆婆逼住了,眼見蛛児的一掌又要打到臉上,氣憤之下,幾欲暈去。

突然間周芷若閃身而上,纖手一伸,架開了蛛児這一掌,説道︰「姊姊且住!」轉頭向金花婆婆道︰「婆婆,適纔我靜住師姊已説得明白,本派同門武藝上雖不及婆婆精湛,却也不容婆婆肆意欺凌。」

金花婆婆笑道︰「這姓丁的女子牙尖齒利,口口聲聲的不服你做掌門,你還來代她出頭麼?」周芷若道︰「本派門戸之事,不與人相干。小女子既受先師之託,雖然本領低微,却也不容外人辱及本派門人。」

金花婆婆笑道︰「好,好,好!」只説得三個「好」字,却已劇烈的咳嗽起來。蛛児遞了一粒藥丸過去,金花婆婆接過服下,喘了一陣氣,突然間雙掌齊出,一掌按在周芷若前胸,一掌按在她的後心,將她身子平平的按在雙掌之間,雙掌著手之處,正是周芷若的致命大穴。她這一招怪異之極,周芷若雖然學武年份不長,究已得了滅絶師太的三分眞傳,不料莫名其妙的便對方制住了前胸後心的要穴,只嚇得花容失色,話也説不出來。

金花婆婆森然道︰「周姑娘,你這掌門人實乃稀鬆平常,難道尊師竟是將峨嵋派掌門的重任,交了給你這麼一個嬌滴滴的小姑娘麼?我瞧你呀,多半是胡吹大氣。」

周芷若定一定心神,尋思︰「她這時手上只須内勁一吐,我心脈立時便被震斷,死於當場。可是我如何能彀墮了師父的威風?」一想到師父,登時勇氣百倍,舉起右手,説道︰「這是峨嵋派掌門鐵環,乃先師親手套在我的手上,豈有虛假?」

金花婆婆一笑,説道︰「要做峨嵋派的掌門,責任非輕,自貴派創派祖師郭襄女俠以降,每一代掌間人肩上都要挑一副重擔,這其中的関鍵,難道尊師也跟你説了麼?我瞧未必。」周芷若道︰「自然跟我説了。」她此言一出口,心頭登時一震︰「她怎麼知道本派的祕密?」金花婆婆道︰「那麼那柄倚天劍呢?」周芷若道︰「這是本派之物,跟你有什麼相干?金花婆婆,我老實跟你説,先師雖然圓寂,峨嵋派並非就此毀了。我落在你的手中,你要殺便殺,若想脅迫我做什麼不應爲之事,那叫休想。本派陥於朝廷奸計,被囚高塔,却有那一個肯降服了?周芷若雖是年輕弱女,既受重任,自知艱巨,早就將生死置之度外。」

張無忌見周芷若胸背要穴倶被金花婆婆按住,生已在呼吸之間,而她兀自如此倔強不服,只怕金花婆婆一怒,立時便傷了她的性命,情急之下,便欲縱出相救。

趙明已知他心意,抓住他的右臂,輕輕一搖,意思説不必忙在一時。只聽金花婆婆哈哈一笑,説道︰「滅絶師太也不算怎麼走眼啊。這個掌門人武功雖弱,性格児倒強。{\upstsl{嗯}},不錯,不錯,武功差的可以練好,江山好改,本性難移。」其實周芷若此早是害怕得六神無主,只是想著師父臨死時的重託,唯有硬著頭皮,捷立不屈,金花婆婆讚她性格堅強,那可將她看錯了。

峨嵋衆同門本來都瞧不起周芷若,但此刻見她不計私嫌,挺身而出的迴護丁敏君,而在強敵的挾持之下,絲毫不失本派威名,心中均各起了對她敬佩之意。靜住長劍一晃,口中幾聲呼哨,峨嵋群弟子倏地散開,各出兵刃,團團將涼亭圍住了。

金花婆婆笑道︰「怎麼樣?」靜住道︰「婆婆劫持峨嵋掌門,意欲何爲?」

金花婆婆咳了幾聲道︰「你們想倚多爲勝?嘿嘿,在我金花婆婆眼下,再多十倍,又有什麼分别?」突然間放開了周芷若,身形晃處,直欺到靜住身前,食中兩指,逕挖她的雙眼。靜住急忙迴劍削她雙臂,只聽得「嘿」的一聲悶哼,身旁已倒了一位同門師妹。原來金花婆婆的手法神異莫測,明攻靜住,左足却已飛出,踢中了一名峨嵋女弟子腰間的穴道。但見她身形在涼亭周遭滴溜溜的轉動,大袖飛舞,間中還傳出幾下咳嗽之聲,峨嵋門人長劍擊刺,竟没一劍能刺中她的衣衫,但男女弟子,却已有七八人被打中穴道倒地。金花婆婆的打穴手法極是毒辣,被打中的都是大聲呼叫,一時廢園之中,淒厲的叫聲此起彼落,聞之心驚。

金花婆婆雙手一拍,回入涼亭,説道︰「周姑娘,你峨嵋派的武功,比之金花婆婆怎麼樣?」周芷若道︰「本派武功當然高於婆婆,當年婆婆敗在先師劍下,難道忘了麼?」金花婆婆怒道︰「滅絶老尼徒仗寶劍之利,那算得什麼?」周芷若道︰「婆婆憑良心説一句,倘若先師和婆婆空手過招,勝負如何?」

金花婆婆沉吟半晌,道︰「不知道。我原想知尊師和我到底誰強誰弱,是以今日到大都來,唉!滅絶師太這一圓逝,武林中少了一位高人。前不見古人,後不見來者,峨嵋派從此衰了。」

那七八名峨嵋弟子的呼號聲啊啊不絶,正似作爲金花婆婆這句話的註脚。靜住等年長弟子用力給他們推拿過血,絲毫不見功效,看來金花婆婆的打穴手法另成一家,非她本人方始解得。張無忌當年治過不少傷在金花婆婆手底的武林健者,知道這個老婆婆下手之毒辣,江湖上罕有能及,有心出去相救,轉念又想︰「這一來幫了周姑娘,却得罪了蛛児。我這位表妹不但對我甚好,而且是骨肉至親,我如何可厚此薄彼?」

只聽金花婆婆道︰「周姑娘,你服了我麼?」周芷若硬著口道︰「本派武功深如大海,不能速成。咱們年歳尚輕,自是不及婆婆,日後進展,却是不可限量。」

金花婆婆笑道︰「妙極妙極!金花婆婆就此告辭。待你日後武功不可限量之時,再來解他們的穴道吧。」説著擕了蛛児之手,轉身便走。周芷若心想這些同門的苦楚,便一時三刻也是難熬,金花婆婆一走,只怕他們痛也痛死了,忙道︰「婆婆慢走。我這幾位同門師姊師兄,還請解救。」金花婆婆道︰「要我相救,那也不難。自今而後,金花婆婆和我這蛛児所到之處,峨嵋門人避道而行。」

周芷若心想︰「我甫任掌門,立時便遇此大敵。倘若答應了此事,峨嵋派那裡還能在武林中立足?這峨嵋一派,豈非就在我手中給毀了?」

金花婆婆見她躊躇不答,笑道︰「你不肯墮了峨嵋派的威名,那也罷了,你將倚天劍借我一用,我就給你解救你的同門。」周芷若道︰「本派師徒陥於朝廷奸計,被囚高塔,這倚天劍寶劍,怎麼還能在咱們手中?」金花婆婆原本也已料到此事,借劍之言,也也不過是萬一的指望,但聽到周芷若如此説,臉上還是掠過一絲失望的神色,突然間厲聲道︰「你要保全峨嵋派聲名,便保不住自己的性命\dash{}」説著從懷中取出一枚丸藥,道︰「這是斷腸裂心的毒藥,你服了下去,我便救人。」

周芷若想起師父的囑咐,柔腸寸斷,尋思︰「師父叫我欺騙張公子,此事我原本幹不了,與其活著受那無窮折磨,還不如一死,一了百了,什麼都不管的乾淨。」當下顫抖著接過毒藥。靜住喝道︰「周師妹,不能吃!」

張無忌見情勢危急,又待躍出阻止,趙明在他耳邉低聲道︰「傻子!假的,不是毒藥。」無忌一怔之間,周芷若已將丸藥送入口中咽下。

靜住等人紛紛呼喝,又要搶上和金花婆婆動手。金花婆婆冷笑道︰「這毒藥麼,藥性一時三刻也不能發作。周姑娘,你跟著我,乖乖的聽話,老婆子一喜歡,説不定便給解藥於你。」説著走到那個被打中穴道的峨嵋門人身畔,在每人身上敲拍數下,那幾人疼痛登止,停了叫喊,只是四肢酸麻,一時仍不能動彈。這幾人眼見周芷若以身試毒,救了自己的苦楚,心中都是十分感激,有人便開言道︰「多謝周師妹!」

金花婆婆拉著周芷若的手,柔聲道︰「乖孩子,你跟著我去,婆婆不會難爲你。」周芷若尚未回答,只覺一股極大的力道拉著自己,身不由主的便騰躍而起。

靜住道︰「周師妹\dash{}」搶上欲待攔阻,斜刺裡一縷指風,勁射而至,却是蛛児從旁發指相襲。靜住左掌揮起一擋,不料蛛児這招乃是虛招,拍的一響,丁敏君臉上已吃了一掌,這「指東打西」正是金花婆婆的武學。但聽得蛛児格格嬌笑,已然掠牆而出。張無忌道︰「快追!」一手拉著趙明,一手拉著小昭,三人同時越牆。靜住等突然見到長草中還躱著三人,無不驚愕。金花婆婆和張無忌的輕功何等高妙,待得峨嵋弟子躍上牆頭,那六人早已没入黑暗之中,不知去向。

張無忌等只追出十餘丈,金花婆婆已然驚覺,脚下絲毫不停,喝問︰「來者是誰?」趙明道︰「留下本派掌門,饒你不死!」低聲向張無忌道︰「你給我掠陣,别現身!」身形一晃。搶上數丈,倚天劍劍尖已指到金花婆婆身後,這一招「金頂佛光」,正是峨嵋派劍法的嫡傳,也虧她聰明過人,竟然在萬法寺中一學之後,使將出來便絲毫不爽。她内勁雖然不足,輕功却已臻上乘,這一招身隨劍去,大具威勢。

金花婆婆聽得背後金刃破風之勢有異,放開了周芷若,急轉身軀。趙明手腕一抖,又是一招「千峰競秀」。金花婆婆識得她手中兵刃正是倚天寶劍,心下又驚又喜,伸手便來搶奪。數招一過,金花婆婆已欺近趙明身前,手指正要搭到她執劍的手腕,不料趙明長劍急轉,使出一招崑崙派的「旋風手」來。

金花婆婆初時見她是個年輕女子,手持倚天劍,使的又是峨嵋嫡系的劍法,自當她是峨嵋弟子。

金花婆婆爲了專心對付滅絶師太,對峨嵋派劍法已鑽研數年,料得趙明功力不過爾爾,這一欺近身,倚天劍定然手到拿來,豈知趙明在危急之中,竟會使出崑崙劍法,這一下金花婆婆武功雖高,可也著了她的道児,急忙著地一滾,方始躱開,但左手衣袖已被劍鋒輕輕帶到,登時削下一大片來。

金花婆婆驚怒之下,欺身再上,趙明知道自己武功可和她差著一大截,不敢和她拆招,只是揮動倚天劍,左刺右劈,東舞西擊,忽而崆峒派劍法,忽而華山派劍法,一招崑崙派的「大漠飛沙」之後,緊跟是一招少林派達摩劍法的「金針渡劫」。每一招均是各派劍法中的精華所在,每一招均具極大威力,再加上倚天劍的鋒鋭,金花婆婆武功雖高,竟是無法逼近她身子的六尺之内。蛛児看得急了,解下腰間長劍,擲給金花婆婆。趙明疾攻七八劍,到第九劍上,金花婆婆不得不用兵刃招架,擦的一聲,長劍斷爲兩截。

金花婆婆臉色大變,倒縱而出,喝道︰「小妮子到底是誰?」趙明笑道︰「你怎地不拔屠龍刀出來?」金花婆婆怒道︰「我若有屠龍刀在手,諒你也非我對手。你敢隨我去一試麼?」

張無忌聽到到提及屠龍刀,心下大奇,只聽趙明道︰「你這老婆子取得到屠龍刀,那倒好了。我只在大都等你,容你去取了刀來再戰。」金花婆婆道︰「你轉過頭來,讓我瞧個分明。」趙明斜過身子,伸出舌頭,左眼閉,右眼開,臉上肌肉扭曲,向她扮了個極怪的鬼臉。金花婆婆大怒,在地下吐了一口唾液,抛下斷劍,擕了蛛児和周芷若快步而去。

張無忌道︰「咱們再追。」趙明道︰「那也不用忙,你跟我來。我包管你的周姑娘安然無恙便是。」無忌道︰「你説什麼屠龍刀?」趙明道︰「我聽這老婆子在廢園中説,她在海外向一位故人借得到了柄寶刀,要和滅絶師太的倚天劍一鬥。『倚天不出,誰與爭鋒?』要和倚天劍爭鋒,拾屠龍刀莫屬,難道她竟向你義父謝老前輩借到了屠龍刀?我適纔仗劍和她相鬥,便是要逼她出刀。可是她手邉又無寶刀,只叫我隨她去一試。似乎她已知屠龍刀的所在,却是無法到手。」無忌沉吟道︰「這倒奇了。」趙明道︰「我料她必去海濱,揚帆出海前去找刀,咱們趕在頭裡,别讓雙眼已盲、心地善良的謝老前輩,受這惡毒的老婆子欺弄。」

張無忌聽了她最後這兩句話,胸口熱血上湧,忙道︰「是,是!」他初時答應趙明去借屠龍刀,只不過是爲了大丈夫千金一諾,不能食言,此刻想到金花婆婆會去和義父爲難,恨不得插翅趕去相救。當下趙明帶著兩人,來到王府之前,向府門前的衛士囑咐了幾句。那衛士連聲答應,回身入内,不久便牽了九匹駿馬,提了一大包金銀出來。趙明等三人騎了三匹馬,讓那六匹馬跟在身後輪流替換,直向東行。

次日清晨,那九匹馬都已疲累不堪,趙明向地方官出示汝陽王調動天下兵馬的金牌,再換了九匹坐騎,當日深夜,已馳抵海邉。

趙明騎馬直入縣城,命縣官急速備好一艘最堅固的大海船,船上舵工、水手、糧食、清水、兵刃、寒衣,一應備齊除此之外所有海船立即驅逐向南,海邉的一百里内,不許另有一艘海船停泊。汝陽王金牌到處,小小的縣官如何敢不奉命唯謹,不到一日,一切均已辦妥。趙明和無忌、小昭三人均換上水手裝束,用油彩抹得臉上黃黃的,再黏上兩撇鼠鬚,更無半點破綻。三人坐在海船之中,專等金花婆婆到來。

這明明郡主料事如神,果然等到傍晩,一輛大車來到海濱,金花婆婆擕著蛛児和周芷若,前來僱船。船上水手早受趙明之囑,諸多推託,直到金花婆婆取出一錠黃金作爲船資,船老大方始勉強答應。金花婆婆等三人一上船,便命揚帆向東。

無邉無際的茫茫大海之中,一葉孤舟,正向東南行駛。

這艘海船船身甚大,船高二層,船頭甲板和左舷右舷均裝有鐵炮,原是蒙古軍的炮船。當年蒙古大軍擬遠征日本,大集舟師,不料一場颶風,將蒙古海軍打得七零八落,東征之舉,歸於泡影,但舟艦的規模,也從那時起遺了下來。這艘大炮船若是泊在岸邉,自是頗顯威武,但到了大海之中,却又成了猶如隨風飄蕩的樹葉一般。

這時張無忌、趙明、小昭三人,化裝了水手,躱在船艙下層。當日趙明一見到這艘船,就知不妙,百密一疏,竟没想到那位縣官加倍巴結,去向水師借了一艘炮船來。臨到上船之時,船中糧食清水均已齊備,而其餘海邉船隻,已遵奉趙明之命,早向南駛出數十里之外。趙明苦笑之下,只有囑咐衆水手在炮口上多掛漁網,在船上裝上幾擔鮮魚,裝作是炮船舊了無用,早已改作了漁船。金花婆婆在海邉到處尋不到船,見有這樣一艘大船,便僱了下來,倒也没瞧出破綻。

其時舟行已有兩日,張無忌和趙明在底艙的窗洞中向外瞧去,只見白天的日頭,晩上的月亮,總是在左舷出現,顯然這船是在逕向南行。其時已是初冬天氣,北風大作,船帆吃飽了風,行駛甚速。無忌已和趙明商量過幾次︰「我那義父是在極北的冰山島上,咱們要去找他,必須北行纔是,怎麼反而南去?」趙明每次總是答道︰「這金花婆婆必定另有古怪。何況這時節没南風,咱們便要北駛,也没法子。」

到得第三日午後,那舵工抽空下艙來向趙明稟報,説道金花婆婆對這一帶海程甚是熟悉,什麼地方有大沙灘,什麼地方有礁石,竟比這舵工還要清楚。張無忌突然心念一動,説道︰「啊,是了!莫非她是要回靈蛇島去?」趙明道︰「什麼靈蛇島?」張無忌道︰「金花婆婆的老家是在靈蛇島啊,她故世的丈夫叫做銀葉先生,靈蛇島金花銀葉,當年威震江湖,難道你没聽説過麼?」趙明{\upstsl{噗}}{\upstsl{哧}}一笑,道︰「你就大得我幾歳,江湖上的事児,倒像是挺内行似的。」張無忌笑道︰「明教的邪魔外道,原比郡主娘娘多知些江湖的閒事。」他二人本是死敵,各統豪傑,打過幾次激烈的硬仗,但在海船的艙底同處數日之後,言笑不禁,又共與金花婆婆爲敵。相互間的隔膜竟是一天少於一天。

那舵工稟報之後,只怕金花婆婆知覺,當即回到掌舵之處。趙明笑道︰「大教主,那就煩你將靈蛇島金花銀葉威震江湖的事跡,説些給我這獨處深宮的小丫頭聽聽。」無忌笑道︰「説來慚愧,銀葉先生是何等樣人,我是一無所知,那位金花婆婆,我却大大的跟她作過一番對。」於是將自己如何於蝴蝶谷中跟「蝶谷醫仙」胡青牛學醫;如何各派人衆被金花婆婆和滅絶師太整得生死不得;如何胡青牛、王難姑夫婦終於又死於金花婆婆手下種種情由,一一向趙明説了。他想胡青牛脾性雖然怪僻,但對自己實在不錯,一想到他夫婦二人的屍體被金花婆婆高高掛在樹上的情景,不由得眼眶紅了。他説這番故事,只是將蛛児要擒自己到靈蛇島去作伴、自己執意不肯、反而將她咬了一口的事略去了不説。爲何要略去此節,自己心中也説不上來,或許怕被趙明聽來頗爲不雅吧?

\chapter{紫衫龍王}

趙明一聲不響的聽完,臉色鄭重,説道︰「張公子,初時我只當這老婆婆只是一位武功極強的高手,原來其中尚有許多恩怨過節,聽你説來,這老婆婆極不好鬥,咱們可千萬大意不得。」張無忌笑道︰「郡主娘娘文武雙全,手下又統率著這許多奇材異能之士,對付區區一個金花婆婆,那也是遊刃有餘了。」趙明笑道︰「就可惜大海之中,没法召喚我手下的衆武士、諸番僧去。」
無忌微微一笑,道︰「這些煮飯的厨子,拉帆的水手,便算不得是江湖上的一流好手,也該算是第二流了吧?」趙明一怔,隨即格格笑了起來,説道︰「佩服,佩服!大教主果然好眼力,須瞞你不過。」

原來趙明回到王府去取馬金之時,暗中已然囑咐衛士,調動了一批下屬,趕到海邉聽由吩咐。這些他是快馬趕程,只比無忌遲到了半天。她所調之人,均未參與萬法寺之戰,從没與無忌朝過相,扮作了厨工、水手之屬。但學武之人,神情舉止自然流露,縱然極力掩飾,張無忌瞧在眼中,心裡早已有數。

趙明聽無忌這麼一説,心中不禁多了一層思量,暗想無忌既然看出,那金花婆婆見多識廣,老奸巨猾,更是早已識破了機関。好在自己人多勢衆,她識破也好,不識破也好,若是動手,她連蛛児在内,終究不過兩人,那也不足爲懼。她既不挑破,自己便不妨假作痴呆。

這幾日之中,無忌最耽心的,便是周芷若服了金花婆婆那顆丸藥後,毒性是否發作,趙明知他心意,見他眉頭一皺,便派人到上艙上去假作送茶送水,察看動靜,每次回報,均説周姑娘言行如常,一無中毒徵狀。這樣幾次之後,無忌也有些不好意思了,靜坐默想之際,又不免想到當日西域雪地中的情境,蛛児如何陪伴自己,如何爲何太沖、武烈、丁敏君等人圍逼之際,尚來與自己見上一面,想到自己曾當著何太沖、武烈等衆人之面,大聲説道︰「姑娘,我誠心願意,娶你爲妻,只盼你别説我不配。」又道︰「從今而後,我會盡力愛護你,照顧你,不論有多少人來跟你爲難,不論有多麼厲害的人來欺侮你,我寧可自己性命不要,也要保護你周全。我要使你心中快樂,忘去了從前的苦處。」

這日他靜坐船艙一角,心中又默念到這幾句話,不禁紅暈上臉。趙明忽道︰「{\upstsl{呸}}!你又想你的周姑娘了!」無忌道︰「没有!」趙明道︰「哼,想就想,不想就不想,難道我管得著麼?男子漢大丈夫,撒什麼謊?」無忌道︰「我幹什麼撒謊?我跟你説,我想的不是周姑娘。」
趙明道︰「你若是想苦頭陀、韋一笑,臉上不會是這樣的神情。那幾個又醜又怪的傢伙,你想到他們之時,會這樣又溫柔,又害臊麼?」無忌不好意思的一笑,道︰「你這人也眞厲害得過了份,别人心裡想的人是俊是醜,你也知道。老實跟你説,我這時候想的人哪,偏偏一點也不好看。」趙明見他説得誠懇,微微一笑,就不再理他,她雖聰明,却也萬萬没料到他所思念的意是船艙上層中那個醜女蛛児。

無忌想到蛛児爲了練那「千蛛絶戸手」的陰毒功夫,弄得容顏凹凸不平,那晩廢園重見,唯覺更損於昔時,言念及此,情不自禁的歎了一口氣。他倒不是惋惜她面容難看,只是覺到她這種邪門功夫越練越深,只怕身子心靈,兩蒙其害。待得想到那日殷利亨説起自己墜崖身亡,蛛児伏地大哭的一番眞情,心下更是感激。他自到光明頂上之後,日日夜夜,不是忙於練功,便是爲明教奔走,幾時能得安安靜靜,想想自己的心事?偶爾雖也記掛著蛛児,也曾命冷謙派人在光明頂四周尋覓,也曾向韋一笑査問,但一直不見蹤跡,此刻見到了蛛児,心下又是深深自責︰「她對我這麼好,可以我對她竟是如此寡情薄義?何以這些時日之中,我竟没將她放在心上?」其實,張無忌做了明教教主之後,他是把自己的私事一槩都抛之腦後了。

趙明忽道︰「你又在懊悔什麼了?」張無忌尚未回答,突聽得船面上傳來一陣{\upstsl{吆}}喝之聲,接著便有水手下來稟道︰「前面已見陸地,老婆子命咱們駛近。」趙明與無忌從窗孔中望將出去,只見數里外是個樹木蔥翠的大島,島的東端奇峰挺拔,聳立著好幾座高山。那船吃飽了風,直駛而前。只一頓飯功夫,已到了島前。那島的東首山石直降入海,並無淺灘,是以那船吃水雖深,却可舶在岸邉。

海船停舶未定,猛聽得山頂傳來一聲長嘯,聲若龍吟,悠悠不絶,雄武威壯,令人聽之精神爲之一振。無忌驀地聽到嘯聲,當眞是驚喜交集,這嘯聲熟悉之極,正是義父金毛獅王謝遜所發。一别十餘年,義父雄風如昔,怎不令他心花怒放?當時也不及細思謝遜如何會從極北的冰火島上來到此處,也顧不得被金花婆婆識破本來面目,急步從木梯走到後梢,向嘯處所發出的山峰上望去,只見四條漢子手執兵刃,正圍著一個身形高大之人在捨死忘生的激鬥。那身形高大之人披著一件灰布長衫,空手而搏,正是金毛獅王謝遜。張無忌一瞥之下,便見義父雙眼雖盲,雖然是以一敵四,雖然是赤手空拳的抵擋四件兵刃,但絲毫不落下風。他從未見過義父施展武功,此刻只瞧了幾招,心下甚喜︰「昔年金毛獅王威震天下,果然是名不虛傳。我義父武功遠在青翼蝠王之上,足可與外公並駕齊驅。」但那四人也是武功了得的高手,那山甚高,從下面望將上去,瞧不明白四人的面目,但見他們個個衣衫襤褸,背上負著若干布袋,看來是丐幫中的子弟長老。旁邉另有三人站著掠陣,似乎倘若這四人支持不住,便即上前相助。只聽一人説道︰「交出屠龍刀\dash{}饒你不死\dash{}寶刀換命\dash{}」山間勁風將他的言語一聲斷斷續續的送將下來,無忌耳音雖靈,但隔得遠了,却也聽不明白。須然只聽得這幾句,已知這一干丐幫人衆,乃是意在劫奪屠龍寶刀。只聽得謝遜哈哈大笑,説道︰「屠龍刀便在我身邉,丐幫的臭賊,有本事便來取去。」他口中説話,手脚上招數半點不緩。

金花婆婆身形一晃,已到了岸上,咳嗽數聲,説道︰「丐幫群俠光降靈蛇島,不來跟老婆子説話,却去騷擾靈蛇島的貴賓,意欲如何?」無忌心道︰「原來這島便是靈蛇島了,聽金花婆婆言中之意,似乎我義父是她請來的客人?我義父當年無論如何不肯離冰火島回歸中原,怎地金花婆婆一請,他便肯來?金花婆婆又怎地知道我義父他老人的所在?」只見山頂上數人一聽山下來了強援,只盼及早拾奪下謝遜,攻得更加緊了。豈知這麼一來,登時犯了武學中的大忌。須知謝遜雙眼已盲,全憑聽取敵人兵刃來路的風聲,以資辨位應敵。這四名丐幫衆出手一快,風聲更響,謝遜長笑一聲,砰的一拳,擊中在一人前胸,那人長聲慘呼,從山頂上直墜下來,拍的一聲巨響,摔得頭蓋破裂,腦漿四濺。旁邉掠陣的三人見情勢不對,其中一人喝道︰「退開!」輕飄飄的一拳擊了出去,這一拳的拳力若有若無,教謝遜無法辨明來路。果然拳力直擊到謝遜身數寸之處,他才知覺,急忙應招,已是手忙脚亂,大爲狼狽。先前打鬥的三人讓身閃開,旁邉掠陣一個老者又加入戰團。此人與先前那人一般的打法,也是輕柔的掌法。數招一過,謝遜左支右絀,迭遇險招。金花婆婆喝道︰「季長老,鄭長老,金毛獅王眼睛不便,你們用這等卑鄙手段,枉爲江湖上的成名英雄。」她一面説,一面撐著拐杖,走上山去。别看她顫巍巍的體態龍鍾,似乎被山風一括,便要摔將下來,那知她身形移動,竟是極快。

但見金花婆婆拐杖在地上一登,身子便乘風而虛般的向前一縱,幾個起落,已到了山腰。蛛児跟隨在後,她武功便不及金花婆婆的精純,但縱躍之際,却也極快,但也看得出她已出全力,不似金花婆婆這等行若無事。張無忌掛念著義父安危,也大跨步登山。趙明跟著上來,低聲道︰「有這老婆子在,獅王無險,你不必出手,隱藏形跡要緊。」無忌點了點頭,反手挽著她上,緊緊跟隨在蛛児身後。這時只看到蛛児婀娜苗條的背影,若是不瞧她的面目,何嘗不是個絶色美女,何嘗輸與趙明、周芷若、小昭三人。他心念一動之下,隨即自責︰「張無忌啊張無忌,你義父身處兇險,這當口你却去瞧人家姑娘,心中品評她相貌身材,美是不美?」其實張無忌既見金花婆婆上山相救,知道這位老婆婆武功高極,義父已無危險,他此時已二十二歳,當年和蛛児曾有婚嫁之約,雖然作得不準,但青年男児,偶興求偶之念,那也是人情之常,不足深責。

四個人片刻間到了山巓。只見謝遜雙手出招極短,緊緊守住門戸,全是防禦的打法,只等敵人的拳脚攻近,這纔以小擒拿手拆解。這般打法一時可保無虞,但要擊敵取勝,却也不能。張無忌站在一棵大松樹之下,眼見義父滿臉皺紋,頭髮已然白多黑少,與當日分手之時,已是蒼老了甚多,想是這十多年來獨處荒島,日子過得甚是艱辛,心下不由得甚是難過,胸口一陣激動,忍不住便要代他打發敵人,撲上前去想認。趙明知他心意,捏一捏他的手掌,搖了搖頭。

只聽金花婆婆説道︰「季長老,你的『陰出掌大九式』馳譽江湖,何必鬼鬼祟祟,變作綿掌的招式?鄭長老更加不成話了,你將『迴風拂柳拳』暗藏在八卦拳中,難道金毛獅王謝大俠便不知道了\dash{}咳咳\dash{}昔年丐幫是江湖上第一行俠仗義的大幫會,唉,近年來每況愈下,越來越不成話了\dash{}咳咳\dash{}」謝遜瞧不見敵人的招式,對敵時十分吃虧,加之那季鄭二長老十分狡獪,出招時故意變式,便謝遜捉摸不定。金花婆婆這一點破,謝遜已然胸有成竹,乘著鄭長老拳法變不變之際,呼的一拳擊出,正好和鄭長老一拳相抵。這一拳威力奇大,幸好鄭長老武功也強,但還是退了兩步,方得拿定樁了。季長老從旁揮掌相護,使謝遜無暇追擊。

張無忌瞧這丐幫二老時,只見那季長老矮矮胖胖、滿臉紅光,倒似個肉莊屠夫,那鄭長老却憔悴枯瘦,面有菜色,纔不折不扣似個丐幫人物。遠處站著個三十歳上下的青年,也是穿著丐幫的服色,但衣衫漿洗得乾乾淨淨,背上竟也負著八個布袋,以他這等年紀,居然做到丐幫中的八袋長老,那也是極爲罕有之事。無忌瞧了兩眼,只覺此人相貌好熟,似在何處見過,一時却想不起來。忽聽那人説道︰「金花婆婆,你明著不助謝遜,這口頭相助,難道不算麼?」金花婆婆冷冷的道︰「閣下也是丐幫中的長老麼?恕老婆子眼拙,倒没會過。」那人笑道︰「在下新任長老不久,婆婆自是不識。在下姓陳,草字友諒。」無忌一聽他自報姓名,登時記起,心道︰「陳友諒,是了!那日太師父帶我往少林寺求醫,有一少年過目不忘,將太師父手錄的『武當九陽功』背得一字不漏,便是此人了。但他是少林子弟,怎地當起丐幫的長老來了?{\upstsl{嗯}},丐幫之中,各門各派的子弟均有,少林子弟授入丐幫,也不足奇。他聰明迥人,若是習得少林派的上乘武功,一進丐幫,自能出人頭地。何況他尚且偸習我太師父的武當九陽功。身兼武當少林兩派之所長,何愁不在丐幫中身居高位。」

金花婆婆厲聲道︰「武當派門下的弟子,也投進了丐幫麼?」張無忌從陳友諒朗聲對答、調勻氣息的内功之中,原已聽出他已頗得武當派内功的心法,聽金花婆婆這麼一叫,心下暗怒︰「這人偸學了我太師父的『武當派九陽功』心法,竟然暗自修練,好一丟臉!」對金花婆婆耳音之敏鋭,不禁甚是佩服。只聽陳友諒笑道︰「在下出身少林,這位老婆婆強換在下門派,好笑啊好笑!」他説這幾句話時,吐氣剛猛,確是九陽功的法門。張無忌於少林、武當兩派的九陽功都曾學過,一聽之下,心想此人兼習兩派内功,各有所成,實是才智過人。驀聽得{\upstsl{吆}}喝之聲大作,鄭長老的左臂又中了謝遜一拳,本來在旁觀鬥的三名丐子弟,又挺兵刃上前圍攻。這三人武功不及季鄭二長老,本來反而礙手礙脚,但謝遜雙目已盲,而且他目盲之後從未和人動手過招,絶無臨敵經驗,今日初逢強敵,全憑聽風辨聲,敵人在拳脚之中再加上兵刃,那就極難辨别方向,片刻之間,肩頭已中了一刀。無忌見情勢危急,正要出手,趙明低聲道︰「金花婆婆豈能不救?」無忌略一遲疑,只見金花婆婆仍是挂著拐杖,微微冷笑,並不上前相援,便在此時,謝遜左腿又被鄭長老踢中了一脚。這一脚力道極其強勁,謝遜一個踉蹌,險些児摔倒。五名丐幫人衆大喜,同時撲上,張無忌手中早已扣了七粒小石子,右手一振,七粒石子分擊五人。這七粒石子還未打到五人身上,猛見黑光一閃,嗤的一聲響,三件兵刃登時削斷,五個人中有四人被齊斬斷,分爲八截,一齊摔下山麓,只有鄭長老斷了一條右臂,跌倒在地,背心上還嵌了張無忌所發的兩粒石子。那四個被斬之人的身上,也均嵌了石子,只是刀斬在先,石子打中在後,無忌這一下出手,倒是變成多餘的了。

這一下變故來得快極,衆人無不心驚。但見謝遜手中提著一柄黑沉沉的大刀,正是號稱「武林至尊」的屠龍寶刀。他橫刀站在山巓,威風凜凜,宛如天神一般。張無忌自幼便見到這柄大刀,却没想到其鋒鋭威猛,竟至如斯。金花婆婆喃喃道︰「武林至尊,寶刀屠龍!武林至尊,寶刀屠龍!」那鄭長老一臂被斬,痛得殺豬似的大叫。陳友諒臉色慘白,朗聲道︰「謝大俠武功蓋世,佩服佩服。這位鄭長老請你下放下山去,在下抵他一命便是,便請謝大俠動手!」此言一出,衆人群相動容,没料到此人倒是個義氣深重的漢子。江湖上最講究的便是一個「義」字,張無忌本來甚是瞧他不起,此刻倒是好生敬重。謝遜道︰「陳友諒,{\upstsl{嗯}},陳友諒,你倒是條好漢,將這姓鄭的抱了去吧,我也不來難爲於你!」陳友諒道︰「在下先行謝過謝大俠不殺之恩,只是丐幫已有五人命喪謝大俠之手,在下十年之内若是習武有成,再來了斷今日的恩仇。」謝遜聽他在此兇險之極的境地下,居然説出日後尋仇的話來,自己只須踏上一步,寶刀一揮,此人萬難逃過,但仍是絲毫不懼,可算得是武林中極有膽色的人物,當下説道︰「老夫若再活得十年,自當領教閣下少林、武當兩派兼修的神功。」陳友諒抱拳向金花婆婆行了一禮,説道︰「丐幫擅闖貴島,這裡謝罪了!」抱起鄭長老,大踏步走下山去。

金花婆婆向張無忌瞪了一眼,冷冷的道︰「你這小老児好準的打穴手法啊。你爲何手中扣了七粒石子?本想一粒打陳友諒,一粒便來打我是不是?」張無忌見她識破了自己扣著七石的原意,却没識破自己本來的面目,當下便不回答,只是微微一笑。金花婆婆厲聲道︰「小老児,你尊姓大名啊?假扮水手,一路跟著我老婆婆,却是爲何?在金花婆婆面前弄鬼,你還要性命不要?」

張無忌不擅撒謊,一怔之下,竟然答不上來。趙明放粗了嗓子道︰「咱們巨鯨幫,向在大海之上,做的是没本錢的買賣。老婆婆出的金子多,便送你一趟,又待如何?這位兄弟瞧著丐幫恃多欺人,出手相援,原是好意,没料到謝大俠武功如此了得,倒顯得咱們多事了。」她學的雖是男子聲調,但仍不免尖聲尖氣,聽來十分刺耳。只是她化裝精妙,活脱是個黃皮精瘦的老児,金花婆婆倒也没瞧出破綻。謝遜左手一揮,道︰「多謝了!你們去吧。唉,金毛獅王虎落平陽,今日反要巨鯨幫相助。一别江湖二十載,武林中能人輩出,我何必再回來?」説到最後這幾句話時,語調中充滿了意氣消沉,感慨傷懷之情。原來張無忌手發七石,勁力之強,世所罕有,謝遜聽得清清楚楚,既震驚武林有這等高手,又自傷今日全仗屠龍刀之助,方得脱困於宵小的圍攻,回思二十餘年前王盤山氣懾群豪的雄風,當眞是如同隔世了。金花婆婆道︰「謝賢弟,我知你不喜旁人相助,是以没有出手,你没見怪吧?」張無忌聽金花婆婆竟然稱他義父爲「賢弟」,心中微覺詫異,只聽謝遜道︰「有什麼見怪不見怪的?你這次回去中原,探聽到了我那無忌孩児什麼訊息?」無忌心頭一震,只覺一隻柔軟的手掌伸了過來緊緊的握住他手,知道趙明不欲自己於此刻上前相認,適才自己没聽趙明的話,貿然發出石子相援,已然做得冒昧,只是関切太過,不敢輕易冒險,此刻忍得一時,却無関礙,只聽金花婆婆道︰「没有!」謝遜長嘆一聲,隔了半晌,纔道︰「韓夫人,咱們兄弟一場,你今日可不能騙我瞎子,我那無忌孩児,當眞還活在世上麼?」

金花婆婆遲疑未答,蛛児突然説道︰「謝大俠\dash{}」金花婆婆左手伸出,緊緊扣住她手腕,瞪眼相視,蛛児便不敢再説下去了。謝遜道︰「殷姑娘,你説,你説!你婆婆在騙我,是不是?」蛛児兩行眼泪,從臉頰上流了下來,金花婆婆右掌舉起,放在她的頭頂,只須蛛児一言説得不合她心意,内力一吐,立時便取了她性命。蛛児答道︰「謝大俠,我婆婆没騙你。這一次咱們去中原,没打聽到張無忌的訊息。」金花婆婆聽她這麼説,臉上掠過了一絲笑意,將右掌提起,離開了她的腦門,但左手仍是扣著她的手腕。謝遜道︰「那麼你們打聽了什麼消息?我明教怎樣了?咱們這些故人怎麼樣?」金花婆婆道︰「不知道。江湖上的事,我没去打聽。我是要先去找峨嵋派的滅絶老尼,報那一劍之仇,其餘的事,老婆子也没放在心上。」

謝遜怒道︰「好啊,韓夫人,那日你在冰火島上,是對我怎樣説來?你説我那張五弟夫婦在武當山上雙雙自刎,我那無忌孩児成爲一個没人照料的孤児,流落江湖,到處被人欺凌,慘不堪言,是也不是?」金花婆婆道︰「不錯!」謝遜又道︰「他説他被人打中了一掌玄冥神掌,日夜苦受熬煎,你在蝴蝶谷中曾親眼見他,要他到靈蛇島來,他却執意不肯,是也不是?」金花婆婆道︰「不錯!我若騙了你,天誅地滅,金花婆婆比江湖上的下三濫還要不如。」謝遜道︰「殷姑娘,你又怎麼説來?」蛛児道︰「我説當時我苦勸他來靈蛇島,他非但不聽,反而咬了我一口。我手背上齒痕猶在,決非假話。」趙明抓著張無忌的手掌忽地緊了一緊,雙目凝視著他,眼中流露出又是取笑,又是怨懟的神色,意思是説︰「好啊,你騙得我好苦,原來這個姑娘識得你在先,你們中間還存著許多糾葛過節。」無忌臉上一紅,想起表妹殷離(即蛛児)對自己的一番古怪情意,心中又是甜蜜又是酸苦。突然之間,趙明抓起無忌的手來,放在口邉,在他手背上狠狠的咬了一口。這一口咬下,無忌手背上登時鮮血迸流,體内的九陽神功自然而然生出抵禦之力,一彈之下,將趙明的嘴角都震破了,也流出血來。但兩人都忍住了不叫出聲。無忌眼望趙明,不知她爲何突然咬自己一口,却見她眼中滿是笑意,臉上暈紅流霞,麗色生春,雖然口唇上黏著兩撇假鬚,仍是不掩甚嬌美絶艷。

張無忌滿腹狐疑,只聽謝遜又道︰「好啊!韓夫人,我只因掛念我那無忌孩児孤苦,這纔萬里迢迢的離了冰火島重回中原。你答應我去探訪無忌,却何必不守諾言?」張無忌眼中的泪水滾來滾去,此時才知義父明知遍地仇家,仍是不避兇險的回到中原,全是爲了自己。只聽金花婆婆道︰「當日咱們怎生説來?我跟你尋訪張無忌,你便借屠龍刀給我。謝賢弟,你借刀於我,老婆子言出如山,自當爲你探訪這少年的確實音訊。」謝遜搖頭道︰「你先將無忌領來,我自然借刀與你。」金花婆婆冷冷的道︰「你信不過我麼?」謝遜道︰「世上之事,難説得很,親如父子兄弟,也有信不過的時候。」無忌知他想起了成崑的往事,心中又是一陣難過。

金花婆婆道︰「那你定是不肯先借刀的了?」謝遜道︰「我放了丐幫的陳友諒下山,從此靈蛇島上再無寧日,不知武林中將有多少仇家前來跟我爲難。金毛獅王早已非復當年,除了這柄屠龍刀外,再也無可倚仗,嘿嘿\dash{}」他突然冷笑數聲,道︰「韓夫人,適纔五人圍攻兄弟,連那位巨鯨幫的好漢也知手中扣上七枚石子,難道你心中不是存著加害於我之意麼?那人會疑心於你,難道我不會疑心麼?你是盼望我命喪丐幫手底,然後你再來撿這現便宜。謝遜眼睛雖瞎,這心可没有瞎。韓夫人,我再問你一句,謝遜到你靈蛇島來,此事十分隱祕,何以丐幫却知道了?」金花婆婆道︰「我正要好好的査個明白。」謝遜伸手在屠龍刀上一彈,放入長袍之内,説道︰「你不肯爲我探訪無忌,那也由你。謝遜唯有重入江湖,再鬧了一個天翻地覆。」説罷仰天一聲清嘯,縱身而起,從西邉山坡上走了下去。但見他行走極是迅捷,越走越遠,直向島北的一座山峰走去。那山頂上孤零零的蓋著一所茅屋,想是他便住在那裡。

金花婆婆等謝遜走遠,回頭向張無忌和趙明瞪了一眼,喝道︰「滾下去!」趙明拉著無忌的手,當即下山,回到船中。無忌道︰「我要瞧義父去。」趙明道︰「當你義父離去之時,金花婆婆目露兇光,你没瞧見麼?」無忌道︰「我也無懼於她。」趙明道︰「我瞧這島中藏著許多詭祕之事。丐幫人衆何以會到靈蛇島來?金花婆婆如何得知你義父的所在?如何能找到冰火島去?這中間實有許多不解之處。你去將金花婆婆一掌打死,原也不難,可是那就什麼也不明白了。」無忌道︰「我也不想將金花婆婆打死,只是義父想得我好苦,我要快去見他。」趙明搖頭道︰「别了十多年啦,也不爭再等一兩天,張公子,我跟你説,咱們固然防金花婆婆,可是更得防那陳友諒。」無忌道︰「那陳友諒麼?此人很重義氣,倒是條漢子。」趙明道︰「你心中眞是這麼想?没騙我麼?」無忌奇道︰「騙你什麼?這陳友諒甘心代鄭長老一死,豈不是十分難得?」趙明一雙妙目凝視著無忌,嘆了口氣,道︰「張公子啊張公子,你是明教教主,要統率多少傑傲不馴的英雄豪傑,如此容易受人之欺,那如何得了?」無忌奇道︰「受人之欺?」趙明道︰「這陳友諒明明在欺騙謝大俠,你眼睛瞧得清清楚楚,怎地會看不出來?」無忌跳了起來,道︰「他在騙我義父?」

趙明道︰「當時謝大俠屠龍刀一揮之下,丐幫高手四死一傷,那陳友諒武功再高,也未必能逃得出屠龍刀刃鋒一割。處此佳境,不是上前拚命送死,便是跪地求饒,可是你想,謝大俠不願自己行蹤被人知曉,陳友諒再磕三百個響頭,未必能哀求得謝大俠心軟,除了假裝仁俠重義,難道還有很好的法子?」她一面説,一面在張無忌的手背上的傷口上敷了一層藥膏,用自己的手帕替他包紮。無忌聽他解釋陳友諒的處境,果是一點不錯,可是回想當時陳友諒慷慨陳辭,語氣中實無半點虛假,仍是將信將疑。

趙明又道︰「好,我再問你一句話︰那陳友諒對謝大俠説這幾句話之時,他雙手怎樣,兩隻脚怎樣?」無忌那時聽著陳友諒説話,時而瞧瞧他臉,時而瞧瞧義父的臉色,没留神陳友諒雙手雙脚如何,但他全身姿勢,其實均已瞧在眼中,旁人不提,他也就忽略了,所謂「視而不見」,便是此意。此刻聽趙明一問,當時的情景,便重新映入腦海之中,説道︰「嘿,那陳友諒右手略舉,左手橫擺,那是武當拳法的一招『獅子搏兔』他兩隻脚麼?{\upstsl{嗯}},是了,這是少林拳中的一招『降魔踢斗式』。難道他口中假裝向我義父求情,其實是意欲偸襲麼?那可不對啊,這兩下招式不管用。」趙明冷笑道︰「張公子,你於世上的人心鬼蜮,可眞明白太少。諒那陳友諒有多大武功,他向謝大俠偸襲,焉能得手?此人聰明機警,乃是第一等的人才,定當有自知之明。倘若他假裝義氣深重的鬼技倆給謝大俠識破了,不肯饒他性命,依他當時所站的位置,他一招『降魔踢斗式』踢的是誰?那一招『獅子搏兔』搏的是那一個?」

張無忌並非呆鈍愚魯之人,只不過對人處處往好的一端去想,以致没去深思陳友諒的詭計,趙明這麼一提,他腦海中一閃,背脊上竟是微微出了一陣冷汗,顫聲道︰「他\dash{}他這一脚踢的是躺在地下的鄭長老,手下抓中的是殷姑娘。」趙明嫣然一笑,道︰「對啦!他一脚踢起鄭長老往謝大俠身前飛去,再抓著那位跟你青梅竹馬、結下嚙手之盟的殷姑娘,往謝大俠身前一推,這麼緩得一緩,他便有機可乘,或能逃得性命。雖然謝大俠神威蓋世,此計未必得售,但除此之外,更無别法。倘若是我,所作所爲自當跟他一模一樣。我直到現下,仍是想不出更好的法子。此人在頃刻之間,機變如此,當眞是了不起的人物。」説著不禁連連讚嘆。張無忌越想越是寒心,世上人心險詐,他自小便經歷得多了,但像陳友諒那樣厲害,倒也少見,過了半晌,説道︰「趙姑娘,你一眼便識破他的機関,只怕比他更是了得。」

趙明臉一沉,道︰「你是譏刺我麼?張公子,我跟你説,你如怕我心地險惡,不如遠遠的避開我爲妙。」無忌笑道︰「那也不必。你對我所使的詭計已多,我事事會防著些児。」趙明微微一笑,道︰「你防得了麼?怎麼你手背上給我下了毒藥,也不知道呢?」無忌一驚,果覺傷口中微覺麻癢,頗有異狀,急忙撕下手帕,伸手背到鼻端一嗅,只聞到一陣甜甜的香氣,不禁叫道︰「啊喲!」知道那是「去腐消肌膏」,原是外科中用作爛去腐肉的消蝕藥膏,給她塗在手背之上,雖然不是什麼了不起的毒藥,但給她牙齒咬出的齒痕,却是爛得更加深了,急忙奔到船尾倒些清水來擦洗個乾淨,趙明跟在他身後,笑吟吟的助他擦洗。無忌在她肩頭上一推,惱道︰「你别走近我,這般惡作劇幹麼?難道人家不痛麼?」那「去腐消肌膏」本身有一種特異的甜香氣息,但趙明在其中調了些自己所用的胭脂,再用自己的手帕給他包紮,教無忌不致發覺。

\chapter{安排毒計}

趙明被他一推,格格笑了起來,説道︰「當眞是狗咬呂洞賓,不識好人心。我是怕你痛得厲害,才用這個法子。」無忌不去理她,氣憤憤的自行回到船艙,閉上了眼睛。趙明跟了進來,叫道︰「張公子!」無忌假裝睡著,趙明叫了兩聲,無忌索性打起呼來。趙明道︰「早知如此,我索性塗上毒藥,取了你的狗命,勝於被你不理不睬。」無忌睜開眼來,道︰「我怎地是狗咬呂洞賓,不識好人心了,你且説説。」趙明笑道︰「我若是説得你信服,你便如何?」無忌道︰「你慣會強辭奪理,我自然辯你不過。」趙明笑道︰「你還没聽我説,心下早已虛了,早知道我是對你一番好意。」

無忌「{\upstsl{呸}}」了一聲道︰「天下有這等好意!傷了我的手背,不來陪個不是,那也罷了,再跟我塗上些毒藥,我寧可少受你些這等好意。」趙明道︰「{\upstsl{嗯}},張無忌,我且問你︰我咬你這口深呢,還是你咬殷姑娘這口深?」無忌臉上一紅,道︰「那\dash{}那是以前的事了,你提它幹麼?」趙明道︰「我偏要提。我要問你,你别顧左右而言他。」無忌道︰「就算是我咬殷姑娘這口深。可那時候她抓住了我,我當時武功不及她,怎麼也擺脱不了,小孩子心中急起上來,只好咬人。你又不是小孩子,我又没有抓住你,要你到靈蛇島來?」趙明笑道︰「這就奇怪了。當時她抓住了你,要你到靈蛇島來,你死也不肯來,怎地現下人家没請你,你却又巴巴的跟了來?究竟是人大心大,什麼也變了。」無忌臉上又是一紅,笑道︰「這是你叫我來的!」趙明聽了這話,臉上也紅了,心中感到一陣甜意,無忌那句話似乎是説︰「她叫我來,我是死也不肯來。你叫我來,我便來了。」

兩人半晌不語,眼光一相對,急忙都避了開去。趙明低下了頭,輕聲道︰「好吧!我跟你説,當年你咬了這殷姑娘一口,她隔了這麼久還是念念不忘於你,我聽她説話的口氣啊,只怕一輩子也忘不了。我也咬你一口,也要叫你一輩子也忘不了我。」無忌聽到這裡,這纔明白她的深意,心中感動,却説不出話來。趙明又道︰「我瞧她手背上的傷痕,你這一咬得很深。我想你咬得深,她也記得深。要是我也像你這般,重重的咬你一口,却狠不了這個心,咬得輕了,只怕你將來忘了我。左思右想,只好先咬你一下,再塗些『去腐消肌散』,把那牙齒印児爛得深些。」無忌先覺好笑,隨即忌到她此舉雖然異想天開,究竟是對自己一番深情,嘆了口氣道︰「我不怪你了。算我是狗咬呂洞賓,不識好人心。其實,你待我如此,用不著這麼,我也決不會忘。」

趙明本來柔情無限,一聽此言,眼中又露出狡獪頑皮之意,笑道︰「你『待我如此』,是説我待你不好呢,還是如此好?張公子,我待你不好的事情很多,待你好的,却是没有一件。」張無忌道︰「以後你多待我好一些,那就成了。」握住她的左手,放到自己口邉,笑道︰「我也來狠狠咬上一口,教你一輩子也忘不了我。」趙明突然一陣嬌羞,撤脱了他手,奔出艙去,一開艙門,險險與小昭撞了個滿懷。趙明吃了一驚,暗想︰「糟糕!我跟他這些言語,莫要都被小丫頭聽去啦,那可羞死人了!」不由得滿臉通紅,奔到了甲板之上。

小昭走到無忌身前,説道︰「公子,我瞧見金花婆婆和那位醜姑娘從那邉走過,每個人都負著一隻大袋子,不知在搗什麼鬼。」無忌{\upstsl{嗯}}了一聲,他適纔和趙明説笑,漸涉於私,突然見到小昭,不免有些羞慚,楞了一楞,纔道︰「是不是走向島北那山上的小屋?」小昭道︰「不是,她二人走向東北,似乎在爭辯什麼。那金花婆婆好似很生氣的樣子。」

張無忌走到船尾,遙遙瞧見趙明俏立船頭,眼望大海,只是不轉過身來,但聽得海中波濤,忽喇急喇的打在船邉。無忌心中,也是如潮水起伏,難以平靜。良久良久,只見太陽從西邉海波中没了下去,島上樹木山峰,慢慢的陰暗朦朧,這纔回進船艙。

無忌用過晩飯,向趙明和小昭道︰「我去探探義父去,你們守在船裡吧,免得人多了被金花婆婆驚覺。」趙明道︰「那你索性再等一個更次,待天色全黑了再去。」無忌道︰「那也説得是。」他一心只長{\upstsl{惦}}記著義父,這一個更次,著實難熬。好容易等得四下裡一片漆黑,張無忌站起身來,向趙明和小昭微微一笑走向艙門。趙明解下腰間倚天劍,道︰「張公子,你帶了此劍防身。」無忌一怔,道︰「你帶著的好。」趙明道︰「不!你此去我有點児擔心。」無忌笑道︰「擔心什麼?」趙明道︰「我也説不上來。金花婆婆詭祕難測,陳友諒鬼計多端,又不知你義父是否相信你就是他那『無忌孩児』\dash{}唉,此島號稱『靈蛇』,説不定島上有什麼厲害的毒物,更何況\dash{}」她説到這裡,住口不説了。無忌道︰「更何況什麼?」趙明舉起自己手來,在口唇邉作個一咬的姿勢,嘻嘻一笑,自己臉却紅了。張無忌知她説的是他表妹殷離,擺了擺手,躍上岸去。趙明叫道︰「接住了!」將倚天劍擲了過來。無忌抄手接住劍柄,心頭又是一熱︰「她對我這等放心,竟連倚天劍也借了給我。」

無忌將劍插在背後,提氣便往島北那山峰奔去。他記著趙明的語語,生怕草中藏有怪蟲毒物,是以只往光禿禿的山石上落脚。不到一頓飯功夫,已奔到那山峰脚下,他抬頭一望,見峰頂那茅屋黑沉沉的,並無燈火,心想︰「義父已安睡了麼?」但隨即想起︰「他老人家雙目已盲,要燈火何用?」便在此時,隱隱聽得左首山腰中傳來幾下説話的聲音。無忌伏底身子,尋聲而往,那聲音却又聽不見了。這時一陣朔風自北吹來,{\upstsl{颳}}得草木獵獵作響,無忌乘著風聲,快步疾進,風聲未歇,只聽得前面四五丈外,一個人壓低著嗓子説道︰「你還不動手,在一旁延延挨挨的搗什麼鬼?」正是金花婆婆的聲音。答話的便是殷離,她道︰「婆婆,你這麼幹,未免太對不起老朋友。謝大俠跟你數十年的交情,他信得過你,纔從冰火島回歸中原。」金花婆婆冷笑道︰「他信得過我?眞是笑話奇談了。他倘若眞是信得過我,幹麼不肯借刀於我。他回歸中原,只是要找尋他的義子,跟我有什麼相干?」張無忌聽了二人的對答,知道金花婆婆在安排什麼毒計,意欲謀害義父,奪取寶刀,當下又向前欺進數丈。黑暗之中,依稀見到金花婆婆佝僂著身子,忽然叮的一聲輕響,她身前發出一下金鐵和山石撞擊之聲,過了一會,又是這麼一響。

無忌大奇,但生怕被二人發覺,不敢再行上前瞧個明白。只聽殷離道︰「婆婆,你要奪他寶刀,明刀明搶的交戰,尚不失爲英雄行逕。靈蛇島金花銀葉,威震江湖,這等事若是傳揚出去,豈不爲天下好漢恥笑?就算奪得屠龍刀來,勝了峨嵋派的女弟子,也没什麼光彩!」金花婆婆大怒,伸直了身子,厲聲道︰「小丫頭,當年是誰在你父親掌底救了你的小命?現下人大了,説不聽婆婆的吩咐!這謝遜跟你非親非故,何以要你一鼓勁児的護著他?你倒説個道理給婆婆聽聽。」她語聲雖然嚴峻,嗓聲却低,似乎只怕被峰頂的謝遜聽到了,其實峰頂和此處相距極遠,只要不是以内力傳送,便是高聲呼喊,也未必能彀聽到。殷離將手中拿著的一袋物事往地下一摔,嗆{\upstsl{啷}}{\upstsl{啷}}一陣響亮,她自己跟著退開了三步。

金花婆婆厲聲道︰「怎樣?你羽毛豐了,自己便想飛了,是不是?」張無忌雖在黑暗之中,仍可見到她晶亮的目光如冷電般威勢迫人。殷離道︰「婆婆,我決不敢忘你救我性命,教我武藝的大恩。可是謝大俠是他\dash{}是他的義父啊。」金花婆婆哈哈一聲乾笑,説道︰「天下竟有你這等痴丫頭,那姓張的小子摔在西域萬丈深谷之中,那是你親耳聽到武烈、武青嬰他們説的。你不死心,硬生生將他們擄了來,詳加拷問,難道這中間還有假麼?這會児那姓張的小子屍骨都化了灰啦,你還念念不忘於他。」殷離道︰「婆婆,我心中可就撇不了他,也許,這就是你説的什麼\dash{}什麼前世的冤孼。」金花婆婆嘆了口氣,語氣大轉溫和,説道︰「别説當年這孩子不肯跟咱們到靈蛇島來,就算跟你成了夫妻,他死也死了,又待怎地?幸虧他死得早,要是這當口還不死啊,見到你這生模樣,怎能愛你?你眼睜睜的瞧著他愛上别個女子,心中怎樣?」

殷離默然不語,顯是無言可答。金花婆婆又道︰「别説旁人,單是咱們擒來的那個峨嵋周姑娘,那般花容月貌,那姓張的小子非動心不可,你殺了周姑娘呢,還是殺那小子?哼哼,你倘若不練這千蛛絶戸手,原是個絶色佳人,現在啊,什麼都完啦。」殷離道︰「他人早死了,我相貌也毀了,還有什麼可説的。可是謝大俠既是他義父,婆婆,咱們便不能動他一根毫毛。婆婆,我只求這件事,另外我什麼也聽你的話。」説著雙膝一曲,跪倒在地。原來她二人遠赴冰火島接回謝遜,途中耽擱了將及一年,以後重入江湖,又是誰也没來往,因之張無忌新任明教教主之事,雖然轟傳武林,金花婆婆和殷離却是一無所知。

金花婆婆沉吟片刻,道︰「好,你起來!」殷離喜道︰「多謝婆婆!」金花婆婆道︰「我答應你不傷他性命,但那柄屠龍刀我却是非取不可\dash{}」殷離道︰「可是\dash{}」金花婆婆截斷她的話頭,喝道︰「别再囉裡囉唆,惹得婆婆生氣。」手一揚,叮的又是一響。但見她雙手連揚,漸漸走遠,叮叮之聲不絶於耳。殷離抱頭坐在一塊石上,輕輕啜泣。張無忌想到她竟對自己一往情深如此,心下大是感激。

過了一會,金花婆婆在十餘丈外喝道︰「拿來!」殷離無可奈何,只得提了那雙布袋,走向金花婆婆之處。無忌走上幾步,低頭一看,這一驚當眞非同小可,只見地下每隔兩三尺,便是一根七八寸長的鋼針,插入山石之中,向上的一端尖利異常,閃閃生光。無忌越想越是心驚,這金花婆婆顯是擔心鬥不過金毛獅王,却在地下插滿了鋼針,欺他眼盲,只須引得他進入針地,就算不死也得重傷。若是發射暗器,謝遜聽風辨器,自可躱得了,但這地下預佈鋼針,無聲無息,雙目失明之人如何能彀抵擋?無忌生平極難動怒,但此刻見了這等毒計,忍不住怒氣勃發,伸手便想拔去鋼針,挑破她的陰謀,但轉念一想︰「這惡婆叫我義父爲『謝賢弟』,昔日和她的交情必是非同尋常,不如待她先和義父破臉,我再來揭破這惡婆的鬼計。今日老天既教我張無忌在此,決不致讓義父受到損傷。」

他心意已決,當下抱膝坐在石後,忽然間又是一陣山風吹來,風聲之中,有如落葉掠地,無忌却聽得出乃是輕功高強之人在悄悄欺近,轉頭往脚步聲來處瞧去,只見一人身形瘦小,脚步輕快,躱躱閃閃的走來,正是那丐幫的長老陳友諒,手中執著一柄薄的彎刀,却用布套遮住了刀光。無忌瞧了他這等鬼鬼祟祟的模樣,暗想趙明料事如神,此人果然並非善類。只聽得金花婆婆長聲叫道︰「謝賢弟,有不怕死的狗賊來啦!」

張無忌吃了一驚,心想金花婆婆好生厲害,難道我的蹤跡讓他發見了?按理説決不致於。只見陳友諒伏身在長草之中,更是一動也不敢動。張無忌幾個起落,又向前搶了數丈。他是要離義父越近越好,以防金花婆婆突施詭計,救援不及。過不多時,一個高大的人影從山前小屋中走了出來,正是謝遜,站在屋前,一言不發。

金花婆婆縱聲説道︰「謝賢弟,你對故人是步步提防,對外人却是十分輕信。你白天放了陳友諒,這會児又來找你啦。」謝遜冷冷的道︰「明槍易躱,暗箭難防。謝遜一生只是吃自己人的虧。那陳友諒又來找我,幹什麼來啦?」金花婆婆道︰「這等奸滑小人,理他作甚?白天你饒了他性命之時,你知道他手上脚下,擺的是什麼招式?他雙手一招『獅子搏兔』未曾使出,脚下蓄勢布力,乃是一招少林派的『降魔踢斗式』,哈哈,哈哈!」這笑聲猶似群鳥夜啼,深宵聽來,極是淒厲。謝遜一怔之下,已知金花婆婆所言不虛,只因自己眼盲,加之君子可欺以方,竟上了陳友諒的當。他淡淡的道︰「謝遜受人之欺,已非首次。此輩宵小,江湖上要多少有多少,多殺一個,少殺一個,又何足道?韓夫人,你也算是我好朋友,當時見到了不理,這時候再來説給我聽,是存心氣我來著?」説到這裡,突然間縱身而起,迅捷無倫的撲到了陳友諒的身前。

陳友諒大駭,大刀劈去。謝遜左手一揚,已將他手中彎刀奪過,拍拍拍連打他三個耳光,右手抓住他後頸,説道︰「我此刻殺你,如同殺雞,只是謝遜有言在先,許你十年之後,再來找我,下次再教我在此島上撞見,咱們當場便決生死。」提起他的身子,輕輕往山坡下擲了出去。眼見那陳友諒落身之處,正是金花婆婆插滿了尖針的,他只要一落下,身受針刺,她佈置了一夜的奸計立時破敗。金花婆婆飛身而前,伸拐杖在他腰間一挑,將他又送出數丈,喝道︰「你再敢踏上我靈蛇島一步,我殺你丐幫一百弟子。金花婆婆説過的話向來作數,今日先賞你一朶金花。」左手一揚,黃光微閃,{\upstsl{噗}}的一聲,一朶金花打在陳友諒左頰的「頰車穴」上,令他一時之間,説不出話來,以免洩漏機密。陳友諒撫住左頰,頭也不回的下山去了。

此時謝遜相距尖陣已不過數丈,張無忌反而落在他後面。須知他内功高出陳友諒何止數倍,屏住呼吸,謝遜和金花婆婆均不知他伏身在旁,陳友諒雖然動作極輕,却還是逃不過這兩位高手的耳音。

金花婆婆回身讚道︰「謝賢弟,你以耳代目,不減其明,此後重振雄風,再可在江湖上縱橫二十年。」謝遜道︰「我可聽不出『獅子搏兔』和『降魔踢斗式』。只要得知無忌孩児的確訊,我已死也瞑目。謝遜身上血債如山,死得再慘也是應該,還説什麼縱橫江湖?」金花婆婆笑道︰「我明教的護法教主,殺幾個人又算什麼?謝賢弟,你將屠龍刀借我一用吧。」謝遜搖頭不答。金花婆婆又道︰「此處形跡已露,你也不能再住。我另行覓個隱僻所在,送你去小住數月,待我持屠龍刀去勝了峨嵋派的大敵,決盡全力爲你探訪張無忌公子。」謝遜又搖了搖頭。金花婆婆道︰「謝賢弟,你還記得『四大法王,紫白金青』這八個字麼?想當年咱們在楊教主手下,鷹王殷賢弟,蝠王韋賢弟,再加你我二人,橫行天下,有誰能擋?今日虎老雄心在,你能讓紫衫龍王任由人欺,不加援手麼?」張無忌聽到這裡,大吃了一驚,心道︰「聽她言中之意,莫非這金花婆婆,竟然是本教四大法王之首的紫衫龍王?天下焉有這等奇事?」只聽謝遜喟然道︰「這些舊事,還提它作甚?老了,大家都老了!」

金花婆婆道︰「謝賢弟,做姊姊的老眼未花,難道看不出二十年來你武功大進?你又何必謙仰?咱們在這世上也没多少時候好活了,依我説啊,明教四大法王乘著没死,該當聯手江湖,再轟轟烈烈的幹它一番事業。」謝遜嘆道︰「殷二哥和韋賢弟,這時候未必還活著。尤其是韋賢弟,他身上寒毒難除,只怕已然不在人世了。」金花婆婆笑道︰「這個你可錯了。我老實跟你説,白眉鷹王和青翼蝠王,眼下都在光明頂上。」謝遜奇道︰「他們又回去光明頂?那幹什麼?」金花婆婆道︰「這是阿離親眼所見。阿離便是殷賢弟的親孫女,她得罪了父親,她父親要殺她。第一次是我救了她,第二次是韋賢弟所救。韋賢弟帶她上光明頂去,中途又給我悄悄偸了出來。阿離,你將六大門派如何圍攻光明頂,跟謝公公説説。」

殷離於是將在西域所見,簡略的説了一遍,只是她未上光明頂,就給金花婆婆擕回,以後光明頂上的一干事故,她就全然不知。謝遜越聽越是焦急,連問︰「後來怎樣?後來怎樣?」終於怒道︰「韓夫人,你雖因爭立教主之事,和衆兄弟不和,但本教有難,你怎能袖手旁觀?你瞧殷二哥和韋賢弟、五散人和五行旗,不是同赴光明頂出力麼?」金花婆婆冷冷的道︰「我取不到屠龍刀,終究是峨嵋派那滅絶老尼手下的敗將,便到光明頂上,也無面目再跟她動手,去了還不是白饒?何況當日我便得知你的所在,迫不及待,便趕到冰火島上來啦。」謝遜問道︰「你如何得知我的所在?是武當派的人説的麼?」金花婆婆道︰「武當派的人怎麼知道?張翠山夫婦受諸派勒逼,寧可自刎,也不肯露你藏身之所,武當門下自然不知。好,今日我什麼也不必瞞你,我在西域撞到一個名叫武烈的人,陰錯陽差,聽到他和女児説話,給我捉摸到了破綻,用酷刑逼他説了出來。」謝遜沉默半晌,纔道︰「這姓武的見過我那無忌孩児,是不是?想是他騙著小孩児家,探聽到了祕密。」張無忌聽到此處,心下慚愧無已,想起當年自己在朱家莊受欺,朱長齡、朱九眞父女以詭計套得自己吐露眞情,倘若義父竟爾因此落入奸人手中,自己可眞是萬死莫贖了。

只聽謝遜又道︰「六大派圍攻明教,豈同小可,我教到底怎樣?幹麼你到冰火島來之時,却瞞住了不説?這一次你回去中原,總聽到些音訊了。」金花婆婆道︰「我跟你説了,有什麼好處?左右不過是聽你埋怨責備。明教興衰亡,早跟老婆子没半點相干。當年光明頂上,左右光明使者夾擊老婆子的事,你是全忘了。老婆子却記得清清楚楚。」謝遜道︰「唉,私怨事小,護教事大。韓夫人,你胸襟未免太狹。」金花婆婆怒道︰「你是男子漢大丈夫,我可氣量窄小的婦道人家。當年我破門出教,立誓和明教再不相干。若非如此,那胡青牛怎能將我當作外人?他爲何一定要我立誓重歸明教,纔肯醫治銀葉先生的毒傷?謝賢弟,我跟你説,這蝶谷醫仙乃是我親手所殺,紫衫龍王早已犯了明教的大戒。我跟明教還能有什麼干係?」謝遜搖了搖頭,道︰「韓夫人,我明白你的心事。你借我屠龍刀去,口中説是對付峨嵋派,實則是要去對付楊逍、范遙。那我更加不能相借。」

金花婆婆咳嗽數聲,道︰「謝賢弟,當年你我的武功,高下如何?」謝遜道︰「四大法王,各有所長。」金花婆婆道︰「今日你壞了一對招子,再跟老婆子相比呢?」謝遜昂然道︰「你要恃強奪刀,是不是?謝遜有屠龍刀在手,抵得過壞了一對招子。」他仰天一聲清嘯,怒聲喝道︰「那玉面火猴跟我相依爲命,在冰火島上伴我二十年,你爲何毒死了牠?我一直隱忍不言,豈難道我當眞不知麼?」

張無忌心頭一震,那玉面火猴當年救過他父母的性命,自己幼時在冰火島上,唯一的遊侶便是這頭靈猴,乍聞牠的死訊,宛似喪失了一位知交好友,説不出的傷心難過。只聽金花婆婆冷冷的笑了一聲,説道︰「這頭子猴児每之見了老婆子總是雙目炯炯,不懷好意,牠身法如電,不下於一位武林高手,老婆子若是一個不防,説不定還要喪生在牠爪底。我想這玉面火猴既然如此靈異,那麼給牠吃的那幾枚水蜜桃,是否曾在毒藥水中浸過,牠也該當分辨出來。不料猴児總是畜生,徒負靈名,將這些水蜜桃吃得乾乾淨淨,還向老婆子拱手作揖,連連道謝呢。」張無忌只聽得怒火如焚,恨不得便要縱身而出,重重打她幾個耳光,一洩心中的悲憤,但轉念一想︰「這老婆子雖然作惡多端,終究是我教下四大護教法王之首,我須得耐心將她收服,以全昔日衆兄弟的義氣。」

謝遜噓了一口長氣,向前踏了一步,一對失明的眸子,瞪視著金花婆婆,神威凜凜,殷離瞧得害怕,向後退了幾步。金花婆婆却佝僂著身子,撐著拐杖,偶爾發出一兩聲咳嗽,看來謝遜只須一伸手,便能將她{\upstsl{砸}}爲肉泥,但她站著一動不動,似乎全没將謝遜放在眼底。張無忌曾見過她數度出手,當眞是快速絶倫,比之韋一笑,另有一種難以言説的詭祕怪異,如鬼如魅,似精似怪。此刻她和謝遜相對而立,一個是劍拔弩張,蓄勢待發,一個却是成竹在胸,好整以暇。無忌心想她排名尚在我外公、義父和韋蝠王之上,眞實的武功必是十分厲害,不禁爲謝遜暗暗擔心。但聽得四下裡鳴啾啾,朔風動樹,却有一番悲涼之意。

兩人相向而立,相距不過丈許,却是誰也不先動手,過了良久,謝遜忽道︰「韓夫人,今日你迫得我非動手不可,違了我們四大法王昔日結義的誓言,謝遜心下好生難受。」金花婆婆道︰「謝賢弟,你心腸向來很軟,我當時眞没料到,武林中那許多成名的英雄豪傑,都是你一手所殺。」謝遜嘆道︰「那是我心傷父母妻児之仇,甚麼也不顧得了。我生平最最不該之事,乃是以七傷拳擊斃了少林派的空見神僧。」金花婆婆凜然一驚,道︰「空見神僧當眞是打死的麼?你甚麼時候,練成了這等厲害的武功。」她本來自信足可對付得了謝遜,待得聽到空見神僧也死在他的拳底,心下始有懼意。

謝遜道︰「你不用害怕。空見神僧只挨打不還手,他是要以廣大無邉的佛法,渡化我這個邪魔外道。」金花婆婆哼了一聲,道︰「這纔是了。老婆子及不上空見神僧,你一十三拳打死空見,不用九拳十拳,便能料理了老婆子啦。」謝遜退了一步,聲調忽變柔和,説道︰「韓夫人,從前在光明頂上,韓大哥和你都待我不錯。那日小弟生病,你夫婦服侍我一月有餘,小弟始終銘感於心。」他拍了拍身上的灰布棉袍,又道︰「我在海外以獸皮爲衣,你給我做這身衣衫,裡裡外外,無不合身,足見光明頂結義之情尚在。你毒死玉面火猴,那也無可如何。你去吧!從此之後,咱們不必再行相見。我只求你傳個訊息出去,要我那無忌孩児到此島來和我一會,做兄弟的足感大德。」

金花婆婆淒然一笑,道︰「你倒記得從前這些情誼。不瞞你説,自從你銀葉大哥一死,我早將世情瞧得淡了,只是世間尚有幾樁怨仇未了,我不能就此撒手而死,相從你銀葉大哥於地下。謝賢弟,光明頂上這些人物,任他武功了得,機謀過人,你老姊姊都没瞧在眼裡,便只對你謝賢弟另眼相看,你可知道其中的緣由麼?」謝遜抬頭向天,沉思半晌,搖頭道︰「謝遜庸庸碌碌,不値賢姊見顧。」

金花婆婆走上幾步,忽然撫著一塊大石,坐了下來,説道︰「昔年光明頂上,只有楊教主夫人和你謝賢弟,紫衫龍王瞧著順眼。爲姊的嫁了銀葉先生,唯有你們二人,没怪我明珠暗投,所投非人。」謝遜也緩緩的坐下,説道︰「韓大哥雖非本教中人,却也英雄了得,衆兄弟力持異議,未免胸襟窄了。唉,六大派圍攻光明頂,不知衆兄弟都無恙否?」金花婆婆笑道︰「謝賢弟,你身在海外,心懸中土,念念不忘舊日兄弟。人生數十年,轉眼即過,何必整日價想著旁人?」兩人此時相距已不過數尺,呼吸可聞,謝遜聽得金花婆婆每説幾句話便咳嗽一聲,説道︰「那年你和丐幫激鬥,肺上中了一劍,纏綿至今,總是不能痊癒麼?」金花婆婆道︰「每到天寒,便咳得厲害些。{\upstsl{嗯}},咳了三十來年,早也慣啦,謝賢弟,我聽你氣息不勻,是否練那七傷拳時傷了内臟?須得多多保重才是。」

謝遜道︰「多謝賢姊関懷。」忽然抬起頭來,向殷離道︰「阿離,你過來。」殷離走到他身前,叫了聲︰「謝公公!」謝遜道︰「你使出全力,戮我一指。」殷離愕然道︰「我不敢。」謝遜笑道︰「你的千蛛絶戸手傷不了我,儘管使勁便了。我是要試試你的功力。」殷離仍道︰「孩児不敢。」又道︰「謝公公,你既和婆婆是當年結義的好友,能有什麼事説不開?不用爭這把刀了吧。」謝遜淒然一笑,道︰「你戮我一指試試。」殷離無奈,取出手帕,包住右手食指,一指戮在謝遜肩頭,驀地裡「啊喲」一聲大叫,向後摔了出去,飛出一丈有餘,騰的一響,坐在地下,便似全身骨骼,根根都已寸斷。金花婆婆不動聲色,道︰「謝賢弟,你好毒的心思,生怕我多了一個幫手,先行出手剪除。」謝遜不答,沉思半晌,道︰「這孩児心腸很好,她戮我這指只使了二三成力,手指上又包了手帕,不運千蛛毒氣傷我,很好很好。若非如此,千蛛毒氣返攻心臟,她此刻已然没命了。」張無忌聽了這幾句話,背上出了一陣冷汗,心想義父明明説是試試阿離的功力,倘若她果眞全力一試,這時候豈非已然斃命?明教中人向來心狠手辣,以我義父之賢,也是在所不免。他却不知謝遜和金花婆婆相交有年,明白對方心意,幾句家常話一説完,便是決不容情的惡鬥,金花婆婆多了殷離一個幫手,於他大大不利,是以用計先行除去,不料殷離對謝遜毫無敵意,這纔保得性命。

謝遜道︰「阿離,你爲什麼一片善心待我?」殷離道︰「你\dash{}你是他的義父,又是\dash{}又是爲他而來。在這世界上,只有你跟我兩人,心中還記著他。」謝遜「啊」了一聲,道︰「没想到你對無忌這麼好,我倒險些児傷了你的性命。你附耳過來。」殷離掙扎著爬起,慢慢走到他的身旁。謝遜將口唇湊在她的耳邉,説道︰「我傳你一套内功的心法,這是我在冰火島上參悟而得,集我畢生武功之大成。」不等殷離答話,便將那内功心法從頭至尾説了一遍。殷離似懂非懂,只是用心暗記。謝遜怕她記不住,又説了兩遍,問道︰「你記住了麼?」殷離道︰「都記得了。」謝遜道︰「你修習五年之後,當有小成。你可知道我傳你功夫的用意麼?」殷離突然哭了出來,説道︰「我\dash{}我知道。可是\dash{}可是我不能。」

謝遜厲聲道︰「你知道什麼?爲什麼不能?」説著左掌蓄勢待發,只要殷離答得不對,立時便斃她於掌下。殷離雙手掩面,説道︰「我知道你要我去尋找無忌,將這功夫轉授於他。我知道你要我練成上乘武功之後,保護無忌,迴護無忌,令他不受壞人的侵害,可是\dash{}可是\dash{}」

\chapter{聖火六令}

殷離説了兩個「可是」,雙手掩面,放聲大哭起來。謝遜站起身來,喝道︰「可是什麼?是我那無忌已然遭遇不測麼?」殷離哭道︰「他\dash{}他早在六年之前,在西域\dash{}在西域墜入深谷而死。」謝遜身子一晃,顫聲道︰「此言當眞?」殷離哭道︰「是眞的。那武烈父女親眼見他喪命。我在他二人身上接連點了七次千蛛手,又七次救他們活命,這等熬煎之下,他們\dash{}他們不能再説假話。」謝遜仰天一嘯,聲音悲壯,兩頰旁老泪滾滾而下。張無忌見義父和表妹爲自己這等哀傷,再也忍耐不住,便欲挺身而出相認。忽聽得金花婆婆道︰「謝賢弟,你那位義児張公子既已殞命,你守著這口屠龍寶刀何用?不如借了於我吧。」謝遜嘶啞著嗓子道︰「你瞞得我好苦。要取寶刀,先取了我這條命去。」輕輕將殷離推在身旁,嘶的一聲,將長袍前襟撕下,向金花婆婆擲了過去,這叫做「割袍斷義」。

當殷離述説張無忌已死的訊息之初,金花婆婆本待阻止,但轉念一想,謝遜一聽到義子身亡,定然心神大亂,拚鬥時雖然多了三分狠勁,却也少了七分謹愼,更易陥入自己所佈的鋼針陣中,當下只是在旁微微冷笑,並不答話。

張無忌心想︰「我該當此時上前,説明眞相,免他二人無謂的傷了義氣。」便在此時,忽聽得左側長草中傳來幾下輕微的呼吸之聲,有人欺到了身旁。這幾下呼吸聲極輕極短,若非張無忌耳音精靈,再也聽不出來,他心念一動︰「原來金花婆婆暗中尚伏下厲害幫手?我倒不可貿然現身。」但聽得刀風呼呼,謝遜已和金花婆婆交上了手。

只見謝遜使開寶刀,有如一條黑龍在他身周盤旋遊走,忽快忽慢,變化若神。金花婆婆忌憚寶刀鋒利,遠遠在他身旁兜著圏子。謝遜時時賣個破綻,金花婆婆毫不畏懼的欺身直進,待他迴刀相砍,隨即極巧妙的避了開去。二人於對方武功素所熟知,料得不能在一二百招中便分高下。謝遜是倚仗寶刀之利,金花婆婆則欺他盲不見物,二人均在自己所長的這一點上尋求取勝之道,反而將招數内力,置之一旁,是以明教兩大高手這番相鬥,却是各逞機智,並非較量眞實武功。

忽聽得颼颼兩聲,黃光閃動,金花婆婆發出了兩朶金花。謝遜屠龍刀一轉,兩朶金花都黏了在刀上。原來那金花乃以純鋼打就,外面鍍以黃金,那鑄造屠龍刀的玄鐵却具極強磁性,遇鐵即吸。這金花乃是金花婆婆當年仗以成名的暗器,施放時變幻多端,謝遜即令雙目健好,也須全力閃避擋格,不料這屠龍刀正是所有暗器的剋星。金花婆婆倏左倏右的連發八朶金花,每一朶均黏在屠龍刀上。此時月黯星稀,夜色慘淡,黯黑的屠龍刀上黏了八朶金花,使將開來,猶如數百隻飛螢在空中亂竄亂舞,突然間金花婆婆咳嗽一聲,一把金花擲出,共有十六七朶,教謝遜一柄屠龍刀黏了東邉的,黏不了西邉。謝遜袍袖揮動,捲去了七八朶,另有八九朶黏在屠龍刀上,喝道︰「韓夫人,你號稱紫衫龍王,名字犯了此刀的忌諱,若再戀戰,於君不利。」金花婆婆打個寒噤,大凡學武之人,性命都在刀口上打滾,最講究口彩忌諱,自己號稱「龍王」,此刀却名「屠龍」,實是大大的不妙,當下陰惻惻的笑道︰「説不定倒是我這殺獅杖先殺了盲眼獅子。」呼的一杖,逕往謝遜肩頭擊去。謝遜沉肩一閃,突然脚下一個踉蹌,「啊」的一聲,這一杖中了他的左肩,雖然力道已卸去了大半,但仍是結結實實的打中了。張無忌大喜,暗中喝了聲彩。

張無忌見謝遜故意裝作閃躱不及,受了一杖,心下便想︰「義父只須將左手袍袖中捲著的金花撒將出去,金花婆婆必向左退。義父一招『千山萬水』亂披風斬去,金花婆婆不敢抵擋寶刀鋒鋭,務必更向左退,接連兩退,蓄勢已盡,那時義父以内力逼出屠龍刀上金花,激射而前,金花婆婆再退不遠,非身受重傷不可。」他心念甫動,果見黃光閃處,謝遜已將左手袍袖捲著的金花撤出,金花婆婆疾向左退。張無忌斗然間想起一事,心叫︰「啊喲,不好,金花婆婆乃是將計就計。」其時他胸中於武學包羅萬有,這兩大高手的攻守趨避,無一不在他算中,但見謝遜的一招「千山萬水」亂披風勢斬出,金花婆婆更向左退。謝遜大喝一聲,寶刀上黏著的十餘金花疾射而前。金花婆婆「啊喲」一聲叫,足下一個踉蹌,向後縱了幾步。

謝遜是個心意決絶的漢子,既已割袍斷義,下手便毫不容情,縱身而起,揮刀向金花婆婆砍去,忽聽得殷離高聲叫道︰「小心脚下有尖針。」謝遜聽到叫聲,一楞之下,收勢已然不及,只聽得颼颼聲響,十餘朶金花猛力射至,乃是金花婆婆令他身在半空,無法收勢而退,這一落下來,雙足正好刺在尖針之上。謝遜無可奈何,只得揮刀格打金花,忽聽得脚底錚錚幾聲響處,他雙足已然著地,竟是安然無恙。他俯身一摸,觸到四周都是七八寸長的鋼針,插在山石之中,尖利無比,只是自己落脚處四枚鋼針,却被人用石子打飛了。謝遜又怒又驚,聽那擲石去針、暗中相助自己之人的手法,正是日間巨鯨幫手擲七石的少年。此人在旁窺視已久,自己竟然絲毫没有察覺,額上不禁出了一陣冷汗。

他二人互施苦肉計,謝遜肩頭是眞的受了一杖,金花婆婆身上也眞的吃了兩朶金花,雖然所傷均非要害,但對方何等勁力,受上了實是不易抵擋。金花婆婆大咳幾下,向著張無忌伏身之處發話道︰「巨鯨幫的小子,你一再干擾老婆子的大事,快留下名來。」張無忌還未回答,突然間黃光一閃,殷離一聲悶哼,已被三朶金花打中。原來金花婆婆已瞧出張無忌武功決不在己之下,自己出手懲治殷離,他定要阻撓,是以面對著無忌説話,乘他絲毫没有防備之際,反手發出金花。這三朶金花深入殷離胸口,乃是致命之傷。

無忌大駭,飛身而起,半空中接住金花婆婆發來的兩朶金花,一落地便將殷離抱在懷中,殷離神智尚未迷糊,見一個小鬍男子抱住自己,急忙伸手撐拒,只一用力,嘴裡便連噴了幾口鮮血。無忌登時醒悟,伸手在自己臉上用力擦了幾下,抹去臉上黏著的鬍子和化裝,露出本來的面目。殷離呆了一呆,叫道︰「阿牛哥哥,是你?」無忌微笑道︰「是我!」殷離心中一寬,登時便暈了過去。無忌見她傷重,不敢便替他取出身上所中暗器,只是點了她神封、靈墟、步廊、通谷諸處穴道,護住她的心脈。只聽得謝遜朗聲道︰「閣下兩次出手相救,謝遜多承大德。」無忌哽咽道︰「義\dash{}義\dash{}你何必\dash{}」

便在此時,忽聽得遠處傳來叮的一聲響,這聲音似乎極輕,又似極響,聽在耳中似乎極是舒服受用,却又似乎是煩燥難當。謝遜、張無忌、金花婆婆聽到這聲音,心頭都是一震,竟比驀地裡聽到晴天霹靂更是吃驚。他三人都是内力高強之人,張無忌九陽神功已成,更是諸邪不侵,但這異音之來,竟是震得他心旌搖動,一刹那間,身子猶如飄浮半空,六神無主,生平從未遭遇過如此經歷。他急忙收攝心神,只聽得那聲音又是一響,這一次却又近了數十丈,在這頃刻之間,這聲音移動得竟是如此迅速。

可是這一下異聲,和第一聲却是截然不同,聲音柔媚宛轉,如靜夜私語,如和風拂柳,但聽在耳裡,同樣的奪魄驚心。張無忌知道來了異人,絲毫不敢怠忽,橫抱殷離,站起身來。突然間{\upstsl{噹}}的一聲巨響,山谷間{\upstsl{嗡}}{\upstsl{嗡}}作聲,如土崩地裂,如百鐘齊鳴,在這巨響聲中,三個人現身眼前。張無忌一瞥之下,只見那三人都是身穿寬大的白袍,其中兩人身形甚高,左首一人却是個女子。三人背月而立,看不清他們面貌,但每人的白袍角上赫然繡著一個火燄之形,竟然是明教中人。

只聽中間那身材最高之人朗聲道︰「明教聖火令到,護教龍王、獅王,還不下跪迎接,更待何時?」他的話聲語調不準,顯得極是生硬。無忌吃了一驚,心道︰「楊教主遺言中説道,本教聖火令自第三十一代教主石教主之時,便失落於幫丐之手,迄今無法取回,怎麼在這三人手中?這是否眞的聖火令?這三人是否本教弟子?」一霎時心中湧起了無數疑竇。只聽金花婆婆道︰「本人早已破門出教,『護教龍王』四字,再也休提。閣下尊姓大名?這聖火令是眞是假,從何處得來?」那人喝道︰「你既已破門出教,尚絮絮何爲?還不快去!」金花婆婆冷冷的道︰「金花婆婆生平受不得旁人半分惡語,當日便楊教主在世,對我也禮敬三分。你是教中何人,對我竟敢大呼小叫?」突然之間,三人身形晃動,同時欺近,三隻左手齊往金花婆婆身上抓去。金花婆婆拐杖一揮,向三人橫掃過去,不料這三人脚下不知如何移動,身形早變。金花婆婆一杖擊空,已被三人的右手同時抓後領,一抖之下,向外遠遠的擲了出去。

以金花婆婆武功之強,便是天下最厲害的三個高手向她圍攻,也不能一招之間便將她身子抓住擲出。但這三個白袍人步法既怪,出手又是配合得妙到毫巓,較之一個人生有三頭六臂,還要法度嚴謹。張無忌情不自禁的「噫」了一聲,只覺這三人的身法、步法、手法,竟是乾坤大挪移的家數,難道這三人居然同時練就了這等高深的武功?這三人初到時那一聲巨響,已將殷離驚醒,她睜開眼來,見無忌將自己橫抱在手臂之中。她只感胸口劇痛,幾乎氣也透不過來,當下閉上了眼睛,除了竭力忍痛,已不能再想什麼。

那三人身子這麼一移,張無忌已得清清楚楚,最高那人虯髯碧眼,另一個黃鬚鷹鼻,竟然都是胡人。那女子一頭黑髮,和華人無異,但眸子極淡,幾乎無色,瓜子臉型,約莫二十歳上下,雖然瞧來詭異,相貌却是甚美。無忌心想︰「原來這三人都是胡人,怪不得語調生硬,説的話又文謅謅的好似背書。」只聽那虯髯人朗聲又道︰「見聖火令如見教主,謝遜何不跪迎?」謝遜道︰「三位到底是誰?若是本教弟子,謝遜該當相識。若非本教中人,聖火令與三位毫不相干。」虯髯人道︰「明教源於何土?」謝遜道︰「源起波斯。」虯髯人道︰「這就是了。我乃波斯明教總教流雲使,另外兩位是妙風使、輝月使。我等奉總教主之命,特從波斯來至中土。」謝遜和無忌都是一怔。無忌讀過楊逍所著的「明教流傳中土記」,知道明教確是從波斯傳來,眼看這三個男女果是波斯胡人,武功身法又是如此,定是不假,當下默不作聲,且聽謝遜如何對答。只聽那黃鬚的妙風使道︰「我教主接獲訊息,得知中土支派教主失蹤,群弟子自相殘殺,本教大趨式微,是以命雲風月三使前來整頓教務。合教上下,齊奉號令,不得有誤。」無忌一聽之下,心中大喜︰「總教主有號令傳來,那是再好也没有了。免得我擔此重任,見識膚淺,不免誤了大事。」

只聽得謝遜説道︰「中土明教雖然出自波斯,但千餘年來獨立成派,自來不受波斯總教管束。三位遠道前來中土,謝遜至感歡忭,跪接云云,却是從何説起。」那虯髯的流雲使伸手入懷,取出兩塊二尺來長,非金非玉的牌來,相互一擊,錚的一聲響,正是無忌第一次所聽到的那古怪聲音。這時相距既近,更是震得人不能自恃。好在那流雲使一擊之下,便不再擊,説道︰「這是中土明教的聖火令,前任姓石的教主不肖,失落於丐幫之手,今由我等取回。自來見聖火令如見教主,謝遜還不聽令?」

謝遜入教之時,聖火令失落已久,從來没有見過,但其神異之處,却是向所耳聞,明教的經書典籍之中,也往往提及,知道這三人所持的六塊玉牌,確是本教的聖火令。何況三人一出手,一招之間,便抓了金花婆婆擲將出去,自己武功和金花婆婆乃在伯仲之間,縱要抗拒,也是無能爲力,當下説道︰「在下相信尊駕所言,但不知尊駕有何吩咐?」流雲使左手一揮,妙風使、輝月使和他均似心意目通,三個人縱身而起,兩個起落,已躍到金花婆婆身側。金花婆婆六朶金花擲出,分擊三使。三使東一閃,西一晃,盡數避開。但見輝月使直欺而前,纖手伸出,點向金花婆婆咽喉。金花婆婆拐杖一封,跟著還擊一杖,突然間金花婆婆騰身而起,後心被流雲使和妙風使抓住,提了起來。這一來她後心要穴爲敵人所制,已全然不能動彈。輝月使搶上三步,左手食指連動,點中了她胸腹的七處穴道。

這幾下對招極是乾淨利落,張無忌看得明白,心道︰「他三人起落身法,未見有過人之處,只是三人配合得巧妙無比。輝月使在前誘敵,其餘二人已神出鬼没的將金花婆婆擒住。但每個人的武功,未必便在金花婆婆之上。」流雲使提著金花婆婆,左手一振,將她輕輕的擲在謝遜身前,説道︰「謝獅王,本教教規,入教之後終身不能叛教。此人自稱破門而出,爲本教叛徒,你先將她首級割下。」謝遜一怔,道︰「中土明教向來無此教規。」流雲使冷冷的道︰「此後中土明教悉奉波斯總教號令。這婆子適纔擺毒計害你,一切全落入咱們眼中,留著便是禍胎,快快將她除了。」謝遜昂然道︰「這位韓夫人昔年待謝某不錯,明教四王,情同金蘭。今日雖然她對謝某無情,謝某却不可無義,不能動手加害。」妙風使哈哈一笑,道︰「中國人婆婆媽媽,有這麼多囉唆。她要害你,你却不去殺她,這算是什麼道理?當眞奇哉怪也,莫明其妙。」謝遜道︰「謝某殺人不貶眼,却不殺同教朋友。」輝月使道︰「非要你殺了她不可。你不殺她,便是不聽號令,咱們先殺了你。」謝遜道︰「三位到中土來,第一件事便勒逼金毛獅王殺了紫衫龍王,這是爲了立威嚇人麼?」輝月使微微一笑,道︰「你雙眼雖瞎,心中倒也明白。快快動手罷!」謝遜仰天長笑,聲動山谷,大聲道︰「我金毛獅王光明磊落,别説不殺同夥朋友,此人即令是謝某的深仇大怨,既被你們擒住,已然無力抗拒,謝某豈能再以白刃相加?」

張無忌聽了義父豪氣干雲的言語,心下暗暗喝采,對這波斯明教三使,漸生反感。只聽妙風使道︰「明教教徒,見聖火令如見教主,你膽敢叛教麼?」謝遜心念一動,昂然説道︰「謝某雙目已盲了二十餘年,你便將聖火令放在我眼前,我也瞧它不見。説什麼『見聖火令如見教主』?」妙風使大怒,道︰「好!那你是決意叛教了?」謝遜道︰「謝某不敢叛教。可是明教的教旨乃是行善去惡,義氣爲重。謝遜寧可自己人頭落地,不幹這等没出息的歹事。」金花婆婆身子不能動彈,謝遜的言語,却是一句句的都聽在耳裡。

張無忌知道義父生死已迫在眉捷,當下輕輕將殷離放在地下,只聽得流雲使道︰「明教中人,不奉聖火令者,一律殺無赦!」謝遜喝道︰「本人是護教法王,即令是教主要殺我,也須開壇秉告天地,申明罪狀。」妙風使嘻嘻笑道︰「明教在波斯好端端,一至中土,便有這許多臭規矩!」三使同時呼嘯,一齊搶了上來。謝遜屠龍刀揮動,護住身子。三使連攻三招,竟然搶不近身。突然之間,三使各執聖火令在手,輝月使欺身直進,左手持令向謝遜天靈蓋上拍了下去。謝遜舉刀一擋,{\upstsl{噹}}的一響,聲音極是怪異。這屠龍刀無堅不摧,可是竟然削不斷聖火令。便在這一瞬之間,流雲使滾身向左,已然一令打在謝遜腿上。謝遜脚下一個踉蹌,妙風使橫令點他後心,突然間手腕一緊,聖火令被人挾手奪了去。他大驚之下,回過身來,只見一個穿著水手裝束的少年,右手中拿著一根聖火令。

張無忌這一下縱身奪令,快速無比,巧妙無比,妙風使竟是事先毫無知覺。流雲使和輝月使驚怒之下,齊從兩側攻上。張無忌身形一轉,向左避開,不意拍的一響,後心已被輝月使一令擊中。那聖火令非金非玉,極是堅硬,這一下打中了,張無忌眼前一黑,幾欲暈去,幸得護體神功立時發生威力,當即鎭懾心神,向前衝出三步。波斯三使毫不放鬆,跟著又圍了上來。張無忌右手持令向流雲使虛晃一招,左手倏地伸出,已抓住了輝月使左手的聖火令,豈知輝月使忽地放手,那聖火令尾端向上一彈,拍的一響,正好打中無忌手腕。他左手五根手指一陣麻木,只得放下左手中已然奪到的聖火令,輝月使纖手伸處,抓口掌中。

張無忌練成乾坤大挪移法以來,再得張三丰指點太極拳中的精奥,縱橫宇内,從無敵手,不意此時一出手便被輝月使這樣一個年輕女子接連打中。第二下打在腕骨之上,若非他的護體神功自然而然將來力卸開,手腕早已折斷。他驚駭之下,不敢再與敵人對攻,凝立當場,要看清楚敵人招數來勢,以定應付方策。波斯三使見他雖然兩次被擊,竟似並未受傷,也已驚奇不已,那是他們生平從未遇到過的情景。妙風使一低頭,一個頭錘向無忌攻來,這種打法,原是武學中大忌,以自己最緊要的部位,送向敵人挨打。無忌端立如山,知他這一招似拙實巧,必定伏下厲害異常的後著,待他的腦袋撞到自己身前一尺之處,這纔向後退了一步,驀地裡流雲使躍身半空,向他頭頂坐了下來。這一招更是怪異,竟是以臀部攻入,天下武學之道雖緊,從未有這種既無用,又笨拙的招數。無忌不動聲色,向旁又是一讓,只覺胸口一痛,已被妙風使用手肘撞中。只是妙風使被他九陽神功一彈,向後倒退三步,跟著又倒三步,甫欲站定,又是倒退三步。

波斯三使愕然變色,輝月使雙手兩根聖火令橫掃,流雲使突然間在無忌跟前連翻三個空心斛斗。張無忌適纔被妙風使手肘這麼一撞,胸口隱隱作痛,忽見流雲使亂翻斛斗,不知是何用意,心想還是遠而避之爲妙,剛向左側踏開一步,不知如何,眼前白光一閃,右肩已被流雲使的聖火令重重擊中了一下。這一招更是匪夷所思,事先既無半點徵兆,而流雲使明明是在半空中大翻觔斗,怎能忽地伸過聖火令來,擊在自己肩頭?無忌驚駭之下,已不敢戀戰,加之肩頭所中這一令勁道頗爲沉重,雖被他九陽神功彈開,却已痛入骨髓。但心知自己只要一退,義父性命不保,今日不理自己生死,無論如何要擊退強敵,保護義父周全,於是深深吸了一口氣,一咬牙,飛身而前,伸掌向流雲使胸口拍去。

流雲使同時的飛身而前,雙手聖火令相互一擊,錚的一響,張無忌心神一盪,身子從半空中直墜下來,但覺腰脅中一陣疼痛,已被妙風使踢中了一脚。砰的一下,妙風使向後摔出,輝月使的聖火令却又擊中了無忌的右臂。

謝遜在一旁聽得明白,知道巨鯨幫中這個少年已是接連吃虧,眼下已不過是在勉力支撐,苦於自己眼盲,無法上前應援,心中焦急萬分。須知他若孤身對敵,當可憑著風聲,分辨敵人兵刃拳脚的來路,但若去相助朋友,怎能分得出那一下是朋友的兵刃,那一下是敵人的拳脚?他屠龍刀揮舞之下,倘若一刀殺了朋友,豈非大大的恨事?耳聽得張無忌已處於接連挨打的局面,當即叫道︰「少俠,你快脱身而走,這是明教的事,與閣下並不相干。少俠今日一再相援,謝遜已是感激不盡。」張無忌大聲道︰「我\dash{}我\dash{}你快走,聽我説,你快走!」只見流雲使一令擊來,張無忌以手中聖火令一擋,拍的一下,如中敗革,似擊破絮,聲音極是難聽。流雲使把捏不定,聖火令脱手向上飛出。張無忌躍起身來,欲待搶奪,突然間嗤的一聲響,後心衣衫被輝月使抓了一大截下來。她手拍上指甲在他背心上劃破了幾條爪痕,隱隱生痛,這麼緩得一緩,那聖火令又被流雲使搶了回來。

經此幾個回合的接戰,張無忌心知憑這三人功力,每一個人都和自己相差甚遠,只是一來武功怪異,二來兵刃神奇,最厲害的是三人聯手,陣法不似陣法,套子不似套子,詭祕陰毒,匪夷所思,只要能彀擊傷其中一人,今日之戰便能獲勝。但他兩次震倒妙風使,每一次他都是若無其事似乎絲毫不受内傷。擊一人則其餘二人首尾相應,張無忌連變數種拳法,始終打不破這三人聯手之局,反而又被聖火令打中了兩下。波斯明教三使這時已不敢以拳脚和他身子相碰,蓋每一次用拳脚擊中在他身上,自己又吃大虧。

謝遜大喝一聲,將屠龍刀豎抱在胸前,縱身躍入戰團,搶到張無忌身旁,説道︰「少俠,用刀!」將屠龍刀遞了給他。張無忌心想仗著這刀神威,或能擊退大敵,當下接了過來。謝遜右足一點,向後退開,在這頃刻之間,後心已重重中了妙風使一拳,只打得他胸腹間五臟六腑似乎都移了位置。這一拳來無影、去無蹤,謝遜竟是聽不到半點風聲。張無忌一刀向流雲使砍了過去,流雲使舉起兩根聖火令,雙手一振,忙加運内力。流雲使的聖火令奪人兵刃,原是手到擒來,千不一失,這一次居然奪不了張無忌手中單刀,大感詫異。輝月使一聲嬌叱,手中兩根聖火令也已架在屠龍刀上,四令奪刀,威力更巨。

張無忌身上已受了七八處傷,雖然均是輕傷,力道究已大爲減弱,這時但感半邉身子發熱,握著刀柄的右手不住發顫。他知此刀乃是義父性命所繫,義父尚不自己身份眞相,居然肯以此刀相借,可説是豪氣干雲之舉,倘若此刀竟在自己手中失去,還有何面目以對義父?驀然間大喝一聲,右臂一伸,體内九陽神功源源激發。流雲、輝月二使臉色齊變,妙風使見情勢不對,一根聖火令又搭到了屠龍刀上。張無忌精神大振,以一抗三,竟是絲毫不餒,心下不禁暗暗自慶,幸好一上來便出其不意的搶得妙風使一枚聖火令,否則六令齊施,自己更是難以抵擋。這時四個人已至各以内力相拚的境地。無忌心想你們和我比拚内力,正是以短攻長,我是得其所哉了。霎時間四人均是凝立不動,各運内力,突然之間,無忌胸口一痛,似手被一枚極細的尖針刺了一下。

這一下刺痛突如其來,直鑽入心肺,張無忌手一鬆,屠龍刀便被五根聖火令吸了過去。無忌猝遇大變,竟是心神不亂,順手拔出腰間倚天劍,一招太極劍法的「圓轉如意」,斜斜的劃了個圏子,同時攻向波斯三使的小腹。三使忙要後躍相避,無忌已將倚天劍插還腰間劍鞘,手一伸,又將屠龍刀奪了過來。這四下失刀、出劍、還劍、奪刀,手法之快,直如閃電,正是乾坤大挪移的第七層功夫。波斯三使「噫」的一聲,大是驚奇。他三人内力修爲,遠遠不及無忌,這一開口出聲,三根聖火令反而被屠龍刀帶了過來。三人急運内力相奪,終於又成相持不下之局。突然之間,無忌胸口又被尖針刺了一下。

這次無忌已有預備,寶刀未曾脱手。但這兩下刺痛似有形,實無質,一股寒氣突破他護體的九陽神功,直侵内臟。無忌情知這是波斯三使一種極陰寒的内力,積貯一點,從聖火令上傳來,攻堅而入。本來以至陰至陽,未必便勝得了九陽神功。只是他的九陽神功遍護全身,這陰勁却是凝聚如絲髮之細,一鑽一閃,一戮一刺,令人難防難當。有如巨象之力雖巨,婦人小児却能以繡花小針刺入其膚。這服陰勁一入無忌體内,立即消失,但便是這麼一刺,可眞疼痛入骨。

輝月使連連運兩下「透骨針」的内勁,但見無忌竟是毫不費力的抵擋了下來,心下更是駭異,又見他腰間懸著寶劍極是鋒鋭,有心一併奪了過來,却是分手不得。妙風使雖然空著左手,但全身勁力都已集於右臂,左手已與癱瘓無異。無忌知道如此僵持下去,敵人尖針一般的陰勁一下一下的刺將過來,自己終將支持不住,可是實無對策。耳聽身後謝遜呼吸粗重,正自一步步的逼近,知他要擊敵助已。只是這時四人内勁佈滿全身,謝遜一拳擊在敵人身上,已與擊打無忌一般無異,是以始終遲遲不敢出手。無忌尋思︰「我和波斯三使並無仇怨,總是要義父先行脱身要緊。但他若知我便是無忌,無論如何不肯捨我而去。」於是朗聲説道︰「謝大俠,這波斯三使武功雖奇,在下要脱身而去,却也不難。請你先行暫避一時,在下事了之後,自當奉還寶刀。」波斯三使聽得他在全力比拚内勁之際,竟能開口説話,洋洋一如平時,心下更驚。

謝遜道︰「少俠高姓大名?」無忌略一避疑,心想此時萬萬不能跟他相認,否則以義父愛已之深,勢必要和波斯三使拚個同歸於盡,以維護自己,當下説道︰「在下姓曾,名阿牛。謝大俠還不遠走,難道是信不過在下,怕我吞没你這柄寶刀麼?」謝遜哈哈大笑,説道︰「曾少俠不必以言語相激。你我肝膽相照,謝遜以垂暮之年,得交你這朋友,實是生平快事。曾少俠,我要以七傷拳打那女子。我一發勁,你撒手棄了屠龍刀。」張無忌知道義父七傷拳的厲害,只要捨得將屠龍刀棄給敵人,一拳便可斃了輝月使,但這麼一來,本教使和波斯總教結下深怨,自己一向諄諄勸誡同教兄弟,務當必和睦爲重,今日自己竟不問來由的殺了總教使者,那裡還像個明教教主?當下忙道︰「且慢!」向流雲使道︰「咱們暫且罷手,在下有幾句話跟三位説明白。」

流雲使點了點頭。張無忌道︰「在下和明教極有関連,三位既持聖火令來此,乃是在下的尊客,適纔無禮,多有得罪。咱們同時各收内力,罷手不鬥如何?」流雲使又連連點頭。張無忌大喜,當時内勁一撤,將屠龍刀收向胸前。只覺波斯三使的内勁同時後撤,突然之間,一股陰勁如刀、如劍、如匕、如鑿,直插入他胸口的「玉堂穴」中。

\chapter{一往情深}

這雖是一股無形無質的陰寒之氣,但刺在張無忌身上,實同鋼戮之利。無忌霎時間,閉氣窒息,全身動彈不得,心中閃電般轉過了無數念頭︰「我死之後,義父也是難逃毒手,想不到波斯總教所遣的使者,竟是如此不顧信義。我那殷離表妹能活命麼?趙姑娘和周姑娘怎樣?小昭,唉,這可憐的孩子!本教救民抗元的大業終將如何?」只見流雲使舉起右手聖火令,便往他天靈蓋上擊將下來。無忌急運内力,衝擊胸口被點中了的「玉堂穴」,總是緩了一步。

忽聽得一個女子聲音大聲叫道︰「中土明教,大隊人馬到了!」流雲使一怔,舉著聖火令的左手停在半空,一時不擊下去。只見一個灰影電射而至,拔出無忌腰間的倚天劍,連人帶劍,直撲入流雲使的懷中。無忌身子雖然不能轉動,眼睛却是瞧得清清楚楚,這人正是趙明,大喜之下緊接著便是大駭,原來趙明所使這一招乃是崑崙派的殺招,叫做「玉碎崑崗」,竟是和敵人同歸於盡的拚命打法。無忌雖不知此招的名稱,却知趙明如此使劍出招,以倚天劍的鋒利,流雲使固當傷在她的劍下。她自己也難逃敵人的毒手。

流雲使初和中原高手過招,在張無忌手下討不到好去,迫得以奸詐取勝,接著便遇到這個不男不女的人物,眼見劍勢凌厲之極,别説三使聯手,即是自保也已有所不能,危急之中,舉起聖火令用力一擋,跟著不顧死活的著地滾了開去。只聽得{\upstsl{噹}}的一聲響,聖火令已將倚天劍架開,但左頰上涼颼颼地,一時也不知自己是存是亡,待得站起身來,伸手一摸,只覺著手處又濕又黏,疼痛異常,原來左頰上一片虯髯已被倚天劍連皮帶肉的削去,若非聖火令乃是奇物,擋得了倚天劍的一擊,半邉腦袋已被削去。

趙明這一招雖然得手,但倚天劍圏了轉來,削去了自己半邉帽子,露出一叢秀髮。原來當張無忌前來和謝遜相會之時,她思前想後,總覺金花婆婆詭祕多詐,陳友諒形跡可疑,這靈蛇島上隱伏著無數危機,越想越是放心不下,便悄悄的跟隨前來。她知道自己輕功未臻上乘,只要略一走近,立時便被發覺,是以只是遠遠躡著。直至無忌出手和波斯三使相鬥,她纔走近。到得無忌和三使比拚内力之時,她芳心暗喜,心想這三個胡人武功雖怪,那裡及得上無忌九陽神功内力的渾厚,突然間無忌開口叫對手罷鬥,趙明心思機靈,正待叫無忌小心,對方的「陰風刀」已然使出,無忌受傷倒地,趙明情急之下,不顧一切的衝出,情知以無忌武功之高,尚且敵不過這三個胡人,自己如何是他們對手?此時不及細想,搶到倚天劍後,便將在萬法寺中向崑崙派學得的一記拚命招數使出來了。

她一擊得手,長劍向妙風使撲出,倚天劍反而跟在身後。連一招叫做「人鬼同途」,乃是崆峒派的絶招,正和崑崙派的玉碎崑崗同一其理,均是趙明知已然輸定,便和敵人拚個「玉石倶焚」。這種打法極其慘烈,少林、峨嵋兩派的佛門武功便無此類招數。須知「玉碎崑崗」和「人鬼同途」都不是敗中取勝,死中求活之招,乃是兩敗倶傷,同赴幽冥之招。當日崑崙、崆峒兩派的高手被趙明囚在萬法寺中,頗受屈辱,比武時功力拚起,却被趙明一一記在心中。

妙風使一見她來勢如此兇悍,大驚之下,突然間全身冰冷,呆立不動。原來此人武功雖高,膽子却是極小,生平戰無不勝,從未遇到如此無法抵擋的劍招,駭佈達於極點,竟致僵立,唯有束手待斃。

趙明的身子已抵在妙風使的聖火令上,手腕一抖,長劍眼看便向她胸前刺去。原來這一招乃是先以自己身體投向敵人兵刃,敵人手中不論是刀是劍,是槍是斧,中在自己身上,勢須略一停留,自己便一劍刺去,敵人武功再高,萬難逃過,妙風使瞧出了此招的厲害,這纔嚇呆。幸好他所用兵器乃是鐵尺般的聖火令,無鋒無刃,趙明以身體抵在其上,竟不受傷,長劍剛向前刺出,後背已被輝月使抱住。

波斯三使聯手迎敵,配合之妙,舉世無儔。趙明一上來兩招拚命打法,竟嚇得三大高手亂了陣脚,直到此時,輝月使自後面抱住了趙明,别瞧她這麼一抱似乎平平無奇,其實拿捏之準,不爽毫髮,應變之速,疾如流星。趙明這一劍雖然凌厲,已然遞不到妙風使身上,她但覺手臂一緊,心知不妙,順著輝月使向後一拉之勢,一劍便往自己小腹刺去。

這一招更是壯烈,屬於武當派劍招,叫做「天地同壽」,却非張三丰所創,乃是殷利亨苦心孤詣的想了出來,本意是要和楊逍同赴地府之用。他自紀曉芙死後,心中除了殺楊逍報仇之外,再無别念,但自知武功非楊逍之敵,師父雖是天下第一高手,自己限於資質悟性,無法學到師父的三四成功夫,反正只求殺得楊逍,自己也不想活了,是以在武當山上想了三招拚命的打法出來。他暗中練劍之時,被張三丰見到,張三丰喟然歎息,心知此種事情難以勸喩,只是將這招劍法取了個「天地同壽」的名稱,意思説人死之後,精神不杇,當可萬古長春,實是殺身成仁、捨生取義的悲壯劍招。殷利亨的大弟子在萬法寺中施展此招,被苦頭陀搶上救出,趙明却在此時使了出來。原來這一招專爲刺殺緊貼在自己身後的敵人之用,利劍穿過自己的小腹,再刺入敵人小腹,輝月使如何能彀躱過?倘若妙風使並未嚇傻,或是流風雲使站得甚近,以他二人和輝月使如同聯成一體的機警,當可救得二女性命。

但事與願違,眼見倚天劍便要洞穿趙明和輝月使的小腹,便在這千鈞一髮之際,張無忌衝穴功成,一伸手便將倚天劍奪了過去。

趙明用力一掙,脱出輝月使的懷抱,她動念迅速之極,取過張無忌手中的那枚聖火令,遠遠的擲了出去,叮的一聲響,跌入了金花婆婆所佈的尖針陣中。這聖火令波斯三使珍同性命,流雲使和輝月使顧不得再和張無忌、趙明對敵,甚至顧不得妙風使的安危,一齊縱身過去撿拾。但只奔出丈餘,便已到了尖針陣中。月黑風高,長草没膝,瞧不清楚聖火令和尖針的所在,兩人只得一路拔針,一路摸索尋令。妙風使猶如大夢初醒,一聲驚呼,跟了過去。

趙明爲救張無忌的性命,適纔這三招使得猶如兔起鶻落,絶無餘暇多想一想,這時驚魂稍定,越想越是害怕,「嚶」的一聲,投入了無忌懷中。無忌一手攬著她,心中説不出的感激,但知波斯三使一尋到聖火令,立時轉身又回,忙道︰「咱們快走!」回過身來,抱起身受重傷的殷離,向謝遜道︰「謝大俠,眼前只有暫避其鋒。」謝遜道︰「是!」俯身替金花婆婆解開了穴道。無忌心想金花婆婆經過這場死裡逃生的大難,自和謝遜前愆盡釋。四個人下山走出數丈,無忌心想殷離雖是自己表妹,終是男女授受不親,於是將她交給金花婆婆抱著。趙明在前引路,其後是金花婆婆和謝遜,無忌斷後,以防敵人追擊。回首但見波斯三使兀自彎了腰,在長草叢中尋覓。無忌這一役慘敗,想起適纔驚險,不禁心有餘悸,又不知殷離受此重傷,是否能彀救活。正行之間,忽聽得謝遜一聲暴喝,一拳向金花婆婆後心打了過去。

只見金花婆婆回手一撩,掠開了謝遜這一拳,已將殷離抛在地下,張無忌吃了一驚,飛身而上,但聽謝遜喝道︰「韓夫人,你何以又要下殺手害殷姑娘?」金花婆婆冷笑道︰「你殺不殺我,是你的事。我殺不殺她,却是我的事,你管得著我麼?」張無忌道︰「有我在此,須容不得你隨便傷人。」金花婆婆道︰「尊駕今日的閒事管得還嫌不彀麼?」張無忌道︰「那未必都是閒事。波斯三使轉眼便來,你還不快走?」金花婆婆冷哼一聲,向西竄了出去,突然間反手擲出三朶金花,直奔殷離後腦。張無忌伸指彈去,只聽得呼呼呼三聲,那三朶金花回襲金花婆婆,破空之聲,比之強弓發硬弩更加厲害。金花婆婆没料到這少年的内力竟是如此深厚,不敢伸手去接,急忙伏地而避。那三朶金花貼著她背心掠過,將她布衫後心整整齊齊的撕去了三條大縫。只嚇得她心中亂跳,頭也不回的去了。

張無忌伸手抱起殷離,忽聽得趙明一聲痛哼,彎下了腰,雙手按住小腹。無忌道︰「怎麼了?」踏上兩步,見她纖纖素手之上,滿是鮮血,手指縫中尚不住有血滲出,原來適纔這一招天地同壽,畢竟還是刺傷了她小腹。無忌心下甚驚,忙問︰「傷得重麼?」只聽得妙風使在尖針陣中歡呼︰「找到了,找到了!」趙明道︰「别管我!快走,快走!」無忌一伸臂,將她也抱了起來,邁開大步,便往山下奔去。趙明道︰「到船上!開船逃走。」張無忌應道︰「是!」一手抱著殷離,一手抱著趙明,足底竟是絲毫不緩,疾馳下山。謝遜在他身後迴護,心下暗自驚異︰「這少年恁地了得,手中抱著二人,竟比我奔得還快。」無忌心亂如麻,手中這兩個少女只要有一個傷重不救,都是畢生的大恨,幸好覺得二人身子都尚溫暖,並無逐漸冷去之象。

那波斯三使找到聖火令後,隨後追來,但這三人的輕功固然不及無忌,比之謝遜也有不迨。張無忌將到船邉,高聲叫道︰「明明郡主有令︰衆水手張帆起錨,急速預備開航!」待得他和謝遜躍上船頭,風帆已然開起。那梢公須得趙明親口號令,上前請示。

趙明失血已多,只低聲道︰「聽\dash{}聽張公子號令\dash{}便是\dash{}」那梢公轉舵開船,待得波斯三使追到岸邉,海船離岸早已數十丈了。

張無忌將趙明和殷離並排放在船艙之中,小昭在旁相助,解開二人衣衫,露出傷口。無忌檢視二人傷勢,見趙明小腹上劍傷深及寸許,流血雖多,性命決可無礙。殷離三朶金花却都中在要害。金花婆婆下手極重,是否能救,實在難説,當下給二人敷藥包紮。殷離早已昏迷不醒,人事不知。趙明泪水盈盈,無忌問她覺得如何,她只是咬牙不答。

謝遜道︰「曾少俠,謝某隔世爲人,不意回到中土,尚能結識你這位義氣深重的朋友。」無忌扶他坐在艙中椅上,伏地便拜,哭道︰「義父,孩児無忌不孝,没能早日前來相接,累義父受盡辛苦。」謝遜大吃一驚,道︰「你\dash{}你説什麼?」無忌道︰「孩児便是張無忌。」謝遜如何能信,只道︰「你\dash{}你説什麼?」無忌道︰「拳學之道在凝神,意在力先方制勝\dash{}」滔滔不絶的背了下去,每一句都是謝遜在冰火島上所授於他的武功要訣。背到百餘句後,謝遜驚喜交集,抓住他的雙臂,道︰「你\dash{}你當眞便是我那無忌孩児?」無忌站起身來,摟住了他,將别來情由,揀要緊的説了一些,自己任明教教主之事,却暫且隱忍不説,以免義父敘教中尊卑,反向自己行禮。謝遜如在夢中,此時不由得他不信,只是翻來覆去的説道︰「老天爺開眼,老天爺開眼!」猛聽得後梢上衆水手叫道︰「敵船追來啦!」

張無忌奔到後梢一望,只見遠遠一艘大船,五帆齊張,乘風追至。黑夜之中瞧不見敵船船身,那五道白帆却是十分觸目。無忌叫道︰「熄燈!」順手拾起梢公喝茶的茶碗,對準桅桿頂上的風燈{\upstsl{砸}}去。嗆{\upstsl{噹}}的一響,風燈熄滅,四下裡登時漆黑一團,只是那風帆既大且白,苦於又不能收蓬。無忌望了一會,見敵船帆多身輕,越逼越近,心下焦急,不知如何是好,暗想只有讓波斯三使上船,跟他們在船艙之中相鬥,當可藉著船艙狹窄之便,使三人不易聯手,作爲障礙,逼令波斯三使各自爲戰。

佈置方定,突然間轟隆一聲巨響,船身猛烈一側,若非艙中諸人個個武功高強,幾乎站立不穩,跟著半空中海水傾潟,直潑進艙來。後梢水手高聲大叫︰「敵船開炮!敵船開炮!」原來這一炮打在船側,幸好並未擊中。趙明向無忌招了招手,無忌低頭道︰「别怕!」趙明聲音微弱,道︰「咱們也有炮!」這一言提醒了張無忌,當即奔上甲板,指揮水手搬開炮上的掩蔽之物,在大炮中裝上火藥鐵彈,點燃藥繩,砰的一聲,一炮還轟了過去。只是這些水手都是趙明手下武士喬裝,武功均雖不弱,發炮海戰却是一竅不通,這一炮轟將出去,落在兩船之間,水柱激起數丈,敵船可是晃也不晃。但這麼一來,敵船見此間有炮,倒是不敢十分逼近。過不多時,敵船又是一炮轟來,正中船頭,船上登時起火。

無忌忙指揮水手,提水救火,忽見上層艙中又冒出一個火頭來。無忌雙手各提一大桶水,踢開艙門,直潑進去,將火頭澆滅了。煙霧中只見一個女子橫臥榻上,正是周芷若,全身都已濕透。無忌抛下水桶,搶進房去,忙道︰「周姑娘,你没事麼?」周芷若點了點頭,只是滿頭滿臉都是水,模樣甚是狼狽,見到無忌突然出現,驚異無比。她雙手一動,嗆{\upstsl{啷}}{\upstsl{啷}}一聲響,原來手脚均被金花婆婆用{\upstsl{銬}}鐐鐵鍊鎖著。無忌到下層艙中取過倚天劍來,削斷{\upstsl{銬}}鐐。周芷若道︰「張教主,你\dash{}你怎麼會到這裡。」無忌還未回答,船身突然間激烈一震。周芷若被{\upstsl{銬}}鐐鎖得久了,手脚都已麻木,足下一軟,直撲在無忌懷裡,無忌忙伸手扶住,窗外火光昭耀,只見她蒼白的臉上飛起兩片紅暈,再點綴著一點點水珠,竟似一株水仙花般清雅秀麗。無忌定了定神,説道︰「咱們到下面船艙去。」兩人剛走到艙門,只覺船隻不住的團團打轉,原來適纔間敵船一炮打來,竟將此船的後舵打得粉碎,連舵手也墜海而死。

那梢公急了,親自去裝火藥發炮,只盼一炮將敵船打沉,不住在炮筒中裝填火藥,用鐵棍樁得實實的,舉起火炮,點燃了藥繩。驀地裡紅光一閃,震天價一聲大響,鋼鐵飛舞,那大炮登時震得粉碎。梢公和大炮旁的衆水手個個炸得血肉橫飛。原來那梢公一味求炮力威猛,火藥裝得多了數倍,炮彈射不出去,反將大炮炸碎。

張無忌和周芷若正走在甲板之上,只覺一般熾烈無比的熱氣衝來,將兩人抛出甲板。無忌想也不想,右手伸出,抓住了一根帆索,左手剛好抓住周芷若的小腿,兩人才算没有落海。但見船上到處是火,轉眼即沉,一瞥眼,見左舷邉縛著一條小船,叫道︰「周姑娘,你跳進小船去\dash{}」這時小昭抱著殷離,謝遜抱著趙明,先後從下層艙中出來。原來適纔這麼一炸,船底炸了一個大洞,海水立時湧了進來。無忌待謝遜小昭一齊坐進小船,揮劍割斷綁縛的繩索,拍的一響,小船掉入海中。無忌湧身輕輕一躍,跳入小船,搶過雙槳,用力划動。

這時那海船燒得正旺,照得海面上一片通紅,張無忌心想只須將小船划到火光照耀不到之處,波斯三使没有見到小船,必以爲衆人盡數葬身大海,就此不再追趕,當下全力扳槳。謝遜抄起一條船板幫著划水,那小船如箭向前飛馳,頃刻間出了火光圏外。只聽那大海船轟隆、轟隆的猛響,船上藏著的火藥不住爆炸,波斯三使的座船追到時不敢逼近,只是遠遠的停著監視,趙明擕來的武士中有幾名識得水性,泅水前往求救,都被波斯三使一一擊死在海中。

張無忌和謝遜片刻也不敢停手,須知若在陸地之上,眞被波斯三使追及,不得已時尚可決一死戰。這時在茫茫大海之中,敵船只須一炮轟來,便是打在離小船數丈以外,{\upstsl{潻}}浪激盪,這小船也是非翻不可。好在二人都是内力修長,直划了半夜,也不疲累。

到得天明,但見滿天烏雲,四下裡都是灰濛濛的濃霧。無忌喜道︰「這大霧來得眞好,只須再有半日,敵人無論如何也找咱們不到的了。」只是其時正當隆冬,各人身上衣衫盡濕,張無忌和謝遜内力深厚,還不算怎樣,周芷若和小昭被北風一吹,忍不住牙関打戰。但小船上一無所有,誰也無法可想。無忌和謝遜早已脱下外衣,蓋在趙明和殷離身上。不料屋漏又逢連夜雨,到得下午,狂風大作,大雨如注,那小船被風力所帶,向南飄浮。木槳早已收起不划,四個人除下八隻鞋子,拚命將船艙中所積的雨水潑到海中。謝遜終於會到無忌,心情極是暢快,眼前處境雖險,却是毫不在意,罵天叱海,在大雨中高聲談笑。小昭天眞瀾漫,竟也是言笑晏晏。只有周芷若自始自終默不作聲,偶而和無忌目光相接,立即便轉頭避開。

謝遜説道︰「無忌,當年我和你父母同乘海船出洋,中途遇到風暴,那可比今日厲害得多了。咱們後來上了冰山,以海豹爲食。只不過當日吹的是南風,把我們送到了極北的冰天雪地之中,今日吹的却是北風。難道老天爺瞧著謝遜不順眼,要再將我充軍到南極仙翁府上,去住他二十年麼?哈哈,哈哈!」他大笑一陣,又道︰「當年你父母一男一女,郎才女貌,正是天作之合,你却帶了四個女孩子,那是怎麼一回事啊?哈哈,哈哈!」周芷若滿臉通紅,低下了頭。小昭却是神色自若,説道︰「謝老爺子,我是服侍公子爺的小丫頭,不算在内。」趙明受傷雖然不輕,却是一直醒目,突然説道︰「謝老爺子,你再胡説八道,等我傷好了,瞧我不老大耳括子打你。」謝遜伸了伸舌頭,笑道︰「你這女孩子倒厲害。」他突然收起笑容,沉吟道︰「{\upstsl{嗯}},昨晩你拚命三招,第一招是崑崙派的『玉碎崑崗』,第二招是崆峒派的『人鬼同途』,第三招是甚麼啊,老頭子孤陋寡聞,可聽不出來了。」趙明暗暗心驚︰「怪不得這位金毛獅王當年名震天下,鬧得江湖上天翻地覆。他雙目不能視物,却能猜到我所使的兩記絶招,當眞名不虛傳。」便道︰「這第三招是武當派的『天地同壽』,似乎是新創招數,難怪老爺子不知。」謝遜嘆道︰「你出全力相救無忌,當然很好,可是又何必拚命,又何必拚命?」趙明道︰「他\dash{}他\dash{}」説到此處,頓了一頓,心中遲疑下面這句話是否該説,終於忍不住哽咽道︰「他\dash{}誰叫他這般情緻纏綿的\dash{}抱著\dash{}抱著殷姑娘。我是不想活了。」説完這句話,已是泪下如雨。四人一聽之下,無不愕然,誰也没想到這位年輕姑娘竟會當衆吐露心事。殊不知趙明是蒙古少女,本和中土深受禮教陶冶的女子大異,要愛便愛,要恨便恨,絶無絲毫忸怩作態,加之扁舟浮海,大雨當頭,每一刻都能舟覆人亡,誰都不知究竟還能活多少時候。

趙明這句話,一個字一個字的送入張無忌耳中,使他心情大是激盪,心想︰「趙姑娘説來是我的大敵,這次我隨她遠赴海外,主旨乃在迎接義父,那想到她對我竟是一往情深如此。」情不自禁,伸過手去握住了她的手,嘴唇湊到她的耳邉,低聲道︰「下次無論如何,不可以再這樣了。」趙明當衆吐露心,話児一説出口,心中已是好生後悔,心想女孩児家没遮攔,這種言語如何可以自己説將出來,豈不是讓他輕賤於我?忽聽張無忌如此深情款款的叮囑自己,不禁又驚又喜,又羞又愛,心下説不出的甜蜜,自覺昨晩三次出生入死,今日海上飄泊受苦,一切都不枉了。

大雨下了一陣,漸漸止歇,濃霧却是越來越重,驀地裡刷的一聲,一尾三十來斤的大魚從海中躍將起來。謝遜右手伸出,五指插入魚腹,將那魚抓入船中,衆人都是喝一聲采。小昭拔出長劍,將大魚部腹刮鱗,切成一塊塊地。各人實在餓了,雖是生魚腥味極重,只得勉強也吃了一些,謝遜却是特别吃得津津有味,他荒島上住了二十餘年,什麼苦也吃過了,豈在乎區區生魚?何況生魚肉只須多嚼一會,慣了魚腥氣息之後,自有一股鮮甜的味道。海上波濤漸漸平靜,大家吃魚後閉上眼養神,小昭第一個先行睡著。趙明握著無忌的手不放,過了一會,心中平安,也慢慢的睡去了。各人昨天這一日一晩的激鬥,眞是累得心力交疲,周芷若和小昭雖未出手接戰,但所受驚嚇也著實不小。大海輕輕晃小舟,有如搖藍,舟中六個人先後入睡。

這一場好睡,足足有四五個多辰。謝遜年老先醒,耳聽得五個青年男女緩緩的呼吸之聲,和海上風聲相應和。趙明和殷離受傷之後,氣息較促,周芷若却是輕而漫長。張無忌一呼一吸之際,若斷若續,竟無明顯分界,謝遜聽得暗暗驚異︰「這孩子内力之深,實是我生平從所未遇。」小昭的呼吸一時快,一時慢,和常人大不相同,顯是練著一種極特異的内功,謝遜眉頭一皺,想起一事,心道︰「這可奇了,難道這孩子竟是\dash{}」忽聽得殷離喝道︰「張無忌,你這臭小子,幹麼不跟我上靈蛇島去?」無忌、趙明、周芷若、小昭等被她這麼一喝,一齊驚醒。只聽她又道︰「我獨個児在島上寂莫孤單\dash{}你幹麼不肯來陪我?你\dash{}你這臭小子,我一劍宰了你\dash{}把你斬成十七八塊,丟到海中餵魚,你\dash{}你\dash{}」無忌伸手一摸她的額頭,竟是著手火燙,知她重傷後發燒,説起胡話來了,無忌雖然醫術通神,但小舟中無湯無藥,實是束手無策,只得撕下一塊衣襟,浸濕了水,貼在她的額上。殷離胡話不止,忽然大聲驚醒︰「爹爹,你\dash{}你别殺媽媽,别殺媽媽!二娘是我害的,你只管殺我好了,跟媽媽毫不相干\dash{}媽媽死啦,媽媽死啦!是我害死了媽媽!嗚嗚嗚嗚\dash{}」哭得十分傷心,無忌柔聲道︰「蛛児,蛛児,你醒醒。你爹不在這児,不用害怕。」殷離怒道︰「是爹爹不好,我纔不怕他呢!他娶二娘、三娘?一個人娶了一個妻子難道不彀麼?爹爹,你三心兩意,喜新棄舊,娶了一個女人又娶一個,害得我媽好苦!你不是我爹爹,你是天下的負心男児,是大惡人!」

無忌聽得惕然心驚,只嚇得面青唇白。原來他適纔間剛做了一個好夢,夢自己娶了趙明,又娶了周芷若,殷離浮腫的相貌也變得美了,和小昭一起也都嫁了自己。在白天從來不敢轉的念頭,在睡夢中忽然都成爲事實,只覺得四個少女個個都好,自己都捨不得和她們分離。他安慰殷離之時,腦海中依稀還存留著夢中帶來的溫馨甜蜜意。

這時無忌聽到殷離惡毒地咒罵父親的言語,憶及昔日在西域光明頂上所見所聞,殷離因不忿母親受欺,殺死了父親的愛妾,自己母親因此自刎,以致舅父殷野王要手刃親生女児。這件慘不忍聞的倫常大變,皆因殷野王用情不專、多娶妻妾之故。他向趙明瞧了一眼,情不自禁的又向周芷若瞧了一眼,想起昨宵的綺夢,内心深處羞慚。

只聽殷離咕咕嚕嚕的説了一些囈語,忽然很苦楚的哀求起來︰「無忌,你跟我去啊,跟我去啊。你在我手背上這麼狠狠的咬了一口,我一點也不恨你。我會一生一世的服侍你、體貼你,當你是我的主人。你别嫌我相貌醜陋,我只要你喜歡,寧願散了全身的武功,棄去千蛛劇毒,跟我初見你時一模一樣\dash{}」這番話説得嬌柔婉轉,無忌那想到這位表妹行事任性異常,喜怒不定,怪僻乖張,内心竟是這般的溫柔,當日蝴蝶谷中一會,她居然會對自己情有獨鍾,如此的始終不忘,只聽她又道︰「無忌,我到處找你,走遍了天涯海角,聽不到你的消息,後來才知你已在西域墜崖身亡。我在西域遇到了個少年曾阿牛,他武功既高,人品又好,説過要娶我爲妻了。」趙明等都知曾阿牛便是無忌的化名,一齊向他瞧去。無忌滿臉通紅,狼狽之極,此時殷離神智昏迷,反不能阻止她不説,倘若出手點她啞穴,她重傷之際,於她身子有損,在趙明、周芷若、小昭三人異樣的目光注視之下,眞恨不得跳入大海,待殷離清醒之後這纔上來。

只聽得殷離喃喃又道︰「那個阿牛哥哥對我這樣説︰『姑娘,我誠心願意,娶你爲妻,只盼你别説我配。』他説︰『從今而後,我會盡力愛護你,照顧你,不論有多少人來跟你爲難,不論有多麼厲害的人來欺侮你,我寧可自己性命不要,也要保護你周全。我要使你心中快樂,忘去了從前的苦處。』無忌,阿牛哥哥的人品可比你好得多啦,他的武功比甚麼峨嵋派的滅絶師太都強。可是我心中自從有了你這個狠心短命的小鬼,便没答應跟他。你短命死了,我便給你守一輩子的活寡。無忌你説,阿離待你好不好啊?當年你不睬我,現在心裡可後悔不後悔啊?」

無忌初時聽她複述自己對她所説的言語,只覺十分{\upstsl{尷}}尬,但後來越聽越是感動,禁不住泪水涔涔而下,只聽殷離輕輕的説道︰「無忌,你在幽冥之中,寂寞麼?孤單麼?我跟婆婆到北海冰火島去找到了你的義父,再要到武當山去掃祭你父母的墳墓,然後到西域你喪生的雪峰上跳將下去,伴你在一起。不過那是要等婆婆百年之後,我不能先來陪你,撇下她孤零零的世上受苦。婆婆待我很好,苦不是她救我,我早給爹爹殺了。我爲了你義父,背叛婆婆,她一定恨我得緊。」在她心中,無忌早已是陰世爲鬼,但無忌本人却明明坐在她身旁。她傷中昏迷,本該語無倫次,前言不接後語,但亂七八糟的瞎説一頓後,跟著的説話便有條有理,和好人無異,這般和一個鬼魅溫柔軟語,海上月明,靜夜孤舟,聽來實是十分的淒迷。要知殷離十年來這般自言自語的慣了,只須一有空間,便獨個児和心目中的無忌説話,吐露心事。她半生説的便是這種話,已是熟極而流,精神一失節制,自然而然的便説出口來。

她接下去的説話却又是東一言,西一語的不成連貫,有時驚叫,有時怒罵,這少女年紀雖輕,心中却已壓抑了無窮的無盡的愁苦。這樣亂叫亂喊好一陣,終於聲音漸低,迷迷糊糊的又睡著了。

\chapter{美若天仙}

五人相對不語,各自想著各人的心事,波濤輕輕打著小舟,只覺清風明月,萬古常存,人生憂患,實是無窮。忽然之間,一聲極溫柔、一聲極細緻的歌聲散在海上︰「到頭這一身,難逃那一日。百歳光陰,七十者稀。急急流年,滔滔逝水。」却是殷離在睡夢中低聲唱著小曲。

曲聲入耳,張無忌心頭心凜,記得在光明頂上祕道之中,出口被成崑堵死,眼見無法脱身,小昭也曾唱過這個曲子,不禁向小昭望去。目光下只見小昭正自痴痴瞧著自己,和他目光一相對,立時轉頭避開。殷離唱了這幾句小曲,接著大唱起歌來,這一回的歌聲却是説不出的詭異,和中土的曲子截然不同,竟如初聞聖火令相擊時的震人心弦,細辨她的歌聲,辭意也和小昭所唱的相同︰「來如流水兮,逝如風;不知何處來兮何所終!」她翻翻覆覆的唱著這兩句曲子,越唱越低,終於歌聲隨著水聲風聲,消没無蹤。各人想到生死無常,一個人飄飄入世,實不知來自何處,不論你如何的英雄豪傑,到頭來不免一死,飄飄出世,又如清風之不知吹向何處,無忌只覺掌裡趙明的纖指如冰,微微顫動。

謝遜忽道︰「這首波斯小曲,是韓夫人教她的,二十餘年前的一天晩上,我在光明頂上也聽到過一次。唉,想不到韓夫人絶情如此,竟會對孩子痛下毒手。」趙明道︰「老爺子,韓夫人怎會唱波斯小曲,這是明教的歌児麼?」謝遜道︰「明教傳自波斯,這首波斯曲子,和明教有些淵源,却不是明教的歌児。這曲子是兩百多年前波斯一位最著名的詩人峨默做的,據説波斯人個個會唱。當日我聽韓夫人唱了這歌,心下難以自已,問起此歌的來歷,她曾詳細説給我聽。

\qyh{}其時波斯大哲野芒設帳授徒,門下有三位傑出的弟子。峨默擅於文學,尼若牟擅於政事,霍山擅於武功。三人意氣相投,相互誓約,他年禍福與共,富貴不忘。後來尼若牟青雲得意,竟做到教主的首相。他兩個舊友前來投奔,尼若牟請於教主,授了霍山的官職。峨默不願居官,只求一筆年金,以便靜居研習天文曆法。飲酒吟詩。尼若牟相待甚厚。

\qyh{}不料霍山雄心勃勃,不肯久居人下,陰謀叛變。事敗後結黨據山,成爲威震天下的一個宗派首領。該派專以殺人爲務,名爲依斯美良派,十字軍時,西域提起『山中老人』霍山,無不心驚色變。其時西域各國君王,喪生於『山中老人』手下者,不計其數。韓夫人言道,極西海外,有一大國,名曰英格蘭,該國國王愛德華得罪了山中老人,被他遣人行刺。那霍山手下武士武功卓絶,愛德華王的衛士抗禦不敵,國王身中毒刃,幸得王后捨身救夫,吸去國王傷口中毒液,國王方得不死。
\footnote{\footnotefon{}〔金庸按〕此事見〈新英國正史〉。}
霍山不顧舊日恩義,更遣人刺殺波斯首相尼若牟。這位波斯首相臨死之時,口吟峨默的詩句,那便是這兩句『來如流水兮逝如風,不知何處來兮何所終』了。
\footnote{\footnotefon{}〔金庸按〕峨默絶句傳諸後世者共一百零一首,我國有郭沫若等人的翻譯。}
韓夫人又道,後來『山中老人』一派的武功,爲波斯明教中人習得。波斯三使的功夫詭異古怪,當是這山中老人的一派相傳了。」

趙明問道︰「老爺子,這個韓夫人的性児,倒很像那個山中老人。你待她仁至義盡,她却陰謀加害於你。」謝遜嘆道︰「世人以怨報德,原是尋常得緊,豈足深怪?」趙明低頭沉吟半晌,忽然説道︰「這位韓夫人位列明教四王之首,武功却不見得高於老爺子啊。昨晩與波斯三使動手之際,她何以又不使千蛛絶戸手的毒招?」

謝遜道︰「千蛛絶戸手?韓夫人不會這等功夫啊。似她這等絶色美人,愛惜容顏過於性命,怎肯練這種功夫?」無忌、趙明、周芷若等都是一怔,心想金花婆婆相貌醜陋,從她目前的模樣瞧來,即使再年輕三四十歳,也決計談不上「絶色美人」四字,鼻低唇厚,四方臉蛋,耳大招風,這面型是改變不來的。趙明笑道︰「老爺子,我瞧金花婆婆美不到那裡去啊。」謝遜道︰「什麼?紫衫龍王美若天仙,二十餘年前乃是武林中第一美人,就算此時年事已高,當年風姿仍當彷彿留存\dash{}唉,我只是看不到吧了。」

趙明聽他説得鄭重,隱約覺得其中頗有蹊蹺,金花婆婆竟是明教四大護教法王之首的紫衫龍王,已是一奇,而這個醜陋佝僂的病驅,居然是當年武林中的第一位美人,更是令人難以置信。她沉吟一會,道︰「老爺子,你名震江湖,王盤山揚刀立威,天下莫不知聞,武功之高,那是不消説的了。白眉鷹王自創教宗,與六大門派分庭抗禮,角逐爭雄,垂二十年。青翼蝠王神出鬼没,那日在萬法寺中威嚇於我,要毀我容貌,今日思之,常有餘悸,金花婆婆武功雖高,機謀雖深,但要位列三位之上,未免不稱,却不知是何緣故?」謝遜道︰「那是殷兄、韋賢弟和我三人甘心情願讓她的。」趙明道︰「爲什麼?」突然格格一笑,道︰「因爲她是天下第一美人,英雄難過美人関,三位大英雄都甘心拜服於石榴裙下麼?」她是番邦女子,不拘尊卑之禮,心中想到,便肆無忌憚的對謝遜開起玩笑來。

那知謝遜並不著惱,反而嘆了口氣,道︰「甘心拜服於石榴裙下的,豈止三人?其時教内教外,盼獲黛綺絲之青睞者,便説一百人,只怕也是少了。」趙明道︰「黛綺絲?這就是韓夫人麼?這名字好怪。」謝遜道︰「她來自波斯,這是波斯名字。」無忌、趙明、周芷若都吃了一驚,齊聲道︰「她是波斯人麼?」謝遜奇道︰「難道你們都瞧不出來麼?」她是中國人和波斯女子的混種,頭髮和眼珠都是黑的,但高鼻深目,膚色如雪,和中原女子大異,一眼便能分辨。趙明道︰「不,不!她是塌鼻頭,{\upstsl{瞇}}著一對小眼,跟你所説的完全不同。張公子,你説是不是?」無忌道︰「是啊。難道她也像苦頭陀一樣,故意自毀容貌?」謝遜道︰「苦頭陀是誰?」無忌道︰「便是明教光明右使范遙。」當下將范遙自毀容貌,到汝陽王府去臥底之事,簡略説了一些。謝遜嘆道︰「范兄此舉,苦心孤詣,大有功於本教,實非常人所能。唉,這也出於韓夫人之所激啊。」趙明好奇心起,道︰「老爺子,你别吞吞吐吐的賣関子了,從頭至尾的説給咱們聽吧。」

謝遜「{\upstsl{嗯}}」了一聲,仰頭向天,怔怔的半晌,緩緩的道︰「二十餘年之前,那時明教在楊破天教主統領之下,好生興旺。這日光明頂上突然來了三位波斯胡人手持波斯總教教主手書,謁見楊教主。那書信中言道,波斯總教中有一位淨善使者,原是中華人氏,到波斯久居其地,加入明教,頗建功勛,娶了一個波斯女子爲妻,生有一女。這位淨善使者於一年前逝世,臨死時心懷故土,遺命要女児回歸中華。總教教主尊重其意,遣人將她女児送來光明頂上,盼中土明教善予照拂。

\qyh{}楊教主自是一口答應,請那女子進來。那少女一進廳堂,登時滿堂生輝,但見她容色照人,明艷不可方物。當她向楊教主盈盈下拜之際,大廳上左右光明使、三法王、五散人、五行旗使,無不震動。護送她來的三個波斯人在光明頂上留了一宵,翌日便即拜别。這位波斯艷女黛綺絲便在光明頂上住了下來。」

趙明笑道︰「老爺子,那時你對這位波斯艷女深深鍾情了,是不是,不用害羞,老老實實的説出來吧。」謝遜搖頭道︰「不!那時我正當新婚,和妻子極是恩愛,妻子又懷了孕,我怎會生有他念?」趙明「哦」了一聲,暗悔失言,她知謝遜的妻児均是爲成崑所殺,這時無意間提起,不免觸動他的心境,忙道︰「對啦,對啦!怪不得她説,當年她嫁與銀葉先生,光明頂人人反對,只有楊教主和你仍是待她很好。想來楊教主的夫人不但是位美人児,而且爲人很厲害,收得丈夫服服貼貼了。」謝遜點頭道︰「你所料不錯。楊教主慷慨豪俠,黛綺絲的年紀足可做他女児。何況波斯總教教主託他照拂,楊教主待她自是仁至義盡,決計不存歹意。楊夫人是教主的師妹,也就是我的師叔。楊教主、成崑、楊夫人是同門師兄妹,楊教主是我大師伯,當年指點過我不少武功,他老人家待我極好的。」成崑殺他全家這場血海深冤,雖然在他心底仇恨愈久愈深,但口中提到成崑的名字之時,却是淡淡的一言帶過,便似説到一個平常人一般。

趙明突然想起一事,問道︰「那位光明右使范遙,據説年青時是個美男子,他對黛綺絲一定是很傾心的了?」謝遜點頭道︰「那是一見鍾情,終於成爲銘心刻骨的相思。其實何止范兄如此,見到黛綺絲之美色而不動心的,只怕很少。不過明教教規很嚴,大家對楊教主又是敬而且畏,人人以禮自持,就是誰對黛綺絲致思慕之忱的,也都是未婚男子,不料黛綺絲容貌雖美,對任何男子都是冷若冰霜,絲毫不假以辭色,不論是誰對她稍露情意,便被她當衆痛斥一頓,令那人羞愧無地,難以下台。楊夫人有意替她撮合,想要她嫁與范遙爲室。黛綺絲竟是一口拒絶,説到後來,她竟當衆橫劍自誓,説她是決計不嫁人的,如果有逼她成婚,她是寧死不屈。這麼一來,衆人的心也都冷了。

\qyh{}過了半年,有一天海外靈蛇島來了一人,自稱姓韓,名叫千葉,是楊教主當年仇人的児子,上光明頂爲父報仇。衆人一看這姓韓的青年人貌不驚人,居然敢單身來向楊教主挑戰,無不哈哈大笑。但楊教主却神色嚴肅,接以大賓之禮,大排筵席的款待。原來楊教主當年和他父親一言不合動手,以一掌『大九天手』擊得他父親重傷。當時他父親言道,日後必報此仇,只是自知自己武功已無法再進,將來不是叫児子來,便是叫女児來。楊教主道︰不論是児子還是女児,他必奉讓三招。那人道︰招是不須讓的,但如何比武,却要他子女選定。楊教主當時便答應了。事過十餘年,楊教主早没將這事放在心上,那知這姓韓的果然遣他児子前來。

\qyh{}衆人都道︰善者不來,來者不善。此人竟敢孤身上得光明頂來,必有驚人的藝業。但楊教主功力之深,幾已達爐火純青之境,武林中再高的能手,也未必勝得他一招半式。這姓韓的能有多大年紀,便有三個五個一齊圍攻,楊教主也不肯放在心上。所擔心的只是不理他要出甚麼爲難的題目。

\qyh{}到第二天上,那韓千葉當衆説明昔日的約言,先把言語擠住楊教主,令他無從食言,然後説了題目出來。原來他要和楊教主同入光明頂上的碧水寒潭之中,一決勝負,輸了的當衆自刎。

\qyh{}他此言一出,衆人盡皆驚得呆了。須知那碧水寒潭冰冷澈骨,雖在盛暑,也是無人敢下,何況其時正當隆冬?楊教主武功雖高,却是不識水性,這一下到碧水寒潭之中,不用比武,凍也凍死了,淹也淹死了。只聽得議事廳中,群雄齊聲斥責。」

張無忌道︰「這件事當眞爲難得緊,大丈夫一言既出,駟馬難追。楊教主當年曾答應過那姓韓的,比武的方法由他子女選擇,這韓千葉選定水戰,按理説楊教主無法推諉。」趙明反握他的手掌,捏了一捏,輕輕笑道︰「是啊,大丈夫一言既出,駟馬難追,身爲明教教主之人,豈能食言而肥,失信於天下?答應了人家的事,總當做到。」

她這話説的是張無忌,再提一下二人之間的誓約。謝遜却那裡知道,便道︰「正是如此。當日韓千葉朗聲説道︰『在下孤身上得光明頂來,原没盼望能活著下山。衆位英雄豪傑儘可將在下亂刀分屍,除了明教之外,江湖上誰也不會知曉。在下只是個無名小卒,殺此區區一人,有何足道?各位要殺便殺,多言無益?』衆人一聽,倒是不便再説甚麼了?」

\qyh{}楊教主沉吟半晌,説道︰『韓児,在下當年確與令尊有約。好漢子光明磊落,這場比武是在下輸了。你要如何處置,悉聽尊便。』韓千葉手腕一翻,亮出一柄晶光燦爛的匕首,對準自己心臟,説道︰『這匕首是先父遺物,在下只求楊教主向這匕首磕上三個響頭。』群雄一聽,無不憤怒,堂堂明教教主,豈能受此屈辱?但楊教主既然認輸,按照江湖規矩,不能不由對方處置。眼前情勢已是十分明白,韓千葉此番是拚死而來,受了楊教主這三個頭後,他立即以匕首往自己心口一插,以免死於明教群豪的手下。

\qyh{}霎時之間,大廳中竟無半點聲息。光明左右使逍遙二仙、白眉鷹王殷兄、彭瑩玉和尚等人,平素均是足智多謀,但當此難題,却也是一籌莫展。韓千葉此舉,明明是要逼死楊教主,以報父親當年一掌之仇,然後自殺。便在這局勢緊張萬分之際,黛綺絲忽然越衆而前,向楊教主道︰『爹爹,他人生了個好児子,你難道便没生個好女児?這位韓爺爲他父親報仇,女児就代爹爹接他招數。上一代歸上一代,下一代歸下一代,不可亂了輩份。』衆人一聽,都是一愕︰『怎麼她叫楊教主作爹爹?』但立即會意︰『黛綺絲是冒充教主的女児,以解此厄。』但各人心中均想︰『瞧她嬌滴滴弱不禁風的模樣,不知是否會武?就算會武,未必能高,至於入碧水寒潭水戰,那是更加不必談起。』

\qyh{}楊教主尚未回答,韓千葉冷笑道︰『姑娘要代父接招,亦無不可。倘若姑娘輸了,在下仍是要楊教主向先父的匕首磕三個頭。』他眼見黛綺絲既美且弱,那裡將她放在眼下?黛綺絲道︰『若是尊駕輸了呢?』韓千葉道︰『要殺要剮,悉聽尊便。』黛綺絲道︰『好!咱們便去碧水寒潭!』説著當先便行。楊教主忙搖手道︰『不可!此事不用你牽涉在内。』黛綺絲道︰『爹爹,你不用擔心。』跟著便盈盈拜了下去,這一拜等於是拜楊教主作爲義父。

\qyh{}楊教主見她顯是滿有把握,兼之除此之外,亦無他法,只得聽她主張。當下衆人一齊來到山北的碧水寒潭。其時北風正烈,只到潭邉一站,已是寒氣逼人,内力稍差的便已覺得不大受用。潭水早已結成厚冰,望下去碧沉沉地,深不見底。楊教主心想不該爲了自己之事要黛綺絲爲他送命,昂然説道︰『乖女児,你這番好意,我心領了,我去接韓兄的高招。』説著除下外袍,取出一柄單刀,他是決意往潭中一跳,從此不再起來了。黛綺絲微微一笑,説道︰『爹爹,女児從小在海邉長大,精熟水性。』説著抽出長劍,飛身躍入潭中,站在冰上,劍往冰上劃了一個尺許見方的圏子,左足踢上,{\upstsl{噹}}的一聲輕響,已是踏低那塊圓冰。身子沉入了潭中。」

其時海上寒風北來,拂動各人的衣衫,謝遜説道︰「當時碧水寒潭之畔的情景,今日思之,便如是昨天剛過的事一般。黛綺絲那日穿了一套淡紫色的衣衫,她在冰上這麼一站,當眞勝如凌波仙子,突然間無聲無息的破冰入潭,旁觀群豪,無不驚異。那韓千葉一見黛綺絲入水的身手,臉上狂傲之色登時收起,手執匕首,跟著躍入了潭中。

\qyh{}那碧水寒潭色作深綠,從上邉望不到二人相鬥的情形,但見潭水不住晃動。過了一會,晃動漸停,但不久潭水又激盪起來。明教群豪都是極爲擔心,眼見他二人下潭已久,在水底豈能長久停留?又過一會,突然一縷殷紅的鮮血,從綠油油的潭水中滲將上來。衆人更是憂患,不知受傷的是韓千葉,還是黛綺絲。驀地裡忽喇一聲響,韓千葉從冰洞中跳了上來,不住的喘息。衆人見他先上,一齊大驚,齊問︰『黛綺絲呢?黛綺絲呢?』只見他空著雙手,他那柄匕首却反插在他左胸,兩邉臉頰上長長的各劃著一條傷痕。衆人正當驚異間,黛綺絲猶似飛魚出水,從潭中躍上,長劍護著身子,在半空中悠閒地轉了個圏子,這纔落在冰上。群豪歡聲大作。楊教主上前握住了她的手,高興得説不出話來。誰都不能料到,這樣千嬌百媚的一位姑娘,水底功夫竟是這般了得。黛綺絲向韓千葉瞧了一眼,説道︰『這人水性不差,念他爲父報仇的孝心,對教主無禮之罪,便饒過了吧?』楊教主自然答允,命神醫胡青牛替他療傷。

\qyh{}當晩光明頂上大排筵席,人人都説黛綺絲是明教大大的功臣,若非她挺身出來解圍,楊教主一世英名付於流水。當下安排職司,楊夫人贈了她一個『紫衫龍王』的美號,和鷹王、獅王、蝠王三王並列。咱們三王心甘情願,讓她位列四王之首,須知她今日這場大功,可將三王過去的功績都蓋下去了。

\qyh{}不料碧水寒潭這一戰,結局竟是大出各人意料之外,韓千葉雖然敗了,不知如何,竟是贏得了黛綺絲的芳心。想是黛綺絲每日前去探傷,病榻之畔,因憐生愛,從歉種情,等到韓千葉傷愈,黛綺絲忽然稟明教主,要下嫁這個青年。各人聽到這個訊息,有的傷心失望,有的憤恨填膺。這韓千葉是本教大敵,當日逼得本教自教主以下,人人狼狽萬狀,黛綺絲忽要嫁他,自然誰都不喜。有些脾氣粗暴的兄弟,當面便出言侮辱。黛綺絲的性子極是剛烈,仗劍站在廳口,朗聲説道︰『從今而後,韓千葉已是我的夫君。那一位侮辱韓郎,便來試試紫衫龍王長劍!』衆人見事情已是如此,只有恨恨而散。她與韓千葉的婚禮極是簡單,衆兄弟倒有一大半没去喝喜酒。只有楊教主和我感謝她這場解圍之德。出力助她排解,使她平安成婚,没出什麼岔子。但韓千葉想要入我明教,終於以反對的人太多,楊教主也不便過拂衆意。

\qyh{}事過不久,楊教主突然失蹤,光明頂上人心惶惶,衆人四下追尋之際,光明右使范遙竟見韓夫人黛綺絲從祕道出來。」張無忌一凜,道︰「她從祕道中出來?」謝遜道︰「不錯。明教教規極嚴,這祕道只有教主一人,方能去得。范遙驚怒之下,當即上前責問。韓夫人説道︰『我已犯了本教的重罪,要殺要剮,悉聽尊便。』當晩群豪大會,韓夫人仍舊是這幾句話,問她入祕道去幹什麼,楊教主到底去了何處,她説一槩不知,至於私入祕道之事,一人作事一身當,多説無益。按理她不是自刎,便當自斷一肢,但一來范遙舊情不忘,竭力替她遮掩,二來我在旁説情,群豪才議定罰她禁閉十年,以思己過。那知黛綺絲説道︰『楊教主不在此處,誰也管不著我。』」

張無忌問道︰「義父,韓夫人私進祕道却是爲何?」謝遜道︰「此事説來話長,明教之中,只我一人得知。當時大家疑心當與楊教主夫婦失蹤有関,但我力證絶無牽連。光明頂大廳之中,群豪三言兩語,越説越僵,終於韓夫人破門出教,説道自今而後,再與中土明教没有干係。她是最先倒出明教之人,即日與韓千葉飄然下峰,不知所蹤。此後教中衆兄弟尋覓教主不得,過了數年,爲爭教主之位,事情越來越僵。白眉鷹兄竟又破門,自創白眉一教。我苦苦相勸,他堅執不聽,哥児倆竟致翻臉。二十餘年前王盤山白眉教揚刀立威,金毛獅王趕去踢他場子,一來是衝著屠龍刀,二來也是爲了出一出當年的惡氣,存心要給殷兄下不了台,讓他知道離了明教之後,未必能成什麼大事。唉,今日思之,却未免太過意氣用事了!」

他長長一聲嘆息之中,蘊藏著無盡辛酸往事,無數江湖風波。

各人沉默半晌,趙明説道︰「老爺子,後來金花銀葉,威震江湖,怎地明教中人都認她不出麼?那銀葉先生,自必是韓千葉了,他又怎生中毒斃命?」謝遜道︰「這中間的經過情形,我便毫不知情。想是他夫婦二人在江湖上行道之時,儘量避開了明教中人。」張無忌拍腿道︰「不錯。金花婆婆從來不與明教中人朝相。六大派圍攻明教之時,她雖到了光明頂上,却不上峰赴援。」趙明沉吟道︰「可是紫衫龍王姿容絶世,怎能變得如此醜陋?那又不是臉上有什麼毀損。」謝遜道︰「依我猜測,她必是用什麼巧妙法児,改易了面容。要知韓夫人一生行事怪僻,其實她内心有説不出的苦處。她畢生在逃避波斯總教來人的追尋,那知臨到暮年,還是無法逃過。」無忌和趙明齊問︰「波斯總教何事尋她?」

謝遜道︰「這是韓夫人一件最大的祕密,本來是不該説的,但我盼望你們回去靈蛇島救她?那是非説不可。」趙明驚道︰「咱們再回靈蛇島去?鬥得過那波斯三使麼?」謝遜不答她的問話,自行敘述往事︰「千百年來,中土明教的教主例由男子出任,波斯總教的教主却一貫是女子。不但是女子,而且是不出嫁的處女。總教的經典中特别鄭重規定,由聖處女任教主,以維護明教的神聖貞潔。每位教主接任之後,便即選定教中高職人士的女児,稱爲『聖女』。此三聖女領職立誓,遊行四方,爲明教立功積德,當教主逝世,教中長老聚會,彙論三聖女所立功德高下,選定立功之聖女繼任教主。但若此三位聖女中有那一人失却貞操,便遭焚身之罰,縱然逃至天涯海角,教中也必遣人追拿,以維聖教貞善。」

他説到這裡,趙明失聲説道︰「難道那韓夫人是總教三聖女之一?」謝遜點頭道︰「正是!當范遙發見她入祕道之前其實我已先行發覺。韓夫人當我是個知己,便將事實眞相,一一告知我。她在碧水寒潭與韓千葉相鬥,水中肌膚相接,竟然情不禁,日後病榻相慰,終成冤孼。兩人成婚之後,她知總教有一日會遣人前來追査,只盼爲總教立一大功,以贖罪愆。她偸入祕道,爲的是找尋『乾坤大挪移』的武功祕譜。這是總教失傳已久的武功心法,中土明教却尚有留存。總教所以遣她前來光明頂,其意便在於此。」

張無忌「啊」的一聲,心中隱隱約約覺得有什麼情事頗爲不妥,但到底何事,一時却想不明白。只聽謝遜道︰「韓夫人數次偸入祕道,始終找不到這武功心法。我知悉後鄭重告誡她,此事犯我教中大規,實難寬容\dash{}」

趙明忽然插嘴道︰「啊,我知道啦。韓夫人所以決意破門出教,爲的是要繼續偸入祕道,她既然不是明教中人,再入祕道便不受什麼拘束了。」謝遜道︰「趙姑娘當眞聰明得緊。但光明頂是本教根本重地,豈容外人任意來去?當時我也猜到了她的用意,韓夫人下山之後,我親自守住祕道口,韓夫人曾私自上山三次,每次都見到我,這纔死了這條心。」他抬起了頭,似乎在想著一件什麼事,突然問道︰「那波斯三使的服色,和中土明教可有什麼不同麼?」張無忌道︰「他們都是身穿白袍,袍角上也繡有紅色火燄\dash{}{\upstsl{嗯}},白袍上滾著黑邉,這是唯一小小的不同。」謝遜一拍船舷,説道︰「是了。總教教主逝世。西域之人,以黑色爲喪服,白袍上鑲以黑邉,那是喪服。他們要選立新教主,是以萬里迢迢的到中土來追査韓夫人的下落。」

張無忌道︰「韓夫人既是來自波斯,必當知曉波斯三使怪異的武功,怎地不到一招,便給他們制住?」趙明笑道︰「你笨死啦。韓夫人這是假裝。她要掩飾自己的身份,自不能露出懂得波斯派武功。依我猜想,謝老爺子倘若聽從波斯三使的言語,下手殺她,韓夫人當有脱身之計。」謝遜搖頭道︰「她不肯顯示自己身份,那倒不錯。但説被波斯三使點中穴道之後,立即能彀脱身,却也未必。她是寧可被我一刀殺死,不願遭那烈火焚身之苦。」趙明道︰「我説中土明教的邪教,那知波斯明教更是邪得可以。爲什麼一定要處女來做教主?爲什麼要將失貞的聖女用火燒死?」謝遜斥道︰「小姑娘胡説八道。每個教派都有歷代相傳的規矩儀典。和尚尼姑不能婚嫁,不可吃葷,那也不是規矩麼?什麼邪不邪的?」

突然間格格聲響,殷離牙関互擊,不住寒顫。張無忌一摸她的額頭,却仍是十分燙手,顯是寒熱交攻,病勢極重,説道︰「義父,孩児也想回靈蛇島去。殷姑娘傷勢不輕,非覓藥救治不可。咱們盡力而爲,便救不得韓夫人,也當救了殷姑娘。」謝遜道︰「不錯。這位殷姑娘對你如此情意深重,焉能不救?周姑娘、趙姑娘你兩位意下如何?」趙明道︰「殷姑娘的傷是要緊的,我的傷是不要緊的。不回靈蛇島去那怎麼成。」周芷若淡淡的道︰「老爺子説回去,咱們便回去。」張無忌道︰「須待大霧散盡,見到星辰,始辨方向。義父,那流雲使連翻兩個空心斛斗,却能以聖火令傷我,那是什麼緣故?」當下兩人研討波斯三使武功的家數,趙明所學甚博,偶爾也參酌所見,但談了半天,終是摸不到三人聯手功夫的要旨所在。

海上大霧,直至陽光出來方散。無忌道︰「咱們自北方南來,現下該當向西北划去纔是。」他和謝遜、周芷若、小昭四人輪流划船。來時順風,此番以人力划回,實是大非易易,好在張無忌和謝遜固是内力深厚,周芷若和小昭也有相當修爲,扳槳划船,只當是鍛鍊武功,一連數日,一葉孤舟,破浪北行。

這幾日中,謝遜皺起了眉頭,苦苦思索波斯三使怪異的武功,除了向無忌詢問幾句之外,什麼話也不説。到得第六天傍晩,謝遜忽然仔細盤問周芷若所學到的峨嵋派功夫,周芷若據實以答。兩人一問一答,直談到深夜。謝遜神情之間,甚是失望,説道︰「少林、武當、峨嵋三派武功,均和九陽眞經有関,和無忌所學一般,都偏自陽剛一路。倘若張三丰眞人在此,以他陽剛陰柔無所不包的博大武學,與無忌聯手,那麼陰陽配合,當可擊潰波斯三使。但遠水救不著近火,韓夫人如落入波斯三使手中,那便如何是好?」周芷若忽道︰「老爺爺,聽説百年前武林之中,有些高人精通九陰眞經,可有這件事麼?」

\chapter{聖女教主}

張無忌在武當山上,曾聽太師父説起過「九陰眞經」之名,知道峨嵋派祖師郭襄女俠之父郭靖神鵰大俠楊過等人,都會得九陰眞經上的武功,但這類功夫太過難練,郭襄雖是郭靖的親生女児,却也未能學得,聽周芷若忽然問起,不禁心中一怔。謝遜道︰「故老相傳是這麼説,但誰也不知道眞假。聽前輩們説得神乎奇技,當今如果眞有誰學得這門武功,和無忌聯手應敵,波斯三使自是應手而除。」周芷若「{\upstsl{嗯}}」的一聲,便不再問。趙明問道︰「周姑娘,你峨嵋派有人會這路武功麼?」周芷若道︰「峨嵋派若是有人具此神功,先師也不會喪身於萬法寺中了。」滅絶師太間接死於趙明手下,周芷若對她痛恨已極,雖是日日夜夜風雨同舟,却從來跟她不交一語。此刻趙明正面相詢,便厲聲頂撞了她一句,她性格溫文,這般説話,已是生平對人最不客氣的言語了。

趙明却不以爲忤,只笑了笑。張無忌不住的扳槳,忽然望著遠處,叫道︰「瞧,瞧!那邉有火光。」各人順著他眼光望去,只見西北角上海天相接之處,微有火光閃動。謝遜雖是無法瞧見,心下却和衆人一般的驚喜,抄起木槳,用力划船。

那火光望去不遠,其實在大海之上,相隔有數十里之遙。兩人划了半天,纔漸漸接近。無忌一看火光所起之處,群山聳立,正是靈蛇島了,説道︰「咱們回來啦!」謝遜猛地裡「啊喲」一聲,叫了起來,説道︰「爲甚爲靈蛇島火光燭天?難道他們要焚燒韓夫人麼?」只聽得咕{\upstsl{咚}}一聲,小昭摔倒在船頭之上。張無忌吃了一驚,縱身過去扶起,但見她雙目緊閉,已然暈去。無忌忙拿捏她人中穴道,將她救醒,問道︰「小昭,你怎麼啦?」小昭雙目含泪,説道︰「我聽説要將人活活燒死,我\dash{}我\dash{}心裡害怕。」無忌安慰她道︰「這是謝老爺的猜測,未必眞是如此。就算韓夫人落入了他們手中,咱們立時趕去,説不定還能趕上相救。」小昭抓住他的手,求懇道︰「張公子,我求你求你,你一定要救韓夫人的性命。」無忌道︰「咱們大夥児盡力而爲。」説著回到船尾,提起木槳,划得比前更快了。

趙明忽道︰「張公子,有兩件事我想了很久,始終不明白,要請你指教。」無忌聽她忽然客氣起來,奇道︰「什麼事?」趙明道︰「那日在綠柳莊外,我遣人攻打令外祖、楊左使各位,是這位小昭姑娘調派人馬抵擋。當眞是強將手下無弱兵,明教教主手下一位小小的丫鬟,居然也有這等能耐,眞是奇了\dash{}」謝遜插口道︰「什麼明教教主?」趙明笑道︰「老爺子,這時候跟你説了罷,你那位義児公子,乃是堂堂明教教主,你反倒是他的屬下。」謝遜將信將疑,一時説不出話來,趙明便將無忌如何出任教主之事,簡略説了一些。但許多細節她也不知,無忌被謝遜追問得緊了,無法再瞞,只得説起六大派如何圍攻光明頂、自己如何在祕道中得獲乾坤大挪移心法等情。謝遜大喜,站起身,便在船艙之中拜到,説道︰「屬下金毛獅王謝遜參見教主。」

無忌急忙跪倒還禮,説道︰「義父不心多禮。楊教主遺命,請義父暫攝教主職位,孩児苦於不克負荷重任,天幸義父無恙歸來,實是本教之福。咱們回到中土之後,教主之位,原是要請義父接任的。」謝遜淒然道︰「你義父雖得歸來,但雙目已瞎,『無恙』兩字,是説不上的了。明教的首領,豈能由失明之人擔任?趙姑娘,你心中有那兩件事不明白?」趙明道︰「我想請問小昭姑娘,這些奇門八卦、陰陽五行之術,是誰教的?你小小年紀,怎地會了這一身出奇的本事?」

小昭道︰「這是我家傳的武功,不値郡主娘娘一笑。」趙明又問︰「令尊是誰?女児如此了得,父母必是名聞天下的高手。」小昭道︰「家父埋名隱姓,何勞郡主動問?難道你想削我幾根指頭,逼問我的武功麼?」莫看她小小年紀,口頭上對趙明竟是絲毫不讓,提到削指之事,更是意欲挑起周芷若敵愾同仇之心。趙明笑了笑,轉頭向無忌道︰「張公子,那晩咱們在大都小酒店中第二次敘會,苦頭陀范遙前來向我作别,他見到小昭姑娘之時,説了兩句什麼話?」張無忌早已將這件事忘了,聽她提起,想了一想,纔道︰「苦大師好像是説,小昭像那位他所相識的敵人。」趙明道︰「不錯。你猜苦大師説小昭姑娘像誰?」無忌道︰「我怎麼猜得到?」

説話之間,小船離靈蛇島更加近了,只見島西一排排的停滿了大船,白帆上繪了紅色火燄,每張帆上都掛著一根黑色飄帶。張無忌皺眉道︰「波斯總教勞師動衆,派來的人不少啊。」趙明道︰「咱們划到島後,揀個隱僻的所在登陸,别讓他們發見了。」無忌點頭道︰「是!」剛划出三四尺,突然間大船上號角聲鳴鳴吹動,砰砰兩響,兩枚炮彈打將過來,一枚落在船左,一枚落在船右,激起兩條水柱,小船晃得幾乎便要翻轉。大船上一人叫道︰「來船划將過來,如若不聽號令,立時轟沉。」無忌暗暗叫苦,心知適纔這兩炮乃是敵船意在示威,故意打在小船兩側,現下相距如此之近,敵人瞄準極易,當眞一炮轟在船中,六個人無一得免,只得划動小船,慢慢靠將過去。

只見三艘敵船的炮口緩緩轉動,始終對準著小船。待小船靠近,大船上放下繩梯。無忌道︰「咱們上去,相機奪船。」謝遜摸到繩梯,第一個爬上大船。周芷若一言不發,俯身抱起趙明,從繩梯攀上船去,跟著便是小昭,無忌抱了殷離,最後一個攀上。只見船上一干人個個黃髮碧眼,身材高大,均是波斯胡人,那流雲使等三使却不在其内。

一個會説華語的波斯人問道︰「你們是誰?到這裡來幹麼?」趙明道︰「咱們飄洋遇險,座船沉没,多蒙相救。」那波斯人將信將疑,轉頭向坐在甲板正中椅上的首領説了幾句波斯話,那首領向手下嘰哩咕嚕的吩咐幾句。小昭突然縱身而起,一掌向那首領擊去。那首領一驚,閃身避過,抓起坐椅,便向小昭{\upstsl{砸}}來。無忌没料到小昭這麼快便即動手,身形一側,欺上三尺,伸指便已將首領點倒。船上數十個波斯人登時大亂,紛紛抽出兵刃,圍了上來。這些人雖然個個身具武功,但與風雲三使相較,相去可是極遠。張無忌右手穩穩抱著殷離,左手東點一指,西拍一掌。謝遜使開屠龍刀,周芷若揮動長劍,再加上小昭身形靈動,片刻之間,已將船上數十名波斯人料理了。十餘人被砍翻在甲板之上,七八人墜入海中,餘下盡數被點中了穴道。霎時之間,海旁呼喊聲,號角聲亂成一片,其餘波斯船隻靠了過來,船上人衆便欲湧上相鬥。張無忌將殷離平放在甲板之上,提起那波斯首領,躍上桅桿,朗聲叫道︰「誰敢上來,我便將此人一掌劈死。」這人在總教中顯是頗有身份,只聽得船上衆人大聲呼喊,無忌雖是一句也聽不懂,但見無人躍上船來,想是此策生效,那些波斯人心存顧忌,一時不敢便來相攻。

無忌躍回甲板,剛放下那個首領,驀地裡背後錚的一聲響,一件兵刃{\upstsl{砸}}了過來。無忌急忙側身相避,反脚踢出,迎面聖火令擊倒,左側又有一根橫掠而至。無忌暗暗叫苦,心想風雲三使來得好快,叫道︰「大家退入船艙。」再提起那個首領,往一根聖火令迎了上去。

輝月使眼見聖火令這一擊正可擊到張無忌左肩,不意他突然會舉起這波斯首領的身子擋架,急忙收令。但如此突然收招,下盤露出空隙,被無忌一腿掃來,險險踢中她的小腿。流雲和妙風兩使自旁急攻,迫使無忌這一腿未能踢實。拆到第九招上,妙風使左手聖火令斜擊甩上,招數怪異無比,堪堪便要點中無忌小腹。張無忌將那波斯首領的身子一沉。妙風使這一招使得古怪,張無忌這一下却也是極其巧妙,只聽得拍的一聲響,這一記聖火令正好打在那波斯人的左頰之上。風雲三使齊聲驚呼,臉色大變,同時向後躍開,交談了幾句波斯話,突然躬身向無忌手中的波斯人行禮,神色極是恭敬。

原來無忌所擒獲的這個波斯首領,乃是波斯總教的十二「寶樹王」之一,號爲「平等王」。這十二寶樹王第一大聖,二者智慧,三者常勝,四者歡喜,五者勤修,六者平等,七者信心,八者忍辱,九者正直,十者功德,十一者齊心,十二者倶明。這十二寶樹王乃是教主座下的十二大經師,身份地位,相當於中土明教的四大護教法王。只是十二寶樹王以精研教義,精通經典爲主,除了第一大聖寶樹王、第三常勝寶樹王、第十功德寶樹王武功卓絶之外,其餘均是平平,比之風雲三使,頗有不及。這次波斯總教爲了尋覓聖女繼承教主之位,十二寶樹王齊來中土。「平等王」失手爲無忌擒獲。風雲三使上船搶救,反而一記聖火令打在平等王的臉上,雖非故意的以下犯上,究是令三人十分惶急。是以不敢再戰,行禮謝罪後即行退去。

無忌喘了一口氣,將平等王橫放在膝蓋之上,知道這人在波斯總教中地位極高,自己一干人脱險求生,勢非著落在他身上不可。俯首察看他臉上傷勢,但見他左頰高高衝起,幸好非致命之傷。想是妙風使一令擊出,已知不對,急忙收力,加之這人也有相當内功,頗有抵禦之勁。

周芷若和小昭收拾甲板上的衆波斯人,將已死的屍自搬入後艙,未死的一一排齊。只見十餘艘波斯大船四下圍住,各船上的大炮都準了無忌等人的座船,每一艘船的船邉上站滿了波斯人,火把照耀下刀劍閃爍,密密麻麻的不知有多少人。無忌暗暗心驚,别説各船開炮轟轟,這成千百人一湧而上,自己便有三頭六臂,也是難以抵擋,縱能仗著絶頂武功脱困,但無論如何不能保護得旁人周全。殷離和趙明身受重傷,更是危險。只聽得一名波斯人以華語朗聲説道︰「金毛獅王聽了,我總教十二寶樹王倶在此間,你得罪總教之罪,諸寶樹王寬於赦免。你速速將船上諸位總教教友獻出,自行開船去罷。」謝遜笑道︰「謝某又不是三歳小児,我一放俘虜,你們船上的大炮還不轟將過來嗎?」那人怒道︰「你就算不放,我們的大炮便不能轟嗎?」謝遜清嘯一聲,説道︰「聖女黛綺絲呢?你們讓她過來,咱們再談别的。」那人低頭和旁人商量了幾句,大聲道︰「黛綺絲犯了總教的大規,當遭焚身之刑,跟你們中土明教有什麼相干?」

謝遜沉吟道︰「我有三個條件,貴方答應了,我們便恭送這裡的總教教友上岸。」那人道︰「什麼條件?」謝遜道︰「第一,要十二寶樹王親口答允,此後明教相親相敬,互不干擾。」那人道︰「{\upstsl{嗯}}!第二呢?」謝遜道︰「你們釋放黛綺絲過船,免了她的失貞之罪,此後不再追究。」那人怒道︰「此事萬萬不可。第三件是什麼?」謝遜道︰「你第二件事也不能答應,何況再説第三件?」那人道︰「好!這第二件事就算允了,第三件不妨説來聽聽。」

謝遜道︰「這第三件嗎?那可易辨之至。你們派一艘小船,跟在我們的座船之後。駛出五十里後,我們見你們不派大船追來,便將俘虜放入小船,任由你們擕走。」那人大怒,喝道︰「胡説九道!胡説九道!」謝遜等都是一怔,不知他説些什麼。趙明笑道︰「此人學説中國話,可學得稀鬆平常。他以爲胡説九道比胡説八道多一道,那便更加荒唐了。」謝遜和張無忌一想不錯,雖然眼前局勢緊張,却也忍不住哈哈大笑起來。

這位在胡説八道上加了一道的人物,乃是第十二位倶明寶樹王。他聽見謝遜等嘻笑,更是惱怒,一聲忽哨,和第十一齊心王縱身躍上船來。無忌搶上前去,一掌推出,往那齊心王胸口推去。齊心王竟不擋架,伸左手往無忌頭頂抓下。無忌眼看自己這一掌要先打到他身上,那知倶明王從斜刺裡雙掌推到,接過了無忌這一掌,齊心王的手指却直抓下來。無忌向前衝了一步,方得避過。原來他二人攻守聯手,便如是個四手四腿之人一般。三個人迅如奔雷閃電般拆了七八招,無忌心下暗驚,這二人比之風雲三使,似乎稍有不及,但武功仍是十分怪異。明明和乾坤大挪移的心法極爲相似,可是一到使用出來,總是大大的變形,根本無法捉摸,然以招數凌厲巧妙而言,却又遠遠不及乾坤大挪移。似乎齊心、倶明二王是兩個瘋子,偶爾學到了一些挪移乾坤的武功,一來來學得不到家,二來神智迷糊,亂踢亂打,常人反倒不易抵禦,但兩人聯守之緊密,和風雲三使如出一轍。無忌勉力抵禦,只戰了個平手,預計再拆二三十招,方可佔到上風。

便在此時,風雲三使齊聲呼嘯,又攻上船來。他三人失手擊了平等王一令,心下甚感驚愧,只盼將他搶回,以功折罪。謝遜舉起平等王左右揮舞,劃成一個個極大的圏子。風雲三使這次如何敢貿然欺前?只是繞著半圓的圏子,想找尋空隙攻上。正門之際,忽聽得倶明王閃哼一聲,已被無忌一脚踢倒。無忌俯身待要擒拿,流雲使和輝月使雙令齊到,妙月使已抱起倶明王,躍回已船。這時齊心王和雲月二使聯手,配合已不如風雲三使嚴謹,加之記掛著倶明王的傷勢,接戰數合之下,眼見難以取勝,便即躍回。

無忌定了定神,説道︰「這一干人似乎學過挪移乾坤之術,偏又學得不像,當眞難以對付。」謝遜道︰「本教的乾坤大挪移心法,本是源於波斯。但數百年前傳入中土之後,波斯本國反而失傳,他們學得只是一些不三不四的皮毛,所以纔派黛綺絲到光明頂來偸回這門武功心法啊。」無忌搖頭道︰「他們武功的基礎甚膚淺,果然只是些不三不四的皮毛,但運用之際,却又十分巧妙。顯然中間有一個重大的関鍵所在,我没揣摩得透。{\upstsl{嗯}},那挪移乾坤的第七層功夫之中,有一些我没練成,難道便是爲此麼?」説著坐著甲板之上,抱頭苦苦思索起來,謝遜等均不出聲,不敢擾亂他的思路。

忽聽得號角此起彼落,一艘大船緩緩駛到,船頭上插了十二面繡金大旗。船頭上設著十二張虎皮交椅,三張空著,其餘九張有人乘坐。那大船駛到近處,便停住了。齊心王和倶明王躍上大船,各在左右最末的一張椅坐了,只有第六張虎皮交椅空著。趙明心念一動,説道︰「咱們抓到的此人和大船上那十一人服色相同,莫非便是十二寶樹王之一嗎?」無忌道︰「我也這麼想。此人身份既高,對方一時當不敢對咱們怎樣\dash{}」下面的話尚未出口,忽見風雲三使押著一人,走到了十一寶樹王之前。

無忌等一見,都吃了一驚,只見那人佝僂著身子,手撐拐杖,正是金花婆婆。坐在第二張椅中的智慧寶樹王向她喝問數語,金花婆婆側著頭,説道︰「你説些什麼?我不懂。」智慧王冷笑一聲,站起身來,左手一探,已揭下了金花婆婆頂上滿頭白髮,露出烏絲如雲。金花婆婆頭一側,向左避讓,智慧王右手倏出,竟在她臉上揭下了一層面皮下來。無忌等看得清楚,智慧王所揭下的乃是一張人皮面具,刹那之間,金花婆婆變成了一個膚如凝脂、杏眼桃腮的美婦,容光照人,端麗難言。無忌心中一動,暗想︰「她和小昭好像啊!」却聽得這句話被趙明説出口來︰「她和小昭好像啊!」

黛綺絲被他揭穿了本來面目,索性將拐杖一抛,只是冷笑。智慧王説了幾句話,她便以波斯話對答。二人一問一答,但見十一位寶樹王的神色越來越是嚴重,無忌等却不懂他二人説些什麼。趙明忽問︰「小昭姑娘,他們説些什麼啊?」小昭流泪道︰「你很聰明,你什麼都知道。却幹麼事先不阻止謝老爺子别説?」趙明奇道︰「阻止他别説什麼?」小昭道︰「他們本來不知金花婆婆是誰,後來知道她是紫衫龍王了,但還没想到紫衫龍王便是聖女黛綺絲。婆婆一番苦心,只盼能將他們騙倒。謝老爺子所提的第二個條款,却要他們釋放聖女黛綺絲,這雖是好,可就瞞不過智慧寶樹王了。謝老爺子目不見物,自不知金花婆婆裝得多像,任誰也能瞞過。趙姑娘,你瞧得清清楚楚,難道想不到麼?」

這一次她却是冤枉了趙明,其實趙明聽了謝遜在海上所説的故事,心中先入爲主,認定金花婆婆便是波斯明教的聖女黛綺絲,一時可没想到,在波斯諸人眼中,她的眞面目其實並未揭破。她待要反唇相稽,但聽小昭説得十分悲苦,隱隱已料到小昭和金花婆婆之間,必有極不尋常的関連,這時倒也不忍在口頭再去刺她,只道︰「小昭妹妹,我確是没想到。若是有意加害金花婆婆,讓我不得好死。」謝遜更是歉仄,當下一句話也不説,心中却打定了主意,寧可自己性命不要,也得援黛綺絲出險。

只聽小昭泣道︰「他們責備金花婆婆,説她既嫁人,又叛教,要\dash{}要放火燒死她。」無忌道︰「小昭,一有可乘之機,我便衝過去救婆婆出來。」他是叫慣了婆婆,其實此時瞧瞧紫衫龍王的本來面目,雖已中年,風姿嫣然,實不減於趙明、周芷若等人,倒似是小昭的一個大姊姊。小昭道︰「不,不!十一寶樹王,再加風雲三使,你是鬥他們不過的,那不過是枉自送了性命。他們這時在商量如何奪回平等王。」

趙明恨恨的道︰「哼!這平等王便活著回去,臉上印著這幾行,醜也醜死啦。」無忌問道︰「什麼臉上印著字?」趙明道︰「那黃鬚使者用聖火令一下子打在他的臉頰之上\dash{}啊,小昭!」她突然間想起一事,説道︰「小昭,你識得波斯文字麼?」小昭道︰「識得。」趙明道︰「你快瞧瞧,這平等王臉上,印著的是些什麼字。」小昭搬起平等王上身,側過他的頭來,只見他左頰高高腫起,三行波斯文深印肉裡。原來每根聖火令上都刻得有文字,妙風使誤擊平等王,竟將聖火令上的文字印在他的肌肉裡了。只是聖火令著肉處不過兩寸寬、三寸長,所印文字殘缺不全。

小昭跟著無忌進入光明頂祕道,曾將乾坤大挪移心法背誦幾遍,雖然不明其理,自己未曾習練,但這武功法門却是記得極熟的
\footnote{\footnotefon{}〔按〕請參閲前文︰無忌在祕道中練至第七層心法時遇有疑難,跳過費解之處不練,小昭代爲一一記誦}
,這時看了平等王臉上的文字,不禁脱口而呼︰「那也是乾坤大挪移的心法!」

張無忌奇道︰「你説那是挪移乾坤的心法?」小昭道︰「不,不是!我初時一見,以爲是了,却又不是。譯成華語,意思是這樣︰『應左則前,須右乃後,三虛七實,無中生有』\dash{}什麼『天方地圓』\dash{}下面的看不到了。」這幾句寥寥十餘字的言語,無忌乍然聽聞,猶如在滿天烏雲之中,驟然間見到電光閃了幾閃,雖然電光過後,四下裡仍是一團漆黑,但這幾下電閃,已讓他在五里濃霧之中看到了出路。他口中喃喃唸道︰「應左則前,須右乃後\dash{}」竭力將這幾句口訣,和所習乾坤大挪移的武功配合起來,只見似是而非,隱隱約約的好像想到了,却又不對。

忽聽小昭叫道︰「張公子,留神!他們已傳下號令︰風雲三使要來向你進攻,勤修王、忍辱王、功德王三王來搶平等王。」謝遜經小昭一提,當即將平等王的身子橫舉在胸口,把屠龍刀抛給無忌,説道︰「你用刀猛砍便是。」趙明也將倚天劍交了給周芷若,此刻同舟共濟,並肩迎敵要緊。張無忌接過屠龍刀,心不在焉的往腰間一插,口中仍在念誦︰「三虛七實,無中生有\dash{}」趙明急道︰「小獃子,這當児可不是參詳武功的時候,快預備迎敵要緊。」一言甫畢,勤修、忍辱、功德三王已縱身過來,伸掌向謝遜攻去。他三人生怕傷了平等王,是以不用兵刃,只使拳掌,只要有一人抓住了平等王的身子,便可出力搶奪。周芷若守在謝遜身旁,每逢勢急,一劍便向平等王身上刺去。勤修王、忍辱王等不得不出掌向周芷若相攻,以免寶劍刺中在平等王身上。

那邉廂張無忌又和風雲三使鬥在一起。他四人數次交手,各自吃過對方的苦頭,誰也不敢大意,數合之後,輝月使一令打來,依照武學的道理,這一招必須打在無忌左肩,那知聖火令在半途古怪怪的轉了個彎,拍的一響,打中在無忌的後頸。無忌一陣劇痛,心頭却登時雪亮,暗暗大叫︰「應左則後,應左則後,對了,對了!」最後忍不住喊出聲來︰「我懂了,我懂了!」原來風雲三使所會的,只不過是挪移乾坤第一層中的入門功夫,但聖火令上另外刻得有詭異的變化用法,以致平等添奇幻。他心念一轉之間,小昭所説的四句口訣已全然明白,只是「天方地圓」甚麼的還無法參悟,心想須得看齊聖火令上的刻字,纔得通曉波斯派武功的精要,突然間一聲清嘯,雙手擒拿而出,「三虛七實」,已將輝月使手中的兩枚聖火令奪了過來,「無中生有」,又將流雲使的兩枚聖火令奪到。兩人一呆之際,無忌已將四枚聖火令擕揣入懷中,雙手分别抓住兩人後領,將兩人擲向敵船。波斯群胡吶喊叫嚷聲中,妙風使縱身而起,逃回已船。此時無忌知了他武功的竅訣,雖然所解的仍極有限,但妙風使的武功在他眼中却已無神祕之可言,右手一探,已抓住他的左脚,硬生生將他在半空中拉了回來,挾手奪下聖火令,舉起妙風使的身子,便往忍辱王的頭頂{\upstsl{砸}}了下去。三王吃了一驚,眼看接戰不利,三人打個手勢,便即躍回。無忌點了妙風使的穴道,擲在脚邉。

無忌這下取勝,來得突兀之至,頃刻之間便從下風轉到上風,趙明等無不驚喜,齊問原由。無忌笑道︰「若非陰差陽錯,平等王臉上吃了這一傢伙,那可糟糕得緊了。小昭,你快將這六根聖火令上的字譯給我聽,快,快!」

各人瞧這六枚聖火令時,但見非金非玉,質地堅硬無比,六枚聖火令長短大小,各各不同,似透明,令中隱隱像有火燄飛騰,實則玉質映光,顏色變幻。每枚令上都刻得有不少波斯文字,别説參透其中深義,便是譯解一遍,也得不少時光。

但無忌心知欲脱眼前之困,非探明波斯派武功的總源不可,當下向周芷若道︰「周姑娘,請你以倚天劍架在平等王的頸中。義父,請你以屠龍刀架在這妙風使的頸中,儘量拖延時刻。」謝遜和周芷若點頭答應。

小昭拿起六枚聖火令,見最短的那枚上文字最少,又是黑黝黝的最不起眼,便將其上文字一句句的譯解出來。無忌聽了一遍,却是一句也不懂,苦苦思索,絲毫不明其意,不由得心中大急。趙明道︰「小昭姑娘,你還是先解打過平等王的那根聖火令。」這一言提醒了小昭,忙核對聖火令上的文字,見是次長的那一根,當即譯解其意,這一次無忌却懂了十之七八。待得一根解完,再解最長那一根時,無忌只聽得幾句,喜道︰「小昭,這六枚聖火令,越長的越淺,這一根上説的都是入門功夫。」

原來六枚聖火令,乃是當年波斯「山中老人」所鑄,上面刻著他畢生武功的精要。六枚聖火令和明教同時傳入中土,向爲中土明教教主的令符,年深日久之後,中土明教已無人識得波斯文字。數十年前,聖火令爲丐幫中人奪去,輾轉間爲波斯商賈所得,復又流入波斯明教。波斯總教鑽研其上文字,數十年間,教中職份較高之人士,人人武功陡進。只是其上所記武功博大精深,便是修爲最高的大聖寶樹王,也只是學得三四成而已。

至於乾坤大挪移心法,本是波斯明教的護教神功,但這種奇妙的武功非常人所能修習,波斯明教的教主規定又須由處女擔任,千百年間接連出了幾位庸庸碌碌的女教主,這心法流留下來的,便十分有限,反倒是中土明教尚留得全份。波斯明教以不到一成的舊傳挪移乾坤武功,和兩三成新得的聖火令武功相結合,變出一門古怪奇詭的功夫出來。該教的首腦情知倘若乾坤大挪移心法能物歸故主,和聖火令上神功相輔相成,那麼明教便能威震天下,他們派遣聖女黛綺絲混入光明頂,其意便在於此。

不料這份心願,却是在中土明教教主張無忌的身上完成。其實波斯明教便是得到了乾坤大挪移心法,若無九陽神功作爲根基,也未必能滲透其中奥妙,可知世事往往講究機緣,未必強求便得。

張無忌盤膝坐在船頭,小昭俯嘴在他耳邉,一句句將聖火令上的文字,説與他聽。這聖火令中所包含的武功,原來奇妙無比,但一法通,萬法通,各種深奥的學問鑽研到了極處,本是殊途同歸。張無忌深明九陽神功,挪移乾坤以及武當派太極拳的拳理,此三種武功乃天竺、波斯、中華三地武學的極致,聖火令上的武功雖奇,究不過是旁門左道之學而達於巓峰而已,其宏廣精深之處,實則遠遠不及上述三種武學。無忌聽小昭譯完六枚聖火令上的文字,倉卒之間,只記得了七八成,所得明白的,又只五六成,但僅此而言,寶樹諸王和風雲三使所顯示的功夫,在他眼中已是瞭如指掌,根本不値一哂。

時光一刻一刻的過去,無忌全心全意浸潤在武學的鑽研之中,無暇顧及身外之務,但趙明和周芷若等却是焦急萬狀,眼見黛綺絲手脚之上都被加上了{\upstsl{銬}}鐐;眼見十一寶樹王聚頭密議;眼見十一王脱下長袍,換上軟甲,眼見十一王的左右呈上十一件奇形怪狀的兵器;眼見前後左右一艘艘船上排滿了波斯胡人;眼見這些胡人彎弓搭箭,將箭頭對準了自身;眼見數十名波斯人手執斧鑿,跳入水中,只待首領令下,便來鑿沉己方的座船。

這時天色漸漸明亮,東方海面之上,半個太陽在水面載浮載沉,放出萬道金光,只聽得居中而坐的大聖寶樹王,大喝一聲,四面大船上鼓聲雷鳴,號角齊動。

\chapter{包藏禍心}

張無忌聽到鼓角之聲,吃了一驚,一抬頭,只見十一位寶樹王各披燦爛生光的金甲,手執兵刃,跳上船來。謝遜和周芷若分執刀劍,架在平等王和妙風使的頸中,十一王見此情景,跳上船頭之後,却也不敢便此逼近,環成半月形,虎視耽耽,伺機而動,周芷若、趙明等見這十一王形相猙獰,身形高大,心下都是暗暗害怕。

智慧王用華語説道︰「爾等快快送出我方教友,饒爾等不死。這幾個教友在吾人眼中,猶如豬狗一般,爾等用刀架在他們頸中,有什麼用?爾等有膽,儘可將他們殺了。波斯聖教之中,這等人成千成萬,殺一兩個有何足惜?」趙明説道︰「爾等不必口出大言,欺騙吾人。吾人知悉,這二人一個是平等寶樹王,一個是妙風使。在爾等明教之中,地位甚高。爾等説他們猶如豬狗一般,爾言差矣,大大差矣!」那智慧王所説的華語,乃是從書本上學來,「爾等」「吾人」云云,大是不倫不類。趙明模仿他的聲調用語,謝遜等聽了,雖在危境之中,意也忍不住微笑。

智慧王眉頭一皺,説道︰「聖教之中,共有三百六十位寶樹王,平等王排名第三百五十九。吾人有使者一千二百人,這妙風使武功平常,毫無用處,爾等快快將他們殺了。」趙明道︰「很好,很好!手執刀劍的朋友,快快將這兩個無用之人殺了。」謝遜道︰「遵命!」舉起屠龍刀,呼的一聲便向平等王頭頂橫劈過去。衆人驚呼聲中,屠龍刀從他頂頭掠過,距頭蓋不到半寸,大片頭髮切削下來,被海風一吹,飄浮空中。謝遜手臂一提,左一刀,右一刀,向平等王兩肩砍落。眼看每一刀均要切掉他的一條臂膀,但刀鋒將及肩頭之際,於是手腕微微一偏,刀鋒將他雙臂衣袖切下了一片。這三下硬砍猛劈,部位竟是如此準確,别説是盲眼之人,便是雙目完好,也是極爲難能。平等王死裡逃生,嚇得幾欲暈去。十一寶樹王、風雲二使目瞪口呆,撟舌不下。

趙明説道︰「你等已見識了中土明教的武功。這位金毛獅王,在中土明教中排名第三千五百零九,爾等若是恃衆取勝,中土明教日後必來報仇,掃蕩爾等總壇,爾等必定抵擋不住,還及早兩家言和的爲是。」智慧王明知趙明所言不實,但一時却無計可施。那大聖寶樹王忽然説了幾句話,小昭叫道︰「張公子,他們要鑿船。」

無忌心中一凜,倘若座船沉了,諸人不識水性,那是非束手成擒不可,身形一晃,已欺到了大聖王身前。智慧王喝道︰「爾等幹什麼?」兩旁功德王和歡喜王手中的一鞭一鎚,同時{\upstsl{砸}}將下來。此時無忌早已熟識波斯派的武功,不躱不閃,雙手伸出,已抓住了兩王的喉嚨,只聽得{\upstsl{噹}}的一聲響,功德王的鐵鞭和歡喜王的八角鎚相互一擊,火花飛濺,兩人已被無忌抓住咽喉要穴,橫拖倒曳的拉了過來。混亂之中張無忌連環踢出四腿,兩脚踢飛了齊心王和忍辱王手中的大砍刀,又兩脚將勤修王和倶明王踢入了水中。只見一個身形高廋的寶樹王撲將過來,雙手各執短劍刺向無忌胸口。

無忌引飛起一脚,踢他手腕。那人雙手突然交叉,刺向無忌小腹。這一招變得靈動之極,無忌急忙躍起,方始避過。原來此人是常勝寶樹王,波斯總教十二王中武功第二。他一擊不中,反手便刺向無忌背心,無忌捏閉了功德王和歡喜王的穴道,將兩王抛入船艙,猱身而上,和常勝王手中雙劍搏擊。此人雖然同是十二寶樹王之一,但武功之強,與餘王大不相同。無忌攻三招,守三招,三進三退,心下暗暗喝采︰「好一個了得的波斯胡人!」

張無忌明白了聖火令上的武功心法之後,未經練習,立時便遭逢強敵,當下一面記憶思索,一面和常勝王搏鬥。最初十餘招間,仗著内力深厚,招數巧妙,保持個不勝不敗之局,到得二十招後,聖火令上的祕訣用在乾坤大挪移功夫上,越來越是得心應手。常勝王號稱「常勝」,生平罕逢對手,今日被無忌剋制得縛手縛脚,那是從所未有之事,又是驚異,又是害怕。鬥到第三十招上,張無忌踏上一步,忽地甲板上一坐,抱住了常勝王的小腿。這招怪異的法門,原爲聖火令上所記,但已是極高深的功夫,常勝王雖然知道,却是從不敢用。無忌一抱之下,十指扣住了他小腿上的「中都」「築賓」兩穴,那是中土武功的拿穴之法。常勝王只覺下半身酸麻異常,長嘆一聲,束手就擒。

無忌忽起愛才之念,説道︰「爾等武功甚佳。余保全爾的英名。快快回去吧!」説著隻手放開。常勝王又是感激,又是羞慚,躍回自己座船。此時謝遜和周芷若已將功德王和歡喜王揪了出來,屠龍刀和倚天劍兩柄利刃之一,均架有波斯明教的兩位重要首領。大聖王見常勝王苦戰落敗,功德王和歡喜王又失陥敵手,就算將敵人座船鑿沉,投鼠忌器,平等王等四人非喪命不可,當下一聲號令,呼召衆人,一齊回歸座船。

趙明朗聲説道︰「爾等快快將黛綺絲送上船來,答應金毛獅王的三個條件。」只見餘一的九位寶樹王低聲商議一陣,智慧王道︰「要答應爾等條款,也無不可。這位青年公子的武功明明是吾人波斯一派,彼從何處學得,吾人有點不明不白。」趙明忍住了笑,正色道︰「爾等本來不明不白,不清不楚,不乾不淨,不三不四。這位青年公子是本教光明使座下的第八位弟子。他的七位師兄,七位師弟不久便到,那時候彼等七上八落,爾等便不亦樂乎,嗚呼哀哉了。」智慧王爲人本極聰明,但華語艱深,趙明的話他只懂得個六七成,情知趙明在大吹法螺,微一沉吟,便道︰「好!將黛綺絲送過船去。」

兩名波斯教徒架起黛綺絲,送到無忌船頭。周芷若長劍一振,叮叮兩聲,登時將她手足上的{\upstsl{銬}}鐐切斷。那兩名波斯教徒見此劍如此鋒利,嚇得打個寒戰,急忙躍回船去。豈知其中一人驚得脚都軟了,竟没能躍上船頭。{\upstsl{噗}}通一聲,跌在海中。

智慧王道︰「爾等快快開船,回歸中土。吾人只派小船,跟隨爾等之後。」張無忌抱拳説道︰「中土明教源出波斯,爾我情若兄弟,今日一場誤會,敬盼各位不可介意。日後請上光明頂來,雙方杯酒言歡。得罪之處,兄弟這裡謝過了。」智慧王哈哈笑道︰「爾武功很好,吾人極是佩服。學而時習之,不亦説乎?有朋自遠方來,不亦説乎?七上八落,不亦樂乎?」無忌等起初聽他掉了兩句書包,心想此人居然知道孔子之言,倒是不易,不料接下去竟是學著趙明説過的兩句話,忍不住都大笑起來。趙明道︰「爾的話説得很好,人之異於波斯人者,幾希!祝爾等多福多壽,來格來饗,禍延先考,無疾而終。」智慧王懂得「多福多壽」四字,只道下面的均是祝壽之辭,笑吟吟連聲説道︰「多謝多謝!」

無忌心想趙明説得高興上來,不知還有多少刁鑽古怪的話要説,身居虎狼之群,夜長夢多,還是及早脱離險境爲是,當下拔起鐵錨,轉過船舵,扯起風帆,將船緩緩駛了出去。四周船上的波斯人見他起錨扯帆,一個人做了十餘名水手之事,神力驚人,盡皆喝采。只見一艘小船抛了一條船纜過來,無忌便將那纜縛在後梢,拖了那小船漸漸遠去。只見小船中端坐二人,一男一女,正是流雲使和輝月使。

張無忌穩穩掌著船舵,向西行駛,見波斯的各艘大船並不追來,駛出數里,遠眺靈蛇島旁的諸船已小不逾尺,仍是停著不動,這纔放心。當下要小昭過來掌舵,到艙中察看殷離的傷勢,見她迷迷糊糊的半睡半醒,雖然未見好轉,病情却也並没更惡。黛綺絲站在船頭,眼望大海,聽到無忌走上甲板,却是並不回頭。無忌見她背影曼妙,秀髮飄拂,後頰膚若白玉,謝遜説她當年乃是武林人中第一美人,此言當眞不虛,遙想碧水潭旁,紫衫如花,長劍勝雪,不知傾倒了多少英雄豪傑。

這船航到傍晩,算來離靈蛇島已有百里,向東望去,海面上並無片帆隻影,波斯總教顯是在要脅之下,不敢追來。無忌道︰「義父,咱們可放了他們麼?」謝遜道︰「好吧!他們便是要追,也追不上了。」無忌於是解開平等、功德、歡喜三王及妙風使的穴道,連聲致歉,放他們回入拖在船梢上的小船中。妙風使道︰「這聖火六令是吾人掌管,失落後其罪非小,亦請一併賜還。」謝遜道︰「聖火令是中土明教教主令符,今日物歸原主,如何能再讓你們擕去。」妙風使絮絮不休,堅執要討還。無忌心想今日須得折服其心,免得日後更多後患,説道︰「我們便是交還於你,你本領太低,還是無法保有。與其被外人奪去,還是存在明教手中的好。」妙風使道︰「外人怎能隨便奪去?」無忌道︰「你若不信,那就試試。」將六根聖火令交給了他。妙風使大喜,剛説得一聲︰「多謝!」張無忌左手一勾,右手一引,早已將六根聖火令一齊奪了過來。妙風使大吃一驚,怒道︰「我尚未拿穩,這個不算。」無忌笑道︰「再試不次,那也不妨。」又將聖火令還了給他。

妙風使先將四枚聖火令揣入懷中,手中執了兩根,見無忌出手來奪,左手一令往無忌手腕上{\upstsl{砸}}將下來。無忌手腕一翻,已抓住他的右臂,拉著他手臂迎將上去,雙令互擊,錚的一聲響,震得人心旌搖動。無忌渾厚内力從他手臂迎將過去,這一擊之下,妙風使兩臂酸痛,全身乏力,便如癱瘓,撒手將聖火令抛在甲板之上。無忌先從他懷中取出四枚聖火令,又拾起甲板上的兩枚,説道︰「如何?是否要再試一次?」妙風使臉如死灰,喃喃的道︰「你不是人,你是魔鬼,你是魔鬼。」舉步待要躍入小船,但一個踉蹌,軟癱跌倒。流雲使躍將上來,抱了他過去。只見小船上扯起風帆,功德王拉住船纜,雙手一拉,拍的一響,船纜崩斷,大小二船登時分開。無忌抱拳説道︰「多多得罪,還祈各位見諒。」只見功德王等人眼中允滿了怨毒之意,掉頭不答。

大船乘風西去,兩船漸距漸遠,忽聽得黛綺絲叱道︰「賊子敢爾!」縱身而起,躍入海中。張無忌吃了一驚,急忙轉舵。只見一股血水,從海中湧了上來,跟著不遠之處,又湧上一股血水,頃刻間共有六股血水湧上。忽喇一響,黛綺絲從水中鑽出,口中咬著一柄短刀,右手抓住一個波斯人的頭髮,踏水而來。無忌忙轉舵將船迎去。但那船船身太大,顧得了轉舵,顧不得落帆,一時在海中慢慢打轉,紫衫龍王水性果然了得,但見她在海中捷若游魚,不多時游到船旁,左手在船邉鐵錨的錨爪上一借力,身子飛起,連著那波斯人一起上了甲板。衆人見了這等情景,心下均已了然。原來波斯人暗藏禍心,待功德王等一干人過了小船,扯起風帆作爲遮掩,暗放熟識水性之人潛到大船之旁,意圖鑿沉無忌等的座船。虧得紫衫龍王見到船旁潛水人吐氣的水泡,躍入海中,殺了六人,還擒得一名活口。正待審問那潛水胡人,驀地裡船尾轟隆一聲巨響,黑煙瀰漫。

但覺得船隻震盪,如中炮擊,後梢上木片粉飛,張無忌等只感一陣炙熱,忙一齊伏低。黛綺絲叫道︰「這等人奸惡如此!」搶到後梢,只見船尾炸了一個大洞,船舵已飛得不知所終,破洞中海水滾滾湧入。趙明向無忌淒然望了一眼,心想︰「敵船不久便即追上,我等當眞是死無葬身之地了。」黛綺絲用波斯話向那被擒的波斯人問了幾句,手一起掌,將他天靈蓋擊得粉碎,一足踢入海中,説道︰「我只發覺他們鑿船,没料到他們竟在咱們船尾上綁上了炸藥。」這時功德王等人所乘的小船早已去得遠了,黛綺絲水性再好,也已無法追上。

衆人黯然相對,束手無策。那大海船船隻甚大,一時三刻之間却也不易沉没。忽然之間,黛綺絲嘰哩咕嚕,向小昭説起波斯話來,小昭也以波斯話回答,兩人一問一答,臉上神色變幻不定。只見小昭向張無忌瞧了一眼,雙頰暈紅,甚是靦腆,黛綺絲却厲聲追問。兩人説了半天,似乎在爭辯什麼,後來黛綺絲似乎在力勸小昭答應什麼事,小昭只是搖頭不允,忽向無忌瞧了一眼,嘆了口氣,説了一個字。黛綺絲伸手摟住了小昭,不住吻她,兩人一齊泪流滿面,小昭抽抽噎噎的哭個不住,黛綺絲却柔聲安慰。張無忌、趙明、周芷若三人面面相覷,全然不解。趙明在無忌耳邉低聲道︰「你瞧,她二人相貌好像!」無忌一凜,只見黛綺絲和小昭都是清秀絶俗的瓜子臉児,高鼻雪膚,秋波流慧,眉目之間,當眞有六七分相似,心中立時想起苦頭陀范遙在大都小酒店中對小昭所説的那兩句話︰「眞像,眞像!」原來所謂「眞像」,乃是説小昭的相貌眞像紫衫龍王,那麼小昭是黛綺絲的妹妹麼?是她的女児麼?

無忌跟著又想起楊逍、楊不悔父女對小昭的加意提防,每當問到楊逍何以對小昭這麼小小一個少女竟然如此忌憚,似當大敵,他却又語焉不詳,這時方始明白,原來楊逍也已瞧出小昭的容貌和紫衫龍王有相似之處,只是並無其他佐證,又見無忌對她加意迴護,是以不便明言。至於小昭故意扭嘴歪鼻,苦心裝成醜女模樣,其用意更是昭然若揭了。

突然之間,無忌想起了一事︰「小昭混上光明頂去幹什麼?她怎麼知曉祕道的入口,那一定是紫衫龍王要她去的,用意顯是在盜取乾坤大挪移的心法。她成我小婢,相伴幾已兩年,我從對她不加防備,這份心法她要抄錄一通,當眞是易如探囊取物。啊喲!我只道她是個天眞瀾漫的少女,那料到她如此工於心計,我兩年來如在夢中,從頭至尾墮入她的殼中。張無忌啊張無忌,你一生信任旁人,事事受人之愚,竟栽在這小丫頭的手中。」想到這裡,不禁大是氣惱。

便在此時,小昭的眼光正向他望了過來。無忌見她神色中柔情無限,實非作偽,心下又是怦然一動,想起光明頂上對戰六大派時,她曾捨身相護自己,兩年來她細心熨貼的服侍自己,決不能是事事相欺,莫非自己冤枉了她?正自遲疑不決,船身劇烈一震,又沉下了一大截。黛綺絲道︰「張教主,你們各位不必驚慌!待會波斯人的船隻到來,我和小昭自有應付之方。紫衫龍王雖是女流之輩,也知一人作事一身當,決不致連累各位。張教主和獅王謝兄待我義重如山。黛綺絲這裡謝過了。」説著盈盈拜倒,無忌和謝遜急忙還禮,心中却想︰「這些波斯人行事歹毒,待會固會將你抓去燒死,也不會放過了咱們。」

那船漸漸下沉,艙中進水,無忌抱起殷離,周芷若抱起趙明,各人爬上桅桿。小昭忽向東方一指,哭出聲來。各人向她手指之處望去,只見遠處海面上帆影點點。過不多時,帆影漸大,正是十餘艘波斯大船鼓風追來。

張無忌心想︰「倘若我是黛綺絲,與其遭焚身之苦,還不如跳在海中,自盡而死。」然而見她神色泰然,毫不驚懼,心下不禁佩服︰「她身居四大法王之首,果不尋常。想當年鷹王、獅王、蝠王都已是成名的年長豪傑,她以一個妙齡少女,位在三王之上,決非僅因一日之功而得。」眼見波斯群船漸漸駛近,又想︰「我得罪諸寶樹王不小,既然落入他們手中,也不盼望再能活命。只是如何想個法児,護得義父和趙姑娘、周姑娘、表妹她們周全。那小昭,唉,寧可她對我不義,不可我待她不仁。」忽然間殷離身子一動,睜開眼來,見已身處於無忌懷中,吃了一驚,道︰「阿牛哥哥,我\dash{}我在那裡?你幹什麼?」無忌道︰「你别驚慌身子覺得怎樣?」殷離搖頭道︰「我\dash{}我一點力氣也没有,什麼都不知道。」

只見十餘艘波斯大船上的炮口,一齊對準了桅桿,張無忌武功再高,以血肉之軀,終究難當大炮中轟出來的炸藥鐵彈。那些波斯船駛到離沉船百餘丈處,便即落帆下錨,生怕駛得過近,被張無忌又搶上船來,擒去一兩位寶樹王,那麼一番計謀,又成泡影。只聽得智慧王哈哈大笑,得意非凡,叫道︰「爾等降不降了?」張無忌朗聲道︰「中土義士,寧死不屈,豈有降理?是好漢子便武功上決一強弱。」智慧王笑道︰「大丈夫鬥智不鬥力,快快束手待擒吧!」黛綺絲突然朗聲説了幾句波斯話,辭氣極是嚴正。智慧王怔了一怔,也答以幾句波斯話。兩人一問一答,説了十幾句話,那大聖王也接嘴相詢。又説了幾句,大船上放下一艘小船,八名水手划槳,駛了過來。

黛綺絲道︰「張教主,我和小昭先行過去,你們稍待片刻。」謝遜厲聲道︰「韓夫人,中土明教待你不薄。本教的安危興衰,繫於無忌一人之身。你若出賣我們,謝某命不足惜。要是損及無忌毫髮,謝某縱爲厲鬼,也決不饒你。」黛綺絲冷笑道︰「你義児是心肝寶貝,我女児便是瓦石泥塵麼?」説著挽了小昭之手,輕輕一躍,落入了小船。八名水手揮槳如飛,划向波斯大艦去了。各人聽了她這兩句話,都是一怔。趙明道︰「那小昭果然是她女児。」

遠遠望見黛綺絲和小昭上了大船,站在船頭,和諸寶樹王説話。那座船不住下沉,桅桿一寸一寸的低下。謝遜嘆道︰「非我族類,其心必異。無忌孩児,我識錯了韓夫人,你識錯了小昭。無忌,大丈夫能屈能伸,咱們暫忍一時之辱,再行俟機脱逃。你肩頭挑著重擔,中原千萬百姓,均盼我明教高舉義旗,驅除韃子,一當時機到來,你自行脱身,決不可顧及旁人。你是一教之主,這中間的輕重大小,可要分辨清楚了。」無忌沉吟未答,趙明「{\upstsl{呸}}」了一聲道︰「自己性命不保了,還什麼韃子不韃子的。你説蒙古人好呢,還是波斯人好?」周芷若一直默不作聲,這時忽道︰「小昭對張公子情意深重,決不致背叛於他。」趙明道︰「你不見紫衫龍王一再逼迫她麼?小昭先是不肯,最後被逼得緊了,終於肯了,還假惺惺地大哭一場呢?」

這時那桅桿離海面已不過丈餘,海中浪濤潑了上來,濺得各人頭臉皆濕。趙明忽然笑道︰「張公子,咱們和你死在一起,倒也乾淨。小昭陰險狡獪,反倒不能跟咱們一起死。」這幾句話雖以玩笑的口吻出之,但含意情致纏綿,無忌聽得甚是感動,心道︰「我不能同時娶她們爲妻,但得和她們同時畢命,也不枉了。」看看趙明,看看周芷若,又看看懷中的殷離。只是殷離又已昏迷不醒,趙周二女均是雙頰酡紅,臉上濺著點點水珠,猶似曉露中的鮮花,趙女粲若玫瑰,周女秀似芝蘭,無忌輕輕嘆了口氣,道︰「却教我如何報答?」

忽聽得十餘艘大船上的波斯人一齊高聲呼叫,吶喊聲和海上波濤相互衝擊。無忌等吃了一驚,凝目向諸船望去。只見每艘船上的波斯人一齊拜伏在甲板之上,向著大艦行禮。大艦上諸寶樹王也是伏在船頭,中間椅上端坐一人,倒似是小昭模樣,只是隔得遠了,瞧不清楚。無忌等大是奇怪,思疑不定,不知這些波斯人在搗甚麼鬼。

但聽得群胡呼喊了一陣,站起身來,仍是不斷的叫喊甚麼,聽那喊聲之中,顯是充滿歡愉,倒似是遭逢什麼喜慶之事一般。只見那艘小船又划了過來,划到近處,小船中赫然坐的是是小昭。她招手説道︰「張公子,各位請同到大艦之上,波斯明教,決計不敢加害。」趙明問道︰「爲什麼?」小昭道︰「各位請過去便知。若有相害之意,小昭如何對得起張公子?」謝遜忽道︰「小昭,你是做了波斯明教的教主麼?」小昭低眉垂首,並不回答,過了片刻,大大的眼睛之中,忽然掛下兩顆晶瑩的眼泪。霎時之間,張無忌耳中{\upstsl{嗡}}的一響,一切前因後果,已是猜到了七八成,心下又是難過,又是感激,説道︰「小昭,你這一切都是爲了我!」小昭側開頭,不敢和他的目光相對。

謝遜嘆道︰「黛綺絲有女如此,不負了紫衫龍王一世英名。無忌,咱們過去吧。」説著躍向小船之中。接著周芷若抱了趙明,跳了過去,終於無忌也抱著殷離跳入小船。八名水手掉過船頭,划向大艦。那小船離大艦尚有二十餘丈,艦上諸寶樹王已一齊躬身迎接教主。紫衫龍王雖是小昭的母親,却也不廢教中的尊卑之禮。衆人登上大艦,小昭吩咐了幾句,早有人恭恭敬敬送上面巾、食物,分别帶著各人入艙換去濕了的衣服。

張無忌見他所處那間房艙極是寬敞,房中珠光寶氣,陳設著不少珍物,剛抹乾身上沾濕的海水,呀的一聲,房門推開,進來一人,正是小昭。她手上拿著一套短衫褲,一件長袍,説道︰「公子,我服侍你換衣。」無忌心中一酸,説道︰「小昭,今日你已是總教的教主,説來我還是你的屬下,如何可再作此事?」小昭求道︰「公子,這是最後的一次。此後咱們東西相隔萬里,會見無日,我便是再想服侍你一次,也是不能的了。」無忌黯然神傷,只得任她和平時一般,助他換上衣衫,幫他扣上衣鈕,結上衣帶,又取出梳子,替他梳好頭髮,無忌見她泪珠盈盈,突然間心中激動,伸手將她嬌小的身軀抱在懷裡。小昭「嚶」的一聲,身子微微顫動。無忌在她櫻唇上深深印了一吻,説道︰「小昭,初時我還怪你欺騙於我,没想到你竟待我這麼好。」

小昭將頭靠在他寬廣的胸脯之上,低聲道︰「公子,我從前確是騙過你的。我媽本是總教三位聖處女之一,奉派前來中土,積立功德,以便回歸波斯,繼任教主。不料她和我爹爹相見後,情難自已,不得不叛教和我爹爹成婚。我是爹爹的遺腹女児,終身從未見過爹爹一面。我媽自知罪重,將聖處女的鐵戒指傳了給我,命我混上光明頂,盜取乾坤大挪移的心法。公子,這件事我是一直在騙你。可是在心中,我却没對你不起。因爲我決不願做波斯明教的教主,我只盼做你的小丫頭,一生一世的服侍你,永遠不離開你。我是跟你説過的,是不是?你也答應過我的,是不是?」無忌點了點頭,抱著她輕柔的身子坐在自己膝上,又吻了吻她。她溫軟的嘴唇上沾著泪水,又是甜蜜,又是苦澀。小昭又道︰「我記得了挪移乾坤的心法,決不是存心背叛於你。若非今日山窮水盡,我決計不會洩露此事\dash{}」

無忌輕聲道︰「現下我都知道了。」小昭幽幽的道︰「我幼年之時,便見媽媽日夜不安,心驚膽戰,遮掩住她好好的容貌,化裝成一個醜樣的老太婆。她又不許我跟她在一起,將我寄養在别人家裡,隔一兩年纔來瞧我一次。這時候我才明白,她爲什麼干冒大險,要和我爹爹成婚。公子,咱們今天若非這樣,别説做教主,便是做波斯的女皇,我也不願。」説到這裡,她雙頰紅暈如火,無忌只覺得抱在懷裡的嬌軀突然熱了起來,心中正自一動,忽聽得黛綺絲的聲音在門外説道︰「小昭,你克制不了情欲,那便送了公子的性命。」

小昭身子一顫,跳了起來,説道︰「公子,你以後莫再記著我。殷姑娘隨我母多年,你一往情深,是你良配。」無忌低聲道︰「小昭,咱們殺將出去,擒得一兩位寶樹王,再要脅他們送回靈蛇島去。」小昭淒然搖頭,道︰「這次他們已學了乖,謝大俠,殷姑娘他們身上,此刻均有波斯的刀劍相加。咱們稍有異動,立時便送了他們性命。」説著打開了艙門。只見黛綺絲站在門口,兩名波斯人手挺長劍,抵住她的背心。那兩名波斯人躬身向小昭行禮,但手中長劍的劍尖,却始終不離黛綺絲的背心。

小昭昂然直至甲板,無忌跟隨其後,果見謝遜等人身後,均有波斯武士挺劍相脅。小昭説道︰「公子,這裡有波斯治傷的靈藥,請你替殷姑娘敷治。」説著用波斯語吩咐了幾句,功德王取出一瓶膏藥,交給無忌。小昭又道︰「我命人送各位回歸中土,咱們就此别過。小昭身在波斯,日日祝公子福體康寧,諸事順遂。」説著聲音又哽咽了。無忌道︰「你身居虎狼之域,一切小心。」小昭點了點頭,吩咐下屬準備船。謝遜殷離趙明周芷若等一一過了船去。小昭將屠龍刀倚天劍六枚聖火令都交了給無忌,淒然一笑,舉手作别。無忌不知説什麼話好,呆立片刻,躍入對船。只聽得小昭所乘的大艦上號角聲嗚嗚響起,兩船一齊揚帆,漸離漸遠。但見小昭俏立船頭,怔怔向無忌的座船望著。兩人之間的海面越拉越廣,終於小昭的座艦成爲一個黑點,終於海上一片漆黑,長風掠動船帆,猶帶嗚咽之聲。

殷離敷了波斯的治傷膏之後,傷勢好得極慢,發燒不退,囈語不止。原來她在海上數日,病中受了風寒,那傷藥只能醫治金創外傷,體内風邪,却非用他藥治之不可。無忌人心焦急,第三日上遙遙望見東首海上有一小島,無忌吩咐舵工向島駛去。那舵工甚是不願,嘰哩咕嚕的爭辯,意思似是,教主只命我送你們回歸中原大陸,却没叫我中途去甚麼荒島。無忌比劃手勢,向他解釋,去荒島乃是採尋草藥,救人性命。那舵工言語不通,只是搖頭。無忌焦躁起來,搶過船舵,掉過船頭東航。

到得島旁,已將在海上多日,波濤激盪,雖然身負武功,却也不免頭暈,此刻上得島來,精神却是爲之一振。那島方圓不過數里,一眼可望到盡頭,但因地氣溫暖,島上長滿了樹木花草。無忌請周芷若看護殷離、趙明,一路分花拂草,尋覓草藥。

那島上花草雖衆,但要採集治病合用的草藥,却也不易。無忌越尋越遠,直到昏黑,仍只找到一味,只得回到原處。周芷若已用枯枝燒起了一堆火,四下裡花香浮動,草木清新,比之船艙中的氣悶侷促,另有一番光景。殷離的精神也好了一些,説道︰「阿牛哥哥,今晩咱們睡在這児,不回船去了。」此議一出,人人讚妙。眼見小島上山溫水軟,也無兇禽猛獸,各人放心安睡,不知東方之既白。

次晨醒轉,無忌揉了揉眼睛,只見那艘波斯船已不在原處。他吃了一驚,奔到海邉四下一望,仍是不見那船的蹤影。

\chapter{鴻飛冥冥}

無忌縱身欲上山崗眺望,只跨出一步,脚下一個踉蹌,險些摔倒,只覺雙脚虛軟無力。那是他從所未有之事,這一驚眞是非同小可,叫道︰「義父,你安好嗎?」却不聽得謝遜回答。無忌心中第一掛懷的是義父的安危,忙奔到謝遜睡臥之處,只見他好端端的睡得正沉,先放了一大半心。趙明、周芷若、殷離三人以男女有别,睡在遠處一塊大石之後。張無忌再奔過去看時,只見周芷若和殷離相對而臥,趙明却已不在該處。一瞥間瞬,只見殷離滿臉是血。無忌俯身一看,見她臉上被利刃劃了十來條傷痕,人已昏迷不醒,忙伸手一搭她脈搏,幸好尚在微微跳動。再看周芷若時,只見她滿頭秀髮被削去了一大片,左耳也被削去半隻,鮮血未曾全凝,可是她臉含微笑,兀自做著好夢。晨曦射下如海棠春夢,嬌麗無限。

無忌看到這般情景,心中連珠價只是叫苦,叫道︰「周姑娘,醒來!周姑娘,醒來!」周芷若只是不醒。無忌伸手去搖她肩頭,周芷若打了個呵欠,吹氣如蘭,側了頭仍是沉睡。無忌知她必是中了迷藥,昨晩出了這許多怪事,自己渾然不覺,此刻又是全身乏力,自己也是中毒無疑。一時叫周芷若不醒,當下又奔到謝遜身旁,叫道︰「義父,義父!」謝遜迷迷糊糊的坐了起來,道︰「怎麼啦?」無忌道︰「糟糕!咱們中了小人之計。」將波斯船隻駛走殷離及周芷若受傷之事簡略一説。謝遜驚道︰「趙姑娘呢?」無忌默然道︰「不見她啊。」吸一口氣,略運内息,只覺四肢虛浮,使不出勁來,衝口便道︰「義父,咱們被人下了『十香軟筋散』之毒。」六派高手被趙明以「十香軟筋散」困倒,一齊擄到大都萬法寺中之事,謝遜早已聽到無忌説過,他站起身來,脚下也是虛飄飄的全無力道。他定了定神,説道︰「那屠龍刀和倚天劍,是否都被她帶走了?」無忌一看身周,刀劍早已不知所終,心下氣惱無比,幾乎要哭出聲來。他倒不是可惜刀劍被盜,只是没料到趙明竟會乘著自己遭逢極大危難之際,忽來落井下石,使出這樣的奸計。

他呆了一呆,掛念殷離的傷勢,忙又到殷周二女身旁,推了推周芷若,她仍是沉睡不醒。他心想︰「我内力最深,是以醒得最早,義父其次。周姑娘内力跟咱二人差得遠了,看來一時難醒。」當下撕了一塊衣襟,替殷離抹去臉上血漬,只見她臉蛋上橫七豎八,都是細細條傷痕。從這傷痕看來,顯然是用倚天劍所劃,殷離自被紫衫龍王金花所傷之後,流血甚多,體内蘊積的千蛛毒液隨血而散,臉上浮腫已退了一大半,幼時俏麗的容顏,這個數日來已略復舊觀,但臉上多了這十幾道劍傷,又變得猙獰可怖。無忌又是心痛,又是惱怒,切齒道︰「趙明啊趙明,但教你撞在我手裡,我不在你臉上也這麼劃下十七八道傷痕,我張無忌枉自爲人了。」定了定神,忙到山邉去採了些止血的草藥,嚼爛了替殷離敷在臉上,又去敷在周芷若的頭皮和耳上。

周芷若打了個呵欠,睜開眼來,忽見無忌伸手在她頭上摸索,羞得滿臉通紅,伸手推開了他的手臂,嗔道︰「你\dash{}你怎麼啦\dash{}」一句話没説完,想是覺得耳上痛楚,伸手一摸,「啊」的一聲,驚呼了出來,跳起身來,道︰「爲什麼?」突然雙膝一軟,撲在無忌的懷中。無忌忙扶住了她,慰道︰「周姑娘,你别怕。」周芷若一眼看到殷離臉上可怖的模樣,忙伸手撫摸自己的臉,道︰「我\dash{}我也是這樣麼?」無忌道︰「不!你只受了些輕傷。」周芷若驚道︰「是那些波斯惡徒幹的麼?我\dash{}我怎地一些児也不知道?」無忌嘆了口氣道︰「只怕是趙姑娘幹的。昨晩的飲食之中,她下了毒。」

周芷若呆了半晌,摸著半邉耳朶,痛哭起來。張無忌慰道︰「周姑娘,幸好你所傷不重,耳朶上受了些損傷,將頭髮披將下來蓋過了,無礙觀瞻。」周芷若道︰「你還説頭髮呢?我頭髮也没有了。」無忌道︰「頂心上少了一片頭皮,兩旁的頭髮可以攏過來掩住。要不然,戴一些假髮\dash{}」周芷若嗔道︰「我爲什麼要戴假髮?到這時候,你還在出力迴護你的趙姑娘。」無忌碰了個莫名其妙的釘子,訕訕的道︰「我纔不迴護她呢?她如此狠心辣手,將殷姑娘傷成這般,我\dash{}我纔不饒她呢。」眼見殷離臉上的模樣,不禁怔怔的掉下泪來。

這時一人昏迷不醒,三人中毒乏力,處身荒島之上,饒是謝遜和張無忌一世英雄,也不由得徬徨失措。無忌盤膝坐下,試一運功,覺得中毒著實不淺。本來中了「十香軟筋散」的毒後,非趙明的獨門解藥不能消解,但他想,與其在此束手待斃,不如以己身超凡入聖的深厚功力,與這「十香軟筋散」的劇毒試相抗衡,當下運起内息,將散在四肢百骸上的毒藥慢慢搬到丹田之中,強行凝聚。這是九陽神功中最爲深奥的逼毒消蟲法,縱然中了最厲害的腐體蝕骨之毒、摧心傷肝之蟲,也能逐步驅出。他用了一個多時辰功,心下略慰,只是此法須以九陽神功爲根基,無法傳授謝遜和周芷若昭行,惟有待自己驅毒淨盡之後,再以神功助謝周二人驅毒。

這等功夫説來簡捷,做起來却是十分繁複,無忌到得第七日上,也只驅除體内三成的毒素。須知十香軟筋散實是非同小可,當日少林神僧空聞、空智,武當派宋遠橋、兪蓮舟,峨峨嵋派滅絶師太等那一個不是内力通神,中毒後竟是半點勁力也使不出來。無忌在七天之中驅得三成毒素,回復一二成功力,當今之世,已是再無第二人了。好在這毒藥只是令人使不出内勁,於身子却是無害。周芷若起初幾日極是著惱,後來倒也漸漸慣了。陪著謝遜搏魚射鳥,燒火煮餐。她晩間在島東一個山洞中獨居,和無忌等離得遠遠地。謝遜雙眼雖盲,却早知她對無忌頗有情意,然她如此儼然守禮,連笑話也不跟無忌多説一句,心中對她好生敬重。無忌却是暗自慚愧,心想趙明之禍,全是由自己而起,這個趙姑娘明明是蒙古的郡主,是明教的對頭死敵,武林中不知有多少高人折在她的手裡,自己對她居然不加防範,當眞是愚不可及了。謝遜和周芷若對他並無一言責備,然他二人越是一句不提,他越是心中難過,有時見到周芷若的眼色,隱隱似説︰「你爲趙明的美色所迷,釀成了這等大禍。」

無忌體内的毒素一天少於一天,殷離的傷勢却是越來越重。荒島上藥草寥寥,無忌空自醫術通神,却是無法救治,他明知殷離的傷是可救治的,然而手邉就是没藥。倘若小島上生有大樹,他早已紮成木筏,冒險内航,偏生島上樹木都是又矮又小,僅彀作柴薪之用。無忌若是不明醫術,那也不過是焦慮而已,此時却如萬把尖刀,日夜在他心頭剜扎。這一晩無忌嚼了些退熱的草藥,餵在殷離口中,眼見她難以下嚥,心中一酸,泪水一滴滴的滴在她的臉上。

殷離忽然睜開眼來,微微一笑,説道︰「阿牛哥哥,你别難過,我要到陰世去見那個狠心短命的小鬼張無忌去了。我要跟他説,世上有一位阿牛哥哥,待我是這樣好,可比張無忌好上千倍萬倍。」無忌喉頭哽咽,一時打不定主意,是否要向她吐露,自己實在就是張無忌。殷離握住了他手,説道︰「阿牛哥,我始終没答應嫁給你,你恨我麼?我猜你是爲了討我喜歡,説著騙騙我的。我相貌醜陋,脾氣乖僻,你怎會要我?」

無忌道︰「不!我没騙你。你是一位情深的意眞的好姑娘,如果得能娶你爲妻,實是我生平之幸。等你身子大好了,咱們諸事料預定當,便即成婚,好不好?」殷離伸出手來,輕輕撫了撫他的面頰,搖頭道︰「阿牛哥哥,我是不能嫁你的。我的心,早就許給了那個狠心的,兇惡的張無忌了\dash{}阿牛哥哥,我有點児害怕,到了陰世,能遇到他麼?他仍舊會對我這樣狠霸霸的麼?」無忌見她説話神智清楚,臉頰潮紅,心下暗驚︰「這是他迴光返照之象,難道她便要畢命於今日嗎?」一時呆呆出神,没聽見她的話,殷離抓住了他手腕,又問了一遍。無忌柔聲道︰「他永遠會待你很好的,當你心肝寶貝児一般。」殷離道︰「能有你待我一半児好麼?」無忌道︰「老天爺在上,張無忌誠心疼你愛你,他早就懊悔小時候待你這般兇狠了。他\dash{}他跟我一般無異,没半點分别。」殷離嘆了口氣,嘴角上帶著一絲微笑,道︰「那\dash{}那我就放心了\dash{}」握著他的手漸漸鬆開,雙目閉目,再也没氣了。

無忌將她屍身抱在懷裡,心想她直到一瞑不視,仍是不知自己便是張無忌,這些日來,她總是昏昏沉沉,無法跟她説知眞相。在她迴光返照的片刻神智清明,却又是什麼也來不及了説了。其實,到了這個地步,説與不説,也没什麼分别。無忌心頭痛楚,竟是哭不出聲來,心中只想︰「若不是趙明損她臉頰,她的傷未必無救。若不是趙明棄了咱們在這荒島之上,只要數日間趕回中原,我定有法子救得她性命。」恨恨的衝口而出︰「趙明啊趙明,你如此心如蛇蠍,有朝一日落在我手中,張無忌決不饒你性命。」

忽聽得背後一個冷冷的聲音説道︰「待得你見到她如花似玉的容貌,那時又下不了手啦。」無忌霍地轉過身來,只見周芷若俏立風中,臉上滿是鄙夷之色。無忌又是傷心,又是慚愧,説道︰「我對著表妹的屍骨發誓,若不手誅妖女,張無忌無顏立於天地之間。」周芷若道︰「那纔是有志氣的男児。」搶上幾步,撫著殷離的屍身,痛哭起來。謝遜聽到哭聲,尋聲而至,得知殷離身亡,也是不禁傷感。

張無忌在山崗上掘了一個墓穴,將殷離葬好,拆下一段樹幹,剝去樹皮,用殷離的匕首在樹幹上刻道︰「張無忌謹立。」一切定當,這纔拜伏在地,痛哭失聲。周芷若勸道︰「古人言道︰兩情若是欠長時,又豈在朝朝暮暮?她對你一往情深,你待她也是仁至義盡。只須你不負了今日之言,殺了趙明爲她報仇,殷家妹子縱在九泉,也是含笑的了。」

張無忌一番傷心,本已凝聚在丹田之中的毒素復又散開,再多費了七八日之功,纔漸行凝聚,待得盡數驅出體外,已是月餘之後了。這小島可和冰火島、靈蛇島不同,島上樹木稀疏,絶無野獸,三人的日子過得萬分艱難。幸而周芷若知無忌心傷殷離之死,惱恨趙明之詐,復又憐惜小昭之去,待他加意的溫柔體貼。

張無忌運神功替謝遜驅去了體内十香軟筋散的毒性後,本該替周芷若驅毒,但想到這驅毒法法,須以一掌於貼對方後腰,一掌貼於臍上小腹,青年男女,怎能如此肌膚相親?但若非這般運功,又不能將自身的九陽眞氣輸入她的體内,一連數日,心下好生躊躇,難以決斷。這日晩間,謝遜忽道︰「無忌,咱們在此島上,你想要過多少日子?」無忌一怔,道︰「那就難説得很,只盼能有船隻經過,救了咱們回歸中土。」謝遜道︰「這一個多月來,遠遠也曾見到船帆的影子麼?」無忌道︰「没有。」謝遜道︰「是了!説不定明天便有船隻來到,但説不定再過一百年也没船經過。」

無忌嘆了口氣道︰「這荒島孤懸海中,非海船航道所經,咱們重回中土的盼望,原是十分渺茫。」謝遜道︰「時候不長,那也没多大害處,但這種劇毒侵肌蝕骨,日子久了,自然五臟六腑都受損傷。」謝遜道︰「是啊。那你怎能不儘早設法給周姑娘驅毒?周姑娘的父母是本教中人,她本人又是峨嵋派一派的掌門,這等溫柔有德的淑女,到那裡求去?難道你嫌她相貌不美麼?」無忌道︰「不,不,周姑娘倘若不美!天下那裡還有美人?」謝遜道︰「那我替你作主,娶了她爲妻室。這男女授受不親的腐禮,就不必顧忌了。」

周芷若本來一旁聽著他父子二人説話,忽聽得説到自己身上來了,羞得滿臉通紅,站起身來便走。謝遜躍起身來,張開雙手,攔在她的身前,笑道︰「别走,别走!這樣媒人,我今日是做定的了。」周芷若道︰「謝老爺子,你爲老不尊!咱們只盼想個法児回中土去,這當児怎地説起這些不三不四的話來?」謝遜哈哈大笑,説道︰「男女好合,乃是終身大事,怎麼是不三不四了?無忌,你父母也是在荒島上自拜天地成婚。他們當日若非破除了這些世俗的禮法,世上那裡有你這個小子?何況今日有你義父爲你作主婚。難道你不喜歡周姑娘麼?不想替她驅除體内的劇毒麼?」周芷若掩了面只是要走,謝遜拉住了她的衣袖,笑道︰「你走到那裡去,明日咱們不見面了麼?啊,我知道了,你是不肯叫我這老瞎子做公公?」周芷若道︰「不,不,不是的。」謝遜道︰「那你是答應?」周芷若只説︰「不,不!」謝遜道︰「你是嫌我這義児太過不成材麼?」

周芷若頓了一頓,説道︰「張公武功卓絶,名揚江湖。得\dash{}得婿如此,更有何求!只是\dash{}只是\dash{}」謝遜道︰「怎麼?」周芷若向無忌微掠了一眼,道︰「他\dash{}他心中實在是喜歡趙姑娘,我是知道的。」

謝遜咬牙道︰「趙明這小賤人害得咱們如此慘法,無忌豈能執迷不悟。無忌,你自己倒説説看。」張無忌心中一片迷惘,想起趙明盈盈笑語,種種動人之處,只覺若能娶趙明爲妻,長自和她相伴,那纔是生平至福,但一轉念間,立時想起殷離臉上橫七豎八、血淋淋的劍傷來,忙道︰「那趙姑娘是我大仇,我要殺了她爲表妹雪恨。」謝遜道︰「照啊,周姑娘,那你還有什麼疑忌?」周芷若低聲道︰「我不放心。除非\dash{}除非你要他\dash{}立下一個誓來。否則我寧可毒發身死,不要他助我驅毒。」謝遜道︰「無忌,快立誓!」無忌雙膝跪地,説道︰「我張無忌若是忘了表妹血仇,天地不容。」周芷若道︰「我要你説得清楚些,對那位趙姑娘怎樣?」謝遜心中暗笑︰「這位周姑娘的醋勁好大,還没過門,便要將丈夫管得服服貼貼。站穩了地步,不讓他日後有翻覆的餘地。」説道︰「無忌,你就説得清楚些。」

張無忌朗聲道︰「妖女趙明爲其韃子皇室出力,若我百姓,傷我武林義士,復又盜我義父寶刀,害我表妹殷離。張無忌有生之日,不敢忘此大仇,如有違者,天厭之,地厭之。」周芷若嫣然一笑,道︰「只怕到了那時候,你又手下容情呢。」謝遜道︰「我説呢,揀日不如撞日,咱們江湖豪傑,還管他什麼婆婆媽媽,繁文褥節,你小兩口子不如今日便拜堂成親吧。這十香軟筋散早一日驅出好一日。」無忌道︰「不!義父,芷若,你們聽我一言。殷姑娘待我情意深重,她自幼便以我爲夫,我心中也已以她爲妻,雖無婚姻之事,却有夫婦之義,她屍骨未寒,我何忍即行另結新歡?」

謝遜沉吟道︰「這話倒也説得是,依你説那便如何?」張無忌道︰「依孩児之見,孩児今日先和周姑娘訂立婚姻之約,助她療毒驅毒,這就方便得多,天幸咱們得回中土,待孩児手刃趙明,奪回屠龍寶刀,交回義父手中,那時再和周姑娘完婚,可説兩全其美。」謝遜笑道︰「你倒想得挺美。要是十年八年,咱們也回不了中土呢?」張無忌道︰「三年之後,不論咱們是否能離此島,就請義父主持孩児的婚事便是。」謝遜點了點頭問周芷若道︰「周姑娘,你説怎樣?」周芷若垂頭不答,隔了半晌,纔道︰「我是個孤伶仃的女孩児家,自己能有什麼主意?一切全憑老爺子作主。」謝遜哈哈笑道︰「很好,很好。咱們三人一言爲定。你小兩口是未婚夫婦,不必再有什麼顧忌。無忌,你給我的小媳婦驅毒吧。」説著大踏步走向山後。

無忌道︰「芷若,我這番苦衷,你能見諒麼?」周芷若微笑道︰「只因是我這個醜樣的,你纔推三阻四,要是換了趙姑娘啊,只怕今晩就\dash{}」説到這裡,轉過了頭,不好意思再説。無忌怦然心動,尋思︰「當大夥児同在小船中飄浮之時,我曾痴心妄想,同娶四美。其實我心中眞正所愛,竟是個無惡不作、陰毒狡猾的小妖女。我枉稱英雄豪傑,心中却如此不分善惡,迷戀美色。」周芷若回過頭來,見他兀自怔怔的出神,站起身來,便要走開。無忌伸手握住她手一拉,不料周芷若功力未復,脚下無力,身子一晃,便倒在他的懷裡,嗔道︰「我是一世受定你的欺侮啦。」

無忌見她輕顰薄怒,楚楚動人,抱著她嬌柔的身子,低聲道︰「芷若,咱倆幼時在漢水中一見,不意終能如我所願。在光明頂我獨鬥崑崙、華山兩派四老之時,多謝你指點救命。」周芷若倚在他的懷裡,説道︰「那日我刺你一劍,你也恨我麼?」張無忌道︰「你没刺正我的心口,我便知你對我暗有情意了。」!周芷若{\upstsl{呸}}了一聲,臉頰暈紅,説道︰「早知如此,當日我一劍刺正你的心口,多少乾淨。也免得以後無窮歳月之中,給你欺侮,受你的氣。」無忌抱著她的雙臂緊了一緊,説道︰「我此後只有加倍疼你愛你。不知咱倆是否能回歸中土,我二人夫婦一體,我怎會給你氣受?」周芷若側過身子,望著他臉,説道︰「要是我做錯了什麼,得罪了你,你會打我、罵我、殺我麼?」無忌和她臉蛋相距不過數寸,只覺她吹氣如蘭,忍不住在她左頰上輕輕一吻,説道︰「似你這等溫文斯文、端莊貞淑的賢妻,那裡會做錯什麼事?」周芷若輕輕撫摸他的後頸,説道︰「便是聖人,也有做錯事的時候。我從小没爹娘教導,難保不會一時胡塗。」無忌道︰「當眞你做錯什麼,我自會好好勸你。」周芷若道︰「你對我決不變心麼?決不會殺我麼?」無忌在她額上又輕吻一下,柔聲道︰「你别胡思亂想了。那有此事?」周芷若顫聲道︰「我要你親口答應我。」無忌笑道︰「好吧!我對你決不變心,決不會殺你。」

周芷若凝視他雙眼,説道︰「我不許你嘻嘻哈哈,要你正正經地説。」無忌笑道︰「你這個小小腦袋之中,不知在想些什麼。」心想︰「總是我對趙明、對小昭、對表妹到處留情,令她難以放心。可是自今而後,那裡便有此事?」於是收起笑容,莊言道︰「芷若,你是我的愛妻。從前三心兩意,只望你既往不咎。我今後對你決不變心。就算你做錯了甚麼,我是重話也不會捨得責備你一句。」周芷若道︰「無忌哥哥,你是男子漢大丈夫,可要記得今晩跟我説過的話。」指著初升的一勾明月,説道︰「天上的月亮,是咱倆的證人。」

張無忌道︰「對,你説得不錯。大上明月,是咱倆證人。」他仍是將周芷若摟在懷裡,望著天邉明月,説道︰「芷若,我一生受過很多很多人的欺騙,從小爲了太過輕信,不知吃過多少苦頭,到底有多少次,這時候記也記不起來了。只有冰火島上,和爹爹、媽媽、義父三個人在一起的時候,那纔没有人世的奸詐機巧。我第一次回中原,一個叫化子弄蛇,騙我探頭到布袋中瞧瞧裡面的蛇。不料他把布袋套在我頭上,將我擒了去。我又那料得到,咱們同生死、共患難的來到這個小島之上,趙姑娘竟會在第一晩的食物之中,便下了十香軟筋散的劇毒?」周芷若笑道︰「你是不到黃河心不死,到得黃河悔已遲。」

無忌心中突然間充滿了幸福之感,説道︰「芷若,你纔眞正是我永遠永遠的親人。你一直待我很好。日後咱們倘若得能回歸中原,你會幫我提防著許多奸滑的小人。有了你這個賢内助,我會少上很多當了。」周芷若搖頭道︰「我是個最不中用的女子,懦弱無能,人又生得蠢。别説和絶頂聰明的趙姑娘天差地遠,便是小昭,這等深刻的心機,我那又及得上她的萬一?你的周姑娘是個老老實實的笨丫頭,難道到今天你還不知道麼?」無忌道︰「只有你這等忠厚賢慧的姑娘,纔不會騙我。」周芷若轉過身來,將臉伏在他的懷裡,柔聲道︰「無忌哥哥,我能和你結爲夫婦,心裡是快樂得了不得,只盼你别因我愚笨無用,將來瞧我不起、欺侮我。我\dash{}我會盡我所能,好好的服侍你。」

兩人坐在海濱,情話綿綿,不知夜之漸深。

次日無忌即以九陽神功,助周芷若驅除體内毒素。運功之下,初時竟是出於意料之外的方便,想是她飲食不多,中毒不如他與謝遜之深。但驅到第七日上,忽覺周芷若體内有一極陰寒的阻力,和他的九陽眞氣相激相抗,周芷若雖盡力克制,亦是不易引導九陽眞氣入體。無忌驚異之下,請教義父。謝遜沉吟半晌,説道︰「這道理我也説不上來,多半是她峨嵋派歷代師父都是女子,所習内功偏於陰柔一路。」無忌點頭稱是。好在周芷若内功修爲和無忌相差極遠,當無忌催動神功之時,她體内陰功終被壓制了下去,但如此運功,却又比替謝遜驅毒時費力得多。無忌隱隱覺得她體内陰勁蘊積未成,但日後成就,竟是非同小可,不禁讚道︰「芷若,尊師滅絶師太實是一代人傑。她傳給你的内功,法門高深之至,此刻我已覺得出來。你遵此用功,日後成就可和我的九陽神功並駕齊驅,各擅勝場。」周芷若笑道︰「你騙我呢!峨嵋派武功!怎能和張大教主的九陽神功、乾坤大挪移法相比?」無忌道︰「芷若,你天性淳厚,武功的招數上雖然所學不多,但内功的根基已紮得極佳。我太師父言道,武學鑽研到最高深時,往往和每人資質有関,而且未必聰明穎悟的便一定能學到最高的境界。據説貴派創派祖師部女俠的父親郭靖大俠,資質便十分魯鈍,可是他武功修爲震古鑠今,太師父説他自己,或者尚未達到郭大俠當年的功力。你峨嵋派内功的法門似尚在武當派之上,依我瞧啊,你將來的成就,當可凌駕滅絶師太之上。」

周芷若橫了他一眼,嬌嗔道︰「你要討好我,也不用説我武功好。我能學到先師十分之一的本事,也就心滿意足了。你幾時把你的九陽神功,挪移乾坤功夫教我一兩手,我纔多謝你呢。」無忌沉吟未答。周芷若道︰「你説我不配做張大教主的徒弟嗎?」無忌道︰「不!我察覺你的内和我學截然不同,那是壓根児相反的路子。要是我來教你,那是世上艱險無比之事。」

周芷若嘆道︰「你不肯教,也就是了。學武功最多是學不成,還能有甚麼危險?」張無忌正色道︰「不,不!我這九陽神功是純粹陽剛的内功,你現下所習的峨嵋派内功,却純是走的陰柔路子。如果你再練我的功夫,陰陽匯於一體,除非是如我太師父這等武學奇才,或許能使之水火相濟,剛柔相調,只要差得一步,那便是走火入魔之禍。{\upstsl{嗯}},等你日後内功大成之時,我那挪移乾坤的心法,你倒是可以學的。」周芷若笑道︰「我是跟你説著玩呢。以後我時時刻刻都跟你在一起,你的武功和的的武功有什麼分别?我生來懶懶散散,你的九陽神功一定難練得緊,你便是逼著我練,我也怕難呢。」無忌聽她如此説,心中甚是甜蜜。

如此情意纏綿,不覺時日之逝。匆匆過了數月,冬去春來,周芷若説自覺體力已然全復,想來毒性已然驅盡。這一日春光明媚,島東幾株桃花開得甚美,無忌折了幾枝桃花,去插在殷離的墓前,想起這位表妹一生困苦,恐怕連一天福也没享過,正自傷神,必聽得海中鷗鳥大聲咕噪。無忌一抬頭,忽見遠處海上一艘帆船,正鼓浪向島上駛來,這一下喜出望外,忙縱聲叫道︰「義父,芷若,有船來啦!」

謝遜和周芷若聽到叫聲,先後奔到無忌身旁。周芷若顫聲道︰「無忌哥哥,怎麼會有船隻到這荒島上來?」無忌道︰「那也眞奇怪得緊,難道是海盜船麼?」不到半個時辰,那帆船已在島外下錨停泊,一艘小船划向島來。無忌等三人迎到海灘,只見小船中的水手都是穿著蒙古水師的軍裝。無忌心中一動︰「難道趙姑娘良心發現,又回到島上來?」斜向周芷若一瞥,只見她秀眉微蹙,胸口起伏,顯是也擔著極大的心事。片刻間小船划到,五名水手上得海漢灘,爲首的一名水師軍官躬身向無忌道︰「這位是張無忌張公子?」無忌道︰「正是。長官何人?」那人聽到無忌自承,神色間極是欣慰,説道︰「小人賤名拔速台,今日找到了公子,當眞幸運之至。小人奉命前來,迎接張公子、謝大俠回歸中土。」他只説張謝二人,却不提周芷若的名字。張無忌還了一揖,説道︰「長官遠來辛苦,却不知是奉何人所遣?」那拔速台道︰「小人是駐防福建的達花魯水師提督麾下,奉勃爾都思將軍之命,前來迎接。勃爾都思將軍一共派出海船八艘,在這一帶閩浙粤三省海尋找公子和謝大俠。想不到倒是小人立下首功。」他言下之意,顯是他的上司許下諾言,誰能找到張無忌的便有升賞。

無忌聽他所説的那些蒙古將軍的名字均不相識,料想那些將軍也是轉輾奉了趙明之命,問道︰「你可知爲何前來接我?」拔速台道︰「勃爾都思將軍吩咐,張公子是大大的貴人,乃是當世的英雄豪傑,命小人找到之後,用心侍候。至於何以迎接公子,小人職位低微,未蒙將軍示知。」周芷若插口問道︰「可是明明郡主之意麼?」拔速台一怔,道︰「明明郡主?小人没福見過。」周芷若冷冷的道︰「甚麼福不福的?」拔速台一怔,道︰「明明郡主乃是我蒙古第一美人,不,乃是天下第一美人,文武全才,是汝陽王爺的愛女。小人怎有福氣一見郡主的金面?」周芷若哼了一聲,不再言語了。

無忌向謝遜道︰「義父,那麼咱們便上船去吧?」謝遜道︰「咱們到那邉山洞中取了隨身物品,便可上船,長官請在此稍候。」拔速台道︰「讓小人和水手們替三位搬行李吧。」謝遜笑道︰「咱們有什麼行李?不敢勞動。」擕了無忌和周芷若的手,走到山後,站定脚步,道︰「趙明忽然派船來接咱們回去,其中必有陰謀,你們想該當如何應付?」

\chapter{丐幫聚會}

無忌道︰「義父,你想趙\dash{}趙明她\dash{}她會在這船上麼?」謝遜道︰「這小妖女若在船上,那倒好辦了。咱們只須留心飲食,免再著了她的道児。」無忌道︰「不錯,咱們把這児收藏著的鹹魚、乾果帶上船去,決不吃喝船上的物事。」謝遜道︰「我料想那趙明決計不在船上,她是欲師那些波斯人的故智,將咱們騙上船去,待航到大海之中,便有蒙古水師其餘的般隻出現,開炮將咱們的座船轟沉。」無忌背上出了一陣冷汗,顫聲道︰「難道她用心竟是如此毒辣?其實,她將咱們放逐在這小島之上,讓咱們自生自滅,永世不得回歸中土,也就是了。咱三人又没什麼事對不起她?」

謝遜冷笑道︰「你將她囚在萬法寺中的六大派高手一齊放了出來,她焉有不記恨之理?再説,明教教主失蹤,此刻教中上下人等,定在大舉訪尋,難保不尋到這荒島上來。只有令咱們葬身海底,那纔是斬草除根。」謝遜哈哈一笑,隨即嘆道︰「無忌孩児,這些執掌軍國重任之人,焉會愛惜人命?像你這般心腸仁慈,蒙古人能橫絶四海、掃蕩百國麼?自古以來,那一個立大功名的英雄不是當機立斷,要殺便殺?别説區區官兵,便是自己父母子女,也顧不得呢。」無忌呆了半晌,説道︰「義父説得是。」他向來知道蒙古人對敵人極是殘忍暴虐,但想對自己部下總須盡力愛惜,此刻聽了謝遜之言,心中不禁呆了半截,自覺此番便算回歸中土,統率中原豪傑驅除韃子,但説到治國平天下,決非自己所能。

周芷若道︰「義父,你説咱們該當如何?」謝遜道︰「你有什麼妙計?」芷若道︰「那麼咱們别上這船吧。跟那蒙古軍官説,咱們在這児住得很好,不想回中原去了。」謝遜笑道︰「那眞是傻丫頭的傻主意。咱們不上船,敵人也決計放咱們不過啊。咱們把這艘船中的官兵都殺盡,他們不能再派十艘八艘來麼?何況中原有多少大事,要無忌回去擔當,怎能讓他老死於這荒島之上?」周芷若俊臉通紅,低聲道︰「還是義父出個主意吧,咱們只聽義父吩咐便是。」謝遜略一沉吟,道︰「咱們須得如此如此。」無忌和周芷若一聽,齊稱妙計。

當下三人盡擕山洞中積儲的食物,搬回小船。無忌更到殷離墓前禱祝一番,灑泪而别,這纔上了大船。無忌在艙前艙後察看一番,果然並無趙明在内,船上水手之中,也無特異礙眼的人物,看模樣均是普通的蒙古官兵。

那船拔錨揚帆之後,只駛出數十丈,無忌反手一搭,已抓住了拔速台的右腕,另一手抽出他腰間佩刀,架在他的後頸,喝道︰「你聽我的號令,命梢公向東行駛!」拔速台大吃一驚,顫聲道︰「張公\dash{}公子,小\dash{}小人没敢得罪你啊。」無忌道︰「你聽我吩咐行事。稍有違抗,我便砍下你的腦袋。」拔速台道︰「是,是!」喝令道︰「梢\dash{}梢公!快向東\dash{}向東行駛。」梢公依言轉舵,那船橫掠小島,向東駛去。

無忌喝道︰「你蒙古人意欲謀害於我,我已識破你們詭計,快快招來!若有虛言,小人你的性命。」説著舉起右掌,往船邉上一拍,只見木屑紛飛,船邉登時缺下一大塊來。船上官兵見到,無不駭然。拔速台道︰「公子明鑒︰小人奉上司之命,迎接公子西歸,此外更無别情。小人只盼立此功勞,得蒙上司升賞,實無半分歹意。」無忌見他説得誠懇,確非虛語,於是放開他的手腕,走到船頭,左手提起一隻鐵錨,右手又提起一隻鐵錨,喝道︰「衆人看清楚了!」雙手一揚,兩隻各重數百斤的大鐵錨一齊飛向半空。衆官兵「嘩」的一聲,齊聲驚喊。

待兩隻大鐵錨落將下來,張無忌使出挪移乾坤的心法,雙手一掠一推,兩隻鐵錨又飛了上去。如此連飛三次,無忌才輕輕接住,將兩隻鐵錨放在船頭。蒙古人從馬上得天下,最佩服武勇之士,見了無忌如此驚人的武功,當眞是如同天神一般,不由得一齊拜伏,説道︰「張公子神勇,世所罕有,小人今日大開眼界。」無忌這麼一顯武功,將一干蒙古官兵收得服服貼貼,再也不敢稍起異心。

掌舵的梢公遵依無忌命令,駕船東駛,直航入大洋之中,一連三天,所見的唯有波濤接天。謝遜料得趙明所遣的炮船,必在閩粤一帶海面守候巡視,現下座船航入大洋已遠,決計不至和趙明的炮船相遇,到第五日上,才命梢公改道向北。這一向北,更是接連駛了二十餘日,直到海中見有浮冰,已知來到北海,憑她趙明再聰明十倍,也難猜到此船的所在,於是再命梢公折向西行,航返中土。這一個多月之中,無忌等不是取用自擕的食物,便是捉捕海中鮮魚爲食,對船上飲食,竟是不沾唇。

這一日午間,遙見西方出現了陸地。蒙古官兵航海已久,眼見歸來,盡皆歡呼。到得傍晩那大船已停泊岸旁,這一帶海岸都是山石,海水甚深,大船可直泊靠岸。謝遜道︰「無忌,你上岸去瞧瞧,這是什麼地方。」無忌答應了,飛身上岸。

一路行來,只見四下裡都是綠油油的森林,地下積雪初融,極是泥濘。走了一陣,樹林更加蔭深,一株參天古柏,都是數人方能合抱。無忌飛身上了一株高樹,四下一望,但見樹木無邉無際,竟是到了林海之中,再無人跡。他想便再向前,也是如此,當下回向船來,尚未走到岸旁,忽聽得一聲慘呼,聲音極是淒厲,正是從船上發出。無忌吃了一驚,展開輕身功夫,飛奔而回,撲上船頭。只見滿船橫七豎八,盡是蒙古官兵的屍首,自拔速台以下,個個屍橫船中,謝遜和周芷若好端端的站著,却不見敵人的蹤影。無忌驚問︰「義父,芷若,你們没事吧?敵人到那裡去了?」謝遜道︰「什麼敵人?你見到敵蹤麼?」無忌道︰「不!這些蒙古人\dash{}」謝遜道︰「是我和芷若殺的。」無忌更是驚奇,道︰「想不到這些韃子一回中土,便膽敢起意害人。」謝遜道︰「他們没敢起意害人,是我要殺了滅口。這些人一死,趙明便不知咱們已回中土。從此她在明裡,咱們在暗裡,找她報仇容易得多了。」

無忌倒抽了口氣,半晌説不出話來。謝遜淡淡的道︰「怎麼?你怪我手段太辣麼?韃子官兵是咱們敵人,用得著以菩薩心腸相待麼?」無忌不語,心想這些人對自己始終服侍唯謹,未有絲毫怠忽,雖説是敵人,但如此殺絶,總覺心中過意不去。謝遜道︰「常言道得好︰量小非君子,無毒不丈夫。己不傷人,人便傷己。那趙明如此對待咱們,咱們便當以其人之道,還治其人之身。」無忌道︰「義父説的是。」謝遜道︰「你放一把火,將船燒了。芷若,搜了屍首身上的金銀,撿三把兵刃防身。」兩人依言而行,在船上放了火,分别躍上岸來。這船船身甚大,直燒到半夜,方始煙飛火滅,連衆人屍首一齊化灰沉入海底,無忌見這麼一來,乾手淨脚,再無半點痕跡,心想義父行事雖然厲害了些,究竟是老江湖,非己所及。

三人胡亂在岸旁睡了一覺,次晨穿林向南而行。走到第二日上,纔遇到七八個採參的客人,一問之下,原來此地竟是関外遼東,距長白山已然不遠。待得和那些採參客人分手,周芷若道︰「義父,是否須得將他們殺了滅口?」無忌喝道︰「芷若你説什麼?這些採參客人又不知咱們是誰。難道咱們一路上見一個便殺一個麼?」

周芷若一呆,登時滿臉脹得通紅,無忌自和她相識以來,從未如此疾言厲色的對她説話。謝遜道︰「依我原意,也是要將這些採參客人殺了。教主既是不願多傷人命,咱們快些設法換了衣服,免露痕跡。」當下三人快步而行,一直走了兩日,纔出森林。見到一家農家,無忌取出銀兩,向老農購買衣服。但那農家極是貧寒,僅有一件老羊皮襖可以出讓。接連走了七八家人家,三人方湊齊了三套汚穢不堪的衣衫。周芷若素來愛潔,聞到衣褲上陳年累積的臭氣,幾欲作嘔。謝遜却十分歡喜,命二人用泥漿塗汚。無忌在水中一照,只見自己活脱成了遼東一丐,趙明便是對面相逢,也未必相識。

三人一路南行,這日來到一處大鎭甸上,那是進関的入經要道。三人走向鎭上一處最大的酒樓,無忌摸出一錠十兩重的銀子交在櫃上,説道︰「待咱們用過酒飯,再行結帳。」他是先怕自己衣衫襤褸,酒樓中不肯送上酒飯。豈知那掌櫃恭恭敬敬的站了起來,雙手將銀兩還給無忌,説道︰「爺們光顧小店,區區酒水粗飯,算得什麼?由小店作東便是。」無忌很是詫異,坐定後,低聲問周芷若道︰「咱們身上可露出什麼破綻?怎地這掌櫃的不肯收受銀子?」周芷若細査三人身上衣服形貌,宛然是三個乞丐,那裡有什麼形跡顯露?謝遜道︰「我聽那掌櫃的語氣之中,頗存懼意,咱們小心些便是。」

他剛説了這句話,只聽樓梯上脚步聲響,走上七個人來,説也湊巧,竟然也都是乞丐的打扮。這七人靠著窗口大模大樣的坐定。只見店小二恭恭敬敬的上前招呼,口中「爺前爺後」,當他們是達官貴人一般。無忌見這些乞丐有的負著五隻布袋,有的負著六隻,都是丐幫中職司頗高的弟子。店小二將酒菜吩咐了下去,尚未送上,又有五六名丐幫子弟上來。片刻之間,這酒樓上絡絡繹繹來了三十餘名丐幫幫衆,其中竟有三人是七袋弟子。無忌這纔恍然大悟,原來丐幫今日在此聚會,酒樓掌櫃誤會他三人也是丐幫中人了。他低聲向謝遜道︰「義父,咱們還是避開了這裡吧,免得多惹事端,看來丐幫今日所到的人不少。」正在此時,店小二送上一大盤牛肉,一隻燒雞,五斤白酒,謝遜腹中正餓,兩個多月來從未好好的飽餐一頓,聞到燒雞的香味,食指大動,説道︰「咱們悶聲不響的吃了酒肉便行,又礙他們什麼事了?」説著端起碗來,骨{\upstsl{嘟}}{\upstsl{嘟}}的喝了半碗白酒。天可憐見,謝遜流落海外二十餘年,直至今日,方得重{\upstsl{嚐}}酒味,這白酒烈而不醇,乃是常釀,在他却是如飲醍醐,似喝瓊漿。

他又是一口,將一碗白酒都喝乾了,忽然低聲道︰「小心,兩個大本領的人物來啦!丐幫中居然有這等人才!」無忌聽到樓梯上的脚步之聲,前面一人左足落脚重,右足落脚輕,後面一人却是一步重、一步輕。單是聽他二人脚步之聲,就知這兩人武功極是奇特。那兩人一走上樓梯頂口,嘩喇喇一陣響,樓上群丐一齊站起。謝遜作個手勢,三人也站起相迎。要知他三人坐在靠裡的偏角上,和衆人一齊坐著,那是極不惹眼,但當人人都站起身來,他三人倘若仍是大模大樣的坐著,只怕當時便有亂子。

只見第一人中等身材,相貌清秀,三絡長鬚,除了身穿乞丐服色,神情模樣,竟似個不第秀才。後面那人滿臉橫肉,虯髯戟張,相貌十分兇猛,只須再黑三分,活像是関公身旁手執大刀的周倉。這二人都是五十多歳的年紀,髯子均已花白,背上各負九隻小小的布袋。這九隻袋子只是表明他們身份,其形體之小,很難裝什麼物事。

張無忌心下尋思︰「百年前丐幫是江湖上第一聲威赫赫的大幫會。聽太師父言道,昔日丐幫幫主洪七公仁俠仗義,武功深湛,不論白道黑道,無不敬服。其後黃幫主、耶律幫主等也均是出類拔萃的人物,豈知數十年來主持非人,丐幫聲望大非昔比。現在幫主史火龍從不在江湖上露面,不知其人如何。這二人背負九袋,在丐幫中除了幫主而外,當以他二人位份最尊,那日靈蛇島上遣人奪我義父屠龍刀,不知和他二人亦有牽連否?」

自從明教的聖火令數十年前爲丐幫中奪去之後,明教和丐幫即是勢同水火。明教曾一再企圖奪回聖火令,雙方仇殺數次。中原武林人物向來認爲明教爲邪魔外道,每逢爭鬥,總是群相出手協助丐幫,是以明教每一回均告失敗。這一次屠龍刀和倚天劍爲趙明盜去,那六根聖火令却仍是藏在張無忌懷中,没有失落。想是趙明忌憚無忌太強,生怕他中了十香軟筋散之後仍有出奇的本領,因此不敢到他懷中搜索。張無忌眼見酒樓上丐幫人多勢衆,絲毫不敢大意,伸手懷中摸了摸那六根聖火令,心想楊破天教主的遺書諄諄以奪回聖火令相囑,莫要一個不小心,又被丐幫奪了回去。

只見那兩名九袋長老走到中間一張大桌旁坐下。那周倉模樣的長老從布袋中摸出一長約四尺的竹棒來,放在桌上。群丐中登時有一半人拜伏在地,説道︰「汚衣派弟子參見掌棒龍頭。」張無忌因丐幫是本教大敵,曾聽楊逍詳細説過丐幫中的情形,知道丐幫歷來分爲汚衣、淨衣兩派。這時見拜伏的群丐個個衣衫極是汚穢,心知那掌棒龍頭便是汚衣派的首領。又見那秀才模樣的長老從布袋中取出一個缺口破缽,雙手捧著放在桌上。其餘衣衫乾淨的群丐便各拜倒,説道︰「淨衣派弟子參見掌缽龍頭。」兩個龍頭右手一揮,説道︰「起來吧!」群丐這纔紛紛歸座。無忌手中捏了一把汗,要知站起來迎接丐幫長老,那也罷了,要他們跪地拜伏,却是萬萬不可。幸好酒樓上亂糟糟一團,他三人又坐在僻處,兩名龍頭長老四隻眼睛望著屋頂,對群丐傲不理睬,因此也没見到他三人並未拜伏。

群丐雖是在酒樓之中飲食,却也不脱乞児的習氣,伸手抓菜,捧碗喝湯,吃得狼藉一團。無忌和謝遜留神傾聽,想聽那兩個龍頭長老説此什麼。不料他二人儘是飲酒吃菜,除了説些「你來一碗!」

\qyh{}這牛肉很香!」之類,一言不涉正事。兩旁群丐更是{\upstsl{吆}}喝猜拳,鬧酒搶菜,嘻嘻哈哈的嚷成一片。待得兩名龍頭長老食畢下樓,群丐也是酒酔飯飽,登時爭先恐後的一鬨而散。

謝遜待群丐散盡,低聲道︰「無忌,你瞧如何?」無忌道︰「丐幫這許多人物在此聚會,決不能是大吃大喝一頓便算。我猜晩間在什麼僻靜之處,定然再行聚集,商量正事。」謝遜點頭道︰「依我之見,亦必如此。丐幫是本教大敵,此事既教咱們撞見了,不能便此放過。須得打探明白,瞧他們是否另有圖謀本教的奸計。」當下三人下樓到櫃面付帳,掌櫃的甚是訝異,説什麼也不肯收。無忌心想︰「看來丐幫鬧得這裡的茶館酒樓都嚇怕了,吃喝不用付錢。只此一端,已可知他們平素的橫行不法。」

三人在僻靜處找了一家小客店歇宿。鎭上丐幫幫衆雖多,但依照向例,無一住店,因此在這客店中倒是不虞撞到丐幫人物。謝遜道︰「無忌,我眼不見物,這種打探訊息的事,幹起來諸多不便,芷若武功不高,陪著你去也幫不了忙,還是偏勞你一人吧。」無忌道︰「正好如此。」他在客店中稍作休息,便即出門。一走到街上,自南端直走到北端,竟没見到一名丐幫弟子。

張無忌尋思︰「不到半個時辰之間,鎭上丐幫幫衆突然人影全無,料想走得不遠。」當下走向一間南貨店,瞪起雙眼,伸拳在櫃台一擊,喝道︰「喂,掌櫃的,我那許多兄弟走向那裡去啦?」櫃面上的店伴見到他這副凶神惡煞的模樣,只道是丐幫中的一個惡丐,個個心驚肉跳,内中一人膽大子較大,指著北方,陪笑道︰「貴幫朋友們絡繹都向北去了。大爺喝茶麼?」無忌唱道︰「不喝!喝什麼他媽的臭茶!」轉身大踏步向北,肚中暗暗好笑。

他快步走出鎭甸不遠,只見左首路旁長草中,人影一閃,一名丐幫弟子站了起來,瞧模樣是要上前喝問。張無忌脚下加快,身子如箭離弦,倏忽而過。那丐幫弟子擦了擦眼睛,還疑心自己眼花,怎地忽然似乎有人,忽然不知去向,無忌心想丐幫沿途佈了卡子,好不戒備森嚴,當下展開輕功,向北疾馳,他眼光何等敏鋭,丐幫佈在樹叢草中、山間石邉的卡子,一一落入他的目中,反倒成爲指引的路標。奔出四五里路,但見三步一崗,五步一卡,哨位越來越密。這些人雖和無忌的武功相差極遠,但青天白日,要盡數避過他們的眼光,却也著實不易,到了後來,只好避過了大路,曲曲折折的繞道而行。眼見一條山道,通向山腰中的一座大廟,無忌料知群丐必在廟中聚會,一提氣,奔向東北角上,再折而向西,繞過群丐的卡子,直欺到廟側。只見廟前一塊大匾上冩著「彌勒神廟」四個大字,廟貌莊嚴,起得甚是雄偉。無忌暗想︰「瞧這模樣,見大幫中重要人物到得不少。我若是混在人叢之中,難免被他們發覺。」四下一打量,見大殿前的庭中左邉一株古松,右邉一株古柏。雙樹蒼挺立,高出殿頂甚多,那松樹更是枝葉密茂,倒可藏身其間。於是繞到廟後,飛身上了屋頂,低伏著身子,走到簷角,輕輕一縱,如一溜煙般落到了松樹之頂,從一根大枝幹後望將出去,心中暗叫一聲︰「僥倖!」殿中風光,盡收眼底。

只見大雄寶殿的地下,兩旁黑壓壓的坐滿了丐幫幫衆,少説也有三百來人。這些人一齊朝内,是以無忌躍上松樹,人影一晃,竟然無人知覺。殿中放著五個蒲團,虛座以待,顯是在等什麼人到來,殿中雖是聚了三四百人,竟無半點聲息,和酒樓上亂糟糟地搶菜爭食的情景渾不相同。無忌心想︰「丐幫享名數百年,近世雖然中衰,昔日典型,究未盡去。那酒樓中的混亂模樣,是平日的神情。由此而觀,幫中長老部勒幫衆,執法實極嚴謹。」

大雄寶殿居中坐著一尊彌勒佛,袒腰露出了一個大肚子,張大了笑口,顯得甚是慈祥。無忌正打量間,忽聽得壁後一人喝道︰「當缽龍頭到!」殿中群丐霍地站起,垂手而立。那秀才模樣的掌缽龍頭手捧破缽,緩步而出,站在右首。那人大喝︰「掌棒龍頭到!」那周倉般的九袋長老雙手高舉一根竹棒,大踏步走了出來,站在左首。那人喝道︰「執法長老到!」只見一個萎靡不振,身形瘦小的老丐走了出來,手中持著一根破竹片。此人脚下輕捷,走動時片塵不起。無忌心道︰「此人好高的輕功,可和本教布袋和尚説不得不相上下,只較韋蝠王稍遜半籌。」又聽那聲音喝道︰「傳功長老到!」這次出來的是一個白鬚白髮的老丐,一根根如銀絲般的鬚髮隨風晃動,臉上似笑非笑,似哭非哭,説不出古怪詭異,這老丐空著雙手,從他身形步法之中,却看不出武功的深淺。這四人將四個蒲團移向下首,只留下中間一個蒲團,然後彎腰躬身,齊聲説道︰「有請幫主大駕!」無忌心中一凜「只聽得丐幫現任幫主名叫『金銀掌』史火龍,武林中極少有人見過他的眞面目,却不知是何等的人物?」

大雄寶殿上群丐一齊躬身,過了良久,只聽得屏風後脚步聲響,大踏步走出一條大漢來。但見他身高七尺,魁梧之極,紅光滿面,竟似一個大官員、大豪紳一般的模樣,右手中嗆{\upstsl{啷}}{\upstsl{啷}}的不住響動,搓弄著兩枚大鐵膽。他身上衣衫雖非富麗,却也決不是乞児模樣。他走到殿中一站,群丐齊聲説道︰「座下弟子,參見幫主大駕。」丐幫幫主史火龍手一揮,説道︰「罷了!小子們都好啊?」群丐道︰「幫主安好。」待丈火龍在中間的蒲團上坐下,各人才分别坐地。

史火龍轉頭向那掌缽龍頭道︰「林兄弟,你把金毛獅王和屠龍刀的事,向大夥児説説。」無忌聽到「金毛獅王和屠龍刀」這幾個字,更是全神灌注的傾聽。那掌缽龍頭站起身來,向幫主打了一躬,轉身説道︰「衆家兄弟︰魔教和本幫爭鬥了六十年,代代成仇。自從魔教教主的令符聖火令落入本幫手中之後,魔教始終處於下風。近來魔教立了一個新教主,名叫張無忌,本幫有人參與圍攻光明頂之役,曾見到此人是個無知少年。諒這等乳臭未乾、黃毛未褪的小児,成得什麼大事?焉能與本幫史幫主的雄才偉略相抗?」群丐歡聲雷動,一齊鼓掌,史火龍臉上頗現得意的神色。

那掌缽龍頭又道︰「只是魔教立了新教主後,本來四分五裂、自相殘殺的局面,登時改觀,倒成了本幫的心腹大患。近一年來,魔教的魔頭們在各路起事,淮泗一帶有韓山童、朱元璋,兩湖一帶有徐壽輝,連敗元兵,佔了不少地方,可説頗成氣候。倘若眞給他們成了大事,逐出韃子,那時候本幫數十萬兄弟們,可都是死無葬身之地了。」群丐大怒{\upstsl{吆}}喝︰「決不能讓他們成事!」

\qyh{}丐幫誓與魔教死拚到底。」

\qyh{}魔教倘佔了天下,本幫兄弟們還有活命嗎?」一時彌勒廟中群丐憤慨激昂,大聲叫嚷。無忌躱在松樹的針葉叢後,尋思︰「想不到我身在海外數月,弟兄們倒是大有所成。丐幫這番顧慮,也非無因,丐幫人數衆多,不可輕侮,若得他們擕手拒元,大事更易成功。該當如何,方得和他們盡釋前嫌,化敵爲友?」

掌缽龍頭待群丐騷嚷稍靜,説道︰「史幫主向來在吹簫山莊靜養,長久不涉江湖,但遇上了這等大事,非得親自主持不可。也是天祐我幫,八袋長老陳友諒結識了一位武當子弟,得到了一個極其重要的訊息。」他提高聲音説道︰「陳長老,請陪宋少俠出來和衆兄弟見見。」壁後有人應道︰「是!」兩個人擕手而出。一個三十來歳年紀,精神奕奕,正是靈蛇島上謝遜饒了他一命的陳友諒。另一個二十七八歳,相貌俊美,腰懸長劍。無忌一見,不禁吃了一驚,原來此人竟是宋遠橋之子宋青書。當時他被趙明囚禁在萬法寺中,得范遙和張無忌救出,那料到竟會和丐幫混在一起。

兩人走到殿中,先向史火龍行禮,再向傳功、執法二長老,掌棒、掌缽二龍頭作了一揖,然後向群丐團團抱拳。掌缽龍頭説道︰「陳長老,你將此事的前因後果,跟衆兄弟説説。」陳友諒擕著宋青書的手,説道︰「衆家兄弟,我幫得蒙宋少俠相助,當眞是天大的機緣。這位宋青書宋少俠,乃是武當派宋遠橋宋大俠的公子,日後武當派的掌門,非他莫屬。那魔教教主張無忌,可説是宋少俠的師弟,因此魔教中的種種情由,宋少俠可説瞭如指掌。數月之前,宋少俠和我説起,魔教的大魔頭金毛獅王謝遜,已到了東海靈蛇島上\dash{}」那執法長老忽然插嘴道︰「武林中找尋金毛獅王,當眞無所不用其極,數十年來始終不知他的下落,宋少俠却何忽然得知?老夫想要請教。」

張無忌心中本來一直存著一個疑團︰「謝遜從極北的冰火島南來靈蛇島,此事該當十分隱祕,何以竟會讓丐幫得知訊息?」這時聽那執法長老問起,自是加倍的留神傾聽。只聽陳友諒道︰「托幫主洪福,一切機緣十分湊巧。東海有一位金花婆婆,不知如何,竟會得知了謝遜的所在。這老婆婆生長海上,精熟航海之事,居然給她找到了謝遜所居的極北荒島,將他接到靈蛇島。那靈蛇島上囚禁著一對年青夫婦,男的名叫衛璧,女的叫作武青嬰,均是大理的一派武學的傳人。乘著金花婆婆前赴中原,他二人殺了看守之人,逃了出來,在山東遇到危難,幸蒙宋少俠搭救,説起各種前因,宋少俠方知金毛獅王是到了靈蛇島。」那執法長老點頭道︰「{\upstsl{嗯}},原來如此。」張無忌心中,也是這樣説道︰「{\upstsl{嗯}},原來如此。」又想︰「衛璧和武青嬰均非正人,當年他們苦心設下巧計,從我口中騙出義父的所在。但幸而如此,紫衫龍王方能獲知義父的下落。當今之世,説到水性和航海之術,只怕很少有人能勝得過紫衫龍王,若不是由她出馬,茫茫北海之中,又有誰能有此能耐,能找尋到這冰火島?縱令是我爹爹媽媽復生,也未必能彀,可見冥冥之中,自有天意。」

陳友諒又道︰「兄弟和宋少俠乃是生死之交,得悉了這個訊息之後,即行會同季鄭二位八袋長老,率同五名七袋弟子,前赴靈蛇島,意欲生擒謝遜,奪獲屠龍寶刀,獻給幫主。不料魔教大幫人馬,也於此時前赴靈蛇島。兄弟們雖然竭力死戰,終於寡不敵衆,季長老和四名七袋弟子爲幫殉難。靈蛇島上的戰況,請鄭長老向幫主稟報。」只見那肢體殘斷的鄭長老從人叢中站起身來,敘述靈蛇島上明教和丐幫之戰。他不説丐幫衆人圍攻謝遜,却説明教如何人多勢衆,自己一干人如何英勇禦敵,最後説到陳友諒捨身救他性命的仗義之處,更是慷慨激昂,口沫橫飛,説謝遜等爲陳友諒的正氣折服,終於不敢動手。

大殿上群丐只聽得聳然動容,齊聲喝采。那傳功長老説道︰「陳兄弟智勇雙全,而如此義氣,更是難得。」陳友諒躬身道︰「做兄弟的承幫主和長老哥哥們的教誨,本幫大義所在,赴湯𨂻火,在所不辭。這區區小事,倒勞鄭長老的稱讚,做兄弟的好生不安。」群丐見他如此謙遜,毫不居功,更是大讚不已。張無忌在樹上越聽越氣,心想世上卑鄙無恥,竟至如此,明明是賣友求生,却變成了仗義救人,只是他做得天衣無縫,連鄭長老也瞧不出破綻,可説是個大大的奸雄。言念及此,心下忽地黯然︰「這奸人的詭計,當時義父被他騙過,我也被他騙過,只是騙不過趙姑娘。唉,趙姑娘聰明多才,人品却是這般\dash{}」

只見那執法長老站起身來,冷冷的道︰「本幫又有這許多兄弟,爲魔教的魔頭們所害,這層血海深仇,咱們便此罷了不成?」群丐大聲鼓噪︰「咱們非替季長老報仇不可!」

\qyh{}踏平光明頂!掃蕩魔教!」

\qyh{}宰了張無忌,宰了謝遜!」

\qyh{}本幫和魔教勢不兩立,見一個殺一個,見兩個殺一雙!」

\qyh{}幫主快下號令,天下丐幫弟子,齊向魔教攻殺!」

執法長老向史火龍道︰「啓稟幫主︰本幫弟子如此群情憤激。報仇雪恨之舉,如何行事,便待幫主示下。」史火龍皺眉頭道︰「這個嘛,這是本幫的大事,{\upstsl{嗯}},{\upstsl{嗯}},須得從長計議。你叫七袋弟子以下的幫衆,暫且退出,咱們好好的商量商量。」執法長老應道︰「是!」轉身喝道︰「奉幫主號令︰七袋弟子以下退出大殿,在廟外相候。」群丐轟然答應,向史火龍等躬身行禮,頃刻間一齊退出廟門,大殿上只剩下八袋長老以上的諸首腦。

\chapter{冤家路狹}

陳友諒走上一步,躬身説道︰「啓稟幫主,這位宋青書宋兄弟於本幫頗有功績,幫主如若恩准,許他投效本幫,以他的身份地位,日後更可爲本幫建立大功。」宋青書道︰「這個,似乎不\dash{}」他只説了一個「不」字,陳友讓兩道鋭利的目光直射到他臉上。宋青書見到陰狠的神色,登時低下了頭,不再説話。史火龍道︰「這個甚好。宋青書投入我幫,可暫居六袋弟子之位,歸八袋長老陳友諒統率。恪遵本幫幫規,爲本幫出力,有功者賞,有過者罰。」宋青書眼中流露出憤恨之色,但隨即竭力克制,上前向史火龍跪下,説道︰「弟子宋青書,向幫主叩頭。多謝幫主開恩,授予立袋弟子之位。」跟著又拜見了衆長老、衆龍頭。

執法長老説道︰「宋兄弟,你既入本幫,就受本幫幫規約束。日後雖然你做到武當派掌門,也得遵從本幫的號令。這個你知道了麼?」宋青書道︰「是。」執法長老語聲嚴厲,又道︰「本幫與武當派雖然同俠義道,究竟路子不同。武當掌門之位,日後定當落在你的身上,何以你却甘心投入本幫?此事須得説個明白。」宋青書向陳友諒望了一眼,説道︰「陳長老待弟子極有恩義,弟子敬慕他的爲人,甘心追隨驥尾。」陳友諒笑道︰「此處並無外人,説出來也無干係。峨嵋派掌人滅絶師太死後,新任掌門人是個年輕美貌的女子,名叫周芷若。此女和宋兄弟青梅竹馬,素有婚姻之約,那知却給魔教的大魔頭張無忌橫刀奪愛,擕赴海外。宋兄弟氣憤不過,求教於我。做兄弟的拍胸膛擔保,誓必助他奪回周女。」張無忌越聽越怒,暗想︰「此人一派胡言,那有此事?」忍不住便要縱身入殿,直斥其非,但終於強抑怒火,繼續傾聽。

史火龍哈哈一笑,説道︰「自來英雄難過美人関,那也無怪其然。一個是武當掌門,一個是峨嵋掌門,不但門當戸對,而且郎才女貌,本來相配得緊啊。」執法長老又問︰「宋兄弟既然受此委屈,何以不求張三丰眞人和宋大俠作主?」陳友諒道︰「宋兄弟言道︰武當派近來頗有與魔教擕手之意。張三丰和他令尊都不願得罪魔教。眼下中原武林之中,唯有本幫和魔教勢均力敵,足可和群魔相抗。」執法長老點頭道︰「那就是了,只須滅得魔教,宰了張無忌那小子,宋兄弟的心願何愁不償。」

張無忌隱身樹中,回想當日在西域大漠之中、光明頂上,宋青書對待周芷若的神情果是頗爲奇特,此刻一印證,才知他早就對周芷若懷有情意。「但爲了一個女子,因而背叛師門背叛親父,人品豈非太差?何況芷若對我柔情蜜意,一片眞心。宋青書縱得丐幫之助,又怎能逼得她的順從?這位宋大哥在江湖上聲名早著,號稱是武當派後起之秀,怎地一愚至斯?」

他心中自嘆息,只聽陳友諒道︰「啓稟教主︰弟子在大都附近,擒得魔教中一名重要人物,和本幫大業,頗有干係,特請幫主發落。」史火龍喜道︰「快帶上來。」陳友諒雙手拍了三下,説道︰「帶那魔頭上來。」張無忌聽得有本教中重要人物爲陳友諒所擒,心下甚是関懷。只見殿後轉出四名丐幫幫衆,手執兵刃,押著一個雙手反綁之人出來,無忌看那人時,見是個二十來歳的青年,相貌甚熟,記得在蝴蝶谷明教大會之中見過,却不知他的姓名。那人臉上滿是氣憤憤的神色,走過陳友諒身畔時,突然一張口,一口濃痰向他臉上吐去。陳友諒閃身避過,反手一掌,正中那人左頰。他臉頰登時腫了起來。押著他的丐幫弟子在他背後一推,喝道︰「見過幫主,跪下,磕頭。」那人一聲咳嗽,又是一口濃痰,直向史火龍臉上吐去。

那人和史火龍相距既近,這一口痰吐將出去勁力又是極足,史火龍低頭一掠,畢竟没能讓過,拍的一聲,正中他的額頭。陳友諒橫掃一腿,將那人踢倒在地,攔在史火龍的身前,指著那人喝道︰「大膽狂徒,你不要性命了麼?」那人罵道︰「老子既是落在你們手中,本就不想活著回去。」陳友諒這麼一攔,史火龍已將額上的濃痰抹去,在下屬之前不致顯得過分狼狽。他隨即倒退兩步,説道︰「啓稟教主,這小子是魔教中的一流高手,武功似乎尚在四大護教法王之上,咱們倒也不能等閒視之。」張無忌聽了此言,初時頗爲詫異,但立即明白,陳友諒故意誇張那人武功,在替幫主遮醜。可是史火龍身爲丐幫的幫主,竟避不開這口濃痰,太過不合情理,同時受了這等侮辱之後,臉上不現憤怒之色,反而顯得有些驚惶失措,似乎怕人發現什麼重大祕密一般,無忌隱隱覺得其中定是另有别情。

執法長老道︰「陳兄弟,此人是誰?」陳友諒道︰「此人名叫韓林児,乃韓山童之子。」無忌暗暗點頭︰「是了。那時蝴蝶谷大會,他一直跟在他父親身後,没跟我説話,是以一時想不起他的名字來。」執法長老喜道︰「啊,他是韓山童之子,陳兄弟,你這場功勞可更大了。啓稟幫主︰韓山童近年來連敗元兵,大建功名,他手下大將朱元璋、徐達、常遇春等人,都是魔教中的厲害人物。咱們擒獲了這小子作爲人質,不愁韓山童不聽命於本幫。」

韓林児破口罵道︰「做你媽的清秋大夢!我爹爹何等英雄豪傑,豈能受你們這些無恥之徒的要脅?我爹爹只聽張教主一人的號令,你丐幫妄想和我明教爭雄,實在太過不自量力。丐幫幫主是你這種人物,給我張教主提鞋児也不配呢。」陳友諒笑嘻嘻的道︰「韓兄弟,你把貴教張教主説得如此英雄了得,咱們大夥児十分仰慕,很想見見他老人家一面。你就給咱們引見引見吧。」韓林児是個忠厚老實之人,不知陳友諒是在用計騙他吐露眞情,便道︰「張教主擔當大事,就是本教兄弟,也輕易見他老人家不著。他那有空閒來見你?」陳友諒笑道︰「江湖上人人都説,張無忌已被元兵擒去,早在大都斬首正法,連首級都已傳送各處,你還在這児胡吹大氣呢。」韓林児大怒,{\upstsl{呸}}的一聲,喝道︰「放你的狗臭屁,韃子能把我張教主擒去?便是有千軍萬馬團團圍住,我教主也能來去自如。張教主大都是去過的,那是去救出六大門派的武林人物。什麼斬首正法,你少嚼蛆吧!」

陳友諒也不生氣,仍是笑嘻嘻的道︰「可是江湖上都這麼説,我也不能相信啊。爲什麼這半年來只聽得明教中有什麼韓山童、徐壽輝,有什麼朱元璋、劉福通、彭瑩玉和尚,却不聽得有一個張無忌?可見他定是死了無疑?」韓林児滿臉通紅,脹得額頭青筋凸了起來,大聲道︰「我爹爹和徐壽輝他們,都是奉張教主的命令行事,怎能和張教主相比?終有一日張教主從海外歸來,教你們知道他老人家武功的厲害。」陳友諒點頭道︰「原來張教主是去了海外,想他是去迎接他義父金毛獅王去了?」韓林児心中一驚,自知失言,一時張大口嘴,説不出話來。

陳友諒輕描淡冩的道︰「張無忌那人武功是算不差的,但生就一副短命橫死之相,有人給他算命,説他活不過今年年初\dash{}」他説到這裡,庭中那株古柏的一根枝幹突然間輕輕一顫,大殿上諸人都没知覺,張無忌却已聽到那枝幹後竟傳出幾下輕微的喘氣之聲,但那人隨即屏氣凝息,克制住了。無忌心想︰「原來古柏中居然也藏得有人。此人比我先到,這麼許久我都没有察覺,此人武功可也高明得很啊。」當下凝目向那古柏枝葉中瞧去。

在枝葉掩映之間,看到了青衫一角,那人躱得極好,衣衫又和古柏同色,若非無忌眼光特佳,可也眞的不易發見。

只聽韓林児怒道︰「張教主宅心仁厚,上天必然福佑。他青春正當,再活一百年也不希奇。」陳友諒嘆道︰「可是世上人心難測啊。聽説他在海外遭奸人陥害,以致爲朝廷擒殺。其實那也不奇,凡是見過張無忌之人,都知他活不過三八二十四歳那一関\dash{}」他還在滔滔不絶的説下去,忽然古柏上青影一晃,一人竄下地來,口中喝道︰「張無忌在此,是誰在咒我短命橫死!」語聲未歇,身子已竄進殿中。站在殿門口的掌棒長老張開大手,往那人後頸抓去。那人輕輕的一側身,已然避開。但見他方巾青衫,神態瀟然,面瑩如玉,眼澄似水,正是女扮男裝的趙明。

無忌突然見趙明現身,心頭大震,又驚又怒,又愛又喜,禁不住輕輕噫了一聲。但大殿上群丐都是在全神提防趙明,誰也没聽到他這聲驚噫。陳友諒當張無忌幼時,曾在少林寺外見過一面,然相隔已久,無法猜想他長大後相貌如何,後來在靈蛇島見到無忌和趙明,那時他二人黏了鬍子,裝作是巨鯨幫中的人物,因此無忌的本來面目,陳友諒並不相識,至於史火龍等人,那更加没見過了。他們只知明教教主乃是個二十來歳的少年,武功極高,見趙明避開掌棒長老這一抓,身法輕靈,已屬一流高手,心下倒均信了二分。但陳友諒見她相貌太美,年紀太輕,話聲中又頗有嬌媚之音,和江湖上所傳張無忌的形貌頗有不同,喝道︰「張無忌早就死了,那裡又鑽出一個假冒貨來?」

趙明怒道︰「張無忌好端端的活著,爲何你口口聲聲咒他?張無忌洪福齊天,長命百歳,等這児的人個個死絶了,他還要活八十年呢。」無忌聽她説這幾句話時語帶悲音,似乎想到將自己抛在荒島上,良心不免自責,但轉念又想︰「這等陰狠忍心之人,講什麼良心自責?張無忌啊張無忌,你對她戀戀不捨,心中儘生些一廂情願的念頭。」陳友諒道︰「你到底是誰?」趙明道︰「我便是明教之主張無忌。你幹麼捉拿我手下兄弟?快快將他放了,有什麼事,衝著我本人來説便是。」忽聽得旁邉一人冷笑道︰「趙明趙姑娘,旁人不識得你,我宋青書難道不識?旁人不識張無忌,我宋青書豈能不識?啓稟幫主︰這女子乃是汝陽王郡王之女明明郡主。她手下高手極多,須得好好提防。」執法長老撮唇呼哨,喝道︰「掌棒長老,你率領衆兄弟赴廟外迎敵,防備敵人攻將入來。」掌棒長老應聲而出,霎時之間,東南西北,四下裡都是丐幫弟子的呼嘯之聲。

趙明見了這等聲勢,臉上微微變色,雙手一拍,牆頭飄下二人,正是玄冥二老的鹿杖客和鶴筆翁。執法長老喝道︰「拿下了!」便有四名七袋子,分撲鹿鶴二老。玄冥雙老武功奇強,只三招之間,四名七袋弟子,均已受傷。那白鬚白髮的傳功長老站起身來,呼的一掌直向鶴筆翁擊去,這一掌風生虎虎,威猛無儔。無忌在樹上看得明白,那正是「降龍十八掌」中的一招「見龍在田」當年謝遜在冰火島上曾約略跟他説過這一招的模樣,只是不明掌法中的精義,使出來時似是而非,想不到這位老丐居然學到了九指神丐洪七公的這招絶技。鶴筆翁識得厲害,全身功力運轉,一招「玄冥神掌」還擊了過去。砰的一聲巨響,雙掌相對,那降龍十八掌是純剛之學,玄冥神掌却是至陰至柔,兩人在自己的看家本領上浸潤數十年,均已練到了九成的功力,以至剛擊至柔,這一對掌,竟是不分上下。傳功長老只覺一股冰冷徹骨的寒氣,自掌心沿著手臂迅速上行。

鶴筆翁跟他一掌相對,竟也隱隱覺得胸口氣血翻湧,心下暗自詫異,向傳功長老瞪目而視,只見他臉上有痛楚之色,雙目如血,正自運功和掌上傳來的陰寒之氣相拒。鶴筆翁心中一喜︰「我還道今日遇上了勁敵,原來你畢竟比我還差半籌。」不等他運功驅除陰毒,上前一步,輕飄飄的又是一掌拍出。他這玄冥神掌掌力所至,籠罩四方,對方除了硬接外,無可閃避,傳功長老無奈,只得又是一招降龍十八掌拍出。

兩人掌力雖有剛柔之分,功力却是難分軒輊。只是傳功長老的掌法承洪七公一脈相傳,純是光明正大的武學,那玄冥神掌之中却另含一股陰毒寒氣。傳功長老和鶴筆翁對掌時並不吃虧,但每對一掌,便須運功驅除寒毒,不但心有二用,而且損耗功力甚巨,對到三掌之後,已是相形見絀。那邉廂鹿杖客使動鹿角杖,雙戰執法長老和掌砵龍頭二人,一時難分高下。掌棒龍頭見傳功長老臉紅如血,一步步的後退,不禁暗自駭異,心想傳功長老學到了降龍十八掌中的十二掌,功力蓋世,乃是本幫的第一高手,怎地反爲不敵這個老児?眼見他對到第七掌時,喘息聲響,白鬚飄動,已微現狼狽之態,雖知傳功長老對敵之時決計不喜旁人相助,但到此地步,與其任由他喪生敵手,折了一世英名,還不如落個以二敵一的不美之名,當下舉起竹棒,一棒向鶴筆翁脚下橫掃過去。他這棒法雖不及「打狗棒法」的神妙,但丐幫弟子能學到棒法者,均已是極強的高手。掌棒龍頭更是其中頂尖児的人物。他一加入戰團,傳功長老便有喘息餘裕,勉強將鶴筆翁敵住。

趙明當玄冥二老到來之時,嬌軀一晃,便欲退走,那知陳友諒抽出長劍,將她擋住。趙明在萬法寺中學得六大門派武功的精髓,反手刷刷刷三劍,一招華山劍法,一招崑崙劍法,第三招已是崆峒派的劍招絶學。待得第四招使出,已是峨嵋派的「降魔大九式」。陳友諒一驚之下,竟是招架不來。趙明長劍圏轉,直刺他的心上,眼前一劍便可洞穿他的胸膛,忽地{\upstsl{噹}}的一聲響,左首一劍橫伸而來,將她這一劍格開了,出招的却是宋青書。

殿上衆人的戰鬥,張無忌隱身在古松之上,看得招招清楚。但見宋青書施展武當劍法,又穩又狠,確已得了宋遠橋的眞傳。陳友諒是少林子弟,從旁夾攻相助。趙明所習絶招雖多,但以一敵二,對手又是少林和武當門下的高弟,時候稍長,已是遮攔多而進攻少。無忌暗暗心焦,心下又感奇怪︰「她爲何只使一柄尋常的長劍?若將倚天劍取將出來,宋陳二人手中兵刃立斷,登時便可闖出重圍。」但見她衣衫單薄,身形苗條,腰間顯然並未藏著倚天劍。無忌焦急了一會,不禁又自責起來︰「張無忌啊張無忌,這小妖女是害死你表妹的兇手,若是被宋青書殺了,正好替表妹報了大仇,何以你反而爲她擔憂?可見你還是戀戀的不捨於她。這不但對不起表妹,可也對不起義父和芷若啊。」

又鬥一陣,丐幫又有幾名高手加入,趙明手下却無旁人來援。鹿杖客見情勢不佳,叫道︰「郡主娘娘,鶴兄弟,久戰不利,咱們先退到庭院之中,乘機走吧。」趙明道︰「很好。這個姓陳的毀謗張公子,説他橫死短命,我氣他不過,你們退走時狠狠的幹他一下子。」玄冥二老齊聲道︰「遵命。娘娘先退便是,這小子交在我們身上。」趙明又道︰「那個韓林児對張公子很是忠心,你們設法救他出來。」鹿杖客道︰「娘娘請先行一步,救人之事,咱兄弟倆俟機行事便了。」他二人在強敵圍攻之中,商議退却救人,竟將對方視若無物。

大殿中鬥得甚緊,丐幫幫主史火龍站在殿角,始終不作一聲。傳功、執法二長老聽得趙明和玄冥二老對答之言,連下號令,命屬下攔截。突然之間,鹿杖客和鶴筆翁撇下對手,猛向史火龍衝了過去,這一下變故來得奇快,史火龍武功再高,只怕也是難擋玄冥二老聯手的這一擊。那知陳友諒當趙明和二老講話之時,料到二老定然以進爲退,要施圍魏救趙之計,已先行繞到史火龍身旁。玄冥二老掌力未到,陳友諒已在史火龍肩頭一推,將他推到了彌勒佛像之後。玄冥二老掌力擊出,{\upstsl{噗}}的一聲輕響,佛像上泥屑紛飛,一尊大的佛像搖搖欲墜。鶴筆翁搶上一步,再補上一掌,一尊兩丈來高的佛像半空中倒將下來。

群丐齊聲驚呼,躍在兩旁相避。趙明乘著這陣大亂,已躍到庭院之中。宋青書和掌棒龍頭劍棒齊施,追擊而至,驀地裡廟門中三條桿棒捲到,往趙明脚下掃去。這三條桿棒使的都是絆繞功夫,趙明既要擋架宋青書的長劍和掌棒龍頭的竹棒,又要閃避這三條桿棒,避開了兩條,却避不開第三條,只覺左脛上一痛,已被一棒擊中,站立不定,人已摔倒。宋青書倒轉劍把,便往趙明後腦{\upstsl{砸}}去,要將她一下{\upstsl{砸}}暈,生擒活捉。

眼見那劍柄距趙明後腦已不到半尺,忽然掌棒龍頭手中的竹棒伸過來在劍柄上一撩,將宋青書的長劍盪開了,但見一條人影飛起,躍出了牆外。宋青書轉過身來,問掌棒龍頭道︰「幹麼放她逃走?」掌棒龍頭道︰「你撩我竹棒幹麼?」宋青書道︰「是你用棒盪開我劍柄的,還説\dash{}」掌棒龍頭道︰「多爭無益,快追!」兩人一齊躍出牆去,只見牆角邉躺著一名七袋弟子,摔得腿骨折斷,爬不起來。掌棒龍頭問道︰「那妖女逃向何方去了?」在牆外守衛的七八名丐幫弟子六聲道︰「没有啊,没見到有人。」掌棒龍頭怒道︰「適才明明有人從這裡躍將出來,你們眼睛都瞎了麼?」一名六袋弟子伸手扶起那跌斷腿骨的七袋弟子,説道︰「適纔便是這位大哥躍牆而去,没再見到有第二個人。」掌棒龍頭搔了搔頭皮,問那七袋弟子道︰「你幹麼躍牆而出?」那七袋弟子哼哼喞喞的道︰「我\dash{}我是被人抓著摔出來的。那妖女好怪異的手法。」

當棒龍頭轉頭對著宋青書滿臉怒色,喝道︰「適纔你用劍柄撩我竹棒,是何用意?你纔入本幫,便來幹吃裡扒外這一套了?」宋青書又驚又怒,説道︰「弟子正要用劍柄{\upstsl{砸}}那妖女,龍頭大哥用棒擋開了我這一{\upstsl{砸}},纔被那妖女逃走。」掌棒龍頭怒道︰「豈有此理?我擋開你的劍柄幹什麼?我在本幫數十年,積功升到掌棒龍頭的高位,難道反來相助外人?我再問你,你好好的爲何不用劍尖刺她,却要倒轉劍柄,假意{\upstsl{砸}}打?哼哼,我老眼未花,須瞞不過去。」

宋青書在武當派中雖是第三輩的少年弟子,但武當門下都知他是未來的掌門人,雖是兪蓮舟,張松溪師叔,對他亦極客氣,從無半句重語,不料在陳友諒挾制之下,無奈投入丐幫,第一日便受掌棒龍頭的惡氣。他是高傲慣了的人,雖知掌棒龍頭在幫中身份地位,比自己這新入幫的要高得多,但此事明明曲在彼方,不肯便此忍氣吞聲,當下反唇相稽,説道︰「『吃裡扒外』四字,可不是胡亂説的,龍頭大哥以此相責,須有人證。小弟適纔這一劍柄{\upstsl{砸}}下去,明明是你用竹棒擋開的,青天白日之下,未必就無旁人目睹。」掌棒龍頭聽他言中之意,反是冤枉自己吃裡扒外,放走了趙明,這口氣如何忍得下去?他本就性如烈火,大聲喝道︰「你這小子不敬長者,可是仗著武當派的聲勢來頭麼?」説著刷的一棒,便往宋青書頭上{\upstsl{砸}}了下去,他是暴怒之下,這一棒勁力極是剛猛。

宋青書胸中一口氣忍不下去,舉起長劍便是一擋。那竹棒之中不知藏有什麼古怪堅硬物事,長劍這一擋竟然削它不斷,宋青書只感虎口隱隱作痛,知道這個掌棒龍頭功力深厚,内力較已爲強。那掌棒龍頭被宋青書這麼一格,小臂微感酸麻,倒也吃了個小小的啞巴虧,喝道︰「姓宋的,你膽敢犯上作亂,是敵人派至本幫來臥底的麼?」説著第二棒又擊了下去。

廟門中突然搶出一人,伸劍在竹棒上一搭,將這一招盪了開去,説道︰「龍頭大哥,請莫生氣。」此人正是八袋長老陳友諒。掌棒龍頭氣呼呼的道︰「陳兄弟,你倒來評評這個理看!」陳友諒道︰「趙明那小妖女呢?」當棒龍頭指著宋青書道︰「是他放了。」宋青書忙道︰「不,是龍頭大哥放的。」

兩人正自爭辯不已,玄冥二老已從廟中呼嘯而出,四下一看,不見趙明人影,知道郡主娘娘已然脱身。兩人心中大定,猛地裡哈哈一聲長笑,四掌齊出,登時有四名丐幫弟子中掌倒地,待得傳功長老、執法長老、掌棒龍頭等人追出時,鹿杖客和鶴筆翁二人早自去得遠了。只聽得兩人陰陰的長笑之聲,已在里許之外。掌棒長老暴跳如雷,喝道︰「大夥児追啊!」陳友諒道︰「不,龍頭大哥,提防敵人暗中設下埋伏。」掌棒龍頭登時醒悟,心中一驚︰「我怎地如此胡塗?單是這兩個老児,已將丐幫鬧得天翻地覆,死傷多人。我孤身追去,不栽個大觔斗才怪。」

心下對陳友諒這一阻止,暗生感激之意,對宋青書的怒氣也就平歇了些。須知玄冥二老功力驚人,適纔大鬧彌勒廟,已使丐幫上下群爲膽寒,雖然單是以他二人,久鬥之下,終究會寡不敵衆,但丐幫不知對方虛實,料想對方或有大批高手,隱伏在後。

當下執法長老査點死傷的弟子,竟有十一人死在玄冥二老手下,另有七身受重傷,至於被彌勒神像倒下來時壓傷的,又有八九人。執法長老分派人衆救護傷者,命掌砵龍頭帶領淨衣派下弟子,在神廟前後搜索敵蹤,若是發見有異,立即傳訊示警。

不提丐幫一番紛擾,且説趙明却是到了何處。原來張無忌見她身受掌棒龍頭及宋青書的夾攻,終於被一名丐幫弟子使桿棒絆倒,宋青書倒轉長劍,便要往她後腦擊去。這一擊可輕可重,輕則令她昏暈,下手稍重,却是立時取了她的性命。張無忌當下更不思索,從古松上縱身而下,使出挪移乾坤的神功,在掌棒龍頭身後推動他手中竹棒,掠過去盪開了宋青書的長劍。張無忌所習的挪移乾坤心法本已神妙無方,這幾個月來在荒島上日長無事,再研習小昭所譯的「聖火令祕訣」,兩者一相結合,比之波斯三使的詭異武功,更是高明了十倍。此刻突然使將出來,雖以掌棒龍頭和宋青書這等高手,竟然也是無法察覺,掌棒龍頭只道是宋青書格開了他的竹棒,宋青書却明明見到掌棒龍頭伸棒過來盪開他的長劍。張無忌乘著他二人一驚的一瞬之間,左手反過來抓住一名七袋弟子,擲出牆外。掌棒龍頭和宋青書更無懷疑,見到一個人影越牆而出,認定是趙明逃了出去,跟著追出,張無忌却已抱起了趙明,如一溜輕煙般飛躍而上了大殿的殿頂。

此刻他輕身功夫實已入了化境,手中抱著一人,縱躍之際,仍是捷如飛鳥。此時乃是午後,青天白日之下,萬物無所遁形,但群丐一窩蜂的跟著掌棒龍頭和宋青書追出廟門,雖有許多人眼睛一花,似乎有什麼東西在頭頂越去,然大殿中彌勒神像倒下後塵沙飛揚,煙霧彌漫,群丐紛紛湧出,廟門前後正自亂成一團。武功高的,在圍攻玄冥二老和趙明,功力較弱的,但求自保,是以竟無一人察覺。

趙明危急中由人救出,身子被抱在一雙堅強有力的臂膀之中,猶似騰雲駕霧般上了廟頂,轉頭一望,耀眼陽光之下,只見那人濃眉俊目,正是張無忌。她幾乎不相信自己的眼睛,叫道︰「是你!」無忌伸手過來,按住了她的嘴巴,四下裡一瞥之間,但見廟左廟右、廟前廟後,都擁滿了丐幫弟子,若要救了趙明就此脱身,原亦不難,但明知丐幫密謀對付明教,武當派中的宋師哥又入了丐幫,不將此事情打聽明白,就此脱身而去,未免可惜。他又見到宋青書和掌棒龍頭爭吵,掌棒龍頭已是目露兇光,丐幫中頗有奸險之輩,説不定宋青書竟會遭了他們毒手。何況韓林児忠心耿耿,務須救出。但見大殿中塵沙飛揚,心想索性涉險進入殿中,覓地躱藏。

他身子向前一竄,從屋簷旁撲了下去,雙足鉤住屋簷,跟著兩腿一縮,人已到了左側一座佛像之後。只見史火龍、傳功長老、執法長老等均已追出廟門,殿中只剩下幾名被佛像壓傷的丐幫弟子,躺在地下呻吟,韓林児却不知已被帶往何處。無忌遊目四顧,一時找不到妥善的躱藏之所。趙明伸指向著一隻大皮鼓一指。那大鼓高高安在一隻大木架上,離地一丈有餘,和右側的巨鐘相對。無忌登時省悟,貼牆繞進,走到皮鼓之後,身子縱起,右手食指在鼓上一劃,嗤的一聲輕響,蒙在鼓上的牛皮已裂開了一條大縫。無忌左足搭在木架的橫撐上,食指又是筆直的一劃,兩劃交叉成一十字。他抱著趙明,輕輕巧巧的從這十字縫中鑽了進去。

這巨鼓製成已久,滿腹塵泥,無忌在灰塵和穢氣之中,却聞到趙明身上發出的陣陣幽香。這皮鼓雖大,但兩人躱在其中,却也轉動不得。趙明靠在無忌身上,嬌喘細細,心情極是激動。無忌心中愛恨交迸,有滿腹怨言要向她責問,苦於置身處却非説話之所,但覺趙明的身子靠在他懷中,將頭偎依在他左肩,根根柔絲,擦到他的臉上。無忌心下一驚︰「我出手相救,已是不該,如何再可和她親暱如此?」伸手用力將她的頭一推,不許她將頭靠在自己肩上。趙明極是生氣,手肘往他胸口撞了過來。無忌借力打力,將她撞來的勁道反彈了轉去,趙明吃痛,幾欲呼叫,無忌早已料到,伸手又將她嘴按住了。

只聽得執法長老的聲音在下面響起︰「啓稟幫主︰敵人已逃走無蹤,屬下不力,未能擒交幫主發落,請幫主降罪。」史火龍道︰「罷了!敵人武功甚高,大家都是親見。執法長老不必自謙。」執法長老道︰「多謝幫主。」接著便是掌棒龍頭指告宋青書放走敵人,宋青書據理而辯,雙方各執一辭,登時殿中情勢極是緊張。史火龍沉吟半晌,道︰「陳兄弟,你瞧見當時實情如何?」陳友諒道︰「啓稟幫主︰掌棒龍頭是本幫元老,所言自無虛假。依兄弟愚見,這姓趙妖女武功怪異,想是她借力打力,以龍頭大哥之棒,盪開了宋兄弟之劍。混亂中雙方不察,致起誤會。」張無忌心下暗讚︰「這陳友諒果是人傑,他不見當時情景,却猜了個八九不離十。」

只聽史火龍道︰「此話極爲有理。兩位兄弟,大家都是爲本幫效力,不必爲此小事傷了兩家和氣。」掌棒龍頭氣憤憤的道︰「就算他\dash{}」陳友諒不待他説完,便即插口道︰「宋兄弟,龍頭大哥德高望重,就是責備你錯了,也當誠心受教。你快向龍頭大哥陪禮。」宋青書無奈,只得上前施了一禮,説道︰「龍頭大哥,適纔小弟多有得罪,還請原恕則個。」那掌棒龍頭滿腔怒氣,竟是發作不出,只是哼了一聲,道︰「罷了!」

\chapter{似嗔似怨}

陳友諒的話聽來似乎是委屈了宋青書,其實許多話都在派掌棒龍頭的不是,他説趙明「以龍頭大哥之棒,盪開了宋兄弟之劍」,又説「龍頭大哥德高望重,就算責備錯你了,也當誠心受教」,丐幫中諸長老都聽了出來。但陳友諒近年來是幫主跟前一個大大的紅人,史火龍對他言聽計從,衆人也就没有什麼話説。

只聽史火龍道︰「陳兄弟,適纔前來搗亂的妖女,乃是汝陽王的親生愛女。魔教是朝廷的對頭,怎麼咱們説到魔教的小魔頭張無忌,這位郡主娘娘反而挺身出來給他出頭?」陳友諒沉吟不答,掌砵龍頭説道︰「我見那郡主娘娘泪光瑩瑩,臉上神色十分氣憤。陳兄弟咒是魔教教主,那郡主娘娘却像是聽到旁人咒他父兄一般,實是令人大惑不解。」宋青書道︰「啓稟幫主︰此中情由我倒知道。」史火龍道︰「宋兄弟請説。」宋青書道︰「魔教雖是事事和朝廷作對,但這位明明郡主對張無忌痴心相戀,恨不得嫁了他纔好,什麼事都出力護得他。」

丐幫群豪聽了此言,都是「啊」的一聲,人人頗出意外。張無忌在巨鼓中聽得清楚,心中也是怦怦亂跳,腦中只是響著︰「那是眞的麼?那是眞的麼?」趙明轉過頭來,雙目瞪視著他。鼓中雖然陰黑,但張無忌目光鋭敏,藉著些微光,已見到她眼中流露出柔情無限。無忌胸口一熱,抱著她的雙臂緊了一緊,便想要往她櫻唇上吻去,突然間想起殷離慘死之狀,一番愛戀登時化作仇恨,右手抓著她的手臂,使勁的一捏。以無忌此時功力,這一捏雖非運出全部的内力,但趙明已是抵受不住,只覺眼前一黑,痛得幾欲暈去,忍不住便要學殷離那樣一句罵了出來︰「你這狠心短命的小鬼。」總算她自制力強,口中没有出聲,但泪水却簌簌的流了出來。這泪水一滴滴的都流在無忌手背之上,又沿著他手臂,流上了他的衣襟。無忌心下剛硬,對她毫不理睬。

但聽得陳友諒問道︰「你怎知道?當眞有這等怪事麼?」宋青書恨恨的道︰「張無忌這小子相貌平平,並無半點英俊瀟灑之處,只是學到了魔教的邪術,許多青年女子便都墮入他的殼中而不自覺。」執法長老點頭道︰「不錯,魔教中的淫邪之徒確有這種採花的法門。峨嵋派的女弟子紀曉芙,不就因受了魔教楊逍的邪術,因而鬧得身敗名裂麼?張無忌的父親張翠山,也是被白眉鷹王之女的妖法所困。那明明郡主,想必是中了這小魔頭的採花邪法,因而失身於他,木已成舟,生米煮成熟飯,便自甘墮落而不能自拔了。」丐幫群豪一齊點頭稱是。傳功長老義憤填膺,説道︰「這等江湖上的敗類,人人得而誅之,否則天下婦女的名節,不知更將有多少喪在這小淫賊的手中。」張無忌只氣得混身發顫,他迄今仍是童子之身,但自峨嵋派絶滅師太起,不知有多少人罵他是淫賊,當眞是有冤無處訴了。至於説趙明失身於已,木已成舟云云,更不知從何説起,想到此處,突然一驚︰「趙姑娘和我相擁相抱,躱在此處,萬萬不能讓他們發覺,否則更是證實了這不白之誣。」

只聽那傳功長老又道︰「峨嵋派周芷若姑娘既落在這小淫賊手中,想必貞潔難保。宋兄弟,此事你也不必放在心上,咱們必然助你奪回愛妻,決不能讓紀曉芙之事,重見於今日。」執法長老道︰「大哥此言甚是。武當派當年庇護不了殷利亨,今日自也庇護不了宋青書。宋兄弟投入本幫,咱們若不給他出這口氣,不助他完成這番心願,他好好的武當派未來掌門,何必到本幫來當一名六袋弟子?」丐幫群豪大聲鼓噪,都説誓當宰了張無忌這淫賊,要助宋青書奪回妻子。趙明將嘴湊到無忌耳邉,輕輕説道︰「你這該死的小淫賊!」

這一句話似嗔似怒,如訴如慕,無忌只聽得心中一蕩,霎時間意亂情迷,極是煩惱︰「倘若她並非如此奸詐陰毒,害死我的表妹,我定當一生和她長相厮守,什麼也不顧得了。」只聽得巨鼓之外,宋青書含含糊糊的向群丐道謝。執法長老爲人甚是精細,又問︰「那淫賊如何迷姦明明郡主,你可知道麼?」宋青書道︰「這中間的細節,外人是無法知悉的了。兄弟只知當時明明郡主率領朝廷武士,來武當山擒拿我太師父,一見那淫賊之面,便即乖乖退去,武當派一場大禍,登時風平浪靜。我三師叔兪岱岩二十餘年前被人折斷肢骨,也是明明郡主贈藥於那淫賊,因而接續了斷骨的。」執法長老道︰「這就是了,想武當派自來是朝廷眼中之釘,那明明郡主若非迷戀淫賊,忘了本性,決不致反而贈藥助敵。如此説來,那淫賊雖然人品不端,對於太師父和衆師叔伯倒還頗有香火之情。」宋青書道︰「{\upstsl{嗯}},我想他不致於全然忘本。」陳友諒道︰「啓稟幫主︰兄弟聽了宋兄弟之言,倒有一計在此,可制得那小淫賊服服貼貼,令魔教上下,盡數聽令於本幫。」史火龍喜道︰「陳兄弟竟然有此妙計,請快快説來。」陳友諒道︰「此間耳目衆多,雖然都是自家兄弟,仍恐洩漏了機密。」

大殿中語聲稍停,只聽得脚步聲響,有十餘人走出殿去,想是只剩下丐幫中最高的幾位首領。只聽陳友諒道︰「此事千萬不能透露半點風聲,宋兄弟,兩位龍頭大哥,咱們前後搜査一遍,且看是否有人偸聽。」只聽得颼颼兩聲,掌棒龍頭和掌砵龍頭已上了屋頂,陳友諒和宋青書在殿前殿後仔細搜査,連未倒的神像之後,帷幕之旁,匾額之内,到處都察看過了。張無忌暗服趙明心思機敏,大殿中除了這巨鼓之外,確無其他更好的藏身處所。

四個人査察已畢,重回殿中。陳友諒低聲道︰「這事還須著落在宋兄弟的身上。」宋青書奇道︰「我?」陳友諒道︰「不錯,掌砵龍頭大哥,請你配幾份『五毒失心散』交由宋兄弟帶上武當山去,暗中下在張眞人和武當諸俠的飲食之中。咱們在山下接應,得手之後,將張眞人和武當諸俠一鼓擒來,那時以此要脅,何愁張無忌這小賊不聽命於本幫?」史火龍首先便道︰「妙計,妙計!」執法長老也道︰「此計不錯。本幫五毒失心散毒性厲害,要在張無忌的飲食之中下毒,他魔教防範周密,難得其便。宋兄弟是武當子弟,所謂家賊難防,當眞是神不知,鬼不覺,手到擒來。」

宋青書躊躇道︰「這個\dash{}這個\dash{}要兄弟毒害家父,那是萬萬不可。」陳友諒道︰「這五毒失心散是本幫的靈藥,不過令人暫時神智迷糊,並不傷身。令尊宋大俠仁俠重義,咱們素來是十分敬仰的,決不致傷他老人家一根毫毛。」宋青書仍是不肯答應,説道︰「兄弟投效本幫,太師父和家父知道後已必重責,這等不孝犯上之事,兄弟萬萬不敢應承。」陳友諒道︰「兄弟,你這事可想不通了。自來成大事者,不拘小節,古人大義滅親,歷朝均有,何況咱們的宗旨乃在對付魔教,擒拿武當諸俠,只不過是箝制張無忌那小淫賊的一種方策而已。」宋青書道︰「兄弟若是做了此事,在江湖上被萬人唾罵,有何面目更立於天地之間?」陳友諒道︰「適纔我爲什麼要八袋長老他們都退出殿去?爲何要上下前後仔細搜査?那就是怕此洩漏出去啊。宋兄弟,你下藥之後,自己也可假作昏迷,咱們將你縛住,和你太師父、尊大人,以及衆師師叔関在一起,誰也不會疑心於你,除了咱們此間七人之外,世上更有何人得知?咱們只有佩服你是個能彀擔當大事的好漢子,誰會笑你?」

宋青書沉吟半晌,囁嚅道︰「幫主和陳大哥有命,小弟原不敢辭。再説小弟新投本幫,自當乘機立功,縱然赴湯𨂻火,也當盡心竭力。只是人生於世,孝義爲本,要小弟算計家父,那是萬萬不可。」丐幫中向來傳統,於「孝」之一字,原是極爲尊祟,群丐聽他如此説,倒是不便如何相強。陳友諒忽地冷笑一聲,説道︰「以下犯上,那是我輩武林中人的大忌,不用宋兄弟説,這個我也明白。但不知莫聲谷莫七俠和宋兄弟如何稱呼?是他輩份高,還是你輩份高?」宋青書不語,隔了良久,忽道︰「好,既是如此,小弟應命就是。但各位須得應承,既不能損傷家父半分,也不能絲毫折辱於他。否則小弟寧可身敗名裂,也不能幹此不孝的勾當。」史十龍、陳友諒等無不大喜,齊聲説道︰「這個咱們自是應承得。宋兄弟跟咱們兄弟相稱,宋大俠便是咱們的尊長,宋兄弟便是不説,咱們也當對他老人家盡子侄之禮。」

張無忌心下起疑︰「宋師哥一直不肯答允,何以陳友諒一提莫七叔,宋師哥便不敢再行推辭,此中定有蹺蹊。看來只有當面問過莫七叔,方知端詳。」只聽執法長老和陳友諒等低聲商議,當張三丰、宋遠橋等人中毒之後,丐幫群豪怎生上山接應。每逢陳友諒如何説,史火龍總是道︰「甚好,甚好!」當砵龍頭道︰「此時方當隆冬,五毒均蟄伏土下,小弟須得走長白山脚上挖掘,多則一月,少則二十日,當可合成五毒失心散。從冰雪之下掘出來的毒物,毒性不顯,服食時不易知覺,對付這等第一流的高手,倒是這等毒物最好。」執法長老道︰「陳兄弟、宋兄弟兩位,陪同掌砵龍頭赴長白山配藥,咱們先行南下。一個月後,在老河口聚齊。今日是十二月初八,準定年後正月初八相會便了。」又道︰「那韓林児落在咱們手中,甚是有用,請掌棒龍頭加意看守,以防魔教截奪。咱們分批而行,免入敵人的耳目。」當下衆人紛紛向幫主告辭,掌砵龍頭和陳友諒、宋青書三人先向北行。片刻之間,彌勒廟前前後後的丐幫人衆散了個乾淨。

張無忌聽得群丐去遠,廟中再無半點聲響,於是從鼓中躍了出來。趙明跟著躍出,理一理身上衣衫,似喜似嗔的橫了無忌一眼。無忌怒道︰「哼,虧你還有臉來見我?」趙明俏臉児一沉,道︰「怎麼啦?我什麼地方得罪大教主啦?」張無忌臉上如罩嚴霜,喝道︰「你要盜那倚天劍和屠龍刀,我不怪你!你將我抛在荒島之上,我也不怪你!可是殷姑娘已然身受重傷,你何以還要再下毒手!這等狠毒的女子,當眞是天下少見。」説到此處,怒火上沖,跨上一步,左右開弓,便是四個耳光。趙明欲待閃避,但在無忌掌力籠罩之下,如何閃避得了?拍拍拍拍四聲響過,她兩邉臉頰登時紅腫起來。

趙明又痛又怒,珠泪滾滾而下,哽咽道︰「你説我盜了倚天劍和屠龍刀,是誰見來?誰説我對殷姑娘下了毒手,你叫她來跟我對質。」張無忌愈加憤怒,大聲道︰「好!我叫你到陰間去跟她對質。」左手一圏,右手一扣,已叉住了她的粉頸,雙手使勁,趙明呼吸不得,右手一指戳向無忌胸口。但無忌有九陽神功護體,這一指戳到,如中敗絮,指上勁力消失得無影無蹤。霎時之間,趙明滿臉紫脹,暈了過去。無忌記著殷離之仇,本待將她扼死,但見了她這等神情,急地心軟,放鬆了雙手,趙明往後便倒。{\upstsl{咚}}的一聲,後腦撞在大殿的青石板上。

過了好一陣,趙明纔悠悠醒轉,見無忌雙目凝望著她,滿臉是擔心的神色,見她睜眼,這纔吁了口氣。趙明問道︰「你説殷姑娘過世了麼?」無忌怒氣又生,喝道︰「被你這麼斬了十七八劍,她\dash{}她難道還活得成麼?」

趙明顫聲道︰「誰\dash{}誰説我斬了她十七八劍?是周姑娘説的,是不是?」無忌道︰「周姑娘決不在背後説旁人壞話,她没親見,不會誣陥於你。」趙明道︰「那麼是殷姑娘自己説的了?」無忌大聲道︰「殷姑娘早不能言語了。那荒島之上,只有咱們五人,難道是義父斬的?是我斬的?是殷姑娘自己斬的?哼,我知道你的心思,你是怕我跟我表妹結爲夫婦,是以下此毒手。我跟你説,她死也好,活也好,我都當她是我妻子。」趙明低頭不語,沉思半晌,又問︰「你怎地回到中原來啦?」無忌冷笑道︰「那倒多蒙你的好心了,你派水師到島上來迎接在下,幸好我義父不似我這等老實無用,咱們纔不墮入你的奸計之中。你派了炮船候在海邉,要開炮轟沉咱們座船,這番心計却是白用了。」

趙明撫著紅腫炙熱的面頰,怔怔的瞧著無忌,眼光中忽然露出憐愛的神色,長長的嘆了口氣。無忌生怕自己心動,屈服於她美色和柔情的引誘之下,將頭轉了開去,突然一登足説道︰「我曾立誓替表妺報仇,算我懦弱無用,今日下不了手。你作惡多端,終須有日再撞在我的手裡!」説著大踏步走出廟門。

他走出十餘丈,聽得趙明追了出來,叫道︰「張無忌,你往那裡去?」無忌道︰「這跟你有什麼相干?」趙明道︰「我有話要問謝大俠和周姑娘,請你帶我去見見他二人。」無忌道︰「我義父下手不容情,你還不是去送死?」趙明冷笑道︰「你義父心狠手辣,可不似你這等胡塗。再説,謝大俠殺了我,你是報了表妹之仇,不是了結一件心事?」無忌道︰「我胡塗什麼?我不願你去見我義父。」趙明微笑道︰「張無忌,你這胡塗小子,你心中實在捨不得我,不肯讓我被謝大俠殺了,是也不是?」無忌被她説中了心事,臉上一紅,喝道︰「你别囉唆!我叫你多行不義必自斃。你最好離得咱們遠遠的,别叫我管不住自己,送了你的性命。」趙明一步步走近身去,説道︰「我這幾句話非問清楚謝大俠和周姑娘不可,我不敢在背後説旁人壞話,當面却須説個明白。」無忌起了好奇之心,道︰「你有什麼話問他們?」趙明道︰「待會你自然知道。我不怕干冒危險,你反而害怕麼?」

無忌略一遲疑,道︰「這是你自己要去的,我義父若下毒手,我須救不得你。」趙明道︰「不用你替我擔心。」無忌怒道︰「我替你擔心?哼!我巴不得你死了纔好。」趙明笑道︰「那你快動手啊。」無忌{\upstsl{呸}}了一聲,不去理她,快步向鎭甸走去。趙明跟在後面。兩人將到鎭甸,無忌停步轉身,説道︰「趙姑娘,我曾答應過你,替你做三件事。第一件是替你找屠龍刀,這件事算是做到了。還有兩件未做,你若跟我去見義父,那是非死不可,你還是走吧,待我替你幹了那兩件事,再去會我義父不遲。」趙明嫣然一笑,説道︰「你是在給自己找個不殺我的原因,我知道你心中是捨不得我。」無忌怒道︰「就算是我不忍心,那又怎樣?」趙明道︰「我很喜歡啊。我一直不知道,你是不是眞心待我,現下可知道了。」無忌嘆了口氣,道︰「趙姑娘,我求求你,你自個児去吧。」趙明搖頭道︰「我非見謝大俠不可。」

無忌拗她不過,只得舉步走向客店,到了謝遜房門之外,在門上敲了兩下,叫道︰「義父!」口中叫門,身子擋在趙明之前。那知叫了兩聲,房中無人回答。無忌一推門,房門却関著,他心下起疑。暗想以義父耳音之靈,自己到了門邉,他便在睡夢之中,也必驚醒,若説出外,何以房門却又閂了?當下手上微微一使勁,拍的一聲門閂繃斷,房門開處,只見謝遜果不在内。

但見一扇窗子却開了一半,想是謝遜從窗中去了。無忌走到周芷若房外,叫了兩聲︰「芷若!」不聽答應,推門進去時,見周芷若也不在内,但炕上衣包,仍是端端正正的放著。無忌道︰「莫非是遇上了敵人?」叫店伴來一問,那店伴説道,不見他二人出去,也没聽到甚麼爭吵打架的聲音。無忌心下稍慰︰「料想是他二人聽到甚麼響動,追尋敵蹤去了。」又想謝遜雙目雖盲,然武功之強,當世罕有其匹,何況有一個精細謹愼的周芷若隨行,當不致出甚麼岔子。他從謝遜窗中躍出去,四下察看一遍,並無異狀,於是又回到房中。

趙明道︰「你見謝大俠不在,爲什麼反而欣慰?」無忌道︰「你又來胡説八道,我幾時欣慰了?」趙明微笑道︰「難道我不會瞧你的臉色麼?你一推開房門,怔了一怔,繃起的臉皮便放鬆了。」無忌不去睬她,自行斜倚在謝遜的炕上。趙明笑吟吟的坐在椅中,説道︰「我知道你是怕謝大俠殺我,幸好他不在,倒免得你爲難。我知道你心中是不捨得我?」無忌怒道︰「不捨得你便怎樣?」趙明笑道︰「我很開心啊。」無忌恨恨的道︰「那你爲什麼幾次三番的來害我?你心中倒捨得我?」趙明突然間粉臉飛紅,輕輕的道︰「不錯,從前我確是想殺了你,但自從綠柳莊上一會之後,我若再起害你之心,我明明特穆爾身遭天誅地滅,萬劫不得超生。」無忌聽她起誓的言語甚是鄭重,便道︰「那爲什麼你爲了一刀一劍,竟將我抛在荒島之上?」趙明道︰「你既認定如此,我是百口難辯,只有等謝大俠、周姑娘回來,咱們四人對質明白。」無忌道︰「你滿口花言巧語,只騙得我一人,須騙不得我義父和周姑娘。」趙明笑道︰「爲什麼你就甘心受我欺騙?因爲你心中喜歡了我,是不是?」無忌忿忿的道︰「是便怎樣?」趙明道︰「我很開心啊。」

無忌見她笑語如花,直是動人,轉過了頭不去看她。趙明道︰「我在樹上枕了半日,肚裡好餓。」大聲叫店伴進來,取出一小錠黃金,命他快去烹煮一席上等酒菜。那店伴見到黃金,服侍得極是周到,水果點心,流水般送將上來,不一會送上酒菜。無忌道︰「咱們等義父回來一起吃。」趙明道︰「謝大俠一到,我性命不保,還是先吃個飽,待會児做個飽鬼的好。」無忌見她話是如此説,但神情舉止之間,却似一切有恃無恐的模樣。趙明又道︰「我這裡金子有的是,待會可叫店伴另整酒席。」無忌冷冷的道︰「我可不敢再跟你一起飲食,誰知你幾時又下十香軟筋散。」趙明臉一沉,説道︰「你不吃就不吃,你肚子餓,我管得著麼?」説罷自己吃了起來。無忌叫厨房裡送了幾張麵餅來,離得趙明遠遠的,自行坐在炕上大嚼。趙明席上是炙羊烤雞、炸肉膾魚,菜式極是豐盛,無忌自管自己吃他的麵餅。趙明吃了一會,忽然泪水一點點的滴在飯碗之中,勉強又吃了幾口,抛下筷子,伏在桌上抽抽噎噎的哭泣。

她哭了一會,抹乾眼泪,似乎心下輕快了許多,望望窗外説道︰「再過一個時辰,天就黑了,那韓林児不知解向何處,若是失了他的蹤跡,倒是不易相救。」無忌一凜,站起身來,道︰「正是,我還是先去救了韓兄弟回來。」趙明道︰「也不怕醜,人家又不是跟你説話,誰要你接口?」無忌見她忽嗔忽羞,忽喜忽愁,心下又是恨,又是愛,當眞不知如何纔好,匆匆將半塊麵餅三口吃完,便走出房去。趙明道︰「我和你同去。」無忌道︰「我不要你跟著我。」趙明道︰「爲甚麼?」無忌道︰「你是害死我表妹的兇手,我豈能和仇人同行?」趙明道︰「好,你去吧!」無忌走出了房門,忽又回身道︰「那你在這裡幹麼?」

趙明道︰「我在這児等你義父回來,跟他説知你去救韓林児去了。」無忌道︰「我義父嫉惡如仇,焉能饒你性命?」趙明嘆了口氣,道︰「那也是我命苦,有什麼法子?」無忌沉吟半晌,道︰「你還是避一避的好,等我回來再説。」趙明搖頭道︰「我也没什麼地方好避。」無忌道︰「好吧!你跟著我一起去救韓林児,再一起回來對質。」趙明笑道︰「這是你要我陪你去的,可不是我死纏著你,非跟你去不可。」無忌道︰「你是我命中的魔星,撞到了你,算是我倒霉。」趙明嫣然一笑,説道︰「你等我片刻。」順手帶上了門。過了好一會,趙明又打開房門,只見她已換上了女裝,貂皮斗篷,大紅錦衣,裝束極是華麗,無忌没想到她隨身的包裹之中,竟帶著如此貴重的衣飾,心想︰「此女詭計多端,行事在在出人意表。」趙明道︰「你呆呆的瞧著我幹麼?我這裡服好看麼?」張無忌道︰「顏如桃李,心似蛇蠍。」趙明哈哈一笑説道︰「多謝張大教主給了我這八字評語。張教主,你也去換一套好看的衣衫吧。」無忌慍道︰「我從小穿著破破爛爛,你若是嫌我衣衫襤褸,儘可不和我同行。」趙明道︰「你别多心。我只是想瞧瞧你穿了一身好看的衣衫之後,是怎生一副模樣。無忌哥哥,你在這児少待,我去給你買衣。反正那些化子們走的是入関的大道,咱們脚下快一些,不怕追他們不上。」也不等無忌回答,自己翩然出門。無忌坐在炕上,心下自責,自己總是不能剛硬,被這個小女子玩弄於掌股之上,明明是她害死了我表妹,仍是這般對她有説有笑,張無忌啊張無忌,你算是什麼男子漢大丈夫?有什麼臉來做明教教主,號令群雄?

久等趙明不歸,眼見天色將黑,心想︰「我幹麼定要等她?不如獨個児去將韓林児救了。」但又尋思︰倘若趙明買了衣衫回來,正好撞上謝遜,被他一掌擊在天靈蓋上,腦漿迸裂,死於非命,衣衫冠履散了一地,想到這等情狀,却又不自禁的不寒而慄。如此坐下又站起,只是胡思亂想,直聽到脚步細碎,幽香襲人,趙明捧了兩個包裹,走進房來。

無忌道︰「等了你這麼久!不用換了,快去追敵人吧。」趙明微笑道︰「已等了這許多時候,也不爭在這更衣的片刻。我已買了兩匹坐騎,連夜可以趕路。」説著解開包裹,將衣褲鞋襪,一件件的取將出來,説道︰「小地方没好東西買,將就著穿,咱們到了大都,再買過貂皮的袍子。」無忌心中一凜,正色道︰「趙姑娘,你想要我貪圖富貴,歸附朝廷,可乘早死了這個心。我張無忌是堂堂大漢子孫,便是裂土封王,也決不能投降蒙古。」趙明嘆了口氣,説道︰「張大教主,你瞧這是蒙古衣衫呢,還是漢人的服色?」説著將一件灰鼠皮袍提了起來。無忌見她所購衣衫都是漢人的裝束,於是點了點頭。趙明轉了個身,説道︰「你瞧我這模樣是蒙古的郡主呢,還是一個平常的漢家女子?」無忌心中怦然一動,先前只見她衣飾華貴,没想到蒙漢之分,此時經她提醒,纔想到她全然是漢人姑娘的打扮。只見她雙頰暈紅,眼中水汪汪的脈脈含情,無忌突然之間,明白了她的用意,説道︰「你\dash{}你\dash{}」趙明低聲道︰「你心中捨不得我,我什麼都彀了。管他什麼元人漢人,我纔不在乎呢。你是漢人,我也就是漢人,你是蒙古人,我也是蒙古人。你心中想的盡是什麼軍國大事,華夷之分,什麼興亡盛衰、權勝威名,無忌哥哥,我心想的,可就只是一個你,你是好人也罷,壞人也罷,是皇帝也罷,乞児也罷,對我都完全一樣。」

無忌心下感動,聽到她這番柔情無限的言語,不自禁的頗爲意亂情迷,隔了半晌,纔道︰「你所以害死我表妹,是爲了妬忌嗎?是怕我娶她爲妻麼?」趙明大聲道︰「殷姑娘不是我害的。你相信也好!不信也好,我便是這句話。」無忌嘆了口氣,道︰「趙姑娘,你對我一番情意,我人非木石,豈有不知?但到了今日這等田地,你又何必再來騙我?」趙明道︰「我從前自以爲聰明伶俐,處處可佔上風,那知世事難料。無忌哥哥,今天咱們不走了,你在這児等謝大俠,我到周姑娘的房中等她。」無忌奇道︰「爲什麼?」趙明道︰「你不用問爲什麼。韓林児的事你不用擔心,我擔保一定救他出來便是。」説著翩然出門,走到周芷若房中,関上了房門。

無忌一時捉摸不透她用意何在,斜倚在炕上,苦苦思索,突然想起︰「莫非她已料想到我和芷若已有婚姻之約,因此害了我表妹一人不彀,又想用計再害芷若?莫非那玄冥二老離開彌勒廟之後,便到這客店中來算計我義父和芷若?」他一想到玄冥二老,心下登時好生驚恐,須知鹿杖客和鶴筆翁武功實在太強,謝遜縱然眼睛不盲,也未必能和任何人一打個平手。他跳起身來,走到趙明房外,説道︰「趙姑娘,你手下的玄冥二老到何處去了?」趙明隔著房門道︰「他二人多半以爲我脱身回去関内,向南追下去了。」無忌道︰「你此話可眞?」趙明冷笑道︰「你既不信我的話,又何必問我?」無忌無言可對,呆立在門外。趙明又道︰「假若我跟你説,我派了玄冥二老,來這客店中害死了謝大俠和你心愛的周姑娘,你信不信?」

這兩句話説中了無忌的心事,他飛起一足,{\upstsl{踼}}開房門,額頭青筋暴露,顫聲道︰「你\dash{}你\dash{}」趙明見他這等模樣,心中也害怕起來,後悔適纔説了這幾句言語,忙道︰「我這是嚇嚇你的,你可别當眞。」無忌凝視著,緩緩説道︰「你不怕到客店中來見我義父,口口聲聲要和他們對質,是不是你明知他二人,現下已經死了?已經不在這世上了?」説著走上兩步,和趙明相距不過三尺,只須手起一掌,立即便能將她斃於掌下。

趙明凝視他的雙眼,正色道︰「張無忌,我跟你説,世上之事,除非親眼目睹,不可輕信,不可妄聽人言,更不可自己胡思亂想。你要殺我,此刻便可動手,待會等你義父回來,你心中怎樣?」無忌定了定神,暗自有些慚愧,説道︰「我義父平安無事,那自是上上大吉。我不許你拿我義父的生死安危來隨口説笑。」趙明點頭道︰「我不該説這些話,是我對你不起,你别見怪。」無忌聽她柔聲認錯,心下倒也軟了,微微一笑,説道︰「我也忒以莽撞,得罪了你。」説著回到了謝遜房中。

但這晩等了一夜,天明睡醒,仍是不見謝遜和周芷若回來。無忌更加擔心起來,胡亂用了些早點,便和趙明商量,到底他二人到了何處。趙明皺眉道︰「這也當眞奇了。我想此間一帶,這些日子中除了丐幫聚會,並無其他江湖人衆出没。不如咱們追上史火龍等一干人,再行設法探聽。」無忌點頭道︰「也只有如此。」當下結算店帳出房,交代掌櫃,當謝遜、周芷若回來,請他們在店中等候。店伴牽過兩匹粟色的駿馬來。無忌見雙駒毛色光潤,腿高體長,是関外極名貴的良駒,不禁喝了聲采。趙明微微一笑,翻身上了馬背。兩騎馬潑剌剌的馳出了鎭甸,向南疾馳而去。旁人但見雙駿如龍,馬上一男一女衣飾華貴,相貌俊美,還道是什麼官宦人家的少年夫妻,並騎踏青。

兩人馳了一日,這天行了二百餘里,途中宿了一宵,次晨又再趕道。

\chapter{梟獐之心}

將到中午時分,朔風陣陣從身後吹來,天上陰沉沉的,灰雲便如壓在頭頂一般,又馳出二十餘里,鵞毛般的雪便一片片的飄將下來。一路上無忌和趙明極少交談,眼見這雪越下越大,無忌仍是一言不發的縱馬前行。這一日途中所經,盡是荒涼的山徑,到得傍晩,雪深近尺,兩匹馬雖然神駿,但在雪中,一提一滑,委實也是支持不住了。無忌見天色越來越黑,縱身站在馬鞍之上,四下一望,不見房屋人煙,心下好生躊躇,説道︰「趙姑娘,你瞧怎生是好?若再趕路,兩匹牲口只怕挨不起。」趙明冷笑道︰「你只知牲口挨不起,却不理人的死活。」無忌被她這麼一説,甚感歉仄,暗想︰「我身有九陽神功,不知疲累寒冷,急於救人,却没去顧她。」

又行一陣,忽聽得忽喇一聲響,一隻獐子從道左竄了出來,奔入了山中。無忌道︰「我去捉來做晩餐。」身隨聲起,躍離馬鞍,跟著那獐子在雪中留下的足跡,直追了下去,轉過一個山坡,暮靄朦朧之中,只見那獐子鑽向一個山洞。無忌一提氣,身子如箭般追了過去。没等那獐子進洞,已一把抓住牠的後頸。那獐子回頭露出利齒,要往無忌手腕上咬去。無忌五指一使勁,喀喇一聲,已將獐子頸骨折斷。見那山洞雖不寬大,但勉強可供二人容身,當下提著獐子,回到趙明身旁,説道︰「那邉有個山洞,我們暫且過一晩再説,你説如何?」趙明點了點頭,忽然臉上一紅,轉過頭去提起韁縱馬先行。

無忌將兩匹馬牽到山坡後兩株大松樹下躱雪,又在各處樹上找尋了二十來根枯枝,在洞口生起火來,只見那山洞倒頗是乾淨,並無獸糞穢跡,向裡望去,黑黝黝的不見盡處,於是將獐子剖剝了,用雪擦洗乾淨,在火堆上烤了起來。趙明除下貂裘,舖在洞中地下。火光熊熊,烘得山洞溫暖如春,無忌偶一回頭,只見火光一明一暗,映得趙明俏臉倍增明艷。兩人相視而嘻,一日來的疲累飢寒,盡化於一笑之中。

獐子烤熟後,兩人各撕一條後腿吃了。無忌在火堆中加些枯柴,斜倚在山洞壁上,説道︰「睡了吧!」趙明嫣然微笑,靠在另一邉石壁上,合上了眼睛。無忌鼻中聞到她身上陣陣幽香,微微睜眼,只見她雙頰暈紅,美若海棠,眞想湊過嘴去吻她一吻,但隨即克制綺念,閉目睡去。

睡到中夜,忽聽得遠遠隱隱傳來馬蹄之聲,無忌一驚而醒,側耳一聽,共是四匹坐騎,自南向北而來,向洞外望去,只見大雪兀自下個不停,心想︰「深夜大雪,如此冒寒趕路,定有十二分的急事。」只聽得馬蹄聲來到近處,忽然停住了,過了一會,馬蹄聲竟是越響越近,顯是走向這山洞而來。無忌一凜︰「這山洞僻處山後,若非那獐子引路,我是決計尋覓不到,怎麼竟然有人跟蹤而至。」隨即省悟︰「是了!咱們在雪地裡留下了足跡,雖是半夜大雪,仍是未能盡數掩去。」這時趙明也已醒覺,低聲道︰「來者或是敵人,咱們雖然不怕,還是避一避的好,且瞧他們是何等樣人。」無忌道︰「他們是從南方來的。」趙明道︰「這纔奇怪啊。」説著抄起洞外白雪,掩熄了火堆。

這時馬蹄聲已然止歇,但聽得四個人踏雪而來,頃刻間已到了洞外數十丈處。無忌低聲道︰「這四人身法好快,竟是極強的高手。」眼見若是出外覓地躱藏,非被那四人發覺不可,正没計較處,趙明拉著他的手掌,縮到了裡洞。那山洞越是向裡,越是狹窄,但竟然甚深,進得一丈有餘,便是一個轉折,忽聽得洞外一人説道︰「這裡有個山洞。」

無忌聽這説話的聲音好熟,正是四師叔張松溪的話聲,甫驚喜間,又聽得另一人道︰「七弟的標記指向此處,説不定曾到過這個山洞。」那却是六俠殷利亨的語音。張無忌正要出聲招呼,趙明伸過手來,按住了他的口,在他耳邉低聲道︰「你跟我住在這裡,給他們見了,多不好意思。」無忌一想不錯,自己和趙明雖是光明磊落,不欺暗室,但一對少年男女,同宿在這山洞之中,給衆師叔伯見了,他們怎信得過自己絶無苟且之事?何況趙明乃是元室的郡主,曾將張松溪、殷利亨等都擒在萬法寺中,頗加折辱,此時仇人相見,極是不便,暗想︰「我還是待張四叔等出洞後,和趙姑娘再分手,再單身趕去厮見,以免{\upstsl{尷}}尬。」

只聽得兪蓮舟的聲音説道︰「咦,這裡有燒過松柴的痕跡,{\upstsl{嗯}},還有獐子的毛皮血漬。」另一人道︰「我一直心中怔忡不定,但願七弟平安無事纔好。」那是宋遠橋的聲音。無忌聽得宋兪張殷四位師叔伯一齊出馬,前來找尋莫聲谷,聽他們話中之意,似乎莫聲谷遇上了強敵,心下也有些掛慮。聽張松溪笑道︰「大師哥愛護七弟,還道他仍是當年少不更事的小師弟,其實近年莫七俠威名赫赫,早非昔比,就算遇上強敵,七弟一人也必對付得了。」殷利亨道︰「我倒不是擔心七弟,反而擔心無忌這孩子不知身在何處。他現下是明教教主,樹大招風,不少人要算計於他。他武功雖高,可惜爲人太過忠厚,不知江湖上風波險惡,只怕墮入奸人的術中。」無忌聽了,心下好生感動,暗想衆位師叔伯待我恩情深重,眞不知如何報答。趙明湊嘴到他耳邉,低聲道︰「我是奸人,此刻你已墮入我的術中,你可知道麼?」

只聽得宋遠橋道︰「七弟到北路尋覓無忌,似乎已找得了什麼線索,只是他在天津客店中匆匆留下的八個字,却叫人猜想不透。」張松溪道︰「『門戸有變,亟須清理。』咱們武當門下,難道還會出什麼敗類不成?莫非無忌這孩子\dash{}」他説到這裡,便説不下去了,聲音之中,暗藏深憂。殷利亨道︰「無忌這孩子決不會做什麼敗壞門戸之事,那是我信得過的。」張松溪道︰「我只是怕趙明這妖女太過厲害,無忌少年人血氣方剛,惑於美色,莫要像他爹爹一般,鬧得身敗名裂\dash{}」四個人不再言語,都是長嘆一聲。

接著聽得火石打火之聲,松柴畢剝聲響,生起火來。那火光映到後洞,雖是經了一層轉折,無忌仍可隱約見到趙明的臉色,只見她似怨似怒,想是聽了張松溪的言語,甚是氣惱。無忌心中却是惕然而驚,尋思︰「張四叔的話倒也有理。我媽媽並没做什麼壞事,已累得我爹爹如此,這趙姑娘殺我表妹、辱我太師父及衆師伯叔,如何是我媽媽之比?」想到此處,一顆心怦怦而跳,暗想︰「若被他們發見我和趙姑娘在此,那我便傾黃河之水,也是洗不清了。」只聽得宋遠橋忽然顫聲道︰「四弟,我心中一直藏著一個疑竇,不便出口。若是説將出來,不免對不起咱們死了的五弟。」張松溪緩緩的道︰「大哥是否擔心無忌會對七弟忽下毒手?」宋遠橋不答。無忌雖不見他的身形,猜想他定是慢慢的點了點頭。

只聽得張松溪道︰「無忌這孩児本性忠厚,按理説是決計不會。我只擔心七弟脾氣太過莽撞,若是逼得無忌急了,令他難於兩全,再加上趙明那奸女安排奸計,從中挑撥是非,那就\dash{}那就\dash{}唉,心心叵測,世事難於逆料,自來英雄難過美人関,只盼無忌在大関頭能把持得定纔好。」殷利亨道︰「大哥,四哥,你們説這些空話,不是杞人憂天麼?七弟未必會遇上什麼兇險。」宋遠橋道︰「可是我見七弟這柄隨身的長劍,可眞令人心驚肉跳,寢食難安。」

兪蓮舟道︰「這件事確也有些費解,咱們練武之人,隨身兵刃不會隨手亂放,何況此劍是師父所賜,當眞是劍在人在,劍亡人\dash{}」説到這個「人」字,驀地住口,下面這個「亡」字硬生生的忍口不言。無忌聽説莫聲谷抛下了師傳長劍,而四位師伯叔更有疑己之意,心中又是擔憂,又是氣苦,突然之間,内洞中傳出一股濃烈的香氣,香氣之中,夾雜著野獸的騷氣,似乎内洞甚深,不是此刻藏有野獸,便是曾有野獸住過。他生怕被宋遠橋等知覺,連大氣也不敢透,拉著趙明之手,輕輕再向内洞,爲防撞到凸出的山石,左手伸在身前,只走了三步,轉了個彎,忽然左手碰到一件軟綿綿之物,似乎是個人體。

張無忌大吃一驚,心念如電︰「不論此人是友是敵,只須稍出微聲,大師伯們立時知覺。」左手直揮而下,連點他胸腹間五處要穴,隨即扣住他的手腕。觸手之處,一片冰冷,那人竟是氣絶已久。無忌借著些微光亮,凝目往那人臉上瞧去,隱隱約約之間,竟覺這死屍便是七師叔莫聲谷。無忌驚惶之下,顧不得是否會被宋遠橋等人發見,抱著那屍體向外走了幾步。光亮漸強,看得清清楚楚,却不是莫聲谷是誰?但見他臉上全無血色,雙目未閉,越顯得怕人。無忌悲憤交集,一時間竟自呆了。

他這麼幾步一走,宋遠橋等已聽到聲音。兪蓮舟喝道︰「裡面有人。」寒光閃動,武當四俠一齊抽出長劍。無忌暗暗叫苦︰「我抱著莫七叔的屍身,藏身此處,這殺叔的罪名,無論如何是逃不掉的了。」想起莫聲谷對自己的種種好處,此刻見他慘遭喪命,心下又是萬分悲痛,霎時間腦海中閃過千百個念頭,却没想到宋遠橋等進來之時,如何爲自己洗刷。

趙明的心思却比他轉得更快,縱身而出,舞動長劍,直闖了出去,刷刷刷刷四劍,倶是峨嵋派拚命的招數,分向四俠刺去。四俠舉劍一擋,趙明早已闖出洞口,飛身上了馬背,反手劍格開張松溪刺來的一劍,伸足在馬腹上一踢,那馬吃痛,疾馳而去。趙明方慶脱險,突然背上一痛,眼前金星亂舞,氣也透不過來,却是吃了兪蓮舟一招飛掌。她伏在馬鞍之上,神智已然迷糊,須知兪蓮舟功力何等深厚,這一掌須未打實,却已令她身受重傷。只聽得武當四俠展開輕功,自後急追而來。趙明心下只想︰「我逃得越遠,他越能出洞脱身。否則這不白之冤,如何能彀洗脱?好在四人都追了出來,没人想到洞中尚有别人。」耳聽得四人越追越近,她伸劍在馬背臀上一刺,那馬吃痛,四蹄如飛,直竄了出去。

無忌見趙明闖出,一怔之間,方才明白她這是調虎離山之計,好救自己脱身,當下抱著莫聲谷的屍身,奔出洞來。耳聽得趙明與武當四俠是向東而去,於是向西疾行。奔出二里有餘,在一塊大岩後將屍身藏好,再回到大路之旁,縱上一株大樹,良久良久,心中仍是怦怦亂跳,想到莫聲谷慘死,又是泪流難止,心想︰「我武當派直是多難如此,不知殺害七師叔的兇手却是何人?」

過了小半個時辰,聽得三騎馬自東南向北而來,雪光反映之下,看到宋遠橋和兪蓮舟各乘一馬,殷利亨和張松溪兩人共騎。只聽得兪蓮舟道︰「今日纔報了萬法寺被囚之辱,出了胸口惡氣。只是她竟也躱在這山洞之中,世事奇幻,出人意表。」殷利亨道︰「四哥,你猜她一個人鬼鬼祟祟的在洞裡幹什麼?」張松溪道︰「那就難猜了。殺了妖女,没有什麼,只有找到了七弟,咱們纔眞的高興。」四個人漸行漸遠,以後的話便聽不見了。

無忌待宋遠橋等四人去遠,忙縱下樹來,循馬蹄在雪中留下的印痕,向東追去,心下説不出的焦急難受,暗想︰「她雖生性狡詐惡毒,這次却確是捨命救我。倘若她竟因此送命,我\dash{}我\dash{}」脚下越奔越快,片刻間便已馳出四五里地,來到一處懸崖邉上。雪地裡但見一大灘殷紅的血漬,地下痕印雜亂,懸崖邉上崩壞了一大片山石,顯是趙明騎馬逃到此處,慌不擇路,連人帶馬,一起摔了下去。無忌叫道︰「趙姑娘,趙姑娘!」連叫四五聲,始終不聽見趙明答應。他更是憂急,向懸崖下望去,見是一個深谷,黑夜之中,没去見到谷底如何。那懸崖陡峭筆立,並無降到谷中的容足之處。

無忌吸一口氣,雙足先伸了下去,面朝崖壁,便向下滑去。這一著原是十分冒險,但他急於救人,已是不及多想。滑下三四丈,又順勢滑下。如此五六次,纔到谷底,著足之處却是軟軟的,急忙躍開,原來是踏在那匹死馬腿上,只見趙明身未離鞍,隻手仍是牢牢的抱著馬頸。無忌伸手一探她的鼻息,尚有細微呼吸,人却已然暈了過去。無忌稍稍放心,此時每跨一步,積雪便深及腰間,竟是舉步維艱。幸好谷中陰暗,一冬的積雪都未熔化,加以趙明身未離鞍,摔下的力道都由那馬承受了去,坐騎登時震死,趙明却只昏暈。無忌搭了搭他脈搏,知道雖然受傷不輕,性命却可無礙,於是將她抱在懷裡,四掌相抵,運功給她療傷。

無忌精通醫理,神功深厚,趙明所受這一掌又是武當派的本門功夫,是以不到半個時辰,趙明已悠悠醒轉。無忌將九陽眞氣源源送入她的體内,又過大半個時辰,天色漸明,趙明哇的一聲,吐出了一大口瘀血,低聲道︰「他們都去了?没見到你吧?」無忌聽她最関心的乃是自己是否會蒙不白之冤,心下好生感激,説道︰「没見到我。你\dash{}你可受了苦啦。」他一面説話,眞氣的傳送仍是絲毫不停。趙明閉上了眼睛,雖是四肢乏力,胸腹之間甚感溫暖舒暢。那九陽眞氣在她體内又運走數轉,趙明回過頭來,笑道︰「你歇歇吧,我好得多啦。」無忌雙臂環抱,圍住了她的腰,將右頰貼在她的左頰,説道︰「你救了我的聲名,那比救我十次性命,更是令我銘感。」趙明格格一笑,説道︰「我是個奸詐惡毒的小妖女,聲名是不在乎的,倒是性命要緊。」

便在此時,忽聽懸崖上傳下一人聲音,朗聲呼道︰「該死的妖女,果然未死,你何以害死莫七俠,快快招來。」正是兪蓮舟的聲音。無忌大吃一驚,不知四位師伯怎地去而復回。趙明道︰「你别轉頭,不可讓他們見到你的臉。」張松溪喝道︰「賊妖女,你不回答,咱們的大石便{\upstsl{砸}}將下來了。」趙明仰頭一望,果見宋遠橋等四人每人都捧著一塊大石,只須順手往下一摔,她和無忌都是性命難保。她在無忌耳邉低聲説道︰「你先撕下皮裘,蒙在臉上,抱著我逃走吧。」無忌依言,撕下裘袍的一角衣襟,蒙在臉上,在腦後打了個結,又將帽低低壓在額上,只露出了雙眼。

原來武當四俠追趕趙明,將她逼入谷底,但這四人行俠江湖,見識何等廣博,料想趙明以郡主之尊,不致孤身而無護衛。四人假意騎馬遠去,行出數里之後,將馬繫在道旁樹上,又悄悄回來搜索。四俠先回山洞,點了火把,深入洞裡,在裡洞只見到兩隻死了的香獐。被什麼野獸咬得血肉模糊,體香兀自未散。四人再搜出洞來,終於見到無忌所留的足印,一路尋去,却發見了莫聲谷的屍體,但見他手足都已被野獸咬壞。四俠悲憤莫名,殷利亨已是哭倒在地。

兪蓮舟拭泪道︰「趙明這妖女武功雖強,但憑她一人,決計害不了七弟。六弟且莫悲傷,咱們須當尋訪到所有的兇手,一一殺了給七弟報仇。」張松溪道︰「咱們隱伏在山洞之側,到得天明,妖女的手下必會尋求來。」武當諸俠之中,以張松溪最是足智多謀,宋遠橋等向來對他言聽計從,當下強止悲傷,各在山洞兩側尋覓岩石藏身守候。到得天明,却不見有趙明手下人尋來,四俠再到趙明墜崖處察看,隱隱聽到説話之聲,向下一望,只見一個錦衣男子抱著趙明,原來這妖女竟是未死。四俠要逼問莫聲谷的死因,不願便用石頭擲死二人。

這雪谷形若深井,四周都是石壁,唯有西北角上有一條狹窄的出路。張松溪喝道︰「兀那元狗,你們從這邉上來,若再延擱,斗大的石塊{\upstsl{砸}}將下來了。」張無忌聽得四師伯誤認自己爲蒙古人,想是自己衣飾華貴,又是跟隨著趙明之故,但見四下裡並無可以隱伏躱避之處,四俠將大石{\upstsl{砸}}將下來,自己縱可跳躍閃避,趙明却是性命難保,眼下只有依言上去,走得一步算一步了,於是抱著趙明,從那窄縫中慢慢爬將上來。他故意顯得武功低微,走幾步便滑跌一下,這條窄縫本是絶難攀援,他更加意做作,大聲喘氣,十分狼狽,搞了半個時辰,摔了十七八交,纔攀到了平地。無忌一出雪谷,本想立即抱了趙明奪路而逃,憑著自己輕功,手中雖然多抱一人,四俠只怕仍是追趕不上。但張松溪極是機靈,瞧出他上山之時的狼狽神態有些做作,早已通知三個師兄弟,四人分佈四角,四柄長劍的劍尖離他身子不及半尺。

宋遠橋狠狠的道︰「賊韃子,你用毛皮蒙住了鬼臉,便逃得了性命麼?武當派莫七俠是誰下手害死的,好好招來!若有半句虛言,我將你這狗韃子千刀萬剮,開肚破膛。」他性子本來恬淡沖和,但眼見莫聲谷死得如此慘法,忍不住口出惡聲,那是數十年來極爲罕有之事。趙明嘆了口氣,説道︰「押魯不花將軍,事已如此,你就對他們説了吧!」跟著湊嘴在無忌耳邉,低聲道︰「用聖火令武功。」

無忌本來極不願對四位師叔伯動武,但形格勢禁,處境實是{\upstsl{尷}}尬之極,驀地裡一咬牙,舉起趙明的身子,便向殷利亨抛了過去,粗著嗓子胡胡大呼,在半空中翻個空心斛斗,伸臂向張松溪抓到。殷利亨一驚之下,順手接住了趙明,呆了一呆,便點了她的穴道,將她摔了出去,在這瞬息之間,無忌已使開聖火令上的怪異武功,拳打宋遠橋,脚踢兪蓮舟,一個頭槌向張松溪撞到,反手却奪了殷利亨手中的長劍。這幾下兔起鶻落,既快且怪。武當四俠廣博,可説是中原武林中的第一流高手,但給張無忌這接連七八招怪招一陣亂打,登時手忙脚亂,竟感難以自保。

那日在靈蛇島上,以張無忌武功之強,遇上波斯明教流雲三使的聖火令招數,也是抵敵不住,何況此時他已學全六枚聖火令上的全部功夫,比之流雲三使,高出何止數倍?這聖火令上所載,本非極深邃的上乘功夫,只是詭異古怪,令人捉摸不定,若在庸手使來,亦非武當派内家正宗武功之敵。但張無忌以九陽神功爲根基,以挪移乾坤心法爲脈絡,加之對武當派武功盡數了然於胸,一招一式,無不攻向武當四俠的空隙之處。鬥到二十餘招時,那聖火令功夫越來越是奇幻莫測。趙明躺在雪中,大聲叫道︰「押魯不花將軍,他們漢人蠻子自以爲了得,咱們蒙古這種祖傳摔角神技,今日叫他們嘗嘗滋味。」張松溪叫道︰「以太極拳自保,這種韃子拳招古怪得緊。」四人立時拳法一變,使開太極拳法,將門戸守得嚴密無比。無忌突然坐倒在地,雙拳猛搥自己胸膛。

武當四俠生平不知遭逢過多少強敵,見識過多少怪招,張無忌乾坤大挪移心法,算得是武學中奇峰突起的功夫了,但這個韃子坐在地下自搥胸膛,不但見所未見,連聽也没聽過。四俠本已收起長劍,各使太極拳守緊門戸,此時一怔之下,宋遠橋、兪蓮舟、張松溪三柄長劍又刺向張無忌身前,殷利亨的長劍已被無忌奪去擲開,但他身邉尚擕著莫聲谷的佩劍,跟著也拔出來刺了過去。張無忌橫腿一掃,原是山中老人在波斯踢起黃沙,襲擊駱駝商隊之用。他是波斯大盜,慣常在沙漠中打劫行商,見有商隊遠遠行來,便坐地搥胸,呼天搶地的哭號,衆行商自必過去探問。他突然間踢起飛沙,迷住衆商眼目,跟著便是長刀疾刺,可在頃刻之間,使數十行商血染黃沙,屍橫大漠,實是一招極陰毒的手法。張無忌以此招踢飛積雪,功效與踢沙相同。

武當四俠在霎時之間,但覺得飛雪撲面,眼睛不能見物,四人應變奇速,立時後躍。但無忌比他們更快,抱住兪蓮舟雙腿,著地一滾,順手已點了他三處大穴,跟著一個斛斗,身在半空,落下時右腿的膝蓋在殷利亨頭頂一跪,竟然撞中了他頂門「五處」和「承光」兩穴。殷利亨一陣暈眩,摔倒在地。宋遠橋飛步來救,無忌身子向後一坐,撞入他的懷中。宋遠橋迴劍不及,左手撤了劍訣,一掌拍出,掌力未吐,胸口已是一麻,被無忌雙肘撞中了穴道。張松溪心下大駭,眼見四人中只剩下一人,無論如何非此人敵手,但同門義重,決計示能獨自逃命,挺起長劍,刷刷刷三劍,向無忌刺了過來。無忌見他身當危難,可是止法沉隱,劍招絲毫不亂,這三劍來得凌厲,但每一劍仍是嚴守武當家法,心下暗暗喝采︰「武當武功,實非尋常,若不是我學到了這一門古怪功夫,要抵擋四位師叔伯的聯手進攻,大非易事。」驀地裡腦袋亂擺,劃著一個個圏子。張松溪不爲所動,不去瞧他搖頭晃腦的裝模作樣,嗤的一聲,長劍破空,直往他胸口刺來。無忌一低頭,似用腦袋往劍尖上迎去,但忽地臥倒向前一撲,張松溪小腹和左腿上四處穴道被點,摔倒在地。

無忌知道所點這四處穴道只能制住下肢,正要往他背心「中樞」「陶道」兩穴各補一次,猛聽得張松溪大聲慘呼,雙眼翻白,上身一陣痙孿,直挺挺的死了過去。無忌這一驚眞是非同小可,心想適纔所點穴道並非重手,别説不會致命,連輕傷也不致於,難道四師伯身有隱疾,陡然間遇此打擊,因而發作麼?他背上刹那間出了一陣冷汗,伸手去探張松溪的鼻息,突然之間,張松溪左手一探,已拉下了他臉上蒙著的衣襟。兩人面面相覷,都是呆了。

過了好半晌,張松溪纔道︰「好無忌,原來是你,不枉了咱們如此待你。」他説話聲音已然哽咽,滿臉憤怒,眼泪却已涔涔而下,説不出是氣惱還是傷心。原來他自知不敵,但想至死不見敵人面目,不知武當四俠喪在何人手中,直是死不瞑目,是以先裝假死,拉下了無忌臉上這一塊皮裘。無忌一來老實,二來對四師伯関心過甚,竟爾没有防備。無忌此刻心境,眞比身受凌遲還要難過,一個人全然傻了,只道︰「四師伯,不是我,不是我\dash{}七師叔不是我\dash{}不是我害的\dash{}」張松溪哈哈大笑,説道︰「很好,很好,你快快將咱們一起殺了。大哥、二哥、六弟,你們都瞧清楚了,這狗韃子不是旁人,竟是咱們鍾愛的無忌孩児。」宋遠橋、兪蓮舟、殷利亨三人身子不能動彈,一齊怔怔的瞪視著無忌。

張無忌此時心境,眞想拾起地下的長劍,往頸中一抹。趙明忽然叫道︰「張無忌,大丈夫忍得一時冤屈,打什麼緊,天下没有不能水落石出之事。你終須找到殺害莫七俠的眞兇,爲他報仇,那纔不枉了武當諸俠愛你一場。」無忌心中一凜,深覺此言有理,説道︰「咱們此刻該當如何?」説著走到她身前,在她背心和腰間諸穴上推宮過血,解開了她被點的穴道。趙明柔聲慰道︰「你别氣苦!你明教中有這許多高手,我手下也不乏才智之士,這眞兇定能擒獲。」張松溪叫道︰「張無忌,你若還有絲毫良心,快快將咱四人殺了。我見不得你跟隨這妖女卿卿我我的醜模樣。」

無忌臉色鐵青,實是没了主意。趙明道︰「咱們當先去救韓林児,再回去找你義父,一路上探訪害你莫七叔的眞兇,探訪害你表妹的兇手。」無忌道︰「什\dash{}什麼?」趙明冷冷的道︰「莫七俠是你殺的麼?爲什麼你四位師伯叔認定是你?殷離是我殺的麼?爲什麼你認定是我?難道只可以你去冤枉旁人,却不容旁人冤枉於你?」這幾句話猶如雷轟電擊一般,直鑽入無忌的耳中,他此刻親身經歷,方自知世事陰差陽錯,往往難以測度,體會到身蒙不白之冤的苦處,「難道趙姑娘她\dash{}她\dash{}竟然和我一樣,也是被人冤枉了麼?」

趙明道︰「你點四位師伯叔的穴道,他們能自行撞開麼?」無忌搖頭道︰「這是聖火令上的奇門功夫,師伯叔們不能自行撞解,過得十二個時辰後,自會解開。」趙明道︰「{\upstsl{嗯}},咱們將四位送到山洞之中,即便離去。在眞兇找到之前,你是不能再跟他們相見的了。」無忌道︰「那山洞中有野獸,有獐子出入來去,莫七叔的屍身,就給野獸咬壞了。」趙明嘆道︰「瞧你方寸大亂,什麼也想不起來。只須有一位上身能彀活動,手中有劍,什麼野獸能侵犯得他們?」無忌道︰「不錯,不錯。」當下將武當四俠抱起,放在一塊大巖岩後以以避風雪。四俠罵不絶口,無忌眼中含泪,並不置答。趙明道︰「四位是武林高人,却如此不明事理。莫七俠倘若是張無忌所害,他此刻一劍將你們殺了滅口,有何難處?他忍心殺得莫七俠,便不忍心加害你們四位。你們若再口出惡言,我趙明每人給你們一個耳光。我是陰毒險惡的妖女,説得出便做得到。當日在萬法寺中,我瞧著張公子的份上,對各位禮敬有加。少林、崑崙、峨嵋、華山、崆峒五派高手,人人被我截去了手指。但我趙明對武當諸俠可有半點禮教不周之處麼?」

宋遠橋等聽了此言,面面相覷,雖然仍是認定張無忌害死了莫聲谷,但生怕趙明當眞出手打人,大丈夫可殺不可辱,被這小妖女打上幾個耳光,那可是生平奇恥。趙明微微一笑,向無忌道︰「你去牽咱們的坐騎來,馱四位去山洞。」無忌猶豫道︰「還是我來抱吧。」趙明心念一動,已知他的心意,冷笑道︰「你武功再高,能同時抱得了四人麼?你怕自己一走開,我便加害四俠。你終始是不相信我。好,我去牽坐騎,你在這裡守著吧。」無忌給她説中了心事,臉上一紅,但確是不敢將四位師伯叔的性命,交託在這個性情難以捉摸的少女手中,便道︰「勞駕你去牽牲口,我在這裡守著四位師伯叔。」趙明冷笑道︰「你再殷勤好心,旁人還是不信你的。你的赤心熱腸,人家只當你是狼心狗肺。」説著轉身便去牽馬。

無忌咀嚼著她這幾句話,只覺她説的似是師伯叔疑心自己,却也是説自己疑心於她。目送著趙明的背影在雪地中漸漸遠去,忽聽得一陣急促的馬蹄聲,沿著大路從北而南的奔來。一前二後,共是三乘。

\chapter{獅王行蹤}

趙明也已聽到這馬蹄聲音,急速奔回,説道︰「有人來了!」無忌向她招了招手。趙明奔到大石之後,伏在無忌身旁,眼見兪蓮舟的身子有一半露在石外,當即將他拉到石後。兪蓮舟怒目而視,説道︰「别碰我!」趙明笑道︰「我偏要拉你,瞧你有什麼法子?」無忌喝道︰「趙姑娘,不得對我師伯無禮。」趙明伸了舌頭,向兪蓮舟裝個鬼臉。

便在此時,一乘馬已奔到不遠之處,其後又有兩乘馬如飛追來,相距約有二三十丈。第一乘馬越奔越近,無忌眼尖,突然低聲道︰「是宋青書大哥!」趙明道︰「快阻住他?」無忌奇道︰「幹什麼?」趙明道︰「你别多問,彌勒佛殿中的話你忘了麼?」無忌心念一動,拾起地下一粒指頭大的冰塊,彈了出去。嗤的一聲,冰塊破空而去,正中宋青書坐騎的前腿。那馬一痛,跪倒在地。宋青書一躍而起,想拉坐騎站起,但那馬一摔之下,左腿已然折斷。宋青書見後面追騎漸近,忙向這邉奔了過來。無忌又是一粒堅冰彈了過去,撞中他右腿穴道。趙明伸出手指,接連四下,點了武當四俠的啞穴,及時制止宋遠橋的呼喚。只聽得宋青書「啊」的一聲叫,滾倒在雪地之中。這麼接連的兩阻,後面兩騎已奔到跟前,却是丐幫的陳友諒和掌砵龍頭。無忌暗自奇怪︰「他二人同去長白山尋覓毒物,配製毒藥,怎麼一逃一追,到了這裡?」跟著又想︰「是了。想是宋大哥天良發現,不肯做此不孝不義之事,幸好撞在我的手裡,倒要救他一救。」

陳友諒和掌砵龍頭翻身下馬,只道宋青書的坐騎久馳之下,氣力不加,以致馬失前蹄,宋青書也因此墮馬受傷,但想他武功不弱,縱然受傷,也必極是輕微,兩人縱身而近,兵刃出手,指住他的身子。無忌指上又扣了一粒冰塊,正要向陳友諒彈去,趙明碰他臂膀,搖了搖手。無忌轉頭瞧她。趙明手指指自己耳朶,再指指宋青書,意思説且聽他們説些什麼。

只聽得當砵龍頭怒道︰「姓宋的,你黑夜中悄悄逃走,意欲何爲?是否想去通風報信,説與你父親知道?」他手中一柄紫金八卦刀,在宋青書頭頂晃來晃去,作勢便要砍下。宋遠橋聽得那八卦刀虛砍的劈風之聲,掛念愛児安危,大是著急。張無忌偶一回頭,見到他眼中焦慮的神色,霎時間變作了求懇,於是點了點頭,示意︰「你一切放心,我決不讓宋大哥身受損傷。」心中却想︰「父母愛子之意,當眞是天高地厚。大師伯對我如此惱怒,恨不得將我千刀萬剮,但知道宋大哥遭到危難,立時便向我求情。但若是大師伯自身遭難,他是英雄肝膽,決計不屑有絲毫示弱求懇之意。」刹那之間又想到宋青書有人関愛,自己却是個無父無母的孤児。

只聽宋青書道︰「我不是去向爹爹報信。」掌砵龍頭道︰「幫主派你跟我去長白山採藥,那麼你何以不告而别?」宋青書道︰「你也是父母所生,你們逼我去加害自己父親,心又何忍?我並非和你們作對,但我不能作此禽獸勾當。」掌砵龍頭厲聲道︰「那你是決意違背幫主號令了?叛幫之人該當如何處置,你知道麼?」宋青書道︰「我是天下罪人,本是不想活了,這幾天我只須一合眼,便見莫七叔來向我索命。他是怨魂不散,纏上了我啦。掌砵龍頭,你一刀將我砍死吧,我多謝你成全了我。」掌砵龍頭舉八卦刀,喝道︰「好!我便成全了你!」陳友諒插口道︰「龍頭大哥,宋兄弟既然執意不肯,殺他也是無益,咱們由他去吧。」掌砵龍頭奇道︰「你説就此放了他?」陳友諒道︰「不錯。他親手害死師叔莫聲谷,自有他本派中人殺他,這種不義之徒的惡血,没的汚了咱們兵刃。」

張無忌當日在彌勒佛廟中,曾聽陳友諒和宋青書説到莫聲谷,有什麼「以下犯上」之言,當時也曾疑心宋青書得罪了師叔,但萬萬料不到,莫聲谷竟會是死在他的手中。宋遠橋等四人雖然目光被岩石遮住,但宋青書的聲音清清楚楚的傳入耳中,無不大爲震動。唯有趙明事先已料到三分,嘴角邉微帶不屑之態。只聽宋青書顫聲道︰「陳大哥你曾發下重誓,決不洩漏此事的機密,只要你不説,我爹爹怎會知道?」陳友諒淡淡一笑,道︰「你只記得我的誓言,却不記得你自己發過的毒誓。你説自今而後,唯我所命。是你先毀約呢,還是我不守信諾?」

宋青書沉吟半晌,説道︰「你要我在太師父和爹爹的飲食之中下毒,我是寧死不爲,你快一劍將我殺了吧。」陳友諒道︰「宋兄弟,常言道識時務者爲俊傑。咱們又不是要你{\upstsl{弒}}父滅祖,只不過下些蒙藥,令他們昏迷一陣。在彌勒佛廟中,你不是早已答應了嗎?」宋青書道︰「不,不!我只答應下蒙藥,但掌砵龍頭捉的是劇毒的蝮蛇、蜈蚣,那是殺人的毒藥,決非尋常蒙汗藥物。」陳友諒悠悠閒閒的收起長劍,説道︰「峨嵋派中的周姑娘美若天人,世上再找不到第二個了,你甘心任她落入張無忌那小子的手中,當眞奇怪。宋兄弟,那日深宵之中,你去偸窺峨嵋諸女的臥室,被你七師叔撞見,一路追了你下來,致有石岡比武、以侄{\upstsl{弒}}叔之事。那爲的是什麼?還不是爲了這位溫柔多情的周姑娘?事情已經做下來了,一不做,二不休,馬入夾道,還能回頭麼?我瞧你爲山九仞,功虧一簣,可惜啊可惜。」

宋青書搖搖晃晃的站了起來,怒道︰「陳友諒,你花言巧語,逼迫於我。那一晩我給莫七叔追上了,敵他不過,我是敗壞武當門風,死在他的手下,倒是一了百了,誰要你出手相助?我是中了你的詭計,以致身敗名裂,難以自拔。」陳友諒笑道︰「很好,很好!莫聲谷背上所中這一掌『震天鐵掌』,是你打的,還是我陳友諒打的?那晩我出手救你性命,又保你名聲,倒是我幹錯了?宋兄弟,你我相交一場,過去之事不必再提。你{\upstsl{弒}}叔之事,我自當守口如瓶,決不洩露片言隻字。山遠水長,咱們後會有期。」宋青書聽他竟肯如此善罷,大起疑心,問道︰「陳\dash{}陳大哥,你\dash{}你要如何對付我?」陳友諒笑道︰「要如何對付你?什麼也没有。我給你瞧一樣物事,這是什麼?」

無忌和趙明躱在岩石之後偸聽,這時很想探頭上來張望一下,瞧陳友諒取了什麼東西出來,但終於強自忍住。只聽宋青書「啊」的一聲,道︰「這\dash{}這是峨嵋派掌門的鐵環,那是周姑娘之物啊,你\dash{}你從何處得來?」無忌聽了,心下也是一凜,暗想︰「我和芷若分手之時,明明見她戴著那枚掌門鐵環,如何會落入陳友諒手中?多半是他假造的膺物,用來騙人。」但聽陳友諒輕輕一笑,説道︰「你瞧瞧仔細,這是眞的還是假的。」隔了片刻,宋青書道︰「我在西域向滅絶師太討教武功,見過他手上這枚指環,看來倒是眞的。」只聽得{\upstsl{噹}}的一聲響,金鐵相撞,陳友諒道︰「若是假造的膺物,這一劍該將它斷爲兩半了。你瞧瞧,指環内『留給襄女』這四個字,不會是假的吧?這是峨嵋派祖師郭襄女俠的遺物玄鐵指環。」宋青書道︰「陳大哥,你\dash{}你從何處得來?周姑娘她人呢?」

陳友諒又是微微一笑,説道︰「掌砵龍頭,咱們走吧,丐幫從此没了這人。」脚步聲響,兩人向北便行。宋青書叫道︰「陳大哥,你回來。周姑娘是落入你手中了麼?她此刻是死是活?」

陳友諒走了回來,微笑道︰「不錯,周姑娘是在咱們手中,天生麗質,我見猶憐。我陳友諒至今未有家室,要是我向幫主求懇,將周姑娘配我爲妻,諒來幫主也必允准。」宋青書喉頭咕{\upstsl{噥}}了一聲,似乎塞住了説不出話來。陳友諒又道︰「本來嘛,君子不奪人之所好,宋兄弟爲了這位周姑娘,闖下了天大的禍事,陳友諒豈能爲了美色而壞兄弟義氣?但你既成了叛幫的罪人,咱們恩斷義絶,什麼也談不上了,是不是?」

宋青書低頭沉吟,内心交戰。張無忌眼角一瞥宋遠橋,只見他臉頰上兩道泪痕,顯是心中悲痛已極。忽聽得宋青書道︰「陳大哥,龍頭大哥,是我做兄弟的一時糊塗,請你兩位原宥,我這裡給你們陪罪啦。」陳友諒哈哈大笑,説道︰「是啊,是啊,那纔是咱們的好兄弟呢。我拍胸膛給你擔保,只須去將蒙汗藥帶到武當山上,悄悄下在各人的茶水之中,你令尊大人性命決然無憂,美佳人周芷若必成你的妻室。咱們有張三丰和武當諸俠在手,不愁張無忌不聽號令,等到丐幫箝制住明教,驅除韃子,得了天下,咱們幫主登了龍位,你我都是開國功臣。封妻蔭子,那是不必説了,連尊大人都要沾你的光呢。」宋青書苦笑道︰「我爹爹淡泊名利,我只盼他老人家不殺我,便心滿意足了。」陳友諒笑道︰「除非尊大人是神仙,能知過去未來,否則焉能知悉其中的過節?宋兄弟,你的脚摔傷了麼?來,咱們倆共乘一騎,到前面鎭上再買脚力。」宋青書道︰「我走得匆忙,小腿在冰塊上撞了一下,也眞倒霉,剛好撞正了『築賓穴』,天下事眞有這般巧法。」原來張無忌這冰塊擲去時用力甚奇,宋青書只顧住掌砵龍頭和陳友諒在後追趕,萬没想到前面岩後竟會有人暗算,只道是自己不小心,剛好將穴道撞正了冰塊尖角。須知此種事亦非出奇,有時無意中手臂在桌子角一撞,竟致片刻酸麻,那便是剛巧碰中穴道了。

陳有諒笑道︰「這那裡是倒霉?這是宋兄弟艷福齊天,命中該有佳人爲妻。若非這麼一撞,咱們追你不上,你執迷不悟起來,你自己固然鬧得身敗名裂,也壞了咱們大事。從此這位香噴噴、嬌滴滴的周姑娘跟陳友諒一世,那不是彩鳳隨鴉,一朶鮮花插在牛糞上了麼?」他言中似是説笑,實則是極厲害的威脅。宋青書「哼」了一聲,道︰「陳大哥,不是做兄弟的不識好歹,信不過你\dash{}」陳友諒不等他説完,插口便道︰「你要見上周姑娘一面,是不是?那容易之至。此刻幫主和衆位長老,都在盧龍,周姑娘也隨大夥在一起。咱們同到盧龍去相會便是。等武當山的大事一了,做哥哥的立時便給你辦喜事,叫你稱心如願,一輩子感激陳友諒大哥,哈哈,哈哈!」

宋青書道︰「好,那麼咱們便上盧龍去。陳大哥,周姑娘怎地會\dash{}會跟著本幫?」陳友諒笑道︰「那是龍頭大哥的功勞了。那日掌棒龍頭和掌砵龍頭在酒樓上喝酒,見有三個面生人混在其中,後來命人一査,其中一位竟然是那位千嬌百媚的周姑娘。掌砵龍頭便派人去將她請了來。你放心,周姑娘平安大吉,毫髮不傷。」無忌暗暗叫苦︰「那日在酒樓之上,原來畢竟還是讓他們瞧了出來。倘若義父並非失明,他老人家定瞧出其中蹊蹺。唉,我和芷若却都蒙在鼓中,兀自不覺。但不知義父也平安否?」

可是陳友諒説話中,却一句不提謝遜,只聽他道︰「周姑娘和你成親後,峨嵋、武當派都要聽丐幫號令,少林派已在我掌握之中,再加上丐幫和明教,聲勢何等浩大?只須打垮了蒙古人,這花花江山嗎,嘿嘿,可要換個主児啦。」

陳友諒説這幾句話時,志得意滿,不但似乎丐幫已得了天下,而且是他陳友諒自己身登大寶,穩坐龍庭。掌砵龍頭和宋青書都跟著他嘿、嘿、嘿的乾笑數聲。陳友諒道︰「咱們走吧。宋兄弟,莫七俠是死在這附近的,他藏屍的山洞似乎離此不遠,是不是?你逃到這裡,忽然馬失前蹄,難道是莫七俠陰魂顯聖麼?哈哈,哈哈!」這幾句話只聽得宋青書毛骨悚然,加快脚步,一跛一拐的去了。

張無忌待三人去遠,忙替宋遠橋等四人解開穴道,拜伏在地,連連磕頭,説道︰「師伯、師叔,侄児身處嫌疑之地,難以自辯,多有得罪,請伯叔們重重責罰。」宋遠橋一聲長嘆,虎目含泪,仰天不語。兪蓮舟忙扶起無忌,説道︰「咱們親如骨肉,這一切不必多説了。眞想不到青書\dash{}青書\dash{}唉,若非咱們親耳聽見,又有誰能彀相信?」宋遠橋刷的一聲,抽出長劍,説道︰「原來七弟撞見青書這小畜生\dash{}這小畜生\dash{}私窺峨嵋女俠寢居,這纔追下來清理門戸。三位師弟,無忌孩児,咱們這便追趕前去,讓我親手宰了這畜生。」説著身影一晃,展開輕功,疾向宋青書追了下去。

張松溪叫道︰「大哥請回,一切從長計議。」宋遠橋理也不理,只是提劍飛奔。張無忌發足追趕,幾個起落,已攔在宋遠橋身前,躬身道︰「大師伯,四師伯有話跟你説,宋大哥一時受人之愚,日後自必自悟,大師伯要責罰於他,也不忙在一時。」宋遠橋哽咽道︰「七弟\dash{}七弟\dash{}做哥哥的好對你不起。」突然回身,迴劍往自己脖子抹去。無忌大驚,一伸手,施展挪移乾坤手法,夾手將他長劍奪了過來,但劍尖終於在他頸中一帶,劃了一道長長的血痕。

這時兪蓮舟等也已追到跟前。張松溪勸道︰「大哥,青書做出這等逆不道的事來,武當門中,人人容他不得。但清理門戸事小,天下百姓的事大,咱們可不能因小失大。」宋遠橋圓睜雙眼,説道︰「你你説清理門戸之事還小了?我\dash{}我生下這等忤逆児子\dash{}」張松溪道︰「聽陳友諒之言,丐幫還想假手青書,謀害吾等師尊,挾制武林諸大門派,{\upstsl{篡}}奪江山。師尊的安危,是本門第一大事,天下武林和蒼生的禍福,更是第一等的大事。青書這孩児多行不義,遲早必遭逆報。咱們還是商量大事要緊。」宋遠橋聽他言之有理,恨恨的還劍入鞘,説道︰「我方寸已亂,便聽四弟吩咐吧。」殷利亨取出金創藥來,替他包紮頸中傷處。

張松溪道︰「丐幫既謀對師尊不利,此刻師尊尚自毫不知情,咱們須得連日連夜,急速趕回武當。這陳友諒雖説要假手於青書,但此種奸徒詭計百出,説不定提早下手,咱們眼前第一要務,是維護師尊的金軀。師尊年事已高,若再有假少林僧報訊之事,吾輩做弟子的萬死莫贖。」説著向站在遠處的趙明瞪了一眼,對她派人謀害張三丰之事,猶有餘憤。宋遠橋背上出了一陣冷汗,顫聲道︰「不錯,不錯。我急於追殺逆子,竟將師尊置諸腦後,輕重倒置,直是氣得胡塗了。」他是血性之人,連道︰「快走,快走!」張松溪向無忌道︰「無忌,搭救周姑娘之事,便由你去辦。事完之後,盼來武當一敘。」無忌道︰「遵奉師伯吩咐。」張松溪低聲道︰「這趙姑娘豺狼之性,你可得千萬小心。宋青書是前車之鑒,好男児大丈夫,決不可爲美色所誤。」張無忌紅著臉點了點頭。

當下武當四俠和無忌將莫聲谷的屍身葬在大石之後,五人痛哭了一場,宋遠橋等四人先行騎馬馳去。趙明慢慢走到無忌身前,説道︰「你四師伯叫你小心,别受我這妖女迷惑,宋青書是前車之鑒,是也不是?」

無忌臉上一紅,笑道︰「你怎麼知道?你有順風耳麼?」趙明哼了一聲,道︰「我説啊,宋大俠他們事後追想,不怪宋青書生就了梟獐之心,反而會怪周姊姊紅顏禍水,毀了一位武當少俠的一生。男人家的心思,我會猜不到麼?」無忌心想她這番話倒也未始没幾分道理,只道︰「宋大師伯他們都是明理君子,焉能胡亂怪人?」趙明冷笑道︰「越是自以爲是君子之人,越是會胡亂怪人。」她頓了一頓,笑道︰「快去救你的周姑娘吧,别要落在宋青書手裡,你可糟糕啊。」無忌又是臉上一紅,道︰「我爲什麼糟糕?」

兩人循著雪中馬蹄的足跡,找到了坐騎,直奔関内。無忌既記掛義父,又想念周芷若,但想丐幫要利用義父來挾制明教,義父如確是落入丐幫手中,當不致對他有所損傷,只是屈辱難免,但芷若冰清玉潔、溫婉賢淑,遇上了陳友諒之奸詐、宋青書之無恥,若遇逼迫,定然難免一死。言念及此,恨不得插翅飛到盧龍。當晩兩人在一家小客店中宿歇,兩匹馬雖是駿馬,但不停蹄的奔馳了大半日,已是疲累不堪,到得客店之中,草料也不肯吃了。無忌躺在炕卜,越想越是擔心,悄悄到趙明窗外一聽,但聽她呼吸調勻,正自香夢沉酣。無忌微一沉吟,到櫃上取過筆硯,撕下一頁帳簿,草草留書,説著事在緊急,決意連夜趕路,事成之後,當謀良唔。將那頁帳簿用石硯壓在桌上,輕輕躍出窗外,展開輕功,向南疾馳而去。

如此晩間以輕功疾追,日間則購買騾馬代步,不數日間已到了盧龍。雖然連日未得安睡,但他内力悠長,竟是並不如何疲累。只是如此快追,按理應當在中途追上陳友諒和宋青書,但一直未曾遇上,想是他晩上趕路之時,陳宋二人和掌砵龍頭正在客店之中睡覺,是以錯過。那盧龍是河北重鎭,唐代爲節度使駐節之地,經宋元之際數度用兵,大受摧破,元氣迄自未復,但仍是人煙稠密,和関外苦寒之地大不相同。無忌走遍盧龍大街小巷,茶樓酒館,説也奇怪,竟是一個乞児也遇不到。無忌心下反喜︰「如此一個大城,街上竟無化子,此事大非尋常。陳友諒説丐幫在此聚會,當非虛言,想是城中大大小小的化子都參見幫主去了。只須尋訪到他們聚會之所,便能探聽到義父和芷若是否被丐幫擒了。」他在城中到處察看,絲毫没有頭緒,又到近郊各處村莊踏勘,仍是不見任何異狀。

到得傍晩,無忌越尋越是焦躁,不由得思念起趙明的好處來︰「若是她在我的身旁,決不致如我這般束手無策。」只得到一家大客中去借宿,用過晩飯後小睡片刻,挨到二更時分,飛身上屋,且看四下裡有何動靜。

他遊目四顧,但見微風動樹,唯聞柝聲處處,更無半點江湖人物聚會的徵象,正煩惱間,忽見東南角上有一座高樓聳起,樓上兀自亮著火光。無忌心想︰「此家若非官宦,便是富紳,和丐幫更牽不上半點干係\dash{}」念頭尚未轉完,突見人影一閃,從樓窗中躍了出來。那人影快速無比,一晃之間,已自隱没,若不是無忌目光敏鋭異常,決計難以發見。他心道︰「莫非有綠林豪客到這大戸人家去做案嗎?這人身法好快,直是第一流的高手。左右無事,便去瞧瞧無妨。」

當下四五個起落,已奔到了那巨宅之旁。張無忌雙足一點,身子如一鶴沖天,翻過了圍牆,突然眼前一亮,只聽得一人聲音説道︰「陳長老也忒煞多事,明明言定正月初八大夥在老河口聚集,却又急足快報,傳下訊來,要咱們在此等候。他又不是幫主,説什麼便得怎麼,當眞豈有此理。」無忌一聽之下,心中大喜,聽這聲音好熟,正是丐幫中人。

那聲音是從靠花園的花廳中傳出,張無忌悄悄掩近,只聽聽得丐幫幫主史火龍説道︰「陳長老足智多謀,他能將武林中尋覓了二十餘年的金毛獅王謝遜擒拿到手,别説本幫無人能及,武林之中,又有那一人能彀辦到\dash{}」無忌又驚又喜,知道義父確是落入了丐幫手中,既是有了著落,只須設法營救便是,丐幫中也無如何了不起的高手,相救義父,當非難事,於是湊眼到長窗縫邉,向裡張望。只見史火龍居中而坐,傳功、執法二長老掌棒龍頭及三位八袋長老坐在下首,還有一個衣袖飾華麗的中年胖子,穿著形貌活脱是個富紳,但背上却也負著六隻布袋。無忌暗暗點頭︰「是了,原來盧龍有一位大財主也是丐幫弟子。叫化子在大財主屋裡聚會,那確是誰也想不到的了。」

只聽史火龍接著説道︰「陳長老既然傳來急訊,要咱們在盧龍相候,定有他的道理。咱們圖謀大事,務當小心謹愼。」掌棒龍頭道︰「幫主明鑒︰江湖上群豪尋覓謝遜,爲的是要奪取武林至尊的屠龍寶刀。現下這把寶刀既不在謝遜身上,不論怎麼軟騙硬嚇,他終是不肯吐露寶刀的所在。咱們徒然得到了一個瞎子,除了請他喝酒吃飯,又有何用?依弟子説,不如給他上些酷刑,瞧他説是不説。」史火龍搖手道︰「不妥,不妥,硬功夫説不定反而壞事。咱們等陳長老到後,再行從長計議。」掌棒龍頭臉有憤憤不平之色,似怪幫主什麼事都聽陳友諒的主張。

史火龍取出一封信來,交給掌棒龍頭,説道︰「馮兄弟,你立刻動身前赴濠州,將我這封信交給韓山童,説他児子在我們這裡,平安無事,只須他投誠本幫,幫主自對他另眼相看。」掌棒龍頭道︰「這送信的小事,似乎不必由弟子親自走這一趟吧?」史火龍臉色微沉,説道︰「馮兄弟,這半年來韓山童等一夥明教人馬,在濠泗一帶鬧得好生興旺。聽説他手下的朱元璋、徐達、常遇春一干人,頗爲英雄了得。送這封信去,乃是要韓山童歸附本幫,一來馮兄弟須得善下説詞,察看他的歸附是眞情還是假意,二來是探聽這一路明教人馬的虛實。馮兄弟肩上的擔子非輕。怎能説是小事?」掌棒龍頭不敢再説什麼,只道︰「謹遵幫主吩咐。」向史火龍行禮,出廳而去。

張無忌再聽下去,只聽他們儘説些日後明教、少林、武當、峨嵋各派歸附之後,丐幫將如何興盛威風,這史火龍的野心,反不及陳友諒之大,聽他言中之意,只須丐幫獨霸江湖,稱雄武林,便已心滿意足,却没想到要得江山、做皇帝。無忌聽了一會,有些厭了,心想︰「看來義父和芷若便被囚在此處,我先去救了他們出來,再將這些大言不慚的乞児懲誡一番。」右足一點,身子如一溜輕煙,上了一株高樹,縱目四下張望,只須見何處丐幫弟子守衛戒備最是嚴密,料想便是囚禁謝遜和周芷若之所。他東西一看,立即便見那高樓下有十來名丐幫弟子手執兵刃,來往巡邏。

張無忌輕輕躍下樹來,掩近高樓,躱在一塊大太湖石之後,待兩名巡邏的丐幫弟子轉身行開,他身子橫射數丈,已竄到樓底的牆脚,施展「壁虎遊牆功」,神不知鬼不覺的便遊了上去。但見樓上燈燭甚亮,他在尋到謝遜和周芷若之前,不願大加驚動,伏身窗外,偸聽房内動靜。

聽了片刻,樓房内竟是半點聲息也無。無忌好生奇怪︰「怎麼一個人也没有?難道竟有高手暗伏在此,能長時間閉住呼吸?」又聽一會,仍是聽不到呼吸之聲,他探身到窗縫中一張,只見桌上一對大臘燭已點去了大半截,室中却無人影。

樓上並排三房,張無忌見東廂房中無人,又到西廂房窗外一張。房中燈火明亮,桌上杯盤狼藉,放著七八人的碗筷,但杯中殘酒未乾,菜肴初動,仍是寂無一人,這些人似乎吃喝未久,便即離房他去。中間房中却是黑洞洞地並無燈光。無忌輕推房門,裡面上著門閂,無忌低聲叫道︰「義父,你在這児麼?」並不聽見房中有人答應。他心想︰「看來義父不在此處,但丐幫人衆如此嚴密戒備,却是爲何?難道有意的實者虛之,虛者實之嗎?」突然之間,鼻中隱隱聞到一陣血腥之氣,從中間房中傳了出來。無忌心頭一驚,左手按在門上,内力微震,格的一聲輕響,門閂從中斷截。無忌立即閃身進房,接住了兩截斷折的門閂,以免掉落地下,發出聲響。

他只跨出一步,脚下便是一絆,相觸處軟軟綿綿地,似是人身。他俯身一摸,却是個屍體。這人氣息早絶,臉上兀自微溫,顯是死去未久。無忌一摸之下,察覺此人,小頭尖腮,並非謝遜,當即放心。但跨出一步,又踏到了兩人的屍身。無忌指住西邉皮壁一戳,刺出兩個小孔,燭光從孔中透了過來。只見地下橫七豎八,躺滿了屍體,盡是丐幫弟子。瞧這些人死去的模樣,都是受了極重的内傷。他提起一屍,撕開衣衫,但見那人胸口拳印宛然,正是七傷拳拳力所傷。無忌大喜︰「原來義父大展神威,擊斃看守人衆,殺出去了。」在房中四下一看,果見牆角上用尖利之物刻著一個火燄的圖形,正是明教的記號。無忌又想︰「不知義父如何會被丐幫所擒?想是他老人家目不見物,難以提防丐幫的詭計。他們若非用蒙汗藥物,便是用絆馬索、倒鉤、漁網之類物事擒他。但門閂在裡倒閂,他又如何出去?這倒奇了。」

一回頭,只見門背板上噴著一灘鮮血,門板外側却淺淺留著一個掌印。無忌微一沉吟,已知其中之理︰「義父留著一人不殺,自己出房之後,叫那人閂上房門,隨即以七傷拳的勁力用在掌上,隔著門板將那人震死。只因隔了一塊門板,掌力猛而不純,以致那人口吐鮮血,是了,適纔我見這樓上有黑影一閃,便是義父脱身而去了\dash{}」但隨即心想︰「那黑影縱躍雖快,但矮短瘦小,決非義父魁梧的身材,此人是誰?」

他走出房外,縮身在門邉向下一張,見衆丐兀自鄭重其事的來回巡邏,對樓上變故全不知情。無忌尋思︰「衆惡丐死去未久,義父定是去得不遠。我何必胡思亂想,只須追上義父,一問便知。咱爺児倆回轉身來,鬧他個天翻地覆,這時方教群丐知我明教手段。」思念及此,豪氣勃發,適纔見那黑影從西南方而去,當下縱身躍起,在一株高樹上一點,一借力,已躍上圍牆的牆頭,俯身査看,果見牆頭轉角處有個纖細的足印,顯是女子所留。無忌好生訝異︰「如此説來,適纔所見的黑影,却是個女子了。武林之中,又有誰有這等高強的輕身功夫?滅絶師太已死,紫衫龍王遠去異國。崑崙派的班淑嫻未必有此功夫,芷若和趙姑娘更是不及,此外再不足論。」但這時深恐追不上謝遜,無暇多想,提氣向西南方疾馳而去。

沿著大路追出數里,來到一處岔道,在樹根草叢中一尋,只見一塊岩石後畫著一個火燄記號,指向西南的小路。無忌大喜,心想義父行蹤已明。立時便可相見。明教中各種聯絡指引的暗號,他曾聽楊逍詳細説過,又見這火燄記號雖是寥寥數劃,但鉤劃蒼勁,顯是出於非常人的手筆,若非謝遜這等文武全才之士,明教中没幾人能畫得出來。

\chapter{獨闖丐幫}

此時張無忌更無懷疑,沿著小路追了下去,直追到沙河驛,天已黎明,在飯店中胡亂買些饅頭麵餅充飢,更向西行,到了棒子鎭上。只見街角牆脚下繪著一個火燄記號,指向一所破破爛爛的祠堂。無忌大喜,心想義父定是藏身其間,走近一看,見匾額上冩著「魏氏宗祠」四個大字。一走進門,只聽得一陣呼{\upstsl{吆}}喝六之聲,大廳上圍著一群潑皮和破落子弟,正自入局賭博,却是個賭場。

賭場莊頭見張無忌衣飾華貴,只道是位大豪客來了,忙笑吟吟的迎將上來,説道︰「公子爺快來擲兩手,你手氣好,殺他三個連莊。」轉頭向衆賭客道︰「快讓位給公子爺,大夥児端定銀子輸錢,可給公子爺雙手捧回府去啊!」張無忌眉頭一皺,見衆賭客中並無江湖人物,提聲叫道︰「義父,義父,你老人家在這児嗎?」隔了一會,不聽有人回答,他又問了幾句。一個潑皮見他不來賭博,却來大呼小叫的擾局,當即應道︰「乖孩児,我老人家在這児,你快快來擲骰子啊。」衆潑皮一聽,登時哄堂大笑。張無忌問那莊頭道︰「你可曾見一位黃頭髮、高身材的大爺進來,是一位雙目失明的大爺?」那莊頭見他不來聚賭,却是來尋人,心中登時淡了,笑道︰「笑話奇談,天下竟有瞎子來賭骰子的麼?除非這瞎子活得不耐煩了。」張無忌追尋義父不見,心中已没好氣,又聽這莊頭和那潑皮出言不遜,辱及義父,一怒之下,踏上一步,一手一個,將那莊頭和潑皮抓了起來,雙手輕輕一送,將二人擲上了屋頂。這二人雖未受傷,却已嚇得殺豬似的大叫起來。無忌推開衆人,拿起賭台上的兩錠大銀,説道︰「大爺借去使使。」揣在懷内,大踏步走出祠堂。衆潑皮驚得呆了,誰敢來追?

他續向西行,不久却又見到了火燄記號。傍晩時分到了豐潤,那是冀北的一座大城。無忌依著記號所指,尋到一處粉牆黑門之外。但見門上銅環擦得晶亮,牆内梅花半開,却是一家幽雅精潔的人家,他拿起門環,輕輕敲了三下。不久脚步細碎,黑門呀的一聲開了,鼻中先聞到一陣濃香,只見應門的是個身穿粉紅皮襖的小鬟,抿嘴一笑,説道︰「公子爺這久不來啦,姐姐想得你好苦,快進來喝茶。」説著又是一笑,向他抛了個媚眼。

張無忌猶如丈二金剛,摸不著頭腦,問道︰「你怎麼識得我?你姊姊是誰?」那小鬟笑道︰「你是明知故問,還假惺惺作態呢,快來吧,别讓我姊姊牽肚掛腸啦。」一伸手,已握住了無忌的右手,引著他向内走去。無忌心下大奇︰「怎地她跟我一見如故?」轉念一想︰「啊,是了,想必芷若寄身此間,知我日内必定循著記號尋來,命這小鬟日夜應門。唉,多日不見,芷若原是牽肚掛腸,想得我苦。」他心中一陣溫馨,便隨著那小鬟,曲曲折折經過一條鵞卵石舖的小徑,穿過一處院落,來到一間廂房之中。只聽得簷間一隻鸚哥尖起嗓子,叫道︰「情哥哥來啦,姊姊,情哥哥來啦。」無忌臉上一紅,心想︰「連鸚哥児也知道了。」

只見房中椅上都舖著錦墊,炭火熊熊,烘得一室皆春,小几上點著一爐香,旁邉放著一張瑤琴。那小鬟轉身出去,不久托著一隻盤子進來,盤中六色果子細點,一壼清茶。那小鬟款款的斟了茶,遞在無忌手中,却在休手腕上輕輕捏了一把。無忌眉頭一皺,心想︰「這丫頭怎地如此輕狂?」礙著周芷若面子,却也不好説她,問道︰「謝老爺子呢?周姑娘在那裡?」那小鬟笑道︰「你問謝老爺子幹麼?喝乾醋麼?我姊姊就來啦,瞧你這急色児的模樣,你啊,好没良心,到咱們這児,心上却又牽掛著什麼周姑娘、王姑娘的。」無忌一怔,道︰「你滿口胡言亂語,瞎扯些什麼?」

那小鬟又是抿嘴一笑,轉身出去。過了一會,只聽得環珮丁冬,帷子掀開,那小鬟扶了一個二十三四歳的女子進來。只見她膚色白膩,眉毛彎彎,却也頗具姿色,右邉嘴角上點著一粒風流痣,眼波盈盈,欲語先笑,體態婀娜,嬝嬝婷婷的迎了上來。無忌只覺濃香襲人,心下甚不自在,只聽那女子道︰「相公貴姓?今児有閒來坐坐,小女子眞是好大的面子。」一面説,左手便搭到了無忌的肩頭。

無忌滿臉通紅,急忙避開,説道︰「賤姓張。有一位謝老爺子和一位姓周的姑娘,可是在這児麼?」那女子笑道︰「這児是梨香院啊,你要找周纖纖,該上碧桃居去。你給那一個小妮子迷得失了魂。上梨香院來找周纖纖了?嘻嘻!」無忌一聽,這纔恍然大悟,原來此處竟是所妓院,身子一閃,便即出門。那小鬟追了出來,叫道︰「公子爺,我家姐姐那一點児比不上周纖纖了?你便一刻児也坐不得。」無忌連連搖手,摸出一錠從賭場搶來的銀子往地下一擲,飛步出門。

這麼一鬧,心神半晌不得寧定,眼見天色將黑,夜晩間只怕錯過了路旁的火燄記號,便向一家客店借宿,用過晩飯後,躺在床上思潮起伏︰「義父怎地又去賭場,又去妓院?他老人家此舉,中間到底含著什麼深意?」朦朦朧朧的和衣躺在床上,睡到中夜,突然間在睡夢中驚醒︰「義父雙目失明,怎能一路上清清楚楚的留下這許多記號?難道是芷若從旁指引?還是敵人故意假冒本教的記號,戲弄於我?甚至是引我入伏?哼,便是龍潭虎穴,好歹也要闖他一闖。」

次晨起身,在豐潤城外又找到了火燄記號,仍是指向西方。張無忌行到午後,到了玉田,只見那記號指向一家大戸人家。這一家門外懸燈結綵,正做喜事,大門外貼著「之子於歸」的紅字,看來是人家嫁女,鑼鼓吹打,賀客盈門,正是三朝回門。無忌這一次學了乖,不敢直入打聽謝遜的下落,混在賀客群中一看,未見異狀,便即出來找尋記號,果在一株大樹旁又找到了火燄的記號。

話休絮煩,那記號引著他自玉田而至三河,更折而向南,直至香河。此時無忌已然想到︰「這多半是丐幫發見了我的蹤跡,使個調虎離山之計,將我遠遠引開,以便自行幹那陰毒的勾當。」他心中雖然焦急,却又不敢不順著記號而行,只怕那記號確是謝遜和周芷若所留。倘若他們正受厲害敵人追擊,一路奔逃,一路留下記號,只盼自己前去救援,自己若是自作聰明,逕返盧龍,義父和芷若竟爾因此遇難,那可如何是好。他心中打定了主意︰「事已至此,我只有跟著這火燄記號,追他個水落石出。」他自香河而寶城,再向大白莊、潘莊,已是趨向東南,再到寧河,更向北行,經豐南、開平、雷莊果然引著他奔馳數日,兜了一個大圏子,重行回行盧龍。

無忌一回盧龍,心下反而平靜,暗想︰「敵人若是引我千里萬里的出去,直到廣東、廣西、貴州,那便如何?幸好是重回盧龍。今日不再暗訪,却是明査,總要著落在這群惡丐身上,要他們交出義父和芷若出來。」當下在酒樓中飽餐了一頓,在故衣店買了一件白色長袍,借了硃筆,在白袍上畫了一個極大的火燄,決意堂堂正正,以明教教主身份,硬闖丐幫總堂。

他換上白袍,大踏步走到那財主巨宅門前,只見兩扇巨大的朱門緊緊閉著,門上碗口大的銅釘閃閃發亮。無忌一掌推出,砰的一聲,兩扇大門飛了起來,直向院子中跌了進去,{\upstsl{乒}}{\upstsl{乒}}{\upstsl{乓}}{\upstsl{乓}}一陣響喨,兩隻大金魚缸打得粉碎。

這數日之中,張無忌既掛令義父和周芷若的安危,又是連遭戲弄,在冀北大繞圏子,這時回到丐幫總舵,決意大鬧一場,一出胸中的怒氣。他一掌劈破大門,大踏步走了進去,舌綻春雷,喝道︰「丐幫衆人聽了,快叫史火龍出來見我。」

院子中站著丐幫的十多名四袋弟子、五袋弟子,見兩扇大門陡然飛起,已是大吃一驚,又見一個白衣少年闖進,登時有七八人同聲呼喝,迎上攔住,紛紛叫道︰「什麼人大呼小叫,到這裡撒野?」張無忌雙臂一振,那七八名丐幫弟子猶如七八綑稲草一般,砰砰連聲,直摔了出去,只撞得一排長窗,盡皆稀爛。無忌穿過大廳,砰的一掌,又撞飛了中門,見中廳上排著筵席,史火龍居中而坐。一干丐幫首領剛聽得大門口喧嘩之聲,正派人出來査詢。那知無忌來得好快,半路上迎住匆匆出來査問的七袋弟子,劈胸截住,便向史火龍擲了過去。

那財主模樣的主人坐在下首,一見那七袋弟子向席上飛來,伸出雙臂,往那人身上抱去。這抱抱個正著,但覺這股勁力排山倒海般撞到,脚下急使「千斤墜」,要待穩住身形,終是這股撞來的力道太強,登登登連退七八步,背心靠在大柱之上,這纔停住。這麼一來,群丐無不駭然,要知那七袋弟子武功甚是不弱,却被來人要抓便抓,手到擒來。那財主武功高強,可是連接個人都接不住,簡直是匪夷所思。

群丐固是驚詫,不料無忌更是驚喜交集,原來那圓桌左首,赫然是周芷若和宋青書二人並肩坐著。無忌一時摸不著頭腦,呆呆望著周芷若,説不出話來。周芷若驚呼一聲︰「無忌哥哥!」站起身來,身子一晃,便委頓在地。無忌吃了一驚,搶上前去,俯身抱起。他身子尚未挺直,背上拍的一聲,砰的一響,已被宋青書擊了一掌,再被另外一名丐幫高手打了一拳。張無忌此時九陽神功早已運遍全身,這一掌一拳打在他背上,掌力拳力盡數卸去。他抱起周芷若,縱身躍回院子,問道︰「我義父呢?」周芷若道︰「我\dash{}我\dash{}」無忌道︰「他老人家可好麼?」周芷若道︰「我被他們點中了穴道,半點功夫也没有了。」無忌只是関心謝遜,又問︰「我義父呢?」周芷若道︰「不知道啊,我被他們擒來此處,一直不知義父他老人家的下落。」無忌在她腰腿関節上推拿了幾下,將她放在地下。那知周芷若被點中穴道的手法,似是丐幫特有的功夫,無忌這兩下推拿竟不奏效。她雙足著地,却無法站直,兩膝一軟,便即坐倒。

群丐紛紛離座,走到階前。史火龍抱拳道︰「閣下便是明教張教主了?」張無忌心想他是一幫之主,倒是不可失了禮數,當下抱拳還禮,説道︰「不敢。在下擅闖貴幫總舵,還乞史幫主恕過無禮之罪。」史火龍道︰「張教主近年來名震江湖,在下仰慕得緊,今日得見尊駕身手,果然名下無虛,佩服佩服。」張無忌道︰「在下來得魯莽,倒讓史幫主見笑了。我義父金毛獅王現居何處,請史幫主請他老人家出來相見。」

史火龍臉上一紅,哈哈一笑,説道︰「張教主年紀輕輕,説話却是如此陰損。咱們好意請謝獅王來敝處盤桓數日,那知獅王不告而别,還下重手傷了敝幫八名弟子,這筆帳不知如何算法?却要請張教主示下。」無忌一怔,心想︰「那八名丐幫弟子,果是我義父用七傷拳所殺。看來他老人家確已不在此間,但到了何處呢?」便道︰「這位周姑娘呢?她什麼地方得罪了貴幫,却將她囚禁在此?」史火龍笑道︰「人道明教張無忌武功雖強,却是個蠻不講理的魔頭\dash{}哈哈\dash{}」

張無忌沉著臉道︰「怎樣?」史火龍道︰「今日一見,果然是樹的影児,人的名児,半點也不錯。」張無忌道︰「我怎樣蠻不講理了?」史火龍道︰「這位周姑娘乃峨嵋派的掌門,她是名門正派的領袖人物,跟貴教旁門左道之士,有什麼干係?這位宋青書兄弟,是武當派後起之秀。他和周姑娘郎才女貌,珠聯璧合,當眞是門當戸對,一對兩好,二人擕手路過此間,丐幫邀他二位作客,共飲一杯,何以明教教主來橫加干預?眞是好笑啊好笑!」群丐聽幫主如此説,都隨聲附和,哈哈大笑起來。

張無忌道︰「若説周姑娘是你們客人,何以你們又點了她的穴道,使她無法站直?」史火龍一愕之間,一時爲之語塞。陳友諒哈哈一笑,説道︰「周姑娘一直好好的在此飲酒,談笑自若,誰説是點了她的穴道?丐幫和峨嵋派淵源極深,世代交好。峨嵋派創派師祖郭女俠,是敝幫上代黃幫主的親生女児,敝幫上代耶律幫主,是郭女俠的親姊夫。武林中若非乳臭小児的無知之輩,這些史實總該知曉。咱們丐幫豈能得罪現任峨嵋派的掌門?張教主信口雌黃,怎不讓天下英雄恥笑?」張無忌冷笑道︰「如此説來,周姑娘是自己點了自己穴道?」陳友諒道︰「那也未必,這児人人親眼目睹,張教主飛蹤過來,強加非禮,一把將周姑娘抱了過去。周姑娘掙扎不服,尊駕自是順手點了她的穴道。張教主,雖説英雄難過美人関,好色之心,人皆有之,可是這般大庭廣衆之間,衆目睽睽之下,張教主這等急色舉動,不是太失自己身份了麼?」

張無忌口才本是遠遠不及陳友諒,被他這麼反咬一口,急怒之下,更是難以分辯,只氣得臉色鐵青,喝道︰「如此説來,你們定是不肯以在下義父的行蹤相告了?」陳友諒道︰「張教主,貴教光明使者楊逍,當年姦勢峨嵋派紀曉芙女俠,引得天人共憤。你别恃武功高強,又來幹這種卑鄙齷齪的勾當。在下是良言相勸,聽不聽由你。」張無忌對周芷若道︰「芷若,你倒説一聲,他們如何擄劫你來此處?」周芷若道︰「我\dash{}我\dash{}我\dash{}」連説了三個「我」字,忽爾身子一斜,暈了過去。

群丐紛紛鼓噪,叫道︰「明教魔頭殺了人啦!」

\qyh{}張無忌逼姦不遂,害死了峨嵋派的掌門!」

\qyh{}殺了淫賊張無忌,爲天下除害。」無忌大怒,踏步而前,便向史火龍衝了過去,心想︰「擒賊先擒王,只要抓住了史火龍,好歹著落在他身上,逼問出我義父的下落。」他向前只衝出兩步,掌棒龍頭和執法長老雙雙攔在他的身前。掌棒龍頭一棒橫掃,執法長老右手鋼鉤,左手鐵拐,兩個人三件兵刃,同時向無忌招呼了過來。

張無忌一聲清嘯,乾坤大挪移心法使出,叮{\upstsl{噹}}一聲響,執法長老右手的鋼鉤格開了掌棒龍頭的鐵棒,左手單拐向他脅下{\upstsl{砸}}了過去。旁邉傳功長老長劍遞出,叫道︰「這小子武功怪異,人人小心了。」刷刷刷三劍,吐勢如虹,連指無忌胸口小腹三處要穴。張無忌見他招數凌厲,叫道︰「好劍法。」側身避開,左手食指點向他大腿「環跳穴」。傳功長老長劍圏轉,劍尖對準張無忌指尖,直戳了過去。這一下變招既快,劍尖所指,更是不差厘毫,單此一劍,已是武林中罕見的高招。張無忌心中暗讚︰「丐幫名揚江湖,百年不衰,幫中臥虎藏龍,果是有傑出的人材。」那日在彌勒廟中,他曾見玄冥二老和丐幫高手交戰,只是身藏柏樹之中,不敢探首觀看,所見不切,此刻親自交手,才知傳功、執法兩位長老,足可列名當世一流高手而無愧色。掌棒龍頭火候較淺,却也只見稍遜一籌而已。

瞬息之間,丐幫三老已和張無忌拆了二十餘招。陳友諒突然高聲叫道︰「擺殺狗陣!」群丐荷荷高呼,刀光似雪,二十一名丐幫高手,各執彎刀,將張無忌圍在垓心。這二十一人或口唱蓮花落,或呻吟呼痛,或高叫︰「老爺、太太、施捨口飯!」或伸拳猛擊自己胸口。張無忌先是一怔,但隨即明白,這些古怪的舉動,均是旨在擾亂敵人心神,只見群丐脚步錯雜,然進退趨避,均是嚴謹有法。

傳功長老喝道︰「且住!」向後退了兩步,橫劍當胸,執法長老和掌棒龍頭也各躍開,那些排成「殺狗陣」的群丐,却仍是奔躍來去,絲毫不停。傳功長老説道︰「張教主,咱們以衆欺寡,原是勝之不武,但丐幫中任何一人,均非張教主對手。除奸殺賊,可顧不得俠義道中單打獨鬥的規矩了。」張無忌微微一笑,道︰「好説,好説。」傳功長老又道︰「咱們人人均有兵刃,張教主却是空手,丐幫所佔便宜,未免太多。張教主要什麼兵刃,儘管吩咐,咱們自當遵命奉上。」張無忌心想︰「這位傳功長老武功既高,人也仗義,與陳友諒這干人倒是頗有不同。」當下説道︰「丐幫之中,並無在下合手的兵刃。跟各位玩玩,又何必掄刀動杖?在下要用兵刃,自己不會取麼?」

他説到此處,身形一晃,已從殺狗陣中閃了出去,雙手分在陳友諒與宋青書二人肩頭一按,夾手奪了二人手中長劍,側身斜退,又回入陣中。他一出一入,二十一名幫衆竟未碰到他一片衣角。群丐正自駭然,只聽張無忌朗聲説道︰「貴幫幹慣了偸雞摸狗的勾當,這『殺狗陣』的名,取得甚好。只是殺狗容易,要想降龍伏虎,此陣便不管用。」説著雙劍一振,一股勁力傳到劍身之上,但聽得喇喀兩響,雙劍從中折斷。掌棒龍頭大呼︰「大夥児上啊。」張無忌一聲清嘯,向左一衝,身子却向右斜了出去,乾坤大挪移手法使將出來,但見白光連連閃動,{\upstsl{噗}}{\upstsl{噗}}之聲不絶,殺狗陣群丐手中的彎刀,都被無忌奪下抛上,一柄柄都插在大廳中間的正樑之上。二十一柄彎刀整整齊齊的列成一排,每把刀都没入木中尺許。

猛聽得陳友諒叫道︰「張無忌,你還不住手?」無忌一回頭,只見陳友諒手中另執柄長劍,劍尖指在周芷若的後心,這一來投鼠忌器,登時受了挾制。張無忌冷笑道︰「百年來江湖上都説『明教、丐幫、少林派』,各教以明教居首,各幫推丐幫爲尊,各位如此作爲,也不怕辱没了洪七公老俠的威名?」傳功長老怒道︰「陳長老,你放開周姑娘,咱們再跟張教主決一死戰。丐幫傾全幫之力,拾奪不下明教的孤身一人,咱們大夥児還有臉面做人麼?」陳友諒笑道︰「大丈夫寧鬥智,不鬥力。張無忌,你還不束手待縛?」張無忌大笑道︰「也罷!今日教張無忌見識了丐幫的威風。」突然間倒退兩步,向後一個空心觔斗,凌空落下,雙足騎在丐幫幫主史火龍的肩頭。他右掌平放在史火龍的頂門,左手拿住了他後頸的經脈。

這一招聖火令上所載武功,竟是如此輕輕易易的得手,連無忌自己也是頗出意料之外。他原意是使一招怪招、出其不意的欺近史火龍,心中早已計算好三下極厲害的後著,要快如閃電的將史火龍擒拿過來,不讓他有動念的時機,須知陳友諒心狠手辣,説不定眞的會向周芷若猛下毒手,那知他心中想好的這三招乾坤大挪移厲害殺手,竟是一招也使不上,史火龍不經招架,便已被擒。無忌騎在史火龍肩頭,猶如児童與大人戲耍一般,形相甚是不雅,但既已制住對方頂門要穴,却也不願縱身下地,以致另生波折。

群丐見幫主被擒,齊聲驚呼。張無忌右手手掌平平按住史火龍頂門的「百會穴」上,那「百會穴」是足太陽經和督脈之交會,最是人身大穴,他掌力只須輕輕一吐,史火龍立時經脈震斷而斃,無藥可救。群丐雖然驚惶,却是誰也不敢動彈。一陣呼喝過後,大廳上突然間一片寂靜,人人睜大雙眼望著張無忌和史火龍,不知如何是好。

正在此時忽聽得屋頂上傳下來輕輕數響琴簫和鳴之聲,似是有數具瑤琴,數枝洞簫同時奏鳴。這樂聲飄渺宛轉,若有若無,但人人聽得十分清楚,只是忽東忽西,不知是從屋頂的那一方傳來。無忌大奇,心中連轉幾個念頭,想不起這琴簫之聲是何含意。忽聽得陳友諒朗聲説道︰「何方高人駕臨丐幫?若是明教的群魔,不妨就此現身,何必裝神弄鬼?」瑤琴聲錚錚錚連響三下,忽見四名白衣少女從東南簷上飄然落入庭中,每人手中都抱著一具瑤琴比尋常的七絃琴短了一半,窄了一半,但具體而微,也是七絃齊備。四名少女落下後分站庭中四方,跟著門外又走進四名黑衣少女,每人手中各執一枝黑色長簫,這簫却比常見的洞簫長了一半。這四名黑衣少女也是分站四角。四白四黑,交叉而立。張無忌於四象八卦的易理所知無多,但見這八個少女所佔方位正八卦不像正八卦,倒八卦不像倒八卦,似乎站得完全錯了,但八人齊錯,中間隱隱又似有脈絡可尋。

八女站定方位,四具瑤琴上響起樂調,接著洞簫加入合奏,樂音極盡柔和幽雅。張無忌雖是不懂音樂,但覺這樂聲溫柔和平,雖處這般極緊張的局面,也願多聽一刻。悠揚的樂聲之中,緩步走進一位身披淡黃輕紗的女子,左手擕著一個十二三歳的女童。那女子約莫二十七八歳年紀,風姿綽約,容貌極美,只是臉色太過蒼白,竟無半點血色。那女童却相貌醜陋,鼻孔朝天,一張闊口,露出兩個大大的門牙,直有兇惡之態。她一手拉著那個美女,另一手却持一根青竹棒。

群丐一見這兩個女子進來,目光不約的集中在那根青竹棒上。張無忌見了這許多女子,自覺仍是騎在史火龍肩頭,未免太過児戲,但陳友諒的劍尖不離周芷若後心,自己又不能輕易放開史火龍。他見群丐人人目不轉睛的瞪著那女童手中的竹棒,似乎天下之大,唯有這根竹棒纔是第一要緊的物事,甚麼白衣少女、黑衣少女、黃衫美女,以及這個醜女童本人,誰都是對之視若無物。無忌暗暗詫異,打量這根竹棒時,只見那棒児通體碧綠,精光溜滑,不知多少年來經過多少人的摩挲把弄,但除此之外,却也别無異處。

那黃衫美女目光一轉,猶似兩道冷電,閃過了大廳上衆人,最後這目光停留在張無忌臉上,冷冰冰的道︰「張教主,你年紀也不少了,正經事不幹,却在這児胡鬧。」這幾句話中微含責備之意,但詞語頗爲親切,猶似長姊教訓幼弟一般。張無忌臉上一紅,分辯道︰「丐幫的陳長老以卑鄙手段,制住我的\dash{}我的同伴,我只好擒住他們幫主。」那美女微微一笑,道︰「將人家幫主當馬騎,不是太過份了一點嗎?我從長安來,道上聽人説明教教主是個小魔頭,今日一見,唉,唉!」説著臻首輕搖,頗有不以爲然的神色,史火龍突然叫起來,「張無忌你這小淫賊,快快下來!」想伸手去扳無忌的腿,可是苦於後頸經脈被拿,混身半點勁道也使不出來。

無忌聽他當著婦道人家的面,斥責自己爲「小淫賊」,又羞又怒,左手一股内力從他後頸透了過去,史火龍全身酸麻難當,忍不住大聲「啊喲,啊喲」的呻吟起來。

群丐見張無忌如此無禮,而且自己幫主却又是如此孱弱,無不憤怒,均覺史火龍在敵人手下居然出聲呻吟,實是大失英雄好漢的身份,别説他是江湖上第一大幫主,便是尋常一個丐幫弟子,也不該對敵人低頭示弱。

陳友諒道︰「張無忌,你放開咱們史幫主,我便收劍如何?」他不等對方答應,當即還劍入鞘。他料得無忌不是言而無信之人,這一著必可收效,果然無忌説道︰「甚好。」身形一晃,已站在周芷若身邉,但是她雙眉深鎖,神情委頓,甫自昏暈中醒轉,不由得甚是憐惜,扶起她身子,坐在庭中一張石鼓凳上。

陳友諒轉向那黃衫美女,拱手説道︰「芳駕惠臨敝幫,不知有何教言?尊姓大名,可得見示否?」他見這黃衫女子年紀已然不小,但仍是穿著未嫁人的閨女衣飾,八女前導,氣派非凡,心中苦苦思索,實想不起武林中有這麼一號人物,而那醜陋女童手中所持的這根綠竹棒,宛如便是丐幫幫主的信物打狗棒,更不知何以會在她手中。

那黃衫美女冷冷的道︰「混元霹靂手成崑在那裡?請他出來相見。」張無忌聽到「混元霹靂手成崑」七字,心下大奇,却見陳友諒臉上陡然變色。但他臉色迅即回復,淡淡的道︰「混元霹靂手成崑?那是金毛獅王謝遜的師父啊。你問明教張教主纔是。」黃衫美女冷笑道︰「閣下是誰?」陳友諒道︰「在下姓陳,草字友諒,乃丐幫的八袋長老。」黃衫美女嘴角向史火龍一撇,問道︰「這傢伙是誰?模樣倒是雄糾糾的一副英雄氣槩,怎地如此膿包?給人略加整治,便即大呼小叫,没半點児光棍。」群丐一聽之下,都感臉上無光,心下暗自羞慚,有些人瞧向史火龍的眼色之中,已帶著三分輕蔑,兩分氣惱。陳友諒道︰「這位便是本幫史幫主。他老人家近來大病初愈,身子不適。你遠來是客,咱們讓你三分。若再胡言亂道,得罪莫怪。」説到最後兩句,已是聲色倶厲。

那黃衫美女神色漠然,向站在巽位上的黑衣少女道︰「小翠,先將那封信還了給他。」那黑衣少女小翠應道︰「是!」從懷中取出一封信來,托在手中。無忌目光敏鋭,一瞥之下,已見封板上冩著︰「面陳韓大爺山童親啓。」,另一行冩著四個小字︰「丐幫史緘。」掌棒龍頭一見那信,登時滿臉紫脹,罵道︰「小賤婢,原來途中一再戲弄老子的偸信賊,便是你這死丫頭。」一橫手中鐵棒,便要撲上前去厮拚。小翠格格一笑,説道︰「我丫頭是丫頭,可是没死。這麼大的人,連封信也看不住,不害羞。」説著纖手一揚,那封信平平穩穩的向掌棒龍頭飛來。他二人相隔三丈有餘,一封信飄揚揚的絶無重量,那黑衣少女居然以内力穩穩送至,内功造詣,實是不弱。掌棒龍頭伸手去抓,但説也奇怪,那信距他尚有三尺,突然間一拐彎,轉向左首,{\upstsl{噗}}的一聲,掉在地下,掌棒龍頭這一抓竟是抓了個空。他一愕之下,正要俯身去拾。張無忌衣袖一捲,送出一股勁風,將那信捲了起來,左手乾坤大挪移神功運出,撥動風勢,已將那信取在手中。旁人不明其理,還道他竟有空中取物的法術,盡皆駭然失色。

張無忌那晩曾見史火龍命掌棒龍頭送信去給韓山童,以韓林児爲要挾,脅他歸降丐幫,此時聽了小翠和掌棒龍頭的對答,已是恍然,知道必是那白衣黑衣少女途中戲耍掌棒龍頭,盜了他的書信,以致他迫得重返盧龍。但掌棒龍頭武功精強,聽他言中之意,竟是直至此時,方知戲耍他的人是誰,那麼這些黑白少女不是有過人的機智,便是身具極高的武功,更可能是那黃衫美女暗中主持,將一位丐幫高手耍得團團亂轉。無忌想到此處,不禁對那黃衫女子好生感激。

\chapter{奸謀揭露}

那黃衫女子説道︰「韓山童起義淮泗,驅逐韃子,道路傳言,都説他信仁好義,不擾百姓。既是這麼一位英雄人物,豈能爲了児子而背叛明教,投降丐幫,張教主,你儘管將這信還他。就算將這信送到韓大爺手中,那也是丐幫自討没趣而已。我只是見這位龍頭大哥胡塗得可笑,又因丐幫中有件大事,須他親自在場,纔截下他的信來。」張無忌抱拳道︰「多謝大姊援手相助,張無忌有禮。」黃衫女子還了一禮,道︰「不必客氣。」張無忌右手一揚,將那信向掌棒龍頭擲去,這一揚之後,手上跟著是一股暗勁送出,這暗勁後發先至,反而搶在那信之前兩尺,但旁人不明乾坤大挪移法的神妙,誰都看不出來。

掌棒龍頭伸手正欲去接那信,突然間被一股無影無蹤的暗勁一撞,騰騰騰連退三步,一個踉蹌,險些児摔倒。那信無人來接,便即掉在地下。掌棒龍頭又驚又怒,俯身拾起,罵道︰「是那一個賤婢暗箭傷人,不算好漢。」他還道是那些黑衣白衣少女之中,有人向他施放了一件奇形暗器。

那黃衫女子搖頭道︰「虧你也是丐幫中的一流高手,不識得張教主這『隔山打牛』的神功。」群丐聽了此言,都是一驚,武林中雖然故老相傳,有這麼一路神妙的武功,能運掌力擊傷人,但一向都以爲那是説説罷了,豈知今日親眼目睹,掌棒龍頭爲暗勁擊得連退三步,那決不是故意假裝,自己硬要出醜。黃衫女子又道︰「聰明反爲聰明誤,世事之奇,往往如此。你們以爲挾按韓林児,便能逼迫韓山童投降麼?那日你在道上接連受阻,以爲改行小道,便能避過麼?嘿嘿,就是避過了,這信送到韓山童手中,於你丐幫也無好處。」陳友諒心中一動,夾手搶過那封信來,只見信皮完好無缺,撕開封皮,抽出信箋,一瞥之下,臉色登時大變。原來一封向韓山童招降的信,已變成丐幫向明教投誠的降書,文字中卑躬屈膝,盡極謙抑,務請明教收錄,俾爲驅趕元虜的馬前先行。

黃衫女子冷笑道︰「不錯。這信我是瞧過啦,可不是我改的。我看了此信,才知掌棒龍頭早已著了人家手脚,上了大當。我念著跟丐幫上代的淵源,不願威名赫赫的天下第一大幫,到今日出醜露乖,纔截了下來。你倒想想,此信倘若由丐幫掌棒龍頭親手送到了明教手中,丐幫今後還有顏面立足於江湖之上麼?」傳功長老、執法長老、掌砵龍頭、掌棒龍頭等先後接過信來,一看之下,無不驚怒,心下却又不禁暗叫︰「慚愧!」果如黃衫女子所言,這封卑辭奴言、没半分骨氣的降書一落入明教之手,丐幫醜名揚於天下,所有丐幫弟子,再難在人前直立,如此説來,黃衫女子截下這封書信,那是幫了丐幫一個大忙了。然則偸換書信,却又是何人?

黑衣少女小翠笑道︰「你們想問︰這封信是誰換的,是不是?」群丐不答,但人人臉上均露出急欲知曉的神色。小翠笑道︰「掌棒龍頭,你除下外袍,便知端的。」掌棒龍頭是個直性子之人,雙手拉住外袍兩邉衣襟一扯,{\upstsl{噗}}{\upstsl{噗}}數聲輕響過去,扣子盡數崩斷。他向後一甩,已將外袍丟下,喝道︰「那便怎地?」只聽得他身後群丐齊聲「咦」的驚呼,似乎瞧到了什麼怪異物事。掌棒龍頭道︰「什麼?」轉過身來,只見六七人指著他的背脊。掌棒龍頭更是焦躁,雙手使勁,撕破内衫前襟,將貼肉的内衫除下,露出一身紮纏糾結的肌肉,揮過内衫一瞧,只見衫上用靛青繪著一隻青色的大蝙蝠,蝙蝠口邉,點著幾滴紅色血點。這蝙蝠雙翼大張,睜獰可怖,正是一頭吸血蝙蝠。

傅功長老、執法長老等齊聲説道︰「青翼蝠王韋一笑!」要知韋一笑從前少到中原,聲名不響,但近年來在江湖上神出鬼没,大顯身手,威名之盛,幾乎有蓋過白眉鷹王之勢。無忌見了那蝙蝠,心下暗喜︰「若非韋兄這等來去無影的輕功,原是難以戲弄這掌棒龍頭於掌股之上。」掌棒龍頭一怔,提起那件内衫,劈臉向張無忌打來,罵道︰「好啊,原來是你這些魔崽子戲弄老夫。」無忌衣袖一拂,那内衫被一股勁風帶得冉冉上升,掛在庭中一株極高的銀杏樹椏枝之上,臨風飄揚,衫上那隻吸血大蝙蝠更是栩栩欲活。

陳友諒心知此事越鬧越臭,只有擱下不理,是爲上策。問那黃衫女子道︰「請問姑娘高姓,不知與咱們有何淵源。」黃衫女子冷笑道︰「跟你們有甚麼淵源?我只跟這根打狗棒有些淵源。」説著向醜女童手中的青竹棒一指。群丐均知打狗棒乃幫主信物,却不明何以會落入旁人手中,各人的眼光均瞧著史火龍,但見他臉色慘白,不知所措。傳功長老問道︰「幫主,這女孩拿著的打狗棒,是假的麼?」史火龍道︰「我\dash{}我看多半是假的。」黃衫女子道︰「好,那麼你將眞的打狗棒取將出來,比對比對。」史火龍道︰「打狗棒是丐幫至寶,怎能輕易示人?我也没隨身擕帶,若有失落,豈不糟糕?」群丐一聽,都覺這句話有些不成體統,身爲丐幫幫主,怎會怕打狗棒失落?那女童高舉青竹棒,大聲説道︰「丐幫諸長老、衆弟子看清楚了。這打狗棒乃本幫世代相傳之物,豈是假的?」群丐聽她口稱「本幫」,暗自驚奇,走近細看,見這棒晶潤如玉,堅硬勝鐵,確是打狗棒無疑。各人面面相覷,不明其理。

黃衫女子哈哈的道︰「素聞丐幫幫主以降龍十八掌及打狗棒法二大神功,馳名天下。小虹,你先向史幫主討教討教降龍十八掌的功夫。小玲,你待小虹姊姊勝了之後,再向史幫主討教討教打狗棒法的功夫。」兩名手持長簫的少女應聲躍出,分站左右。陳友諒怒道︰「姑娘不肯見示姓名,已是没將丐幫放在眼中,更令兩名小婢向我們幫主挑戰,江湖上焉有這個道理?史幫主,待弟子先料理了這兩個丫鬟,再來領教這位姑娘的高招。咱們要瞧瞧到底是何方高人,如此輕視丐幫。」史火龍道︰「甚好,就請陳長老下場。」陳友諒刷的一聲拔出長劍,緩步走到中庭。

那小虹道︰「姑娘叫我討教降龍十八掌,你眞的會這路掌法麼?降龍十八掌要用劍麼?」陳友應諒喝道︰「史幫主何等身分,怎能跟你小丫頭動手過招?降龍十八掌的神功,豈是你小丫頭輕易見得的麼?」説著又踏上一步。黃衫女子向張無忌道︰「張教主,我求你一件事,行不行?」張無忌道︰「姑娘請説。」黃衫女子道︰「請你將這姓陳的{\upstsl{攆}}了開去,將那冒充史幫主的大騙子揪將出來。」

張無忌初時一交手便將史火龍擒住,覺得他功夫實在平庸之極,心下早已起疑,又見他事事聽陳友諒指點,自己没半點主意,憑他武功、識見,決不能爲丐幫的幫主,這時聽黃衫女子説他是「冒充幫主的大騙子」,前後一加印證,已自明白了六七成,一點頭,已欺到史火龍身前。史火龍一招「沖天炮」打來、砰的一拳,打在張無忌胸口,張無忌哈哈大笑,説道︰「降龍十八掌神功,是如此膿包嗎?」一伸手,抓住他胸口衣襟,將他提了出來。陳友諒自知非張無忌敵手,不等他動手,自行退入了人叢之中。

突然之間,那醜女童放聲大哭,撲將上來,抓住史火龍亂撕亂打,叫道︰「你害死我爹爹,害死我爹爹,你這個千刀萬剮的惡賊。」抓住他頭髮一扯,史火龍滿頭頭髮,忽然跌落,露出油光晶亮的一個光頭。

原來史火龍竟是個禿頭,頭上戴的是假髮。他被張無忌拿住後心,半點勁道也使不出來。他身材高大,那女童一陣亂打,小拳頭只打到他的肚子。張無忌手臂一拗,將他腦袋掀了下來,那女童亂抓之下,忽然間抓下了他一塊鼻子,却無鮮血流出。衆人驚奇之下,凝目細著,原來他鼻子低塌,那高挺的鼻子也是假裝的。群丐一陣大嘩,齊問︰「你是誰?怎地來冒充史幫主?」無忌提起他身子,重重往下一頓,只摔得他七葷八素,半晌説不出話來,無忌微微一笑,自行退開,心想此人冒充史火龍眞相大白,自有群丐跟他算帳。

掌棒龍頭性如烈火,上前左右開弓,拍拍拍拍的打了七八個重重的耳光,那假幫主雙頰紅腫,大叫︰「不干我事,不干我事。是陳\dash{}陳長老叫我幹的。」執法長老心頭一凜,喝道︰「陳友諒呢?」不料陳友諒一見事情敗露,早已逃之夭夭。執法長老道︰「快追他回來!」早有數名七袋弟子應聲而出,追出門去。

黨棒龍頭罵道︰「直娘賊!你是什麼東西,要老子向你磕頭,叫你幫主。」提起蒲扇大的巴掌,又要往他臉上摑去,執法長老忙伸手格開,説道︰「馮兄弟不可魯莽。你一掌打死了他,什麼事都査不出來了。」他轉身向那黃衫女子抱拳行禮,恭恭敬敬的道︰「若非姑娘拆穿此人奸謀,咱們至今兀自蒙在鼓裡,姑娘芳名可能見示否,敝幫上下,同仰大德。」黃衫女子淡淡一笑,道︰「小女子幽居深山,自來不與外人往還,姓甚名誰,自己也早忘了。至於這一位小妹妹,你們之中難道没一人認得她麼?」群丐瞧著這個女童,没一人認得。傳功長老忽地心念一動,踏上一步,道︰「她\dash{}她\dash{}好像史幫主夫人哪\dash{}莫非\dash{}莫非\dash{}」黃衫女子道︰「不錯,她姓史名紅石,便是史火龍史幫主的獨生愛女。史幫主臨危之時,命他大弟子王嘯天抱了這孩子,擕帶打狗棒前來找我,替他報仇雪恥。只可惜王嘯天苦戰脱力,傷重難治,但終於將這孩子送到了我手裡。」傳功長老道︰「姑\dash{}姑娘!你説史幫主已經歸天了?他\dash{}他老人家是怎麼死的?」

原來史火龍在二十餘年之前,便因苦練降龍十八掌,内力不濟,得了上半身癱瘓之症,雙臂不能動彈,自此擕同妻子,到各處深山尋覓靈藥治病,將丐幫幫務,交與傳功、執法二長老,掌捧、掌砵二龍頭分工處理。只因幫務主持乏人,二長老、二龍頭不相統屬,各管各的。汚衣派和淨衣派又積不相能,以數偌大一個丐幫,漸趨式微。待這假幫主最近突然現身,年輕的丐幫弟子,從未見過幫主,而傳功長老等人,和史火龍一别數十年,見這假幫主相貌依稀相似,誰會想到竟會是假冒的?

黃衫女子嘆了口氣道︰「史幫主是喪在混元霹靂手成崑的手下。」張無忌「咦」了一聲,滿腹疑竇,心想自己在光明頂上,親眼見到成崑死於舅舅手下,屍橫就地,怎麼會去殺死史火龍?難道是他在上光明頂之前幹的事麼?問道,「請問姑娘,史幫主喪生,已有多久了?」黃衫女子道︰「去年十月初六,距今兩月有餘。」黃無忌道︰「這就奇了。不知何以知道是成崑那老賊下的毒手。」黃衫女子道︰「王嘯天言道︰他師父史幫主和一老者連對一十二掌,那老者嘔血而走。史幫主自知傷重不治,料想老者三日之後,必定元氣恢復,重來尋釁,當即向王嘯天囑咐後事,説出仇人姓名,乃是混元霹靂手成崑。史幫主雙臂癱瘓之症,其時已愈了九成,他曾得降龍十八掌中十二掌眞傳,武功之強,已是世所罕有,但竭盡全力,十二掌使完,仍是難逃敵人毒手。」女童史紅石聽到這裡,放聲大哭起來。

傳功長老臉現悲憤之色,掏出一塊髒髒的手帕,替史紅石擦去泪水,説道︰「小世妹。幫主之仇,即我幫上下數萬弟子之仇,咱們終當擒住那混元霹靂手成崑,碎屍萬段,以報令尊的大恨。不知令堂現在何處?」史紅石指著黃杉女子,道︰「我媽媽在楊姊姊家裡養傷。」衆人直至此時,方知那黃衫美女姓楊,至於她是何等人物,仍是猜不到半點端倪。

黃衫女手輕輕嘆了口氣,説道︰「史夫人也挨了成崑一掌,傷勢著實不輕,長途跋涉來到舍下,至今昏迷不醒,是否能彀痊可,那也\dash{}那也難説。」執法長老悔恨的道︰「這成崑不知跟老幫主有何仇怨,竟爾下此毒手?」黃衫女子道︰「史幫主遺言道︰他和這成崑素不相識,仇怨兩字,更是無從説起。因此他老人家直到臨終,仍是不明其中之理,據他推測。多半是丐幫中人什麼地方得罪了成崑,因而找到史幫主頭上。」執法長老沉吟道︰「這成崑爲了躱避謝遜,數十年前便已在江湖上消聲匿跡,不知所終,丐幫弟子怎能和他結仇?看來其中必有重大的誤會。」

掌砵龍頭一直在旁靜聽,一言不發,這時突然抓起一柄彎刀,架在那假冒史火龍的禿子頸中,喝道︰「你叫什麼名字?何以假冒幫主?快快説來,若有半字虛言,哼,哼!」説著手起一刀,將一張椅子劈爲兩半。那禿子嚇得魂不附體,道︰「我\dash{}我説\dash{}小人名叫癩頭黿劉敖,本是山西解縣亂石岡山寨中的一名頭目,這天下寨做没本錢的買賣,撞到了陳友諒陳大爺,還有陳大爺的師父。陳大爺一脚將小人踢翻了,提劍正要砍殺,小人連忙磕頭求饒。陳大爺對小人左瞧右瞧,忽然説道︰『師父,這小賊挺像咱們前天所見那個人哪。』他師父搖頭道︰「嘿嘿,年紀不對,鼻子塌了,又是個禿頭。』陳大爺笑道︰『弟子有法子弄他像來。』於是叫小人跟著他們到解縣,住在客店之中。陳大爺去弄了些石膏,裝高了小人鼻子,又叫我戴上白頭假髮的髮,喬扮成這等模樣\dash{}各位老爺,小人便有天大的膽子,也不敢來戲弄諸位,只是陳大爺這麼説,小人只好這麼幹,小人狗命一條,全捏在他手裡,那\dash{}那是無可奈何,小人家中尚有八十歳的老娘,衆位大爺饒命則個。」説著雙膝跪倒,磕頭便如搗蒜。

執法長老沉吟道︰「陳友諒出身少林派,他師父是少林寺的高僧,他\dash{}他還有什麼師父?」這一言提醒了張無忌,當即接口道︰「不錯,他師父便是成崑。」於是將成崑化名圓眞,混入少林寺拜神僧空見爲師,自己幼時曾在少林寺中遭圓眞毒手等情簡略説了,其後又敘述圓眞如何偸襲光明頂,終於爲白眉鷹王殷天正所擊斃,但屍身却又突然失蹤。掌缽龍頭和執法長老齊聲道︰「此事已無可疑。在光明頂上,成崑乃是假死,混亂之中,悄悄溜走。」傳功長老怒道︰「看來罪魁禍首,竟是陳友諒這奸賊。他師徒二人野心勃勃,妄圖獨霸天下,是以害死史幫主,命這小毛賊冒充史郡主,做他們傀儡,再想進一步挾制明教,籠絡少林、武當、峨嵋三大派。這奸計不可謂不毒,野心不可謂不大。宋青書呢?宋青書到那裡去了?」各人這些時中只注視著丐幫主、黃衫女子、史紅石等人,没防到宋青書竟也步著陳友諒後塵,不知何時溜之大吉了。説到此時,各方面一加印證,陳友諒的奸計終於全盤暴露,傳功長老向黃衫女子深深一揖,説道︰「姑娘有大德於敝幫,丐幫不知何以爲報。」黃衫女子淡淡一笑,道︰「我先人和貴幫上代淵源甚深,些些微勞,何足掛齒?這位史家小妹妹,你們好好照顧。」説了這幾句話,躬身一禮,黃影一閃、已掠上屋頂。傳功長老叫道︰「姑娘且請留步。」

只見那四名黑衣少女,四名白衣少女一齊躍上屋頂,琴聲丁冬、蕭聲嗚咽,片刻間琴蕭之聲飄然遠引,曲未終而人已不見,倏然而來,倏然而去,衆人心下均感一陣惘然。

傳功長老擕了史紅石的手,向張無忌道︰「張教主,且請進廳内説話。」群丐恭恭敬敬的站在一旁,請張無忌先行。無忌也不客氣,走進廳内,和傳功長老等分賓主坐定,周芷若坐在他的肩下。無忌第一件事便是関心謝遜的下落,請問了傳功長老、執法長老諸人的姓名後,便道︰「曹長老,我義父金毛獅王若在貴幫,便請出來相見。」傅功長老嘆了口氣,道︰「陳友諒這奸賊玩弄手段,累得丐幫愧對天下英雄。不瞞張教主説,謝大俠和這位周姑娘,確是咱們在関外合力請來敝幫,其時謝大俠身染疾病,昏迷在床。咱們没經動手過招,就請他大駕到了此間。八日之前的晩間,謝大俠突然擊斃了看守他的敝幫弟子,脱身而去。所斃丐幫人衆,棺木尚停在後院未葬,張教主若是不信,可請移駕到後院審察。」無忌見他言語誠懇,何況那晩丐幫弟子屍橫斗室之情,則是自己親眼目睹,便道︰「曹長老既如此説,在下焉敢不信?」尋思︰「義父離去的那晩,我曾見一個身形苗條的女子黑影。輕功極高,又在牆頭見到一個女子的足印,莫非是那位黃衫女子麼?」

於是問史紅石道︰「小妹妹,這位楊姊姊之家住何處?你從前識得她麼?」史紅石搖頭道︰「我從前不識。王大哥聽了我爹爹吩咐,帶著這根竹棒児,和媽媽同我一起坐車,路上遇到惡人,打了一架,王大哥又傷了。咱們一起坐車走了好幾天。又上山去。王大哥走不動了,便在地下爬,後來到了一座樹林外邉,王大哥大叫幾聲。後來一位穿黑衣的小姊姊出來,後來楊姊姊出來,問了王大哥許多話,拿這棒児去了半天。後來王大哥死了,媽媽又昏了過去。楊姊姊便帶了我,又帶了八個穿白衣裳、黑衣裳的小姊姊,坐了車子來啦。」她年紀幼小,説不出個所以然,問到地名日子,也是一槩不知,從她口中,竟是探不到半點端倪。

傳功長老道︰「貴教韓山童大爺的公子,却在敝幫。」他轉頭吩咐了幾句,一名丐幫弟子匆匆進去。過不多時,只聽得韓林児破口大罵的聲音,從後堂傳了過來︰「你們這些個個不得好死的臭叫化,又來欺騙老子!咱們張教主身份何等尊貴,豈能駕臨你們這臭叫化窩來。你乘早送老子上了西天,鬼鬼祟崇的奸計,一槩不管用。」丐幫衆長老聽了這些罵聲,臉上均現羞慚之色。張無忌心想︰「這位韓大哥確是個忠義赤膽、鐵錚錚的好男児。」心下敬他爲人,站起身來,搶上幾步,見韓林児從後壁大踏步走將出來,便道︰「韓大哥,我在這裡,這幾天委屈了你啦。」韓林児一見張無忌,一怔之下,心中大喜,當即跪下拜倒,説道︰「張教主,果然是你老人家來啦,這可想煞了小人。你快傳下號令,將這些臭叫化児殺他個乾淨。」

張無忌含笑扶起,説道︰「韓大哥,丐幫諸位長老也是中了旁人奸計,致生誤會。此刻已分解明白,韓大哥瞧在兄弟面上,不必介意。」韓林児站起身來,向傳功長老等怒目而視,本想痛罵幾句,出一出心中怨氣,只是教主如此吩咐,只得強自忍耐。執法長老道︰「張教主今日光降,實是敝幫莫大榮寵,快整治筵席,大夥児一來向張教主接風,二來向峨嵋派周掌門致歉,三來向韓大哥陪罪。」早有衆弟子答應了下去。無忌心懸義父,有許多話要向周芷若詢問,實是無心飲食,當即抱拳説道︰「諸位美意,在下多謝了。只是在下急於尋訪義父,只好日後再行叨擾,莫怪,莫怪。」

傳功長老等挽留再三,張無忌見其意誠,只得留下與宴。席間丐幫諸高手又鄭重謝罪,並説即當派丐幫中弟子,四出尋訪謝遜下落,一有訊息,立即遣急足報與明教知道。張無忌謝了,與諸長老、龍頭痛飲而散。丐幫衆高手見他年紀雖輕,但武功既高而絶無傲人之態,豁達大度,殷殷以擕手共抗韃子爲勉,衆人均是大爲心折,席上訂交,直送至盧龍城外十里,方始分手。

張無忌、周芷若、韓林児三人騎著丐幫所贈駿馬,沿著官道南下。韓林児對教主十分恭謹,不敢並騎而行、遠遠跟在後面,沿途倒水奉茶,猶如奴僕一般的服侍張周二人。張無忌過意不去,數次説道︰「韓大哥,你雖是我教下兄弟,但我敬你爲人,在公事上你聽我號令,日常相處,咱們平輩論交,便如兄弟朋友一般。」韓林児甚是惶恐,道︰「屬下對教主死心塌地的敬仰,平輩論交,如何克當?平時無緣多親近教主,今日得小小盡心,服侍教主,實是屬下生平之幸。」周芷若微笑道︰「我不是你教主,你却不必對我這般恭敬。」韓林児道︰「周姑娘是天人一般的人物,小人能跟你説幾句話,已是一輩子的大幸。言語粗魯,姑娘莫怪。」周芷若見他説得誠懇,眼中所流露的嚴謹崇敬,當眞是將自己當作了天仙天神。她自知容色清麗,所有青年男子遇到自己,無不心搖神馳,但如韓林児這般五體投地的拜倒,却也是生平從所未遇,少女情懷,實不禁暗自欣喜。

無忌詳細問她當日被丐幫擒獲的經過,周芷若言道︰那日無忌出了客店,去偵視丐幫有無密謀,去後不久。謝遜突然渾身顫抖,胡言亂講起來。她心中害怕,竭力對他安靜,但謝遜似乎不認得了她,在店房中亂跳亂竄,過了一會,便即癱瘓在地,人事不知,便在此時,丐幫中有六七名高手同時搶進房來,她來不及抽劍抵禦,便被點中了穴道,和謝遜二人同時被送到盧龍。無忌幼時便知義父因練七傷拳傷了心脈,兼之全家爲成崑所害,偶爾會心智錯亂,只是没料到他,偏在這當口發作,以致無法抵擋丐幫的侵襲,兩人琢磨謝遜不知此刻到了何處,周芷若亦感茫無頭緒。無忌道︰「京師是各路人物會聚之處,咱們南下路過,便可去大都打探一下消息。我想青翼蝠王韋兄手中,多半會有若干線索。」周芷若抿嘴笑道︰「你去大都啊,當眞是想見韋一笑麼?」張無忌知她言中之意,不禁臉上一紅,道︰「那也不一定找得到韋兄。咱們要打探義父的所在,若能遇上韋兄、苦頭陀、楊左使他們,總能幫我出一些主意。」周芷若微笑道︰「有一位神機妙算、足智多謀的人児,你到大都去找她,更能幫你出一些好生意,楊左使、苦頭陀他們,萬萬不及這位姑娘聰明。」張無忌一直不敢跟她説起在彌勒佛廟中與趙明邂逅相遇之事,這時聽他提及趙明,不由得神色問頗爲{\upstsl{尷}}尬忸怩,道︰「你總是念念不忘趙姑娘,高興起來便損我兩句。」周芷若笑道︰「也不知是我念念不忘她呢,還是另有一人念念不忘於她。你自己作賊心虛,當我瞧不出你心中有鬼麼?」

張無忌天性至誠,心想自己與周芷若已有白頭之約,此後生死與共,兩情不貳,什麼事都不該隱瞞於她,當下提起勇氣,説道︰「芷若,有一件事我該當與你説,請你别生氣。」周芷若道︰「我該生氣便生氣,不該生氣便不生氣。」無忌聽了這兩句話,心中一窒,暗想自己曾對她發下重誓,決意殺了趙明,爲表妹殷離報仇,但與趙明相見後非但不殺,反而和她荒郊共宿,連騎並行,這番話説出來實是心中有愧。他不善作偽,自覺羞慚,神色間便盡數顯了出來。

無忌沉吟之間,雙騎已奔近一處小鎭,眼見天色不早,當下找一家小客店投宿。晩飯過後,無忌又替周芷若背心穴道上推拿了一陣,雖然仍非解這奇異的點穴之法,但點穴後爲時已久,血脈運轉,被封住的穴道終於也自行解開了。無忌心下暗想︰「丐幫諸長老的武功雖非極強,點穴手法却大是神妙。芷若心性高傲,不肯在席間求他們解穴,那出手點穴之人居然也假裝忘記了。嘿嘿,這些化子們死要面子,一敗塗地之餘,勉強在點穴法上佔一些上風,也是好的。」

周芷若嫌客店中一股汚穢的霉氣,道︰「咱們到外面走走,活活血脈。」張無忌道︰「好!」擕了她手,走到鎭外。其時夕陽在山,西邉天上晩霞如血,兩人閒步一會,在一株大樹下坐了,但見太陽慢慢鑽入地下,周遭暮色漸漸逼來。於是張無忌將彌勒佛廟中如何遇見趙明、如何發現莫聲谷的屍體、如何和宋遠橋等相會、如何循著明教的火燄記號在冀北大兜圏子等情一一説了,説到最後,雙手握著周芷若的手,道︰「芷若,你是我未過門的妻子,咱倆夫婦一體,我什麼事也不會瞞你。趙姑娘堅要再見我義父一面,説有幾句要緊的話問他。我當時便起了疑心。此刻回思,越想越是害怕。」説到最後這幾句話,聲音也發顫了。周芷若道︰「你害怕甚麼?」張無忌只覺掌中的一雙小手寒冷如冰、也是輕輕發抖,便道︰「我想起義父患有失心瘋之症,一發作起來,人事不知。當年他瘋疾大發,竟圖向我媽媽非禮,他一對眼睛,便是因此被我媽媽打瞎的。當我出生之時,義父又想殺死我爸爸媽媽,幸而聽到我的哭聲,這纔神智清醒。我怕\dash{}我怕\dash{}」

周芷若道︰「你怕甚麼?」無忌嘆了口氣,道︰「此話我本來不該説,但我確是擔心,我那表妹是\dash{}是\dash{}義父殺的。」周芷若跳起身來。顫聲道︰「謝大俠仁俠仗義,對咱們後輩更是慈愛,怎會去殺殷姑娘?」無忌道︰「我這只是空口猜測,當然作不得準。就算我表妹眞爲義父所殺,那也是他老人家舊疾突發,猶如夢魘一般,不是他老人家生性殘忍。唉,這一切帳,都該算在成崑那惡賊身上。」周芷若沉思半晌,搖頭道︰「不對,不對!難道咱們齊中『十香軟筋散』之毒,也是義父他老人作的手脚?他又從何處得這毒藥?」無忌眼前猶如罩了一團濃霧,瞧不出半點光亮。只聽周芷若冷冷的道︰「無忌哥哥,你是千方百計,在想替趙明開脱洗刷。」無忌道︰「倘若趙姑娘她眞是兇手,她躱避義父尚自不及,何以執意要見義父,説有幾句要緊話問他?」周芷若冷笑道︰「這位姑娘機變無雙,她要爲自己洗脱罪名,難道還想不出什麼巧妙法児麼?」她語聲突轉溫柔,偎倚在他身上,説道︰「無忌哥哥,你是天下第一等的忠厚老實之人,説到聰明智謀,如何能是趙姑娘的對手?」無忌嘆了口氣,覺得她所言確甚有理,伸臂輕摟住她柔軟的身子,低聲道︰「芷若,我只覺世事煩惱不盡,即如親如義父,也令我起了疑心,我只盼驅走韃子的大事一了,你我隱居深山,共享清福,再也不理這塵世之事了。」周芷若道︰「你是明教的教主,倘若天如人願,眞能逐走了胡虜,那時天下大事。都在你明教掌握之中,如何能容你去享清福?」張無忌道︰「我才幹不足以勝任教主,更不想當教主。要是明教掌握重權,這一教之主,更非由一位英明智哲之士來擔當不可。」周芷若道︰「你年紀尚輕,目下才幹不足,難道不會學麼?再説,我是峨嵋一派的掌門,肩頭擔子甚重。師父將這掌門人的鐵環授我之時,命我務當光大本門,就算你能隱居山林,我却没這般福氣呢。」

\chapter{花燭春宵}

無忌撫摸她手指上的鐵指環,道︰「那日我見這指環落在陳友諒的手中,心裡焦急得了不得,只怕你受了奸人的欺辱,恨不得插翅飛到你的身邉。芷若,我没能早日救你脱險,這些日子中,你可受了委屈啦,這鐵指環,他們怎麼又還了你?」周芷若道︰「是武當派的宋青書少俠拿來還我的。」無忌聽她提到宋青書的名字,眼前突然出現她與宋青書併肩共席、在丐幫花廳上飲酒的情景,問道︰「宋青書對你很好,是不是?」周芷若聽他語聲有異,問道︰「什麼叫做『對你很好』?」無忌道︰「没什麼?我只是隨便問問,這位宋師哥對你一往情深,不惜叛派逆父,{\upstsl{弒}}叔謀祖,對你自是很好的了。」

周芷若仰頭望著東邉初升的新月,幽幽的道︰「你待我只要能有他一半的好,我就心滿意足的了。」無忌道︰「我固是不及宋師哥這般情痴,要我爲你做這些不孝不義之事,那是萬萬不能。」周芷若道︰「爲了我,你是不能。爲趙姑娘,你偏能彀。你在那小島上曾立過重誓,定當殺此妖女,以替殷姑娘報仇。可是你一見她面,早將這些誓言忘得乾乾淨淨了。」無忌道︰「芷若,要是我査明屠龍刀和倚天劍確是趙姑娘所盜,害死了表妹的惡事確是她所幹,我自不饒她。但若她是清白無辜,我總不能無端端的殺她。説不定我當日在小島上立的誓,是立錯了。」

芷若不語,無忌道︰「是我説錯了麼?」周芷若道︰「不!是想起我自己在萬法寺的高塔之上,也曾在師父跟前發過重誓。只恨我在那小島上對你以身相許之時,不肯把這重誓説了出來。」無忌驚道︰「你\dash{}你發過什麼重誓?」周芷若道︰「那時我對師父發誓説,要是我日後嫁你爲妻,我父母死在地下不得安穩,我師父化爲厲鬼,日夕向我糾纏,我跟你生的子孫男的世世爲奴,女的代代爲娼。」

張無忌一聽到這幾句如此毒辣的惡誓,不禁身子發抖,隔了半晌,纔道︰「芷若,那是作不得數的。你師父只道明教是爲非作惡的魔教,我是奸邪無恥的淫賊,纔逼你發此重誓。她老人家若是得知眞相,定能教你免了此誓。」周芷若泪流滿面,道︰「可是她\dash{}她老人家已經不知道啦。」説著撲在無忌懷裡,抽抽噎噎的哭個不休。無忌撫摸她的柔髮,慰道︰「你師父若是地下有知,定然不會怪你背誓。難道我眞是個奸邪無恥的淫賊?」周芷若抱著他腰,説道︰「你現下還不是,可是你將來受了趙明的蠱惑,説不定\dash{}説不定便奸邪無恥了。」無忌伸指在她臉頰上輕輕一彈,笑道︰「你也把我忒也瞧得小了。良人者,所仰望終身者也。你的夫君是這樣的人麼?」周芷若抬起頭來,臉頰上兀自帶著晶晶珠泪,眼中却已全是笑意,説道︰「也不羞,你已是我的夫君了麼?你再跟那趙明小妖女鬼鬼祟祟,我纔不要你呢。誰保得定你不會如那宋青書一般,爲了一個女子,便做出許多卑鄙的勾當。」無忌低下頭去,在她臉頰上一吻,笑道︰「誰叫你天仙下凡,咱們凡夫俗子,焉能自持?這是你爹爹媽媽不好,生得你太美,可害死咱們男人啦!」

突然之間,三丈外一株大樹之後,「嘿嘿」連聲,傳來兩下冷笑,無忌正將周芷若摟在懷裡,一愕之間,只見一個人影連晃幾晃,已遠遠去了,周芷若一躍而起,蒼白著臉,顫聲道︰「是趙明!她一直跟在咱們身後。」無忌聽這兩下冷笑,確是女子的聲音,只是難以斷定是否眞是趙明,遲疑道︰「眞是她麼?她跟著咱們幹麼?」周芷若怒道︰「她喜歡你啊,還假惺惺的裝不知道呢。你們多半暗中約好了的,這般裝神扮鬼的來耍弄我。」無忌連叫冤枉。周芷若俏立寒風之中,思前想後,不由得怔怔的掉下泪來。

無忌左手輕輕摟住她肩頭,右手伸袖替她擦去泪水,柔聲道︰「怎麼好端端地又流起泪來?若是我約趙姑娘來此,教我天誅地滅。你想想,要是我心中對她好,又知她人在左近,怎會跟你瘋瘋癲癲的説些親熱話児?那不是故意讓她難堪麼?」周芷若心想這話倒也不錯,嘆口氣道︰「無忌哥哥,我心中一直難以平定。」無忌道︰「爲什麼?」周芷若道︰「我總是忘不了對師父發過的重誓,又想這趙明定然放不過我,不論武功智謀,我都跟她差得太遠。」無忌道︰「我盡心竭力,護你周全。她膽敢傷我愛妻的一根毫毛,我豈肯饒她?」周芷若道︰「若是我不幸死在她的手裡。那也罷了,只怪我自己命苦。怕的是你受了她的迷惑,信了她花言巧語,中了她的圏套機関,却來殺我,那時我纔死不瞑目呢。」無忌道︰「那當眞是杞人憂天了。世上多少害過我、得罪過我的人,我都不殺,怎麼反而會殺你?」他解開衣襟,露出胸口的劍疤,笑道︰「這一劍是你刺的?你越是刺得我深,我越是愛你。」

周芷若伸纖纖素手,撫摸他胸口的傷痕,心中若不勝情,突然臉色蒼白,説道︰「一報還一招,將來你便是一劍將我刺死,我也不悔。」無忌伸臂將她樓在懷裡,道︰「待咱們找到義父,請他老人家替咱倆主婚,自後咱二人行坐不離,白頭偕老。只要你喜歡,再刺我幾劍都成,我重話児也不説你一句,這麼著,你彀便宜了吧?」周芷若將臉頰貼在他火熱的胸膛之上,聞到他肌膚間男子的氣息,低聲道︰「但願你大丈夫言而有信,不忘了今日之言。」兩人偎倚良久,直至中宵,風露漸重,方回客店分别就寢。

次晨三人繼續南行,一路没再發見趙明的蹤跡,不一日已來到大都。進城時已是傍晩,只見合城男女,都在灑水掃地,將街道里巷,掃得乾乾淨淨,每一家門口都擺了一張香案,無忌等投了客店,問店伴城中有何大事,店小二道︰「客官遠來不知,却也撞得眞巧,合該有眼福,明日是大遊皇城啊。」無忌道︰「什麼叫遊皇城?」店小二道︰「明児是一年一度皇上大遊皇城的日子。皇上要到慶壽寺供香,數萬男男女女扮戲遊行,頭尾三十餘里。那纔叫好看哩。客官今晩早些安息,明児起一個早,到王德殿門外去佔個座児,要是你眼光好,皇上、皇后、貴妃、太子、公主,個個都能瞧見。你想想,咱們做小百姓的,若不是住在京師,那裡有親眼見到皇上的福氣?」韓林児聽得不耐煩起來,斥道︰「認賊作父,無恥漢奸。韃子的皇帝,有什麼好看?」店小二睜大了眼睛,指著他道︰「你\dash{}你説這種話,不是造反麼?你不怕殺頭要?」韓林児道︰「你是漢人,韃子害得咱們多慘,你居然皇上長、皇上短的,還有半點骨氣麼?」那店小二見他兇霸霸的,轉身便欲出去。周芷若手起一指,點中了他背上的穴道,道︰「此人出去,定然多口,只怕不久有官兵前來拿人。」説著一脚將他踢到了床底,笑道︰「且餓他幾日,咱們走的時候再放他。」過不多時,掌櫃的在外面大叫︰「阿福,阿福,又在那裡{\upstsl{嘮}}叨個没完没了啦!快給三號房客人打臉水。」韓林児忍住好笑,拍桌叫道︰「快送酒飯來,大爺們餓啦。」過了一會,另一名店小二送酒飯進來,不斷自言自語︰「阿福這小子定是去皇城瞧煙花去啦,這小子正經事不幹,便是貪玩。」

次日清晨,無忌剛起床,便聽得門外一片喧嘩。他走到門口,只見無數男女,都是衣衫光鮮,一齊向北湧去,人人嘻嘻哈哈,比過年還要熱鬧,炮仗之聲,四面八方的響個不停。周芷若也到了門口,道︰「咱們也瞧瞧去。」

張無忌道︰「我和汝陽王府中的武士們動過手,莫要被他們認了出來,既要去瞧,須得喬裝改扮一下。」當下和周芷若、韓林児三人扮成了村漢村女的模樣,用泥水塗黃了臉頰雙手,跟著街上人衆,湧向皇城。

其時卯末辰初,皇城内外,已是人山人海,幾無立足之地。張無忌雙臂前伸,輕輕開道,終於在延春門外一家大戸人家的屋簷之下,擠到了三個空位。立定不久,已聽得鑼聲{\upstsl{噹}}{\upstsl{噹}},自遠而來。衆百姓齊呼︰「來啦,來啦!」人人延頸而望。鑼聲漸近漸響,敲到近處,只見見一百零八名長大漢子,一色青衣,左手各提一面徑長三尺的大鑼,右手鑼鎚齊起齊落。一百零八面大鑼{\upstsl{噹}}的一聲一齊響了出來,直是震耳欲聾。鑼隊過去,跟著是三百六十人的鼓隊,其後是漢人的細樂吹打、西域琵琶隊、蒙古號角隊,每一隊少則百餘人,多則千人。樂隊行完,只見兩面素緞大旗,高擎而至。一面旗上書著「安邦護國」,一面旗上書著「鎭邪伏魔」,旁附許多金光閃閃的梵文。大旗前後各有二百蒙古精兵衛護,長刀勝雪,鐵矛如雲,四百人騎的一色白馬。衆百姓見了這等威武氣槩,都大聲歡呼起來。無忌心下暗自感嘆︰「外省百姓,對蒙古官兵無不恨之切骨,這京師人士却是身爲亡國奴而不知恥,想是數十年來日日見到蒙古朝廷的威風,竟忘了自己是亡國之身了。」

兩面大旗剛經過無忌身邉不久,突然間西首人叢中白光連閃,兩排飛刀直射出來,逕奔兩根旗桿。每排飛刀均是連串七柄,七把飛刀整整齊齊的插在旗桿之上。那旗桿雖粗,但連受七把飛刀的砍削,立時折斷,呼呼兩響,從半空中倒將下來。只聽得慘叫之聲大作,十餘人被旗桿壓在下面。在旁瞧熱鬧的衆百姓大呼小叫,紛紛逃避,登時亂作一團。

這一下變起倉卒,張無忌等也是大出意料之外。韓林児大喜之下,正要喝采,驀地裡一隻軟綿綿的手掌伸了過來,按住他的口上,却是周芷若及時制止他的呼喝。只見四百名蒙古兵各持兵刃,在人叢中插索搗亂之人。無忌見這十四柄飛刀發射的手勁甚是凌厲,顯是武林高手所爲,只是閒人阻隔,没能瞧見放刀之人是誰。連他都没法見到,蒙古官兵只自亂哄哄的瞎搜一陣。過不多時,人叢中有七八名漢子被橫拖直曳的拉了出來,口中大叫︰「冤枉\dash{}」蒙古兵刀矛齊下,立時將這些漢子殺死在大街之上。韓林児大是氣憤,説道︰「放飛刀的人早已走了,憑這些膿包,也捉得到麼?却來亂殺良民出氣。」周芷若低聲道︰「韓大哥禁聲!咱們是來瞧遊皇城,不是來大鬧皇城。」韓林児道︰「是。」不敢再説什麼了。

亂了一陣,後邉樂聲又起,過來的一隊隊都是吞刀吐火的雜耍,各種西域秘技,看到衆百姓喝采不迭,適纔血濺街心的慘劇,似乎已忘了個乾淨,其後是一隊隊的傀儡戲、皮影戲,更後是駿馬拖拉的綵車,每輛車上都有俊童美女扮飾的戲文,什麼「白娘娘水浸金山」「唐三藏西天取經」「唐明皇遊明宮」「李存孝打虎」「劉関張三戲呂布」「張生跳粉牆」,爭奇鬥勝,極盡精功。張無忌等三人一向生長於窮鄕僻壤,那裡見過這許多繁華氣象,都是不禁暗嘆今日大開眼界。

這些綵車之上,都插有錦旗,書明「臣湖廣行省左丞相某某貢奉」「臣江浙行省右丞相某某貢奉」等字樣。越到後來,貢奉者的官爵愈大,綵車愈是華麗,扮飾戲文男女的身上,也是越加珠光寶氣,髮釵頸鍊,竟然都是極貴重的翡翠寶石。原來這些蒙古王公大臣,一來爲討皇帝歡喜,二來各自誇耀豪富,都是不惜工本的裝點貢奉綵車。

絲竹悠揚聲中,一輛裝扮著「白兔記」戲文的綵車過去,忽然樂聲一變,音樂單調古拙,綵牽上一面白布旗子,冩的是「周公流放管、蔡」,車中一個中年漢子手捧朝笏,扮演周公,旁邉坐著一個穿天子衣冠的小孩。扮演成王。管叔蔡叔交頭接耳,向周公指指點點。接著一輛綵車,旗上冩的是「王莽假仁假義」,車中的王莽,白粉塗面,雙手滿持金銀,向一群寒酸士人施捨。其後是四面布旗,冩著四句詩道︰

\begin{quotation}
周公恐懼流言日\hskip8pt王莽謙恭下士時

若使當時便身死\hskip8pt千古眞偽有誰知
\end{quotation}

張無忌見了這兩齣戲文,心中一動︰「天下是非黑白,固非易知。周公是大聖人,當他流放管叔蔡叔之時,人人説他有{\upstsl{篡}}位之意。王莽是大奸臣,但起初謙恭下士,舉世莫不歌功頌德。所謂路遙知馬力,日久見人心,世事眞偽,實非朝夕之際可辨。」又想︰「這二輛綵車與衆大不相同,其中顯是隱藏深意,主理之人,却是個頗有學識的人物。」正沉吟閒,忽聽得一聲破鑼,一輛綵車由兩匹瘦馬拉了過來。那車子樸素無華,衆百姓遙遙望見,已哄笑起來,都道︰「這等破爛傢生,也來遊皇城,那不笑掉了大夥児的下巴麼?」

車子漸近,無忌看得分明,不由得大吃一驚,只見車中一個大漢黃髮垂肩、雙目緊閉,盤膝坐在榻上,扮的却不是金毛獅王謝遜是誰?旁邉一個青衣美貌少女,手捧茶碗,殷勤服侍,相貌雖不如周芷若之清麗絶俗,但衣飾打扮,和周芷若當日在風外之時全然一模一樣。韓林児失聲道︰「周姑娘,這人好像你啊。」周芷若哼了一聲,並不回答。無忌回過頭去,只見她臉色鐵青,胸口起伏不定,知她心中極是惱怒,於是伸手握住了她的右手,一時猜不透這輛彩車是何用意。

這車之後,跟著一輛車上仍是一旦一淨,分别扮演謝遜和周芷若。只見那旦角笑嘻嘻繞到淨角背後,伸出兩指,突然在假謝遜背心上用力一戳。假謝遜「啊」的一聲大叫,倒撞下榻,假周芷若伸足將他踏住,提劍欲殺。衆百姓大聲喝采︰「好啊,好啊,快殺了他。」第三輛車上仍是假謝遜和假周芷若二人,另有六七名丐幫幫衆,將假謝遜和假周芷若擒住。張無忌此時更無懷疑,情知這三車戲文,定是趙明命人扮演,料知他和周芷若要到大都來,是以要裝神弄鬼,羞辱周芷若一番。他一俯身,從地下拾起六粒小石子,中指輕彈,嗤嗤連響,將三輛車前的六匹瘦馬右眼睛打瞎了。小石貫腦而入,六馬哀嘶連連,倒地而斃。三輛彩車翻了過來,車上的旦角淨角滾了一地,街上又是一陣大亂。無忌這六粒小石從衣袖之中發出,街上人衆千萬,但除了周芷若和韓林児,誰也不知是他暗中做的手脚。

周芷若咬著下唇,輕聲道︰「這妖女如此辱我,我\dash{}我\dash{}」説到這裡,聲音已然哽咽了。張無忌只覺她纖手冰冷,身子顫抖,忙慰道︰「芷若,這小渾蛋甚麼希奇古怪也想得出來,你别理會。只須我對你一片眞心,旁人挑撥離間,我如何能信?」周芷若頓了一頓,忽道︰「啊,我想起來了。那日,義父本是好端端地,突然間身子一顫,摔倒在地,跟著便胡言亂語的發起瘋來,莫非\dash{}莫非當時這妖女眞是伏在客店中的暗處,向義父後心施發暗器?」張無忌沉吟道︰「她若是做了手脚,再趕來彌勒廟,時候也趕來得及,也説不定是玄冥二老施的暗算。」

説話之間,蒙古官兵已彈壓住衆百姓,拉開死馬,後面一輛輛彩車又絡繹而來。無忌和周芷若只是想著適才情事,也無心觀看車上戲文。彩車過完,只聽得梵唱陣陣,一隊隊身披大紅袈裟的番僧邁步而來。

衆番僧過後,只聽得鐵甲鏘鏘,二千名鐵甲御林軍各持長矛,列隊而過,跟著是三千名弓箭手。弓箭手過盡,香煙繚繞,一尊尊神像坐在轎中,身穿錦衣的伕役抬著經過,甚麼土地、城隍、靈官、韋陀、財神、東嶽,共是三百六十尊神像,最後一神却是関聖帝君。衆百姓喃喃念佛,有的便跪下膜拜。

神像過完,手持金瓜、金錘的儀仗隊開道,羽扇寶傘,一對對的過去。衆百姓齊道︰「皇上來啦,皇上來啦。」遠遠望見一座黃綢大轎,三十二名錦衣侍衛抬著,遠遠而來。張無忌凝目瞧那蒙古皇帝,只見他面目憔悴,雖然雙頰潮紅,但一望而知是荒於酒色。雙目深陥,顯得萎靡不振。皇太子騎馬隨侍,這太子倒是頗有英氣,背負鑲金嵌玉的長弓,不脱蒙古健児本色。

韓林児在張無忌耳邉低聲道︰「教主,你何不撲上前去,一掌劈死了這韃子皇帝,也好爲天下百姓除一大害?」張無忌{\upstsl{嗯}}了一聲,沉吟未答。韓林児又道︰「韃子皇帝身旁護衛雖多高手,未必能擋住教主的一擊。」無忌左首一人忽然説道︰「不妥,不妥。以暴易暴,未見其可也。」張無忌。韓林児、周芷若一齊吃了一驚,向這人看去,却是個五十來歳的賣藥郎中,背負藥囊,右手拿著個虎撐。那人雙手拇指翹起、並列胸前,做了個明教的手勢,低聲道︰「彭瑩玉拜見教主。教主貴體無恙,千萬之喜。」

無忌大喜,道︰「啊,你是彭\dash{}」他化裝之術極是巧妙,站在無忌身旁已久,無忌等三人竟是毫没察覺。彭瑩玉低聲道︰「此間非説話之所。鞋子皇帝除他不得。」張無忌素知他極有見識,點了點頭,不再言語,伸出右手去抓住了他左手,心頭喜之不勝。

皇帝和皇太子過後,又是三千名鐵甲御林軍,其後成千成萬的百姓跟著瞧熱鬧。街旁衆百姓都道︰「瞧皇后娘娘公主娘娘去。」人人一湧而西。周芷若道︰「咱們也去瞧瞧。」四人擠入人叢,隨著衆百姓到了玉德殿外,只見七座金脊綵樓聳然而立,樓外御林軍手執藤條,驅趕閒人。百姓雖衆。但張無忌等四人既要擠前,自也輕而易舉,不久便到了綵樓之前。中間最高一座綵樓,皇帝居中坐在龍椅上,旁邉兩位皇后,都是中年的胖胖婦人,全身包裹在珠玉寶石之中,説不盡的燦爛光華。皇太子坐於左邉下首,右邉下首坐著一個二十來歳的女子,身穿綿袍,想必是公主了。無忌一瞥之下,只見左首第二座綵樓中,一個少女身穿貂裘,頸垂珠練,巧笑嫣然,美目流盼,正是趙明。這綵樓居中坐著一位長鬚王爺,相貌威嚴,自是趙明的父親汝陽王察罕特穆爾。趙兄之兄庫庫特穆爾在樓上來回閒步,鷹視虎步,極見驃悍。

周芷若瞧著兩位皇后,呆呆出神,不禁走得太近。突然之間,一名御林軍揚起藤鞭,劈頭向她擊了下來。無忌右手輕帶,已抓住了鞭梢,只須一揮,便將他摔得鼻青目腫,但隨即放手,轉身退入人叢。此時衆番僧正在綵樓前排演「天魔大陣」,五百人敲動法器。左右盤旋,縱高伏低。陣法變幻極盡巧妙,衆百姓歡聲雷動,皆大讚嘆。周芷若向趙明凝望半晌,嘆了口氣,道︰「咱們回去了吧?」

四人從人叢中擠了出來,回到客店。彭瑩玉向張無忌行參見之禮,各道别來情由。無忌問起謝遜消息,彭瑩玉却是甫從淮泗來到大都,未知謝遜已回中原。他説起朱元璋、徐達、常遇春等年來攻城略地,甚立戰功,明教聲威大振。韓林児道︰「彭大師,適纔咱們搶上綵樓,一刀將韃子皇帝砍了,豈非一勞永逸?」彭瑩玉搖頭道︰「這皇帝昏庸無道,正是咱們大大的幫手,豈可殺他?」

韓林児奇道︰「韃子皇帝昏庸無道,苦害百姓,怎麼反而是咱們大大的幫手了?」彭瑩玉道︰「韓兄弟有所不知,韃子皇帝任用番僧,朝政紊亂,又命賈魯開掘黃河,勞民傷財,弄得天怒人怨。咱們近年來連勝韃子,你道咱們這些烏合之衆,當眞是縱橫天下的蒙古精兵之敵麼?只不過衆百姓恨極了韃子,一有戰鬥,人人群起而攻,韃子兵好漢敵不過人多,是以落敗,這胡塗皇帝不用好官。汝陽王善能用兵,韃子皇帝處處防他,事事掣肘,生怕他立功太大,搶了他的皇位,因此近來削減他兵權,只是派些吹牛拍馬的酒囊飯袋來領兵。蒙古兵再會打仗,也給這些混蛋將軍害死了。這韃子皇帝,可不是咱們大大的幫手麼?」

這番話只聽得韓林児如夢初醒,連連點頭稱是。彭瑩玉又道︰「咱們若是殺了韃子皇帝,皇太子接位,新皇縱然昏庸,總比他的胡塗老子好些。倘若他起用一批能征慣戰的宿將來打咱們,那就糟了。」張無忌道︰「幸得大師及時提醒,否則今日咱們只怕已壞了大事。」韓林児連打自己嘴巴,罵道︰「該死,該死!瞧你這小子以後還敢胡説八道麼?」登時把張無忌、周芷若、彭瑩玉逗得都笑了。

彭瑩玉又道︰「教主是千金之體,肩上擔負著驅虜復國的重任,也不宜干冒大險,效那博浪之一擊。屬下見皇帝身旁的護衛之中,高手著實不少,教主雖然神勇絶倫,但終須防寡不敵衆。萬一失手,如何是好?」張無忌拱手道︰「謹領大師的金玉良言。」周芷若忽然嘆了一口氣,道︰「彭大師這話當眞半點不錯,你怎能輕身冒險?要知待得咱們大事一成,坐在這彩樓龍椅之中的,便是你張教主了。」韓林児拍手道︰「那時候啊,教主做了皇帝,周姑娘做了皇后娘娘,彭大師和楊左使做個左右丞相,豈不美哉!」周芷若雙頰暈紅,含羞低頭,但眉梢眼角間,實是不勝之喜。張無忌連連搖手,道︰「韓兄弟,此言不可再説。本教只圖拯救天下百姓於水火之中,功成身退,不貪富貴,那纔是大丈夫光明磊落的行徑。」彭瑩玉笑道︰「教主胸襟固非常人所及,只不過到了那時候,黃袍加身,你想推也推不掉的。當年陳橋兵變之時,趙匡胤何嘗想做皇帝呢?」張無忌只道︰「不可,不可!我若有非份之想,教我天誅地滅,不得好死。」周芷若聽他説得決絶,臉色微變,眼望窗外,説道︰「明教教徒做皇帝,那也不稀奇。當年我爹爹自立爲王,倘若成事,他老人家不就是皇上麼?」彭瑩玉黯然嘆道︰「不錯,只可惜當年周子旺周師兄造反不成,否則周姑娘好端端的便是一位公主娘娘。」周芷若冷笑道︰「哼!汝陽王的郡主,那有什麼希罕了?偏偏有人當她了不起,目不轉睛地瞧著出神。我若是個男子漢,想娶韃子的皇親國戚,那也得娶皇帝的公主,做個駙馬爺纔算體面。」彭瑩玉和韓林児只當她是隨口説笑,都哈哈大笑起來。張無忌却神色頗爲{\upstsl{尷}}尬,尋思︰「芷若素來溫文靦靦,如何今日説起這些話來?想是今日我瞧趙姑娘時,被她冷眼旁觀,心中不快,是以這時乘時發作。唉,那也是她一番愛我的深情厚意。」

當下彭瑩玉向無忌稟報各地明教起事抗元的情形,雖是有勝有敗,聲勢却是日盛一日,只可惜各大門派、各大幫會妬忌者有之、牽制者有之,未能齊心協力,以致許多起義都是功敗垂成,倘若武林群豪能開一大會,同心反元,那麼大事定然能就。張無忌道︰「大師此言甚是,待得會見楊左使後,咱們好好計議一番。」四人談了一會,用過酒飯,無忌道︰「我和彭大師到街上走走,打聽義父的消息。」

他想韓林児性子直,在京師中見到什麼不平之事,立時便會揮拳相向,闖出禍來,便道︰「韓兄弟,你和芷若今晩别出去了,便在客店中歇歇。」韓林児道︰「是,教主諸多小心!」當下張無忌和彭瑩玉言定一個向西,一個向東,二鼓前回到客店會合。

張無忌出店後信步向西,一路上聽到衆百姓紛紛談論,説的都是今日「遊皇城」的豪華熱鬧,那當眞是一年勝於一年。有人説道︰「南方明教造反,今天関帝菩薩遊行時眼中大放煞氣,反賊定能撲滅。」又有人道︰「明教有彌勒菩薩保佑,看來関聖帝君和彌勒佛將有一場大戰。」又有人道︰「賈魯大人拉伕挖掘黃河,挖出一個獨眼石人,那有人背上刻有兩行字道︰『莫道石人一隻眼,挑動黃河天下反』,這是運數使然,勉強不來的。」

張無忌對這些愚民之言,也無意多聽,信步之間,越走越是靜僻,驀地抬頭,竟是到了那日與趙明會飲的小酒店門外。無忌心中一驚,暗想︰「怎地無意之間,又來到此處了?難道我心中對趙姑娘竟是如此撇不開、放不下嗎?」只見店門半掩,門内靜悄悄地,似乎並無酒客。無忌稍一遲疑,輕輕推門走下進去,見櫃台一名店伴伏在桌上打盹。他走進内堂,只見角落裡那張方桌上點著一枝明滅不定的臘燭,桌旁坐著一人,臉朝向裡。這張方桌正是他和趙明兩次飲酒的所在,除了這一位酒客之外,堂上更無旁人。無忌一怔之下,走近一步,那人聽見聲音,霍地站起。燭影搖晃,映在那人臉上,竟是趙明。

她和無忌兩人都没料到居然又會在此地和對方相見,不禁都是「啊」的一聲,叫了出來。趙明低聲道︰「你\dash{}你怎麼會到這裡?」語聲顫抖,顯是心中極爲激動。無忌道︰「我閒步經過,便進來瞧瞧,那知道\dash{}」他走到桌邉,只見趙明對面另有一副杯筷,問道︰「尚有旁人來要?」趙明臉上一紅,道︰「没有了。前兩次我跟你在這裡飲酒,你坐在我對面,所以\dash{}所以我叫店小二仍是多放一副杯筷。」無忌心中感激,只見桌上的四碟酒菜,竟和第一次趙明約他來飲酒時一般無異,她一番柔情深意,此刻方從心底體會到了,不由得伸出手去,握住了她的雙手,道︰「趙姑娘!」趙明道︰「只恨,只恨自己生在蒙古王家,做了你的對頭\dash{}」

突然之間,窗外「嘿嘿」兩聲冷笑,一物飛了進來。拍的一聲,打滅了燭火,堂中登時漆黑一團。無忌和趙明聽這冷笑之聲,都知是周芷若所發,一時彷徨無主,追出去也不是,留在這裡也不是。耳聽得屋頂之聲細碎,周芷若如一陣風般去了。

趙明低聲道︰「你和她已有白首之約,是也不是!」無忌道︰「是,我原是不該瞞你。」趙明道︰「那日我在樹後,聽到你跟她這般甜言蜜語,恨不得自己立時死了,恨不得自己從來没生在這世界上。那日我冷笑兩聲。她一報還一報,也來冷笑兩聲,可是\dash{}可是你却没跟我説過半句教我心中歡喜的話児。」無忌心下歉仄,道︰「趙姑娘,我不該到這児來,不該再和你相見。我的心已有所屬,決不應再惹你煩惱。你是金枝玉葉之身,從此將我這個山村野夫忘記了吧。」趙明拿起他的手來,撫著他手背上的疤痕,道︰「這是我咬傷你的,你武功再高,醫道再精,總是去不了這個傷疤。你自己手背上的傷疤也去不了,能除去我心上的傷疤麼?」突然之間,雙臂摟住他的頭頸,在他唇上深深一吻。張無忌迷惘之間。趙明用力一口,將他上唇咬得出血。跟著在他肩頭一推,身子竄出窗去,叫道︰「你這小淫賊,我恨你。我恨你!」

\chapter{濠州大會}

韓林児在張無忌、彭瑩玉出店後,向周芷若道︰「周姑娘,你早些安歇。」不敢多説一句話,便起身走向自己房中。周芷若微笑道︰「韓大哥,你怕了我麼?連在我面前多坐一會也不肯。」韓林児漲紅了臉。忙道︰「不,不!」可是脚步邁得更加快了,一走進自己房中,立刻帶上房門,上了閂,心下怦怦亂跳,定了定神,躺在炕上,眼前出現的只是周芷若嬌艷清麗的容顏,溫和柔軟的話聲,心下但想︰「周姑娘日後成了教主夫人,我跟在教主身畔,好好的幹,拚命立些功勞。周姑娘一喜歡,就會説︰『韓大哥,這一趟可辛苦你啦!』那時候啊,我韓林児纔不枉了這一生。」

他躺在炕上出了會神,微笑著朦朧睡去,睡到半夜,忽聽得門上輕輕幾下剝啄之聲。韓林児翻身坐起。問道︰「是誰?」只聽得周芷若在門外説道︰「是我。你開門,我有話跟你説。」韓林児道︰「是,是。赤足便去開門,拔去門閂,忙回身點亮了臘燭。只見周芷若雙目紅腫,神色大異,韓林児嚇了一跳,問道︰「周姑娘,你\dash{}你\dash{}」頓了一頓 底下的話便説不下去了,突然靈機一動。飛奔出房,道︰「我去打水來給你洗臉。」過不多時,赤著雙足,捧了一盆洗臉水進來。周芷若淒然一笑,以手支頤,呆呆的望著燭火。韓林児道︰「你\dash{}你洗臉吧。」周芷若一言不發,搖了搖頭,忽然怔怔的流下泪來。韓林児嚇得呆了,垂手站著,不知周芷若爲何生氣煩惱,更不知她要跟自己説什麼話。

這般僵持良久,忽然拍的一聲輕響,燭花爆了開來,周芷若身子一顫,從沉思中醒了過來,輕輕「{\upstsl{嗯}}」的一聲,站起身來。韓林児大聲道︰「周姑娘,是誰對你不住,姓韓的這就拔刀子找他去,我便是性命不要。也得在他身上戳幾個透明窟進。你請説吧!」周芷若淒然搖了搖頭,走出房去。她進房來坐了半晌,似有滿腹心事傾吐,却是一個字也不説,便又出去。倒教韓林児這莽撞漢子半點摸不看頭腦,呆呆站著,連連握拳搥頭。

他自知是粗人,没法明白女孩児家頭髮般細的心事,想了一會毫無頭緒,耳聽得遠處{\upstsl{噹}}{\upstsl{噹}}{\upstsl{噹}}的打著三更,心想︰「怎地教主和彭大師還没有回來?」只得上炕又睡。朦朧間剛要合眼,忽聽得砰砰一聲,東邉房中似乎有一張椅子倒在地下,那房正是周芷若所居。韓林児一躍出房,月光掩映之下,房中窗上映出一個黑影,似是懸空而掛。兀自微微搖晃。韓林児大吃一驚,叫道︰「周姑娘,周姑娘!」伸手推門,房門却是反閂著。他用肩頭使勁一撞,撞斷門閂,搶進房去,忙打火褶點亮了臘燭,只見周芷若雙足臨空,頭頸套在繩圏之中,那繩子却掛在樑上。他這一驚當眞是魂飛天外,急忙縱身一躍,用力一扯,崩斷了繩子,將周芷若放在床上,一探她鼻息,竟己氣絶。他縱聲大叫︰「周姑娘,周姑娘,你\dash{}你有什麼想不開,幹麼出此下\dash{}下策?」説到後來,竟然喉頭塞住了,再也説不出來。

忽聽得房門外一人道︰「韓大哥,什麼事?」走進一人,正是張無忌。他見此情景,也是如同陡遇雷轟,顫抖著雙手,解去周芷若頸中繩索,一摸她胸口,幸喜一顆心尚自微微跳動。無忌喜道︰「不礙事,救得活的。」伸手在她背心小腹穴道上推拿數下,一股九陽眞氣從掌心傳了過去,來回一撞,周芷若「哇」的一聲,哭了出來。韓林児大喜道︰「好啦,好啦。周姑娘活轉了。」周芷若睜開眼來,見到無忌,哭道︰「你幹什麼理我?讓我死了乾淨。」忽地見到無忌上唇血漬,更有幾個細細的齒痕,心頭怒火上升,一伸手,重重打了他一個耳光。

韓林児大吃一驚,心想毆打教主,那還了得?但周芷若在他心目中却又是敬若天神,一時之間心中大爲胡塗,不知説什麼好。突然間有人伸手在他肩頭輕輕拍了兩下,韓林児回過頭去,見是彭瑩玉,喜道︰「彭大師,你回來啦,快快,快來勸勸周姑娘。」彭瑩玉笑道︰「勸什麼?咱們到外面走走吧。」韓林児急道︰「不,不成啊,要是打起架來,周姑娘不是教主的敵手。」彭瑩玉哈哈大笑,道︰「胡塗兄弟!難道咱兩個幫周姑娘,就能打贏教主了麼?我説教主一定打不贏周姑娘。」説著使個眼色,拉著韓林児使出店房,韓林児却兀自不住回頭,関懷之情,見於顏色。

周芷若忍不住{\upstsl{噗}}{\upstsl{哧}}一笑,但撲在床上,又抽抽噎噎的哭了起來。張無忌坐在床邉,輕拍她的肩頭,柔聲道︰「芷若,我確不是約好了跟她相見,當眞是陰錯陽差、誤打誤撞碰見的。」周芷若雙足亂踢,哭道︰「我不信,我不信。不管你説什麼鬼話,以後别想再叫我相信。」無忌嘆道︰「『周公恐懼流言日,王莽謙恭下士時』,世上的事情,原是極易引起誤會\dash{}」周芷若不等他説完,霍地坐了起來,説道︰「那郡主娘娘用這些詩句來損我,你倒念念有辭,老是記在心裡。你瞧你的嘴唇,也不害羞,成什麼樣子?」説到這裡,臉蛋児却飛紅了。無忌心想今日之事,已百喙難辯,反正自己已決意與周芷若白頭偕老、之死靡它,只有動之以情,令她漸漸淡忘。燭光下見她俏臉暈紅頸中深深一根繩印,兩邉腫了上來,心想若非韓林児及早察覺施救,待得自己回店,只怕她已是香殞玉碎,回天乏術,終成大恨。不禁又是慚愧,又是愛惜,伸臂抱住她,向她櫻唇上吻去。周芷若轉頭閃避,怒道︰「你跟人家不乾不淨,又來惹我。當我是好欺的麼?」無忌雙臂一緊,令她動彈不得,終於在她唇上深深吻了下去。周芷若掙扎不脱,一個心却也漸漸軟了。

無忌心想自己和她雖是名分已定,終是未婚夫妻,深宵共處一室,雖是不及於亂,却不免有瓜田李下之嫌,於彭瑩玉、韓林児等人臉上須不好著,於是放開了她,説道︰「芷若,你好好休息,一切明日咱們再談。我若是再瞞了你去見趙姑娘,任你千刀萬剮,死而無怨。」周芷若蒼白的臉上紅撲撲地,胸口起伏不定,喘氣道︰「胡説八道什麼?你明知我不會將你千刀萬剮。」無忌笑道︰「那麼你{\upstsl{{\upstsl{刴}}}}了我的雙足好不好?」周芷若低下了頭,眼泪撲簌簌的如珠而落。

無忌這一來又不好走了,重行坐在她的身旁,伸臂摟住她肩頭,道︰「怎麼又傷心啦?」周芷若只理哭泣不語。無忌問之再三,不料越是問得緊,她越是傷心。無忌罰誓賭咒,説決不負心薄倖。周芷若雙手蒙著臉道︰「我是怨自己命苦,不是怪你。」無忌道︰「咱們小時候,大家命苦。韃子在中國作威作福,誰都是多苦多難。以後,咱倆結成夫妻,又將韃子趕了出去,只有歡喜,没有傷心了。」周芷若抬起頭來,正色道︰「無忌哥哥,我知道你對我一片眞心,只不過趙明那小妖女想誘惑你,却不是你三心兩意。可是\dash{}可是我是不配做你夫人的了。我本想一死了之,那知韓林児這傻瓜偏偏救活了我。我死了一次没勇氣再死了。我\dash{}我要學師父一樣,削髮爲尼。唉,咱們峨嵋派的掌門,終究是没一個嫁人的。」無忌道︰「那到底是爲什麼?你恨趙姑娘誣陥你害我義父麼?」周芷若凝視他雙目,問道︰「你信不信?」無忌道︰「我自然不信。」周芷若道︰「你不信就好了,本來誰都不會相信。」無忌道︰「那麼,又爲了什麼?」周芷若咬了咬牙,説道︰「因爲\dash{}因爲\dash{}」

她説了兩個「因爲」,背轉身去,道︰「無忌哥哥,你只當從來没見過我,從此别記得我這薄命之人。你去娶了趙姑娘也好,另娶淑女也好,我\dash{}我都是不管的了。」突然間雙足一登,身子從窗口穿了出去,直上屋頂。無忌一呆︰「她身法如此輕靈曼妙,好高的武功啊。」一時不及細想,跟著追出?只見周芷若向東疾奔。無忌三個起落,突然間繞到她的身前。周芷若收足不及,撞入了他的懷中。無忌雙臂一張,將她摟住了。該處正是流經大都城内一條小河河畔,無忌半扶半抱的將她帶到河邉一塊石上,偎倚著坐下,柔聲道︰「芷若,咱倆夫婦一體,你有什麼爲難之事,都是和我的一樣,儘管説將出來,也好讓我爲你分憂。你獨自個悶在心裡,却是何苦?」

周芷若將臉伏在他的胸前,哭道︰「我\dash{}受了人家欺負,已經\dash{}已經不是清白之身了。我肚中已有了孼種,怎能\dash{}怎能再跟你結爲夫婦?」這幾句話聽在無忌耳中,委實猶似晴天霹靂一般,半晌説不出話來。

周芷若緩緩站起身來,説道︰「這是命該如此,你慢慢的,將我忘記了吧。」無忌兀自怔怔的呆著,不相信她所説的話竟是眞的。周芷若輕嘆一聲,轉身便走。無忌一躍而起,拉住她手,顫聲道︰「是\dash{}是宋青書這賊子麼?」周芷若點了點頭,流泪道︰「在丐幫之中,我被點中穴道,無力抗拒\dash{}」無忌緊緊抱住了她,説道︰「這又非你的過失。事已如此,煩惱也是枉然。芷若,這是你遭難,我只有更加愛你憐你。咱們明日立時動身,回到淮泗,告知本教兄弟,我便跟你成親。你肚中\dash{}肚中孩児,便算是我的,於你清白,絶無半點損害。」周芷若低聲道︰「你何必好言慰我?我已非黃花閨女,怎能再做教主夫人?」

張無忌道︰「你\dash{}你也是把我瞧得小了,張無忌是豪傑男児,豈如俗人之見?縱是你一時胡塗,自行失足,我也能不咎已往,何況這是意外之災?」周芷若心中感激,道︰「無忌哥哥,你當眞待我這麼好嗎?我\dash{}我只怕你是騙我的。」無忌道︰「我待你的好處,以後你才知道,現下我還没起始待你好呢。」周芷若撲在他的懷裡,感極而泣,過了一會,説道︰「你用些藥物,先替我將這孼種打了下來。」無忌道︰「不可。打胎之事既傷天和,於你身子又是大大有損。」心下暗想︰「她失陥丐幫,前後不過一月,怎地已知懷孕?説不定是她胡思亂想。」一搭她的脈搏,亦無胎象,但想這種事情不便多問,自己醫術雖精,所專却是治傷療毒,於婦科一道,原是所知不多。只聽周芷若又道︰「這孼種是個女的,那也罷了,倘若是個男児,日後天如人願,你登極做了皇帝,難道要這孼種來做太子?乘早還是打了,免貽無窮之患。」無忌嘆道︰「這『皇帝』兩字,再也休提。我這種村野匹夫,絶無覬覦大寶之意,若教衆師兄弟們聽見,只道我一己貪圖富貴,反而冷了心腸。」周芷若道︰「我也不是強要你做皇帝,但若天命所歸,你推也推不掉的。你待我這麼好,我自當設法圖報。周芷若雖是個弱女子,可是機緣巧起來,説不定我便能助你做了天子。我爹爹事敗身亡,我命中無公主之份,却又有誰知道我不能當皇后娘娘?」無忌聽她説得熱切,笑道︰「皇后娘娘未必及得上峨嵋掌門之尊。好了,明児一早咱們還要趕路。我的皇后娘娘,請駕回宮,早些安歇吧。」滿天愁雲慘霧,便在兩人中一笑之間,化作飛煙而散。

次日清晨,無忌囑咐彭瑩玉續留大都三日,打聽謝遜的訊息,自己偕同周芷若、韓林児,南下前赴淮泗。到山東境内,便見蒙古敗兵,曳甲丟盔,蜂湧而來。

無忌等見到大隊敗兵,便避道而行,後來見到一兵落單,抓住了一加逼問,知道韓山童在淮北連打幾個大勝仗,殺得元軍連失數處要地。張無忌等不勝之喜,加緊趕路,一到魯皖邉界,已全是明教義軍的天下。義軍中有人認得韓林児,急足報到元帥府。三人將近濠州時,韓山童等已知張無忌到來,率領了朱元璋、徐達、常遇春、鄧愈、湯和等大將,迎出三十里外。衆人久别重逢,倶各大喜。韓山童親手向張無忌獻上酒菜,鑼鼓喧天,兵甲耀眼,擁入濠州城中。周芷若騎在馬上,跟隨在無忌之後,左顧右盼,心想這番風光,雖是不及大都皇帝皇后「遊皇城」的華麗輝煌,却也頗足以快慰平生。

到得城中,衆將逐一躬身參見。當晩濠州城中大開筵席,恭迎教主駕到,韓山童聽児子説起身遭丐幫擒獲,全仗教主相救,更是一再稱謝。無忌在城中歇了數日,楊逍、范遙、殷天正、殷野王、鐵冠道人、説不得、周顚、五行旗諸掌旗使等得到訊息,陸續從各地來會,城中宴會不斷。又過數日,青翼蝠王韋一笑和彭瑩玉也前脚後脚的到達。

無忌急問謝遜的音訊。彭瑩玉在大都未得絲毫消息。韋一笑却道︰「屬下在河北遇見丐幫的掌棒龍頭,意圖不利於本教,屬下開了他一個小小玩笑。只怕金毛獅王失陥於丐幫也未可知。當時屬下未知謝兄已回中原,否則定當前赴丐幫一探,」無忌當下説起謝遜被丐幫擒去又復失蹤的種種情由,楊逍、范遙、殷天正等倶是足智多謀之士,反覆思量,均無頭緒。范遙道︰「這個擕帶八名少女的黃衫女子,不知是何來歷,説不好謝兄的行蹤,要著落在她身上尋訪出來。」殷天正道︰「那個到處繪下本教記號,引得教主在冀北大兜圏子之人,想必與此事有極大関連。」群豪雖是見多識廣,却無一人説得出這黃衫女子究是何等樣人,只得勸無忌且自寬心,都道︰「這黃衫女子的言語行事,對教主顯無惡意。金毛獅王若是落在她的手中,定然無恙。瞧此女之意,最多不過想探詢屠龍寶刀的下落而已。」無忌掛懷難釋,一時却也無可如何,只得派出五行旗下教衆,分頭赴各處打聽。

明教義軍大戰數場,雖均獲勝,損折也極慘重,兩三月内,義軍忙於休養整頓、招募新兵,不克再與元軍大戰。彭瑩玉那晩見到周芷若自盡,雖是不明底細,但猜想得到這對青年男女之間,不是醋海興波,便是大鬧别扭。范遙等又知無忌與趙明之間,感情頗不尋常,若是明教教主和蒙古郡主結親,於抗元復國的大業,實是爲害非小,眼見目下並無大事,倶勸無忌早日與周芷若完婚。無忌想起周芷若已懷有身孕,此事原也延擱不得,當即允可。殷天正擇定三月十五爲黃道吉日。明教上上下下喜氣洋洋,都爲教主的婚事忙了起來。

此時明教威震天下,東路韓山童在淮泗一帶迭克大城,西路徐春輝在鄂北豫南,也是連敗元兵。教主大婚的喜訊一傳出,武林人士的賀禮便如潮水般湧到。崑崙、崆峒等自居名門正派,與明教向有嫌隙,但一來大都萬法寺中張無忌曾出手相救,已於各派有恩,二來周芷若是峨嵋掌門,是以各派掌們也都遣人送禮到賀。武當派張三丰自己不到,親書「佳児佳婦」四字立軸。一部手抄的「太極拳經」,命宋遠橋、兪連舟、殷利亨三大弟子到賀。其時楊不悔已與殷利亨成婚,跟著丈夫來到濠州。張無忌笑著上前請安,大聲叫道︰「六師嬸!」楊不悔滿臉通紅,拉著無忌的手,回塵前事,心中又是歡喜,又是傷感。

無忌生怕陳友諒、宋青書奸心未息,乘機爲害,當下派韋一笑爲謝禮使,前赴武當。

無忌暗中將宋青書害死莫聲谷、又圖謀害張三丰之事,詳細跟韋一笑説了,囑咐他上武當山拜見張三丰後,便與兪岱岩、張松溪爲伴,防備陳友諒的奸謀,須待宋遠橋等回歸武當,再行告辭。韋一笑聽了陳友諒與宋青書的作爲。一張瘦臉氣得鐵青,狠狠的道︰「自遵教主的訓諭,韋一笑不敢再吸人血,這一次撞到了這兩個奸賊,非將他二人吸個血乾皮枯不可。」無忌忙道︰「那陳友諒嘛,韋兄不妨順手除去。宋青書是我宋大師伯的獨生愛子,武當派未來的掌門,且由武當派自行清理門戸,免傷我宋大師伯之意。」靠一笑答應了,拜别而去。

到得三月初十,峨嵋衆女俠擕帶禮物,來到濠州,丁敏君託人帶來賀禮,人却未到。三月十五正日,明教上下人衆個個換了新衣。拜天地的禮堂設在濠州第一大富紳的廳上,懸燈結綵,裝點得花團錦簇。殷天正爲男方主婚,常遇春爲女方主婚。鐵冠道人爲濠州總巡,佈置教中弟子,四下巡査,以防敵人混入搗亂。湯和統率義軍精兵,在城外駐紮防敵。這日上午,少林派、華山派也派人送禮到賀。

申時一刻,吉時已屆,號炮連聲鳴響。楊逍和范遙將衆賀客迎到大廳,贊禮生朗聲贊禮,殷利亨和韓林児陪著張無忌出來。絲竹之聲響起,衆人眼前一亮。八位峨嵋派青年女俠,陪著周芷若婀婀娜娜的步出大廳。男左女右,新郎新娘並肩而立。贊禮生朗聲喝道︰「拜天!」張無忌和周芷若正要在紅氈毹上倒拜,忽聽得大門外一人嬌聲喝道︰「且慢!」青影一閃,一個青衣少女笑吟吟的站在庭中,却是趙明。

群豪一見趙明,登時紛紛呼喝起來。明教和各大門派中的高手,許多都吃過趙明的苦頭,没料到她竟敢孤身闖入險地。有的性子莽撞些的,便欲上前動手。楊逍雙臂一張,也喝一聲︰「且慢!」向衆人道︰「今日是敝教教主和峨嵋派掌門大喜之日,趙姑娘光臨到賀,便是咱們嘉賓。衆位且瞧峨嵋派和明教的薄面,將舊日樑子,放過一邉,不得對趙姑娘無禮。」他向説不得和彭瑩玉使個眼色,兩人已知其意,繞到後堂,即行出去査察,且看趙明帶了多少高手同來。楊逍向趙明道︰「趙姑娘請這邉上坐觀禮,回頭在下再敬姑娘三杯水酒。」

趙明微微一笑,説道︰「我有幾句話跟張教主説,説畢便去,容日再行叨擾,」楊逍道︰「趙姑娘有什麼話,待行禮之後再説不遲。」趙明道︰「行禮之後,已經遲了。」

楊逍和范遙對望一眼,知她今自是存心前來攪局,無論如何要立時設法阻止,免得將一場喜慶大事,鬧得{\upstsl{尷}}尬狼狽,舉座不歡。楊逍踏上兩步,説道︰「咱們今日賓主盡禮,趙姑娘務請自重。」他心下已打定了主意。趙明若再搗亂,只有迅速出手,點了她的穴道,制住她再説。趙明向范遙道︰「苦大師,人家要對我動手,你幫不幫我!」范遙眉頭一皺,説道︰「郡主,世上不如意事十居八九,既是如此,也是勉強不來。」趙明笑吟吟的道︰「我偏要勉強。」轉頭向無忌道︰「張無忌,你是明教教主,男子漢大丈夫,説過的話作不作數?」

無忌一見趙明到來,心中早已怦怦亂跳,只盼楊逍能打開僵局,勸得她好好離去,聽她突然問到自己。只得答道︰「我説的話,自然作數。」趙明道︰「那日我救了你殷六叔之命,你答應替我做三件事,不得有違,是也不是?」張無忌道︰「不錯。你要我借屠龍寶刀一瞧,你不但已瞧到了,還將寶刀盜了去。」這數十年來,江湖上人人関心這「武林至尊」屠龍刀的下落,一聽已入趙明手中,登時群情聳動,吵嚷起來。

趙明道︰「到底屠龍刀是在誰的手中,只有金毛獅王謝大俠才知,你不妨去問他一問。」謝遜已返中原之事,武林群豪多不知聞,聽到趙明提及「金毛獅王」,衆人喧嘩之聲登寂,張無忌道︰「我義父現居何處,我日夕掛念,甚盼姑娘示知。」趙明笑了一笑,説道︰「我要你做三件事,言定只須不違武林中俠義之道,你就須得遵從。借屠龍刀一觀之事,雖然做得不算道地,但這把刀我總算是看到了,後來寶刀被盜,也不能怪你。這第一件事,算你已經辦到。現下我有第二件事要辦,張無忌,當著天下衆位英雄豪傑之前,你可不能言而無信。」無忌道︰「你要我辦什麼事?」

楊逍插口道︰「趙姑娘,你有什麼事奉託於敝教教主,既有約定在先,只要不背武林道義,别説張教主可應允,便是敝教上下,也當盡心竭力。此刻是張教主和新夫人參拜天地的良辰吉時,别事暫且擱在一旁,請勿多言阻撓。」他説到後來,口氣也頗爲嚴厲。趙明却是神色自若,竟似没將這位威震江湖的明教光明左使放在心上,懶洋洋道︰「我這件事,更是要緊,片刻也延擱不得。」她突然走上幾步,到了張無忌身前,提高脚跟,在他耳邉輕輕説道︰「這第二件事,是要你今天不得與周姑娘拜堂成親。」

張無忌呆了一呆,道︰「什麼。」趙明道︰「這就是第二件事了,至於第三件,以後我想到了再跟你説。」她這幾句話雖然説得甚輕,但周芷若、宋遠橋、殷利亨,以及陪伴新娘的八位蛾嵋女俠,却都聽見了,各人都是不禁色爲之變。峨嵋八女身上雖然没擕帶兵刃,但裡袖中均是暗扣拳掌,倘若趙明再説不遜之言,辱及峨嵋掌門。那就免不了給她吃些苦頭。張無忌搖頭道︰「此事恕難從命。」趙明道︰「你答應過的話不作數麼?」無忌道︰「咱們言明在先,不得違背俠義之道。我和周姑娘既有夫婦之約,倘若依你所言,那便是違背了這個『義』字。」趙明冷笑道︰「你若和她成婚,那纔眞是不孝不義。大都遊皇城之時,難道你没見到你義父如何遭人暗算?」張無忌怒火上沖,大聲道︰「趙姑娘,今日我敬你是客,讓你三分,若再在此胡説八道,得罪莫怪。」趙明道︰「這第二件事,你是不肯聽從的了?」無忌心腸甚軟,想起她以郡主之尊,不惜抛頭露面,在群豪之前求懇自己廢棄婚事,原是出於對自己的一片痴心,不由得聲音柔和了些,道︰「趙姑娘,事已如此,你一切\dash{}一切看開些吧,我張無忌是村野匹夫,如何\dash{}如何\dash{}」説了兩個「如何」,不知怎生接口才好。

趙明道︰「好,你瞧瞧這是什麼?」張開右掌,伸到無忌面前。無忌一看之下,大吃一驚,全身發抖,道︰「這\dash{}這是我\dash{}」趙明迅速合攏手掌,將那物揣入了懷裡,道︰「我這第二件事你依不依從,全由得你。」説著轉身便向大門外走去。她掌中到底有什麼東西,何以令無忌一見之下竟是這等驚惶失措,抑是誰也無法瞧見。周芷若鳳冠霞帔,雙目被紅布遮住了。只聽得無忌和趙明的對答,更是見不到半點外間的物事。

張無忌急道︰「趙\dash{}趙姑娘,且請留步。」趙明道︰「你要就隨我來,要就快些和新娘子拜堂成親。男児漢狐疑不決,别遺終身之恨。」她口中朗朗説著這幾句話,脚下並不停留,直向大門外走去。無忌叫道︰「趙姑娘且慢,一切從長計議。」眼見趙明反而加快脚步,急忙搶上前去,叫道︰「好,我依你,今日便不成婚。」趙明停了一停,道︰「那你跟我來。」無忌搶上兩步。回頭看周芷若亭亭站著,心中歉仄無已,待要向她解釋幾句,却見趙明又在向外走去,眼前之事緊急萬分,須得當機立斷,一咬牙,便追向趙明身後。

張無忌剛追到大門邉,突然間身旁紅影一閃,一個人追到趙明身後,紅袖中伸出一雙纖纖素手,五根手指向趙明頭頂插了下去。這一下兔起鵠落,迅捷無比,出手的正是新娘周芷若。無忌心念一動︰「這一招好厲害?芷若從何處學得如此精妙的武功。」眼見她五根手指已將趙明的頂門籠罩住了,趙明雖是學過各家各派的精妙招術,竟是無法解脱這五指齊抓之厄,五根插將下去,立是破腦之禍。當下無忌不及細想,左足一點,竄上前去便扣周芷若的脈門。周芷若左手手肘倏地撞了過來,波的一聲輕響,正好撞中在無忌胸口。無忌體内九陽神功立時發動,卸去了這一撞的勁力。但已感胸腹間氣血翻湧,脚下微一踉蹌。范遙眼見危急,故主情殷,一掌向周芷若肩頭推去。周芷若左手微揮,輕輕一拂,巧妙無比的拂中了范遙手腕穴道。范遙半身酸麻,再也無法出手。

但總算這麼阻得一阻,趙明已向前搶了半步,避開了腦門要害,只感肩頭一陣劇痛,周芷若右手五指已插入趙明右肩近頸之處,無忌「啊」的一聲,伸掌向周芷若推去。周芷若頭上所罩的紅布並未揭去,可是聽風辨形,左掌迴轉,便斬無忌手腕。無忌決不想傷害周芷若,只是見她招數太過凌厲,一招之間便能要了趙明的性命,迫於無奈。只有招架勸阻。不料周芷若上身不動,下身不移,雙手連施八下險招。無忌使出乾坤大挪移心法,這纔擋住,這八攻八守,只是在電光石火般的一瞬之間過去。大廳上群豪屏氣凝息,無不驚得呆了。趙明肩受重傷,摔倒在地,五個傷孔中血如泉湧,登時使染紅了半邉衣裳。

周芷若霍地住手不攻,説道︰「張無忌,你受這妖女迷惑,竟然捨我而去。」無忌道︰「芷若,請你諒解我的苦衷。咱倆婚姻之約,張無忌絶無反悔,只是稍遲數日\dash{}」周芷若冷冷的道︰「你去了便休再回來,只盼你日後不要反悔。」趙明咬牙站起,一言不發的向外便走,肩頭鮮血,流得滿地都是。群豪雖然見過江湖上不少異事,但今日親見二女爭夫,血濺華堂,新娘子頭遮紅巾而毀傷情敵,無不神眩心驚,誰也説不出話來。

張無忌一頓足,説道︰「義父於我恩重如山。情深似海,芷若,芷若,盼你體諒。」説著追了趙明出去。殷天正、楊逍、宋遠橋、殷利亨等不明其中原因,誰也不敢攔阻。

周芷若霍地伸手扯下頭上紅巾,朗聲説道︰「各位親眼所見,是他負我,非我負他,自今而後。周芷若和姓張的恩斷義絶。」説著揭下頭頂珠冠,伸手抓去,手掌中抓了一把珍珠,抛開鳳冠,雙手一搓,滿掌珍珠盡數變成粉末,簌簌而落,説道︰「我周芷若不雪今日之辱,有如此珠。」殷天正、宋遠橋、楊逍待欲善言相慰,要她候張無忌歸來,問明再説,却見周芷若雙手一扯,嗤的一響,一件繡滿金花的錦袍撕成兩片,抛在地下。她飛身而起,美妙無比約在半空中一個轉折,上了屋頂。楊逍、殷天正等一齊追上,只見她輕飄飄的有如一片紅雲,向東而去,輕功之佳,竟似不下於青翼蝠王韋一笑。楊逍等料知追趕不上,征了半晌,重行回入廳來。

一場喜慶大事被趙明這麼一鬧,轉眼間風流雲散,明教上下固感臉上無光,前來道賀的群豪也是十分没趣。來人紛紛猜測,不知趙明拿了什麼要緊物事給張無忌看了,以致害得他急急追出,聽無忌言中會意,似乎此事和謝遜有重大関連,但其中眞相,却是誰也不知。峨嵋衆女俠低聲商議幾句,便即氣憤憤的告辭。殷天正連連致歉,説務當率領張無忌,前來峨嵋金頂,鄭重陪罪,再辦婚事,千萬不可傷了兩家和氣。

\chapter{千里赴難}

峨嵋衆女不置可否,當即分頭前去尋覓周芷若,群雌粥粥。痛斥男子漢薄倖無良,那也不在話下。

原來趙明握在手掌中給張無忌觀看之物,乃是一束金黃色的頭髮。無忌一看之下,登時認出這是謝遜的頭髮。要知謝遜所練内功與衆不同,更是生具異稟,因此中年以後,一頭長髮轉爲金黃之色,但這顏色和西域色目人的金髮,却又是截然有異,無忌一看之下,便能分辨。他想謝遜的頭髮既被趙明割下一截,想必身子已落入她掌握之中,即便不然,她也必知曉謝遜的下落。他對謝遜和親生父親並無分别,一見金髮,只覺普天之下,更無一事比救出義父更加要緊。他心知趙明既持此髮而來,只要自己和周芷若拜了天地,趙明一怒之下,不是去殺了謝遜,便是於他大大不利,可是當著群豪之前,却又不能向周芷若解釋其中苦衷。要知衆賀客之中,除了明教和武當派諸人之外,幾乎人人欲得謝遜而甘心,不是報復昔日謝遜大肆殺戮之仇。便是意圖奪取屠龍寶刀。是以他一見趙明奔出,明知萬分對不起周芷若,終是以義父性命爲重,跟著追來。

他一出大門,只見趙明提氣疾奔,肩頭鮮血,沿著大街一路灑將過去。無忌吸一口氣,竄出數丈,當即攔在趙明身前,説道︰「趙姑娘,你别逼我做不義之人,受天下英雄唾罵。」趙明肩頭受傷極重。初時憑著一口眞氣支持,勉力而行,她一聽無忌之言,説道︰「你\dash{}你\dash{}」眞氣一洩,登時摔倒在地。無忌俯身道︰「你先跟我説,我義父在那裡?」趙明道︰「你帶著我去救他,我跟\dash{}跟你\dash{}指路。」無忌道︰「他老人家性命可是無恙?」趙明有氣没力的道︰「你義父\dash{}義父落了成崑手中。」

張無忌聽到「成崑」兩字,這一驚當眞是心膽倶裂。他此時已知當日成崑在光明頂上乃是詐死,此人武功既高,計謀又富,謝遜和他仇深似海,既是落入他的手中,則兇險不可言喩。趙明道︰「你一個人不成,叫\dash{}叫楊逍他們同去\dash{}」一面説,一面伸手指向西方,突然間,腦袋向後一仰,已是暈了過去。張無忌想像義父此刻身歷的苦楚,五内如焚,抱起趙明,匆匆撕下衣襟,替她裹了傷口,招手命街旁一個明教教徒過來,囑咐道︰「你快去稟報楊左使,命他急速率領衆人,向西趕來,説我有要事吩咐。」那教徒垂手答應,飛奔著前去稟報。

無忌心想能早到一刻便好一刻,世事難料,説不定便因半刻之間的延擱,便致救不到謝遜的性命,當下抱起趙明,快步走到城門邉,命守門將士牽過一匹健馬,飛身而上,向西急馳。

馳了十餘里,只覺懷中趙明的身子漸漸寒冷,伸手搭一搭她的脈膊,更是跳得十分微弱,無忌驚慌起來,揭開她傷口裹著的衣襟,只見五個指孔深及肩骨,傷口旁都變成紫黑之色,顯然中了劇毒。無忌大是驚疑︰「芷若是峨嵋弟子,如何會使這種陰毒武功?她出招之凌厲狠辣,更比滅絶師太尤爲了得,實是令人大惑不解。」眼見若不急救,趙明登時便要毒發身死,他一身新郎裝束,身邉如何會擕帶得療毒的藥品?微一沉吟,當即跨下馬背,抱著趙明縱身往左首的山上竄去。他凝目草叢之中。尋找去毒的草藥,可是一時之間,連最尋常的草藥也無法找到。

無忌一顆心怦怦亂跳,翻過一座山又是一座山,口中只是喃喃禱祝。突然間眼睛一亮,只見右前方一條小瀑布旁,生有四五朶紅色小花,那正是去毒的妙藥,無忌大喜,輕輕將趙明放在地下。越過兩道山澗,走到瀑布之旁,正要俯身去摘那紅花,忽聽得身後一個女子聲音喝道︰「住手!」

無忌轉過頭來,只見隔澗站著三個女子,中間一人身材瘦長,身穿尼姑裝束,他識得是峨嵋派弟子靜慧,另外兩個玄衣少女,也是峨嵋弟子,却不知姓名。只見靜慧手持長劍,滿臉怒容,喝道︰「張教主,你在這裡幹什麼?」張無忌反手一抓,已將三朶紅花摘在手裡,深恐一加延擱,便救不了趙明性命,當即將紅花放在口中咀嚼口含含糊糊的道︰「靜慧師太,你身上可帶得有『佛光去毒丹』?」那「佛光去毒丹」,乃是峨嵋派的去毒聖藥,功效可比這些小紅花強得多了,峨嵋弟子,下山行道,身上大都擕帶,一來治病救人,二來自防不測。

靜慧道︰「我有便怎樣?無便如何?」無忌道︰「這位趙姑娘身中劇毒,請師太施捨三枚靈丹,救她一救。」靜慧長眉軒起,厲聲道︰「這妖女是害死師父的兇手,峨嵋弟子,人人恨不得食其肉而寢其皮。哼,哼!她身中劇毒,那正是惡貫滿盈,張教主,我又來問你,今日是你和本派掌門的大喜日子,何以受了這妖女蠱惑,三言兩語。便\dash{}便抱了她離開喜堂?你置我掌門人於何地?置峨嵋派於何地?」張無忌一揖到地,説道︰「靜慧師太,我救人要緊,實有説不出的苦衷,一切只好請各位原諒。我愛芷若之心,至死不變,皇天后土,實所共鑒。」靜慧聽到他説「救人要緊」四字,只道他所説要救之人便是趙明,決没想到另行牽涉謝遜在内,心下更是憤怒,大聲道︰「當時好端端地,這妖女又没受傷?就算你要救她,儘可與我掌門人行禮成婚之後,再行施救。哼,當眞是花言巧語,一派胡言。」

無忌聽她言語糾纏,心知多挨一刻,趙明肩頭的毒傷便重一刻,當下眉頭一皺,搶到趙明身旁,撕開她肩頭一些衣服,將口中嚼爛的紅花,敷到她傷口之上。便只這片刻的耽擱,傷口附近的肌肉更紫更黑,腫得更加高了,不由得暗暗心驚,趙明若是便此傷發而死,不但令他慘痛難當,而且謝遜和成崑眼下却在何處,一時也是不易知曉,茫茫人海,却向何處找去?説不定謝遜竟爾遭了成崑毒手,不及相救,那可是千古之恨了。

他雙手顫抖,正替趙明敷藥,忽聽得腦後金刃劈風,嗤的一劍,刺了過來。無忌左手一帶,三根手指平平貼在劍刃之上,一推一掠,已將靜慧這一劍化解了開去。他一招並不回頭,但聽風辨器之準,實已到了化境,須知這一手「推三阻四」只須有厘毫之差,三根手指便給長劍削了下來。常人若非高出對手數籌,便是面面相對,也不敢輕易使用,何況背後出招,盲目卸劍?

靜慧這一劍被他輕描淡冩的用三指化開,剛要再度出招,那知對方這一招餘力未盡,她身形一晃,踉踉蹌蹌的跌開三步。靜慧又驚又怒,明知不是無忌的對手,但一來今日之事實在辱人太甚,難以容忍,二來眼前的趙明正是害死師父的大仇人,峨嵋弟子無一不是痛心泣血,發下毒誓,務必殺之而甘心。眼見這大仇人身受毒傷,昏迷不醒,只須纏住張無忌使他不得施救,多半不須用劍,便能殺敵報仇,當下縱聲喝道︰「柯師妹、歐師妹,一齊上啊!」兩個玄衣少女長劍出鞘,一齊向張無忌攻到。

無忌苦笑道︰「我和三位無冤無仇,何必苦苦相逼?」一面説,一面揮動左手,以乾坤大挪移的心法,見招拆招,卸開三人劍招,右手不停的用紅花敷傷。靜慧三人長劍如虹,化成濛濛劍氣,圍繞在無忌身旁,竟是刺不到他一片衣角。靜慧猛地裡大喝一聲,長劍顫動,疾向躺在地下的趙明身上刺去。無忌「嘿」的一聲,左手中指彈出。{\upstsl{噹}}的一響,靜慧只覺虎口劇痛,再也拿捏不住,青光閃閃,三尺長劍飛向天空。

這柄長劍飛到天空,拍的一響,斷成兩截,兩段斷劍相距丈餘,急速落下。靜慧手中没了兵刃,倏出一指,點向無忌後心死穴。無忌見她下手毒辣,心下不禁有氣︰「你便是要阻我救人,也不必制我死命。」左手伸轉,在她手腕上一搭、撲的一下,將靜慧的身子直摔了出去。柯歐二女見師姊連吃兩次大虧,嚇得不敢再行上前動手。

無忌敷完藥後,見趙明氣息微弱,傷處黑氣漸漸擴大,蔓延到了胸前和背脊,情知這紅花解不了劇毒,回過身來,對靜慧道︰「靜慧師太。你是佛門子弟,慈悲爲本,請賜三枚佛光去毒丹,張無忌終身感激大恩。」靜慧怒道︰「叫你救活了這妖女,便是我峨嵋派大敵。周掌門從此和你恩斷義絶,再也無可挽回。」那姓歐的少女一直想勸無忌幾句,只是師姊在前,没她説話的地步,這時再也忍耐不住,説道︰「張教主,你和我周師姊這樣\dash{}這樣好,何必爲了這妖女\dash{}而這樣\dash{}這樣\dash{}你還是回去,和周師姊\dash{}和周師姊\dash{}吧?」她説了這幾句話,已是脹得粉臉通紅。無忌聽她雖然辭不達意,説得斷斷績續,但語意却是極爲誠懇,也不禁有些感動,説道︰「多謝姑娘美意,但我不能見死不救。」但見趙明肩頭一片黑氣越來越濃,皺眉道︰「姑娘,請你施我三枚佛光去毒丹,張無忌必當重報。」這姓歐的少女心軟,伸手入懷便要去取丹藥,眼光向靜慧一瞥,只見她滿臉煞氣,心中一驚。一隻手雖然摸到了藥瓶,却不敢從懷中掏將出來,靜慧喝道︰「歐師妹,你忘了恩師的血仇麼?你若將丹藥給人,我當場一掌便劈死了你。」

無忌怒道︰「妳不給那也罷了,何以攔阻旁人?」靜慧對無忌的武功實是頗爲忌憚,雙掌交錯,護在胸前,一步步向後的退開,叫道︰「柯師妹、歐師妹,咱們走!」

她這一示怯,意欲逃走,登時引起了無忌搶藥之心。他雙眉一軒,説道︰「靜慧師太,我救人要緊,你再不給藥,在下可要得罪了。」説著向靜慧身前走去。靜慧左掌一揚,右掌從左掌掌底穿出,一股勁風。向無忌面門撲來。無忌身形微斜,讓她手掌掠著自己左頰而過,便是相差寸許,没能打著,就是這要一側,左手突然翻轉,已點中了她左肩的穴道。

靜慧上身被制。飛起右足,踹到無忌腿上,這一脚踢得極快,無忌也不退讓,却將她踹來之力反震了回去。靜慧只感右足足底「湧泉穴」中一股熱氣湧將上來,登時全身酸麻,再也動彈不得。

那姓歐少女求道︰「張教主。你别\dash{}别傷我師姊。」無忌道︰「我不傷她。請你從她懷中取丹藥給我。」靜慧喝道︰「歐師妹,峨嵋弟子,寧死不屈,你敢動我一動?」

無忌見兩個少女神情猶豫,此刻趙明生死懸於一線,再也顧不得什麼「男女授受不親」的俗禮,便伸手到靜慧懷中去取丹藥,靜慧「{\upstsl{呸}}」的一聲,一口唾沫向無忌臉上噴去。無忌側頭讓開,手中摸到三個小磁瓶,便取了出來。正在此時,那姓柯少女連人帶劍,直向無忌後心撲到。

無忌閃身一讓,眼見她劍勢兇猛,生怕收不住招而刺中靜慧,當下右手一轉,引開她的劍尖,拿著了三枚磁瓶的左手,却因此無意中碰到了靜慧右肩琵琶骨處的肌膚。無忌吃了一驚,急忙縮手,不敢再和靜慧目光相對,拔開三隻磁瓶的瓶塞,分辨藥性,將三枚佛光去毒丹嚼爛了,一半餵入她的口中,一半敷在她的肩頭,心想她中毒甚深,三枚丹藥只怕不彀,索性將一瓶佛光去毒丹揣在懷内,説道︰「得罪!」解開靜慧的穴道,抱起趙明,向西便奔。忽聽得身後那姓歐少女驚呼︰「師姊,不可。」

無忌回過頭來,只見青光一閃,靜慧左手持劍,將自己右肩齊琵琶骨處卸了下來,霎時間滿地都是鮮血,靜慧身子搖晃,却並不摔倒。無忌大吃一驚,知道是自己適纔取藥之時闖的禍,爲了避開柯姓少女一劍。左手撞到了靜慧肩胸之間的肌膚。她是出家清修的女尼,身子被男人碰到,引爲奇恥大辱,憤激之際,竟爾出此烈性行逕。無忌身如雷閃颼颼颼出指如風,連點她傷處附近七八處穴道,止住猶似泉湧的血流。靜慧厲聲道︰「魔教惡賊,滾開!」便在此時。遠處連連響起哨聲,那柯姓少女取出竹哨,放在口中,與之應答。無忌知道這是峨嵋派招呼同儕的訊號,一回頭,只見七八人疾馳而來。

無忌心想峨嵋後援到來,靜慧的性命當可無礙,但自己若與群女朝相,只有越加糾纏不清,當下回身抱起趙明,飛奔而去。那柯姓少女却也不敢追趕。無忌生怕再與峨嵋弟子撞到,不敢行走大路,只是落荒而走。

奔出三十餘里,趙明嚶嚀一聲,醒了過來,低聲道︰「我\dash{}我可還活著麼?」無忌見佛光去毒丹生效,心中大喜,笑道︰「你覺得怎樣?」趙明道︰「肩上癢得很,唉,周姑娘這一手功夫當眞厲害。」無忌將她輕輕放下,再著她肩頭時,只見黑氣絲毫不淡,只是趙明的脈搏却已不如先前微弱。無忌略一沉吟,知道丹藥不足以拔毒,於是俯口到她肩頭,將傷口中毒血,一口口的吸將出來,吐在地下,腥臭之氣,沖鼻欲嘔。趙明星眸迴斜,伸手撫著無忌的頭髮,嘆道︰「無忌哥哥,這中間的原委,你想到了嗎?」

無忌吸完毒血,到山溪中去嗽了口,回來坐在她的身畔,問道︰「什麼原委?」趙明道︰「周姑娘是名門正派的弟子,怎地會這種陰毒的邪門武功?」無忌道︰「我一直很覺奇怪,不知是誰教她的?」趙明嫣然一笑,道︰「定是魔教那派的小賊教的了。」無忌笑道︰「魔教中魔頭雖多,誰也不會這種武功,只有青翼蝠王吸人頸血,張無忌吸人毒血,差相彷彿。」趙明斜倚在他身上,説道︰「今日耽誤了你的洞房花燭,你怪我不怪?」不知如何,無忌此刻心中甚感喜樂,除了掛念謝遜安危之外,反覺比將要與周芷若拜堂成親之時,更是平安舒暢,到底是什麼原因,却也説不上來,然而要他承認喜歡趙明攪翻了喜事,可又説不出口,只得道︰「我自然怪你,日後你與那一位英俊瀟灑的郡馬爺拜堂之時,我來大大搗亂一場,決不讓你太太平平的做新娘子。」趙明蒼白的險上一紅,笑道︰「你來搗亂,我一劍殺了你。」無忌忽然嘆了口氣,黯然不語。趙明道︰「你嘆什麼氣?」無忌道︰「不知道那位郡馬爺前生做了什麼善事,修來這樣的好福氣。」趙明笑道︰「你現下再修,也還來得及。」無忌心中抨然一動,道︰「什麼?」趙明臉一紅,不再接口了。

説到這裡,兩人誰也不好意思往下深談,休息一會,無忌再替她敷藥,抱起她身子,又向西行。趙明靠在他肩頭,粉頰和他左臉相貼,無忌鼻中聞到的是粉香脂香,手中抱的是溫玉軟玉,不由得意馬心猿,神魂飄飄,倘若不是去營救義父,眞的要放慢脚步,在這荒山野嶺中慢慢的走它一輩子了。

這一晩便在濠州之西的荒山中露宿一夜,次日無忌和趙明到了一處小鎭,買了一匹健馬。趙明毒傷極難拔淨,身子虛弱,無力單獨騎馬,只好靠在無忌身上,兩人同鞍而坐。如此行了五日,已到河南境内,這日正行之間,忽見前面塵頭大起,有百餘騎疾馳而來,只聽得鐵甲鏘鏘,正是蒙古的騎兵。無忌將馬勒在一旁,讓開了道。

那些蒙古騎兵見張無忌衣飾華貴,手中抱著一個青年女子,也均不以爲意,從無忌身旁縱騎而過。數百名騎兵走完,隔著數十丈處,又是一隊騎者,但這群人行列並不整齊,或前或後,行得疏疏落落,無忌一瞥之下。暗叫︰「不好!」急忙轉過了頭,原來他見到人群之中,竟有趙明手下的「神箭八雄」在内。他雖無所畏懼,但與這些人撞見,總是多生枝節。

這二十餘人從無忌身旁行過,只因無忌和趙明的臉朝向道旁,神箭八雄竟無一人知覺。待這一批人過完,無忌拉過馬頭,正要向前再行,忽聽得蹄聲輕捷,三乘馬如煙如霧的衝到。中間是匹白馬,馬上乘客錦袍金冠,手揮長鞭,兩旁各是一匹栗馬,鞍上赫然是鹿杖客和鶴筆翁玄冥二老。無忌待要轉身,鹿杖客却同時看到了二人,叫道︰「郡主娘娘休慌,救駕的來了。」鶴筆翁縱聲長嘯,聲音遠遠的傳了出去。「神箭八雄」等聽到嘯聲,一齊圏轉馬頭,登時將無忌圏在中間。

無忌一怔,眼睛向懷中的趙明望却,臉上似説︰「你安排下伏兵,向我襲擊嗎?」却見趙明神色頗爲憂急,登知自己錯怪了,心中立時舒坦,只要知道趙明並非出賣自己,那麼任何危難,均可鎭定應付。只聽趙明道︰「哥哥,没想到在這裡見到你,爹爹好吧?」無忌聽她叫出「哥哥」兩字,纔留神騎白馬的那個錦袍青年,認得他是趙明之兄庫庫特穆爾,漢名叫作王保保。無忌曾在大都見過他兩面,只因全神貫注在玄冥二老身上,没去留心這個看來武功並不甚高的青年。

王保保乍見嬌妹,不禁又驚又喜。無忌認得他,他却不識無忌,皺眉道︰「妹子,你\dash{}你\dash{}」趙明道︰「哥哥,我中了敵人暗算,身受毒傷,幸蒙這位張公子救援,否則今天見不到哥哥了。」鹿杖客將嘴湊到王保保耳邉,低聲道︰「小王爺,那便是魔教的教主張無忌。」王保保久聞張無忌之名,只道趙明受他挾制,在他脅迫之下,方出此言,右手一揮,玄冥二老已欺到無忌左右五尺之處,神箭八雄中的四雄也各彎弓搭箭,對準無忌的後心。王保保道︰「張教主,閣下是一教之主,武林中成名的豪傑,欺侮舍妹一個弱女子,豈不教人恥笑,快快將她放下,今日饒你不死。」趙明道︰「哥哥,你何出此言?張公子確是有恩於我,怎説得上『欺侮』二字?」王保保認定妹子是在敵人淫威之下,不得不如此説,朗聲道︰「張教主,你武功再強,總是雙拳難敵四手,快快放下我妹子,今日咱們兩下各不相犯。我王保保言而有信,不須多疑。」

無忌心想︰「趙姑娘毒傷甚重,隨著我千里奔波,不易痊可,既與她兄長相遇,還是讓她隨兄而去,由王府名醫調治,於她身子有益。」便道︰「趙姑娘。令兄要接你回去,咱們便此别過,只請示知我義父所在,我自去設法相救。咱們後會有期。」説到這裡,心下甚是黯然神傷,明知和她漢蒙異族,官民殊途,雙方仇怨甚深,但臨别之際,實是不勝戀戀之情。

不料趙明道︰「一路上我没跟你説謝大俠的所在,内中自有深意,我只答應帶你前去找他,却不能告訴你地方了。」無忌一怔,道︰「你重傷未愈,跟著我長途跋涉,大是不宜,還是與令兄同歸的爲是。」趙明目光中滿是執拗之色,道︰「你若是撇下我,便不知謝大俠的所在。我身子一天好似一天,路上走走,反而好得快,回到王府去。可悶也悶死了我。」無忌向王保保道︰「小王爺,你勸勸令妹吧。」王保保大奇,心念一轉,冷笑道︰「嘿嘿,你裝模作樣,弄什麼鬼?你手掌按在我妹妹死穴之下,她自是只好遵你吩咐,口中胡説八道。」無忌一躍而起,縱身下地。

神箭八雄中有二人只道他要出手向王保保襲擊,颼颼兩箭,挾著極強的勁風,向無忌背心射了過來。無忌有心要顯顏色,左手一引一帶,使出乾坤大挪移神功,兩枝狼牙箭回轉頭去,勁風更厲,拍拍兩音,將發箭二人手中的長弓弓背劈斷了。若非那二人閃避得快,還得身受重傷,只見雙箭餘勢不衰,疾插入地,箭尾鵰翎兀自顫動不已。衆人見他這等功夫,除了玄冥二老外,自王保保以下,無不駭然變色。

無忌離趙明遠遠地,説道︰「趙姑娘,你先回府養好傷勢,我等再謀良晤。」趙明搖頭道︰「王府中的醫生,那裡有你醫道高明?你送佛送上西天吧。」王保保見無忌已遠離妹子,但妹子仍是執意與他同行。不由得又驚又恐。向玄冥二老道︰「有煩兩位保護舍妹,咱們走!」玄冥二老齊聲應道︰「是!」走到趙明馬旁。趙明朗聲道︰「鹿鶴二位先生,我有要事須隨同張教主前去辦理,正嫌勢孤力弱,你二位隨我同去吧。」玄冥二老向王保保望了一眼道︰「魔教的魔頭行事邪僻,郡主不宜和他多所交往,還是跟小王爺一起回府的爲是。」趙明秀眉微蹙,道︰「兩位現下只聽我哥哥的話,不聽我話了麼?」鹿杖客陪笑道︰「小王爺是爲郡主娘娘好,他的金石良言,乃是出於愛護郡主的至意。」趙明停了一聲,向王保保道︰「哥哥,我行走江湖,早得爹爹允可,你不用爲我擔憂,我自己會當心的。你見到爹爹時,代我問候請安。」

王保保知道父親向來寵愛嬌女,原是不敢過份逼迫,但任由她孤身一人隨魔教的教主而去,無論如何不能放心,見趙明伏在馬鞍之上,嬌弱無力,却是提韁欲往西,當即張開雙臂攔住,道︰「賢妹,爹爹隨後便來,你稍待片刻,稟明了爹爹再走不遲。」趙明笑道︰「爹爹一到,我便走不成了。哥哥,我不管你的事,你也别來管我。」王保保再向張無忌打量,見他長身玉立,面目英俊,聽著妹子的語氣,顯已鍾情於他,心想明教造反作亂,乃是大大的叛逆,朝廷的對頭,妹子竟然受此魔頭蠱惑,爲禍非小,當下左手一揮,喝道︰「先將這魔頭拿下了。」鹿杖客揮動鹿杖,鶴筆翁舞起鶴筆。化作一片黃光,兩團黑氣,齊向無忌身上罩下。玄冥二老功力深厚,較之殷天正,謝遜等人猶有過之,二老聯手夾攻,那幾乎是從所未有之事,無忌也是絲毫不敢怠慢,凝神應敵。

趙明深知玄冥二老的厲害,無忌武功雖強,但以一敵二,手中又無兵刃,只怕折了威名,叫道︰「玄冥二老,你們若是傷了張教主,我稟明爹爹,可不能相饒。」王保保怒道︰「亂臣賊子,人人得而誅之。玄冥二老,你們殺了這小魔頭,父親和我均有重賞。」他頓了一頓,又道︰「鹿先生,小王加贈四名美女,定教你稱心如意。」他兄妹二人,一個要殺,一個下令説不得損傷,倒使玄冥二老左右做人難了。鹿杖客向師弟使個眼色,低聲道︰「捉活的。」無忌突然展開聖火令上所載武功,上身微斜,右臂彎過,從莫名其妙的方位轉了過來,拍的一下,重重打了鹿杖客一個耳光,喝道︰「你倒捉捉看。」鹿杖客突然間吃了這個大虧,又驚又怒,但他究竟是第一流高手,心神絲毫不亂,將一根鹿頭杖使得風雨不透。無忌欲侍再使偸襲,打倒一人,一時之間竟是無法可施。

趙明馬韁一提,縱馬便行,王保保馬鞭揮出,刷的一鞭打在她坐騎的左眼之上。那馬吃痛,長聲嘶鳴,前足提了起來。趙明傷後虛弱,險些児從鞍上摔下,怒道︰「哥哥,你定要阻攔麼?」王保保道︰「好妹子,你今日聽我的話。哥哥慢慢跟你陪罪。」

趙明道︰「哥哥,你今日若是阻我,有一個人不免死於非命。這位張教主從此恨我入骨,你妹子\dash{}你妹子也難以活命。」王保保道︰「妹子説那裡話來?汝陽王府中高手如雲,自能保護你周全。這小魔王别説出手傷你,便是要再見你一面,也未必能彀。」趙明嘆道︰「我就怕不能再見他。那我\dash{}我是不想活了。」蒙古女子不甚拘泥禮法,他兄妹二人又是情誼素篤,素來無話不説,趙明情急之下,竟是毫不隱瞞,將傾心於張無忌的心意坦然説了出來。

王保保怒道︰「妹子你忒也胡塗,你是蒙古王族,堂堂的金枝玉葉,怎能向蠻子賤狗垂青?若讓爹爹得知,豈不氣壞了他老人家?」左手一揮,登時又有三名高手上前夾攻張無忌。只是無忌和玄冥二老此時各運神功,數丈方圓之内勁風如刀,那三名高手竟是插不下手去。趙明叫道︰「張公子,你要救義父,須得先救我。」王保保見妹子意不可回,莫要眞是阻她不住,父親面前如何交代!當下猿臂一伸,將她抱了過來,放在身前鞍上,雙腿一夾,縱馬便行。趙明的武功本較兄長爲高,但重傷之下,四肢全無力氣,無可抗拒,只有張口大呼︰「張公子救我,張公子救我!」

無忌呼呼兩掌,使的是十成勁力,將玄冥二老逼得倒退三步,身形一晃。展開輕功,向王保保馬後追來。玄冥二名和那些高手提氣急追,要待纏住無忌。以便小王爺脱身。但無忌每當五人追近。便呼呼呼向後拍出數掌,使的均是降龍十八掌中那一招「神龍擺尾」,他這一招掌法雖未學得至精至妙,然九陽神功威力奇大,每掌拍出。玄冥二老使須收脚閃避,不敢直攖其鋒。如此連阻三阻,無忌已是追及奔馬。縱身躍在半空。抓住王保保後頸。他這一抓之中,暗藏拿穴手法,王保保上身登時一陣酸麻,雙臂放開了趙明,身子已被無忌提起,向鹿杖客投擲過去,鹿杖客只怕小王爺受傷。急忙張臂接住,無忌却已抱起趙明,躍離馬背,向左首山坡上奔去。

鶴筆翁和其餘高手大聲吼喝,隨後追來。可是這山峰高達數百丈,登高追逐,最是考較輕功,玄冥二名内力極強,輕功在武林中却非一流,反是另外四五人遠在鶴筆翁前頭。無忌在山上拾起幾枚石子,連珠擲出,登時有二人被打中要害,骨碌碌的滾下山來。餘人暗自吃驚,雖在小王爺監視之下不敢停步,脚下却是放得緩了。眼見無忌抱趙明越奔越遠,再也追趕不上。王保保氣得破口大罵,連叫︰「放箭,放箭!」自己彎弓搭箭,颼的一箭,向無忌後心射去,他弓力甚勁,但終於相距太遠,箭尖離無忌後心尚有丈餘,一枝箭便掉在地下。

趙明抱著無忌頭頸,知道衆人已追趕不上,一顆心纔算落地,嘆了口氣道︰「總算我有先見之明,没告知你謝大俠的所在,否則你這没良心的小魔頭,焉肯出力救我。」無忌轉過一個山坳,脚下仍是絲毫不緩,説道︰「你跟我説了,自己回府養傷,豈不兩全其美?又何苦既得罪了兄長,又陪著我吃苦?」趙明道︰「我既決意跟著你吃苦,這位兄長嘛,遲早總是要得罪的。我只怕你不許我跟著你,别的我什麼都不在乎。」無忌雖知她對自己甚好,但有時念及,總想這不過是少女懷春,一時意動,没料到她竟是糞土富貴,棄尊榮猶如敝屣,一往情深若此,低下頭去,但見她蒼白憔悴的臉上情意盈盈,眼波流動,説不盡的嬌媚無限,忍不住俯下頭去」在她微微顫勁的櫻唇上一吻。

一吻之下,趙明滿臉通紅,激動之餘,竟爾暈了過云,無忌深明醫理,料知無妨。心中却又加深了一層感激,突然想起︰「芷若待我,那有這般好!」

\chapter{迴護情郎}

趙明暈去一陣,便即醒轉,見無忌若有所思,問道︰「你在想甚麼?定是想周姑娘了?」無忌也不隱瞞,點了點頭,道︰「我想到有些對她不起,辜負了她。」趙明道︰「你後悔不後悔?」無忌道︰「當時我要跟她拜堂成親,想到你時,不由得好生傷心,此刻想到了她,却又對她好生抱歉。」趙明笑道︰「那你心中對我愛得多些,是不是?」無忌道︰「我老實跟你説吧,我是對你又愛又恨,對芷若是又敬又怕。」趙明笑道︰「哈哈!我寧可你對我又愛又怕,對她是又敬又恨。」無忌笑道︰「現下又不同了,我對你是又恨又怕,恨的是你拆散了我美滿良緣,怕的是你不肯賠我。」趙明道︰「賠什麼?」無忌笑道「今日要你以身相代,賠還我的洞房花燭。」趙明滿臉飛紅,忙道︰「不,不!那要將來跟我爹爹説好\dash{}等我向哥哥賠禮疏通,這纔\dash{}這纔\dash{}」無忌道︰「要是你爸爸一定不肯呢?」趙明嘆道︰「那時我嫁魔隨魔,只好跟著你這小魔頭,自己也做個小魔婆了。」無忌扳起了臉,喝道︰「大膽的妖婦,跟著張無忌這淫賊造反作亂,該當何罪?」趙明也扳起了臉,正色道︰「罰你二人在世上做對快活夫妻,白頭偕老,死後打入十八層地獄,萬劫不得超生。」

兩人説到這裡,一齊哈哈大笑,忽聽得前面一人朗聲道︰「郡主娘娘,小僧在此恭候多時。」聲音清越,却震得滿山鳴響,顯是内力十分深厚。無忌吃了一驚,急忙止步,只見山後轉出三名番僧,一人穿紅,一人穿黃,第三人極爲矮小,却是身披金色袈裟。那穿紅袍的番僧雙手合什,躬身説道︰「小僧奉王爺之命,迎接郡主回府。」

趙明並不認得三僧,説道︰「三位從何處來?怎地我並不相識?」紅衣番僧道︰「小僧摩罕法!」指著穿金色袈裟的番僧道︰「這位是小僧的師伯鳩尊者!」指著穿黃袍的番僧道︰「這位是小僧的師兄摩罕聖。我三人從西天竺來,投入汝陽王爺府中,適逢郡主外出,是以今日方得拜見。」説著三人躬身行禮。無忌眉頭微蹙,尋思︰「這三人的功力已是不弱,他師伯和師兄當更加了得。我以一敵三,未必能勝,何況手中又抱著一人。」趙明道︰「你們等在這裡幹麼麼?」摩罕聖舉了舉手上的一隻白鴿,並不説話。趙明早知這是兄長的白鴿傳訊,通知了父親,是以被這三人迎頭截住,看來父親手下高人盡出,四處攔阻,不只這三個番僧而已。

她一側頭,見無忌臉有憂色,於是湊嘴到他耳邉,低聲道︰「這三個和尚很難打發麼?」無忌點了點頭。趙明微一沉吟,心念已決,在他耳邉低聲道︰「謝大俠的所在,我便跟你説了。日後你是否負我,憑你良心。」她想無忌一人要脱身而去,當是易如反掌,自己不能爲了一己私情,累得謝遜性命。無忌這時却是捨不得和她分離,道︰「你不用擔心,咱們衝過去再説。」眼見山道狹窄,左邉下臨深谷,右邉是陡削的絶壁,除了硬衝,更無别法。只聽摩罕法道︰「郡主身上有傷,王爺極是擔心,盼咐小僧,速速迎接郡主芳駕。」他雖是天竺人,華語倒説得頗爲明白。那鳩尊者和摩罕聖却一言不發。鳩尊者尖嘴削腮,垂首低眉,宛如入定。摩罕聖却是挺胸凸肚,氣勢雄壯。

趙明道︰「我爹爹在那裡?」摩罕法道︰「王爺便在山下相候,渴欲一見愛女傷勢如何。」趙明笑道︰「你的中國話説得很好啊。好吧!張公子,咱們走吧。」她是要走到一處較易脱身的所在,再行覓路逃走,擠在這狹狹的山道之中,實無施展餘地。那知摩罕法從背上取下一隻布袋,迎風一展,成了一隻軟兜,他拿著一端,摩罕聖握住了另一端。

摩罕法恭恭敬敬的道︰「請郡主坐轎。」趙明笑道︰「我不愛坐轎,就是喜歡他抱著。」無忌情知多言敗事,大踏步使往前闖去。這三名得到飛鴿傳書。已知無忌是個厲害的勁敵,摩罕聖手肘一彎,一肘便向胸口撞來。無忌縱身而起,躍過鳩尊者頭頂。突覺一股冷冰冰的寒風,直襲下盤。無忌左手劈出,和鳩尊者對了一掌,猛覺這股陰寒的掌風變成熾熱異常,原來鳩尊者一掌之中,頃刻間陰陽互變,的是極奇幻、極高明的掌法,非中土之所有。無忌所習九陽神功,得之於來自天竺的達摩祖師,他一聽到摩罕法自稱亦來自天竺,便早深具戒心,絲毫不敢怠忽,這一掌乃是用了八成力,鳩尊者猛哼一聲,向後退了三步,無忌却是借了他一推之力,向山下縱出七八丈遠,抱著趙明,向前急奔。交換這一掌後,他已試出鳩尊者功力較己尚差一籌,掌法雖然奇妙,那也遠遠不及自己的乾坤大挪移心法,認眞較量武功,自己可操勝算。

只聽得三名番僧嘰哩咕嚕的叫喊,自後緊緊追來,輕功竟是大爲不弱,但無忌内力雄渾,雖是懷抱趙明,脚下可越奔越快,將這三名番僧抛得老遠。翻過一道山嶺,眼見三僧已是追趕不上,正想覓條岔路躱開,忽聽得號角之聲嗚嗚吹起,三十餘名蒙古弓箭手快步而來,攔住了當路,兩旁山坡上也突然出現蒙古官兵,擂木巨石,紛紛打下。只是他們不敢傷害趙明,但求截住他二人的去路,矢石倒不向無忌身上招呼。無忌見此路不通,忙向嶺左的山坡上欺去,忽聽得鑼聲{\upstsl{噹}}{\upstsl{噹}},山峰上紅旗招展,一排弓箭手排在嶺上。原來四下裡都有伏兵,已是身陥重圍之中,無忌若是單身一人,原可冒險衝出,但擕同趙明後身手究不靈便,倘若她身中一矢一石,不幸傷及要害,豈非終身憾事。

他微一沉吟,索性回頭奔去,行不到半里,只見三名番僧飛步而至。無忌將趙明往地下一放,場道︰「要性命的,快快讓道,否則莫怪我手下無情。」鳩尊者踏上一步,一招「排山掌」,雙掌齊出,當胸向無忌推到。無忌心想到此地步,力強者勝,縱然將這番僧擊落深谷之中,那也是無可如何了,當下左掌揮出,一引一帶,將對方這股雄猛無儔的掌力撞了回去。鳩尊者叫道︰「阿米阿米哄!阿米阿米哄止!」似是唸咒,又似罵人。趙明不肯吃虧,叫道︰「你纔阿米阿米哄!」

只見鳩尊者登登登退了三步,摩罕聖和摩罕法兩名番僧伸掌抵住他的背心,將他推了回來。鳩尊者招式不變,又是一招「排山掌」擊至。張無忌心想今日要帶著趙明越出重圍,用力之地尚多,不願跟他硬拚,耗費眞力,當下又以挪移乾坤心法,將他勁力化開,不料手指剛觸及他掌緣,突然間如磁吸鐵,手指竟和他掌緣牢牢黏住了。鳩尊者大叫︰「阿米阿米哄!阿米阿米哄!」無忌連掙兩掙,都是没能掙脱,只得運起九陽神功,反擊過去。

這一次居然没將鳩尊者推動,但見摩罕聖、摩罕法二僧四隻手出力抵在鳩尊者背心,三名番僧六眼圓睜,神情猙獰可怖。無忌猛然想起︰「曾聽太師父言道,天竺武功之中,有一種併體連功之法。這三個番僧集三人之力和我對掌,倒是不易取勝。」他生怕後面追兵到來,利在速戰速決,不耐久耗,一聲清嘯,手上已加了一成力。只見三番僧額頭登時大汗淋漓,頂上升起絲絲白氣,突然間哇的一聲,摩罕法噴出一口鮮血。説也奇怪,他這口鮮血一噴,顯是受了不輕的内傷,但無忌却感對方推來的勁力反而增了一成。無忌體内眞氣鼓盪,手上勁力再增。摩罕聖滿臉通紅,張口一枝血箭,噴向鳩尊者頸中。無忌只覺對方掌力如潮而至,洶湧澎湃,莫可與禦。

無忌倒退兩步,將那股巨力卸脱了五成,再運勁反擊過去。三番僧眼見不支,摩罕聖和摩罕法全身搖晃,差一點便要跌倒。鳩尊者一張口,一口鮮血向無忌臉上噴來。無忌側身一讓,胸口猛地受到對方掌力,猶如萬斤巨鎚之一擊,但覺丹田中氣血翻湧,也似要嘔出一口鮮血,方始暢快。他萬没料到這三名天竺僧的内功如此怪異,噴一口鮮血,勁力便強一成,但從三人神情看來,顯然已是強弩之末,只須再支持片刻,三人非脱力衰竭不可。他定一定神,九陽神功源源發出,拍的一聲,摩罕法左足跪在地上,手掌仍未離開鳩尊者後心。

無忌心中正自一喜,忽聽得背後脚步輕響,一人輕飄飄的一掌拍了過來。無忌吃了一驚,左掌向後拍出,待要將這掌化開,不料他的乾坤大挪移心法,全恃九陽神功爲根基,此時全力對付身前三名番僧,拍向身後這一掌只不過平時的二成力道。但覺一股陰寒之氣從手掌中直傳過來,霎時間上身發顫,已擋不住前後四名高手的同時夾擊,身形一晃,便即俯身撲倒。趙明驚呼︰「鹿先生,住手!」原來正是鹿杖客以玄冥神掌急施龍擊。

趙明撲上前去,遮住無忌身子,喝道︰「那一個敢再動手。」鹿杖客本想補上一掌,就此結果了這個生平第一勁敵的性命,見郡主如此相護,只得罷手退開。他縱聲長嘯,示意已然得手,招呼同伴趕來,並道︰「郡主娘娘,王爺只盼郡主回府,並無他意。此人是大逆不道的反叛,郡主何苦如此?」趙明本想狠狠申斥他一番,但轉念一想,莫要激動他的怒氣,竟爾傷了無忌性命,當下忍住了口邉言語,扶起無忌的身子。

過不多時,鸞鈴聲響,三騎馬從山道上馳來,一是鶴筆翁,一是王保保,最後一人竟是汝陽王親自到了。三人馳到近處,翻身下馬,汝陽王皺眉道︰「明明,你幹麼不聽哥哥的話,在這裡胡鬧?」趙明眼泪奪眶而出,説道︰「爹!你叫人這樣欺負女児。」汝陽王上前幾步,伸手要去拉她。趙明右手一翻,白光閃動,從懷中取出一柄匕首,抵在自己胸口,叫道︰「爹,你不依我,女児今日死在你的面前。」汝陽王嚇得退後兩步,道︰「有話好説。你要怎樣?趙明伸左手拉開自己右肩衣衫,扯下繃帶,露出五個指孔,其時毒氣已去,傷口未愈,血肉模糊,更是可怖。汝陽王見她傷得這樣厲害,心疼愛女,連聲道︰「怎樣了?怎樣了?幹麼傷得這等厲害?」

趙明指著鹿杖客道︰「這人心存不良。意欲奸淫女児,我抵死不從,他\dash{}他\dash{}便抓得我這樣,求爹爹\dash{}爹爹作主。」鹿杖客只嚇得魂飛天外,忙道︰「小人斗膽也不敢,豈\dash{}豈有此事?」汝陽王向他瞪目怒視,哼了一聲,道︰「好大的膽子!韓姬之事,我已寬恩不加追究,却又冒犯我女児起來了。拿下!」

這時他隨侍的武士已先後趕到,一聽王爺喝道︰「拿下」,雖知鹿杖客武功了得,還是有四名武士欺近身去。鹿杖客又驚又怒,心想他父女骨肉至親,郡主惱我傷她情郎,竟來反咬我一口,常言道「疏不間親」,郡主又是詭計多端,我怎爭得過她?當下揮出一掌,將四名武士逼退,嘆氣道︰「師弟,咱們走吧!」鶴筆翁尚自遲疑。趙明叫道︰「鶴先生,你是好人,不像你師兄是好色之徒,快將你師兄拿下,我爹爹升你的官,重重有賞。」玄冥二老武功卓絶,只是熱中於功名利祿,這纔以一代高手的身份,投身王府以供驅策。鶴筆翁素知師兄好色貪淫,聽了趙明之言,倒也信了七八成,升官之賞又令他怦然心動,只是他與鹿杖客同門至好,却又下不了手,一時猶豫難決。

鹿杖客臉色慘然,顫聲道︰「師弟,你要升官發財,便來拿我吧。」鶴筆翁嘆道︰「師哥,咱們走吧!」和鹿杖客並肩而行。玄冥二老威震京師,汝陽王府中衆武士對之敬若天人,誰敢出來阻擋?汝陽王雖是連聲呼喝,衆武士也只虛張聲勢、裝模作樣一番,眼見玄冥二老揚長下山去了。

汝陽王道︰「明明,你既已受傷,快跟我回去調治。」趙明指著張無忌道︰「這位張公子見鹿杖客欺侮我,路見不平,出手相助。哥哥不明就裡,反説他是什麼叛逆反賊,爹爹,我有一件大事要跟張公子去辦,事成之後。再同他來一起叩見爹爹。」汝陽王聽她言中之意,竟是要委身下嫁無忌。他聽児子説過,這人乃是明教教主。汝陽王這次離京南下,便是爲了調兵遣將,對付淮河和豫那一帶的明教反賊,如何肯讓女児隨此人而去?於是問道︰「你哥哥説,這人是魔教的教主,這没假吧?」趙明道︰「哥哥最會胡説八道。爹爹,你瞧他有多大年紀,怎能做反叛的頭腦?」汝陽王見張無忌不過二十一二歳,受傷後臉色憔悴,失去英挺秀拔之氣,更加不像一個統率數十萬軍的大首領。但他素知女児狡譎多智,又想明教爲禍邦國,此人就算不是教主,只怕也是魔教中的重要人物,須縱他不得,便道︰「將他帶到城裡,細細盤問。只要不是魔教中人,我自有升賞。」他這樣説,已是顧到了女児的面子,免得她當著這許多人面前恃寵撒嬌。

四名武士答應了,便走近身來。趙明哭道︰「爹爹,你眞要逼死女児麼?」匕首向胸口刺進半寸,鮮血登時染紅衣衫。汝陽王驚道︰「明明,千萬不可胡鬧。」趙明哭道︰「爹爹,女児不孝,已私下和張公子結成夫婦,腹中有了他的骨肉。你要殺他,不如先殺了女児。」

她此言一出,不但汝陽王和王保保大吃一驚,張無忌也是大出意料之外,雖知她是全力相護,却也萬料不到她竟會捏造這種謊言。汝陽王連連跥脚,道︰「此話可眞?此話可眞?」趙明道︰「這等可恥之事,女児若非迫不得已,豈肯當衆輕賤自身、羞辱父兄?爹爹你就算少生了女児這個人,放女児去吧。」汝陽王雙手不住扯著自己鬍子,滿額都是冷汗。他命將統兵、交鋒破敵,都是一言立決,但今日遇上了愛女這等{\upstsl{尷}}尬事,竟是束手無策。王保保道︰「妹子,你和張公子都已受傷,且暫伺爹爹回去,請名醫調理,然後由爹爹主婚,明媒正娶。爹爹得一乘龍快婿,我也有一位英雄妹夫,豈不是好?」他這番話説得好聽,趙明却早知乃是緩兵之計,張無忌一落入他們手中,焉有命在?只怕立時便將他送到大都,斬首示衆,便道︰「爹爹,事已如此,女児嫁雞隨雞、嫁犬隨犬,是死是活,我都隨定張公子了。你和哥哥有什計謀,那也瞞不過我,終是枉費心機。眼下只有兩條路,你肯饒女児一命,就此罷休。你要女児死,原也不費吹灰之力。」

汝陽王怒道︰「明明,你可要想明白了,你跟了這反賊去,從此不再是我女児。」趙明柔腸百轉,原也捨不得爹爹哥哥,想起平時父兄對自己的疼愛憐惜,心中有如刀割,但自己只要稍一遲疑,登時便送了無忌性命,眼下只有先救情郎,日後再求父兄原諒,便道︰「爹爹,哥哥,這都是明明不好,你\dash{}你們饒了我吧。」

汝陽王見女児意不可回,深悔平日溺愛太過,放縱她行走江湖,以致做出這等事來,素知她從小任性,倘加威逼。她定然刺心自殺,不由得長嘆一聲,泪水潸潸而下,嗚咽道︰「明明,你多加保重。爹爹去了\dash{}你一切小心。」趙明點了點頭,不敢再向父親多望一眼。

汝陽王轉身緩緩走下山去,左右牽過坐騎,他恍如不聞不見,並不上馬,走出十餘丈,他突然回過身來,説道︰「明明,你的傷不礙事麼?身上帶得有錢麼?」趙明含泪顆了點頭。汝陽王對左右道︰「把我的兩匹馬去給郡主。」左右衛士答應了,將馬牽到趙明身旁,擁著汝陽王走下山去。鳩尊者等三名天竺僧委頓在地,無法站起,六名王府武士兩個服侍一個,扶著跟在後面。過不多時,衆人走得乾乾淨淨,山道下只剩得無忌和趙明二人。

無忌盤膝而坐,潛運神功,將鹿杖客這一掌中所含的陰寒之氣,慢慢逼了出來。只是鹿杖客這一掌偸襲,適逢他以全力和天竺三僧較量内勁,後背藩籬盡撤,失了護體眞氣,以致受傷著實不輕。他以九陽眞氣在體内轉了三轉,嘔出兩口瘀血,纔去了胸口閉塞之氣,睜開眼來,只見趙明滿臉都是擔憂的神色。無忌柔聲道︰「趙姑娘,這可苦了你啦。」趙明道︰「這會児你還是叫我「趙姑娘」麼?我不是朝廷的人了,也不是郡主了,你\dash{}你心裡,還當我是個小妖女麼?」無忌慢慢站起身來,説道︰「我問你一句話,你得據實告我。我表妹殷離臉上的劍傷,到底是不是你割的?」趙明道︰「不是?」無忌道︰「那麼是誰下的毒手?」趙明道︰「我不能跟你説。只要你見到謝大俠,他自會跟你説知詳情。」無忌奇道︰「我義父知道詳情?」趙明道︰「你身上内傷未愈,多問徒亂心意。我只跟你説,倘若你査明實據,殷姑娘確是爲我所害,不用你下手,我自會在你面前自刎而死。」無忌聽她説得斬釘截鐵,不由得不信,沉吟半晌,道︰「那多半是波斯明教那艘船上的水手之中,暗伏高手,施展什麼邪法,半夜裡將咱們一起迷倒,害了我表妹,盜去了倚天劍和屠龍刀。由此看來,尋出義父之後,非到波斯走一遭不可。見見小昭!」

趙明抿嘴一笑道︰「你自己想去見見小昭,便捏造些緣由出來。我勸你不用胡思亂想了,早些養好了傷,咱們上少林寺是正經。」無忌奇道︰「上少林寺幹麼?」趙明道︰「救謝大俠啊。」無忌更是奇怪,道︰「我義父是在少林寺麼?怎麼會在少林寺之中?」趙明道︰「這中間的原委曲折,我是不知内情,但謝大俠身在少林寺内,却是千眞萬確,我跟你説,我手下有一死士,削髮爲僧,在少林寺出家。這是他遞出來的訊息。」無忌道︰「嘿!好厲害!」這「好厲害」三字,也不知是讚趙明的手段,還是説局勢的險惡,説了這話後,便即低頭沉思。他心中一覺煩惱,牽動内息,忍不住哇的一聲,又吐了一口血。

趙明急道︰「早知你傷得這等要緊,又是這等沉不住氣,我便不跟你説了。」無忌坐下地來,靠在山石之上,待要寧神靜息,但関心則亂,總是無法鎭定,説道︰「少林神僧空見,是被我義父以七傷拳打死的,少林僧俗上下,二十餘年來誓報此仇。我義父落入了他們手中,那裡還有命在?」趙明道︰「你不用著急,有一件東西却救得謝大俠的性命。」無忌忙問︰「什麼東西?」趙明道︰「屠龍寶刀。」無忌一動念間,已然明白,屠龍刀號稱「武林至尊」,少林派數百年來領袖武林,對這把寶刀自是欲得之而甘心,他們爲了得刀,必不肯輕易加害謝遜,只是一番折辱,定然難免。

趙明又道︰「我想救謝大俠之事,還是你我二人暗中下手的爲是。明教英雄雖衆,但如大舉進襲少林,雙方損折必多。少林派倘若眼見抵擋不住明教進攻,謝大俠即將救出,説不定使出下策,下手將謝大俠害了。」無忌聽她想得周到,心中不禁感激,道︰「明妹,你説得是。」

趙明第一次聽他叫自己爲「明妹」,心中説不出的甜蜜,但一轉念間,想到父母之恩,戚友之親,從此付諸東流,一去不可復返,又是不禁神傷。無忌知道她的心意,却也無從勸慰,只是想︰「她此生已然託付於我,我不知如何方能報答她的深情厚意?芷若和我有婚姻之約,我却又如何能彀相負?唉!眼前之事,終是設法救出義父要緊,這等児女之情,且自放在一旁。」他強力著站了起來,説道︰「咱們走吧!」

趙明見他臉色灰白,知他受傷著實不輕,秀眉微蹙,道︰「我爹爹愛我憐我,倒是不妨,只怕哥哥不肯相饒。不出兩個時辰,他又會派人來捉拿咱倆回去。」無忌點了點頭,他見王保保行事果決,是個極厲害的人物,原也不肯如此輕易罷手。目下兩人都是身受重傷,若是西去少林,實是步步荊棘,一時彷徨無策。趙明道︰「無忌哥哥,咱們急須離開此處險地,到了山下,再定行止。」無忌點了點頭,蹣跚著去牽過坐騎。待要上馬,只感胸口,一陣劇痛,竟是跨不上去。趙明右臂用力,咬著牙一推,將他送上了馬背,但這麼一用力,胸口被匕首刺傷的傷口又流出不少鮮血。她掙扎著也上了馬背,坐在無忌身後。本來是無忌扶她,一現下反而變成她扶無忌。二人喘息半晌,這纔縱馬前行,另一匹便跟在其後。

二人共騎,緩緩下山。趙明料想父親不致變計,哥哥當著父親之面,也不敢派人前來生事,但一兩個時辰之後,只要哥哥能設法暫時離開父親,一切便甚難料。二人下得山來,索性往大路上走去,折而東行,以免和王保保撞面。行得片刻,便走上了一道小路,趙明和無忌稍稍寬心,二人商量,便是王保保遣人追拿,也不易尋到這條偏僻小路上來。只要挨到天黑,入了深山中便有轉機。正行之間,忽聽得身後馬蹄聲響,兩匹馬急馳而來。趙尹花容失色,抱著無忌的腰,説道︰「我哥哥來得好快,咱倆苦命,終於難脱他的毒手。無忌哥哥,讓我跟他回府,設法求懇爹爹,咱們徐圖後會,天長地久,終不相負。」無忌苦笑道︰「令兄未必便肯放過了我。」剛説了這句話,身後兩乘馬相距已不過數十丈。趙明拉馬讓在道旁,拔出匕首,心意已決,若有迴旋餘地,自當以計脱身,要是哥哥決意殺害無忌,兩人便死在一塊。

那兩乘馬奔到身旁,却不停留,馬上乘者是兩名蒙古士兵,經過二人身旁,只是忽忽一瞥,便即越過前行。趙明心中剛叫出一聲︰「謝天謝地,原來只是兩個尋常小兵,非爲追尋我等而來。」兩名元兵却已勤慢了馬,商量了幾句,忽然圏轉馬頭,馳到二人身旁。一名滿腮鬍子的元兵喝道︰「兀那兩名蠻子,這裡好馬是那裡偸來的?」趙明一聽他的口氣,便知他見了父親所贈的駿馬,起意眼紅。汝陽王這兩匹馬,原是神駿無儔,兼之金鐙銀勒,華貴非凡。蒙古人愛馬如命,見了焉有不動心之理?趙明心想︰「這雖是爹爹相贈,但這兩個惡賊,若是恃強相奪,也只有給了他們。」打蒙古話道︰「你們是那一位將軍的麾下?竟敢對我如此無禮?」那蒙古兵一怔。問道︰「小姐是誰?」他見趙明和無忌衣飾華貴,跨下兩匹馬更是非同小可,再聽她蒙古話説得流利,倒也不敢放肆。

趙明道︰「我是花児不赤將軍的小姐,這是我的哥哥。咱二人路上遇盜,身上受了傷。」兩名蒙古兵相互望了一眼,突然放聲大笑。那鬍子兵大聲道︰「一不做,二不休,索性殺了這兩個娃娃再説。」抽出腰刀,縱馬便向無忌頭上砍來。趙明驚道︰「你們幹什麼?我告知將軍,教你二人四馬分屍而死。」

\qyh{}四馬分屍」是蒙古軍中重刑,犯法者四肢分縛於四匹馬上,一聲令下,長鞭揮處,四馬齊奔,登時將犯人撕爲四截,最是殘忍的刑罰。那絡腮鬍的蒙古兵獰笑道︰「花児不赤打不過明教叛軍,却亂斬部屬,拿咱們小兵出氣。昨日大軍譁變,早將你父親砍爲肉醬。在這児撞到你這兩隻小狗,那是再好不過。」説著一刀當頭砍下。趙明一提韁繩,縱馬避過,那兵正待追殺,另一個年紀較輕的元兵叫道︰「别殺這花朶児似的小姑娘,咱哥児倆先圖個風流快活。」那鬍子道︰「妙極,妙極!」趙明聽了此言,心念微動,便即縱身下馬,向道旁逃去。

兩名蒙古兵好色一齊下馬追來。趙明「啊喲」一聲,便即摔倒。那鬍子兵撲將上去,伸手欲按趙明背心。趙明手肘一撞,正中他胸口要穴,那鬍子兵哼也不哼,滾倒在旁。另一元兵没看清他已中暗算,跟著撲上,趙明依樣葫蘆,又撞中了他的穴道。這兩下撞穴,在她平時即是不費吹灰之力,此刻却累得氣喘吁吁,滿頭都是冷汗。她支撐了起來,却去扶無忌下馬,喝道︰「你這兩個犯上作亂的狗賊,還要性命不要?」兩名蒙古兵穴道被撞後,只覺上半身麻木不仁,雙手半點也動彈不得,下肢略有知覺,却也是酸痛難言,只道趙明跟著便要取他二人性命,那知聽她言中之意,竟有一線生機,忙道︰「姑娘饒命!花児不赤將軍並非小人下手加害。」趙明道︰「好,若是依得我一事,便饒了你二人的狗命。」兩名元兵不理是何難事,當即答應︰「依得!依得!」

趙明指著自己的坐騎,道︰「你二人騎了這兩匹馬,急向東行,一日一夜之内,必須馳出三百里地,越快越好,不得有誤。」二人面面相覷,做夢也想不到她的吩咐竟是如此。那鬍子兵道︰「姑娘,小人天大的膽子也不敢再\dash{}」趙明截住他話頭道︰「事機緊迫,快快上馬。路上倘若有人問起,你只須説這兩匹馬是市上買的,千萬不可提及我二人的形貌,知道了麼?」那二名蒙古兵仍是將信將疑,但禁不住趙明連聲催促,心想此舉縱然有詐,也勝於當場被她用匕首刺死,於是告了罪,一步步挨將過去,忍住脚底猶似萬針齊攢的疼痛,翻身上鞍。總算蒙古人自幼生長於馬背之上,騎馬比走路還要容易,雖然手足僵硬,仍能控馬前行。二兵均是一般的心意,生怕趙明中時胡思亂想,突然却又翻悔,待那馬行出十餘丈,雙腿急夾,縱馬疾馳而去。

無忌嘆道︰「明妹,你當眞智計無雙,令兄手下見到這兩匹駿馬,定料我二人已向東去。咱們此刻却又向何而行?」趙明道︰「咱是向西南方去了。」當下二人上了蒙古兵留下的坐騎,在荒野間不依道路,逕向西南。

這一路盡是嶇崎亂石,荊棘叢生,只刺得兩匹馬腿上鮮血淋漓,一跛一躓,一個時辰只行得二十來里。天色將黑,忽見山樹中一縷炊煙,梟梟升起,無忌喜道︰「前面有人家,咱們便去借宿。」趙明點頭稱是,二人行到近處,却見大樹掩映間露出黃牆一角,原來是座廟宇。趙明扶無忌下得馬來,將兩匹馬的馬頭朝向西方,從地下拾起一根荊枝,在馬臀上狂鞭數下。兩匹馬長聲悲嘶,快奔而去。趙明到處佈伏疑陣,但求引開王保保的追兵,至於失馬逃遁更是艱難,却也顧不得許多了,眼前是破釜沉舟,行得一步便算一步。二人相將扶持,挨到廟前,只見那屋宇倒還齊整,大門上匾額冩著︰「中嶽神廟」四字。趙明提起門環,敲了三下,隔了半晌無力答應,又敲了三下。忽聽得門内一個陰惻惻的聲音道︰「是人是鬼,到這裡來挺屍麼?」無忌聽這人語音頗具内功,竟甚個武林人物。心中微驚,向趙明望了一眼。

\chapter{偽裝和尚}

趙明尚未有何示意,只聽得「格格」聲響,那門緩緩開了。從那兩扇木門開動維艱的聲音中聽來,顯然這兩扇門極少開関。木門後出現一個人影,其時暮色蒼茫,他又身子有光,看不清此人面貌,但見他光頭僧衣,是個和尚。無忌道︰「在下兄妹二人,途中遇盜,身受重傷,欲在寶刹借宿一宵,請大師慈悲。」那人「哼」的一聲,險側側的道︰「出家人素不與人方便,不收。」便欲関門。趙明忙道︰「與人方便,自己方便,你未必没有好處。」那和尚道︰「什麼好處?」趙明伸手到耳邉摘下一對鑲珠的耳環,每隻耳環上都有一顆小指頭大小的珍珠,燦爛暈光,的是珍物。她將這對耳環遞了過去,交在那和尚手中。

那和尚一看這對珍珠耳環,再打量無忌與趙明二人,説道︰「好吧,與人方便,自己方便。」側身讓在一旁。趙明扶著無忌走了進去,那和尚引著二人穿過大殿和院子,來到東首的廂房,説道︰「你們就在這児住。」那房中無燈無火,黑洞洞地,趙明在床上一摸,床上只是一張草蓆,更無别物。只聽得外面一個十分洪亮的聲音叫道︰「郝四弟,你領誰進來了?」那和尚答道︰「兩個借宿的客人。」一面説,一面跨步出門。趙明道︰「師傅,請你佈施兩碗飯,一碟素菜。」那和尚道︰「出家人吃十方,不佈施!」説著揚長而去。趙明恨恨的道︰「這和尚可惡!無忌哥哥,你肚子很餓了吧?咱們得弄些吃的纔成。」

突然間院子中脚步聲響,共有七八人走來,火光一閃,房門被人用力推開,兩名僧人高舉燭台,照射無忌和趙明兩人。無忌一瞥之下,高高矮矮共是八名僧人,有的粗眉巨眼,有的滿臉橫肉,竟無一個善相之人。一個滿臉皺紋的老僧道︰「你們身上還有多少金銀珠寶,一起都拿出來。」趙明道︰「拿出來幹什麼?」老僧道︰「兩位施主有緣來此,正好撞到小廟要大做法事,重修山門,再裝金身。兩位身上的金銀珠寶,一起施捨出來。倘若吝嗇不肯,得罪了菩薩,那就麻煩了。」趙明怒道︰「那不是強盜行逕麼?」那老僧道︰「罪過,罪過。咱們八兄弟殺人放火,原是做的強盜勾當,最近被魔教逼得存身不住,只好改裝了和尚避禍。兩位施主有緣,肥羊自己送上門來,那倒是千載難逢之事。」

無忌和趙明一聽,不禁大吃一驚,没想到這八個和尚乃是大盜改裝。這老僧既是直言不諱,自是存心要殺了二人,決不致自吐隱事之後又再相饒。另一名僧人獰笑道︰「女施主不用害怕,咱們八個和尚強盜正少一位押廟夫人,你生得這般花容月貌,當眞是觀世音下凡,妙極!妙極!」趙明從懷裡掏出七八綻黃金,一串珠鍊,放在桌上,説道︰「財物珠寶,盡在於此。咱兄妹也是武林中人,各位須顧全江湖上義氣。」那老僧笑道︰「兩位是武林中人,那是再好也没有了,不知是那一派的門下?」趙明道︰「咱們是少林子弟。」其時少林派是武林中第一大派,趙明只盼這八人便算不是身出少林旁系,親友之中,多少也有人與少林有些淵源。

不料她此言一出,八名僧人一起哈哈大笑,説道︰「是少林子弟,那是再好也没有了。咱們鬥不過少林寺的老和尚,正好拿你們這兩個娃娃出氣。」説著伸手便來拉趙明手腕。趙明一縮手,那僧人拉了個空。無忌見眼前情勢危急之極,自己與趙明身上傷重,萬難抵敵,這幾年來會過多少武林中的成名人物,却難道今日反喪住於這八個三四流的小盜手中?不管怎樣?總不能眼睜睜的看著趙明受辱,便道︰「明妹,你躱在我身後,我來料理這八名小賊。」

趙明空有滿腹智計,到此也是束手無策,問道︰「你們是什麼人?」那老僧道︰「咱們是少林寺逐出來的叛徒,遇到别派的江湖人馬,倒還手下留情,但若碰到少林子弟,那是非殺不可。小姑娘,這位兄弟本來要你做個押廟夫人,現下知道你是少林門下,咱們只有先姦後殺,留不得活口了。」無忌低沉著嗓子,道︰「好哇,你們是圓眞惡僧的門下,是也不是?」那老僧「咦」的一聲,道︰「這倒奇了,你怎麼知道?」趙明接口道︰「咱們正是要上少林寺去,會見陳友諒大哥,推舉圓眞大師作少林方丈。」那老借道︰「善哉善哉!我佛如來,渡厄大千。」趙明道︰「是啊,咱們正好齊心合力,共成善舉。」

她此言一出,八名僧人登時哈哈大笑,原來這八名僧人確是圓眞和陳友諒一黨,由陳友諒引入,拜在圓眞門下。八人出身綠林,各有一手不弱武功,得到圓眞指點後,更是進了一層,近年來圓眞圖謀方丈一席之心甚急,四處收羅人才。只是少林寺戒律精嚴,每收一名弟子,均由執掌戒律的監寺詳加盤問,査明出身來歷,圓眞難以爲所欲爲。於是由陳友諒設計,招引各路都會豪傑、江洋大盜在寺外拜師,作爲圓眞的弟子,却不身入少林。只待時機到來,共舉大事。圓眞的武功何等深湛,只一出手,便令江湖豪士群相懾服,這些武林人物中來素慕少林派名門正派的威望,二來又見圓眞神功絶技,見所未見,自是皆願拜師。便有數人不願背叛本門的,圓與立刻下手除却,是以奸謀經營已久,却不敗露。那老僧口稱「我佛如來,渡厄大千」,却是他們這一黨見面的暗號,若是本黨中人,只須答以「花開見佛,心即蓬萊」,互相便知。趙明絶頂機智,一聽到老僧口氣中露出是圓眞弟子,便推算到圓眞圖謀方丈之位的心意,可是他們約定的暗號,却如何得知?

一名矮矮胖胖的僧人道︰「富大哥,這小妮子説什麼推舉我師作少林方丈,這訊息從何處得來?事関重大,却是不可不問。」這八人雖是落髮作了和尚,但相互間仍是「大哥」「二哥」相稱,不脱舊日綠林的習氣。無忌一聽他八人笑聲,便知要糟,苦於全身眞氣雖不渙散,但重傷後無法凝聚,不能在拳脚上使將出來,危急之際收束心神,強行聚氣,只覺熱烘烘的眞氣東一團、西一塊,始終難以依著脈絡運行。只見那老僧猶如鳥爪的五根手指伸了出來,便向趙明抓去。趙明無力擋架,只得身子一縮,避向了裡床。無忌俯首閉目,盤膝而坐。只盼能恢復得二三成功力,便能打發這八名惡賊了。

那矮胖僧人見無忌在這當口兀自大模大樣的運氣打坐,心下惱怒,喝道︰「這小子不知死活,老子先送他上西天去,免得在這裡礙手礙脚!」説著右臂抬起,骨骼格格作響,僧袍中似乎有氣鼓起,呼的一拳,打向無忌的「膻中穴」。趙明看得危急,一聲驚呼,只見那矮胖僧人一拳打後,右臂軟軟垂下,雙目圓睜,却是站著一動也不動了。那老僧吃了一驚,伸手拉了他一把,那胖僧應手而倒,竟已死去。餘下各僧友驚又怒,紛紛喝道︰「這小子有妖法,有邪術!」原來無忌傷後眞氣難凝,不能聚以傷敵,但體内的九陽神功却並未失去。那胖僧運勁於臂,全力擊向他的「膻中穴」。無忌的九陽神功攻敵不足,護身却是有餘,將敵人打來的拳勁盡數反彈過去不算,更因對方這麼強力一擊,引動了他體内九陽眞氣,勁上加勁,力中貫力,那胖僧如何抵受得住,立時便即斃命。

那老僧見多識廣,却知這是無忌借力打力之技,並非妖法邪術,自恃雙手鐵沙掌無堅不摧,左一掌,右一掌,呼呼拍出。

這老僧的鐵砂掌功夫,在綠林中也是赫赫有名,有個外號叫作「神砂破天手」。當那胖僧一拳打中張無忌的「膻中穴」而斃命,這老僧在旁看得清楚,只道無忌胸口裝有毒箭、毒刺之類利器,是以避開他胸口要害,雙掌都擊向他露在袖外的下臂。準擬先打折他的雙臂,同時震傷他的内臟,再行慢慢收拾。那知這兩掌斷樹裂石的掌力,撞到張無忌手臂之上,激動他體内九陽眞氣。反激而去。那老僧倒撞出去,其勢如箭,喀喇一聲大響,撞破窗格,一頭碰在庭中一株大槐樹上,腦漿迸裂,立時死於非命。

這老僧破窗而出,餘下各僧一時未知他的死活,同時有三僧齊向張無忌夾攻,一僧雙拳搗向無忌太陽穴,一僧以「雙龍搶珠」之招,伸指挖他眼珠,另一僧飛起右足。踢向他的丹田。無忌稍一低頭,避開雙眼,讓他兩指戮在眉間,但聽得砰砰、啊喲、{\upstsl{噗}}{\upstsl{噗}}數聲連響,三僧先後震死。第三僧飛足猛踢,力道極是強勁,竟將他這條右腿硬生生的震斷成爲兩截。無忌丹田處受了這一腿,眞氣鼓盪,右半邉身子中各處脈絡竟有貫穿模樣,不禁心下暗喜︰「可惜這惡僧震死得太早,要是他在我丹田上多踢幾脚,反能助我早復功力。看來我受傷雖重,恢復倒是不難,只須有十天到半月的養息,便能盡復舊觀。」

八僧中死了五僧,餘下三名惡僧嚇得魂飛天外,爭先恐後的搶出門去。只見老僧大哥死在樹旁,死狀甚慘,三僧更是害怕。三個人直奔到廟門之外,不見無忌追趕出來,三人站定了商議。一個道︰「這小子定是有邪法。」另一個道︰「我看不是邪法,他有極高的内功,反激出來傷人。」第三人道︰「不錯,咱們好歹要給死去了的兄弟報仇。」這三人雖然平素作惡多端,但頗有江湖好漢的義氣,八兄弟曾立下重誓,同生共死,決不相負。只是雖有決死之心,却明知不是無忌敵手,三人商議了半晌,一人忽道︰「這小子顯是受傷甚重,否則何以不追將出來?」另一人大喜道︰「不錯,多半他不會走動。五個兄弟以拳脚打他,他能以内力反激,咱們用兵刃砍他刺他,難道他當眞有銅筋鐵骨不成?」三個人商量定當,一人挺了柄長茅,一人提刀,一人持劍,走到院子之中。

只見東廂房中靜悄悄地,並無人聲。三人往撞破了窗格子中一張,只見張無忌仍是盤膝而坐,模樣極是疲累,身子搖搖晃晃,直有隨時摔倒的模樣。趙明拿著一塊手帕,在替他額頭抹汗。三僧使個眼色,總是不敢便此衝入。一僧高聲叫道︰「臭小子,有種的便出來,跟老爺鬥三百回合。」另一僧罵道︰「這小子有什麼本事,便只會使妖法害人。那是江湖間下三濫的把戲,卑鄙下流,無恥之尤。」他三人你一言,我一語,見無忌既不答話,又不下床,膽子越來越大,辱罵的言語也是越來越骯髒,不但無忌的祖宗十八代給他罵了個狗血淋頭,連趙明也變成了天下最淫亂的女賊。張趙二人有生以來,從未聽見過如此厲害的汚言穢語,大槩佛門弟子中口出惡言的,再也無人能勝得過這三位大和尚了。

無忌和趙明聽在耳裡,心中却並不生氣,他二人這時最擔心的,不是三僧再來尋仇,而是怕他們嚇得一去不回。此間離嵩山少林寺不遠,這三僧若去告知了成崑,那就性命休矣。無忌之傷不到十天以外,萬難痊可,即使成崑不至,只要來得一兩個二流高手,例如陳友諒之類的人物,無忌就要無法抵擋。兩人重傷之下,既要逃避王保保的追索,又要防備成崑一黨的襲擊,前後夾攻,絶無倖理,因此見這三僧去而復回,反而暗暗喜歡。

張無忌連受五僧襲擊,體内九陽眞氣反而有若干處所漸行凝聚,雖然仍是難以發勁傷敵,但心下已不若先前的驚惶擔憂。突然間砰的一聲,一僧飛脚踢開房門,搶了進來,青光閃處,紅纓抖動,他手中正是挺著一柄長矛。趙明叫聲︰「啊喲!」急將手中的匕首遞給無忌。無忌搖頭不接,不由得暗暗叫苦︰「我手上半點勁力也無,縱有兵刃,如何却敵?我血肉之軀,却不能抵擋兵器。」動念未已,颼的一聲,那長矛捲起一個槍花,紅櫻散開,矛頭已向胸口刺到。

他這一矛刺得快,趙明的念頭却也轉得快,伸手到無忌懷中,摸出一塊聖火令,對準矛頭來路,擋在無忌胸口。{\upstsl{噹}}的一響,矛頭正好戳在聖火令上。這聖火令以倚天劍之利尚自不能削斷,矛頭刺將上去,自是絲毫無損。這一刺之勁激動無忌體内九陽神功,反彈出去,但聽得「啊\dash{}」的一聲慘叫,矛桿直插入那僧人胸口。這僧人尚未摔倒,第二名僧人的單刀已砍向無忌頭頂。趙明深恐一塊聖火令擋不住單刀的刃鋒,雙手各持一塊聖火令,急速在無忌頭頂一放。這其間當眞是間不容髮,又是{\upstsl{噹}}的一聲響,單刀反彈,刀背將那惡僧的額骨撞得粉碎,但趙明的左手小指,却也被刀鋒切去了半寸長的一節、危急之際,竟自未感疼痛。

第三名僧人持劍剛進門口,便見兩名同伴幾是同時殞命,他便再是同仇敵愾,也已無勇氣上前厮殺,大叫一聲,向外便奔。趙明叫道︰「不能讓他逃走了。」一塊聖火令從窗子中擲將出去,準頭極佳,却是全無力量,没碰到那人身子,聖火令便已落地。無忌一把抱住她身子,叫道︰「再擲!」以胸口稍行凝聚的眞氣,從她背心傳入。趙明左手的聖火令再度擲出,那僧人只須再奔兩步,便躱到了照壁之後,但聖火令去勢奇快,穿背而入,更從前胸透出,餘勁未衰,拍的一響,嵌入了照壁之中。

無忌和趙明一擲出這聖火令,同時昏暈,相擁看跌下床來。這時廂房内死了六僧,庭中死了二僧,無忌和趙明昏倒在血泊之中。荒山小廟,冷月窺人,頃刻間更無半點聲息。

過了良久,趙明先行醒轉,迷迷糊糊之中,先伸手一探無忌鼻息,呼吸雖是微弱,却是悠長平穩。她支撐著站起身來,無力將無忌扶上床上,只得將他身子拉平,抬起他的頭,枕在一名死僧的身上。她坐在死人堆裡,不住喘氣。又過半晌,無忌睜開眼來,叫道︰「明妹,你\dash{}你在那裡?」趙明嫣然一笑,清冷的月光從窗中照將進來,兩人看到對方臉上都是鮮血,本來神情甚是可怖,但劫後餘生,却覺説不出的俊美可愛,各自張臂,便已相擁在一起。

這番劇戰,先前殺那七僧,可説是未花半分力氣,全是借力打力,但最後以聖火令飛擲第八名惡僧,二人全是大傷元氣。這一晩二人均是無力動彈,只有躺在死人堆中,靜候精神恢復。趙明包紮了左手小指的傷處,止住流血,累得迷迷糊糊的又睡著了。

這一睡直到次日中午,二人方始先後醒轉。無忌打坐運氣,調息大半個時辰,精神爲之一振,撐身站了起來,肚裡已是餓得咕咕直叫,摸到厨下,只見一鍋飯一半已成黑灰,另一半也已焦臭難聞。他伸手抓了兩口吃了,盛了一碗,送到房中去給趙明。趙明笑道︰「今日情景,比之大都小酒店中,却是如何?」無忌笑道︰「此間樂,不思蜀!」

趙明道︰「這等狼狽,只可天知地知,你知我知,實不足爲外人道也。」兩人相對大笑,伸手在一隻碗中抓取焦飯而食,只覺滋味之美,猶勝山珍海味。一碗飯尚未吃完,忽聽得遠處山道之上,傳來了馬蹄和山石相擊之聲。

嗆{\upstsl{啷}}一聲,盛著焦飯的瓦碗掉在地下,打得粉碎。趙明與無忌面面相覷,兩顆心怦怦跳動,耳聽得馳來的共是兩匹馬,到了廟門前戈然而止,接著門環四響,有人打門,稍停片刻,又是門環四響。無忌低聲道︰「怎麼辦?」只聽得門外一人叫道︰「上官三哥,是我秦老五啊。」趙明道︰「他們就要破門而入,咱們且裝死人,隨機應變。」兩人伏在死人推裡,臉孔向下。剛伏好身子,便聽得砰的一聲巨響,廟門被人大力撞開,從這撞門的聲勢中聽來,來人膂力大是不小。趙明心念一動,道︰「你伏在門邉,擋住二人的退路。」無忌點點頭,爬到了門檻之旁。

緊跟著便聽得兩聲驚呼,刷刷聲響,進廟的兩人拔出兵刃,顯已見到了庭中的兩具屍首。一人低聲道︰「小心,防備敵人暗算。」另一人大聲喝道︰「好朋友,鬼鬼祟祟的躱著是什麼英雄?有種的出來跟老子決一死戰。」這人音聲粗豪,中氣充沛,諒必是那推門的大力士了。他連喝數聲,靜聽四下裡並無半點聲息,説道︰「賊子早去遠了。」另一個嗓音嘶啞的人道︰「四處査一査,莫要中了敵人的詭計。」那秦老五道︰「壽老弟,你往東邉搜,我往西邉搜。」那姓壽的見到庭中二人死得如此可怖,不禁膽寒,道︰「只怕敵人人多,咱們聚在一起,免得落單。」秦老五未置可否,那姓壽的突然「咦」的一聲,指著東廂房,道︰「裡\dash{}裡面還有死人!」兩人走到門邉,但見小小一間房中,死屍橫七豎八的躺了一地,秦老五饒是大膽,也不由得心中發毛,道︰「這中\dash{}中嶽神廟裡的八位兄弟,一齊喪命,不知是什麼人下的毒手?」姓壽的道︰「秦五哥,咱們急速回寺,報知師父知道。」秦老五沉吟道︰「師父叮囑咱們,須得趕快將請帖送出,趕著在端午節開『屠獅英雄會』,要是誤了師父的事,那可吃罪不起。」

無忌聽到「屠獅英雄會」五字,微一沉吟,不禁驚、喜、慚、怒,百感齊生,心想︰「他師父大撒請帖,開什麼屠獅英雄會,自是招集天下英雄,要當衆殺害義父,由此觀之,在端午節之前,義父性命倒是無礙。但我身爲明教之主,竟不能保護義父周全,害得他老人家落入奸人手中,苦受折辱,不孝不義,莫此爲甚。」他越想越怒,恨不得立時手刃這兩個奸人,但又怕二人見機,脱身逃走,自己却是無力追逐,唯有待他二人進房,然後截住退路,依樣葫蘆,以九陽眞氣反震之力鋤奸。不料這二人見房中盡是死屍,腥臭撲鼻,不願進房,只是站在中庭商量。

那姓壽的道︰「這等大事,也得及早稟告師父纔好。」秦老五道︰「這樣吧,咱哥児倆分頭行事,我去送請帖,你回少林去稟告師父。」姓壽的又擔心在道上遇到敵人,躊躇未答。秦老五惱起上來,道︰「那麼任你挑選,你愛送請帖,那也由得你。」姓壽的沉吟片刻,終於覺得還是回山較爲安全,道︰「聽憑秦五哥吩咐,我回山稟告便是。」二人商議定當,便要出寺。趙明身子一動,低聲呻吟了兩聲。

秦壽二人吃了一驚,一齊回過頭來,只見趙明又動了兩動,這時看得清楚,却是一個女子。秦老五奇道︰「這女子是誰?」走進房去。姓壽的膽子雖小,但一來見她是個女子,二來是重傷垂死之人,絲毫不加忌憚,跟著走了進去,正要伸手去扳趙明肩頭,無忌一聲咳嗽,坐起身來,盤膝運氣,雙目似閉非閉。秦壽二人突然児無忌坐起,臉上全是血漬,神態却又是這等可怖,一齊大驚。那姓壽的叫道︰「不好,這是屍變。這僵屍陰魂不散,秦五哥須得小心。」一縱身便跳上了床。秦老五叫道︰「僵屍作怪,姓秦的可不來怕你。」一刀便往無忌頭頂砍下。

無忌手中早已握好了兩枚聖火令,眼見單刀砍下,便將聖火令往頭頂一放,{\upstsl{噹}}的一響,刀刃砍在聖火令上,反彈回去,又是將那秦老五的額頭撞得腦漿迸裂,立時斃命。那姓壽的手中握著一柄鬼頭刀,手臂只是發抖,想要向無忌身上砍去,却只是不敢。無忌只等他砍劈過來,便可用九陽眞氣反撞,但若他嚇得並不動手,竟爾從窗中跳了出去,或者逕而闖門直出,只要不碰無忌的身子,反是無法傷他。趙明見他久久不動,心下也是不禁焦躁︰「看來這膽小鬼竟是嚇得魂飛魄散,不敢向無忌哥哥動手,要是他抛刀逃走,咱們可奈他不得。」只見他牙関相擊,格格作響,突然間拍的一聲,鬼頭刀掉在地下。無忌道︰「你有種便來砍我一刀,打我一拳。」那人道︰「小\dash{}小的没種,不\dash{}不敢跟大人動手。」無忌道︰「那麼你踢我一脚試試。」那人道︰「小的\dash{}小的更加不敢。」無忌怒道︰「你如此膿包,待會只有死得更慘,快向我砍上兩刀。我若見你手勁不差,説不定反饒了你的性命。」那人道︰「是,是!」俯身拾起了鬼頭刀,一眼瞥見秦老五頭骨破碎的慘狀,心想敵人神通廣大,已到了動念傷人的地步,我還是苦苦哀求饒命的爲是,當下雙膝一軟,已是跪倒在地,磕頭道︰「老爺饒命,老爺饒命!」

趙明好生生氣,哼了一聲道︰「武林中居然有這等没出息的奴才。」那人道︰「是,是!小的没出息,没出息,眞是奴才,眞是奴才。」他不敢出手,張無忌倒是無計可施。突然心念一動,喝道︰「過來。」那人忙道︰「是!」向前爬了幾步,仍是跪著。無忌伸出雙手,將兩根拇指按在他眼珠之上,喝道︰「我先挖出你的眼珠。」他手上雖然全無勁力,但眼珠是柔軟之物,再輕微的力道也是抵受不起,那人危急之中,不及細想,伸手用力將無忌雙臂一推。無忌只求他這麼一推,便可借用他的力道,手臂向下一滑,已是點中他乳下的「神封」「步廊」兩處穴道。這兩指點穴,乃是借用那姓壽的一推之力。雖與無忌平時出手勁力強弱大相懸殊,但因部位恰好,那人只感全身一陣酸麻,撲倒在地,大聲求懇︰「老爺饒命,老爺饒命。」

趙明知道無忌這一下點穴,只能暫時制住,不到半個時辰,那人穴道自解,屆時又有一番麻煩,又想有許多事要向他査明,不便此時取他性命,便道︰「你已被這位爺台點中了死穴,你吸一口氣,左胸肋角是否隱隱生疼?」那人依言吸氣,果覺左胸的幾根肋骨處頗爲疼痛,其實這是一時氣血閉塞的應有之象,那人不知,更是大聲哀求起來。趙明道︰「要饒你不難,須得連續下金針半月,方能解去死穴。」那人磕頭道︰「姑娘救得小人之命,做牛做馬,也供姑娘驅使。姑娘但有所命,決不敢有半點違抗。」趙明嫣然一笑,道︰「似你這等江湖人物,我倒是第一次看見,好吧,你去拾一塊磚頭來。」那人忙應道︰「是,是!」蹣跚著走出,到院子中去撿磚頭。

無忌低聲問︰「要磚頭幹什麼?」趙明微笑道︰「山人自有妙計。」那人拿了塊一磚頭,恭恭敬敬的走到趙明面前。趙明在髮上拔下一隻金釵,將釵尖對準他肩頭「缺盆穴」,道︰「我先用金針解開你上身的脈絡,免得死穴之氣上衝入腦,那就無救了。但不知那位爺台肯不肯饒你性命?」那人眼望無忌,滿是哀懇之色,無忌也點了點頭。那人大喜,道︰「這位大爺答應了,姑娘快快下手。」趙明道︰「{\upstsl{嗯}},你怕不怕痛?」那人道︰「小人只怕死,不怕痛。」趙明道︰「很好!你用磚頭在金釵尾上用力敲擊一下。」那人心想金釵插入肩頭,這是皮肉之傷,毫不皺眉,提起磚頭使在釵尾用力一擊。

磚頭一擊之下,金釵直刺入那人「缺盆穴」中,那人不痛不酸,反而覺得有一陣舒適之感,對趙明更增幾分信心,不絶口的道謝。趙明命他拔出金釵,又在他魂門、魄戸、天柱、庫房等七八處穴道上各刺一釵。張無忌微微一笑,道︰「好了,好了!」站起身來。要知那人穴道上受了這些攢刺,十日之内,只須發足一奔,百里内便即氣阻而死。他若是逃出廟去,定然生怕無忌追來,那時自必竭力快跑,趙明這幾下刺穴立即發作,便制了他的死命。

趙明道︰「你去打兩盆水,給我們洗臉,然後去做飯。你若是要死,不妨在飯菜之中下些毒藥,咱三人同歸於盡。」那人道︰「小的不敢,小的不敢。」

這麼一來,無忌和趙明倒多了一個侍僕。趙明問他姓名,原來那人姓壽,名叫南山,有個外號叫作「萬壽無疆」,却是江湖上朋友取笑他臨陣畏縮,一輩子不會被人打死之意。他雖隨著一干綠林好漢拜在圓眞門下,圓眞却嫌他根骨太差,人品猥崽,只差他跑腿辦事,從來没傳授過什麼武功。那壽南山被點了穴道,力氣不失,被趙明差來差去、極是賣力。他將九具屍首拖到後園中埋葬了,提水洗淨廟中血漬。最妙的此人武功不成,烹調手段却高,做幾碗菜肴,無忌和趙明吃來大加誇讚。

待得諸事定當,張趙二人盤問那「屠獅英雄會」的詳情。壽南山倒是毫不隱瞞,只可惜他地位卑微,旁人瞧他不起,許多事都没跟他説。壽南山只知少林寺方丈空聞大師派圓眞主持這次大會,由空聞和空智兩位神僧出面,廣撒英雄帖,邀請天下各門派、各幫會的英雄好漢,於端午節齊集少林寺,會商要事。無忌要過那英雄帖一看,只見那是邀請雲南點蒼派浮塵子、古松子、歸藏子等劍客的請柬。點蒼諸劍成名已久,但隱居滇南,從來不和中原武林人士交往。這次少林派連點蒼諸劍也邀到了,可見這次大會賓客之衆,規模之盛。少林派領袖武林,二大神僧親自出面邀請,接柬之人不論有何要事,都是決計不會不到。無忌見那請柬上只是寥寥數字,書明「敬請端陽佳節,聚會少林,與天下英雄樽酒共歡」,並無「屠獅」字樣,便問︰「幹麼那秦老五説這會叫作『屠獅英雄會』?」壽南山臉有得色,道︰「張爺有所不知,我師父擒獲了一個鼎鼎大名的人物,叫作金毛獅王謝遜。咱少林派這番在天下英雄之前大大的露一露臉,當衆宰殺這隻金毛獅子,所以這個大會嘛,叫作『屠獅英雄會』。」無忌強忍怒氣,又問︰「這位金毛獅王是何等人物,你可看見了麼?你師父如何將他擒來?這人現下関在何處?」壽南山道︰「這金毛獅王哪,嘿嘿,那可是厲害無比,足足有小人兩個那麼高,手膀比小人的大腿還粗。不説别的,單是他一對精光閃閃的眼睛瞧你一眼,你登時便魂飛魄散,不用動手,便已輸了\dash{}」無忌和趙明對望,聽他説謝遜雙目精光閃閃,顯是信口胡吹,只聽他又道︰「我師父跟他鬥了七日七夜,不分勝敗,後來我師父怒了,使出威震天下的『擒龍伏虎功』來,這纔將他收伏。現下是関在咱們寺中山後的石洞之内,身上縛了八根純鋼打就的鍊條\dash{}」

無忌越聽越怒,喝道︰「我問你話,便該據實而言,胡説八道,瞧我要了你的狗命!金毛獅王謝大俠雙目失明,説什麼雙眼精光閃閃?」壽南山的牛皮當場被人戳穿,忙道︰「是,是!想必是小人看錯了。」無忌道︰「到底你有没有見到他老人家?謝大俠是怎麼一副相貌、你且説説看。」壽南山實在未見過謝遜,知道再吹牛皮,不免有性命之憂,忙道︰「小人不敢相欺,其實是聽師兄們説的。」

\chapter{屠獅大會}

無忌最想知道的,乃是謝遜被囚的所在,但反覆探詢,壽南山確是不知,料想這是機密大事,他原也無所得悉,只索罷了。好在端陽節距今二月有餘,時日大是從容,養傷痊癒後前去相救,儘來得及。三人在這中嶽神廟中過了數日,倒也安然辦事,少林寺中並未派人前來聯絡。到得第八日上,趙明之傷已痊可了七八成,無忌體内眞氣逐步貫通,四肢漸漸有力,其時若有敵人到來,傷敵雖仍不足,逃生却已有餘。那壽南山盡心竭力的服侍,不敢稍有異志。趙明笑道︰「萬壽無疆,你這胚子學武是不成的,做個管家倒是上等人材。」壽南山苦笑道︰「姑娘説得好。」

又過十日,無忌和趙明,傷勢痊癒,每日吃著壽南山精心烹調的美食,兩人紅光滿面,精力充沛。無忌忌和趙明商議,如何到少林寺中營救謝遜。趙明道︰「本來最好的法子,乃是眞的點了『萬壽無疆』死穴,派他回去少林寺打探。只是這人太過膿包,要是被成崑或陳友諒瞧出破綻,反而壞了大事。這樣吧,咱二人先到少室山脚下,相機行事。只是咱二人的打扮却得變一變。」無忌道︰「喬裝作什麼?剃了光頭,做和尚尼姑嗎?」趙明臉上微微一紅,碎道︰「{\upstsl{呸}}!虧你想得出!一個小和尚,帶著一個小尼姑,整天晃來晃去,成什麼樣子?」無忌笑道︰「那麼咱倆扮成一對鄕下夫妻,到少室山脚下種田砍柴去。」趙明一笑,道︰「兄妹不成麼?要是扮了夫妻,給周姑娘瞧見、我這左邉肩上又得多五個指頭窟窿。」無忌也是一笑,不便再説下去,細細向壽南山問明少林寺中的居室内情,便道︰「你身上被點了的死穴,已解了十之八九?只是你這一生必須居於南方,只要一見冰雪,立刻送命。你此去得急速南行,住的地方越熱越好,倘若受一點點風寒,有什麼傷風咳嗽,那可危險得緊。」説著替他前胸後背,一陣推宮過血,解了他的死穴。壽南山信以爲眞。拜别二人,一出廟門便向南行,這一生果然長居南方蠻荒之地,小心保養,不敢傷風,竟爾得享高壽,直至明朝建文年間方死,當眞應了「萬壽無疆」的外號。

張趙二人待他走遠,一把火將中嶽神廟燒成白地。走出二十餘里,到自家農家,各買了一套男女農民的衣衫,到荒野處換上,將原來衣衫掘地埋了,慢慢走到少室山下。到得離少林寺七八里處,途中已三次遇到寺中僧人。趙明道︰「咱們不能再向前行了。」見山道旁兩間茅舍。門前有一片菜地,一個老農正在澆菜,便道︰「向他借宿去。」無忌走上前去,行了個禮,説道︰「老丈,借光,咱兄妹行得倦了,討碗水喝。」那老農恍若不聞,不理不睬,只是搯著一飄飄糞水,往菜根上潑去。無忌又説了一遍,那老農仍是不理。忽然啊的一聲,柴靡推開,走出一個白髮如銀的婆婆,笑道︰「我老伴耳聾口啞,客官有什麼吩咐?」無忌道︰「我妹子走不動了,想討碗水喝。」那婆婆道︰「請進來吧。」

二人跟著入内,只見屋内收拾得甚是整潔,板桌木凳,抹得乾乾淨淨,老婆婆的一套粗布衣裙,也是洗得一塵不染。趙明心中喜歡,喝過了水,取出一錠銀子,笑道︰「婆婆,我哥哥帶我去外婆家,我路上脚抽筋。走不動了,今児晩上想在婆婆家裡借宿一宵。等明児清早再趕路。」那婆婆道︰「借宿一宵不妨,也不用什麼銀子。只是咱們只有一間房,一張床,我和老伴就算讓了出來,你兄妹二人也不能一床睡啊。嘿嘿,小姑娘,你跟婆婆説老實話,是不是背父私奔,跟了情哥哥逃了出來啊?」趙明給她説中了眞情,不由得滿臉通紅,暗想這婆婆的眼力好厲害,聽她説話口氣,不似尋常農家老婦,當下向她多打量了幾眼。

但見她雖是弓腰曲背,却是脚骨輕健,雙日開闔之間,炯炯有神。説不定竟是身負絶藝,趙明情知無忌還像個尋常農民,自己的容貌舉止、説話神態,決計不似農女,便悄悄説道︰「婆婆既已猜到,我也不能相瞞。這位曾哥哥,是我自幼的相好,我爹爹嫌他家中貧窮,不肯答應婚事,我媽媽見我尋死覓活,便作主叫我跟了他\dash{}他出來\dash{}我媽媽説,過得三年兩載,咱們有了娃娃,再回家去,爹爹就是不肯也只好肯了。」她説這番話時滿臉飛紅,不時偸偸向無忌望上幾眼,目光中深孕情意,又道︰「我家在大都算是有面子的人家,爹爹在朝中又做個官児,咱倆若是給人抓了,那可是天大的禍事。婆婆,我跟你説了,你可千萬别告訴人。」

那婆婆呵呵而笑,連連點頭,道︰「我年輕時節,也是個風流人物。你放心,我把我的房讓給你小夫妻。此處離大都千里之遙,包你無人追來,就是有人跟你爲難,婆婆也不能袖手旁觀。」她見趙明溫柔美麗,一上來便將自己的隱私説與她聽,心下竟是十分好感,決意出力相助,玉成她倆的好事。趙明聽了她這幾句話,更知她是個武林人物,此處距少林寺極近,不知她與成崑是友是敵,當其要處處小心,不能露出半分破綻,於是盈盈拜倒,説道︰「婆婆肯替咱二人作主,那眞是多謝了。牛哥,你快來謝過婆婆。」無忌依言過來,作揖道謝。

那婆婆當即讓了自己的房出來,在堂上用木板另行搭了一張床,墊些稲草,舖上一張草席。趙明將無忌拉到房中,將自己編的故事輕輕説了。無忌點頭道︰「澆菜那個老農本領更大,你瞧出來了麼?」趙明道︰「啊,我倒看不出。」無忌道︰「他肩挑一擔糞水,行得極慢,可是兩隻糞桶竟無半點晃動,那是很高的内力修爲。」趙明道︰「比起你來怎麼樣。」無忌笑道︰「我來試試,也不知成不成,」説著一把將她抱起,抗在肩頭,作挑擔之狀,趙明格格笑道︰「你將我當糞桶麼?」那婆婆聽得他二人親熱笑謔之聲,先前心頭存著的些微疑心,立時盡去。

當晩二人和那老農夫婦同桌共餐,居然有雞有肉。無忌和趙明故意偸偸捏一捏手,碰一碰肘,便如一對熱戀私奔的情侶,蜜裡調油,片刻分捨不得。那婆婆瞧在眼裡、只是微笑,那老農却不聞不見,只管低頭吃飯。飯後無忌和趙明入房,閂上了門。兩人在飯桌上這般眞眞假假的調笑,不由得都動了情。趙明俏臉紅暈,低聲道︰「咱們這是假的,可作不得異。」無忌一把將她樓在懷裡,吻了吻她,低聲道︰「倘若是假的,三年兩載,怎生得有個娃娃?」趙明羞道︰「胚,原來你躱在一旁,把我的話都偸聽去啦。」

無忌雖和她言笑不禁,但總是想到自己和周芷若已有婚姻之約,雖盼將來一雙兩好,總須和周芷若成婚之後,再説得上趙明之事。此刻溫香在抱,不免意亂情迷,但他頗能自制,只親親她的櫻唇粉頰,便將她扶上床去,自行躺在床前的一張板凳之上,調息用功,九陽眞氣運轉十二周天,便即睡去。

趙明却是翻來覆去,一時難以入睡,直至遠遠聽得雞鳴之聲,已是深宵,正朦朦朧朧間,忽聽得極輕的脚步聲響,自遠而近,迅速異常的搶到了門前。她伸手去推無忌,恰好無忌也已聞聲醒覺,伸手過來推她,雙手相觸,輕輕握住了。只聽得門外一個清朗的聲音説道︰「杜氏賢伉儷請了,故人夜訪,得嫌無禮否?」過了半晌,門内那婆婆的聲音説道︰「是青海三劍麼?我夫婦從川西遠避到此,算是怕了你青海玉眞觀了。殺人也不過頭點地,又何必趕盡殺絶,如此苦苦相逼?」

門外那人哈哈一笑,説道︰「你二位要是當眞怕了,向咱們磕三個響頭,玉眞觀既往不咎,前事一筆勾銷。」只聽得板門啊的一聲開了,那婆婆道︰「請進!」其時滿月初殘,銀光瀉地,無忌和趙明從板壁縫中望將出去,只見門外站的是三個黃冠道人。中間一人短鬚截張,又矮又胖,説道︰「賢伉儷是磕頭陪罪呢,還是雙鉤長劍上一決生死?」那婆婆尚未回答,那聾啞老頭已大踏步而出,只聽得霹霹拍拍,他全身骨骼猶如爆豆般響了起來,顯是在運一種特異的内勁。跟著那婆婆往丈夫身旁一站,雙手舞了幾個柔軟的圏子,便如二八少女翩翩起舞一般。

那短鬚道人道︰「杜老先生幹麼一言不發?不屑跟青海三劍交談麼?」那婆婆道︰「拙夫耳朶聾了,聽不到三位的言語。」短鬚道人咦的一聲,道︰「杜老先生的聽風辨器之術乃武林一絶,怎地耳朶聾了?可惜啊可惜。」他身旁那個更胖的道人刷的一聲。抽出長劍,道︰「賢伉儷怎地不用兵刃?」那婆婆雙手一舉,每隻手掌中青光閃爍,各有三柄不到半尺長的短刀,雙手共是六柄。聾啞老頭跟著揚手,雙掌之中也是六柄短刀,只見他左手刀滾到右手,右手刀滾到左手,便似手指交叉一般,純熟無比,三個道人見了他夫婦的特異兵刃,一齊吃了一驚,武林中還未見過這種兵器,説是飛刀吧,但飛刀還未有這般使法的。

原來這聾啞老頭姓杜,名叫百當,向以雙鉤威震川西。他妻子叫作易三娘,善使鏈子槍。二人多年前和青海玉眞觀結下了怨仇,交戰數場,互有勝敗。杜氏夫婦眼見一來寡不敵衆,二來這場怨仇自小事而起,結得甚是無謂,於是咬牙棄了川西的大片基業,遠走他鄕,不意今晩又遇怨家對頭。那三個道人是玉眞觀第二代弟子中的好手,短鬚道人叫雲鶴,胖子道人叫馬法通,第三個瘦瘦小小的道人叫雲燕,劍法上均有頗深的造諳,合稱「青海三劍」。

馮法通雖然身材臃腫,生相蠢笨,其實爲人甚多智計,他一見杜氏夫婦兵刃出來,竟是捨棄了浸潤數十年的拿手兵器不用,知道他夫婦在這十二柄短刀之上,必有極厲害極怪異的招數,當下長劍一振,肅然吟道︰「三才劍陣天地人。」雲鶴接口道︰「電逐星馳出玉眞。」三名道人脚步錯開。登時將杜氏二老圍在垓心。無忌見三名道人忽左忽右,穿來插去,似三才而非三才,三柄長劍織成一道光網,却不向對方遞招。待那三道走到七八步時,無忌已瞧出其中之理,尋思︰「這三名道人好生狡猾,口中叫明這是三才劍陣,其實暗藏正反五行。只要敵人信以爲眞,按天地人三才方位去破解,立時陥身五行殺傷。他三個人而排五行劍陣,每個人要管到一個以上的生剋變化,這輕功和劍法上的造詣,果然相當不凡。」

杜氏夫婦背與背相靠,四隻手銀光閃閃,十二柄短刀交換舞動,原來兩人不但雙手的短刀交互轉換,而且杜百當的短刀交到了易三娘手裡,易三娘的短刀交到了杜百當手裡。但每一柄刀決不脱手抛擲,却是老老實實的遞來遞去。趙明瞧得奇怪,問道︰「無忌哥哥,他們在變什麼戲法?」無忌皺眉不答,又看一會,忽然道︰「啊,我知道了,他是怕我義父的獅子吼。」趙明道︰「什麼獅子吼?」無忌連連點頭,忽地冷笑道︰「哼,就憑這點児功夫,也想屠獅伏虎麼?」

趙明莫名其妙,道︰「你打什麼啞謎?自言自語的,叫人聽得老大納悶?」無忌低聲道︰「這五個人都是找義父的仇人。那老頭怕我義父的獅子吼,故意刺聾了自己耳朶\dash{}」只總得{\upstsl{噹}}{\upstsl{噹}}{\upstsl{噹}}{\upstsl{噹}},密如聯珠般的一陣響聲過去,五個人已交上了手。

青海三劍連攻五次,均被杜氏夫婦擋開。這對老夫婦手中的十二柄短刀盤旋往復,月光下聯成了三道光環,繞在二人身旁,守得嚴密無比。青海三劍五攻不入,當即轉爲守禦。杜百當猱身而進,短刀疾取馬法通小腹,武學中有言道︰「一寸長,一寸強。一寸短,一寸險。」他這些短刀長不逾五寸,當眞是險到了極處,只見他刷刷刷三刀,全是進攻的殺著,決不防及自身。雲鶴、雲燕長劍刺來。均被易三娘以短刀架開。原來他夫婦練就了這套刀法,一攻一守,配合得天衣無縫,攻者專攻而守者專守,不須兼顧。馬法通被他三刀之下,通得手忙脚亂,連連退避。杜百當撲入了他的懷中,刀刀不離要害、越來越是驚險。

雲鶴一聲長嘯,劍招亦變,與雲燕兩把長劍從旁插入,組成一道劍網,將杜百當攔到了三尺以外。三劍聯防,眞是水也潑不進去。無忌又是輕輕冷笑一聲,在趙明耳邉説道︰「這套刀法和劍法,都是練就專門對付我義父的。你瞧他們守多攻少,守長於攻,再打一天一晩也分不了勝負。」果然杜百當數攻不入,隨即棄攻轉守。趙明細看五人的招數,確是攻者平平無奇,守者却是全無破綻,低聲道︰「金毛獅王武功卓絶,這五個傢伙單靠守禦,焉能取勝?」但見五人刀來劍往,連變了七八種招數,兀自難分勝敗。馬法通突然喝道︰「且住!」托地跳出圏子。杜百當颼颼兩下撲擊也向後退開,銀髯飄動,自具一股威勢。

馬法通道︰「賢伉儷這套刀法,練來是屠獅用的?」易三娘咦的一聲,道︰「你倒訊息靈通。」馬法通道︰「杜老先生與謝遜有殺兄之仇,這等大仇,自是非報不可。既是探得對頭在少林寺中,何以不及早求個了斷?」易三娘側目斜睨,道︰「這是我夫婦的私事,不勞道長掛懷。」馬法通道︰「玉眞觀和賢夫婦的樑子,原是小事一件,豈値得如此性命相撲?咱們不如化敵爲友,聯手去找謝遜如何?」易三娘道︰「玉眞觀和謝遜也有樑子?」馬法通道︰「樑子倒是没有,嘿嘿。」易三娘道︰「既和謝遜並無樑子,何以苦心孤詣的練這套劍法?咱們雙方招數殊途同歸,都是剋制七傷拳用的。」馬法通道︰「三娘好眼力!眞人面前不説假話,玉眞觀只是想借屠龍寶刀一觀。」易三娘點了點頭,伸手指在杜百當掌心飛快的冩了幾個字。杜百當也伸指在她掌心冩字。夫婦倆以指代舌,談了一會。易三娘道︰「咱夫婦倆只求報仇,便是送了自己性命,也所甘願,於屠龍刀絶無染指之意。」馬法通大喜道︰「那好極了。咱們五人聯手,賢夫婦殺人報仇,玉眞觀得一柄寶刀。齊心合力,易成大功,雙方各遂所願,不傷和氣。」當下五個人擊掌爲誓,立下了毒咒,杜氏夫婦便請三道進屋喝茶,詳議報仇奪刀之策。

青海三劍進屋坐定,見隔房門板緊閉,不免多瞧了幾眼。易三娘笑道︰「三位不必起疑,都是大都來的一對小夫妻,私奔離家,女的似玉女一般,男的却是個粗魯漢子,都不會半點武功的。」馬法通爲人甚是謹細,道︰「三娘莫怪,非是我不信三娘之言,只是咱們所圖謀的事関重大,頗遭天下豪傑之忌,若是走漏了消息\dash{}」易三娘笑道︰「咱們鬥了半天,這小兩口兀自睡得死豬一般。馬道爺既是不信。親眼去瞧瞧也是好的。」説著便去推門,那門在裹面上了閂。

無忌心想倘若此刻打發了這五人,反而失了營救義父的頭緒,當即抱起趙明和衣睡倒在床,只匆匆忙忙的除下鞋子,拉棉被蓋在身上。只聽得拍的一聲響,門閂已被雲鶴使内勁震斷。易三娘手持燭台,走了進來,青海三劍跟隨其後。

無忌見到燭光,睡眼惺忪的望著易三娘,一臉茫然之色。馬法通颼的一劍,往他咽喉刺了過去,這一招又狠又疾,端的厲害。無忌「啊」的一聲驚呼,却是不知閃避,上身向前一撞,似乎反而送到劍尖上去。馬法通縮手迴劍,心想此人果然半點不會武功,若是武學之士,膽子再大,也決不敢不避此劍。他那知無忌的武功勝他十倍,不但事先明知他是假意相試,就算他眞的有意傷人,劍尖刺到無忌的咽喉肌膚,也是萬難加害。

趙明唔的一聲,仍未醒轉。雲鶴道︰「易三娘説的不錯,出去吧。」五人又回到了廳上。無忌跳下床來,穿上了鞋子,只聽馬法通道︰「賢伉儷可是拿準了,謝遜確是在少林寺中?」易三娘道︰「此節已是千眞萬確。少林寺送出英雄帖,端陽節在寺中大開屠獅之會。倘若他們没擒到謝遜,當著普天下英雄之面,這個大人怎能丟得起?」馬法通{\upstsl{嗯}}了一聲,又道︰「少林派的空見神僧死在謝遜拳下,少林僧俗弟子,自是非報此仇不可。賢伉儷只須在端陽節進得寺去,睜開眼來瞧著仇人引頸就戮,不須花半分力氣,便報了血仇。杜老先生又何必毀了一對耳朶,又甘冒得罪少林派的奇險?」易三娘冷笑道︰「咱老夫妻的獨生愛児,無辜爲謝遜這惡賊所傷,咱夫婦和他仇深似海,報這等殺子之仇,焉能假手旁人?咱們一遇上姓謝這惡賊,老婆子第一步便是刺聾自己雙耳。咱夫婦但求與他同歸於盡,嘿嘿,咱從我愛児爲他所害,咱老夫妻於人世早已一無所戀。得罪少林派也好,得罪武當派也好,大不了是千刀萬剮,何足道哉?」

無忌隔房聽著她這番話,只覺怨毒之深,直是令人驚心動魄,心想︰「義父當年受了成崑的荼毒,一口怨氣發洩在許多無辜之人身上。這對杜氏夫婦看來原非歹人,只是心傷愛子慘死,這纔處心積慮的要殺我義父報仇。這等仇怨要説調處吧,那是萬萬不能,我只有救出義父,遠而避之,免得更增罪孼。」這時只聽得鄰室五人半點聲息也無,從板壁縫中張去。見杜氏夫婦和馬法通三人手指上醮了茶水,在板桌上冩字,心想︰「這五人當眞小心,明知我並非江湖中人。猶恐洩漏了機密。唉,我義父在江湖間怨家極衆,覬覦屠龍刀的人更多,不等端陽節到便要提前下手的,只怕不計其數,這等人不是苦心孤謂,便是藝高手辣,少林寺只要稍有疏忽,義父便遭大禍。那是越早救了他出來越好。」

這五個人以指冩字,密議了半夜,竟是一宵不睡。無忌自在板凳上睡了兩個多時辰,也不去理會。次晨起身,只見青海三劍已然不在。無忌對易三娘道︰「婆婆,昨晩三位道爺手裡拿著明晃晃的刀子,幹什麼來啊?我起初還道是捉拿咱們來著,嚇得了不得,後來才知不是。」易三娘聽他管長劍叫作刀子,心下暗暗好笑,淡淡的道︰「他們走錯了路,喝了碗茶便走了。曾小哥,吃過中飯後,咱們要挑三把柴到寺裡去賣。你幫著挑一組成不成?寺裡的和尚問起,我説你是咱們児子。這可不是佔你便宜,那只是免得寺裡疑心。你媳婦花朶児一般的人物,可别出去走動。」她雖似和無忌商量,實却是斬釘截鐵般下了號令,叫無忌推辭不得。無忌一聽之下,已然明白︰「她見我眞是個鄕下人,要我陪著混進少林寺去察看動靜,那是再好也没有。」便道︰「婆婆怎麼説,小子便怎麼幹,只求你收留咱兩口児,咱兩人東逃西奔,没一天平安。」

到得午後,無忌隨著杜氏夫婦,各自挑了一擔乾柴,往少林寺走去。他頭戴斗笠,腰插短斧,赤足穿一雙麻鞋,三個人中,獨有他挑的一擔柴最大。趙明站在門邉,微笑著目送他遠去。

杜氏夫婦雖是脚力雄健,但故意走得甚慢,氣喘叮叮的,到了少林寺外的山亭之中,放下柴擔休息。山亭中有兩名僧人坐著閒談,見到無忌等三人,也不以爲意。易三娘除下包頭的粗布,抹了抹汗,又伸手過去替無忌抹汗,道︰「乖孩子,累了麼?」無忌初時有些不好意思,但聽她言語之中,頗蓄深情,不像是故意做作,不禁望了她一眼。只見易三娘泪水在眼眶中轉來轉去,知道她是念及自己被謝遜所殺了的那個孩子,但見她情致纏綿的凝視自己,似乎盼望自己答話,無忍心下不忍,便道︰「媽,我不累。你老人家累了。」他一聲「媽」叫了出口,想起自己母親,心下也是不禁傷戚。易三娘聽他叫了一聲「媽」,泪水已忍不住流了下來,假意用包頭布擦汗,擦的却是泪水。杜百當站起身來,挑了柴擔。左手一揮,便走出了山亭,他知老妻觸景生情,憶起了亡児,説不定露出破綻,被那個僧人瞧破了機関。無忌走將過去,在易三娘柴擔上取下兩綑乾柴,放在自己柴擔之上,道︰「媽,咱們走吧。」易三娘見他如此體貼,心想︰「我那孩児今日若在世上,比這少年年紀大得多,我孫児也抱了幾個啦。」一時怔怔的不能移步,眼見無忌挑擔走出山亭,這纔跟著走出,心情激動之下,脚步不禁有些蹣跚,無忌回過身來,伸手相扶。一名僧人道︰「這少年倒是孝順,可算難得。」另一名僧人道︰「婆婆,你這柴是挑到寺裡去賣的麼?這幾日方丈下了法旨,不讓外人進寺,你别去吧。」易三娘好生失望,心想︰「少林寺果然防範周密,那是不易混進去了。」杜百當走出數丈後,見他一人不即跟來,便停步相候。

另一名道人道︰「這一家鄕下人母慈子孝,咱們就行個方便,師弟,你帶他們從後門進香積厨去,監寺若是知道了,便説是來慣賣柴的鄕人,料也無妨,」那僧人道︰「是,監寺不讓外人入寺,那是防備閒雜人等。這些忠厚老實的鄕人,何必斷了他們生計?」於是領著杜氏夫婦和無忌,轉到後門進寺,將三把乾柴挑到柴房,自有管香積厨的僧人算了柴錢。易三娘道︰「咱們有上好的大白菜,我叫阿牛明児送幾斤來,那是不用錢的,送給師傅們{\upstsl{嚐}}新。」引她來的那僧人笑道︰「從明児起,你不能再來了。監寺知道,怪罪下來,咱們可擔待不起。」管香積厨的僧人向無忌打量了幾眼,忽道︰「端陽前後,寺中要多上一千餘位客人,挑水破柴,説什麼也忙不過來。這位兄弟倒生得健旺,你來幫忙兩個月,算五錢銀子一個月的工錢給你如何?」

易三娘大喜,忙道︰「那再好也没有了,阿牛在家裡也没什麼要緊事,就在寺裡幫助師傅們打打雜,賺幾兩銀子幫補幫補,也是好的。」無忌一想不妥︰「少林寺中很多人相識於我,偶爾來厨房走走,那還罷了,在寺中一住兩月,非給人認了出來不可。」説道︰「媽,我媳婦児\dash{}」易三娘心想這等天賜良機,眞是可遇而不可求,説道︰「你媳婦児好好在家中,還怕你媽虧待了她嗎?你在這児,聽師傅們話,不可偸懶,媽和你媳婦過得幾天,便來探你。這麼大的小子,離開媽一天也不成,你還要媽餵奶把尿不成?」説著伸手理了理他的頭髮,眼光中充滿慈愛之色。

那管香積厨的僧人已煩惱多日,料想端陽大會前後,天下英雄聚會,這飯菜茶水。實是難以打發。監寺雖已增撥了不少人手,但寺中這些和尚不是勤於清修,便是鑽研武功,厨房中的粗笨事務,誰都不肯去幹,被監寺委派了到那是無可奈何,但在厨房中大模大樣,有許多輩份均比管香積厨的僧人爲高,更加差之不動。他見無忌誠樸勤懇,一心一意想留他下來,不住的勸説。

無忌心中早已是千肯萬肯,只是故意裝著躊躇,待那引他入寺的僧人也從旁相勸,這纔勉強答應,説道︰「師傅,最好你一個月給我六錢銀子,我五錢銀子給我媽,一錢銀子給我媳婦買花布\dash{}」管香積厨的僧人呵呵笑道︰「咱們一言爲定,六錢就是六錢。」易三娘又叮囑了幾句,這纔同了杜百當慢慢下山。無忌追將出去,道︰「媽,我媳婦児請你多多照看,易三娘道︰「我理會得,你放心便是。」

無忌回來請問那管香積厨的僧人法名,原來叫作慧止。當下跟入厨房,劈柴搬炭、燒火挑水,忙了個不亦樂乎,他故意在搬炭之時,滿臉塗得黑黑地,再加上頭髮蓬鬆,水缸中一照,當眞是誰也認不出來了。當晩無忌便在香積厨房的小屋之中,與衆火工睡在一起。他知少林寺中臥虎藏龍,往往火工之中也有身懷絶技之人,是以處處小心。

如此過了七八日,易三娘帶著趙明來探望了他兩次。無忌做事勤力,從早到晩,什麼粗工都做,慧止固然歡喜,旁的火工也均和他極爲投機。無忌不敢探問訊息,只是豎起耳朶,從各人閒談之中,尋找線索,心想義父既是囚在寺中,定然有人送飯,只須著落在送飯的人身上,總可訪到義父被囚的所在,那知耐心等了數日,竟是瞧不出半點端倪。

到得第九日晩間,無忌睡到半夜,忽聽得半里外隱隱有呼喝之聲。他心中一動,悄悄起來,一見四下無人知覺,較即展開輕功,循聲趕去,只聽得那聲音來自寺左的樹林之中,無忌生怕自己蹤跡敗露,一縱身上了一株大樹,査明樹後草中無人隱伏,這纔從此樹躍至彼樹,逐漸移近,這時林中兵刃相交,已有數人鬥在一起。無忌隱身在樹後一看,密林中一片黑暗,瞧不清人影,但見刀光縱橫,劍影閃動,六個人分成兩邉相鬥。

無忌看了數招,從那劍光之中,已看出三個使劍的便是青海三劍,但見這三人佈開了正反五行的「假三才陣」,守得甚是繁密,在旁相攻的乃是三個僧人。各使戒刀,破陣直進。拆到二十餘招時,{\upstsl{噗}}的一聲響,青海三劍中一人中刀倒地。假三才陣一破,餘下二人更加不是對手,更拆數招,一人「啊」的一聲慘呼,被砍斃命,聽聲音是那矮胖子馬法通。餘下一人右臂帶傷,兀自死戰。一名僧人低聲喝道︰「且住!」三把戒刀將他團團圍住,却不再攻。一個蒼老的聲音道︰「你青海玉眞觀和我少林派向來無冤無仇,何故夤夜來犯?」青海三劍中餘下那人乃是雲鶴,慘然道︰「咱師兄弟三人既然敗陣,只怨自己學藝不精,更有什麼好問?」那蒼老的聲音冷笑道︰「你們是爲謝遜而來,還是爲了想屠龍刀?嘿嘿,没聽説謝遜曾殺過玉眞觀中人!諒必是爲了寶刀啦。憑這點児玩藝,也想來闖蕩少林寺麼?少林寺領袖武林千餘年,没想到被人如此小看了。」雲鶴乘他説得高興,刷的一劍,中鋒直進。那僧人急忙閃避,終於慢了一步,被他一劍刺中左肩。旁邉二僧雙刀齊下,雲鶴登時身首異處。

三名僧人一言不發,提起青海三劍的屍身,快步便向寺中走去。無忌正想跟隨前去瞧個究竟,忽聽得右前方長草之中,有人輕輕呼吸,暗道︰「好險!原來尚有埋伏。」當下靜伏不動,過了小半個時辰,纔聽得草中有人輕輕擊掌二下,遠處有人擊掌相應,只見前後左右,六名僧人長身而起,或持禪杖,或挺刀劍,散作扇形回入寺中。

無忌待那六人走遠,纔回到小屋,同睡的衆火工兀自好夢不醒。無忌心下暗歎︰「若非親眼得見,怎知在這片刻之間,三條好漢已靜悄悄的死於非命。」自經此役,他知少林寺防範之周,迥非尋常,更是多加了一分小心。

\chapter{雷震電閃}

又過數日,已是四月中旬,天氣漸熱,離端陽節也是一天近似一天。無忌心想︰「憑著我在香積厨下幹這粗活,終難探知義父的所在,今晩須得冒險往各處査察。」他知道自己武功雖較少林寺中每一人都高,但寺中高手如雲,倘是單憑一人之力,明搶硬奪,定然救不出謝遜,只有暗中下手,方能救人出險。這日晩上無忌睡到三更時分,悄悄出來,縱身上了屋頂,躱在屋脊之後,身形甫定,便見兩條人影自南而北,輕飄飄的掠過,僧袍鼓風,戒刀映月,正是寺中的巡査僧人。

無忌待那二僧過去,向前縱了數丈,但聽得瓦面上脚步聲響。又有二僧縱躍而過,但見此來彼去,穿梭相似,顯是少林寺知道這幾日中將有不少武林高手前來探寺,是以巡査之嚴,恐怕皇宮内院也有所不及。無忌見了這等情景?知道若再前往,定然被人識破,只得廢然而返。

挨過三日,這一晩雷聲大作,突然間下起傾盆大雨來。無忌大喜,暗道︰「天助我也。」但見那雨越下越大,四下裡一片漆黑,無忌閃身走向前殿,心想︰「羅漢堂、達摩堂、藏經閣、方丈精舍四處,最是少林寺的根本要地,我逐一探將過去。」只是少林寺中屋宇重重,摸不到何處是羅漢堂、何處是藏經閣。他躱躱閃閃的信步而行,來到一道長廊,突覺這條長廊依稀相識,記起幼時隨太師父來少林寺求「少林九陽功」,曾到過這條廊上,由此而左,通向成崑所居的小室。他微一沉吟,心道︰「且探探這惡賊去,或者從他身上,能尋到義父的所在。」當下追憶舊日走過的路途,沿著一條鵞卵石舖的小徑,穿過一片竹林,果然到了成崑所居的小室之外。無忌心中砰砰跳動,深知成崑武功深湛,陰險奸猾,若是發見了他的蹤跡,後果如何,實是難以逆料。這時他全身早已濕透,黃豆大的雨點打在臉上手上,一滴滴的反彈出去,他一個箭步,欺到小舍的窗下,只聽得裡面有人正在説話。無忌只聽得幾個字,便知是方丈空聞大師的聲音。

只聽他説道︰「爲了這金毛獅王,一月來少林派已殺了二十三人,多造殺孼,實非我佛慈悲之意。明教光明左使楊逍、右使范遙,白眉鷹王殷天正、青翼蝠王韋一笑,先後遣使來寺,求我放了謝遜\dash{}」無忌聽到此處,心下大是喜慰,暗道︰「我外公和楊左使等也已得訊息,原來曾派託人來過。」只聽空聞續道︰「本寺雖加推托,但明教豈肯就此罷休,他張教主武功出神入化,始終不見現身,只怕暗中更有圖謀。我和空智師弟蒙他相救,欠過人家的恩情,若是他親自來求,我等如何對答?今日三位師叔細細盤問謝遜殺害空見師兄的詳情,謝遜始終閉自不答。此事當眞難處,師弟師侄,你二位有何高見?」只聽一個蒼老陰沉的聲音輕輕咳嗽一聲,正是改名圓眞的成崑,他説道︰「方丈師叔忒也多慮,謝遜由三位太師叔看守,那是萬無一失的了。英雄大會関涉我少林派千百年的興衰榮辱,魔教的一些小恩小怨,方丈師叔不必掛懷。何況此事是魔教暗中勾結朝廷,來和六大門派爲難,方丈師叔難道不知麼?」

空聞奇道︰「怎地是明教勾結朝廷?」圓眞道︰「明教張教主本要和蛾嵋派掌門人周姑娘結親,成婚之日,汝陽王的郡主娘娘突然擕同那姓張的小子出走,此事轟傳江湖,方丈師叔必有所聞。」空聞道︰「不錯,聽説有這一回事。」圓眞道︰「那郡主娘娘手下,有一個得力部屬,叫做苦頭陀,兩位師叔在萬法寺中想必會過。」空智憶及此事,猶有餘憤,説道︰「哼,此間大事一了,我倒接再上大都,找這頭陀會會。」圓眞道︰「兩位師叔可知這頭陀是誰?」

空智道︰「這這苦頭陀所知甚博,似乎各家各派的武功均有涉獵,却看不出他的門道來。」圓眞道︰「苦頭陀便是魔教的光明右使范遙。」空聞和空智齊聲驚道︰「此話當眞?」圓眞道︰「圓眞焉敢欺瞞師叔?屆時他若膽敢前來本寺,兩位師叔一見便知。」空智沉吟道︰「如此説來,張無忌和那郡主確是暗中勾結,由郡主出面、擒了六大門派中的首領人物,再中張無忌賣好救人。」圓眞道︰「十有八九,便是如此。」空聞却道︰「我見那張教主忠厚俠義,似乎不是這等樣人,咱們可不能怪錯了好人。」圓眞道︰「方丈師叔明鑒,常言道︰知人知面不知心。那謝遜是張無忌的義父,魔教自會不顧一切的圖謀相救。到得屠獅大會之中,一切自有分曉。」接著三人商議如何接待賓客、如何抵擋敵人劫奪謝遜,又盤算各門派中有那些好手。無忌聽著三人商議,圓眞和空智力圖挑動各派互鬥,待得數敗倶傷之後,少林派再出面收卞莊刺虎之利,壓服各派,名正言順的掌管屠龍刀,殺了謝遜祭奠空見。空閒則力持鄭重,似乎對明教不敢輕侮。

空智道︰「第一要緊之事,説來説去,還是如何迫使謝遜在端陽節前吐露屠龍刀所在,否則這次屠獅大會變得無聲無臭,反而折了本派的威望。」空聞道︰「師弟所言極是。咱們須得在會中揚刀立威,説道這武林至尊的屠龍寶刀已歸本派掌管,那時本派號令天下,那就莫敢不從了。」空智道︰「好,就是如此,圓眞,你再設法去跟謝遜談談,勸他交出寶刀,咱們便饒他一命。」圓眞道︰「是!謹遵兩位師叔吩咐,包在圓眞身上,端陽大會之前,定能取得寶刀。」脚步之聲輕響,圓眞走了出來。

無忌心下大喜,但知這三位少林僧武功高極,只要稍有響動,立時便被査覺,若是三人一齊出手,自己只怕難以取勝,最多不過是自謀脱身,要救義父却是千難萬難了。當下屏息不動,見圓眞瘦長的身形向北首走去,手中撐著一把油紙傘,急雨打在傘上,浙瀝作響。無忌待他走出十餘丈,這纔輕輕向前移步。跟隨其後。大雨之下,寺頂和各處的巡査都鬆了許多,無忌以牆角、樹幹爲掩蔽,一路追攝,雨聲既大,他輕功又強,幸喜無人發覺。只見圓眞躍過寺後圍牆,逕向北去。無忌心想︰「原來義父被囚在寺外,難怪寺中不著絲毫跡,始終探聽不到頭緒。」他不敢公然躍牆而出,將身子貼在牆邉,慢慢遊了上去,到得牆頂,等牆外巡査的僧人走過,這纔躍下。一條條雨線之中,但見圓眞的傘頂已在百丈之外,折而向左,走向一座小小的山峰,跟著便迅速異常的攀上峰去。

圓眞是謝遜之師,此時已是個七十餘歳的老人,但身手仍是矯捷無比,只見他上山時雨傘決不晃動,却是冉冉上昇,宛如有人用長索將他吊上山去一般。無忌快步走近,到了山脚之下,正要跟著上峰,忽見山道旁樹叢中白光一閃,有人執著兵刃埋伏。無忌急忙停步,只過得片刻,見樹叢中先後竄出四人,三前一後,齊向峰頂奔去。無忌見那山峰上唯有幾株蒼松,並無房屋,不知謝遜被囚在何處,見四下更無旁人,當下展開輕功,跟著上峰,前面這四人的輕功大是不弱,當眞登高山如履平地,但無忌吸一口氣,加快脚步,追到離那四人只不過二十來丈。黑暗之中,只依稀看得出其中一個是女子,三個男子身穿俗家裝束,顯然並非少林寺中僧人。無忌尋思︰「這四人多半也是來向我義父爲雖的了,讓他們先和圓眞鬥一個你死我活。我且不忙插手。」將到峰頂,那四人奔得更加快了,無忌突然認出了其中二人身形︰「啊,那是崑崙派的何太沖和班淑嫻夫婦。」

猛聽得圓眞一聲長嘯,倏地轉過身來,疾衝下山,原來他早已察覺到身後有人。無忌應變快極,黑暗中一見他轉身下山,立時隱入道旁草叢,伏地爬行,向左移了數十丈,只聽得兵刃相交,鏗然聲響,圓眞已和來人動上了手。從那兵刃撞擊的聲音中聽來,乃是二人對付圓眞一人。無忌心下一動︰「尚有二人不上前圍攻,那是向峰頂找我義父去了。」當下從亂草叢中急攀上山。

到得峰頂,只見光禿禿地一片平地,只有三株蒼松,作品字形排列,枝幹插向天空。無忌暗暗奇怪︰「難道義父並非囚在此處?」聽得右首草叢中簌簌聲響,有人爬動,跟著便聽得班淑嫻道︰「咱們急速動手,薩師弟和南師弟未必絆得住這少林僧。」何太沖道︰「不錯。」兩人長身而起,撲向三株松樹。無忌生怕謝遜便在近處。遭了何太沖夫婦的毒手,不敢有半分大意,跟著便在草叢中爬行向前。突然之間,只聽得何太沖「嘿」的一聲,似乎已經受傷。無忌抬頭一看,見何太沖夫婦身處三株松樹之間,長劍揮舞,似在與人動手,但對敵之人却一個也瞧不見,偶爾傳出拍拍拍幾下悶響,似是長劍與什麼古怪的兵刃相撞。無忌心下大奇,更爬前幾步,凝目一看,不禁大吃一驚,原來斜對面兩株松樹樹幹都向内凹入一洞,剛好容納一人,每一株樹的凹洞中均坐著一名老僧,手舞黑色長索,攻向何太沖夫婦。一株松樹背向無忌。他瞧不見樹中情景,但樹旁也有一根黑索揮出,想必樹中亦有一僧。黑夜中漆黑一團,三根長索通體黝黑無光,舞動之時瞧不見半點影子。何太沖夫婦急舞長劍,嚴密守禦,只因瞧不見敵人兵刃來路,絶無反擊的餘地。這三根長索似緩實急,却又無半點風聲,滂陀大雨之下,黑夜孤峰之上,三名老僧行若鬼魅,説不盡的詭異。

何氏夫婦連聲叫嚷,急欲脱出這品字形的三面包圍,但每次向外衝擊,總是被長索擋了回來。無忌暗暗驚訝,見黑索揮動時無聲無息,這三名老僧的内功實已到了返照空明的境界,説到功力之純,比自己遠有過之,心想︰「圓眞説道,我義父交由他三位太師叔看守,看來這三位老僧便是空聞,空智的師叔。他每個人都身具七八十年的功力,我以一敵三,那是萬難取勝。」正焦躁間,已聽得「啊」的一聲慘叫,何太沖背上中了一索,從圏子中直摔出來,眼見得是不活了。班淑嫻又驚又悲,一個疏神,三鞭齊下,只打得她腦漿迸裂,四肢齊折,不成人形。跟著一根黑索一抖,將班淑嫻的屍身從圏子中抛出。

圓眞邉鬥邉退,叫道︰「相好的,有種的便到這裡領死。」那姓薩和姓南的兩個壯漢,都是崑崙派中的健者,明知圓眞是誘敵之計,却是毫不氣餒的挺劍直上。圓眞和這二人相鬥,以武功論原是不輸,但要一舉格殺二人,却是有所不能,最多傷得一人,餘下一人便會脱身逃走,當下引得二人追向松樹之間來。二人離松樹尚有數丈,突然見到何太沖的屍身,一齊停步。突然間兩根長索從腦後無聲無息的圏到,各自繞住了一人的腰間,長索一抖,將二人從數百丈高的山峰上抛了下去。兩人在山下撞得早已斃命,但身在中空時發住的慘呼,兀自纏繞數峰之間,回聲不絶。

無忌伏在草中,見三名老僧在片刻間連斃崑崙派的四位絶頂高手,舉重若輕,遊刃有餘,武功之高,實是生平罕見,比之鹿杖客和鶴筆翁,似乎猶有過之,縱不如太師父張三丰之深不可測,却也到了神而明之的境界。少林派中居然尚有這等元老,只怕連張三丰和楊逍也均不知。無忌心中怦怦亂跳,伏在草叢中一動也不敢動。

只見圓眞接連兩腿,將何太沖和班淑嫻的屍身踢入了深谷之中。屍身墜下,過了好一陣纔傳上兩響鬱悶的聲音。無忌暗想︰「何太沖夫婦雖然對我以怨報德,又圖害我義父,劫奪寶刀,但總是武學中的一派宗匠,不意落得如此下場,令人浩歎。」只聽得圓眞恭恭敬敬的道︰「三位太師叔神功蓋世,舉手之間便斃了崑崙派的四大高手,圓眞欽仰無已,不可言宣。」一名老僧哼了一聲,並不回答。圓眞又道︰「圓眞奉方丈師叔之命,謹來向三位太師叔請安,並有幾句話要對那囚徒言講。」一個枯槁的聲音道︰「空見師侄德高藝深,我三人最爲眷愛,原期他發揚我少林一派武學,不幸命喪此奸人之手。我三人坐関數十年,早已不聞塵務,這次看在空見師侄面上,纔到這山峰上來。這奸人既是死有餘辜,一刀殺了便是,何必諸多囉唆,擾我三人清修?」

圓眞躬身道︰「太師叔吩咐得是。只因方丈師叔言道,我恩師雖是爲此奸人謀害,但我恩師何等功夫,豈是這奸人一人之力所能加害?將他囚在此間,煩勞三位太師叔坐守,一來引得這奸人的同黨來救,好將當年害我恩師的仇人逐一除去,不使漏網。二來要他交出屠龍寶刀,以免該刀落入别派手中,{\upstsl{篡}}竊武林至尊的名頭,折了本派千百年的威望。」無忌聽到這裡,不由得暗暗切齒,心道︰「圓眞這惡賊當眞是千刀萬剮,難抵其罪,一番花言巧語,請出這三位數十年不問世事的高僧來,假他三人之手,屠戮武林中的高手。」只聽得一名老僧哼了一聲,道︰「你跟他講吧。」

此時大雨兀自未止,雷聲隆隆,愈增威勢,只見圓眞走到三株松樹之間,跪在地下,對著地面説道︰「謝遜,你想清楚了嗎?只須你説出收藏屠龍刀的所在,我立時便放你走路。」無忌大是奇怪︰「怎地他對著地面説話,難道此處有中地牢,我義父囚在其中?」忽聽得一個聲音清越的老僧怒道︰「圓眞,出家人不打誑語,你何以騙他?他若是説出藏刀的所在,難道你眞放了他麼?」圓眞道︰「太師叔明鑒︰弟子心想,恩師之仇雖深,但兩者相權,還是以本派威望爲重。只須他説出藏刀之處,本派得了寶刀,咱們便放他逃生,三年之後,弟子再去找他爲恩師報仇。」那老僧道︰「這也罷了。武林中仁義爲先,言出如箭,縱對大奸大惡,咱們少林子弟也不能失信於人。」圓眞躬身道︰「謹奉太師叔教誨。」

無忌越聽越覺這三名少林僧不但武功卓絶,且是有德的高僧,只是墮入了圓眞的奸計而不自覺。只聽圓眞又向地下喝道︰「謝遜,我太師叔的話,你可聽見了麼?三位老人家答應放你逃生。」忽聽得地底下傳上來一個聲音道︰「成崑,你還有臉來跟我説話麼?」無忌一聽到這聲音雄渾蒼涼,正是義父的口音,不由得心中大震,恨不得立時撲上前去,一掌擊斃成崑,將謝遜救了出來。但想到三位少林高僧鬼神莫測的奇技,知道自己一現身,三條黑索便招呼過來,即使成崑不出手,自己也不是這三位高僧聯手之敵,當下強自克制,尋思︰「待那圓眞惡僧走後,我上前拜見三僧,説明這中間的原委曲折。他三位佛法精深,不能不明是非。」

反聽得圓眞歎道︰「謝遜,你我年紀都大了,往日的恩恩怨怨,又何必苦苦掛在心頭?不到二十年,你我同歸黃土。我有虧待你之處,也有過對你不錯的日子。從前的事,一筆勾銷了吧。」謝遜聽他絮絮而語,並不理睬,待他停口,便道︰「成崑,你還有臉跟我説話麼?」圓眞説了半天,見只他是這一句話,不由得怒氣上衝,喝道︰「我念著昔日的恩義,對你始終没下毒手,哼,你還記得我的『萬蟻攢心指』麼?」

無忌一聽到「萬蟻攢心指」五字,不由怒火上衝,他曾聽謝遜説過,那是一種最爲陰狠毒辣的武功,中此指者,有如千千萬萬隻螞蟻在五臟六腑一齊咬囓,搔不著摸不到,却是痛癢雖當,直至自己將全身肌肉一塊塊撕爛,仍是不得氣絶。他心意已決,倘若圓眞要向謝遜下此毒手,那時須顧不到三僧難敵,非捨命相救義父不可。只聽謝遜在地牢中仍是這句話︰「成崑,你還有臉跟我説話麼?」

圓眞冷冷的道︰「我且容你再想三天,三天之後,若再不説出屠龍刀的所在。你仔細捉摸萬蟻攢心的滋味吧。」説著站起身來,向三僧禮拜,走下山去。

無忌待他走遠,正欲長身向三僧訴説,突覺身周氣流略有異狀,這一下襲擊事先竟無半點朕兆,無忌一驚之下,著地滾開,只覺兩條長長的物事,從臉上橫掠而過,相距不逾半尺,去勢奇急,即是絶無勁風,正是三高僧的兩條黑索。無忌只滾出丈餘,又是一條黑索向他胸口點到,這一次那黑索如長矛、如白桿,化成一條筆直的兵刃,疾刺而至,同時另外兩條黑索,也是從身後纏了過來。無忌初時見崑崙派四大高手轉瞬間便命喪三條黑索之下,已知這三位少林林僧的武功奇幻難測,此刻身當其難,更是千鈞一髮,性命懸於呼吸之間。他左手一翻,抓住當胸點來的那條黑索,正想從旁甩去,突覺那條長索一抖,一股排山倒海的内勁向胸口撞到,這内勁只要中得實了,當場便是肋骨斷折,五臟齊碎。好張無忌,便在這電光石火般的一刹那間,右手後揮,撥開了從身後襲至的兩條黑索,左手乾坤大挪移心法混著九陽神功,一提一送,身隨勁起,颼的一聲,身子直衝上天。

正在此時,天空中白光耀眼,三四道閃電齊亮,只聽得一位高僧「{\upstsl{嗯}}」的一聲,對無忌的功夫頗感驚異。這幾道閃電照亮了無忌身形。三位高僧抬頭上望,見這身具絶頂神功的高手竟是一個面目汚穢的鄕下少年,更是驚訝。三條黑索便如三頭張牙舞爪的墨龍相似,從下面急升而上,長及五丈,分從三面捲向無忌身子。無忌藉著電光,一瞥間已看清了三僧的容貌,坐在東北角那僧臉色漆黑,有如生鐵;西北角那僧枯黃如槁木;正南方那僧却是臉色慘白如紙。三僧均是面頰深陥,瘦得全無肌肉,黃臉的僧人眇了一目,三個僧人五道目光映著閃電,更顯得燦然有神。

眼見三根黑索將捲上身來,無忌一撥一帶,一捲一纏,借著三人的勁力,將三根黑索捲在一起。這一招手勢,却是張三丰所傳的武當派太極心法,勁成渾圓,三根黑索上所帶的内勁立時被牽引得絞成一團。只聽得轟隆隆幾聲響喨,三個霹靂連續而至,這天地雷電之威,直是驚心動魄。無忌在半空中翻了一個筋斗,左足在一株松樹的枝幹上一勾,身子已然定住,叫道︰「後學晩輩,明教教主張無忌,拜見三位高僧。」説著一足站在松幹,一足凌空,躬身行禮。那松樹的枝幹隨著他這一拜之勢,猶似波浪般上下起伏,無忌却見穩穩站住,姿勢極是美妙。他雖躬身行禮,但居高臨下,不落半點下風。

三高僧一覺黑索被他内勁帶動,相互纏繞,反手一抖,三索便即分開。三僧適纔三招九式,每一式中都隱藏數十招變化,數十下殺手,那知無忌將這三招九式一一化開,儘管化解時每一式都是險到了極處,稍有厘毫之差,便是筋折骨斷、喪生殞命之禍,仍是顯得揮灑自若,履險如夷。三高僧一生之中,從未遇到過如此高強敵手,不禁一齊心下駭然。他們却不知無忌化解這三招九式,實已竭盡生平全力,正借著松樹枝幹的高低起伏,暗自調勻丹田中已亂成一團的眞氣。

無忌適纔所使武功,包括九陽神功、乾坤大挪移、太極拳,而最後半空中十個筋斗,却是聖火令上所載的心法。那三位少林高僧雖然各是身懷絶技,但坐関數十年,不聞世事,於無忌這四種功夫竟是一種也没見過,只是隱約覺得,他的内勁和少林九陽功似是一路,但雄渾精微之處,遠較少林派神功爲勝。待得聽他自行通名,竟是明教教主,三僧心中的欽佩和驚訝之情,登時化爲滿腔怒火。那臉色慘白的老僧森然道︰「老納還道是何方高人降臨,却原來是魔教的大魔頭到了。老衲師兄弟三人坐関數十年,遠離少林寺數百里之遙,不但不理俗務,連本寺大事,也是素來不加聞問。不意今日得與魔教教主相逢,實是生平之幸。」

張無忌聽他左一句「魔頭」,右一句「魔教」,顯是對本教惡感極深,不由得大是躊躇,不知如何開口申述纔是。只聽那黃臉眇目的老僧説道︰只聽那黃臉眇目的老僧説道︰「魔教教主是楊破天啊!怎麼是閣下?」張無忌道︰「楊教主逝世已近三十年了。」那黃臉老僧「啊」的一聲,不再説話,這一聲驚呼之中,蘊藏著無限的傷心和失望。無忌心想︰「他聽到楊教主逝世的訊息,極是難過,想來他當年和楊教主定是交情甚深。義父是楊教主的舊部,我一面動以故人之情,一面再説出楊教主爲圓眞氣死的原由,且看如何?」便道︰「大師想必識得楊教主了?」黃臉老僧道︰「自然識得。老納若非識得大英雄楊破天,何致成爲獨眼之人?咱師兄弟三人,又何必坐這三十餘年的枯禪?」這幾句話説得平平淡淡,但其中所含的沉痛和怨毒,却是既深且巨。無忌心中暗叫︰「糟糕,糟糕。」從他言語中聽來,這老僧的一隻眼睛,便是壞在楊破天手中,而他師兄弟三人坐枯禪一坐三十餘年,痛下苦功,就是爲了要找楊破天報仇。這時聽得楊破天已死,自是不免大失所望了。

忽然間那黃臉老僧一聲清嘯,説道︰「楊破天既死,咱三人的深仇大怨,只好著落在現任教主身上。張教主,老納法名渡厄,這位白臉師弟,法名渡劫,這位黑臉師弟,法名渡難。空見、空聞、空智、空性,都是咱們師侄。空見、空性二人,都是死在貴教手下。到底魔教使了什麼卑鄙無恥的手段,咱們也不想追究。貴教主既然來到此地,自是有恃無恐。數十年來恩恩怨怨,咱們武功上一作了斷便是。」

無忌道︰「晩輩此來,只在營救義父金毛獅王謝大俠,與貴派並無樑子。空見神僧雖爲我義父失手所傷,這中間頗有曲折。至於空性神僧之死,與敝派却是全無瓜葛。三位不可專聽一面之辭,須得明辨是非才好。」白臉老僧渡劫道︰「依你説來,空性爲何人所害?」無忌皺眉道︰「據晩輩所知,空性神僧是死於朝廷汝陽王府的武士手下。」渡劫道︰「汝陽王府的衆武士爲何人率領?」無忌道︰「汝陽王之女,漢名趙明。」渡劫道︰「我聽圓眞言道,此女已然和貴教聯手作了一路,她叛君叛父,投誠明教,此言是眞是假?」這渡劫的辭鋒咄咄逼人,一步緊於一步,張無忌不擅説謊,只得道︰「不錯,她\dash{}她現下\dash{}現下棄暗投明。」渡劫朗聲道︰「殺空見的,是魔教的金毛獅王謝遜,殺空性的是貴教的趙明。這個趙明更攻破少林寺。將我合寺弟子,一鼓擒去,最不可恕者,竟在本寺祖師達摩老祖面壁參禪的石像之上,刻以侮辱之言。再加上我師兄的一隻眼珠,咱三人合起來一百年的枯禪,張教主,這筆帳不跟你算,却跟誰算去?」

無忌長歎一聲,心想自己既是承認收容趙明,她以往的過惡,只有一古腦児的承攬在自己身上,至於楊破天和謝遜昔日給下的仇恕,時至今日、渡劫之言不錯︰我若不擔當,誰來擔當?

張無忌身子挺直,勁貫足尖,那條起伏不已的枝幹突然定住。紋絲不動,朗聲説道︰「三位老禪師既如此説,晩輩無可逃責,一切罪愆,便由晩輩一人承當便是,但我義父傷及空見神僧,内中實有無數苦衷,還請三位老禪師恕過。」渡厄道︰「你憑著什麼,敢來替謝遜説情?難道我師兄弟三人,便殺你不得麼?」無忌心想事已至此,只有奮力一拚,便道︰「晩輩以一敵三,萬萬不是三位的對手,請那一位老禪師賜教?」白臉老僧渡劫道︰「咱們單打獨鬥,並無勝你把握。這等血海深仇,説不上江湖規矩,好魔頭,你下來領死吧,阿彌陀佛!」他口中一宣佛號,渡厄、渡難二僧齊聲應道︰「我佛慈悲!」三根黑索倏揮飛起,疾向無忌身上捲來。

無忌身子一沉,從三條黑索間竄了下來,雙足尚未著地,半空中一變身形,向渡難撲了過去。渡難左掌一立,猛地翻出,一股極猛的勁風向無忌小腹擊出。無忌轉身卸勁,以乾坤大挪移心法將他勁力化解了開去,便在此時,渡厄和渡劫的兩根黑索同時捲到。無忌滴溜溜轉了半個圏子,堪堪避開。渡劫雙拳猛揮。無聲無息的打了過來。無忌在三株松樹之間,見招拆招,驀地裡一掌劈出,將數百類黃豆大的雨點挾著一股勁風向渡厄飛了過去。渡厄側頭一讓,還是有數十顆打在臉上,竟是隱隱作痛,他喝了一聲︰「好小子!」黑索一抖,轉成兩個圓圏,從半空中往無忌頭頂套下。無忌身如箭飛,既避索圏,又攻向渡劫。他越鬥越是心驚,只覺身周的氣流在三條黑索和三股掌風激盪之下,竟似漸漸凝聚成膠一般。他自習成武功以來,從未遇到過如此高強的對手,三僧不但招數精巧,内勁更是雄厚無比。無忌初時七成守禦,尚有三成攻勢,但鬥到二百餘招時,漸感體内眞氣不純,唯有只守不攻,以圖自保。

他的九陽神功本來用之不盡,愈使愈強,但其時在三僧聯攻之下,每一招均須耗費極大的内力,慢慢感到了後勁不繼,這又是他自臨敵以來從未經歷過之事。再拆數十招,他暗自尋思︰「再鬥下去,只有徒自送命。留得青山在,不怕没柴燒。今日且自脱身、待去約得外公、揚左使、范右使、韋蝠王,咱們五人合力,定可勝得三僧,那時再來營救義父。」當下向渡厄急攻三招,待要搶出圏子,不料三條黑索所組成的圏子已如銅牆鐵壁相似,無忌數次衝擊,均被攔了回來。非但無法脱身,反而被渡難的黑索在腰間掃了一下,拉去了一大片皮肉。這黑索不知是用何種物事製成,柔若遊絲却又堅逾鋼鐵。無忌心下大驚︰「原來三僧聯手,有如一體,這等心意相通的功夫,世間當眞有人能做到麼?」他那知渡厄、渡劫、渡難三僧坐這三十餘年的枯禪,最大的功夫便是用在「心意相通」之上,一人動念,其餘二人立即意會,此種心靈感應説來甚是玄妙,但三人在斗室中相對三十餘年,專心致志以練感應,心意有如一體,亦非奇事,他又想︰「由此觀之,縱然我約得外公等數位高手同來,亦未必能攻破他三人心意相通所組成的堅壁。難道我義父終於是無法救出,我今日要死在此地?」

他心中一急,精神略散,肩頭登時被渡劫五指掃中,痛入骨髓,無忌一動念間,心道︰「我死不足惜,義父的冤屈却須代他申雪。義父一生高傲,既是落入人手,決不肯以一言半語爲自己辯解。」當下朗聲説道︰「三位老禪師,晩輩今日被困,性命難保,大丈夫死則死耳,何足道哉?有一事却須言明\dash{}」呼呼兩聲,兩條黑索分從左右襲到,張無忌左撥右帶,化開來勁,繼續説道︰「那圓眞俗家姓名,叫作成崑,外號混元霹靂手,乃是我義父的業師\dash{}」

\chapter{長嘯下山}

三位少林高僧見他一面拆招化勁,一面吐聲説話,這等内功修爲,實非自己所能,不由得更增了幾分忌憚之意。但這三僧認定明教乃是無惡不作的魔教,這教主武功越高,爲害世人越大,眼見他身陥重圍,已然無法脱困,正好乘機除去,我是積下了無量功德,是以一言不發,黑索和掌力加緊施爲。張無忌繼續説道︰「三位老禪師須當知曉,這成崑和明教教主楊破天,凡是同門師兄弟,他二人同戀師妹,那位師妹却終於成了楊教主的夫人。成崑心下不忿,是以和明教結下了深仇大怨\dash{}」他原原本本,將成崑如何處心積慮要毀明教、如何與楊夫人私通幽會以致激死楊破天、如何假酔圖姦謝遜之妻,殺其全家,如何逼得謝遜亂殺武林人士,如何拜空見神僧爲師,誘使空見身受謝遜一十三拳、如何失信不出,使空見飲恨而終\dash{}渡厄等三僧越聽越是心驚,這些事蹟似乎件件匪所夷思,但件件入情入理,無不若合符節。渡厄手上的黑索首先緩了下來。

無忌又道︰「晩輩不知楊教主如何與渡厄大師結仇,只怕其中有奸人挑撥是非,此人定是這圓眞無疑。渡厄大師不妨回思往事,印證晩輩是否虛言相欺。」渡厄{\upstsl{嗯}}的一聲,停鞭不發,低頭沉吟,説道︰「那也有些道理。老衲與楊破天結仇,這成崑爲我出了大力,後來他意欲拜老納爲師,老納向來不收弟子,這纔引薦他拜在空見師侄的門下。如此説來,那是他有意安排的了?」無忌道︰「不特如此,目下他更覬覦少林寺掌門方丈之位,收羅黨羽,陰謀密計,要害了空聞神僧\dash{}」這句話尚未説畢,突然間隆隆聲響,一塊巨大的圓石從左首向三株松樹間發將進來。渡厄喝道︰「什麼人?」黑索揮動,拍拍兩響,繫在脚石之上,只打得石屑飛舞。圓石後突然竄出一條人影,撲向無忌,寒光閃動,一柄短刀刺向無忌咽喉。

這一下來得突兀之極,無忌正自全力擋架渡劫、渡難二僧的黑索和拳掌,全没防到忽然竟會有人偸襲,黑暗中只覺風聲颯然,短刀的刀尖已刺到喉數,危急中身手斜刺向旁射出,嗤的一聲啊,短刀已將他胸口衣服劃破了一條大縫,只須有厘毫之差,便是開膛破胸之禍。此人一擊不中,藉著那大石掩身,已滾出三僧黑索的圏子。無忌暗叫一聲︰「好險!」喝道︰「成崑惡賊,有種的便跟我對質,想殺人滅口麼?」適纔短刀那一刺,他雖未看清人形,但以對方身法之捷,出手之狠,内勁之強,除成崑外更無旁人。少林三僧的三條黑索猶如三隻長手,伸將出去,捲向大石,一迴一揮之間,將那千餘斤的大石抬了起來,直貫出去,成崑却已遠遠的下山去了。

渡厄道︰「當眞是圓眞麼?」渡難道︰「確然是他。」渡厄道︰「若非他作賊心虛何必\dash{}」剛説了「何必」兩字,驀地裡四面八方呼嘯連連,撲上七八條人影,當先一人喝道︰「少林和尚枉爲佛徒,殺害這許多人命,不怕罪孼麼?大夥児齊上。」八個人各挺兵刃,向三位老僧攻了上去。

無忌坐在三僧之間,只見這八人中有三人持劍,其餘五人或刀或鞭,個個武學精強,霎時間便和三禪師的黑索鬥在一起。無忌看了一會,見那三個使劍的劍招,和數日前死在少林僧手下的青海三劍乃是一路,但變化精微,勁力雄渾,却顯是在青海三劍之上,想必是青海派中長輩的佼佼人物,這三人合力攻擊渡厄。另有三人合攻渡劫,餘下二人則聯手對付渡難。渡難的對手雖只二人,但這二人的武功却比其餘各人又高出一籌。各人鬥到二十餘招時,無忌已看出渡難漸落下風,渡厄却是穩佔先手,以一敵三,兀自行有餘力。

又拆十餘招,渡厄看出渡難應付維艱,當下黑索一抖,偸空向渡難的兩名對手晃了過去。那二人都是身形極高,黑鬚飄動,年事已高,手脚却是極爲矯捷,一個手使一對判官筆,另一個使打穴橛,均是點穴打穴的名家。渡厄和渡難均知這二人甚是了得,此刻身在數丈之外,已隱然感到他二人兵刃上發出來的勁風,倘若被他二人欺近身來,施展短兵刃上的長處,勢必更爲厲害。青海派的三柄長劍上壓力一鬆,慢慢又扳回劣勢。這麼一來,變成渡劫以一敵三,渡厄、渡難二僧却是以二敵五,一時間相持不下,成了個不勝不敗的局面。

無忌只看得暗暗稱奇︰「這八個人的武功,都是足可與韋蝠王相頡頑。只比滅絶師太稍遜,却似在何太沖之上。但這八人的來歷,除了三個是青海派外,其餘五人我一槩不知。可見天下之大,草莽間臥虎藏龍,不知隱伏著多少默默無聞的英雄好漢。」

十一個人拆到一百餘招時,少林三僧的黑索漸漸收短。黑索一短,揮動時可節省内力,但攻人時的靈動,却也減了幾分。更鬥數十招,三僧的黑索又縮短了六七尺。那兩名黑鬚老人越鬥越近,兵刃上的威力大增,尋瑕抵隙,只盼撲到三僧的身邉。但少林三僧的黑索收短後,守禦相應嚴密,三條黑索組成的圏子上似有無窮彈力,黑鬚老人每次變招搶攻,均被這黑索之圏彈了出來。這時三僧已聯成一氣,成爲以三敵八之勢。

少林三僧一面惡戰,一面心下暗暗叫苦,與這八人相鬥,再久也不致落敗,只須將黑索再縮短八尺,那便組成了「金剛伏魔圏」,别説八名敵人,便是十六人,三十二人,那也攻不進來,可是這圏子之中,却隱伏著一個心腹之患的強敵。張無忌這時一出手,内外夾攻,立時便取了少林三僧的性命。三僧見他盤膝而坐,似乎在等待良機,要讓自己三人和外敵拚到雙方筋疲力竭,他再來收漁人之利。這時三僧的内功已施展到了淋漓盡致,有心要呼喚向山下少林寺求援,却是開口不得,這當児只要輕輕吐一個字立時氣血翻湧,縱非立時斃命,也是身受内傷,成爲廢人。三僧心下都是自責過於自大,當強敵來攻之初,竟未出聲通知本寺人衆。否則只要達摩堂或羅漢堂有幾名好手來援,便可克敵取勝。

這情勢無忌自也早已看出,這時要取三僧性命,且是舉手之勞,但想大丈夫不可乘人之危,何況三僧只是受了圓眞瞞騙,並無可死之道,而殺了三僧後獨力應付外面八敵,亦是同樣的艱難,眼見雙方勝負非一時可決,他低頭一看,只見一塊極巨的岩石壓住地牢之口,只露出一縫,作爲謝遜呼吸與傳遞食物之用。這巨石重達數千斤,絶非一二人之力所能推動,但張無忌在光明頂地道之中,學得乾坤大挪移心法後,曾推開厚達丈許的石門,與彼相較,這塊巨石也不見更重過那扇門,只是此處地下光禿禿地,較難著手。他心想時機稍縱即逝,若是相鬥的雙方分了勝敗,或是少林寺有人來援,便救不了義父,當下跪在石旁,雙掌推住巨石,使出乾坤大挪移心法,勁力一到,那巨石便即緩緩移動。

那巨石移開不到一尺,突然間背後風動勁到,渡難一掌向他背心拍了下來。張無忌卸勁借力,拍的一聲響,他背上衣衫碎了一大塊,在狂風暴雨之中,片片作蝴蝶飛舞,但渡難這一掌的掌力。却給他傳到了巨石之上,隆隆一響,那巨石立時又移開一尺。這掌力雖是卸去,未受内傷,但初受之際,他全身力道盡數用來推石,背心上也是痛入臟腑。渡難一掌虛耗,黑索上露出破綻,一名黑鬚老人立時撲進索圏。

少林三僧的軟索均是擅於遠攻,不利近擊,那黑鬚老者一搶進圏子,右手點穴橛便向渡難左乳下打去。渡難左手肘掌,運勁逼開他點穴橛的一擊。黑鬚老者左手食指疾伸,戳向渡難的「膻中穴」。渡難暗叫︰「不好了!」那料到他「一指禪」的點穴功夫,竟比他打穴橛的打穴更是厲害,危急之下,只得右手撤開黑索,豎掌一封,護住胸口,跟著姆指、食指、中指三指翻出,立時反攻。他雖將這黑鬚老者擋住了,但黑索離手,那使判官筆的老者當即搶前。少林三僧中三條黑索去其一,眼見「金剛伏魔圏」已被攻破。

突然之間,那條摔在地下的黑索索頭昂起,便如一條假死的毒蛇忽地反噬,呼嘯而出,向那使判官筆的老者額頭點去,索頭未到,索上所挾勁風已令對方一陣氣窒。那老者急舉判官筆擋架,索筆相交,一震之下,雙臂酸麻,左手判官筆險些脱手飛出,右手判官筆被震得擊向地下山石,只擊得石屑紛飛,火花四濺。那條黑索展將開來,將青海派三劍又逼得退出丈許,「金剛伏魔圏」不但回復原狀,威力更勝於前。少林三僧驚喜交集之下,只見黑索的另一端竟是持在張無忌手中。他並未練過「金剛伏魔圏」的功夫,説到心意相通、動念便知的配合無間,那是不及渡難,但内力之剛猛,却是無與倫比,一條黑索上所發出的内勁,眞如排山倒海一般,向著四面八方逼去。渡厄與渡劫的兩條黑索在旁相助,登時逼得索外七人連連倒退。

渡難專心致志對付那黑鬚老者,不論武功和内力修爲,都是勝了一籌,他坐在松樹穴中,並不起身,十指拍、戳、彈、勾、點、拂、擒、拿,數招之間,便令那黑鬚老者迭遇險招。那老者見同伴七人處境也均不利,當下一聲怒吼,從圏中躍出。張無忌將黑索往渡難手中一塞,俯身運起乾坤大挪移心法,又將壓在地牢上的巨石推開了尺許,對著露出來的洞穴説道︰「義父,孩児無忌救援來遲,你能出來麼?」謝遜道︰「我不出來。好孩子,你快快走吧!」無忌大奇,道︰「義父,你是被人點中了穴道,還是身有{\upstsl{銬}}鍊?」也不等謝遜回答,便即縱身躍入地牢,{\upstsl{噗}}的一聲,水花濺起,原來地牢中積水齊腰,謝遜半個身子浸在水裡。

無忌心中悲苦,伸手抱著謝遜,在他手足上一摸,並無{\upstsl{銬}}鍊等物,再在他幾處主要穴道上一加推拿,也非被人下了手脚。當下抱著他的臂膀,一躍而上,兩個人濕淋淋的飛出地牢,坐在巨石之上。無忌道︰「他兩下裡劇鬥方酣,此時脱身,最好不過。義父,咱們走吧。」説著挽住他手臂,便欲拔步。謝遜却坐在石上,動也不動,抱膝説道︰「孩子,我生平最大的罪孼,乃是殺了空見大師。你義父若是落入旁人之手,那是勢須奮戰到底,但今日是囚在少林寺中,我甘心受戮,還了空見大師這條性命。」無忌急道︰「你失手傷了空見大師,那是成崑這惡賊奸計擺佈,何況義父你全家血仇未報,豈能死在成崑的手下?」謝遜嘆道︰「我這一個多月來,在這地牢中每日聽著三位高僧誦經唸佛,聽著山下少林寺中傳來的晨鐘暮鼓,回思往事,你義父手下染了這許多鮮血,實是百死難續。唉,種種因果報應,我比成崑作更多,好孩子,你别管我,自己快下山去吧。」

無忌越聽越急,大聲道︰「義父,你不肯走,我可要用強了。」説著轉過身來,抓住謝遜雙手,便往自己背上一負。只聽得山道上人聲喧嘩,有數人大聲叫道︰「什麼人到少林寺來撒野?」一陣踐水急奔之聲,十餘人搶上山來。無忌持住謝遜雙腿,正要起步,突然間後心「大椎穴」一麻,雙手無力,只得放開了謝遜,急得幾乎要哭了出來,叫道︰「義父,你\dash{}你何苦如此?」

謝遜道︰「好孩子,我所受冤屈,你已對三位高僧分説明白。我所作的罪孼,却須由我自己身受報應。你此時若再不去,我的仇怨誰來代我報復?」説到最後二句,聲音突然提高。無忌心中一凜,但見十餘名少林僧各執禪杖戒刀,向那八人攻了上去{\upstsl{乒}}{\upstsl{乒}}{\upstsl{乓}}{\upstsl{乓}}交手數合,那持判官筆的黑鬚老者情知再鬥下去,今日難逃公道,只是功敗垂成,被一名無名少年壞了大事,心下實是大大的不忿,朗聲喝道︰「請問松間少年高姓大名,河間郝密、卜泰,願知是那一位高人橫加干預。」渡厄黑索一揚,説道︰「明教張教主,天下第一高手,河間雙煞神怎地不知?」持判官筆的郝密「噫」的一聲,雙筆一揚,縱出圏子,其餘七人跟著退了出去。少林僧待要攔阻,但武功上比那八人遜了一籌。八人併肩一衝,一齊下山去了。

渡厄等三僧對謝遜與張無忌對答之言,盡數聽在耳裡,又想到適纔無忌就算不是乘人之危,只須袖手旁觀,兩不相助,當卜泰破了「金剛伏魔圏」攻到身邉之時,以河間雙煞下手之辣,此刻三僧早已不在人世。三僧放下黑索,站起身來,向無忌合什爲禮,齊聲道︰「多感張教主大德。」無忌急忙還禮,説道︰「份所當爲,何足掛齒?」渡厄道︰「今日之事,老衲原當讓謝遜隨同張教主而去,適纔張教主眞要救人,老納須是無力阻攔。只是老衲師兄弟三人奉本寺方丈之命,看守謝遜,佛前立下重誓,若非我三人性命不在,決不能放謝遜脱身。此事関涉本派千百年的榮辱,還請張教主見諒。」無忌哼了一聲,並不回答。

渡厄又道︰「老衲喪眼之仇,今日是揭過了。張教主要救謝遜,可請隨時駕臨,只須殺敗老衲師兄弟三人,立時可陪獅王同去,張教主多約幫手,車輪戰也好,一湧而上也好,咱師兄弟只是三人應戰。在張教主再度駕臨之前,老衲三人自當維護謝遜周全,決不容圓眞辱他一言半語、傷他一毫一髮。」

無忌向謝遜望了一眼,黑暗中只見他一個巨大的身影,長髮披肩,低首而立,似乎心中深自懺悔昔日罪惡,無復當年神威凜廉的雄風。無忌泪水幾欲奪眶而出,尋思︰「今日我是打不過他們的了,這三僧既如此説,義父又不肯走,只有約了外公、楊左使、范右使他們再來鬥過。但那三條黑索組成的勁圏便如銅牆鐵壁相似,適纔若不是渡難在我背上打了一掌,卸了勁力,那卜泰萬萬攻不進來。下次縱有外公和左右光明使相助,是否能彀破得,實未可必。唉,眼下也只有走一步算一步了。」便道︰「既是如此,數日間便當再來領教三位的高招。」回身抱著謝遜的腰,道︰「義父,孩児走了。」謝遜點了點頭,撫摸他的頭髮,説道︰「你不必再來救我,我是決意不走的了。好孩子,盼你事事逢凶化吉,不負你爹娘和我的期望。你當學你爹爹,不可學你義父。」無忌道︰「爹爹和義父都是英雄好漢。只不過爹爹運氣好,義父運氣不好。一般的大丈夫,都是孩児的好榜樣。」説著躬身一拜,身形晃處,已自出了三株松樹圍成的樹子,向少林三僧一舉手,展開輕功,倏忽不見,但聽他清嘯之聲,片刻間已在里許之外。衆僧相顧駭然,説不出話來,各人早聞明教張教主武功卓絶,抑没想到神妙至斯。

張無忌既見形跡已露,索性顯一手功夫,好教少林僧衆心生忌憚,善待謝遜。他這一聲清嘯鼓足了中氣,綿綿不絶,在大雷雨中飛揚而出,有若一條長龍行經空際。他足下施展全力,越奔越快,嘯聲也是越來越響,少林寺中千餘僧衆一齊在夢中驚醒,直至那嘯聲漸去漸遠,方始紛紛議論。空聞、空智等得報是張無忌到了,均是平增一番憂患。張無忌一聲龍吟般的清嘯,奔出數里,突然道旁一株柳樹後有聲叫道︰「喂!」跟著一個黑影躍了出來,正是趙明。無忌停嘯止步,伸手挽住了她,見她全身被大雨淋濕了,髮上臉上,水珠不斷流下。趙明道︰「怎麼啦?跟少林寺的禿頭們動過手了麼?」無忌道︰「是。」趙明道︰「謝大俠怎樣了?有没見到?」無忌挽著她手臂,在大雨中緩步而行,將適才情事簡略的説了。趙明沉吟半晌,道︰「你有没有問他如何失手遭擒?」無忌道︰「我只想著怎地教義父脱險,没空問到這些閒事。」趙明嘆了口氣,不再作聲。無忌道︰「你不高興麼?」趙明道︰「在你是閒事,在我就是要緊事。好啦,等救出了謝大俠,再問也不遲。我只怕\dash{}」無忌道︰「怕什麼?」你擔心咱們救不了義父?」趙明道︰「明教比少林派強得多,要救謝大俠,終究是辦得到的,我就怕謝大俠決心一死,以殉空見神僧。」無忌也是擔心著這件事,問道︰「你説會麼?」趙明道︰「但願不會。」

二人一路言講,走到了杜氏夫婦的茅舍之前,趙明笑道︰「你行跡已露,不能再瞞他二人了。」見茅舍之門半掩,便伸手推開。他搖了搖身子,抖去一些濕水,踏步進去,忽然聞到一陣血腥之氣。他心下一驚,左手反掌將趙明推到門外,黑暗中突然有人伸手抓來。這一抓無聲無息,快捷無倫,待得驚覺,五根手指已觸到面頰。無忌此時已不及閃避,一足飛出。逕踢那人胸口。那人反手一勾,肘錘打向無忌腿上環跳穴,黑暗中招數極是狠辣。無忌只須縮腿一讓,敵人左手就挖去了自己的一對眼珠,當即提手虛抓,他料敵奇準。這麼一抓,剛好將敵人的左手拿在掌中,便在此時,環跳穴上一麻,立足不定,右腿跪倒。

他正要乘勢扭斷敵人的手腕,只覺掌中所握的那隻手掌溫軟柔滑,乃是女子的手掌,心中一動,没重下手,只是提起那人身子,往外甩出,撲的一聲,右肩劇痛,已中了敵人一刀。那人一躍出屋。一掌向趙明臉上拍去。無忌知道趙明擋不了這一掌,非當場斃命不可,忍痛縱起,也是一掌拍出,雙掌相交,仍是没半點聲息,無忌一掌陽剛之勁,全爲對方陰柔的内力化去。那人一擊不中,更不再擊,借著這對掌之力,縱出數丈以外,一晃身間,便在黑暗中隱没不見。

趙明驚道︰「是誰?」無忌「嘿」了一聲,懷中火摺已被大雨淋濕,打不了火,知道自己右肩上插了敵人的短匕,生怕匕上有毒,不即拔出,道︰「你點亮了燈。」趙明到厨下取出火刀火石,點亮油燈,一見無忌肩頭的匕首,大吃一驚。無忌看了看刀鋒,並未餵有毒藥,笑道︰「些些外傷,無関緊要。」左手三根手指拈住匕首之柄,便拔了出來,一轉頭,只見杜百當和易三娘縮身在屋角之中,當下顧不得止住傷口流血,搶上去一看,只見二人早已死去多時。

趙明驚道︰「我出去時,他二人尚自好好地。」無忌點點頭,等趙明替他裹好傷口,拿起那匕首一看,正是杜氏夫婦所使的兵刃,再著屋中,只見樑上、柱上、桌上、地下,插滿了短刀,顯是敵人曾與杜氏夫婦一番劇鬥,將他夫婦的短刀一一打得出手,這纔動手加害。趙明心下駭然,道︰「這人武功厲害得很啊。」無忌想起適纔小室中摸黑相鬥,雖只三招兩式,却是兇險到了極處,若非料到那人要來抓自己眼珠,不但此時已成了瞎人,只怕自己與趙明都已屍橫就地。但看杜百當和易三娘的屍身時,只見胸口數十根肋骨,根根斷成數截,連背後的肋骨也是如此,顯是爲一種極陰狠可極厲害的掌力所傷。

張無忌數經大敵,什麼兇險的情景也都遭遇,但回想適纔暗室中這三下兔起鶻落般的交手,不由越想越驚。今晩兩場惡鬥,第一場以一敵三,歷時甚久,但驚心動魄之處,遠不如第二場瞬息間的三招兩式。趙明又問︰「那是誰?」無忌搖頭不答。趙明突然間已知是誰,眼中流露出恐懼神色,呆了半晌,撲向無忌懷中,嚇得哭了出來。兩人心下均知,若不是趙明聽到無忌嘯聲,大雨中奔將出去迎接,鬼使神差的逃過了一難。那麼此刻死在屋角中的已不是兩人而是三人了。

無忌輕拍她的背脊,柔聲安慰。趙明道︰「那人只想殺我,却累得杜氏夫婦死於非命。」無忌道︰「這幾日中,你千萬不可離開我身邉。」沉吟片刻,又道︰「不到一年間,何以功力武功進展如此迅速?」當世除我之外,只怕無人能護得你周全。」

次日清晨,無忌拿了杜百當鋤地的鋤頭,挖了個深坑,將杜氏夫婦埋了,與趙明一齊跪下來拜了幾拜。剛站起身來,忽聽得山坳中少林寺裡鐘聲{\upstsl{噹}}{\upstsl{噹}}不絶,撞得甚是緊急,接著東面放起青色煙火,直衝上天、南方紅色、西方白色,北方黑色,數里外更升起黃色煙火。五道煙火,將少林寺圍在中間。無忌叫道︰「明教五行旗五旗齊到,那是正面跟少林派幹起來啦,咱們快去。」忽忽與趙明換了衣服,洗去手臉的汚泥,快步向少林寺奔去。只行出數里,便見一隊白衣的明教教聚手執黃色小旗,緩緩向山上行去。

無忌叫道︰「顏旗主在麼?」厚土旗掌旗使顏垣聽到叫聲,回頭一看,見是無忌,大喜之下,急忙上前行禮,説道︰「厚土旗顏垣,參見教主。」旗下教衆歡聲雷動。一齊拜伏在地。衆人這次由光明左使楊逍、光明右使范遙二人率領,盡集教中高手,向少林寺要人。明知必有一番周折,説不定要大動干戈,只是到處尋不著教主,不免有群龍無首之感,但事在緊急,不能等到端陽節正日,天下英雄群聚少林,那時再來討人。就得與舉世群雄爲敵了。衆人商議之下,均覺既是無法稟明教主,只得權宜爲計,於端陽節前十日齊上少林寺來。

張無忌慰勉幾句,早有教衆吹起號角,報知教主到來。過不多時,楊逍、范遙、殷天正、韋一笑、殷野王、周顚、彭瑩玉、説不得、鐵冠道人等人。先後從各處聚集,只是鋭金、巨木、洪水、烈火四旗教衆,分四面圍住了少林寺,不敢擅離所佔方位。衆人參見教主,無不大喜。楊逍與范遙謝過擅專之罪,無忌道︰「各位不須過謙,大家齊心合力來救謝法王,原是本教兄弟大夥児的義氣,本人心下感激,有何怪罪了。」當下將自己混入少林寺,昨晩已和渡厄等三僧動手的事簡略説了。衆人聽説一切都是成崑的奸謀,盡皆氣憤,周顚和鐵冠道人更是破口大罵起來。無忌道︰「今日本教以堂堂之師,向少林方丈要人,最好是别傷了和氣。萬不得已動手,咱們第一是救謝法王,第二是捉拿成崑,此外不可濫傷無華。」衆人齊聲應諾。無忌又向趙明道。「明妹,最好你喬裝一下,别讓少林僧衆認出身份,以免多生事端。」要知當日趙明擄了少林衆僧囚在大都,與少林派已結下極深的怨仇。趙明笑道︰「顏大哥,我扮作你旗下的一名小兄弟吧!」顏垣雖不明白她與無忌之間的瓜葛,但聽教主呼之爲妹,二人神情親密,當即遵命,叫一名旗下兄弟除下外袍,讓趙明披上,趙明奔入山後樹林,匆匆改扮,搽黑了面頰,從林中出來時,已變成一個面目猙獰的黑瘦漢子。當下號角吹動,群豪列隊上山。少林寺中早已接到明教拜山的拜帖,空智禪師率領僧聚,在山亭中迎候。

空智聽了圓眞之言,深信少林僧衆被趙明用計擒往大都囚禁,削斷手指,逼授武功,乃是明教與汝陽王暗中勾結,安排下的奸計,後來張無忌出手相救,更是假意賣好,另有陰謀,是以此刻一見明教大舉上山,臉上神色極是陰沉,合什行了一禮,什麼話也不説。無忌抱拳道︰「敝教有事向貴派奉懇,專誠上山拜見方丈神僧。」空智點了點頭,説道︰「請!」引著明教群豪走向山門。空聞方丈聽説張無忌親自到來,不願失了武林中的禮數。率領達摩堂、羅漢堂、藏經閣各處首座高僧,在山門外迎接,請群豪到大雄寶殿之上,分賓主坐下,小沙彌送上清茶。

空聞和張無忌、楊逍、殷天正等人寒暄了幾句,便即默然。無忌説道︰「方丈神僧,咱們是無事不登三寶殿,特地求懇方丈瞧在武林一脈,開釋敝教護教法王謝法王,大恩大德,日後必當補報。」空聞道︰「阿彌陀佛,出家人慈悲爲本,戒嗔戒殺,原是不該和謝施主爲難。不過老衲師兄空見,命喪謝施主之手,張教主是一教之主,也當明白武林中的規矩。」張無忌道︰「此中另有原故,可也怪不得謝法王。」於是將空見甘願受拳,以化解武林中一場大冤孼的經過説了。空聞等只聽得一半,便即口宣佛號,一齊恭恭敬敬的站起。空聞目中含泪,顫聲道︰「善哉善哉!空見師兄以大願力行此大善事,功德非小。」有幾名和尚口中低聲唸經,對空見之仁俠高義,無不敬佩。明教群豪也一齊站起,致欽仰之意。

無忌又道︰「謝法王失手傷了空見神僧,至感後悔,但事後細細回想,此事的罪魁禍首,實是貴寺的圓眞大師。」他見圓眞不在殿上,道︰「請圓眞大師出來,當面對質,分辨是非。」周顚插口道︰「是啊。在光明頂上這禿驢裝假死,却活了過來,鬼鬼祟祟,是什麼好東西了?快叫他滾了出來。」那日他在光明頂上吃了圓眞的大虧後,一直記恨。無忌忙道︰「周先住不可在方丈大師之前無禮。」周顚道︰「我是罵圓眞那禿驢,又不是罵方丈那禿\dash{}」這「禿」字一出口,知道不對,急忙伸手按住自己嘴巴。

空智想起空見、和空性的慘死,本已十分悲憤,見周顚出言無禮,更加了幾分惱怒,説道︰「然則空性師弟之死,張教主却又如何解釋?」無忌道︰「空性神僧血性過人,豪爽俠義,在下當日在光明頂上以武相會,極是欽佩。不幸身遭大難,在下甚是悼惜。此是奸人暗算,與敝教無涉。」空智冷笑道︰「張教主倒推得忒煞乾淨。然則汝陽王郡主與明教聯手之事,那也是假的了?」無忌臉上一紅,道︰「郡主與她父兄不洽,投身敝教。郡主往日對貴寺諸多不敬之處,在下自當令她上山拜佛,鄭重謝罪。」空智喝道︰「張教主花言巧語,於事何補?你身爲一教之主,信口胡言亂語,豈不令天下英雄恥笑?」張無忌想到殺空性,擒衆僧之事,確是趙明大大的不該,雖與明教無涉,但她目下却是託身於己,可不能推委不理。正爲難間,鐵冠道人已厲聲道︰「空智大師,我教主敬你是前輩高僧,給足了你面子,你可須知自重。我教主守信重義,豈能説一句假話?你辱我教主,便是辱我明教百萬之衆。縱我教主寬宏大量,不予計較,咱們做部屬的却不能善干罷休。」此時明教教衆在淮河、豫鄧一帶攻城掠地,招兵買馬,説是「百萬之衆」,確非流誇之言。

空智冷笑道︰「百萬之衆便怎地?莫非要將少林寺踏爲平地?魔教辱我少林,原非自今日始。咱們失手被擒,囚於萬法寺中,只怨自己學藝不精,自來邪正不兩立,那也没有什麼。嘿嘿,『先誅少林,再滅武當,唯我明教,武林稱王』好威風,好煞氣!」

\chapter{平手相鬥}

無忌等一聽此言,登時記起,「先誅少林,再滅武當,唯我明教,武林稱王」這十六個字,乃是當日趙明手下武士將少林僧衆擒去之後,以金剛大力指手法冩在達摩石像臉上。其時苦頭陀范遙身在汝陽王府,心向明教,一待衆人出寺,便即飛身回到達摩堂,將那石像移轉,仍作面壁之態,以免趙明嫁禍於明教的陰謀得逞。後來楊逍等發覺了這十六個字,但看過之後,仍將石像移正,没料想還是給少林僧衆知悉。無忌口才不佳,又想到這是趙明的傑作,内心有愧,不禁無言可答。

楊逍却道︰「空智大師所云,好教咱們大是不解。敝教張教主去世的尊大人,乃是武當弟子張五俠,此事江湖上盡人皆知。咱們就算再狂妄萬倍,也決不敢辱及教主的先人。再説在石上刻字的金剛大力指手法,乃是少林派的不傳神技,敝教教下兄弟身手平庸,無人能會此等高深功夫。空智大師於各家各派武功,無所不窺,當知在下所説是否花言巧語,胡説八道,天下英雄恥笑誰來?」這一席話振振有辭,立時令空智爲之語塞。

空聞方丈一來修爲日久,心性慈和,二來終究以大局爲重,心知明教勢大,若是雙方正面動手,只怕傳之千百年的少林古刹,不免要在自己手中毀去,便道︰「各位空言爭論,於事無益,請隨老柄前赴達摩堂,瞻仰初祖法像,誰是誰非,便知端的。」無忌道︰「如此甚好。」他見趙明混在顏垣手下的厚土旗教衆之中,並未隨入大殿,料想不致爲少林僧衆發覺,心下又放寬了幾分。

當下知客僧在前領路,一行人衆,行向達摩堂來。那達摩堂乃是少林寺中前輩高僧修眞養性之所,行輩較低的僧衆,輕易不敢窺堂門一步。達塵堂首座職份雖尊,對堂中諸僧,却也十分恭敬。到得堂前,只見板門緊閉,空智説道︰「方丈肅請明教衆位施主,前來達摩堂瞻仰初祖法像。」衆人在堂前站立片刻,不聞門内有何聲響,達摩堂的首座便伸手輕輕推開了板門。只見堂中有九位老僧,一齊閉目在蒲團上打坐。那打坐的姿式却是各人不同,或跪或蹲,或臥或曲,有的雙手高舉,有的獨脚上翹。無忌等一見,均知那是在修習上乘的佛家内功,這些古怪姿式,顯是從五百羅漢的法像中演化出來。九位高僧對方丈駕臨不聞不問,不言不動,猶似披塑木彫一般。

無忌尋思︰「那日我等上少林寺來,在達摩堂中但見到九個破爛蒲團。明妹擄去囚在萬法寺中的僧衆,也無這九位老僧在内。不知當日這九僧到了何處?」空聞、空智等對這九僧也是視若無睹,只是躬身向面朝牆壁的達摩石像下拜。空聞道︰「弟子驚動初祖法像,尚請原宥。」拜罷。吩咐六名弟子恭移法身。六名弟子依言上前,一齊雙手合什,默祝了幾句,然後三人一邉,分列兩旁,臂上使勁、將這二千餘斤的大石像轉了過來。

這石像只轉過一半,達摩堂上衆人不約而同的一聲驚呼,只見那石像口眼耳鼻,盡皆完好,竟是没半點破損。這一來不但空聞、空智等大吃一驚,張無忌等也是大出意料之外。衆人以前明明見到,這達摩石像的整張臉孔被人削平,冩上︰「先誅少林」等十六個字,何以此時却已變得完好無缺?空智上前伸手一摸,見那石像的面目乃是從整塊巨石雕成,絶非另行雕刻一張臉孔鑲嵌而上。霎時之間,人人面面相覷,説不出話來。這個二千餘斤的巨大石像在外面雕好之後,悄悄運進寺來,將原來的大石像換將出去,這一進一出,那是多大的工程,少林寺近數月來守衛何等嚴密,别説這等兩件龐然大物,便是一盆一缽之微,也是不能隨便擕進擕出。

楊逍見群僧驚愕萬狀,抓住良機便道︰「貴寺福澤深厚,功德無量,達摩老祖顯聖,補好了被奸人損毀的法像,實乃可喜可賀。」説著便向達摩石像跪拜下去。張無忌等跟著一齊拜倒。空聞、空智等群僧只得還禮。空聞等雖不信老祖顯聖云云的鬼話,但想多半是明教暗中做了手脚,不論如何,總是向本派補過、道歉,各人心中存著的氣惱,不由得均是消解了三分。

空聞道︰「石像既已完好如初,此事不必再提。」揮手命六名弟子推著石像轉身面壁,又道︰「昨晩張教主降臨,已與老納三位師叔朝過相,渡厄師叔和張教主訂下約會,只須張教主破得我三位師叔的『金剛伏魔圏』,任憑將謝施主帶走,此事可是有的?」張無忌道︰「不錯,渡厄大師確有此言。但在下深佩三位高僧武功高深,自知不是敵手,昨晩已折在三位高僧手下,敗軍之將,何敢言勇?」空聞道︰「阿彌陀佛,張教主言重了。昨晩勝負未分,三位師叔頗感教主高義。」楊逍、范遙等聽無忌説過渡厄等三僧武功精妙,凡是學武之人,均盼一觀爲快。殷天正道︰「既是少林衆位高僧執意武學上一見高低,教主,咱們不自量力,只好領教少林派的絶學。好在咱們是爲相救謝兄弟而來,實逼處此,無可奈何,並非膽敢到領袖武林的少林寺來撒野。」張無忌對外公之言向來極是尊重,又想除此之外,也是别無善法,便道︰「兄弟們聽到在下耀揚三位老僧神功蓋世,都説三位高僧坐関數十年,武林中誰也不知,今日大夥児有幸拜見,實是生平之幸。」空智舉手道︰「請!」領群豪走向寺後山峰。

明教洪水旗下教衆,在韋旗使唐洋率領之下,散在山峰脚邉,空聞等視若無睹,逕行上峰。空聞、空智合什走向松樹之旁,躬身稟報。渡厄道︰「楊破天的仇怨化解了,初祖法像的事也揭過了,好得很,好得很。張教主,你們幾位上來動手?」楊逍見三僧身形矮小瘦削,嵌在松樹幹中,便像是三具僵屍人乾,但幾句話却是説得山谷鳴響,顯是内力深厚之極,不由得聳然動容。

無忌尋思︰「昨晩我一人是鬥他三人不過,咱們今日人多,倘若一湧而上,一來施展不開。二來倚多爲勝,也是折了本教的威風。多了不好,少了不成,咱們三個對他三個,最是公平。」便道︰「昨晩在下見識到三位的神功,大開眼界,原是不敢再在三位面前出醜。但謝法王與在下有父子之恩,與衆位兄弟有朋友之義,咱們縱然不自量力,那也是非救他不可。在下想請兩位教中兄弟相助,以三敵三,平手領教。」渡厄淡淡的道︰「張教主不必過謙。貴教倘再有一位武功和教主不相伯仲的。那麼只須兩位聯手,便能殺了咱三個老禿。但若老納所料不錯,如教主這等身手之人,舉世再無第二位,那麼還是人多一些,一齊上來的好。」周顚、鐵冠道人等你瞧瞧我,我瞧瞧你,都想這老禿驢好生狂妄,竟將天下英雄視若無物,只是語氣之中,總算自承不及張教主,説舉世無人能與教主平手,倒還算客氣。張無忌道︰「敝教雖是旁門左道,不足與貴派名門抗衡,但數百年的基業,也有一些人才。在下因緣時會,暫代教主之職,其實論到才識武功,敝教中勝於在下者,何止車載斗量,韋蝠王,請你將這份名帖,呈上三位高僧。」説著取出一張名帖,上面自張無忌、楊逍、范遙、殷天正、韋一笑以下,書就此次拜山群豪的姓名。

韋一笑知道教主要自己顯示一下當世無雙的輕功,好教少林寺借不敢小覷了明教中的人物,當下躬身應諾,接過名帖,身子並未站直,竟不轉身,便即反彈而出,猶如一溜輕煙,相隨十餘丈間,便飄到了三株松樹之間,雙掌一翻,將名帖送交渡厄。

渡厄等三僧見他一晃之間,便即到了自己跟前,輕功之佳,實是生平罕見,何況他是倒退反彈,那更是匪夷所思,不由得讚道︰「好輕功!」少林群僧個個是識貨的,登時采聲雷動。明教群豪雖知韋一笑輕功了得,但這般倒退反彈的身手,却也是初次見到,只是各人不便稱讚自家人,儘管心下佩服,却是默不作聲。

渡厄微微欠身,伸手接過名帖。他右手五根手指一搭到名帖,韋一笑全身一麻,宛似受到電震,胸口發熱,身子幾欲軟倒。他大驚之下,急忙運功支撐,渡厄已將名帖取了過去,從名帖上傳來的這一股内勁也即消失。韋一笑臉色一變,暗想這眇目老僧的内勁當眞是深不可測,不取多所逗留,斜身一讓,從一片長草上滑了過來,回到張無忌身旁。這一手「草上飛」的輕功,雖非特異,但練到這般猶如凌虛飄行,那也是神乎其技的了。空聞、空智等均想︰「此人輕功造詣到了如此地步,固是得了高人傳授,但也出於天賦,看來他是生就異稟,旁人縱是苦練,也決計到不了這等境界。」

渡厄説道︰「張教主既是決意三人下場,除了教主與這位韋蝠王外,還有那一位前來指教?」張無忌道︰「韋蝠王已領教過大師的内勁神功,在下想請明教左右光明使者相助。」渡厄心中一動︰「這少年好鋭利的眼光,適纔我隔帖傅勁,只是一瞬間之事,居然被他看了出來。什麼左右光明使者,難道比這姓韋的武功更高麼?」他坐関年久,於楊逍的名頭竟是没聽見過,至於范遙,則長年來隱姓埋名,旁人原也不知。楊范二人聽得無忌提及自己名字,當即踏前一步,躬身道︰「謹遵教主號令。」無忌道︰「三位高僧使的是軟兵刃,咱們用什麼兵刃好?」須知張、楊、范三人平時臨敵均是空手,今日面對勁敵,不能托大不用兵刃,三人一法通,萬法通,什麼兵刃都能使用,無忌此言,乃是就著二人方便。楊逍道︰「聽憑教主吩咐便是。」

無忌微一沉吟,心想︰「昨晩河間雙煞以短攻長,倒也頗佔便宜。」便從懷中取出,六枚聖火令來,將四枚分給了楊范二人,説道︰「咱們上少林拜山,不敢擕帶兇器,這是本教鎭山之寶,大家對付著使吧。」楊范二人躬身接過,正要請示方略。空智突然大聲道︰「苦頭陀,咱們在萬法寺中結下的樑子,豈能就此揭過?來來來,待老衲先領教你的高招。老衲今日没服十香軟筋散,各人手下見眞章吧。」要知那日空智被囚在萬法寺中,一肚皮的怨氣未曾發洩,今日見到范遙,一直盡力抑制心下怒火,此刻却是再也忍耐不住了。范遙淡淡一笑,道︰「在下奉教主號令,攻打『金剛伏魔圏』大師要報昔日之仇,待此事過後,再行奉陪。」空智從身旁弟子手中接過長劍,喝道︰「你不自量力和我三位師叔動手,不死也必重傷。我這仇是報不了的啦。」范遙笑道︰「我死在令師叔手下,也是一樣。」空智冷笑道︰「明教中既除閣下之外,更無别位高手,那也罷了。」

他這句話原是激將之計,明教群豪豈有不知?但覺若是嚥了這口氣下去,倒教少林派將本教瞧得小了。以位望而論,范遙之下便是白眉鷹王殷天正。無忌覺外公年邁,不便請他出手,正想請舅父殷野王出馬,殷天正踏上一步,道︰「教主,屬下殷天正討令。」無忌道︰「外公年邁,便請舅舅\dash{}」殷天正道︰「我年紀再大,也大不過這三位高僧。少林派有碩德耆宿,我明教便無老將麼?」無忌知外公武功深湛,決不在楊逍、范遙之下,比舅舅高出甚多,若是由他出戰,當多幾分把握,説道︰「好,范右使留些力氣,待會向空智神僧領教,便請外公相助孩兄。」

殷天正道︰「遵命!」從范遙手中接過了聖火雙令。空聞方丈朗聲道︰「三位師叔,這位殷老英雄,人稱白眉鷹王,當年自創白眉教,獨力與六大門派相抗衡,眞是了不起的英雄好漢。這位楊先生,内功外功倶臻化境,是明教中的第一流人物,崑崙、峨嵋兩派的高手,曾有不少敗在他的手下。」渡劫乾笑數聲,説道︰「幸會,幸會!且看少林門下弟子,身手如何?」三僧黑索一抖,猶似三條墨龍一般,圍成了三層圏子。

張無忌昨晩與三僧動手時伸手不見五指,全憑黑索上所發出的勁氣,以辨認敵方兵刃來路,此時方當午初,艷陽照空,連三僧臉上每一條皺紋都瞧得清清楚楚。他倒轉聖火令,抱拳一躬身,説道︰「得罪了!」側身便攻了上去。楊逍飛身向左,殷天正大喝一聲,舉起右手聖火令,便往渡難的黑索上擊落。「{\upstsl{噹}}嗚」一響,索令相擊。這兩件奇形兵刃相互碰撞,發出的聲音也是十分的古怪刺耳。兩人手臂都是一震,心道︰「好厲害!」均知是遇到了生平罕逢的勁敵。

無忌心下尋思︰「這『金剛伏魔圏』招數嚴密,我等雖是三人聯手,也決非三五百招之内所能攻破,且耗費他三僧的内勁,徐尋破綻。」一見黑索纏到,便使聖火令以之硬碰硬的對攻,他體内九陽神功愈運愈強,綿綿不絶,永無止歇。旁觀衆人但覺六人的兵刃上捲起層層旋風,寒氣逼人而來,不由得一步步的退開。鬥到一頓飯時分,無忌等三人已將索圏壓得縮小了丈許圓徑。然而三僧的索圏壓小,抗力越強,三人每攻前一步,便此先前要多花幾倍力氣。楊逍與殷天正越鬥越是駭異,起初尚是以三敵三的局面,到得半個時辰之後,楊殷二人漸漸支持不住,成爲二人合鬥渡難。無忌却是一人對付渡厄、渡劫二僧。

殷天正走的至是剛猛路子,楊逍却是忽柔忽剛,變化無方。這六人之中,以揚逍的武功最爲好看,那兩柄聖火令在他手中盤旋飛舞,忽而成劍,忽而成刀,忽而作短槍刺、打、纏、拍,忽而作判官筆點、戳、捺、挑,更有時左手匕首,右手水刺,忽地又變成右手鋼鞭,左手鐵尺,百忙中尚自雙令互擊,發出啞啞之聲以擾亂敵人心神。相鬥未及四百招,已連變了二十二種兵刃,每種兵刃均是兩套招式,一共四十四套招式。空智於少林派七十二絶藝得其十八,范遙自負於天下武學無所不窺,但此刻見楊逍神技一至於斯,都不由得暗自歎服。周顚與楊逍素有心病,曾數次和他爭鬥,此刻越著越是慚愧︰「楊逍這龜児子原來一直讓著我。先前我只道他武功只此我稍高,每次動手,碰巧運氣好,這纔勝我一招半式。豈知我周顚跟他龜児子差著這麼老大一截。」

但不論楊逍如何變招,渡難一條黑索分敵二人,仍甚綽綽有餘。衆人只見殷天正頭上白霧升起,知他内力已發揮到了極致,一件白布長袍慢慢鼓起,衣内充滿了氣流。他每踏一步,脚底便是一個足印,鬥到將近一個時辰,圍著三株松樹之外,已被他踏出了一圏足印。陡然之間,殷天正將右手聖火令交於左手,將渡難的黑索一壓,右手一招劈空掌便向渡難擊了過去。渡難左手一起,五指虛抓,握成空拳,也是一掌劈出。空聞空智等一齊「憶」了一聲,聲音中充滿了驚訝佩服之情。原來渡難還他這一掌,乃是少林七十二絶藝中之中的「小須彌掌」。這種掌力極難練成,那是不必説了,縱然練成之後,每次出掌,也須坐馬運氣,凝神良久,始能將内勁聚於丹田,那知渡難要出掌便出掌,一動念間,就將這「小須彌掌」拍了出來,跟著黑索一抖,又向楊逍撲擊而至。

但渡難以「小須彌掌」與殷天正對掌,黑索上的勁力便弱了一大半。他正以巧補弱,只見那黑索滾動飛舞,宛若靈蛇亂顫,楊逍的兩根聖火令也是變化無窮。旁觀衆人的目光,大半集中去瞧他二人相鬥。殷天正凝神提氣,一掌掌的拍出,忽而跨前兩步,忽而又倒退兩步。那邉張無忌以一敵二,三人的招式都是平淡無奇,所有的拚鬥,都是在内勁上施展。這種拚鬥比之殷天正的鬥力和楊逍的鬥巧,其實更是兇險十倍,只要内勁被對方一逼上岔路,不是立時氣絶身亡,便是走火入魔,那時發瘋癱瘓,均是常事,只是這種險到極處的比拚,只有身歷其境的局中人方知其中的甘苦,旁觀者武功再高,也無法從他三人的招式中辨認出來。

眼見六人相鬥,已過了一個多時辰,太陽由偏東而當頭直射,更漸漸偏西。空聞空智、范遙、韋一笑等第一流的高手,這時已看出了雙方勝負之機。但見殷天正頭頂的白氣越來越濃,而渡劫坐在其中的那棵大松樹,枝幹上的針葉竟不住的搖晃顫動,可知渡厄和渡劫二僧功力究有高下,鬥到此時,渡劫背靠松樹,須得借大樹之力,方能與無忌的九陽神功相抗。倘若殷天王先行支持不住,那便是明教輸了,若是渡劫先一步難以抵擋,則是少林派落敗。

出手相鬥的六人更加明白這中間的関鍵所在。殷天正與渡難此拚掌力,拚到五十餘掌之後,知道自己終非他的敵手,心想︰「咱們今日之事,以救謝兄弟爲重。我個人的勝負榮辱、何足道哉?何況輸在少林派前輩高人手下,也不能説是損了我白眉鷹王的威名。」臂下拚得一掌,便向後退出一步,再拚得十餘掌,已是返到數丈之外。那知「小須彌掌」乃少林派七十二絶藝之中,渡難在這掌法上浸淫數十載,威力實是非同小可,殷天正退一步,這小須彌擘的掌力跟著進擊一步,勁力竟是絲毫不以路程拉遠而稍衰。

楊逍尋思︰「這位少林高僧果眞了得,我聖火令上招數再變,終究也是奈何不了他。殷白眉獨受内勁,時候長了只怕支持不住。」兩根聖火令一合,想要挾住黑索,跟他也來個硬碰硬的鬥力,以分殷天正的重擔。不料聖火令剛要挾到黑索,渡難手腕一抖,那黑索的索頭直昂上來,撞向揚逍面門,楊逍心念如電,聖火令脱手,向渡難胸口急擲過去,雙掌一翻,已抓住索頭,一招「倒曳九牛尾」,猛力向外急拉。

渡難見他兵刃出手,當作暗器般打來,勁道極猛,左手上肘一沉,便往下向左胸的一枚聖火令壓去,同時身子略側,讓開飛向左胸的那枚聖火令。没料到左肘壓下了一枚聖火令,另一枚突然間中道轉向,呼的一聲,斜刺射向渡劫。原來這六人之中,以楊逍最工機心,他擅於斜擲暗器,兩枚聖火令中,攻渡難的是虛,攻渡劫的那枚之上,方用上了全身内勁。

渡劫正與張無忌全力相抗,眼見渡難對付楊殷二人,已是穩佔上風,那想得到楊逍竟會忽發奇想以此怪異的手法偸襲,一驚之下,聖火令已到面門。渡劫心神微亂,輕輕伸起兩指,將那枚聖火令挾了下來。但其時他與張無忌全神貫注的比拚内勁,那容得這麼心神一分,霎時之間,他存身其内的大松樹搖晃不止,樹上松針紛紛下墜,便如空中下了一陣急雨。張無忌一覺對方破綻大露,這乾坤大挪移心法最擅於尋瑕抵隙,對方百計防護,尚且不穩,何況自呈敗弱?他手指上五股勁氣,登時絲絲作響,疾攻過去。片刻間拍拍有聲,渡劫那棵松樹上一根根小枝也震得落了下來。

渡厄眼見勢危,霍地站起,身形一晃,已到了渡劫身旁,伸出左手,搭在他的肩頭。渡劫得師兄渡厄相助,重行穩住。那邉廂渡難與殷天正,楊逍也已到了各以眞力相拚,生死決於俄頃的地步。楊逍拉著黑索一端,向外扯奪,股天正却以破山碎碑的雄渾掌力,不絶向渡難抵壓過來。兩大高手一拉一推,兩股勁力恰恰相反,渡難身處其間,雖也吃力萬分,却是絲毫不現敗象。

旁觀的明教群豪和少林僧衆眼見這等情景,知道這場拚鬥下來,不僅分出勝敗而已,六大高手之中,只怕有半數要命喪當場。偌大一座山峰之上,刹時間竟無半點聲息,群雄泰半汗濕衣背,没一個不是提心吊膽,爲自己一方的人擔憂。

便在這萬籟倶寂之際,忽聽得三株松樹之間的地底下,一個低沉的聲音説起話來︰「楊左使、殷大哥、無忌孩児,我謝遜雙手染滿血跡,早已死有餘辜,今日你們爲救我而來,與少林寺三位高僧爭鬥,若是雙方再有損傷,謝遜更是百死莫贖。無忌孩児,你快快率同本教兄弟,退出少林寺去。否則我立時自絶經脈,以免多增罪孼。」這聲音雖低,但遠遠傳送而出,峰頂衆人,無不聽得清清楚楚,正是謝遜以「獅子吼」神功在地牢中説話。當年他在王盤山上,用獅子吼震死各幫各派無數豪士,此刻雖非以此神功傷人,但衆人耳鼓仍是震得{\upstsl{嗡}}{\upstsl{嗡}}作響,相顧失色。

無忌知道義父言出如山,決不肯爲了一己脱困,致令旁人再有損傷,眼前情勢,倘若力拚到底,自己雖是無恙,但外公、楊逍、渡劫、渡難四人,必定不免,正躊躇間,只聽謝遜大聲喝道︰「無忌,你還不去麼?」無忌道︰「是!謹遵義父吩咐。」他退後一步,朗聲説道︰「三位高僧的『金剛伏魔圏』果然神妙,今日明教無法攻破,他日再行領教。外公、楊左使,咱們收手吧!」説著勁氣一收,將渡厄、渡劫二僧黑索上所發出的内勁一彈而回。楊逍與殷天正聽到他的號令,苦於正與渡難全力相拚,無法收手。若是收回内勁,立時便被渡難的勁氣所傷。渡難此刻也是欲罷不能。張無忌走到殷天正之前,雙掌一揮,接過了渡難與殷天正分以左右襲來的掌力,跟著伸出聖火令,搭在渡難的黑索中端。那黑索正被楊逍與渡難拉得如繃緊了的弓弦一般,無忌的聖火令一搭上去,乾坤大挪移的神功登時將兩端傳來的猛勁化解了。黑索軟軟垂下,落在地下。楊逍手快,一把搶著。

渡難臉色一變,正欲發話。楊逍雙手擁著黑索,走近幾步,説道︰「奉還大師兵刃。」渡劫已知他的心意,將身旁的兩枚聖火令抬了起來,交還給他。自經適纔這一戰,三位少林高僧已收起先前的狂傲之心,知道若是拚將下去,勢必是個兩敗倶傷的局面,己方三人實是無法佔得對方的上風。渡厄説道︰「老納閉関數十年,重得見識當世賢豪,至感欣幸。張教主,貴教英才濟濟,閣下更是出類拔萃,唯望以此大好身手多爲蒼生造福。少作傷天害理之事。」張無忌躬身道︰「多謝大師指教。」渡厄道︰「我師兄弟三人,在此恭候張教主大駕三度蒞臨。」無忌道︰「恭候是不敢,然自當再來領教。謝法王是在下義父,恩同親生。」渡厄長嘆一聲,閉目不語。

無忌率同楊逍諸人,拱手與空聞、空智等人作别,走下山中。彭瑩玉傳出訊號,撤回五行旗人衆。離寺十里,厚土旗教衆倚山搭了十餘座竹棚,以供衆人住宿。無忌悶悶不樂,心想本教之中,無人的武功能比楊逍與外公更高,就算換上范遙與韋一笑,那也不過是和今日的局面相若,天下那裡去找一兩位勝於他們的高手,來破這金剛伏魔圏?彭瑩玉猜中他的心事,道︰「教主?你怎地忘了張眞人?」。

無忌躊躇道︰「倘若我太師父肯下山相助,和我二人聯手,破了這『金剛伏魔圏』定可辦到。但一來此舉大傷少林、武當兩派的和氣,太師父未必肯允。二來太師父一百多歳的年紀,武學修爲雖已到了爐火純青的境界,究竟年紀衰邁,若有失閃,如何是好?只怕宋大師伯他們也決計不肯\dash{}」突然之間,殷天正站起身來。哈哈笑道︰「張眞人如肯下山,定然馬到成功,妙極,妙極!」乾笑幾聲。張大了口。聲音忽然啞了。

群豪見他笑容滿臉,直挺挺的站著,都覺奇怪。楊逍道︰「殷兄,你想張眞人能下山出手麼?」他連問兩次,殷天正只是不答,身子也一動不動。無忌吃了一驚,伸手一搭他的脈搏,不料心脈早停,竟已氣絶身亡。原來他適纔苦戰渡難,耗竭了全部力氣,加之年事已高,竟然油盡燈枯,張無忌心中一痛,抱著他的屍身,哭了出來。殷野王搶了上來,更是呼天搶地的大哭。群豪念及同教的義氣,無不愴然泪下。訊息傳出,明教中有許多教衆原屬白眉教旗下,登時哭聲震動山谷。

這數日間,群豪忙於料理殷天正的喪事,眼見各門派、各幫會的武林人物絡繹上山。這些人仰慕段天正的威名,都到竹棚中他靈前弔祭。空聞、空智等已親自前來祭過,並派了十八名僧人,做法事爲殷天正超度。但十八名僧人只唸了幾句經,便給殷野王手執哭喪棒轟了出去,周顚更在一旁大罵︰「少林禿驢,假仁假義。」這數日中,張無忌憂心如搗,和楊逍、彭瑩玉、趙明等商議數次,均甚不得善法。趙明會想設法將「十香軟筋散」的毒藥,下在渡厄三僧的飲食之中,又説要去召鹿杖客、鶴筆翁二人來和無忌聯手,但無忌和楊逍等均覺不妥。

彈指間端陽正日已到,張無忌率領明教群豪,來到少林寺中。少林寺前殿後殿、左廂右廂,到處都擠滿了各路的英雄好漢。聚人均知此次英雄大會,乃是爲謝遜而開。各路武林人物之中,有的是謝遜的仇人,圖在會中報仇雪恨,有的覬覦屠龍刀,妄想奪得寶刀,成爲武林至尊,有的是相互間有私人恩怨,要乘機作一了斷,極大多數却爲瞧熱鬧而來,少林寺中派出百餘位知賓接待,分别獻茶,引著在寺中各處休息。

\chapter{技震群雄}

衆賓客坐定後,少林群僧一批批的出來,按著圓、慧、法、相、莊各字輩,與天下群雄見禮,最後是空智神僧,身後跟著達摩堂中那九名老僧,來到廣場正中,合什行禮,口宣佛號,説道︰「今日得蒙天下英雄賞臉,降臨敝寺,少林上下,盡感光寵。只是方丈師兄突患急病,無法起床與各位相見,命老衲鄭重致歉。」無忌心下微覺奇怪︰「那日空聞大師到外公廳前弔祭,臉上絶無病容,精神矍鑠,他這等内功深厚之人,怎能突然害病?難道是受了什麼傷?」

只聽空智又道︰「金毛獅王謝遜爲禍武林,罪孼深重,此次幸爲本寺所擒。少林派不敢自專,恭請各位名重武林之士,齊來敝寺,共商處置之策。」空智本來生得愁眉苦臉,這時説話更是没精打采,似乎頗爲擔心空聞的疾苦。這英雄大會自從當年在荊紫関舉行之後,近百年來並未再開,可説是江湖上第一等的盛事,但主持者臨時生病,群雄不由得均感掃興。無忌四下裡打量,不見圓眞和陳友諒露面,心想︰「那日晩上我向渡厄等三位高僧揭破圓眞的奸謀,不知寺中是否已予處置?空聞大師忽地托病,不知是否與此事有関?」

空智説完,便即合什退下,忽見東南角上站起一人,身形魁梧,一部黑白相間的鬍鬚隨風飛舞,貌相甚是威嚴,手掌心{\upstsl{噹}}{\upstsl{啷}}{\upstsl{啷}}地玩弄著三枚大鐵膽,却是川東老拳師夏胄。只聽他聲若洪鐘,説道︰「這謝遜作惡多端,既教貴派擒來,那是造福武林,實非淺鮮。空聞、空智兩位神僧太過謙抑,這等惡人,當時一刀殺却,也就是了,何必再問旁?今日既是天下英雄聚會,咱們此會便叫作屠獅大會。將這謝遜凌遲處死,每人吃他一口肉、飲他一口血,替無辜死在他手下的朋友們報仇,豈不痛快?」原來這夏胄有個親兄弟便爲謝遜所殺,數十年來只想找謝遜報仇。他此言一出,四周便有數十人隨聲附和,都説早殺了的爲是。

混亂之中,忽聽得一個陰惻惻的聲音説道︰「謝遜是明教的護教法王,少林派倘若不怕得罪明教,早就一刀將他殺了,何必邀大夥児來此分擔罪責?我説夏老拳師,你有點老胡塗啦,老兄弟勸你一句,還是明哲保身的爲是。」這番話説得陰陽怪氣,但傳在衆人耳中,仍是清清楚楚,衆人往聲音來處瞧去。却看不見是誰,原來那人身材矮小,説話時又不站起,坐在人叢之中,誰也看不見他的相貌。

夏胄大聲道︰「是『酔不死』的司徒兄弟麼?那謝遜與我有殺弟之仇,大丈夫一人做事一人當,請少林衆高僧將他牽將出來,老夫一刀將他殺了。魔教群魔找上身來,儘管衝著我川東姓夏的便是。」那「酔不死」司徒千鍾又是陰惻惻的一笑,説道︰「夏大哥,江湖上人人皆知,那把武至尊的屠龍刀,乃是落在謝遜手中。少林派既得謝遜,豈有不得寶刀之理?人家殺謝遜是虛,揚刀立威纔是大事。我説空智大師哪,你也不用假惺惺來啦,痛痛快快將那屠龍刀取將出來,讓大夥児開開眼界是正經。你少林派千百年來就是武林中的頭児腦児,有此刀不爲多,無此刀不爲少,總之是武林至尊就是。」原來司徒千鍾此人一生玩世不恭,不拜師,不收徒,一個人閒雲野鶴,不屬任何門派幫會,生平極少與人動手,誰也不知他的武功底細,説起來冷嘲熱諷,却往往一語中的。

當下群雄中便有七八人跟著説道︰「此言有理。請少林派取出屠龍刀來,讓大夥児開開眼界。」空智緩緩説道︰「屠龍刀不在敝寺,老衲一生之中也從未見過,不知世上是否眞有此刀。」群雄一聽,立時紛紛議論起來,廣場上一片嘈雜,與會諸人原先都認定此會必與屠龍刀有莫大関連,豈知空智竟是一口否認,誰都大出意料之外。

空智身後跟著九位老僧,均是身披大紅袈裟,待群雄嘈雜之聲稍息,九僧中一名老僧踏上兩步,朗聲説道︰「屠龍刀在謝遜手中,此事天下皆知。可是本派雖擒獲了謝遜,屠龍刀却不在他的身邉。本寺方丈以此事有関武林氣運,曾詳加盤査,那謝遜桀驁不馴,抵死不言。今日英雄盛會,一來是商酌處置謝遜之方,二來是向衆位英雄打聽那屠龍刀的下落。衆位英雄中有得知音訊者,便請明言。」群雄面面相覷,誰都接不上口。那「酔不死」司徒千鍾却又陰陽怪氣的説道︰「武林中百年來傳言道︰『武林至尊,寶刀屠龍,號令天下,莫敢不從。倚天不出,誰與爭鋒?』除了屠龍刀,尚有倚天劍,這柄倚天寶劍哪,本來聽説是在峨嵋派手中,可是西域光明頂一戰,却也從此不知所終。今日此會雖叫做英雄大會,峨嵋派的英雌們難道就不能來麼?」衆人聽到最後這句話,却不禁鬨然大笑起來。

轟笑聲中,一名知客僧大聲報道︰「丐幫史幫主,率領丐幫諸長老、諸弟子到。」無忌聽到「史幫主」三字,心下大奇︰「丐幫史火龍幫主早已死在圓眞手下,如何又出來一位史幫主?」空智説道︰「有請!」丐幫是江湖上第一大幫會,空智不肯慢客,親自迎了出去。只見寺旁小路上來了一列一百五十餘人,都是衣衫襤褸的漢子。丐幫近年來聲勢雖已不如往時,究竟百足之蟲,死而不僵,在江湖上仍有極大潛力,群雄誰也不敢輕視,一大半都站了起來。但見當先是兩位老年丐者,張無忌認得是傳功長老和執法長老。兩位老丐身後,却是一個十二三歳的醜陋女童,鼻孔朝天,闊口中露出兩枚大大的門牙,正是史火龍之女史紅石,她手中持著一根綠色竹棒,乃是丐幫幫主的信物打狗棒。史紅石之後則是掌棒龍頭、掌砵龍頭,其後依次是八袋長老、七袋弟子、六袋弟子。丐幫這次到英雄大會的,最低的也是六袋弟子。

空智見持打狗棒的乃是一個女童,心下躊躇,不知幫主是誰,該當向誰説話纔是,只得合什行禮,含糊道︰「少林僧衆恭迎丐群雄大駕。」群丐一齊抱拳行禮,傳功長老説道︰「敝幫前幫主不幸歸天,衆長老公決,立史幫主之女史紅石姑娘爲幫主,這一位便是敝幫新幫主。」説著向史紅石一指,空智和群雄是一呆,心想江湖上向來有言道︰「明教、丐幫、少林派」,各教門以明教居首,各幫會推丐幫爲尊,各門派則以少林派爲第一。明教立了個二十餘歳的少年張無忌當教主,已經令人嘖嘖稱奇,不料丐幫更推這樣一個小女孩作幫主,若非從丐幫長老口中説出,那是誰也不肯相信的。空智不願缺了禮數,合什道︰「少林門下空智,參見史幫主。」史紅石福了福還禮,囁囁嚅嚅的對答不出。傳功長老道︰「敝幫幫主年幼,一切幫務,暫由兄弟及執法長老二人處決。空智神僧乃前輩大德,多禮甚不敢當。」兩人謙虛了幾句,群丐自入竹棚中歸座。

丐幫人數衆多,半晌方始坐定。無忌見一百五十餘名丐幫弟子,人人身上戴孝,臉上均有悲憤之色,有些弟子背上的布袋之中,更有物蠕蠕而動,顯是有所爲而來,心下暗喜,剛跟楊逍説得一句︰「咱們到了一批好幫手。」只見傳功、執法二長老,掌棒、掌砵二龍頭,引著史紅石來到明教棚前。傳功長老抱拳行禮,説道︰「張教主,金毛獅王失陥,敝幫有好大的干係,咱們今日寧可性命不在,也要保護謝獅王周全,一來報教主前日的恩德,以贖咱們的罪愆;二來也是替史故幫主報仇雪恨。丐幫上下,齊聽教主號令。」張無忌急忙還禮,説道︰「不敢。」傳功長老這一番話説得甚是響亮,故意要讓廣場上人人聽見。

群雄聽在耳裡,都是一楞︰「丐幫幾時與明教結成了死黨啦?」除了極少在江湖上走的隱居俠士之外,衆人大抵均知年前丐幫參與圍攻光明頂之事,雙方一場血戰,死傷均衆,最後攻上光明頂的丐幫幫衆幾乎全軍覆没。此刻傳功長老公然聲言全幫齊聽張無忌號令,又説爲史幫主報仇雪恨云云,誰都摸不著頭腦。傳功長老幾句話説畢,丐幫衆弟子一齊站起,大聲説道︰「謹奉張教主號令,赴湯𨂻火,在所不辭。」

傳功長老回過身來,大聲説道︰「我丐幫與少林派向來無怨無仇,敝幫一直尊少林派是武林第一大門派,縱有些微嫌隙,咱們也必儘量克制忍讓,從來不敢有所得罪。敝幫自史火龍史幫主以下好生佩服少林四大神僧德高望重,足爲學武之士的表率模楷。史幫主歸隱已久,靜居養病,數十年來不與江湖人士往還,不知何故,竟遭少林高僧的毒手\dash{}」他説到這裡,廣場上一齊「啊」的一聲驚呼,連空智也是大出意料之外。

只聽傳功長老接著説道︰「咱們今日到此,不敢自居英雄,來赴這英雄大會,只是要請空聞方丈指點迷津,咱們史幫主什麼地方得罪少林派,以致少林高僧要趕盡殺絶,連史夫人也保不了性命?」

空智合什説道︰「阿彌陀佛,史幫主不幸仙逝,老衲今日尚是第一次聽到訊息。長老口口聲聲説是敝派弟子所爲,只怕其中有什麼誤會,還請長老言明當時詳情。」

傳功長老道︰「空聞、空智兩位大師佛法精深,咱們豈能誣賴?便請大師請貴寺一位高僧、一位俗家子弟出來對質。」空智道︰「長老吩咐,自當遵命,不知長老要命那二人出來?」傳功長老道︰「是\dash{}」他只説個「是」字,突然間張口結舌,説不下去了。

空智吃了一驚,身形不晃,欺近他的身去,抓住他的右腕,但覺肌膚尚熱,脈息已停。空智更驚,叫道︰「長老,長老!」看他顏面時,只見眉心正中有一顆香頭大般的細黑點,竟是要害處中了絶毒的暗器。空智大聲説道︰「各位英雄明鑒,這位丐幫長老中了絶毒暗器,不幸身亡,我少林派可決計不使這等陰狠的暗器。」

丐幫幫衆一聽此言,登時大嘩,數十人搶到傳功長老的屍身之旁。掌砵龍龍從懷中取出一塊吸鐵石,放在傳功長老眉心,吸出一枚細如牛毛,長才寸許的銀針來。

丐幫諸長老見多識廣,情知空智之言不虛,這等陰毒暗器,第一名門正派的少林派是決計不使的,然而在光天化日,衆目睽睽之下,竟然有人放暗器偸襲,無一人能予察覺,此人武功之高,實是不可思議。

執法長老等均想,傳功長老向南而立,這暗器必是從南方射來,其時南方陽光耀眼,傳功長又是心情十分憤激,以至未及提防這等極細微的暗器。

衆長老怒目向空智身後瞧去,只見九名身披大紅袈裟的達摩堂老僧眼睛半閃,垂眉而立,在這九僧之後,一排排黃衣僧人、灰衣僧人,實是無法分辨到底是誰下的手脚,然而兇手必是少林僧,那是絶無可疑的了。執法長老朗聲長笑,雙目中泪珠却是滾滾而下,説道︰「空智大師還説咱們冤枉了少林派,眼下之事,却有何話説?」掌砵龍頭最是性急,手中鐵棒一揚,喝道︰「我們今日跟少林派拚了。」但聽得廣場上嗆{\upstsl{啷}}{\upstsl{啷}}兵刃亂響,丐幫幫衆紛紛取出兵刃,一百五十餘人一齊躍到了廣場正中。

空智臉色慘然,回頭向著少林群僧,緩緩説道︰「本寺自達摩老祖西來,建下基業,千百年來歷世僧侶勤修佛法,精持戒律,雖因學武防身,致與江湖英豪來往,然而從來不敢作什傷天害理之事。方丈師兄和我早已勘破世情,豈再戀此紅塵\dash{}」

空智説到這裡,一反手,從一名少林僧手中搶過一條鑌鐵禪杖,伸手一擲,一條長達丈許的鐵禪杖没入地下泥中,霎時間無影無蹤。熟悉武林掌故的英豪均知,少林僧以禪杖插地,那是示意眼前之事須得以死相拚,決心大開殺戒,只是像他這般隨手一揮,便將一條長大禪杖没入泥中,如此功力却是世所罕見。其時丐幫和少林僧雙方劍拔弩張,大戰一觸即發,廣場上群雄人人提心在手,對空智這手功夫,竟是誰都忘了喝采。

空智的目光從少林群僧的臉上一個個望了過去,緩緩説道︰「這枚毒針是誰所發?大丈夫敢作敢當,給我站了出來。」便在此時,無忌心念一動,想起了一事︰昔年他母親殷素素喬裝他父親張翠山模樣,以毒針殺死少林僧,令他父親含冤莫白。但白眉教的金針與此銀針形狀大不相同,針上毒性也是截然有異,從丐幫傳功長老的死狀看來,銀針上所帶劇毒似乎是西域一種見血封喉的「心一跳」毒蟲所練。所謂「心一跳」,是説這種毒蟲的劇毒一與熱血相混,中毒者的心臟只跳得一跳,便即停止。

無忌早知史火龍是被圓眞所殺,看來少林群僧之中,隱伏著不少圓眞的黨羽,所以發這毒針傷害傳功長老性命,便是要阻止他説出圓眞的名字,無忌眼力雖然敏鋭,却也没瞧出誰是發射毒針的兇手。空智説了這番話後,數百名少林僧一言不發,有的只是説︰「阿彌陀佛,罪過,罪過!」

掌棒龍頭大聲道︰「殺害史幫主的兇手是誰,丐幫數萬弟子無一不知。你們想殺人滅口,除非將天下丐幫弟子,個個殺了。這個殺人的和尚,便是圓眞\dash{}」

他剛説到「圓眞」兩字,掌砵龍頭忽地飛身搶在他的面前,鐵砵一舉,叮的一聲輕響,將一枚銀針接在砵中。這枚銀針仍是不知從何方而來,只是掌砵龍頭全神貫注的防備,陽光下只是銀光微一閃鑠,便舉砵接過,只要稍稍慢得半步,掌棒龍頭便又死於非命。空智身形一挫,已經到了達摩堂九僧的身後,迅捷無倫的飛起一腿,砰的一聲,將左起第四名老僧踢了出來,跟著一把抓住他的後領,提身而起,説道︰「空如,原來是你,你也和圓眞勾結在一起了。」右手拉住他的前襟往下一扯,嗤的一聲響,衣襟破裂,露出腰間一個小小的鋼筒。筒頭鑽著一個細孔。原來這鋼筒中裝有強力彈簧,只須伸手在懷中一按筒上機括,孔中便射出餵毒銀針。發射這暗器不須抬臂揮手,即使二人相對而立,只隔數尺,也是看不出對方已發射了暗器。掌棒龍頭悲憤與驚怒交集,提起鐵棒橫掃過去,便將這空如打得腦漿迸裂而死。這空如乃是和四大神僧同輩的老僧,雖不是上代方丈嫡傳的弟子,但在少林派中輩份武功均高,只因被空智擒住後拿著脈穴,掙扎不得,掌棒龍頭一棒掃來,他竟是無法躱閃。群雄又是齊聲驚叫。

空智一呆,向掌棒龍頭怒目而視,心想︰「你這人忒也魯莽,也不問問清楚。」

正混亂間,廣場外忽然飄進四名玄衣女尼,手中各執拂塵,朗聲説道︰「峨嵋派掌門周芷若,率領門下弟子,拜見少林寺空聞方丈。」空智放下空如的屍身,説道︰「請進!」不動聲色的迎了出去,達摩堂下剩下的八名老僧仍是跟在他的身後,適纔一幕慘劇,竟如同並未發生過一般。四名女尼行禮後倒退,轉身回出,飄然而來,飄然而去,難得的是四個人齊進齊退,宛似一人,脚下更是輕盈翩逸,有如行雲流水,凌波步虛。

張無忌聽得周芷若到來,登時滿臉通紅,偸眼向趙明看去。趙明也正望著他。二人目光相觸,趙明眼色中似笑非笑,嘴角微斜,似有輕蔑之意,也不知是嘲笑張無忌的狼狽失措,還是瞧不起峨嵋派虛張聲勢。

峨嵋派衆女俠却不同丐幫般自行來到廣場,直待空智率同群僧出迎,這纔列隊而進,但見八九十名女弟子一色的玄衣,其中大半是落髮的女尼,一小半是老年、中年、妙齡女子。女弟子走完,相距丈餘,一位秀麗絶俗的青衫女郎緩步而前,正是峨嵋派掌門周芷若。無忌見她容顏清減,頗見憔悴之色,心下又是憐惜,又是慚愧。

在周芷若身後相隔數丈,則是二十餘名男弟子,身穿玄色長袍,大多彬彬儒雅,不似别派的武林人物那麼雄健飛揚。每名男弟子手中都提著一隻木盒,或長或短。百餘名峨嵋人衆,身上和手中均不帶兵刃,這些木盒之中,顯然都是兵器。群雄一見之下,心中暗讚︰「峨嵋派甚是知禮,兵刃不露,那是敬重少林派之意了。」張無忌待峨嵋派衆人坐定,走上前去,向周芷若長揖倒地,含羞帶愧,説道︰「周姊姊,張無忌請罪來了。」

峨嵋派中十餘名弟子霍地站了起來,個個柳眉倒豎,極是憤怒。周芷若萬福回禮,説道︰「不敢,張教主何須多禮?别來安好。」臉色平靜,也不知她是喜是怒。

張無忌心下忡怔不定,説道︰「芷若,那日爲了急於救援義父,致誤大禮,我心下好生過意不去。」他見峨嵋派站著的女弟子之中,有當日斷臂的靜慧在内,上前也是一揖,説道︰「張無忌多多得罪,甘心領責。」靜慧身子一側,不受他這禮,却是一言不發。周芷若道︰「聽説謝大俠失陥在少林寺中,張教主英雄蓋世,想必已經救出來了。」

張無忌臉上一紅,道︰「少林派衆高僧修爲深湛,明教已輸了一仗,我外公不幸因此仙逝。」周芷若道︰「殷老先生一世英雄,可惜,可惜!」

張無忌見她既不發怒,也不露絲毫喜色,不知她心中如何打算,自己每説一句話,總是被她一個軟釘子碰了回來,眞是老大没趣。但轉念一想,那日與她成婚之日,自己當著無數賓客之面,竟隨趙明飄然而去,當時周芷若心中的難過,比今日自己的小小没趣重過何步千倍萬倍,當下説道︰「待會相救義父,還望念在昔日之情,賜予援手。」

他一説這幾句詁,心中忽然一動︰「這半年來芷若功力大進,那日喜堂之上,連苦頭陀范遙這等身手,一招之間便被她逼開。明妹學兼各家各派之所長,更是險些被她斃於當場。想來凡是接住峨嵋掌門之人,她派中另有密傳的武功祕笈,她悟性高於滅絶師太,以致青出於藍,更勝於藍。倘若她肯和我聯手,只怕便能攻破金剛伏魔圏了。」想到這裡,不禁喜形於色,説道︰「芷若,我有一事求你。」

周芷若臉色忽然一板,説道︰「張教主,請你自重,咱們男女有别,不可再用舊時稱謂。」她伸手向身後一招,説道︰「青書,你過來,將咱們的事向張教主説説。」

只見一條滿臉虯髯的漢子走了過來,抱拳道︰「張教主,你好。」無忌一聽聲音,正是宋青書,仔細辨認,纔認出是他,原來他大加化裝,扮得又老又醜,掩飾了本來面目。無忌抱拳道︰「原來是宋師哥,一向安好。」宋青書微微一笑,道︰「説起來還得多謝張教主纔是。那日你要與内子成婚,偏生臨時反悔\dash{}」張無忌聽到「與内子成婚」這五個字,大吃了一驚,顫聲問道︰「什麼?」宋青書道︰「我這段美滿姻緣,倒要多謝張教主作成了。」陡然之間,無忌想起周芷若自殺那晩所説的話來,她自稱陥身丐幫之時,爲宋青書所汚,腹中留下了他的孼種。霎時之間,無忌猶似五雷轟頂,呆呆站著,眼中瞧出來一片白茫茫地,耳中聽到無數雜亂的聲音,却半點不知旁人在説些什麼,過了良久,只覺有人挽住他的臂膀,説道︰「教主,請回去吧!」

張無忌定了定神,一斜眼,見挽住自己手臂的却是韓林児。只見他臉上充滿了愁苦悲憤之色,對周芷若道︰「周姑娘,我教主乃是個大仁大義的英雄,那日事出誤會,你便嫁了這個\dash{}這個\dash{}哼,哼!」他本想痛罵宋青書幾句,但又礙著周芷若的面子,話到口邉,却又忍了下去。張無忌對趙明雖是情根深重,但一直覺得自己與周芷若已有婚姻之約,當日爲了營救義父,迫不得已纔隨趙明而去,料想周芷若溫柔和順,只須向她袒誠説明其中情由,再大大的陪個不是,定能得她原恕,豈知她一怒之下,竟然嫁了宋青書,這時心中的痛楚,竟比昔時在光明頂上被她刺了一劍,更加難受。

他回過頭來,只見周芷若伸出皓如白玉的纖手,向宋青書招了招。

宋青書得意洋洋的走到她身旁,挨著她坐了嘴角邉似笑非笑,向張無忌道︰「咱們成親之時,並没大撒帖子,只是峨嵋派的同門到賀道喜。這杯喜酒,日後還該補請你纔是。」張無忌想説一句「多謝了」,但這三個字竟是説不出口。

韓林児拉著他臂膀,道︰「教主,這種人别去理他。」宋青書哈哈一笑,道︰「韓大哥,這杯喜酒,屆時也少不了你。」韓林児在地下吐了一口唾沫,恨恨的道︰「我便是喝三缸馬尿,也勝過喝你的倒霉死人酒。」張無忌知他性子直率,在這児當衆與人爭吵甚是不好看相,嘆了一口氣,挽著韓林児的手臂便走。

只聽得丐幫的掌棒龍頭大著嗓子,正與一名少林僧爭得甚是激烈。張無忌與周芷若宋青書這些言語,是在西北角峨嵋派的竹棚前所説,低聲細氣,並未惹人注意。廣場上群雄,一直都在聽丐幫與少林派的爭執。無忌回到明教的竹棚坐定,心頭極是煩亂。只聽那穿大紅袈裟的少林僧説道︰「我説圓眞師兄和陳友諒都不在本寺,貴幫定然不信。貴幫傳功長老不幸喪命,敝派空如師叔已然抵命,還有什麼説的?」

掌棒龍頭道︰「閣下要想搜査少林僧,未免枉妄了一點吧?區區一個丐幫,未必有此能耐。」掌棒龍頭怒道︰「你瞧不起丐幫,好,我先領教領教。」那少林僧道︰「千百年來,也不知有多少英雄好漢駕臨少林,敝派雖然多是酒囊飯袋之輩,仗著老祖慈悲,少林寺却也没教人燒了。」他二人越説越僵,眼看就要動手,空智坐在一旁,却並不干預。忽聽得「酔不死」司徒千鍾陰陽怪氣的聲音説道︰「今日天下英雄齊集少林,有的遠從千里之外趕來,難道是爲瞧丐幫報仇來麼?」

川東老英雄夏胄大聲道︰「不錯。丐幫與少林派的樑子,暫請擱在一旁,慢慢算帳不遲,咱們先料理了謝遜那奸賊再説。」掌棒龍頭怒道︰「你口中可别不乾不淨,金毛獅王謝大俠,乃是明教四大護教法王之一,什麼奸賊不奸賊的?」夏胄聲若洪鐘,大聲道︰「你怕明教,我可不怕明教。魔教中出了這種豬狗不如、狼心狗肺的奸賊,還尊他一聲英雄俠士麼?」楊逍身形一晃,走到廣場心中,抱拳團團行了一禮,説道︰「在下明教光明左使,有一言要向天下英雄分説。敝教謝獅王昔年殺傷無辜,確有不是之處\dash{}」夏胄道︰「哼,人都給他殺了,憑你輕描淡冩的幾句話,便能令死人復生麼?」

楊逍昂然道︰「咱們行走江湖,過的是刀頭上舐血的日子,活到今日,那一個手上不帶著幾條人命?武功強的,多殺幾人,學藝不精的,命喪人手,每殺一個人都要抵命,嘿嘿,這廣場上數千位英雄好漢,留下來的只怕寥寥無幾的了。夏老英雄,你一生之中,從未殺過人麼?」

楊逍這一句話登時將夏胄問得啞口無言。其時天下大亂,四方擾攘,武林人士行走江湖若非殺人,便是被殺,頗難獨善其身,手下不帶絲毫血漬者,除了少林派、峨嵋派若干僧尼之外,可説極是罕有。這川東大豪夏胄生性暴躁,傷人不計其數,他一時語塞,呆了一呆,纔道︰「歹人該殺,好人便不該殺。這謝遜和明教衆魔頭一模一樣,專做傷天害理之事,我恨不得千刀萬剮,食其肉寢其皮。哼哼,姓楊的,我瞧你也不是好東西。」他雖知明教中厲害的人物甚多,但想今日與會者大都是明教的對頭,己方人衆而對方勢孤,既要殺謝遜爲弟復仇,勢必與明教血戰一場不可,因此言語中再也不留絲毫地步。

明教竹棚中一人尖聲尖氣的説道︰「夏胄,你説我是不是好東西?」夏胄向説話之人瞧去,只見他削腮尖嘴,臉上灰沉沉地無半分血,不知他是何等樣人物,喝道︰「我不知你是誰。既是魔教中人,自然也不是好東西了。」司馬千鍾插口道︰「夏兄,這一位你也不識得麼?那是明教四大護教法王之一的青翼蝠王。」夏胄道︰「{\upstsl{呸}},{\upstsl{呸}}!吸血魔鬼!」突然之間,群雄眼前一花,只見韋一笑已欺到了夏胄身前,他二人相隔十餘丈,不知韋一笑如何在頃刻間一閃即至。韋一笑提起手來,劈劈拍拍四響,打了他四下耳光,手肘一探,已撞了他小腹上的穴道。夏胄武功本來也非泛泛,韋一笑若憑眞實功夫與他相鬥,至少也得拆到五十招上,方能勝他,但韋一笑的輕身功夫實在太快,如電而至,攻了他一個措手不及,夏胄待要招架,已然著了他的道児。群雄驚呼聲中,只見明教竹棚中又是一條白影竄出。這白影的身法雖不及韋一笑那麼迅電閃電一般,却也是疾逾奔馬。

只見那白影來到夏胄身前,一隻布袋張了開來,兜頭罩下,將夏胄裹在布袋之中,往肩頭一揹,群雄這纔看清,乃是一個笑嘻嘻的僧人,正是布袋和尚説不得。原來韋一笑偸襲成功,順勢點了夏胄的穴道,説不得用布袋罩他時,夏胄已全無招架之力。説不得笑道︰「好東西,你是好東西,和尚揹回家去,慢慢的煮來吃了!」他肩頭雖是負著一人,脚下仍是輕飄飄地,毫不費力的回歸竹棚。

這一幕詭異之極的怪事倏然而起,倏然而止,夏胄身旁雖有十來個好友和弟子,但誰都不及救援。待得韋一笑和説不得回歸竹棚就坐,那十來人才拔出兵刃,趕到明教棚前,紛紛喝罵要人。説不得拉開布袋之口,笑道︰「你們都給我回去,安安靜靜地坐著,大會一完,我自將他好好放了出來。你們不聽話麼,和尚就在這布袋中拉一泡尿,拉一頓屎。你們信是不信?」一面説,一面便作解開自己褲帶之狀。那十餘人氣得臉容變色,但想明教這一干人無惡不作,説得出便做得到,要憑武力奪人是辦不到的了,倘若這賊禿眞在夏胄頭上撤一泡尿,夏胄非自殺不可。各人你看看我,我看看你,只得垂頭喪氣的回去。旁觀群雄又是駭異,又是好笑,上山之時,本來個個興高采烈,要看如何屠戮謝遜,此刻見了明教二豪的身手,這纔覺得今日之會大是兇險,縱然殺得謝遜,只怕這廣場上也非染滿鮮血、伏屍遍地不可。

只見司馬千鍾左手拿著一隻酒杯,右手提著一隻酒葫蘆,搖頭晃腦的走到廣場中心,説道︰「今日當眞有好大的熱鬧瞧,有的要殺謝遜,有的要救謝遜,可是説來説去,這謝遜到底是否在這少林寺中,却是老大一個疑竇。我説空智大師哪,你不如將金毛獅王請了出來,先讓大夥児見上一見。然後要殺要救的雙方,各憑眞實的本領,結結棍棍的打上一場,豈不有趣?」

\chapter{比武較量}

司馬千鍾這番話一説,廣場上群雄倒有一大半轟然叫好。楊逍心想︰「謝獅王怨家太多,明教縱與丐幫聯手,也不足與天下英雄相抗,不如從屠龍刀上著眼,攪成個群相角逐的局面。」於是朗聲説道︰「衆位英雄今日齊集少林,一來是與謝獅王各有恩怨未了,二來嘛,嘿嘿,只怕也想見識見識這把屠龍寶刀。倘若依司馬先生所説,大夥児一場混戰,那麼這把寶刀歸誰所有呢?」群雄一聽,倒也有理,這數千人之中,眞正與謝遜有血深仇的,也不過百餘人而已,其餘衆人一想到那「武林至尊」四字,實是禁不住怦然心動。

只見一個黑鬚老者站了起來,説道︰「那屠龍刀現下是在何人手中,還請楊左使示下。」楊逍道︰「此節敝人不明,正要請教空智禪師。」空智搖了搖頭,默然不語。群雄心下均是暗暗不滿,心想︰「少林派是英雄大會的主人,但空聞方丈臨時裝病不出,這個名滿天下的空智神僧,却又是一副不死不活的神氣,不知在弄什麼玄虛。」一個身穿高葛長袍的中年漢子站起來説道︰「空智禪師既説不知,那麼謝獅王必是知道的了。咱們請他出來,問他一問。然後各憑手底玩藝見眞章,誰的武功天下第一,那麼名副其實,自然而然的是『武林至尊』,不管這把刀是在誰的手中,都該交與這位武林至尊。依我説啊,大夥児先議定了這節,免得事後爭執。若有不服的,天下英雄群起而攻之。衆位意下如何?」無忌認得這説話之人,正是那晩圍攻少林僧金剛伏魔圏的青海派三大高手之一。司馬千鍾道︰「那不是打擂台麼?我看有點大大児的一妥。」那黃袍漢子冷然道︰「有何不妥?依閣下之見,不比武功,是要比酒量了?那一個千鍾不酔,那一個酔而不死,便是武林至尊了?」衆人一聽,都轟然大笑起來。人叢中有人怪聲説道︰「這還比什麼,這武林至尊,自是『酔不死』司馬先生!」

司馬千鍾斜過酒葫蘆,在杯中倒了一杯酒,仰脖子喝了,一本正經的道︰「不敢,不敢!要説到『酒林至尊』,我『酔不死』或許還有三分指望。這『武林至尊』四字哪,哈哈,不敢當啊,不敢當。」他對著那黃袍漢子道︰「閣下既提此議,武學上自有超凡入聖的造詣,在下眼拙,却不知閣下尊姓大名。」那黃袍漢子冷冷的道︰「在下是青海派葉長青,喝酒的本事和裝丑角的玩藝,都不及閣下。」言下之意,是説武功上的修爲,只怕要比閣下強得多了。司馬千鍾側頭想了半晌,説道︰「青海派,没聽見過。葉長青,{\upstsl{嗯}},{\upstsl{嗯}},没聽見過。」衆人暗想︰「這司馬老児好大的膽子,侮辱葉長青一人,那也罷了,他言語中竟然侮辱青海一派,難道他身後有什麼強大靠山?還是跟青海派有什麼解不開的仇怨?單憑這兩句話,青海派只怕立時便要出手。」

其實司馬千鍾孤身一人,並無靠山,他跟青海派也無什麼樑子,只是生性狂妄,喜歡口舌招尤,雖然生平因此而吃了不少苦頭,却始終改不了這個脾氣。葉長青甚是陰沉,心中已然動了殺機,但臉上不動聲色,問道︰「閣下既説比武之議不妥,比灌黃湯嘛,閣下又是喝遍天下無敵手,那便如何是好,倒要請教。」司馬千鍾道︰「要説喝遍天下無敵手,此事談何容易。想當年我在濟南府\dash{}」他正要勞勞叨叨的説下去,人叢中有人喝道︰「酔不死,别在這児發酒瘋啦,大夥児没空聽你胡説八道。」又有人道︰「到底謝遜的事怎樣?屠龍刀的事怎樣?」另有人道︰「空智禪師,你是今日英雄大會的主人,叫咱們乾耗著,算是怎麼一會子事?」衆人你一言,我一語,都是催著司馬千鍾快些走開,要空智拿一句言語出來。

這些人在人叢中紛紛議論,或遠或近,聲音來自四面八方。司馬千鍾道︰「江陵府黑風寨的史老大,你不用性急,你的黑沙掌雖然厲害,未必便打遍天下無敵手,鄱陽湖的水底金鰲侯兄弟,那謝遜獅王的武功水陸倶能,你别欺他不會水底功夫,何況人家還有一位紫衫龍王没出面,嘿嘿,鰲魚豈是龍王之比?青陽山的吳三郎,你是用劍的,便是奪到屠龍刀,你又不會使,瞎起個什麼勁?\dash{}」這司馬千鍾説話瘋瘋癲癲,却當眞有過人之長,相識既廣,耳音又是絶佳,從一片嘈雜的人聲之中,居然將一個説話之人指名道姓的叫了出來,無一有誤。群雄見他顯了這手功夫,却也忍不住喝采。

只見空智身後一名達摩堂僧人站了起來,説道︰「少林派忝爲主人,不幸空聞方丈突患重病,盛會主持無人,倒讓各位見笑了。謝遜和屠龍刀二事,其實一而二,二而一,儘可合併辦理。以老衲之見,適纔青海派這位葉施主説得甚是有理。與會群雄,英才濟濟,只須各人露上一手,最後那一位藝壓當場,謝遜歸他處置,屠龍刀也由他執掌,群雄歸心,豈不是好?」張無忌低聲詢問彭瑩玉,説這話的僧人是誰。彭瑩玉搖頭道︰「屬下不知。這僧人未參與圍攻光明頂之役,也没被郡主娘娘擒入萬法寺中,可是他一再搶在空智大師的前頭説話,似乎在寺中位份甚是不低。」趙明低聲道︰「這人十九是圓眞一黨。我猜想空聞方丈已落在圓眞手中,空智大師受了這群叛徒挾制,以致萎靡氣沮。」

無忌心中一凜,道︰「彭師傅以爲如何?」彭瑩玉道︰「郡主的猜測也是大爲有理。只是少林寺中高手如雲,圓眞竟敢公然犯上作亂,膽子忒也大了。」無忌道︰「圓眞佈置已久。第一次想瓦解本教,第二次意圖統制丐幫,兩次奸謀均是功敗垂成,這一我想他是要少林派的掌門方丈。」趙明道︰「單是做掌門方丈,也還不彀。」無忌道︰「少林派是武林中第一門派,做到掌門方丈,那已是登峰造極,不能再高了。」趙明道︰「武林至尊呢?那不是比少林派的掌門方丈更高麼?」無忌一呆,道︰「他想做武林至尊?」趙明道︰「無忌哥哥,周姊姊嫁了旁人,你神魂不定,甚麼事也不會想了。」無忌被她説中了心事,臉上一紅,心道︰「張無忌,你不可只管顧念児女之情,將今日營救義父的大事擱在一旁。」定了定神,心想圓眞深謀遠慮,今日這英雄大會,顯然也是他一力促成,其中定有奸謀,便道︰「明妹,你猜圓眞有何詭計?」

趙明道︰「圓眞此極工心計,智謀百出\dash{}」周顚一直在旁聽著他二人低聲説話,這時忍不住插口道︰「郡主娘娘,你的智謀也不輸於圓眞。」趙明笑道︰「過獎了。」周顚道︰「不是過獎\dash{}」彭瑩玉道︰「顚兄,你别打斷郡主娘娘的話。」周顚怒道︰「你先别打斷我的話!」彭瑩玉笑了笑,不再説話,知道若是跟他糾纏下去爭上一兩個時辰也是弄不清楚,還是乘早收口的乾淨。周顚道︰「你怎麼不説話了?」彭瑩玉道︰「你叫我别打斷你的話,我就不打斷你的話。」周顚道︰「可是你已經打斷過了。」彭瑩玉道︰「那你再接下去説就是。」周顚道︰「我忘了,説不下去啦。」

趙明笑了笑,道︰「我想圓眞若是單想做少林寺方丈,不必請天下英雄來此。謝大俠既已落入他的手中,何必又要叫群雄比武爭奪?無忌哥哥,説到武功之強,只怕當今之世,無人及得上你,此節圓眞不會不知。決不能這般好心,安排下群雄大會,讓你技勝群雄,成爲武林至尊,然後將謝大俠和屠龍刀獻上給你。」張無忌、彭瑩玉、周顚三人一齊點頭,問道︰「你猜他有何詭計?」

這時楊逍已回到張無忌身旁,插口道︰「我也一直在想,圓眞這厮奸謀定是不小\dash{}」周顚忍不住又道︰「圓眞是本教的大對頭,郡主娘娘,從前你也是本教的大對頭。圓眞這厮詭計百出,郡主娘娘,你也是詭計百出。你兩個児倒有點児差不多。」楊逍喝道︰「你又在瘋瘋癲癲的瞎説了。」趙明微微一笑,道︰「周先生之言甚是有理,倘若我是圓眞,我該當如何圖謀呢?{\upstsl{嗯}},第一,我勸空聞方丈大撒英雄帖,請得天下英雄來到少林寺。想那空聞方丈佛法精深,原是個慈悲和平之人,自來不喜多事,但我只須提起空見和空性兩位神僧,空聞方丈念著師兄之情,自必允可。再者,少林寺若是殺了謝大俠,和明教深似海,以他一派之力,未必擋得住明教的進襲,但若往天下英雄頭上一推,明教總不能將與會的數千好漢一古腦児的給宰了。」

衆人都點頭稱是。趙明又道︰「英雄大會一開成,我自己也不露臉,叫人以謝大俠與屠龍刀爲餌,鼓動群雄自相爭鬥殘殺。明教勢必與群雄爲敵,鬥到後來,不論誰勝誰敗,明教的衆高手少説也當損折一半,元氣大傷。」張無忌道︰「正是。此節我原也想到了,但義父對我恩重如山,與衆兄弟又是數十年的交情,咱們豈能坐視不救?唉,咱們上山没幾天,外祖父已然仙逝,圓眞這厮定是躱在暗中拍手稱快。」趙明道︰「鬥到最後,武功第一的名號多半是張教主所得,於是少林群僧説道︰『張教主技壓群雄,實乃可敬可賀,本寺謹將謝大俠交於教主,請教主到寺後山峰頂上去迎取便是。』於是大夥児一齊來到峰頂,張教主便須獨力去破那金剛伏魔圏。若是旁人上前相助。圓眞的黨羽便道︰『技壓群雄的是張無忌張教主,跟旁人可不相干,閣下還是站在一旁的爲妙。』張教主奪得這武功天下第一的名頭,就算身上毫不帶傷,也不知耗了多少内力神功,到那時如何是這三位老僧之敵?結果謝大俠是救不出,反而自己死在三株蒼松之間。冷月淒風,伴著一代大俠張無忌的屍首,豈不妙哉?」

群豪聽到這裡,都是臉上變色,心想這番話確不是危言聳聽,張無忌血性過人,不論多麼艱苦危難,總是非救謝遜不可,縱然送了自己性命,也是絶無反悔。圓眞此計看準了無忌的性子,教他明知是刀山油鍋,也要跳將進去。趙明嘆了口氣,説道︰「這麼一來,明教是毀定了,圓眞再使奸計,毒死空聞,却將罪名推在空智大師的頭上,這一著安排起來十分容易,只須證據捏得造確實,不由得少林僧衆不信。於是各黨羽一力推舉,他老人家順理成章的當上了方丈。他老人家一聲號令,群雄圍攻明教,以多勝少。聚而殲之,那時候武功天下第一的名號,除了他老人家之外,只怕旁人也爭奪不去。屠龍刀不出現便罷,若在江湖上現了蹤跡,天下英雄人人皆知,這把寶刀的正主児,乃是少林寺方丈圓眞神僧。寶刀的得主若不給他老人家送去,只怕多多不便哪。」

她説得聲音甚低,除了竹棚中這一角之外,旁人均不能聽見,但這番話一説完,周顚伸手在自己大腿上用力一拍,叫道︰「正是,正是!好大的奸謀。」他這幾句話却是十分響亮,廣場上倒有一大半人都聽見了,各人的眼光一齊望到明教這一邉來。司馬千鍾道︰「什麼奸謀了?説給老夫聽聽成不成?」周顚道︰「這話是不能説的。老夫一心想挑撥離間,要天下英雄自相殘殺,拚個你死我活,這話説了出來,豈不是不靈了麼?」司馬千鍾笑道︰「妙極,妙極!却不知如何挑撥離間,願聞其詳。」周顚大聲道︰「我心中有一個陰謀毒計,要假意説道︰屠龍刀是在老子這裡,那一個武功最強,老子就將屠龍刀給他\dash{}」司馬千鍾叫道︰「好計策!好計策!那便如何?」趙明與張無忌對望了一眼,心想︰「這酒鬼跟咱們無親無故,倒要幫忙得緊。」周顚大聲説道︰「你想這屠龍刀號稱『武林至尊』,那一個不想出全力爭奪?於是瘋子給酒鬼殺了,酒鬼給和尚殺了,和尚給道士殺了,道士給姑娘殺了\dash{}殺了個天翻地覆,血流成河,嗚呼,屍橫遍野,不亦樂乎。」

群雄一聽,都是慄然心驚,均想這人説話雖是瘋瘋癲癲,却是無一而非至理。崆峒派的二老宗維俠站起身來,説道︰「這位周先生言之有理。咱們明人不説暗話,各家各派對這把屠龍刀嗎,都是有點児眼紅,可是人人爲了它鬧個身敗名裂,甚至是全派覆滅,那可有點児犯不著。我想大夥児得想個計較,以武會友,點到爲止,雖分勝敗,却是不傷和氣。各位以爲如何?」原來光明頂一役,張無忌以德報怨,替他治好了因練七傷拳而蓄積的内傷,宗維俠好生感激,崆峒派這次上少林寺來,原有相助明教救援謝遜之意。

司馬千鍾笑道︰「我瞧你好大的個児,却是怕死。既不帶彩,又不傷命,這場比武有什麼看頭。」崆峒派的四老常敬之性子極易暴燥,怒道︰「要傷你這酒鬼,那也不用叫你帶彩。」司馬千鍾道︰「我不過是句玩児,常四先生何必這麼大的火氣?誰不知道崆峒派的七傷拳殺人不見血。少林寺的空見神僧,不也是死在七傷拳之下麼?我司馬酒鬼這幾根老骨頭,如何是空見之比?」群雄均想︰「這酒鬼出口便是傷人,既得罪崆峒派,又損了少林派。他在江湖上打滾,居然給他混到這麼年紀還不死,倒也是奇事一樁。」宗維俠却不去睬他,朗聲道︰「依在下之見,每一門派,每一幫會教門,各推兩位高手出來,分别較量武藝。最後那一派武功最高,謝大俠與屠龍刀都憑他處置。」群雄轟然鼓掌,都説這法子最妙。張無忌留心看空智身後的少林群僧,大多是皺起眉頭,頗有不悦之色,知道趙明識穿圓眞的奸謀,破了他挑撥群雄自相殘殺之計。

只見一個白面微鬚的中年漢子站起身來,手搖描金摺扇,神情甚是瀟灑,説道︰「在下覺得宗二俠此議甚是。咱們比武較量之時,雖説是點到爲止,但兵刃拳脚上不生眼睛,若有失手,那也是各安天命。同門同派的師友,可不許出來挑戰報復,否則又是糾纏不清,鬥了個没有了局。」群雄都道︰「不錯,正該如此。」司馬千鍾尖著嗓子,説道︰「這一位兄台好英俊的人物,説話又是哈聲哈氣的,想必是湘南衡陽府的歐陽兄台了?」那人摺扇搖了兩搖,笑道︰「不敢,正是區區,你捧我一句,損我一句,剛好抵過。」司馬千鍾道︰「歐陽兄和我好像都是孤魂野鬼,不屬什麼幫會門派。我好酒,你好色,咱哥児倆創一個『酒色派』,咱們酒色派兩大高手併肩子齊上,會一會天下衆高手如何?」群雄哈哈大笑,覺得這司馬千鍾不住的插科打諢,逗人樂子,使會場平添不少笑聲,減少了許多暗中潛伏的煞氣。原來這白臉漢子名叫歐陽牧之,一共娶了十二名姬妾,武功雖強,却是極少闖蕩江湖,整日價倚紅偎翠,享這溫柔鄕之樂。

歐陽牧之笑道︰「若是跟你聯手組派,我這副身家可不彀你喝酒。各位,説到比武較藝,咱們可得推舉幾位年高德劭,衆望所歸的前輩出來作個公證纔是。以免你説你贏,我説我贏,爭執個不休。」司馬千鍾笑道︰「輸贏自己不知道麼?誰似你這般胡賴不要臉。」宗維俠道︰「還是推舉幾個公證人的好,少林派是主人,空智僧人自然是一位了。」司馬千鍾指著説不得布袋道︰「我推舉這布袋児裡的川東大俠夏胄夏老英雄。」

説不得提起布袋,向司馬千鍾擲了過去,笑道︰「公證人來啦!」司馬千鍾抛下葫蘆酒杯,便去解布袋上的繩子,不料説不得打繩結的本事另有一功,那綑縛袋口的繩子又是金絲混和魚瞟所纏成,司馬千鍾用盡力氣,竟是解之不開。説不得哈哈大笑,縱身而前,左手提起布袋,拿到自己背後,右手接著,十根手指扭了幾扭,又提到身前,就是這麼在身前身後兜了一個圏子,布袋上的繩結已然鬆開。他倒轉袋裡一抖,夏胄滾了出來。司馬千鍾忙伸手解了他的穴道。夏胄在黑漆一團的袋中悶了半天,突然間陽光耀眼,又見廣場上成千對眼睛一齊望著自己,不由得羞愧欲死,翻手拔出身邉短劍,便往自己胸口插了下去。司馬千鍾夾手奪過,笑道︰「夏兄何必如此心拙?」人叢中一個矮矮胖胖的漢子大聲説道︰「這位布袋中的大俠,只怕没資格做公證人,我推舉長白山的孫老爺子。」又有一個中年婦人説道︰「浙東雙義威震江南,他兩兄弟正直無私,正好作公證人。」群雄你一言,我一語,霎時之間推舉了十餘人出來,均是江湖上頗具聲望的豪傑。

正紛擾不決之際,峨嵋派中一個老尼姑冷冷的道︰「推舉什麼公證人了?壓根児便用不著。」她話聲並不十分響亮,但清清楚楚的鑽入各人耳中,顯然内力修爲頗是了得。司馬千鍾笑道︰「請教這位師太,何以不用公證人?」那老尼道︰「二人相鬥,活的是贏,死的便輸,閻王爺是公證人。」一衆人聽了這句冷森森的話,背上均是感到一絲涼意。司馬千鍾道︰「咱們以武會友,又無深仇大冤,何必動手便判生死?出家人慈悲爲本,這位師太之言,也不怕佛祖嗔怪麼?」那老尼冷冷的道︰「你跟旁人説話瘋瘋癲癲,在峨嵋弟子跟前,可得給我規矩些。」司馬千鍾拾起葫蘆酒杯,斟了一杯酒,笑道︰「嘖嘖嘖!好厲害的峨嵋派。常言道好男不與女鬥,好酒鬼不與尼姑鬥!」舉起酒杯,剛放到唇邉,突然間{\upstsl{嗖}}{\upstsl{嗖}}兩響,破空之聲極強,兩枚小小念珠激射而至,一枚打中酒杯,一枚打中葫蘆,跟著又是一枚射至。正中他的胸中。

只聽得彭彭三聲巨響,三枚念珠炸了開來,葫蘆酒杯登時粉碎,司馬千鍾胸口炸了個大洞。他身子被炸力一撞,向後摔出數丈,全身衣服立時著火。夏胄上前撲打,只見司馬千鍾已然氣絶,臉上兀自帶著笑意。可見那三枚念珠飛射之速,司馬千鍾直至臨死,絲毫没想到大禍已然臨頭。這一下奇變猶如晴空打了個焦雷,群雄中不乏見多識廣之士,可是誰也没見過如此迅厲害的暗器。

周顚説道︰「乖乖不得了!這是什麼暗器?」楊逍低聲道︰「聽説西域阿拉伯國,有一種叫做『霹靂雷火彈』的暗器,中藏烈性炸藥,用強力彈簧機括發射。看來這老尼所用,便是這個傢伙了。」只見夏胄抱著司馬千鍾燒得焦黑的屍身,向著峨嵋派説道︰「我這位司馬兄弟雖然口頭上尖酸刻薄,只不過生性滑稽,心地却是仁厚,一生之中,從未做過任何傷天害理之事。今日天下英雄均在此間,可有那一位能説他幹過何等惡行?」群雄盡皆默然。夏胄指著那老尼道︰「峨嵋派號稱是俠義道的名門正派,豈知竟會使用這種歹毒暗器。武林中雖説力強者勝,却也走不過一個『理』字。請問這位師太上下?」那老尼道︰「我叫靜迦。這位袋中大俠在此指手劃脚,意欲如何?」夏胄慘然道︰「姓夏的學藝不精,慘受明教魔頭的凌辱,那是姓夏的本領不濟,却不損在下一生俠義之名。靜迦師太,你如此狠毒,對得起貴派祖師郭襄女俠麼?」

峨嵋派群弟子聽他提到創派祖師的名諱,一齊站起身來。靜迦一張方臉,兩條長眉斜身豎起,喝道︰「本派祖師的名諱,豈是你這混蛋隨便叫得的?」夏胄道︰「峨嵋弟子多行不義,沾辱祖師的名頭。别説郭女俠,便是滅絶師太當年,縱然心狠手辣,劍底却也不誅無罪之人。似你這等濫殺無辜,你掌門人竟然縱容不管。嘿嘿,峨嵋派還想在江湖上立足麼?」靜迦道︰「你再胡言半句,這酒鬼便是你的榜樣。」夏胄正氣凜然,大踏步走上三步,説道︰「峨嵋掌門若不清理門戸,峨嵋派自此將爲天下不齒。」群雄與峨嵋弟子數千道目光,一齊望向周芷若,却見她向靜迦緩點了點頭。彭彭兩聲響過去,靜迦的霹靂雷火彈射出,夏胄的胸口和小腹各炸了一洞,衣衫著火,但他爲人極其倔強,雖已氣絶,身子兀自直立不倒,手中仍抱著司馬千鍾的屍體。

群雄面面相覷,都是驚得呆了,過了片刻,數百人鼓噪起來,責罵峨嵋派的不是。韋一笑和説不得對視一眼,點了點頭,兩人奔到夏胄的屍身之前,跪地拜倒。説不得道︰「夏老英雄,我二人不知英雄仁義,適纔多有得罪。好教我兄弟羞愧無地。」二人提起手掌,拍拍拍拍幾響,各自打了自己幾下耳光,四邉臉頰登時紅腫。二人撲熄了兩具屍身上的火燄,抱入明教的竹棚。張無忌見周芷若突然變得如此狠心,心下好生難過。

群雄鼓噪聲中,周芷若在宋青書耳邉低聲説了幾句話。宋青書點了點頭,緩步走到廣場正中,朗聲説道︰「今日群雄相聚,原不是詩酒風流之會,前來調琴鼓瑟,論文作對。既然不免動到兵刃拳脚,那就保不定死傷。這位夏老英雄適纔言道,司馬先生平生未有歹行,責備本派靜迦太濫傷無辜。衆位英雄復又群相鼓噪,似有不滿本派之意。兄弟倒要請教︰咱們今日比武較量,是否先得査明各人的品行德性?大聖大賢,那纔是千萬傷害不得,窮兇極惡之輩,就不妨任意屠殺?」群雄一時語塞,均覺他的話倒也並非無理。宋青書原是言辭十分便給,又道︰「若説這屠龍刀是有德者居之,咱們何必再提『比武較量』四字?不如大家齊赴山東,去到曲阜大成孔夫子的文廟之中,恭請孔聖人的後代收下。但若説到這個『武』字,較量之際只顧生死勝敗,恐怕顧不得對方是『無辜』還是『有辜』了。」群雄中便有人説道︰「不錯,刀槍無眼,咱們原就説過不能尋仇報復。」

兪蓮舟和殷利亨聽著宋青書的説話,口音越聽越像,只是他滿臉短鬚,又是口口聲聲「本派,本派」,顯是峨嵋派的男弟子,不由得大起疑竇。兪蓮舟起身道︰「請教閣下尊姓大名。」宋青書見到二師叔,積威之下,不禁有些害怕,窒了一窒,纔道︰「無名後輩,不勞兪二俠下問。」兪蓮舟厲聲道︰「閣下不住口的説『比武較量』,想必武學上有過人的造詣了。家師幼時曾受過貴派郭女俠的大恩,累有嚴訓,武當弟子不敢與峨嵋派動手。在下要問個明白,閣下是否眞是峨嵋弟子,姓甚名誰?大丈夫光明磊落,有何可以隱瞞之處?」

周芷若拂塵微舉,説道︰「兪二俠,本座也不必瞞你。此人是本座夫君,姓宋名青書,原本系出武當,此刻却已轉入峨嵋門下。兪二俠有何話説,只管衝著本座言講便是。」她這幾句話聲音清朗,冷冷説來,猶如水激寒冰風動碎玉,加之容貌清麗,出塵如仙,廣場上數千豪傑,誰都不作一聲,人人凝氣屏息的傾聽。宋青書伸手在臉上一抹,拉去黏著的短鬚,一整衣冠,登時成爲一個臉如冠玉的英俊少年。群雄一看之下,心中暗暗喝采︰「好一對神仙美眷!」

兪蓮舟想起他戕害七弟莫聲谷的罪行,不由得氣憤填膺,但他一向生性深沉,近年來年事漸高,修爲日益精湛,心中雖是狂怒,臉上仍是淡淡的,只是雙目神光如電,往宋青書臉上掃去。宋青書心下慚愧,不由得低下頭去。周芷若道︰「外子脱離武當,投入峨嵋,今日當著天下英雄之前,正式佈示。兪二俠,張眞人顧念舊日情誼,不許武當弟子與本派爲敵,那是他老人家的義氣,可也正是他人家保全武當威名的聰明處。」殷利亨再也忍不住,跳了出來,指著周芷若道︰「周姑娘,你年幼之時遭遇危難,是我師父出手相救,薦你到峨嵋門下。雖然家師施恩不望報,可是你今日言語之中,顯是説我武當派浪得虛名,遠不及峨嵋派諸位女俠。\dash{}這對得住家師麼?」

周芷若淡淡一笑,道︰「武當諸俠威震江湖,均有眞才實學,宋大俠更是我的公公,本座豈敢説各位浪得虛名?至於武當、峨嵋兩派,各有所傳,各有所學,也難説誰高誰低。昔年本派郭師祖有恩於張眞人,張眞人後來有恩於本座,那就兩相抵過,咱們誰也不欠誰的恩情。兪二俠、殷六俠,武當弟子不得與峨嵋派動手的規矩,咱們就此免了吧。」廣場四周各處竹棚之中,群雄竊竊私議,都説︰「這位年青掌門人好大的口氣,聽她言中之意,似乎峨嵋派拿得定能勝武當派。這位兪二俠内功外功倶已登峰造極,今日會中,只怕以他武功最強,有望奪得屠龍寶刀。難道峨嵋派單憑一件厲害歹毒暗器,便想獨霸江湖麼?」

殷利亨心中激動,想到七弟莫聲谷慘死,不由得流下泪來,叫道︰「青書\dash{}青書!你\dash{}你何以害死你\dash{}你七叔\dash{}」説到「七叔」兩字,突然間放聲大哭。群雄面面相覷,好不奇怪︰「武當殷六俠多大的聲名,竟會當衆大哭?」兪蓮舟走上前去,挽在殷利亨的右臂,朗聲説道︰「天下英雄聽著,武當不幸,出了宋青書這叛逆弟子。在下七弟莫聲谷,便被這\dash{}」突然間颼颼兩響急劇的破空之聲,兩枚「霹靂雷火彈」向兪蓮舟胸口射了過去。張無忌大叫一聲︰「啊喲!」待要撲將上去搶救,但那雷火彈來得實在太快,説到便到,他事先又絲毫没想到峨嵋派竟會如此不顧武林道義,在衆目睽睽之下驀然偸襲,他身法再快,也已不及趕到。

這一下,兪蓮舟也是頗出意外,倘若側身急避,那雷火彈飛將過去,勢必傷了不少丐幫弟子。他生就一副俠義心腸,心想這雷火彈是對付自己而來,要爲的是殺人滅口,以免當衆暴露宋青書犯上叛父的罪行,要是自己閃身避難,不免害死許多無辜。就這麼心念如電的一閃,兩枚雷火彈上先後射到,兪蓮舟雙掌一翻,使出太極拳中一招「雲手」雙掌柔到了極處,空明若虛,將兩枚雷火彈彈射來的急勁盡數化去,輕輕的托在掌心。只見他雙掌向天,平托胸前,兩枚雷火彈在他掌心快速無倫的滴溜溜亂轉。

竹棚中群雄一齊站起身來,數千道目光齊集於他兩隻手心,每個人的心似乎都停了跳動,生怕這兩枚活物一般的雷火彈隨時都會炸將開來。原來這太極拳中的柔勁乃是天下武功中至柔的功夫,正如太極拳中所謂「一羽不能加,蠅蟲不能落」,由粘而虛,隨曲就伸,以「耄耋御衆之形」,而致「英雄所向無敵」。兪蓮舟近年來勤修苦練,已深得張三丰的眞傳,適纔見到司馬千鍾和夏胄先後在此彈下喪命,知道此彈遇物即炸,厲害無比,無可奈何之中,冒險以生平絶學一擋,果然柔能克剛,兩枚雷火彈被他掌心的柔勁制住,就似鑽入了一片黏稠之物中間一般,只是急速旋轉,却不爆炸。但聽得颼颼兩聲,峨嵋派中人又有兩枚雷火彈向他擲了過來。

\chapter{廣場濺血}

殷利亨站在師兄身旁,見又是兩枚霹靂雷火彈射來,當即雙掌一揚,迎著雷火彈接去,待得手掌與雷火彈將觸未觸之際,雙手施出太極拳中「手揮琵琶式」,將雷火彈輕輕攏住,脚下「金雞獨立式」,左足著地,右足懸空,全身急轉,宛似一枚陀螺。原來殷利亨精於劍術,太極拳上的造詣不如師兄深厚,眼見兪蓮舟接那兩枚雷火彈頗爲吃力,自己掌力只要稍有半分用得實了,那歹毒暗器立時便會爆炸,是以全身急轉,雙掌虛帶雷火彈在空中一圏圏的轉動,以化去擲來的勁力。愈蓮舟掌心化勁,殷利亨則是空中化勁,在武功上是稍遜半籌,但一眼望去,却是他急速轉身的身法好看得多。他轉到三十餘轉時,四面八方采聲雷動,雷火彈勁力已衰,豈知颼颼聲響,又是八枚雷火彈擲了過來。

兪蓮舟與殷利亨齊聲暴喝,各將手中的雷火彈擲將出去。武當派弟子不用暗器,却練就一項接器打器的絶技,接到敵人的暗器之後,反擲出去,能以一打二、以一擊三。他二人擲出四枚雷火彈,互相撞擊,將面對八枚雷火彈一齊撞中。廣場上彭彭之聲震耳欲聾,黑煙彌漫,鼻中聞到的盡是硝磺火藥之氣。愈殷二人擲出雷火彈後,當即縱身後躍,退至十餘丈外,以防峨嵋派再接再厲,層出不窮的將雷火彈擲將過來,終究是難以抵擋。

群雄見到這雷火彈如此厲害,無不駭然,心想當世除了武當派這兩位高手之外,只怕没有多少人能接得住,雖然輕功極佳之人可以閃身躱避,但若擲彈之人以「滿天花雨」手法打出,使數枚雷火彈互相碰撞,一經爆炸,身法再快也是躱閃不得的了。只見華山派竹棚中一個身形極高的人站了起來,朗聲説道︰「峨嵋派與人較量武功,就是這般倚多爲勝嗎?」此人正是華山二老之一的高老者,當年在光明頂上,曾與何太沖夫婦聯手和張無忌相鬥。峨嵋派的靜迦説道︰「武功之道千變萬化,力強者勝,力弱者敗。咱們又不是迂腐騰騰的讀書人,事事要講規矩道理,天下也没這麼多規矩道理好講。」群雄見峨嵋派中雖然大都是女流之輩,但其蠻不講理,竟是勝於男子。華山派的高老者和她們理論,却也不敢走近,只是站在自己的竹棚之中,隔得遠遠地説話,生怕對方將霸氣無雙的霹靂雷火彈發了出來。

張無忌心想︰「芷若嫁於宋師哥,實非本心所願,想當日她和我流落海外,雙棲孤島,何等相親相愛?咱們山盟海誓,互不相負,言猶在耳,豈能毀之一旦?這都是我實在對不起她,竟在拜堂成親的大喜之日,當著滿堂賓客之前,和明妹雙雙出走。芷若是一派掌門,千金之體,我這般欺負凌辱於她,怎不教她切齒惱恨?今日峨嵋派許多倒行逆施,實則都是種因於我。」想到這裡,心下越來越是不安,又從竹棚中出來,走到峨嵋派之前,向周芷若道︰「芷若,種種都是我對你不起,你也不必自暴自棄。宋師哥害死莫七叔,此事終須作個了斷。我瞧宋師哥不如隨同兪二伯、殷六叔回返武當,向宋大師伯領罪的爲是。」周芷若冷笑道︰「張教主,我先前還道你是個好漢子,只不過行事胡塗而已,不料竟是個卑鄙小人。大丈夫一人作事一身當,你害死了莫七俠,何以却將罪名推在外子頭上?」

張無忌吃了一驚,道︰「你\dash{}你説我害死莫七叔?我\dash{}那有此事?」周芷若道︰「令尊和令堂,是如何死的?不是幹了對不起人的事,自殺身死麼?你三師伯兪岱岩一世英雄,不是害死在令堂手中,以致終身殘廢麼?令尊以堂堂名門弟子,不是見色起意,與白眉邪教中的妖女苟合成婚麼?張教主,你年紀輕輕,可把令尊令堂這許多德行學了個齊全!」

張無忌滿臉通紅,氣得全身發抖。周芷若倘若是罵他自己,念著昔日情義,自不會和他計較,但她這些惡毒的言語,句句是辱及他的父母。無忌臉色由紅轉白,忍不住便要發作,但轉念又想︰「芷若知我甚深,料得只有侮辱我爹爹媽媽,方能激得我發怒失態。唉,千錯萬錯,總是那日我在婚禮中捨她而去的不是。」牙齒咬著下唇皮,轉身便走。忽聽得峨嵋派中一人大聲説道︰「想不到明教張教主竟是卑鄙懦怯的小人,見到咱們霹靂雷火彈的厲害,挾了尾巴便逃。」張無忌回過身來,見説話的是只餘獨臂的靜慧,不禁嘆了口氣,心道︰「她遭逢不幸,亦因是我而起,我又何必跟她一般見識。」只聽得身後嘲笑之聲,越來越響,張無忌不再理會,回歸明教的竹棚。

楊逍冷笑説道︰「霹靂雷火彈雕蟲小技,何足道哉?既奈何不了武當二俠,自亦奈何不了武當嫡傳的張教主,你們峨嵋派要倚多爲勝,且讓你們見識見識倚多爲勝的手段。」左手一揮,一個白衣童子雙手奉上一個小小的木架,架上插滿了數十面五色小旗。楊逍執起一面白旗。手一揚,那白旗落在廣場中心,插在地下。群雄見白旗連桿不到二尺,旗中繡了一個明教的火燄記號,不知他鬧什麼玄虛。便在此時,楊逍身後一人揮出一枚白色火箭,急升上天。只聽得脚步聲響,一隊頭裹白布的明教教徒奔了進來,共是五百人,每人彎弓搭箭,颼颼聲響,五百枝箭整整齊齊的插在白旗周圍,排成一個圓圏。原來這正是吳勁草統率下的鋭金旗。

群雄未及喝采,鋭金旗教衆已拔出背後標槍,搶上數步,揮手擲出,五百枚標槍一齊插在箭圏之内。衆人跟著又搶上三步,各自拔出腰間短斧。群雄眼前光芒閃動,五百柄短斧呼嘯而前,砍在地下,排成一圏。短斧、標槍、長箭,三般兵刃圍成三個圏子,各不相混。任你武功通天,在這一千五百件長短兵刃的夾擊之下,霎息間便成肉泥。原來鋭金旗當年在西域與峨嵋派一場惡戰,損折極重,連掌旗使莊錚也死在滅絶師太的倚天劍下,其後痛定思痛,排了這個無堅不摧的陣勢出來。近年來明教聲勢大盛,五行旗各旗教衆相應擴充,鋭金旗下教衆已有四五千人。這五百名投槍擲斧之士,乃是十中選一的健者,武功本來已有相當根底,再在明師指點下練得年餘,已成爲一支可上戰陣,可作單鬥的勁旅。

群雄相顧失色,均想︰「明教楊左使這枝白色小旗擲向何處,這一千五百件兵刃便跟著投向何處。峨嵋派的霹靂雷火彈再厲害,傷人終究有限,擲出十枚,就算每一枚都打中,也不過傷得十人,如何是明教鋭金旗之比?」各人又想︰「倘若明教突然反臉,將咱們聚而殲之,那便如何?各門各派、各幫各會的好漢雖然人人武功甚強,却是一批烏合之衆,可不比明教的精鋭之師習練已久,指揮下得心應手。」群雄心下各自惴惴不安,竟然忘了對鋭金旗顯示的精妙功夫喝采。

楊逍喝道︰「鋭金旗退,巨木旗進!」鋭金旗五百名教衆抬起羽箭槍斧,奔到明教竹棚之前,躬身向張無忌行禮,隨即返身奔出廣場。楊逍一面青旗擲出,插在白旗之旁,廣場旁只聽得巨木砰彭撞擊,五百名巨木旗教衆青布包頭,每十個人抬一根巨木,互相碰撞,快步奔來。那些巨木裝有鐵鉤,每根巨木至少也有千斤之重,各人挽住一隻鉤鉤,脚下步子極是整齊。突然間一聲叱喝,五十根巨木抛擲出手,有的高,有的低,有的在左,有的在右,但每根巨木飛出,迎面必有一根巨木對準了撞到,五十根巨木分成二十五對,竟無一根落空。

但聽得砰砰砰砰巨響不絶,五十根巨木分成二十五對,相互衝撞。每根巨木都是重逾千斤,相互撞擊之下,聲勢實是驚人,若是青旗附近有人站著,不論縱高躍低,左閃右避,總是免不了被巨木撞到。巨木旗這一路陣法,原是從攻城戰中演化出來,攻城者抬了大木,衝擊城門,再堅固的城門也會被巨木撞開。血肉之軀在這許多大木衝撞之下,豈不立成了肉泥?巨木旗的五百名幫衆待木材撞後落地,搶上前去抓住木材上的鐵鉤,回身奔出,相距十餘丈之遙,只待發令者再度擲出青旗,又可二次抬木撞擊。楊逍喝道︰「巨木旗退,由木生火!」右手一揮,一面紅色小旗擲入廣場。

但見頭裹青巾的明教教衆退開,五百名頭裹紅巾的烈火旗教衆搶了進來,各人手持噴筒,一陣噴射,廣場中心全是黑黝黝的稠油,烈火旗掌旗使揮手擲出一枚硫磺火彈,石油遇火,登時烈焔衝天,燒了起來。要知明教總壇光明頂附近盛産石油,石中日夜不停有油噴出,遇火即燃。烈火旗人衆每人背負鐵箱,箱中盛滿石油,不論燒屋燒人,均是難以抵擋。

楊逍又道︰「烈火旗退,洪水旗揚威。」黑旗飛處,五百名頭裹黑巾的烈火旗下教衆搶進廣場。這洪水旗所擕傢生,又是與别旗大不相同,共是二十部水龍,放出二十條餓狼,張牙舞爪,在廣場上咆哮起來,便欲四散咬人,群雄大奇,心想這些惡狼跟「洪水」兩字有何干係?只聽得唐洋喝道︰「噴水!」一百名教衆手持陶瓷噴筒,一百股水箭向惡狼身上射了過去。説也奇怪,那二十頭惡狼一遇水箭,立時便倒,大聲悲{\upstsl{嗥}},片刻間皮破肉爛,變成一團團焦炭模樣。原來洪水旗所嘖水箭,乃是劇毒的酸質腐蝕藥水,從硫磺、硝石等類藥物中提煉製成。群雄見了這等驚心動魄之狀,不由得毛骨悚然,背上出了一身冷汗,均想︰「這些毒水倘若不是射向群狼,却是射在我的身上,那便如何?」只見洪水旗教衆提起二十部水龍上的龍頭,虛擬作勢,對著群狼,顯而易見,水龍中也是裝滿了毒水,若加發射,不但水盛,且可及遠。

楊逍喝道︰「洪水旗退,厚土旗收拾殘局。」一面小小黃旗揮出,不料這黃旗的旗桿底下裝著炸藥,拍一聲,炸了開來。只見一群頭裹黃巾的明教徒奔進廣場,各人背上負著一隻布囊,人數比金、木、水、火四旗少得多,只有百人。這百人剛圍成一個圏子,突然之間,轟的一聲大響,塵土飛揚,廣場中心陥落下去,露出一個徑長三四丈的大洞。跟著大洞四周泥土紛紛跳動,鑽出一個個頭戴鐵盆、手持鐵鏟的漢子來。四百條大漢驀地裡從地底鑽出,群雄都是大吃一驚,「啊」的一聲叫了起來。原來這四百教衆早就從遠處打了地道、鑽到廣場中心的地底,只待令到,便即破土而出。那大洞的地底也已挖掘一空,用木板木條撐住,號令到時,一抽木條,整塊地面便陥了下去。這一來,狼屍、毒水、石油、焦土等物一齊陥入地底。一百名教衆揮動鐵鏟,在大洞虛擊三下。群雄看得明白,若是有人跌入洞中,要待躍上,勢必被這一百柄鐵鏟擊了下去。跟著一包包石灰、鐵沙、石子倒入洞中,片刻間便將大洞和數百個小洞填平。四百柄鐵鏟此起彼落,好看已極。掌旗使一聲令下,五百教衆齊向張無忌行禮。那廣場中心填了鐵灰石灰,平滑如鏡,比先前更是堅硬了十倍。

群雄心中明白︰「倘若我站在廣場中心,口出侮慢明教之言,此刻只怕已被活埋在地底了。」這一來,明教五行旗大顯神威,小加操演,旁觀群雄無不駭然失色。

群雄人人心中明白,近年來明教在津泗豫鄂諸地造反,攻城略地,連敗元軍,現下他們是將兵法戰陣之學,用於武林豪士間的群毆,人數既衆,部勒又嚴,加之習練有素,自不是尋常江湖門派之所能匹敵。楊逍收兵以後,將插著小旗的木架交與身後童子,冷冷的瞧著周芷若,一言不發,但這無言之意却是十分清楚︰「你説要倚多爲勝,憑你峨嵋派百餘男女弟子,能與我明教數千之衆相抗麼?」

廣場上群雄各人想著各人的心事,一時間寂靜無嘩。過了好一會,空智身後一名達摩堂老僧站起來説道︰「適才明教操演行軍打仗的陣法,模樣倒是好看,但到底管不管用,能不能制勝克敵,咱們不是元帥將軍,學的也不是孫吳兵法,只怕誰也説不上來\dash{}」衆人均知他這幾句話是違心之論,只不過煞一煞明教的威風,將五行旗的厲害處輕輕一言帶過。周顚叫道︰「不知管不管用,那容易得很,少林寺派些大和尚出來試上一試,立見分曉。」那老僧置之不理,繼續説自己的話︰「咱們今日是天下英雄之會,藝高者勝。咱們講究的是單打獨鬥,説到倚多爲勝,武林中没聽説有這個規矩。」歐陽牧之道︰「倚多爲勝,武林中確是没這個規矩,然則霹靂雷火彈、毒火毒水這些玩意児,許不許用?」那老僧微一沉吟,道︰「下場比試的人用暗器,那是可以的。有些旁門左道之士,喜歡在暗器上加些毒藥毒水,那也無法禁止。但若旁人偸襲,那是壞了大會的規矩,大夥児須得群起而攻之。衆位意下如何?」群雄中一大半轟然叫好,都説該當如此。

崆峒派的唐文亮説道︰「在下另有一言,不論何人連勝兩陣之後,便須下場休息,以便快復内力元氣。否則車輪戰的幹將起來,任你通天本事,也不能一口氣從頭勝到尾。再者,各門各派各幫各會之中,如已有二人敗陣,不得再派人上場,否則的話,咱們這裡數千英雄,每個人都出手打上一架,只怕三個月也打不完,少林寺糧草再豐,可也給大夥児吃喝窮了,一百年元氣難復。」衆人轟笑聲中,均説這兩條規矩有理。要知唐文亮感激張無忌當年在光明頂上接骨之恩,有心盼他得勝,獨冠群雄,所以提出這兩條規矩,都是意在幫他節省力氣。彭瑩玉笑道︰「老唐三倒是識得大體,看來崆峒派今日幫咱們是幫定啦。咱們除了教主之外另由一位出陣?」明教衆高手誰都躍躍欲試,只是均知這一件事擔當極其重大,須得竭盡全力。先將與會的英雄打敗一大半,留給教主的強敵越少越好,他才能保留力氣,以竟全功。倘若只勝得寥寥數人,便被人打敗,留下一副重擔給教主獨挑,自己損折威名事小,負累了本教、謝遜、和教主却是事大。再者貿然請纓,不免自以爲除教主外本人武功最強,傷了同教間的義氣,是以誰都不敢出聲。周顚道︰「教主,我周顚不是怕死,只不過武功彀不上頂児尖児,出去徒然出醜。」

張無忌一個個瞧過去,心想︰「楊左使、范右使、韋蝠王、布袋師傅、鐵冠道長諸位各負絶藝,均可去得。其中范右使武學最博,不論對手是何家數,他都有取勝之道,還是請范右使出馬的爲是。」便道︰「本來各門兄弟任誰去都是一樣,但楊左使曾隨我攻打金剛伏魔圏,韋蝠王與布袋師傅曾生擒夏胄,都已出過力氣。這一次本座想請范右使出手。」范遙大喜,躬身道︰「遵命!多謝教主看重!」明教群豪素知范遙武功了得,均無異言。趙明却道︰「范大師,我求你一件事,你肯不肯答應?」范遙道︰「郡主但有所命,自當遵從。」群豪一齊望著趙明,不知她要説出什麼話來。

趙明道︰「少林派的空智大師與你的樑子未解,倘若你與他先鬥了上來,勝敗之數,未易逆料,縱然勝得了他,那也是筋疲力盡了。」范遙點了點頭,心知空智神僧成名數十年,看上去愁眉苦臉、一副短命夭折之相,其實内功外功倶臻上乘。趙明道︰「你不妨去和他訂個約會,言明日後再到大都萬法寺去單打獨鬥,一決勝敗。」楊逍和范遙齊聲説道︰「妙計,妙計!」均知空智和范遙一訂後約,今日便不能動手,趙明此計,實是給明教去了一個強敵。

其時各處木棚之中,各門派幫會的群雄正自交頭接耳,推舉本派出戰的人選,有幾處木棚中更有人大聲爭鬧,顯是對人選意見不一。范遙走到主棚之前站定,向著空智一抱拳,説道︰「空智大師,你有膽量没有?敢不敢再上大都萬法寺走一遭?」空智一聽到「萬法寺」三字,那是他生平的奇恥大辱,登時臉上的皺紋更加深了,細小的眼縫中神光湛湛,説道︰「幹什麼?」范遙道︰「咱二人在萬法寺結下怨仇,便當在萬法寺了結。你空智大師德高望重,在下也不免薄有虛名,若是你勝了我,江湖上便道強龍不壓地頭蛇,你大師只不過佔了地利之便。若是在下僥倖勝得一招半式,無知之輩加油添醬,只怕要説苦頭陀上得少林寺來,打敗了寺中第一高手。如果大師不怕觸景生情,今年八月中秋月明之夕,在下便在萬法寺塔上塔下,討教大師幾手絶藝。」

空智對范遙的武功也是頗爲忌憚,加之寺中方有大變,這時他心中盤算的只是如何應付這場千載浩劫,已無心緒與范遙動手,再被他這麼一激,登時點頭,當即説道︰「好,今年八月中秋,咱們在萬法寺相會,不見不散。」范遙抱拳施了一禮,便即退下,他走了七八步,只聽空智緩緩説道︰「范施主,今日你一心要救金毛獅王,不敢和我動手,是也不是?」范遙一凜,立定了足步,心想︰「這和尚畢竟識穿了我的心事。」他是個豪爽的漢子,不願虛假隱瞞,回頭哈哈一笑,説道︰「在下並無勝你把握。」空智微笑道︰「老衲也是並無勝得施主的把握。」須知武功到了上乘境界之人,相互間自然而然會生出英雄重英雄、好漢惜好漢的心情。空智見范遙漸漸走遠,不禁長嘆一聲。

廣場中人聲漸靜,那達摩堂的老僧朗聲説道︰「咱們便依衆英雄議定的規矩,起手比武。刀槍拳脚無眼,格殺不論,各安天命。最後那一個門派幫會武功最強,謝遜和屠龍刀都歸其所有。」無忌眉頭微皺,心想︰「這和尚生怕人家下手不重,唯恐各派怨仇結得不深,那裡是空見、空聞這些神僧們慈悲的心腸?」既是議定了每人勝得兩場,便須下來休息,先比遲比倒無多大分别,登時便有人出來叫陣,有人上前挑戰,片刻間場中有六個人分成三對較量。趙明自在萬法寺習得六大門派的絶藝後,根基雖然尚淺,識見却是不凡,站在無忌與范遙之間,低聲議論出戰六人的武功,預料誰勝誰敗,居然説得頭頭是道。只一盞茶時分,三對中已有兩對分了輸贏,只有一對尚在纏鬥,跟著又有人向勝者挑戰,仍是六人分成三對相鬥的局面。新上場的兩對分别動用了兵刃。如此上上落落,十之八九是有人流血受傷,方始分出勝敗。無忌心想︰「如此相鬥,武林中各派非失和不可,任何一派敗在對方手中,即使無人喪命受傷,以後仍會輾轉報復,豈非釀成自相殘殺的極大災禍?」只見場中崑崙派中一個中年道士以長劍刺傷了巨鯨幫的一條大漢,丐幫的執法長老則將華山派的矮老者一掌劈得口噴鮮血。

華山派的高老者見師兄受傷,破口大罵︰「臭叫化,爛叫化!」縱身出來,便欲向丐幫的執法長老挑戰。矮老者抓住他手臂,低聲道︰「師弟,你鬥他不過,暫且咽下了這口氣。」高老者怒道︰「鬥不過也要鬥!」他嘴裡雖這般説,其實他生平最是信從師兄的話,又知師兄的武藝與自己招數相同而修爲較深,師兄尚且敗陣,自己也是非輸不可。被矮老者拉著,不住口的亂罵,却回到了竹棚之中。

接著那執法長老又勝了「梅花刀」的掌門人,連勝兩陣,在丐幫幫衆如雷掌聲之中,得意洋洋的退回。

如此你來我往,廣場上比試了兩個時辰,紅日偏西,出來挑戰之人也是武功越來越強。許多人本來雄心勃勃,滿心要在這英雄大會中吐氣揚眉,人前逞威,但一見到旁人武功,才知自己原來不過是井底之蛙,不登泰山,不知天地之大,就此不敢出場。到得申牌時分丐幫的掌砵龍頭出場挑戰,將湘西排教中的彭四娘栽了一個大觔斗。彭四娘衣衫背心裂開了一條大縫,羞慚無地的退下。掌砵龍頭眼望峨嵋派人衆,冷笑道︰「女娘們能有什麼眞實本領?不是靠了刀劍之利,便得靠暗器古怪。這位彭四娘練到這等功夫,那也是極不容易的了。」周芷若低聲向宋青書説了幾句,宋青書點了點頭,緩步出場,向掌砵龍頭拱了拱手,道︰「龍頭大哥,我向你領教幾下高招。」

掌砵龍頭一見宋青書,氣得臉上發青,大聲道︰「姓宋的,你這奸賊奉了陳友諒之命,混入我丐幫來。害死我史幫主之事,只怕你這奸賊也有一份。今日你還有臉來見我麼?」宋青書冷笑道︰「江湖上混跡敵窩,刺探機密,乃是常事,只怪你們這群丐化子瞎了眼睛,識不出宋大爺的本來面目。」掌砵龍頭大罵︰「你連你親生老子的武當派也能背叛,什麼事做不出來?你對父不孝,將來對妻也必不義。峨嵋派非在你手中大大栽個觔斗不可。」宋青書怒得臉上無半點血色,道︰「你放屁放完了麼?」掌砵龍頭更不打話,呼的一掌便擊了過去。宋青書迴身卸開,反手輕輕一拂,以峨嵋派的「金頂綿掌」相抗。掌砵龍頭惱他混入丐幫,騙過衆人,手下招招殺著,狠辣異常,竟是性命相搏,並非尋常的比武較量。

他這一拚命,宋青書便落了下風。要知掌砵龍頭於加盟丐幫之前,已是江湖上成名的豪傑,他在丐幫中的地位僅次於幫主及傳功、執法二長老,掌底造詣大是不凡。宋青書是武當派第三代弟子中的佼佼人物,但初習峨嵋派的「金頂綿掌」,究是不甚純熟,這套掌法中的精微奥妙變化,無法施展出來。只見他鬥到四五十合之後,已是迭逢險招,這一遇險,自然而然以武當派的「綿掌」拆解。這是他自幼浸潤的武功,已練了二十餘年,得心應手,威力甚強,與「金頂綿掌」外表上有些彷彿,運勁拆招的法門却是大不相同。

旁人不明就理,還道他漸漸挽回頹勢,殷利亨却是越看越怒,叫道︰「宋青書,你這小子好不要臉!你反出武當,如何還用武當派的功夫救命?你不要你爹爹,怎地却要你爹爹所傳的武功?」宋青書臉上一紅,叫道︰「武當派的功夫有什麼希罕?你看清楚了!」左手突然在掌砵龍頭眼前上圏下鉤、左旋右轉,連變了七八種花樣,突然間右手一伸,{\upstsl{噗}}的一響,五根手指直插入掌砵龍頭的腦門。旁觀群雄一怔之間,只見他五根手指血淋淋的提了起來,掌砵龍頭翻身栽倒,立時氣絶。

宋青書冷笑道︰「武當派有這等功夫麼?」群雄驚叫聲中,丐幫中搶上了七八人,有的扶起掌砵龍頭的屍身,其餘的便向宋青書攻去。

圍攻宋青書的六人,均是丐幫中七袋弟子,以上的高手,其中四人,各使兵刃,霎時之間宋青書便險象環生。空智大師身後一名胖大和尚高聲喝道︰「丐幫諸君倚多爲勝,這不是壞了今日英雄大會的規矩麼?」他這兩句話聲音響亮異常,震得各人耳鼓{\upstsl{嗡}}{\upstsl{嗡}}作響。執法長老叫道︰「衆人且退,讓本座替掌砵龍頭報仇。」丐幫群弟子向後躍開,抱著掌砵龍頭的屍身,退歸竹棚,人人滿臉憤容,向宋青書怒目而視。旁觀群雄均想︰「雖説比武較量之際格殺不論,但這姓宋的出手也忒煞毒辣了些。」

張無忌立時所想到的,只是趙明肩頭的五個爪印,以及那晩暗裡茅舍中一對老夫婦屍橫就地的可怖情景,顫聲道︰「楊左使,峨嵋派中何以有這種邪惡的武功?」楊逍搖頭道︰「屬下從没見過這等功夫。但峨嵋派祖師郭女俠外號叫作『小東邪』,武功若是帶著三分邪氣,却也不奇。」

二人説話之間,宋青書已和執法長老鬥在一起。這位執法長老身形瘦小,行動快捷無論,十根手指如鉤錐,竟是以鷹爪功與宋青書對攻,顯然他也是擅長指功,要用手指在宋青書天靈蓋上也戳五個窟窿,好替掌砵龍頭報仇,宋青書初仍以「金頂綿掌」功夫和他拆解十餘招,鬥到深澗,執法長老喝一聲︰「小狗賊!」左手五指已搭上了宋青書腦門,正要透勁而入,宋青書右手已一探,{\upstsl{噗}}的一聲響,五根手指已抓斷了他的喉管。執法長老身子向前撲倒,左手上勁力未衰,插入了地下,血流滿地,登時氣絶。

周芷若這一次却佔了先機,做個手勢,八名弟子各持長劍,縱身而出,每兩個弟子背脊靠背脊,分佔四個方位,將宋青書圍在中間,丐幫若再上動手,立時便是群毆的局面。一名達摩堂的老僧朗聲道︰「羅漢堂下的三十六弟子聽令!」他手掌互擊三下,三十六名身披黃袍的少林僧躍將出來,十八名手執禪杖,十八名手執戒刀,前前後後,散在廣場各處,似陣法又不似陣法,却是守住了各個扼要處所。那老僧説道︰「奉空智師叔法旨,羅漢堂三十六弟子監守大會的規矩。今日大會中比武較量,若是有人恃衆欺寡,或是在旁暗助,便是天下武林的公敵。我少林寺忝爲主人,須當維繫公道。三十六弟子嚴加査察,不論何人犯規,當場便予格殺,決不容情。」三十六名弟子轟然答應,虎視耽耽的望著廣場中心。這麼一來,峨嵋派防護在先,少林派監視於旁,丐幫衆弟子雖然群情悲憤,却也不敢貿然上前動手,只是高聲怒罵,將執法長老的屍身抬了下來。

趙明向范遙低聲道︰「苦大師,没想到峨嵋派尚有這一手絶招,那日在萬法中,滅絶師太寧死不肯出塔比武,只怕就是爲此。」范遙搖了搖頭,心下苦思拆解這一招的法子。他呆了半晌,突然向張無忌道︰「屬下向你請教一路武功。」雙掌按在桌上,伸出左手一根小指,右手一根小指、一前一後,靈活無比的連續動了七下,低聲道︰「我雙臂如此連攻,只須纏到了這小子的手臂,内力運出,便能震斷他的手臂関節,他指力再厲害,也教他無所施其技。」張無忌也伸出兩根手指,左鉤右搭,道︰「小心他以指力戳你手臂。」范遙點頭稱是,道︰「我以擒拿手抓他手腕,十八路鴛鴦連環踢他下盤。」無忌道︰「猛攻八十一招,叫他無法喘息。」

只見他二人四根手指此進彼退,快速無倫的攻拒來去,范遙忽然微笑道︰「教主這幾下太過神妙,這小子除了指力外武功有限,這幾招料他施展不出。」張無忌微微一笑,道︰「他施展不出這三招,那麼范右使你已然勝了。」

\chapter{變幻萬端}

張無忌左手的手指轉了兩個圏,右手的手指突然從圏中穿出,鉤住了范遙的手指,微笑不語。范遙一怔之下,大喜道︰「多謝教主指點,屬下佩服得緊。這四招匪夷所思,大開眼屬下茅塞,我眞恨不得拜你爲師纔好。」張無忌道︰「這是我太師父所傳太極拳法中的『亂環訣』,要旨是左手所劃的幾個圓圏。這姓宋的雖然出自武當,料他未能悟到這些精微之處。」范遙成竹在胸,已有制勝宋青書的把握,只是宋青書連勝兩場,按規矩應當退場休息,須得待他再度出場,然後上前挑戰。

趙明微微一笑,神情甚是愉悦,走到了一旁。無忌走到她的身邉,低聲道︰「明妹,什麼事這生喜歡?」趙明玉頰暈紅,低下了頭,道︰「你傳授范左使這幾招武功,只是讓他震斷了宋青書的手臂,何以不教他取了那姓宋的性命?」無忌道︰「宋青書雖是多行不義,終究是我大師伯的獨生愛児,該當由我大師伯自行清理門才是。我若叫范右使取了他性命,可對不起大師伯。」趙明笑道︰「你殺了他,周家姊姊成了寡婦,你重收覆水,豈不甚佳?」無忌握住了她的手,笑道︰「你許不許我?」趙明道︰「我是求之不得,等你三心兩意之時,好讓她用手指在你胸口戳上五個窟窿。」

當無忌與范遙拆招、與趙明説笑之際,宋青書已在峨嵋八女的衛護之下,退回竹棚。群雄見到適纔宋青書殺人這驚心動魄的兩幕,誰都不禁心寒,各人靜以觀變,不願出來以身犯險。過了片刻,宋青書又飄然出場,抱拳道︰「在下休息已畢,更有那一位英雄賜教。」范遙叫道︰「讓我領教峨嵋派的絶學。」正要縱身而出,突然一個灰影一晃,站在宋青書之前,向范遙道︰「范先生,請讓我一讓。」只見此人氣度凝重,雙足不丁不八的站著,抱元守一,正是武當二俠兪蓮舟。宋青書從小就怕這位師叔,但見他屏息運氣,嚴陣臨敵,知道今日之事,已不再是武當山上的授藝拆招,而是生死相撲,雖然他另行學得了奇門武功,終究不免膽怯。兪蓮舟抱拳道︰「宋少俠請!」這一行禮,口中又如此稱呼,那是明明白白的顯示,他對宋青書不敢有絲毫輕視,却也已無半分香火之情。宋青書一言不發,躬身行了一禮。兪蓮舟呼的一掌,便迎面劈了過來。

兪蓮舟成名三十餘年,但武林親眼見過他一顯身手的,却是寥寥無幾。江湖上素知武當派武功注重以柔克剛,招式緩慢而變化精微,豈知兪蓮舟雙掌如風,招式奇快,頃刻間宋青書腰間分别中了一腿一掌。宋青書大駭︰「太師父和爹爹均是要我做武當第三代掌門,決不致有什麼武功祕而不授。兪二叔這套快拳快腿,招式雖是武當一路,但變化如此之快,顯是犯了本門功夫之大忌,偏生又這等厲害!」待要施展周芷若所教他的指上功夫,却被兪蓮舟逼得氣也喘不過來,當下只得連連倒退,竭力守住門戸。

群雄全神貫注的瞧著二人相鬥,眼下雖是兪蓮舟佔著上風,然適纔宋青書抓殺丐幫二老,均是反敗爲勝,從劣勢中突出殺著,此事未必不能重演。却見兪蓮舟越打越快,可是一招一式却是清清楚楚,便如擅於唱曲的名家,雖是唱到了極快之處,但板眼吐字,仍無半點模糊。群雄均站了起來,有些站在後面的,索性登上桌椅,心下無不讚嘆︰「武當二俠名不虛傳,這一口氣不停的急攻,招式上竟無重複之處。」虧得宋青書是武當嫡傳弟子,對兪蓮舟拳脚中精微的變化都曾學過,只是如此快鬥,却是生平第一遭。廣場上黃塵飛揚,化成一團濃霧,將兪宋二人裹住。

二人本是近身而鬥,猛聽得拍的一聲響,雙掌相交,兪蓮舟與宋青書一齊向後躍開,兩團黃霧分了開來。兪蓮舟尚未站定,復又猱身而前。殷利亨掛懷師兄安危,不自禁的走到場邉,手按劍柄,目不轉睛的望著場中。這相鬥的二人都是武當高手,宋青書生死繫於一線,全力相拚,早已顧不得門派之别,所使的全是自幼練起的武當功夫。二人的拳脚招式,殷利亨盡數了然於胸,知道每一招均是致命的殺著,情切関心,比之旁人更是緊張了幾倍。好在見兪蓮舟越打越佔上風,若非防宋青書突出五指穿洞的陰毒殺手,處處預留地步,早已將他斃於掌底。張無忌也頗擔心,手中暗持兩枚聖火令,倘若兪蓮舟眞有性命之憂,那也顧不得大會規矩,非出手相救不可。

但見塵沙越揚越高,宋青書突然左手五指箕張,向兪蓮舟右肩抓了過來。兪蓮舟在百招之前便在等他施展這一手。要知宋青書抓斃丐幫二老,出手的情景被兪蓮舟瞧得清清楚楚,倘若事先並無二老遭殃,他突然間首次遇到這種陰損之極的殺手,就算不死,也得重傷,既是見識在先,心中已然盤算好應付之。宋青書練此爪法未久,變化不多,此時再抓,與起先兩下仍是大同小異。兪蓮舟右肩斜閃,左手憑空亂劃了幾個圏子。趙明與范遙忍不住齊聲「噫」的一下驚呼,原來兪蓮舟這兩下圏子,正是張無忌指點范遙的太極拳「亂環訣」。趙明與范遙一見之下,便知宋青書要糟,果然「噫」聲未畢,宋青書右手五指抓向兪蓮舟咽喉。無忌大怒,低罵︰「該死,該死!」丐幫執法長老便是命喪於這一抓之下,宋青書對師叔居然也下此毒手。但見兪蓮舟雙臂一圏一轉,使出六合勁中的鑽翻螺旋二勁,已將宋青書雙臂圏住,格格兩響,宋青書雙臂骨節寸斷。兪蓮舟喝道︰「今日替七弟報仇!」兩手一合,一招「雙風貫耳」,雙拳擊在他的左右兩耳。這一招綿勁中蓄,宋青書立時頭骨碎裂。

他身子尚未跌倒,兪蓮舟正待補上一脚,當場送了他的性命,驀地裡青影閃動,一條長鞭迎面擊來。兪蓮舟急忙後躍避過,那長鞭快速無倫的連連進招,正是峨嵋派掌門周芷若爲夫復仇來了。

兪蓮舟連退三步,那知周芷若鞭法奇幻,三招間便已將他圏住,忽地軟鞭一抖,收了回來,左手抓住鞭梢,冷冷的道︰「此時取你性命,諒你不服。取兵刃來!」殷利亨刷的一聲拔出長劍,上前説道︰「我來接姑娘的高招。」周芷若冷冷的瞪了他一眼,轉身去看宋青書傷勢,只見他雙目突出,七孔流血,軟癱在地,眼見性命不保。峨嵋派搶上三名男弟子,將他抬了下去。周芷若回過頭來,指著兪蓮舟道︰「先殺了你,再殺姓殷的不遲。」兪蓮舟適纔竭盡全力,竟是無法從她的鞭圏中脱出,心中好生駭異,他愛護師弟,心想︰「我跟她鬥上一場,就算死在她的鞭下,六弟至少可瞧出她鞭法的端倪。他死裡逃生,便多了幾分指望。」回手去接殷利亨手中的長劍。殷利亨也瞧出局勢兇險無比,憑著師兄弟二人的武功,想逃出她長鞭的一擊,機會極是渺茫,他和師兄是同樣的心思,寧可自身先攖其鋒,好讓師兄察看她鞭法的要旨,當下不肯遞劍,説道︰「師哥,我先上場。」

兪蓮舟向他望了一眼,數十載同門學藝、親如手足的情誼,猛地裡湧上心頭,心念猶似電閃,想起兪岱岩殘廢,張翠山自殺,莫聲谷慘死,武當七俠只剩其四,今日看來又有二俠畢命於此,這般六弟武功雖強,感情極是軟弱,倘若自己先死,他心神大亂,保不定要在群俠之前出醜,損了本派的顏面。

兪蓮舟尋思︰「若我先死,六弟萬難跟我報仇,他也決計不肯偸生逃命,勢必是師兄弟二人同時畢命於斯,於事無補。若他先死,我瞧出這女子鞭法中的精義,或能跟她拚個同歸於盡。」當下點頭,道︰「六弟多支持一刻好一刻。」殷利亨想起妻子楊不悔已有身孕,不由自主向楊逍與張無忌這邉望去,轉念又想︰「我死之後,不悔與孩児總會有人照料,何必婆婆媽媽的去囑咐求人!於是長劍一舉,目視劍尖,心無旁騖,跟著含胸拔背、沉肩{\upstsl{墬}}肘,説道︰「掌門人請賜招!」他年紀雖比周芷若大得多,但周芷若此刻是峨嵋派掌門,他絲毫没缺了禮數。兪蓮舟見他以「太極劍」起手式應敵,知道六弟這次是以師門絶學與強敵周旋,便緩緩向後退開。

周芷若道︰「你進招吧!」殷利亨心想對方出手如電,若是被她一佔先機,極難平反,當下左足踏上,劍交左手,一招「三環套月」他第一劍便是虛虛實實,以左手劍攻敵,劍尖上光芒閃爍,嗤嗤發出輕微響聲。旁觀群雄忍不住震天{\upstsl{吆}}喝了聲彩。周芷若斜身閃開,殷利亨跟著便是「大魁星」,「燕子抄水」,長劍在空中劃成大圓,右手劍訣戳出,竟似也發出嗤嗤微聲。周芷若纖腰輕擺,一一避過,説道︰「殷六俠,我讓你三招,以報昔日武當山上故人之情。」這「情」字一出口,軟鞭便如靈蛇顫動,直奔殷利亨胸口,殷利亨奔身向左,那軟鞭竟被半路中彎了過來。殷利亨一招「風擺荷葉」,是劍削了出去,鞭劍相交,輕輕擦的一響,殷利亨只覺虎口發熱,長劍險些児脱出手去,心中吃了一驚︰「我只道她招式怪異,内力非我之敵,不料她内勁也是奇詭莫測。」當下顧不得自身生死安危,將一套太極劍法使得圓轉如意,嚴密異常的守住門戸。

周芷若手中的軟鞭猶似一條柔絲,竟如没半分重量。身子忽東忽西,忽進忽退,在殷利亨身周飄蕩不定。張無忌越看越奇,心想︰「她手中軟鞭運用,比之渡厄、渡難、渡劫三高僧,又是截然不同。」他初時只道峨嵋派中另有邪門武功,但此刻看了她如鬼魅的身手,與滅絶師太實是大異奇趣,心下隱隱竟有層恐懼之感。范遙忽道︰「她是鬼,不是人!」這句話正説中張無忌的心事,不禁身子一顫,倘若不是在廣場上陽光耀眼,四周站滿了人,眞要疑心周芷若已死,鬼魂持鞭與殷利亨相鬥,他見識過無數門派的怪異武功,但像周芷若這般如風吹柳絮,如水送浮萍,實非人間氣象,不由得想︰「難道她當眞有妖法不成?還是什麼怪物附體?」

周芷若武功精奇,然殷利亨這套太極劍法,乃張三丰晩年繼太極拳所創,可説是近世登峰造極的劍法,殷利亨功勁一加運開,綿綿不絶,雖是傷不了周芷若,但只求自保,却也是絶無破綻。只不過人人都已看了出來,殷利亨已然輸定,所差者只是他活著敗陣,還是死著敗陣。

忽聽得一人怪聲怪氣的叫道︰「啊喲,宋青書快斷氣啦,周大掌門,你不給老公送終,做寡婦也不光彩哪!」衆人往聲音來處望去,原來却是周顚。知道武當派弟子生平最注重養氣調息,一到上陣交鋒,個個有「泰山崩於前而色不變、麋鹿興於左而目不瞬」的修爲,是以有意助殷利亨一臂之力,擾亂周芷若的心神。他見周芷若並不回頭,手下也不加快,於是又叫道︰「喂喂,峨嵋派的周芷若姑娘,你老公要噎氣啦,有幾句話吩咐你,他説他在外面有三七二十一個私生子,他死了之後,要你好好給他撫養,免得他死不瞑目。你到底答應呢還是不答應?」

群雄聽他胡説八道,有的忍不住便笑出聲來,周芷若却是便如没有聽見。周顚又叫道︰「啊喲,乖乖不得了!滅絶師太,近來你老人家身子好啊。多日不見,你老人家越來越硬朗了。」突然之間,周芷若身形一閃一晃,倒退了數丈,長鞭從右肩一甩向後,鞭頭向周顚面門。她本來與周顚相隔數丈,但軟鞭説到便到,正如天外遊龍,夭矯而至。周顚正自口沫橫飛的説得高興,那料得到周芷若在惡鬥之中,竟會突然舉鞭襲擊。他一呆之下,長鞭已打到面前,周芷若並不回身,然而背後竟似生了眼睛一般,鞭梢直指他的鼻尖。

周芷若長鞭一甩出,左手食中二指向殷利亨連連戳去,一連七指,全是對向他頭臉與前胸的重穴。殷利亨不及攻敵,要待圏轉長劍削她手臂,時間上也有分刻之差,只得使招「鳳點頭」,矮身避開。其時明教竹棚中拍的一聲,跟著嗆{\upstsl{啷}}{\upstsl{啷}}一陣亂響,原來楊逍站在周顚近旁,眼明手快,一掌拍起身前的木桌,擋了周芷若一鞭。木桌被長鞭一擊,登時木屑橫飛,桌上的茶壼、茶碗也是四下亂擲,各人身上濺了不少瓷片熱茶。

周芷若一擊不中,不再理睬周顚,一條軟鞭疾風暴雨般向殷利亨攻擊。兪蓮舟在旁看了半晌,始終無法捉摸到周芷若鞭法的精要所在,暗想︰「若是我上前相鬥,這套太極劍法也無法使得比六弟更好。若是鬥得久,她女子内力不足,咱們或能以韌力取勝。」他見殷利亨劍法中吞吐開合、陰陽動靜,實已到了恩師張三丰平時所指點的絶詣,心想師弟一生中從未施展過如此高明的劍術,今日面臨生死関頭,竟將劍法中最精要之處都發揮了出來,咱們武當門的武功講究愈戰愈強,時候拖得越久,越有不敗之望。

不料他臉上忽憂忽喜的神情,都給周芷若看在眼裡,她朗聲叱道︰「兪二叔,你喜歡什麼?殷六叔爲人好,我才容他鬥到二百招後纔取他性命,以免他一世英名,付於流水。待會你上來啊,三十招内我便叫你血濺黃沙。瞧仔細了!」突然間長鞭抖動,繞成一個個大大小小的圏子,將殷利亨裹在其中。太極拳和太極劍都講究運勁成圏,周芷若所劃的許多圓圏,方向與殷利亨的劍圏相同,只是却快了十倍二十倍。殷利亨劍上勁力被她這麼一帶,登時身不由主,連轉了幾個身,青光一閃,長劍脱手上揚。

一條長鞭如蟒蛇般捲了攏來,鞭頭對準殷利亨天靈蓋上{\upstsl{砸}}了下去。兪蓮舟縱身而起,捨命去抓軟鞭的鞭梢。周芷若裙底飛出一腿,正中兪蓮舟腰脅。便在這千鈞一髮之際,一人從旁搶至,猿臂一伸,那長鞭呼呼數響,都纏上他的手臂,正是張無忌出手救人,以乾坤大挪移心法,轉移長鞭的勁道。周芷若變招奇速,右手放開鞭梢柄,雙掌拼力,向張無忌胸前擊到。無忌若是一卸勁,這雙掌之力剛好擊正殷利亨臉盤。他右手被軟鞭纏住,未能掙脱,只得左手一掌拍出,以硬拚硬。

不料二人三掌相接,無忌猛覺周芷若雙掌中竟無半分勁力,心下大駭︰「啊喲不好!她與六叔苦鬥二百餘招,以一年輕弱女,和六叔這武當名手比勁較力,竟已到了油盡燈乾的境地,我這勁力往前一送,豈非當場要了她的性命。」他與周芷若恩怨糾纏,究竟舊情不絶,危急中忙收手勁。他初時一掌拍出,知道周芷若此時武功與自己已然相差不遠,可説是生平從所未逢的強敵,心中早是絲毫不敢怠忽,加之單掌迎雙掌,這一掌竟是出了十成力。這十成力道剛向外吐,便即察覺對方力盡,急忙硬生生的收回,這原是犯了武學的大忌,等於以十成掌力回擊自身。

何況在這間不容髮之際突然回收,用力更是奇猛,但張無忌此時於自己内勁收發由心,這股強力回撞,是一時氣窒,絶無大礙,不料他掌上這十成力剛一回收,突覺對方掌力猶似洪水決堤,勢不可當的衝了過來。無忌大吃一驚,知道已中周芷若暗算,胸口砰的一聲,已被周芷若雙掌擊中。那是他自己的掌力再加上周芷若的掌力,並世兩大高手合擊之下,無忌護體的九陽神功雖然雄渾,却也抵擋不住。何況周芷若這兩掌之力,乃是乘隙而進,正是無忌舊力已盡、新力未生的時候擊至。張無忌身子向後一仰,眼前一黑,一口鮮血噴了出來。周芷若情知若憑眞實功夫,自己不是無忌對手,一招偸襲成功,左手跟著前探,五指便抓向無忌的胸口。

無忌雖是身受重傷,心神未亂,眼見這一抓到來,立時便是開膛破胸之禍,勉強向後移了數寸。嗤的一響,周芷若五指抓破了他胸口衣衫,露出前胸肌膚。

周芷若右手五指跟著便要進襲,那時兪蓮舟被她一腿踢倒,正中穴道,動彈不得,殷利亨撲上想要救援,也已不及,眼見無忌難逃此劫,周芷若一瞥之下,忽然見到無忌胸口露出一個傷疤。她心念一動,想起那是昔日光明頂上自己用倚天劍刺傷於他,突然間天良發現,右手五指距他胸膛不到半尺,竟是抓不下去,心想︰「那時我奉師命刺他,他毫不避讓。今日他也是爲了不肯傷我,這纔容我得手,難道我竟下手殺了他麼?」

她稍一遲疑,韋一笑、殷利亨、楊逍、范遙四人同時撲到。韋一笑在無忌身前一擋。楊范二人分襲周芷若左右,殷利亨已抱著無忌逃了開去。這時場中一陣大亂,峨嵋派群弟子和少林僧紛紛呼喝,手執兵刃,趕了上來。楊逍與范遙見無忌已然脱身,與周芷若拆得數招,便不再戀戰,韋一笑扶起兪蓮舟,一齊回到竹棚之中。峨嵋、少林兩派人衆見場中罷鬥,也便退開。趙明本也搶上救援,只是身法不及韋楊諸人迅速,中途遇上,但見無忌嘴邉都是鮮血,只嚇得臉如白紙。無忌強笑道︰「不礙事,運運氣便好。」衆人扶著他在竹棚中地下坐定,無忌緩引九陽神功,調理内傷。

周芷若叫道︰「那一位英雄前來賜教?」范遙束了束腰帶,大踏步走出。無忌道︰「范右使,我下令,你不可出戰,咱們\dash{}咱們認輸\dash{}」一口氣岔了道,又是兩口鮮血噴出。范遙對教主之令不敢不從,倘若堅持出戰,勢必引得無忌傷勢加劇,何況出戰只是盡心竭力,枉自送了性命,却於本教無補。

周芷若站在廣場中心,又説了兩遍。適纔張無忌迴力自傷,只有他與周芷若二人方才明白,旁人都以爲周芷若掌力怪異,無忌力所不敵,而周芷若不下殺手,饒了無忌性命,却是人所共見。她以一個年輕姑娘,連敗殷利亨、兪蓮舟三位一等一的高手,武功之奇,實是匪夷所思。群雄中雖有不少身負絶學之士,但自忖決計比不上兪、殷、張三人,以他三人尚且不敵,旁人更加不必上去送命。周芷若站在場中,山風吹動衫裙,飄飄欲仙,原是個嬌柔無力的弱女,但周圍來自三山五嶽、四面八方的數千英雄好漢,竟無一人敢再上前挑戰。

周芷若又待片刻,仍是無人上前。那達摩堂的老僧走了出來,合什説道︰「峨嵋派掌門人宋夫人技冠群雄,武功爲天下第一。有那一位英雄不服?」他連問三聲,周顚噓了三次,却無人正式不服。那老僧道︰「既是如此,咱們便依英雄大會事先議定,金毛獅王謝遜由峨嵋派宋夫人處置。屠龍寶刀在何人手中,也請一併交出,由宋夫人收管。這是群雄公決,任誰不得異言。」

張無忌正在調勻内息,鼓動九陽眞氣,治療重傷,漸漸入於返虛空明的境界,猛聽得那老僧説到「金毛獅王謝遜交由峨嵋派掌門人宋夫人處置」這句話,心頭一震,險險又是一口血噴將出來。趙明坐在一旁,全神貫注的照料,見他突然身子發抖,臉色大變,明白他的心意,便柔聲説道︰「無忌哥哥,義父若是由周姊姊處置,那是最好不過。她適纔不忍心下手害你,可見對你仍是情意深重。她既盼你重捨舊好,決不能害了義父,你儘管放心療傷便是。」無忌一想不錯,心頭大寬。眼見太陽正從山後下去,廣場上漸漸黑了下來。

只聽那老僧又道︰「金毛獅王謝遜,囚於山後某地。今日天時已晩,各位想必餓了。明日中午,咱們仍舊聚此地,由老衲引導宋夫人前去開関釋囚。那時咱們再見識宋夫人並世無雙的武功。」張無忌、楊逍、范遙等都向趙明望了一眼,心中都道︰「果然是你料得不錯。少林派暗中另有陰謀。周芷若武功再強,却也不能打敗渡厄等三位高僧,只怕她非送命在小山峰上不可,仍由少林稱雄逞強。」這時周芷若已奔回竹棚,察看宋青書的傷勢。群雄見周芷若雖是奪得了「武功天下第一」的名頭,此間大事却未了結,心中各有各的打算,誰也不下山去。那老僧又道︰「各位英雄聽著︰各位來到本寺,均是少林的嘉賓,各位相互間若有恩怨糾葛,務請瞧在敝派薄面,暫忍一時,請勿在少室山上了結,否則便是瞧不起少林派了。各位用過晩飯以後,前山各處,儘可隨意遊覽。後山是敝派藏經授藝之所,請各位自重留步。」

當下范遙抱起張無忌,回到明教自搭的茅棚之中。無忌所受掌傷雖重,但服了九粒他平時煉製的靈丹,再以九陽眞氣輸導藥力,到得深夜二更時分,吐出三口瘀血,内傷盡去。楊逍、范遙、兪蓮舟、殷利亨等均是又驚又喜,均讚他内功修爲實是罕見罕聞,常人受了這等重傷,縱有高手調治,至少也得將養一兩個月,方能去瘀順氣,他却能在幾個時辰内痊可,若非親見,當眞難信。

無忌吃了兩碗飯,將養片刻,站起身來,説道︰「我出去一會児。」他是教主之尊,既不説什麼事,旁人自也不便相詢。殷利亨道︰「你重傷剛癒,一切小心。」無忌應道︰「是!」見趙明臉上神色極是関懷,向她微微一笑,意思説︰「你放心吧!」眞氣流轉,精神爲之一振,逕到少林寺外,向知客僧人説道︰「在下有事要見峨嵋掌門,相煩引路。」那知客僧見是明教教主,心下甚是害怕,忙恭恭敬敬的道︰「是,是!小僧引路,張教主請這邉來。」引著他向西走去,約莫行了里許,指著幾間小屋,道︰「峨嵋派都住在那邉,僧尼有别,小僧不便深夜近前。」其實他是深怕張無忌又去和周芷若動手,這當世兩大高手厮拚起來,自己一個不巧,便受了池魚之殃。無忌笑道︰「你若是回去説起此事,驚動旁人,我不如點了你的穴道,在此等我半夜如何?」那知客僧忙道︰「小僧決不敢説,教主放心。」急急忙忙的轉身便去。

張無忌緩步走到小屋之前,相距十餘丈,黑暗中便見兩名女尼飛身過來,手執長劍,攔在身前,叱道︰「夤夜之中,何人駕到?」無忌抱拳道︰「明教張無忌,求見貴派掌門宋夫人。」那兩名女尼一見是張無忌到了,都是大吃了一驚,一名年長的女尼道︰「張\dash{}張教主\dash{}請暫候,我\dash{}我去稟報。」她雖強自鎭定,但聲音發顫,轉身没走了幾步,便摸出竹哨吹了起來。

峨嵋派今日吐氣揚眉,在天下群雄之前,掌門人力敗當世三位高手,嚇得數千鬚眉男子無一敢上前挑戰,眞是開派以來從所未有之盛事。但峨嵋派今日殺丐幫二老、敗武當二俠、傷明教教主,得罪的人著實不少,何況周芷若號稱武功第一,不知有多少英雄惱恨妬忌,這一晩身處險地,強敵環伺之下,戒備得十分嚴密。那女尼哨子一響,四周立時撲出二十餘人,黑夜中劍光閃動,分佈各處。張無忌也不理會,雙手負背後,靜立當地。

那女尼進了小屋稟報,過了片刻,便即回身出來,説道︰「敝掌門人言道,男女有别,晩間不便相見。請張教主迴步。」張無忌道︰「在下頗通醫術,願爲宋青書少俠療傷,别無他意。」那女尼一怔,又進去稟報,隔了良久,這纔出來,説道︰「掌門人有請。」張無忌拍了拍腰間,顯示未擕帶兵刃,隨著那女尼走進小屋。祇見周芷若坐在一旁,以手支頤,怔怔出神,聽得張無忌進來,竟不回頭,那女尼斟了一杯清茶放在桌上,便退了出去,輕輕帶上了門,廳上更無旁人。一枝白燭忽明忽暗,照著周芷若一身素淡的青衣,景情甚是淒涼。張無忌心中一酸,低聲道︰「宋師哥傷勢如何?待我瞧瞧他去。」

周芷若仍是並不回頭,冷冷的道︰「他頭骨震碎,傷勢極重,多半不能活了。不知能不能過今晩。」無忌道︰「你知我醫術不壞,願盡力施救。」周芷若問道︰「你爲什麼要救他?」無忌一怔道︰「我對你不起,心下萬分抱愧,何況今日你手底留情,饒了我性命。宋師哥受傷,我自當竭力。」周芷若道︰「你手底留情在先,我豈有不知?你若能救活宋大哥,要我如何報答?」張無忌道︰「一命換一命。」周芷若向内堂指了指,道︰「他在裡面。」無忌走向房門一張,只見房内黑漆一團,並無燈光,於是拿起燭台,走了進去。周芷若始終一手支頤,坐在桌旁,身子全不動彈。

無忌揭開青妙帳子,燭光下只見宋青書突出,五官歪曲,容顏十分可怕,呼吸微弱,早已人事不知。無忌按了按他的手腕,但覺脈息混亂,忽快忽慢,肌膚冰冷,若不立即施救,果然是難以挨過當晩,再輕摸他的頭骨,察覺前額與後腦骨共有四塊碎裂。要知兪蓮舟雙之力何等厲害,這一招「雙風貫耳」運上了十成内勁,若不是宋青書内功也有相當根底,當場便已斃命。無忌放下帳子,將燭台放在桌上,坐在竹椅上,凝思治療的法子。他自得蝶谷醫仙胡青牛的傳授,醫術之精,當世已無其匹,但宋青書受的實是致命重傷,要救他性命,最多只有一成把握。

他細細思量了一頓飯時分,走到外室,説道︰「宋夫人,能否救得宋師哥的性命,我殊難斷言,是否容我一試?」周芷若道︰「倘若你救他不得,世間也無第二人能彀。」張無忌道︰「縱然救得他性命,但容貌武功,只怕難復舊觀。」周芷若道︰「你究竟不是神仙。我知道你會盡心竭力,救活了他,以便自己問心無愧的去做朝廷郡馬。」張無忌心頭一震,自問並無此意,但此事也不便置辯,當下又回到房中,揭開宋青書身上蓋的薄被,點了他的八處穴道,十指輕柔,以一股若有若無之力,將他碎裂的頭骨一扶正。然後從懷中取出一雙金盒,以小指挑了一團黑色藥膏,雙手搓得勻淨,輕輕塗在宋青書頭骨碎處。

這黑色藥膏便是「黑玉斷續膏」,乃是西域少林寺療傷接骨的無上聖藥。當年張無忌向趙明乞得,用以接續兪岱岩與殷利亨二人的四肢斷骨,尚有剩餘。他掌内九陽眞氣源源送出,將藥力透入宋青書各處斷骨。

\chapter{黃衫女子}

無忌送完藥力,見宋青書頭臉上無甚變化,心下甚喜,知道救活他性命的把握又多了幾成。他自己重傷初癒,這麼一運内勁,不由得又感心跳氣喘,站在床前調勻内息半晌,這纔回到外房,將燭台放在桌上。燭光映處,見周芷若臉色蒼白異常,隱隱聽得屋外輕輕的脚步之聲,知是峨嵋派群弟子正在巡邏守衛,便道︰「宋師哥的傷或能治癒,你可放心。」周芷若道︰「你没救他的把握,我也没救謝大俠的把握。」

無忌心想︰「明日她要去攻打金剛伏魔圏,峨嵋派中縱有一二高手相助,十九也難成事,説不定反而送了她的性命。」説道︰「你可知我義父囚禁之處的情形麼?」周芷若道︰「不知。少林派設下什麼厲害的埋伏?」無忌於是將謝遜如何被囚入山頂地牢、少林三老僧如何守禦、自己如何兩度攻打均告失敗的經過説了一遍。周芷若默默聽完,道︰「這等説來,你既破不了,我是更加無濟於事。」無忌突然心中一動,説道︰「芷若,倘是我二人聯手,大功可成。我以純陽至剛的力道,牽纏住三位高僧的長鞭。你以陰柔之力乘隙而入,一進入伏魔圏中,内外夾攻,便能取勝。」周芷若冷笑道︰「咱們從前曾有婚姻之約,我丈夫此刻却命在垂危,加之今日我没傷你性命,旁人定然説我對你舊情猶存。倘若再邀你相助,天下英雄人人要罵我不知廉恥、水性楊花。」無忌急道︰「咱們只須問心無愧,旁人言語,理他作甚?」周芷若道︰「倘若我問心有愧呢?」

無忌一呆,接不上口,只道︰「你\dash{}你\dash{}」周芷若道︰「張教主,咱們二人孤男寡女,深宵共處,已惹物議。你快請吧!」無忌站起身來,深深一揖,道︰「宋夫人,你自幼待我很好,盼你再賜我一次恩德。張無忌有生之年,不敢想忘高義。」周芷若默不作聲,既不答應,亦不拒絶。她自始至終没有回過頭來,無忌無法見到她的臉色,待要再低聲下氣的相求,周芷若高聲道︰「靜慧師姊,送客!」

呀的一聲,房門打開,靜慧站在門外,一手執著長劍,氣憤憤的瞧著無忌。張無忌心想義父的生死在此一舉,自己的顏面屈辱,何足道哉,突然間跪在地下,向周芷若磕了四個頭,道︰「宋夫人,盼妳垂憐。」周芷若的身子仍如石像般一動不動。靜慧喝道︰「張無忌,掌門人叫你出去,你還糾纏些什麼?當眞是武林敗類,無恥之尤。」她還道無忌乘著宋青書將死,又來求周芷若重行締婚。張無忌嘆了口氣,起身出門。

他回到明教的茅棚之前,趙明迎了上來道︰「宋青書的傷有救,是不是?又用我的黑玉斷續膏去做好人。」無忌道︰「咦!你眞是料事如神。他傷勢是否能救,此刻還不能説。」趙明嘆了口氣,道︰「你想救了宋青書的性命,來換謝大俠,無忌哥哥,你是越弄越糟,一點也不懂人家的心事。」無忌奇道︰「爲什麼?這個我可不明白了。」趙明道︰「你用盡心血來救宋青書,那便是説一點也不顧念周姊姊對你的情意。你想她惱也不惱?」無忌一怔,無言可答,倘説周芷若願意自己丈夫傷重不治,那是絶無是理,但她確是説過︰「我知道你會盡心竭力,救活了他,以便自己問心無愧的去做朝廷郡馬。」這兩句話中,果是頗有怨對之意。趙明道︰「你救了宋青書的性命,現在又後悔了,是不是?」不等張無忌回答,微微一笑,便即翩然進了自己居室。

無忌坐在石上,對著一彎冷月,呆呆出神,回思一生經過,自從離開冰火島後,不一載而父母雙亡,自此而後,可説没一日不是身在憂患之中,自己一心求好,但往往事與願違。早知如此,與父母同在冰火島此生終老,豈不是好?

五月初六清晨,少林寺鐘聲鐺鐺撞起,群雄又集在廣場之中。那達摩院的老僧這次更不向空智請示,便即站了出來,朗聲説道︰「衆位英雄請了,昨日比武較量,峨嵋派掌門人,宋夫人藝冠群英,便請宋夫人至山後破関,提取金毛獅王謝遜。老僧領路。」説著當先便行。峨嵋派靜慧等八名女尼跟隨其後,接著便是周芷若與峨嵋群弟子。衆英雄更在後面,一齊向山後走去。張無忌見周芷若衣飾一如昨日,並未戴喪,知道宋青書未死,心想︰「他既挨得過昨晩,看來性命能保。」

衆人上得山峰,只見三位高僧仍是盤膝坐在松樹之下。那達摩院老僧道︰「金毛獅王囚於三株蒼松間的地牢中,看守地牢的是敝派長老。宋夫人武功天下無雙,勝了敝派這三位長老,便可破牢取人。咱們大夥児再瞻仰宋夫人的身手。」楊逍見無忌臉色不定,在他耳邉悄聲説道︰「教主寬心。韋蝠王、説不得二位,已率領五行旗人衆伏在峰下。峨嵋派若不肯將謝獅王交出,咱們只好用強。」無忌皺眉道︰「這是壞了大會的規矩,有失信義。」楊逍道︰「我怕宋夫人將刀劍架在謝獅王頸中,咱們動手時投鼠忌器。信義什麼的,也顧不得這許多了。」趙明也悄聲道︰「謝大俠仇人極多,咱們要防備人叢中暗器偸襲。」楊逍道︰「范右使、鐵冠道長、周兄、彭大師四位已分佔四角,防人偸襲。」趙明低聲道︰「最好是若有人用暗器偸襲,咱們就可乘機動手,搶了謝大俠便走。天下英雄也不能怪咱失了信義。要是風平浪靜,楊左使,不如你暗中派人假裝襲擊謝大俠,紛擾之中,咱們混水摸魚搶人。」楊逍笑道︰「此計大妙。」當下便去派遣人手。

張無忌明知此舉甚不光明磊落,但爲了相救義父,那也只好無所顧忌,心中又不禁感激趙明,暗想︰「明妹和楊左使均有臨事決難的大才,難得他二人商商量量,極是投機,我可就没這種本事。」

只聽周芷若道︰「三位高僧既是少林派長老,自是武學深湛。要本座以一敵三,非但不公,抑且不敬。」那達摩院老僧道︰「宋夫人要添一二人相助,亦無不可。」周芷若道︰「本座承天下英雄相讓,僥倖奪魁,所仗著不過是先師滅絶師太祕傳的本派武功。若是以三敵三,縱然得勝,也未能顯得先師當年教本座的一番苦心,但如以一敵三,又是對主不恭。這樣吧,我叫一位昨日傷在本座手下、傷勢尚未痊可的小子聯手。這小子當年曾被先師三掌擊得口吐鮮血,天下皆知。如此便不損先師威名。」張無忌一聽之下,心中大喜︰「謝天謝地,她果然允我之請。」只聽周芷若道︰「張無忌,你出來吧。」

明教群豪除了楊逍等數人之外,都是不明其中原由,但聽她小子長、小子短的侮辱本教教主,盡皆憤恨難平。不料張無忌臉有喜色,走了出來,長揖到地,説道︰「宋夫人昨日手下留情,饒了小子性命。」他心中已然打定了主意︰「她當衆辱我,不過是爲峨嵋派掙個顏面,再報那日婚禮中新郎遁走的羞恥。爲了義父,我是委曲求全到底。」周芷若道︰「你重傷未癒,我也不要你眞的幫手,只不過作個樣子而已。」張無忌道︰「是,一切遵命而行,不敢有違。」

周芷若取出軟鞭,右手一抖,鞭子登時捲成十多個大大小小的圏子,好看已極,左手翻處,青光閃動,露出了一柄短刀。群雄昨日已見識了她軟鞭的威力,不意她左手尚能同時用刀,一長一短,一柔一剛,那是兩種截然相異的兵刃。群雄驚佩之下,精神都爲之一振。

張無忌從懷中摸出兩枚聖火令來,向前走了兩步,突然脚下一個踉蹌,故意又咳嗽幾聲,顯得重傷未愈,自保也是十分勉強,待會若是勝了少林三僧,好讓群雄都説全是周芷若的功勞。只見渡厄等三僧緩緩將長鞭抖了出來,周芷若靠到無忌身邉,低聲問道︰「你曾立誓爲你表妹報仇,倘若害她的兇手是你義父,你還救他不救?」無忌一怔,道︰「義父有時心智失常,作不得數。」

正在此時,忽聽得峰腰裡傳來輕輕數聲琴簫和鳴之聲。無忌心中一喜,只聽得瑤琴錚錚錚連響三下,四名白衣少女翩然上峰,手中各抱一具短琴,跟著簫聲抑揚,四名黑衣少女手執長簫走上山峰來。黑白相間,八名少女分佔八個方位,琴簫齊奏,樂音極是柔雅。一個身披淡黃輕紗的美女從樂聲中緩步上峰,果然是當日無忌在盧龍丐幫中會過之人。

丐幫的女童幫主史紅石一見,奔將過去,撲在她的懷裡,叫道︰「楊姊姊,楊姊姊!咱們的掌老和龍頭,都給人害了!」説著手指周芷若,道︰「是她峨嵋派和少林派下的毒手。」那黃衣少女點頭道︰「我都知道了。哼!『九陰白骨爪』未必便是天下最強的武功。」她上峰來時這等聲勢,人又美貌飄逸,人人的目光都在瞧她,這兩句話更是清清楚楚的送到了各人耳中。群雄一凜之下,心想︰「峨嵋派這路爪法,便是百年前馳名江湖的陰毒武功『九陰白骨爪』麼?」年紀較長的武林人士,都曾聽過「九陰白骨爪」的名字,但均知這門武功陰毒過甚,久已失傳,誰也没有見過。黃衫女子擕著史紅石的手,走入丐幫之中,便在一塊山石上坐了。

周芷若道︰「這女子是誰?」張無忌道︰「我只見過她一次,不知她的姓名來歷。」周芷若道︰「她不是姓楊麼?」無忌道︰「我也是此刻首次聽見。」周芷若哼了一聲,道︰「動手吧!」長鞭一抖,捲向渡難的長鞭,身子一借勢,便從三株蒼松間落了下去。她第一招便直攻敵人中央,狠辣迅捷,膽識之強,縱是第一流江湖老手,也是有所不及。群雄只見她身在半空,如一雙青鶴凌空撲擊而下,身法曼妙無比。她右手的軟鞭與渡難的長鞭纏在一起,既借其力,又使渡難的兵刃暫時無法使用。渡厄和渡劫雙鞭齊揚,分從左右擊至。張無忌直搶而前,脚下一躓,一個筋斗摔了過去。群雄咦的一聲,只道他傷後立足不定。

那知道張無忌這一招使的乃是聖火令上所載的古波斯武功,身法怪異,已達極點,他似是向前摔跌,雙手聖火令却已向渡難胸口拍了過去。其時渡難長鞭正與周芷若的鞭子纏住未分,不能迴鞭抵擋,渡厄、渡劫和他同一體,一見勢危,立時捨却周芷若,雙鞭向無忌身上擊來。兩條長鞭矯夭若遊龍,眼見無忌性命不保,不料他在地下一個打滾,狼狽萬狀的滾向渡厄身邉,渡厄左手向他肩頭戳去,無忌左掌以挪移乾坤之力化開,身子一晃,和身向渡劫撞到。

原來他今日一意要捧周芷若成名,將擊敗少林三高僧的尊榮,盡數歸於這位峨嵋掌門,自己只求救出謝遜,是以使的全是古波斯武功,東滾一轉,西摔一交,要多難看就有多難看,要多狼狽就有多狼狽。旁觀群雄之中,原本不乏識見卓超的人物,但一來這路古波斯武功實是太怪,二來從未有人在中土用過,三來昨日張無忌身受重傷乃是人所共見,因此初時最多没瞧出破綻。拆到數十招後,只周芷若身形忽高忽低,飄忽無方,張無忌越來越是招架不住,手忙足亂,竟似比一個初學武功的莽漢尤有不如,但不論情勢如何兇險,他總是能在千鈞一髮之際避開了對方的殺著。

旁觀群雄中年事較長,心智機敏的便知其中必有蹊蹺,多半張無忌所使的,乃是「酔八仙」一類的功夫,看上去顚三倒四,實際中奇奥變化,這一類武功,比之正路功夫可又難上許多了。可是這門古波斯的武功,若是單獨對付渡厄或渡劫、渡難一人,對方定然鬧個手足無措,便如無忌初逢風雲三使時那麼落於下風。但這三位少林高僧枯禪坐將下來,心意相通,任誰一人招數中露出破綻空隙,其餘二人立時予以補足。無忌種種怪異身法,本來每一招都足以迷亂敵人眼光,似左實右,似前實後,只要判斷略一錯誤,立時便上了他的大當,但三高僧鞭隨心動,對無忌的諸番做作竟是視而不見。拆到七八十招時,無忌怪招縱是層出不窮,却是没能損及三高僧分毫。鬥近百招,無忌只覺三高僧長鞭上威力漸強,自己身法却慢慢的澀滯起來,已無初鬥時的靈動自如。

原來無忌所使武功,有小半已入魔道,三高僧的「金剛伏魔圏」,正是以佛力伏魔的精妙大法。旁人只見無忌越鬥越是精神,其實他心靈中魔頭漸長,只須再鬥百招,那就全然處於三高僧佛門上乘武功的克制之下,不由自主的狂舞不休。三高僧不須出手,他自己便制了自己死命。要知明教被稱爲「魔教」,亦非全無道理,而這路古波斯武功的創立人「山中老人」,更是殺人不眨眼的大惡魔。無忌初時照練,倒也不覺如何,此刻乍逢勁敵,將這路武功中的精微處盡數發揮出來,心靈漸受感應,突然間哈哈仰天三笑,聲音中竟是充滿了邪惡的奸詐之意。

他三笑方罷,猛聽得三株蒼松間的地牢中傳出誦經之聲,正是義父謝遜的聲音。只聽他蒼老的聲音緩緩誦唸佛經︰「爾時須菩提
\footnote{\footnotefon{}〔按〕須菩提是在舍衛國聽佛説金剛經的長老}
聞説是經,深解義趣,涕泪悲泣,而白佛言︰『希有世尊,佛説如是甚深經典。我從昔來所得慧眼,未曾得聞如是之經。世尊,若復有人得聞是經,信心清淨,即生實相\dash{}』」無忌邉鬥邉聽,自謝遜的誦經聲一起,少林三高僧長鞭上的威力也即收斂,只聽謝遜繼續誦道︰「『世尊,我今得聞如是經典,信解受持,不足爲難。若當來世,後五百歳,其有衆生得聞是經,信解受持,是人即爲第一希有,何以故?此人無我相、無人相、無衆生相、無壽者相\dash{}』」

無忌聽到此處,心中思潮起伏,知道義父自被囚於峰頂地牢,每日聽少林三高僧講經,上次明明可以脱身,却是自知孼重罪深,堅決不肯離去,難道他聽了數月經文之後,終於大澈大悟麼?那經中言道︰「若當來世,後五百歳,其有衆生得聞是經,信解受持。」在義父心中,這五百年後之人,便是他了。只聽他又唸佛經道︰「佛告須菩提︰『如是,如是!若復有人,得聞是經,不驚,不怖,不畏,當知是人甚爲希有\dash{}如我昔爲歌利王割截身體,我於爾時,無我相、無人相、無衆生相、無壽者相。何以故?我於往昔節節支解時,若有我相、人相、衆生相、壽者相,應生瞋恨\dash{}是故,菩薩須離一切相。』」

張無忌於佛經精義,原本不解,但謝遜所唸經文,句句涉及他的自身,文義甚是明白,那顯然是説,世間一切全是空幻,對於我自己的身體,别人的身體,心中全不必牽念,即使别人將我身體割截,節節支解,因爲我根本不當自己的身體,所以他絶無惱恨之心。「義父修爲若此,是否叫我不必爲他煩惱,不必出力救他脱險?」

張無忌施展聖火令上的古波斯武功,只因對手是三位中土第一流的佛家高手,到得百餘招後,魔由心生,他已漸漸受到自己心中魔頭的牽制,正自一步步的踏入危境,忽聽得謝遜在地牢中誦經之聲。原來少林三僧三條鞭組成「金剛伏魔圏」,原是以「金剛經」爲最高旨義,最後要做到「無我相、無人相、無衆生相、無壽者相」,於人我之分、生死之别,全部視作空幻。只是少林三僧修爲雖高,臨敵時總是忘不了克敵制勝的念頭,雖將自己生死置於度外,人我之分却是無法除去,因此這「金剛伏魔圏」的威力,還不能練到極致。但這數月來,三僧對謝遜所講的,便是這部「金剛經」。

無忌一聽到佛經,手下招數不停,心中却想到了經文中的含義,魔意消退,這路古波斯武功立時不能連貫,刷的一聲,渡劫的長鞭抽到了他的左肩。無忌左肩一沉,不由自主的使出了挪移乾坤心法,配以九陽神功,登時將擊來的勁力卸去,心念微動︰「我用這路古波斯武功實是難以取勝。」斜眼看周芷若時,見她左支右絀,也已呈現敗象,暗想︰「今日之勢,事難兩全。我若不出全力,芷若一敗,救義父之事便無指望了。」一聲清嘯,使開兩根聖火令,著著進攻。謝遜誦經之聲並未停止,但無忌凝神施展乾坤大挪移心法,没能再去聽他所唸經文的含義。他儘量將三僧的長鞭接到自己手上,以便讓周芷若能尋到空隙,攻入圏内。

他這一全力施展,三僧祇覺鞭上壓力漸沉,迫得各運内力與之抗禦。三僧中渡厄修爲最高,深體必須除却「人我四相」,但渡難、渡劫二僧爭雄鬥勝的念頭一盛,著了世間相的形跡,渡厄的鞭法非和他二人相配不可。旁觀群雄一見無忌改了武功的招數,三株蒼松間的爭鬥越來越是激烈。祇見三僧頭頂漸漸現出一團淡淡的水氣,知道那是額上汗水爲内力一逼,化作了蒸氣,可見五個人已到了各以内力相拚的境地。張無忌頭頂也有水氣現出,却是筆直一條,又細又長的聚而不散,顯是他内力深厚,更勝三僧。群豪昨日人人見到他身受重傷,那知他祇是一宵之間,便即痊癒,内力之深厚,已達化境,適纔的摔跌滾動,全是假意做作。即是武學平平之輩,也都看了出來。

周芷若却不與三僧正面交鋒,祇在圏外游鬥,見到金剛伏魔圏上生出破綻,便即縱身而前,一遇長鞭攔截,立時翩若驚鴻般躍開。這麼一交鋒,張無忌和她武學修爲的高下,登時再也無法隱瞞,旁觀群雄中已有人竊竊私議︰「近來年武林中傳言︰明教張教主武功之強,當今獨步。果然是名不虛傳。」

\qyh{}昨天他是故意讓這位宋夫人的,這叫做好男不與女鬥啊。」

\qyh{}什麼好男不與女鬥?宋夫人本來是張教主的妻子,你知不知道?這叫做故刀情深!」

\qyh{}{\upstsl{呸}}!只有故劍情深,那裡有故刀情深?」

\qyh{}刀跟劍都是兵器,有什麼分别?」

少林三僧和張無忌的招數越打越是緩慢,變化越來越是精微。上得少林寺來參與英雄大會之人,個個都是江湖上成名的人物,有的本身武功雖非一流,識見大都不凡,此刻見到這場拚鬥到了這等高深的境界,無不嘆爲觀止。周芷若的武功純以奇幻見長,制服武當二俠,已是她成就的峰巓,説到内功修爲,比之兪蓮舟、殷利亨尚是遠爲不如。這時張無忌與少林三僧各以眞實本領相拚,半分不能取巧,她竟是插不下手去,有時軟鞭一晃上前進攻,在四人的内勁上一碰,立時不由自主的彈了出來。

又鬥小半個時辰,張無忌體内九陽神功急速流動,聖火令上發出嗤嗤聲響。少林三僧的臉色本來各自不同,這時却都是殷紅如血,僧袍都鼓了起來。

少林三僧身上僧袍高高鼓起,便似爲大風所充,但張無忌的衣衫却是並無異狀,這般情景高下已判,倘若無忌是以一對一、甚而是以一敵二,早已獲勝。張無忌所練的九陽眞氣原本渾厚無倫,再加上張三丰指點,學得太極拳中練氣之法,更是愈鬥愈盛,最能持久,他原可再拚一兩個時辰,以待三僧氣衰力竭。但少林三僧拚到此時,也已瞧出久戰於已不利,突然間三僧齊聲高喝,三條長鞭急速轉動,鞭影縱橫,似眞似幻。無忌凝視敵鞭來勢,一一拆解,心下暗自焦急︰「芷若武功雖奇,究竟所學時日無多,尚比不上外公和楊左使二人聯手的威力。我獨力難支,看來今日又要落敗了。這次再救不出義父,那便如何是好?」

他心中一急,心浮氣粗,招數上威力稍減,三僧乘機跟著進擊,更是險象環生。無忌腦中如電光火石般一閃,想起昔年冰火島上謝遜對他的慈愛,又想謝遜所以眼盲之後仍是干冒大險,重入江湖,全是爲了自己,今日若是救他不得,實是不願獨活。眼見渡難一鞭自身後遙遙兜至,張無忌突出怪招,左手一舉,竟讓這一鞭擊中手臂,只是用了挪移乾坤之法,將鞭力卸去,右手聖火令擋住渡厄、渡劫雙雙攻來的兩鞭,身子忽然大鳥般向左撲了出去,空中一個迤旋,已將渡難那條長鞭在他所坐的蒼松上繞了一圏。

這一招直是匪夷所思,渡難的長鞭一纏上松樹,登時無兵刃可用。無忌左臂力振,向後急拉,要將他長鞭深深嵌入松樹樹幹。渡難大驚之下,向後力奪。無忌變招奇速,順著他的力道扯了過去。那松樹的樹幹雖粗,但樹根處已有大半被三僧挖空,用以遮蔽風雨。此刻被一條堅韌無比的長鞭纏住,由無忌和渡難兩股極大的力道同時拉扯,只聽得喀喇喇一聲巨響,那松樹在挖空處折斷,從半空中倒將下來。

張無忌得理不讓人,當渡厄、渡劫二僧一愕之際,雙掌齊施,大喝一聲,推向渡厄身居的那株蒼松。這兩掌上的掌力,乃是他畢生功力所聚,那松樹抵受不住,當即折斷。兩株斷下的松樹連枝帶葉,一齊壓向渡劫所居的松樹上去。這株松樹倒下時已有數千斤的力道,無忌飛身而起,雙足更在第三株松樹上一蹬,那松樹又即斷折,在半空中搖搖晃晃,緩緩倒下。

其時松樹折斷聲、旁觀群雄驚呼聲混成一片,張無忌手中兩枚聖火令便力向渡厄、渡劫擲了過去。渡厄、渡劫既要閃避從空倒下的松樹,又要應付無忌擲來的聖火令,武功雖強,却也鬧了個手忙足亂。無忌身子一矮,貼地滾過尚未著地的樹幹,已到了金剛伏魔圏的中心,使出挪移乾坤心法,雙掌一搓一推,立時便將蓋在地牢上的大石推開,叫道︰「義父,快出來!」他有過前車之鑑,生怕謝遜又不肯出來,以致適纔這番僥倖成功的冒險盡付東流,不待謝遜答應,探手下去,抓住謝遜的後心,一提便提了上來。

便在此時,渡厄和渡劫雙鞭齊到,無忌迫得放下謝遜,懷中又掏出兩枚聖火令,向二僧擲出,雙手快如電閃,抓住了兩條長鞭的鞭頭。渡厄、渡劫正要各運内力與無忌奪鞭,聖火令已擲到面門,雙令之到,快得直無思量餘地,兩僧只得撒手棄鞭,急向後躍,這纔避開了聖火令之一擊。當眞是説時遲,那時快,渡厄和渡劫向後躍開之時,渡難左掌已向無忌胸口拍到。無忌叫道︰「芷若,快絆住他!」斜身一閃,抱起了謝遜,只須將他救出了三松之間,少林派便無話説。周芷若哼了一聲,微一遲疑,渡難又是一掌拍到。無忌身子一轉,避開背心要穴,讓渡難這一掌擊中自己肩頭。

張無忌抱了謝遜,正要從三株斷松間走了出來,謝遜道︰「無忌孩児,我一生罪孼深重,在此處聽經懺悔,正是心安理得。你何必救我出去?」説著便要掙扎下地。無忌知道義父武功既高,若是堅決不肯出去,倒難應付,説道︰「義父,孩児得罪了!」右手五指閃了幾閃,點了他大腿與胸腹間的數處穴道,讓他暫時動彈不得。

就是這麼一遲疑,渡厄、渡難、渡劫三僧的手掌同時拍了過來,同聲喝道︰「留下人來!」張無忌見三人的掌力將四面八方都籠蓋住了,手掌未到,掌風已是森然逼人,只得將謝遜放在地下,出掌抵住,叫道︰「芷若,快將義父抱了出去。」他雙掌搖晃成圏,運掌力與三僧對抗,使三僧無一能抽身攔阻周芷若。這是乾坤大挪移心法中最高深的功夫之一,他掌力游走不定,虛虛實實,將三僧的掌力黏住了。但這門功夫,也是最耗眞力,比之適纔比拚内力,那是辛苦得多。

張無忌雖知此舉費神,難以持久,但想周芷若抱了謝遜出去,只是傾刻間的事,那時便可設法脱離三僧掌上的黏力。周芷若一躍進圏,到了謝遜身畔。謝遜喝道︰「{\upstsl{呸}},賤人!\dash{}」周芷若一伸手,便點了他的啞穴,叱道︰「姓謝的,我好意救你,何以出口傷人?你罪行滔天,命懸我手,難道我便殺你不得麼?」説著舉起右手,五指成爪,便要往謝遜天靈蓋上抓了下去。無忌一見大急,忙道︰「芷若,不可!」其時他與少林三僧正自各以絶學相拚,少林三僧雖無殺他之意,但到了這等生死決於俄頃的関頭,不是傷敵,便是己亡,實無半點容讓的餘裕。張無忌一開口,眞氣稍洩,三僧的掌力便排山倒海推將過來,只得催力抗禦。雙方各運「黏」字訣,非分勝敗,難以脱身。

周芷若手爪舉在半空,却不下擊,斜眼冷睨張無忌,冷笑道︰「張無忌,那日濠州城中,你在婚禮中捨我而去,可料到有今日之事麼?」張無忌心分三用,既擔心謝遜被她抓死,又恨她在這緊急関頭來算舊帳,何況少林三僧的掌力源源而至,縱然專心凝神的應付,最後也非落敗不可,這一心神混亂,更是大禍臨頭。他額上冷汗涔涔而下,霎時之間,前胸後背,衣衫都已被大汗濕透。

楊逍、范遙、韋一笑、説不得、兪蓮舟、殷利亨等看到這般情景,無不大驚失色。這些人均是義氣深重,只教救得張無忌,縱然犧牲自己性命,也是絶無悔恨,但各人均知自己功力不及無忌與少林三僧,别説從中拆解,便是上前襲擊少林三僧,三僧也會輕易而舉的將外力轉移到無忌身上,令他受力更重,那是救之適足害之了。崆峒五老唐文亮、宗維俠、常敬之等感懷張無忌昔日之德,也是頓足搓手,十分焦急。空智忽然叫道︰「三位師叔,張教主曾於本派有恩,傷之不義,務請手下留情。」楊逍、范遙等聽他這等説,都好生感激。但三僧和無忌的比拚已到了難解難分的地步,無忌原無傷害三僧之心,三僧念著日前無忌相助解圍,也早就相俟機罷手,只是雙方均是騎虎難下。三僧神遊物外,對空智的叫聲聽而不聞,根本就不知他在説些什麼,其實便算得知,却也無能爲力。

韋一笑身形一晃,如一溜輕煙般閃竹斷松之間,便待向周芷若撲去,却見周芷若右手作勢,懸在半空,自己只須上前這麼一撲,她手抓立時便向謝遜頭頂插下。謝遜若死,張無忌心中大悲,登時便會死在三僧掌力之下。韋一笑與周芷若相距不到一丈,却是呆呆定住,不敢上前動手。一時之間,山峰上每個人都似成了一座石像,誰都一動不動,也是誰都不出一聲。驀地裡周顚哈哈一笑,踏步上前。

\chapter{怨恩了了}

楊逍吃了一驚,喝道︰「顚兄,不可魯莽。」周顚竟不理會,走到少林三僧之前,嘻皮笑臉的説道︰「三位大和尚,吃狗肉不吃?」一伸手,從懷中掏出一隻煮熟了的狗腿,在渡厄面前晃來晃去。原來這兩日少林寺中供應的都是素齋,周顚好酒愛肉,接連幾日青菜豆腐,如何能挨?昨日晩間偸了一隻狗,宰來吃了一個飽,尚留著一條狗腿,此刻事急,便去擾亂少林三僧的心神。楊逍等一見,盡皆大喜,心想︰「周顚平時行事瘋瘋癲癲,這一著却大是高招。」須知少林三僧與張無忌比拚内力,関鍵全在於專志凝神,周顚上前胡鬧,祇須有一僧動了嗔怒,心神微分,無忌便可得勝。

三僧視而不見,毫不理會。周顚拿起狗腿張口便咬,説道︰「好香氣,好滋味!三位大和尚,吃一口試試。」他見三僧絲毫不動聲色,當下將狗腿挨到渡厄口邉,待要塞到他的口中。旁觀的少林群僧紛紛呼喝︰「兀那癲子,快快退下!」周顚將狗腿往前一送,剛碰到渡厄口唇,突然間手臂一震,半身酸麻,拍的一聲,狗腿掉在地上。原來渡厄此時内勁佈滿全身,以至「蠅蟲不能落」的境界,四肢百骸一遇外力相加,立時反彈出來。周顚叫道︰「啊喲喲!了不起,了不起!你不吃狗肉,那也罷了,何必將我好好一條狗腿,掉在地下弄得稀髒?我要你賠,我要你賠?」他手舞足𨂻,大叫大嚷。不料三僧修爲深湛,絲毫不受外魔所擾。周顚右手一翻,從懷中取出一柄短刀,叫道︰「你不領情吃我的狗腿,老子今日和你拚了。」一刀在自己臉上一劃,登時鮮血淋漓。

群雄驚呼聲中,周顚又用短刀在自己臉上一劃,一張臉血肉模糊,甚是猙獰可怖。這等情景本來不論是誰見了都要心驚動魄,但少林三僧神遊外物,五官倶失其用,不但見不到周顚自殘的情景,連周顚這個人出現在身前也均不知。周顚大聲叫道︰「好和尚,你不賠還我的狗腿,我死在你的面前了吧!」舉起短刀,便往自己心窩中插了下去。要知周顚性本忠義,眼見教主命在俄頃,決心捨生自殺,以擾亂三僧的心智。

他這一刀插下,驀地裡黃衫一閃,一個人飛身過來,手腕一翻,夾手將他的短刀奪去,跟著斜身而前,五指伸張,往周芷若頭頂插了下去,所用手法,與宋青書殺斃丐幫長老的姿式全然相同。周芷若的手指與謝遜頂門相距雖然不過尺許,但敵人襲來的身法實在太快,只得翻手上托,解開了襲來的這一招。

張無忌的内勁之強,並不輸與三僧聯手,只是「物我兩忘」的枯禪功夫,却是不及三僧,於外界事物,做不到視而不見、聽而不聞的地步,是以周芷若對謝遜一加威脅,他立時便心神大亂。待得周顚上前胡鬧,進而抽刀自盡,張無忌一一瞧在眼裡,更是焦急。正在這内息大亂、轉眼便要噴血而亡的當児,鼻中忽然聞到一陣淡淡的香氣,那黃衫女子躍身進來,奪去周顚手中短刀,出招攻向周芷若,解去了謝遜的危難。

張無忌心中一喜,内功立長,將三僧攻過來的勁力一一化解,霎時之間便成了個相持不下的局面。渡厄等雖於外界事物不聞不見,但對雙方内勁的消長,却是辨析入微,陡然間察覺到無忌内勁大張,可是又不反守爲攻,這正是消除雙方危難的最佳時機,三僧心意相通,立時將己方内勁微微一收。張無忌跟著收了一分勁力,三僧亦收一分。如此你收一分,我收一分,不到一盞茶時分,雙方勁力收盡,四人同時哈哈一笑,一齊站起身來。張無忌長揖到地,渡厄、渡劫、渡難三僧同時合什還禮。四人齊聲説道︰「佩服,佩服!」

張無忌回過頭去,只見那黃衫女子和周芷若鬥得正緊。黃衫女子一雙空手,周芷若右手鞭,左手刀,却兀自落於下風。那黃衫女子的武功似乎與周芷若乃是一路,飄忽靈動,變幻無方,但舉手抬足之間,却是正而不邪,如説周芷若形似鬼魁,那黃衫女子便是態擬神仙。無忌只看得兩眼,已知黃衫女子有勝無敗,義父絶無危險,但見那黃衫女子出手中頗有引逗之意,似要看明白周芷若武學的底細,若是當眞求勝,早已將周芷若打倒了。

渡厄説道︰「善哉,善哉!張教主,你雖勝不得我三人,我三人也勝不得你。謝居士,你請自便吧!」説著上前解開了謝遜身上的穴道,説道︰「謝居士,放下屠刀,立地成佛。我佛門戸廣大,世間無不可渡之人。你我在這山峰上共處多日,那也是有緣。」謝遜站起身來,説道︰「我佛慈悲,多蒙三位大師指點明路,謝遜感激不盡。」

只聽那黃衣女子一聲清叱,左手一翻,已奪下周芷若手中長鞭,跟著手肘撞中了她胸口穴道,右手成爪,伸在她的頭頂,説道︰「你要不要也嘗嘗『九陰白骨爪』的滋味?」周芷若動彈不得,閉目待死。謝遜雙目雖然不能見物,但於周遭一切情景,却是聽得十分明白,上前一揖,説道︰「姑娘救我父子二人性命,深感大德。這位周姑娘若不悔悟,多行不義,終有遭報之日。求懇姑娘今日暫且饒她。」黃衫女子道︰「金毛獅王悔改得好快啊。」身形一晃,便即退開。

張無忌擕了謝遜之手,正要並肩走開,謝遜忽道︰「且慢!」指著少林僧衆中的一名老僧叫道︰「混元霹靂手成崑,請你站將出來。當著天下衆英雄之前,將各種前因後果分説明白。」群雄吃了一驚,一齊向他手指之人望去。只見這老僧弓著背脊,形容猥瑣,僧袍也是十分破舊,相貌與成崑截然不同。無忌正待説︰「他不是成崑」,只聽謝遜又道︰「成崑,你改了相貌,聲音却是改不了。你一聲咳嗽,我便知你是誰。」那老僧獰笑道︰「誰來聽你這瞎子胡説八道。」他一開口説話,張無忌立時辨認了出來,那日光明頂上他身處布袋之中,曾聽成崑長篇大論的説話,對他語音記得清清楚楚。此刻他雖故意逼緊喉嚨,身形容貌更是喬裝得十分巧妙,但語音終究難變。張無忌縱身一躍,截斷了他的後路,説道︰「圓眞大師,成崑前輩,大丈夫光明磊落,何不以本來面目示人?」

成崑喬裝改扮,潛伏在人叢之中,始終不露破綻,那知當黃衫女子制服周芷若之際,他大出意料之外,忍不住輕輕咳嗽一聲,偏生謝遜雙眼盲後耳音特靈,對他又是記著銘心刻骨的血仇。就謝遜而言,這一聲咳嗽不啻是個晴天霹靂,立時便將他認了出來。成崑在寺中暗伏了大批黨羽,今日原想挑得與會群雄自相殘殺,逼出屠龍刀的下落,害死謝遜,最後更謀害空聞、空智,自己接任少林寺方丈之位。那知他的計謀雖巧,每一件事到頭來都是另生變故,無不事與願違。他見張無忌擋住去路,知道事已敗露,長身大喝︰「少林僧衆聽者︰魔教搗亂佛地,藐視本派,衆僧一齊動手,格殺勿論。」他手下黨羽紛紛答應,抽出兵刃便要上前動手。

空智忍氣已久,一直顧念著師兄安危,只得受本寺叛徒挾制,此刻聽圓眞號令僧衆與明教動手,情知一場混戰下來,本寺僧衆不知將受到多大的損傷,權衡輕重,空聞一人事小,闔寺僧衆的性命事大,當下喝道︰「少林弟子不得莽撞。空聞方丈已落入這叛徒圓眞手中,衆弟子先擒此叛徒,再救方丈。」霎時間峰頂上亂成一團。

混亂之中,張無忌見周芷若委頓在地,臉上充滿沮喪失望,心下大是不忍,當即上前解開她的穴道,扶她起身。周芷若一揮手,推開他的手臂,逕自躍回峨嵋群弟子之間。

只聽謝遜朗聲説道︰「今日之事,全自成崑與我二人身上所起,種種恩怨糾纏,須當由我二人了結。師父,我一身本事是你所授;成崑,我全家是你所殺。你的大恩大仇,今日咱們來算個總帳。」成崑見空智不顧一切的出聲號令,終究少林寺中正派者多而自己黨羽較少,看來接掌少林方丈職位的圖謀,終究也歸鏡花水月,到頭來一場空幻,何況明教與渡厄等三僧若是聯手,更難抵敵。他心思機敏,一計不成,二計又生,心想︰「謝遜作惡多端,我若制服了他,大可將一切罪行盡數推在他的頭上。他的武功皆我所授,他雙眼又盲,難道我還對付他不了?」於是説道︰「謝遜,江湖上有多少英雄好漢,命喪你手。今日更招引明教的大批魔頭,來少林擾亂佛門福地,與天下英雄爲敵。我深悔當年傳授了你武功,此刻非得清理門戸、整治一下你這欺師叛祖的逆徒不可。」説著大踏步走到謝遜面前。

謝遜高聲道︰「四方英雄聽者,我謝遜的武功,原是這位成崑所授,可是他逼姦我妻不遂、殺我父母妻児。師尊雖親,總親不過我親生的爹娘。我找他報仇,該是不該?」四下裡群雄轟然叫道︰「該當報仇,該當報仇!」成崑一言不發,呼的一掌,便向謝遜頭上劈了過去。謝遜頭一偏,讓過頂門要害,拍的一響,這一掌打在他的肩頭。謝遜哼的一聲,並不還手,説道︰「成崑,當年你傳我這一招『長虹經天』之際,説道若是擊中敵身,便當運混元一氣功傷敵,爲什麼不運功啊?是不是年紀老了,無功可運了?」原來成崑知己知彼,明白謝遜的武功極是了得,第一招只是虛招,没料到居然對方竟是不閃不躱,一擊而中。但他這一招上全没用上勁力,是以謝遜並未受傷。

成崑左手一引,右手一掌拍出。謝遜身子一斜,仍不還招。成崑連環踢出,拍拍兩響,謝遜脅下連中兩腿。這兩腿的勁力却是厲害無比,饒是謝遜體格粗壯,可也禁受不起,哇的一聲,一大口血噴將出來。無忌急叫︰「義父,還招啊!你怎能儘受打不還手。」謝遜身子搖晃了幾下,苦笑道︰「他是我師父,受他兩腿一掌,原也應該。」

成崑心中暗叫︰「倒霉,倒霉!我只道他對我仇深似海,一上來就是拚命,早知他還讓我三招,我可得痛下殺手了。」見謝遜這一掌來得凌厲,當即左手一引,卸開他的掌力,身子轉了半個圏子,已旋到他的身後,欺他眼不物,一掌無聲無息的從他背後按了過去。謝遜便如親眼所見,反足踢去,成崑輕輕一躍,從半空中如一隻老鷹般撲擊下來。他年已古稀,身手之矯捷竟是絲毫不輸於少年。謝遜雙手上托一彈,成崑下擊之勢被阻,又彈了上去,在半空中輕輕一個迴旋,又撲擊下來,身法之美妙,實是罕見。

兩人這一搭上手,以快打快,轉瞬間便拆了七八十招。謝遜雙目雖然不能見物,却佔了一個便宜。要知他一身武功全是成崑所授,他的拳脚成崑固所深悉,而成崑各種招數,謝遜也無不了然於胸。事過數十年,二人内功修爲倶各大進,但武功上拳脚的招術,仍是本門的解數。謝遜不必用眼,便知自己這一掌過去,對方將如何拆招,而跟著來的一招,多半是那幾種變化中的一種。加上他年紀比成崑小了十餘歳,氣血較壯,冰火島上奇寒酷熱的鍛鍊,於内力修爲大有好處,因之最初一百餘招中,竟是絲毫不落下風。

謝遜與成崑仇深似海,苦候數十年,此刻方始交上了手,張無忌本來料他定要不顧性命的撲擊,與成崑鬥個兩敗倶傷,那知謝遜一招一式,全是沉穩異常,將自己門戸守得極是嚴密。無忌初時略覺詫異,又看了數十招,當即領悟,成崑武功之強,實是不輸於渡劫、渡難等三僧,謝遜若是上來便逞血氣之勇,只怕支持不到五百招以上。他師徒二人於對方功夫修爲,自是知己知彼,謝遜心中仇恨越深,手上越是謹愼,生怕自己先毀在成崑手下,報不了父母妻児的血仇。

堪堪拆到二百餘招,謝遜大喝一聲,呼的一拳擊出。崆峒派的常敬之叫道︰「七傷拳!」只見謝遜左右雙拳連繼續擊出,威猛無儔,崆峒諸老相顧駭然。這七傷拳乃是謝遜從崆峒派盜得拳譜而學成,但拳上威力之強,遠過於崆峒嫡派的唐文亮、常敬之之諸老。成崑左掌一帶,待他又是一拳擊到時,右掌平推出去。拍的一響,拳掌相交,謝遜鬚髮倶張,威風凜凜的站著不動,成崑却是連退三步,旁觀群雄中許多人都喝起采來。原來謝遜與成崑結仇經過,江湖上傳聞已遍,衆人雖惱恨他出手太辣,濫傷無辜,但也覺他所遇太慘,不免寄以同情之心,旁觀人衆一大半是盼他得勝。

只見謝遜搶上三步,跟著又是呼呼兩拳擊出,成崑還了兩掌,復退三步。無忌暗叫︰「不好!成崑用的是少林九陽功,那是他拜空見神僧爲師後學來的功夫,成崑深悉其中関鍵所在,故示以弱,却將少林九陽功使將出來。謝遜每一拳打出,成崑受了他拳力的七成,以少林九陽功化解,其餘三成却反激回去。謝遜呼呼打出十二掌,成崑連退數十步,外表看來似是謝遜大佔上風,其實内傷越受越重。」

無忌心中焦急萬分,這是義父一生夢寐以求的復仇機緣,自己無論如何不能插手相助,但如此再鬥得數十拳,謝遜勢必嘔血身亡。空智突然冷冷的道︰「圓眞,我師兄當年傳你這少林九陽功,是教你用來害人的麼?」成崑冷笑一聲道︰「我恩師命喪七傷拳下,今日我是爲恩師報仇雪恥。」趙明突然大聲説道︰「空見神僧的九陽功修爲在你之上,他爲什麼不能抵擋七傷拳?空見大師是否害在你這奸賊手裡的。是你騙得他老人家出頭化解冤孼,騙得他老人家挨打不還手。嘿嘿,你看,你看,你背後站的是何人,滿臉是血,怒目指著你的背心,這不是空見神僧麼?」成崑明知是假,但他作了虧心事後,總是内疚神明,不由自主的打了個寒噤。正在此時,謝遜又是一拳擊到,成崑還了一掌,身子一晃,竟没後退,分心之下,眞氣走得岔了,被這一拳打得胸口氣血翻湧,當即展開輕身功夫,在謝遜身旁遊走,片刻後方調勻了氣息。

趙明叫道︰「空見神僧,你緊緊釘住他,不錯,就是這樣,在他後頸中呵些冷風。你死在徒児手中,他也必死在徒児手中,這叫做一報還一報,老天爺有眼,報應不爽。」成崑給她叫得心中發毛,疑心生暗鬼,隱隱似覺後頸中果然有陣陣冷風吹襲。其實這峰頂上終年山風不絶,加之他二人縱躍來去的打鬥,後心自然有風。趙明見他微有遲疑之態,又叫︰「啊,成崑,你回過頭來看看背後。你不敢回頭麼?你瞧瞧地下黑影,爲什麼二人打鬥,却有三個黑影。」成崑情不自禁的一低頭,果見兩個人影多了一個黑影,心中一窒,謝遜呼的一拳打了過來。成崑不及運神功化解,硬碰硬的還拳相擊,砰的一響,二人各以眞力相抗,都是身子一晃,向後退了一步。成崑這纔看清,所多的那個黑影,原來是那株斷折了的半截松樹所投。

成崑久戰謝遜不下,心中正自焦躁,暗想︰「他是我徒児,雙眼又盲了,我竟然仍是奈何他不得,我的心腹在旁瞧著也是不服。眼下情勢險惡,唯有制住這叛徒,一來可挾制明教,二來挑動與他有仇之人。如能逼問出屠龍刀所在,那是更好不過,否則至少也能自求脱身。」突然間見到斷松的黑影,心念一動,移步換形,悄没聲的向斷松處退了兩步。謝遜一拳擊出,搶上兩步,成崑又退兩步,想要引他絆倒在斷松之上。謝遜正待上前追擊,張無忌叫道︰「義父,小心脚下。」謝遜心中一凜,向旁跨開,便是這麼稍一遲疑,成崑已找到空隙,一掌無聲無息的拍到,正印在謝遜胸口,掌力一吐,謝遜向後便倒。

成崑一脚向他頭蓋上踹去,謝遜一個打滾,又站了起來,嘴角邉不住流出鮮血,成崑寂然不動,一掌緩緩伸出。要知謝遜與他相鬥,全仗熟悉招數,輔以聽風辨形,此刻成崑一招得手,領悟到招數越慢,出手無聲,謝遜便越是難以提防,這一掌慢慢移到謝遜面門,一按一翻,一掌又是打在他的肩頭。謝遜身子一晃,強力撐住。群雄中許多人看得不服,紛紛叫嚷起來︰「亮眼人打瞎子,用這等卑鄙手段!」成崑不理,又是緩緩一掌拍出。謝遜凝神頃聽,他手掌略抖,立時舉手招架,格開了這一掌。無忌見他滿頭黃髮飛舞,嘴角都是鮮血,心下又憤又急,情知這般鬥將下去,那是非死在成崑手中不可,只是謝遜一世英雄,在這當口自己若是出手相助,縱是殺得成崑,謝遜也是雖生猶死,英名盡行付與流水。他找住趙明的手,急道︰「明妹,快想個計較纔好。」趙明道︰「你能偸發暗器,打瞎了老賊雙目麼?」無忌搖頭道︰「義父寧死也不肯讓我做這等事!」

忽然間日色漸暗,似乎烏雲蔽天,有人叫了起來︰「天狗吃太陽,天狗吃太陽!」\footnote{\footnotefon{}日食通常只發生在夏曆初一(朔日),估計是爲了照顧劇情。三聯版徑改之。}無忌抬頭一看,只見一輪紅日缺了一片,正是日食之象。四下裡喧聲漸響,有的抬頭望日,有的仍是目不轉睛的凝神瞧著成謝二人打鬥,有的心中驚恐,竟跪下來向著太陽磕拜。趙明叫道︰「成崑老賊,你作惡多端,老天爺也不饒你,這不是示警懲罰於你嗎?你今日壽元已終,死後上刀山,下油鍋,萬劫不得超生。」成崑本已心虛,但見四下裡越來越黑,聽趙明這麼一叫,更是膽怯,雙掌呼呼接連拍出,便欲脱身逃走下山,但謝遜一心報仇,於四周變故全不大理會,緊緊纏住了他,令成崑難以抽身。猛聽得山峰下雄雞喔喔而啼,片刻之間,太陽已全被月亮的陰影遮住,遠遠更傳來獸吼犬吠之聲。群雄雖均膽大,但身處空曠之地,陡遭天變,心中無不惴惴。這一次日蝕甚是奇怪,日光竟被遮得半點不露,人人眼前黑漆一團,伸手不見五指。張無忌握住趙明的手,雖是用力凝視,也已瞧不見謝遜和成崑相鬥的情景。

這一日月無光,成崑登時成了瞎子,初時還隱隱約約的看到謝遜身形遊動的影子,到得後來,竟如雙眼蒙了一塊厚厚的黑布。他急速後躍,只盼遠離謝遜,但謝遜一招快似一招,黑影中只聽得成崑「啊」的一聲慘叫,胸口已被一招七傷拳擊中。成崑究是老謀深算,知道自己一拳受傷不輕,若再後躍,只有連續中拳,黑暗中當即招數一變,以「小擒拿手」禦敵。這「小擒拿手」原是黑暗中近身搏擊之用,講究應變奇速,不必用眼觀看,手指、手掌、手臂、手肘任何一處碰到敵人身體,立時擒拿抓打、撕拍鉤碰。謝遜大喝一聲,也以「小擒拿手」對付。兩人所使的武術招數並無分别,群雄只聽得黑暗中呼喝連連,夾雜著拳掌與肉體相碰之聲,迅如爆豆,想是兩人均是全速相攻。

張無忌心中怦怦亂跳,暗想義父若是遭到兇險,便欲出手相救,也不可得,極目凝視,也是無法辨别二人的身形。

謝遜瞧不見天象奇變,但數招之間,已覺察到成崑拳脚之來,往往著於空處,再聽到旁人大叫︰「天狗吃太陽,天狗吃太陽。」登時明白了這中間的道理。他雙眼已盲了二十餘年,聽聲辨形的功夫早已練得爛熟,以耳代目,行之已慣。成崑却是陡然間成了瞎子,亂打亂拿,雙方優劣之勢,立時逆轉。謝遜加速進擊,心想日食之變片間便即過去,只須太陽稍露光芒,自己暫時所佔的上風便即失却。成崑步步退後,謝遜則是著著進逼。成崑心中驚懼,饒是他平時老奸巨滑,但此刻心智失常,竟没有想到緊守門戸,以待日食之過,只想黑暗中相鬥於自己大大不利,務須及早料理謝遜,是以脚下不住倒退,兩條手臂却是使得猶如疾風驟雨一般,「小擒拿手」中的毒招狠著,加快的施展。

驀地裡謝遜雙掌一分,搶擊成崑脅下。成崑大喜,叫一聲︰「著!」右手食中二指,取向謝遜雙手。這一招「雙龍搶珠」,招式原非極奇,只是挾在「小擒拿手」中使將出來,却有極大的威力,對方側頭一避,他左手橫掃一掌,非擊中他太陽要穴不可。那知謝遜不閃不避,也喝一聲︰「著!」也是一招「雙龍搶珠」使出,食中二指插向他的雙目。成崑二指插中謝遜眼珠,腦海中如電光石火一閃︰「糟糕!」跟著自己雙眼一痛,已被謝遜二指插中。

這時月影輕移,太陽周圍露出一圏日暈。群雄只見成崑和謝遜均是雙目流血,相對不動。二人所受的傷一模一樣,但謝遜雙眼早盲,再被成崑二指插中,只不過是皮肉受損,並無所失。成崑却變成了盲人。謝遜冷笑道︰「瞎子的滋味好不好過?」呼的一拳擊去,成崑目不見物,無法閃避,這一招「七傷拳」正打在他的胸口。謝遜左手跟著又是一拳,成崑倒退數步,摔在斷松之上,口中鮮血狂噴。忽聽得渡厄説道︰「因果報應,善哉善哉!」謝遜一呆,第三拳没再擊去,説道︰「我本當打你一十三拳七傷拳。但你武功全失,雙目已盲,從此成爲廢人,再也不能在世間爲惡。餘下的一十一拳,那也不用打了。」

群雄見他大獲全勝,都歡呼起來。謝遜突然坐倒在地,全身骨骼格格亂響,無忌大吃一驚,知道他是逆運内息,要散盡全身武功,此舉非同小可,忙道︰「義父,使不得!」搶上前去,正待伸手按上他的背心以九陽神功制止。謝遜猛地裡躍起身來,伸手在自己胸口砰的狠擊一拳,口中鮮血狂噴。無忌忙伸手扶住,只覺他手掌衰弱無力,知他功力已失,再難以復原了。謝遜指著成崑説道︰「成崑你殺我全家,我今日毀你雙目,除了你的武功,以此相報。師父,我的一身武功是你所授,今日我盡數毀了,還了給你。從此我和你無恩無怨,你永遠瞧不見我,我也永遠瞧不見你。」成崑雙手按著自己眼睛,痛哼一聲,並不回答。

群雄面面相覷,那想到這一場師徒相拚竟會如此收場。謝遜朗聲説道︰「我謝遜作惡多端,原没想能活到今日,天下英雄之中,有那一位的父兄師友曾爲謝某所害,便請來取了謝某的性命去。無忌,你不得阻止,更不得事後報復,免增你義父罪孼。」張無忌含泪答應。群雄中雖有不少人與他怨仇極深,但見他報復自己全家血仇,只是廢去成崑的武功,他此刻武功也毀了,若是上前刺他一劍,打他一拳,却也不是英雄行徑。人叢中忽然走出一條漢子,説道︰「謝遜,我父親一指鎭普南邱英雄傷在你的拳下,我給先父報仇來啦!」説著走到身前。

謝遜道︰「不錯,令尊是在下所害,便請邱兄動手。」那姓邱的漢子拔刀在手,走上兩步。張無忌心中一片混亂,若不出手阻止,義父眼下便要命喪這漢子刀下,但若將這漢子打發了,只怕反令義父有生之年,更增煩惱,何況他雙目已盲,武功全失,活在世上是否尚有人生之樂,實在也難説得很。他身子發顫,不由自主的也踏上了兩步。謝遜喝道︰「無忌孩児,如你阻人報仇,對我大大的不孝。我死之後,你到地牢中細細察看便知一切。」那姓邱的漢子舉刀當胸,突然眼中垂下泪來,一口唾沫,吐到了謝遜臉上,哽咽道︰「先父一世英雄,如他老人家在天之靈,見我手刃一個武功全失的盲人,定然惱我不肖\dash{}」嗆{\upstsl{啷}}一聲,單刀落地,掩面奔入人叢之中。

跟著又有一個中年婦人走出,説道︰「謝遜,我爲兄長陰陽判官秦鵬飛報仇來啦。」走到謝遜面門,也是一口唾沫吐到了他臉上,大哭走開。無忌見義父接連受辱,始終直立不動,心中痛如刀割。須知武林豪士於生死看得甚輕,却決計不能受辱,所謂「士可殺而不可辱」,這二人每人一口唾沫吐在他的臉上,實是最大的侮辱,謝遜却是安然忍受,可知他於過去所作罪孼,當眞痛悔到了極點。人叢中一個又一個的出來,有的打謝遜兩記耳光,有的踢他一脚,更有人破口痛罵,謝遜始終低頭忍受,既不退避,更不惡言相報。

如此接連三十餘人,一一將謝遜侮辱了一番,到最後一名長鬚道人出來,稽首説道︰「貧道太虛子,我兩位師兄,命喪謝大俠拳底。貧道今日得見謝大俠仁心英風,深自慚愧,貧道劍下也曾殺過無數黑白兩道的豪傑。我若找你報仇,旁人也可找我報仇。」説著拔出長劍,左手振指一彈,{\upstsl{噹}}的一聲,長劍斷兩截。他將斷劍投在地下,稽首行禮而去。

群雄竊竊私議,這太虛子江湖上其名不著,武功却是如此了得,更難得的是心胸寬廣,能彀自責,看來再没有人出來向謝遜爲難了。不料群議未畢,峨嵋派中走出一名中年女尼,走到謝遜身前,説道︰「殺夫之仇,我也是一口唾沫了結了吧!」説著口一張,一口唾沫向謝遜額頭吐了過去。那知這口唾沫勢夾勁風,中間竟是混著一枚棗核鋼釘。謝遜聽得風聲有異,微微苦笑,並不閃避,心想︰「我此刻方死,已然遲了。」

驀地裡黃影一閃,那黃衫女子衣袖拂處,將這枚棗核釘捲在袖中,喝道︰「這位師太法名如何稱呼?」那女尼見突擊不中,臉中微驚惶之色,道︰「我叫靜照。」黃衫女子道︰「{\upstsl{嗯}},靜照,靜照。你出家之前的丈夫叫什麼名字?怎地爲謝大俠所害?」靜照怒道︰「這跟你有什麼相干?要你多管什麼閒事?」黃衫女子道︰「謝大俠懺悔前罪,若是有人爲報父兄師友的大仇,縱是將他千刀萬剮,謝大俠均是甘願忍受,旁人原也不能干預。但若有人心懷叵測,意圖混水摸魚,殺人滅口,那可人人管得。」靜照道︰「我和謝遜無怨無仇,何必要殺人滅\dash{}」底下這「口」字尚未説出,斗然間知道錯了話,急忙停住,臉色慘白,不禁向周芷若望了一眼。黃衫女子道︰「不錯,你跟謝大俠無怨無仇,何故要殺人滅口?哼,峨嵋派靜字輩十二女尼之中,靜玄、靜虛、靜空、靜慧、靜照,均是閨女出家,何來丈夫?」靜照一言不發,掉頭便走。

黃衫女子喝道︰「這麼容易便走了?」搶上兩步,伸掌往她肩頭抓去。靜照斜身卸肩,避開了她這一抓。黃衫女子右手食指戳向她的腰間,跟著飛起一脚,踢中了她腿上環跳穴。靜照哼了一聲摔倒在地。黃衫女子笑道︰「周姑娘,這殺人滅口之計好毒啊。」

\chapter{共舉義旗}

周芷若冷冷的道︰「靜照師姊向謝遜報仇,説什麼殺人滅口?」她左手一揮,道︰「這児無數名門正派的弟子,不明邪正之别,甘願與旁門妖魔混在一起。峨嵋派可犯不著趕這淌混水,咱們走吧。」峨嵋派一聲照應,都站了起來,有些人望著躺在地下的靜照,不知掌門人是否發令相救,還是置之不理。

空智走到成崑身前,喝道︰「圓眞,快叫人放開方丈。老方丈若有三長兩短,你的罪孼可就更大了。」成崑苦笑道︰「事已至此,大家同歸於盡。此刻我便要放了空聞這老和尚,然已來不及了。你又不是瞎子,這時還瞧不見火燄嗎?」空智一呆,回頭向峰下瞧去,果見寺中黑煙和火舌冒起,驚道︰「達摩堂走火!快,快去救火。」群僧一陣大亂,紛紛便要奔下山去。忽見達摩堂四周一條條白龍般的水柱,齊向火燄中灌落,霎時間便將火頭壓了下去。稟報道︰「啓稟師叔祖,圓眞手下的叛徒縱火焚燒達摩堂,幸得明教洪水旗下衆英雄仗義,已將烈火撲滅。」空智走到張無忌身前,合什禮拜,説道︰「少林千年古刹免遭火劫,全出張教主大恩大德,合寺僧侶粉身難報。」張無忌還禮遜謝,道︰「此事份所當爲,大師不必多禮。」

空智道︰「空聞師兄被這叛徒囚於達摩院中,火勢雖滅,不知師兄安危如何。張教主與衆位英雄少待,老衲須得前去察看。」成崑哈哈大笑,道︰「空聞身上澆滿了牛豬油,火頭一起,早已了帳。洪水旗救得了達摩院,救不得老方丈。」忽然峰腰傳來一人聲音,説道︰「洪水旗救不得,還有厚土旗呢。」却是范遙的聲音。他話聲甫畢,便和厚土旗掌旗使顏垣奔上峰來,兩人擕扶著一位老僧,正是少林寺方丈空聞。但見三人均是衣衫焦爛,頭髮鬚眉都被燒得稀稀落落,狼狽不堪。空智急步上前,抱住空聞,叫道︰「師兄,師兄,你身子安好?師弟無能,罪該萬死。」空聞微笑道︰「全仗這位范施主和顏施主從地道中穿將出來相救,否則你我焉有再見之日。」空智駭然道︰「明教厚土旗穿地之能,一神至此。」向范遙、顏垣深禮致謝,並道︰「范施主,老僧先前無禮冒犯,尚請原宥。大都萬法寺之約,老僧是不敢去的了。」要知武林人士訂下比武的約會,若是食言不到,比之較技服輸可要丟臉萬倍。空智自甘毀約,可知他對范遙冒險相救師兄的大德,實是感激無已。兩人本來互相佩服,經此一事,更是傾心接納,從此成爲至交好友,此是後話不提。

原來成崑事先計劃周詳,於英雄大會舉行的前夕,出其不意的點中了空聞穴道,將他囚在達摩院中。院中放滿硝磺柴草等引火之物,分派心腹看守,脅迫空智一切須聽自己吩咐,否則立時縱火,焚死空聞。其後事與願違,一切均非事先意料所及,一敗塗地之餘,便傳出號令,命心腹縱火,那是他破釜沉舟的最後一著棋子。只盼群雄與僧衆忙於救火,他心腹人等便可乘亂將他救下山去。不料楊逍一到少室山下,未曾與無忌見面,便命厚土旗打下地道,通入少林寺中,本意是設法相救謝遜,可是謝遜却並非囚於寺内。達摩大石像的突然掉換,便是厚土旗人衆在地道内暗中所使手脚。

後來無忌與周芷若聯手攻打金剛伏魔圏,待得成崑現身,正式與空智破臉,趙明與楊逍便瞧出端倪。二人計議之下,命范遙率領洪水、厚土兩旗,潛入寺中俟機相救空聞。只是成崑的佈置極是毒辣,空聞雖是救出,却燒死了三名厚土旗的兄弟。

范遙與顏垣冒煙突火,救出空聞,但三人也被烈火燒得鬚眉倶焦,若不是從地道中脱險,勢必葬身火窟,待得洪水旗撲滅火燄,已是遲了。那達摩院及鄰近幾間僧舍爲火所焚,幸而未曾蔓延,大雄寶殿、藏經閣、羅漢堂等要地未遭波及。

空智低聲與空聞商議了幾句,傳下法旨,將成崑手下黨羽盡數拘禁於後殿待命。成崑在少林寺日久,結納的徒黨著實不少,但魁首受制,方丈出險,衆黨羽眼看大勢已去,當不敢抗拒,在羅漢堂首座率領僧衆押送之下,垂頭喪氣的下峰。無忌回首看周芷若時,祇見峨嵋派人衆早已乘亂走了,祇靜照仍是躺在地下。

無忌走到那黃衫女子跟前,長揖説道︰「張無忌承姊姊兩番援手,大德不敢言謝。只盼示知芳名,以便無忌日夕心中感懷。」黃衫女子微微一笑,説道︰「終南山後,活死人墓,神鵰俠侶,絶跡江湖。」説著{\upstsl{襝}}衽爲禮,手一招,帶著身穿黑衫白衫的八名少女,飄然而去。無忌追上一步,道︰「姊姊請留步。」那黃衫女子竟不理會,自行下峰去了。丐幫的小幫主史紅石叫道︰「楊姊姊,楊姊姊!」只聽得峰腰間傳來那女子的聲音道︰「丐幫大事,請張教主一力承擔。」張無忌朗聲道︰「無忌遵命。」那女子道︰「多謝了!」這「多謝了」三字遙遙送來,相距已遠,仍是清晰異常。無忌心下帳惘,一寧神,走到謝遜身邉,只叫了聲︰「義父!」泪如雨下。謝遜笑道︰「痴孩子!你義父承三位高僧點化,大徹大悟,畢生罪孼一一化解,你該當代我歡喜纔是,有什麼可難過的?我廢去武功有何足惜,難道將來再用以爲非作歹麼?」無忌應道︰「是!」謝遜走到空聞身前,跪下説道︰「弟子罪孼深重,盼方丈收留,賜予剃度。」空聞尚未回答,渡厄道︰「你過來,老僧收你爲徒。」謝遜道︰「弟子不敢望此福緣。」要知他拜空聞爲師,乃是「圓」字輩弟子,若拜渡厄爲師,敘「空」字輩排行,和空聞、空智便是師兄弟稱呼了。渡厄喝道︰「咄!空字是空,圓亦是空,我相人相,好不懵懂!」謝遜一怔,登即領悟,什麼師父弟子,於佛家盡是虛幻,便説偈道︰

\begin{quotation}
\qyh{}師父是空\hskip8pt弟子亦是空\hskip8pt無罪無業\hskip8pt無德無功!」
\end{quotation}

渡厄哈哈笑道︰「善哉,善哉!你歸我門下,仍是叫謝遜,你懂了麼?」謝遜道︰「弟子懂得。牛屎謝遜,皆是虛影,身既無物,何況於名?」

要知謝遜文武全才,於諸子百家之學,無所不窺,一旦得渡厄點化,立悟佛家精義,自此歸於佛門,終成一代碩德高僧。張無忌又是歡喜又是悲傷,一時説不出話來。渡厄道︰「去休,去休!纔得悟道,莫要更入魔障!」擕了謝遜之手,與渡劫、渡難緩步下峰。空聞、空智、張無忌等一齊躬身相送。金毛獅王三十年前名動江湖,做下了無數驚世駭俗的事來,今日身入空門,群雄無不感嘆。

空聞説道︰「衆英雄光臨敝寺,説來慚愧,敝寺忽生内變,多有得罪,招待極是不周。衆英雄散處四方,今日一會,未知何時重得相聚,且請寺中坐地。」當一群雄下峰入寺。少林寺中開出素餐接待。衆僧侶做起法事,替會中不幸喪命的英雄超度。群雄逐一祭弔致哀。

大事雖了,張無忌心中仍有許多不明之處,謝遜去得匆匆,不少疑團未及相詢,但料想関鍵所在,必與周芷若有関。他宅心忠厚,念及舊情,心想這些疑團也不必一一剖明,以致更損周芷若的名聲。他用過齋飯後,與史紅石及丐幫諸長老在西廂房中敘話,商議丐幫的大事,忽有教衆進來説道︰「教主,武當張四俠到來,有要事相商。」無忌一驚站起。

張無忌聽到張松溪突然到來,吃了一驚︰「莫非太師父有什麼不測?」急忙搶步出去,來到大殿,向張松溪拜倒在地,見他神色並無異狀,這纔放心,問道︰「太師父安好?」張松溪道︰「師父他老人家安好。我在武當山下得到訊息,元兵鐵騎二萬,開向少林寺來,窺測其意,顯是不利於英雄大會,是以星夜前來報信。」無忌道︰「咱們急速説與方丈知曉。」當下二人同至後院,告知空聞。空聞沉吟道︰「此事牽涉甚大,當與群雄共議。」於是命寺衆撞起鐘來,邀集衆英雄同到大雄寶殿之中。張松溪一説此訊,群雄都是一驚,登時便紛紛議論起來。

血氣壯盛的便道︰「乘著天下英雄在此,咱們迎下山去,先殺他一個措手不及。」老成持重的則道︰「元兵來往調動,原是常事,未必是來跟咱們爲難。」張松溪道︰「在下會聽蒙古話,親耳聽到韃子的軍官號令,確是殺向少林寺來。」空聞道︰「衆位英雄,看來朝廷得知咱們在此聚會,只道定是不利於朝廷。咱們人人身有武功,原是要殺韃子,兵來將擋,水來土淹,何足道哉?\dash{}」他話未説完,群雄中已有人喝起了采來。空聞續道︰「只是咱們江湖豪士,慣於單打獨鬥,比的若不是兵刃拳脚,便是内功暗器。這等馬前馬後、長槍大戟的交戰,咱們頗不擅長,依老衲之見,不如衆英雄便即散去如何?」

群雄面面相覷,默不作聲。張無忌道︰「咱們若不是就此散去,一來韃子只道咱們畏懼於彼,不免長他人志氣;二來少林寺中諸位師父們如何?」空聞微笑道︰「元兵來到寺中,一見寺中皆是僧人,並無江湖豪士,那也無可如何。這叫作乘興而來,敗興而返。」群雄知道空聞所以如此説,實是出於一番好意,要知道這次英雄大會乃少林派所邀集,雖不願由此生禍,致令群雄血濺少室山頭。但群雄個個都是血性之人,要他們臨敵退縮,那是決計不肯的。何況朝廷既是出動大軍,決不能撲了個空便即安然返防,定要騷擾少林,説不定將衆僧侶盡數擒拿而去,一把火將寺燒了。蒙古兵向來暴虐,殺人放火,原是慣事。楊逍説道︰「方丈與衆位英雄在此,在下本是不該多嘴。然韃子施虐,人人有抗敵之責。以在下之見,咱們設法將韃子引了開去,在别的地方好好跟他們鬥上一鬥,免得千年古刹,受這戰火之厄。」

群雄紛紛叫好,説道︰「正該如此。」正議論間,忽聽得寺門外馬蹄聲奔得甚急,兩騎馬疾馳而來。蹄聲到門外戛然而止,兩名漢子在知客僧接引之下,匆匆進來。群雄一看服色,却是明教的教衆。二人走到張無忌身前,躬身行禮,一人稟報道︰「啓稟教主︰韃子兵先鋒五千,攻向少林寺來,説道寺中諸位師父們聚衆造反,要踏平少林。凡是光\dash{}光\dash{}」空聞微笑道︰「你要説光頭和尚,是不是?那也不用忌諱,但説便是。」那人道︰「一路上好多位僧人已被韃子兵殺了。他們説道︰『光頭的都不是好人,有頭髮的也不是好人,凡遇身邉帶兵刃的便一槩殺了。』」

許多人哇哇叫了起來,都道︰「不跟韃子兵拚個你死我活,恥爲黃帝子孫。」其時宋室淪亡雖是將近百年,但草莽英豪,始終將蒙古官兵視成夷狄,不肯服其管束。各門各派,各幫各會相互間固是私鬥不休,然而不論如何結下深仇大怨,從來無人肯去借朝廷官府之力來爲難對方。這次英雄大會之中,絶大多數豪士均未能一顯身手,這時聽説蒙古殺到,各人熱血沸騰,盡皆奮身欲起。張無忌朗聲説道︰「衆位英雄,今日正是男児漢殺敵報國之時。少林寺英雄大會,自此名揚千秋!」

大殿上歡呼聲,喊叫聲,嚷成一片。張無忌道︰「便請空聞方丈發號施令,咱明教上下,盡聽指揮。」空聞道︰「張教主説那裡話來?敝派僧衆雖曾學過一些拳脚,但於行軍打仗,却是一竅不通。近年來明教創下偌大事業,江湖上誰個不知聞?唯有明教人衆,方足與韃子大軍相抗。咱們公推張教主爲武林盟主,相率天下豪傑,與韃子周旋。」張無忌還待遜辭,群雄已大聲喝采。要知張無忌雖年輕不足服衆,但武功之強,適才力鬥少林三僧時已是人所共見,而明教韓山童、徐壽輝、朱元璋等各路人馬,在淮泗、豫鄂各地起事,連獲勝利,更非其餘門派可及。各派各幫的豪士均想除了明教之外,確是無人能當此盟主的大任。

張無忌道︰「這盟主一席,責任奇重,在下於用兵一道,實非所長,還請各位另推賢能的爲是。」正謙讓間,忽聽得山下喊聲大振,鋭金旗的兩名教衆奔馳入殿,報道︰「蒙古兵殺上山來了。」張無忌道︰「鋭金、洪水兩旗,先擋頭陣。周顚先生、鐵冠道長,你兩位各助一旗。」周顚和鐵冠道人應聲而出。

局勢緊急,不容無忌再行推辭,祇得分派道︰「説不得師父,請你持我聖火令去,就近調本派援兵,上山應援。」説不得接命而去。大殿中衆英雄武功雖高,却均是不相統屬的烏合之衆,聽得元兵殺到,各抽兵刃,紛紛湧出。楊逍低聲道︰「教主,你若不發號施令,衆人亂殺一陣,那是非敗不可。」無忌點了點頭,當即搶步出殿,來到半山亭中察看,祇見蒙古兵先鋒千餘已攻到山腰,但被鋭金旗一輸硬弩標槍,驅了回去。放眼遠望,一隊隊蒙古兵蜿蜒而來,軍容甚盛。其時距成吉斯汗與拔都威震異域之時已遠,但蒙古鐵騎竟習練有素,仍是舉世無匹的精兵。

忽聽得左首喊聲大震,許多女尼和少年男女逃上山來,却是峨嵋派人衆,想是途遇蒙古官兵,又被逼了回來。周芷若和靜慧、靜照等浴血斷後,十多名漢子抬著擔架等物,被蒙古兵包圍在内,周芷若率衆數度衝殺了數十名蒙古兵,始終無法救出陥入重圍的同門,無忌暗叫道︰「不好!這擔架上的是宋師哥!」叫道︰「烈火旗兩旗掩護!韋兄、范楊二使,隨我救人。」縱身衝將下去。兩名蒙古兵挺長矛刺將過來。無忌一手抓住一枝長矛,運勁一抖,兩名蒙古兵摔將下去。他掉轉矛頭,雙矛銀光閃閃,猶似雙龍入海般捲入人叢。韋一笑、楊逍、范遙、彭瑩玉等跟隨其後,蒙古兵當之辟易,登時將周芷若等一干人都隔在身後。范遙呼一拳擊出,將一名蒙古兵十夫長的臉打得稀爛,搶過擔架中的傷者,夾在脅下,轉身便走。無忌見周芷若滿臉是血,又衝入蒙古兵中,忙叫道︰「芷若,芷若,宋大哥救回來啦!」周芷若並不理會,揮鞭向前攻打,只是山道狹窄,擠滿了人,一時衝不過去。無忌見尚有兩名峨嵋弟子抬著個擔架,陥入包圍,正挺兵與蒙古兵死戰,心道︰「難道宋師哥是在那個擔架之上?」斜身躍起,兩柄長矛在山壁上交互刺戳,以手代足,如踏高蹺般搶了過去。相距尚有丈餘,只見兩名峨嵋弟子先後中刀中箭,骨碌碌的滾下山去。

無忌大吃一驚,飛身躍起,左手長矛阻住擔架下落,見擔架中人那人全身都裹在白布之中,只露出了一張臉,正是宋青書。無忌抛去長矛,將他橫抱在手,只覺他身子沉重異常,白布中硬崩崩的似乎尚有别物。一時也不及細想,只怕扭動他震碎了的頭骨,左閃右避,躱開蒙古兵攢刺來的馬刀長矛,脚下却是走得平穩異常。只見張松溪和殷利亨雙雙攻到,手持長劍,護在他身子兩側。兩柄長劍倏刺倏收,蒙古兵紛紛中劍。

張松溪和殷利亨二人這麼一擋,無忌抱著宋青書穩穩的走一山來。數百蒙古兵列隊上衝,彭瑩玉叫道︰「烈火旗動手!」這一聲令下,烈火旗的教衆從噴筒中噴出石油,一枝枝火箭射將出去,登時烈燄奔騰,當先的二百餘名蒙古兵身上著火,一團團火球般滾下山去。那邉廂洪水旗從水龍中澆出毒水,也有數百名蒙古兵身中烈性毒水。死傷狼藉。群雄乘機上前衝殺。山腰裡蒙古兵的萬夫長下令鳴金收兵,衆兵將前隊改成後隊,強弓射住陣脚,不令群雄追擊,緩緩退了下去。彭瑩玉嘆道︰「韃子雖敗不亂,確的是天下精兵。」只見蒙古兵直退到山脚下,如扇面般散開,看來一時不致再上山進攻。無忌下令道︰「鋭金、洪水、烈火三旗守住上山要道。巨木、厚土二旗,急速伐木搬土,構築壁壘,以防敵軍衝擊。」五行旗各掌旗使齊聲接令,分别指揮下屬佈防。

群雄先前大都身負武功,均想縱然殺不盡韃子官兵,若求自保,總非難事。但適纔一陣交鋒,見識到了蒙古軍的威力,才知行軍打仗,和單打獨鬥的比武實是大不相同。千千萬萬人一擁而上,勢如潮水,如周芷若這等厲害之極的人物,在人潮中也是無所施其技。四面八方都是刀槍劍戟,亂砍亂殺,平時所學的什麼見招拆招,内勁外功,全都用不著。若不是明教五行旗以陣法抵擋陣法,這時少室山頭只怕已是慘不堪言,少林寺也是烈火中成了一片瓦礫了。待見蒙古兵退下,群雄這纔紛紛議論,方想到爲什麼前朝儘多英雄豪傑之士,却將大好江山淪亡在韃子手中。倒是少林僧衆頗有規律,一隊隊少年僧衆手執禪杖戒刀,在年長僧侶率領之下,分布各處要地守禦,但寡不敵衆,顯是也擋不住二萬餘蒙古精兵的全力衝擊。

無忌將宋青書輕輕放在地下,一探他的鼻息,幸喜尚有呼吸,回頭想招呼周芷若過來,却不見人影,問道︰「宋夫人呢?」衆人適纔忙於驅退蒙古官兵,誰都没留心周芷若到了何處。峨嵋群弟子這時對明教也消了幾分敵意,均説没見到掌門人。無忌怕宋青書在混亂中身上受傷,於是解開裹在他身上的白布察看。

他身上白布一共裹了三層,待得第三層解開,嗆{\upstsl{啷}}{\upstsl{啷}}幾聲響亮,跌出四件斷折了的兵刃出來。無忌吃了一驚,叫道︰「屠龍刀,倚天劍!」群雄聽得「屠龍刀」三字,紛紛圍了上來,但見屠龍刀斷成了兩截,倚天劍也是斷成了兩截。

無忌提起半截屠龍刀只覺入手仍是頗爲沉重,心中百感交集,自己父母爲此刀而喪命,近二十餘年來,江湖上紛擾不休,都是爲了此刀。群雄聚雙少林,主要也是爲了這柄寶刀,那想到寶刀出現,竟是刀劍齊折,已無用處,他舉起斷刀一看,只見斷截之處中空,可藏物事,那倚天劍也是這樣,但刀劍中均是空空如也,如果曾藏過什麼物事,却也早給人取去了。楊逍嘆道︰「周姑娘一身驚人武功,原來是從此刀劍中而來。」

張無忌雖是心地仁厚,却也決非蠢人,他一看到斷刀斷劍,心下已是恍然︰原來小島上當晩刀劍齊失,却是周芷若取了去。不知她使下什麼手脚,放逐趙明、害死殷離,再以刀劍互砍,兩柄天下最鋒鋭的利器就此兩敗倶傷。她取出藏在刀劍中的祕笈,暗中修練。他越想越是明白︰「是了,當時在小島之上,我用九陽神功替她驅除毒素,她體内竟生出一種怪力,隱隱與我的神功相抗,越到後來,這股怪力越強,顯是她修習的内功日有進境。唉!時日迫促,她爲了急於求成,不及好好紮下内功根基,以致所習的均是可以速成的陰毒外功,雖然厲害,終究達不到眞正爐火純青的峰巓境界。」

張無忌正自沉吟,鋭金旗掌旗使吳勁草走上前來,説道︰「啓稟教主,屬下是鐵匠出身,學過鑄造刀劍之法,待屬下試試,不知是否能將這寶刀寶劍接續完好。」楊逍喜道︰「吳旗使鑄劍之術,天下無雙,教主不妨命他一試。」無忌點頭道︰「這柄利器如此斷了,確也可惜。吳旗使試試也好。」吳勁草向烈火旗掌旗使夏炎説道︰「鑄刀鑄劍,関鍵在於火候,須得夏兄相助一臂之力。看這模樣,韃子一時不會攻山,咱哥児倆便即動手如何?」夏炎笑道︰「生柴燒火,那是兄弟的拿手本事。」

於是二人指揮屬下,搬石搭起一座爐子,這爐子搭得甚高,只露出一個不到一尺口徑的火孔。烈火旗中各種燃料均是現成,頃刻間便生起一爐熊熊大火。吳勁草目不轉睛的望著爐火,他身旁放著十餘件兵刃,一看爐火變色,便將兵刃放入爐中試探火性,待見爐火自青變白,當下雙手各執鋼鉗,鉗起兩截屠龍刀,拚在一起,拿到火燄中鎔燒。衆人見他上身脱得赤條條地,火星濺在身上,恍如不覺,直是全神貫注,心不旁鶩。無忌心想︰「鑄造刀劍雖是小道,其中却也有大學問、大本領在。若是尋常鐵匠,單是這等高熱已便抵受不住。」忽聽得拍拍兩聲,拉扯風箱的兩名烈火旗教衆暈倒在地。夏炎和烈火旗掌旗使搶上前去,拖開暈倒的兩人,親自拉扯風箱鼓風。這兩人内功修爲均是不弱,這一使勁鼓風,爐火直竄上來,火燄高達丈許,蔚爲奇觀。

吳勁草突然叫道︰「不好!」縱身後躍,一臉沮喪之色。衆人吃了一驚,看他手中時,只見兩柄鋼鉗均已燒鎔,屠龍刀却是毫無動靜。吳勁草搖頭道︰「屬下無能。這屠龍寶刀果然是名不虛傳。」夏炎和烈火旗副使暫停扯風,退在一旁,他二人全身衣褲都已汗濕,便似從水中爬起來一般。趙明忽道︰「無忌哥哥,你那些聖火令,不是連屠龍刀也砍不動麼?」無忌道︰「啊,是了!」六枚聖火令除將一枚交於説不得調山下兵,剩下尚有五枚,他從懷中取了出來,交給吳勁草道︰「刀劍不能復原,那也罷了。聖火令是本教世傳的寶物,可不能損毀。」吳勁草接令一看,見五枚聖火令非金非鐵,堅硬無比,在手中掂了掂斤兩,低頭沉思,臉上神色十分古怪。

無忌道︰「若無把握,不必冒險。」吳勁草一凜,從沉思中醒轉,説道︰「屬下多有不是,請教主原宥。這聖火令乃用白金、玄鐵,混和金剛砂等物鑄就,烈火決不能熔。屬下大是疑惑,不知當年如何鑄成,眞乃匪夷所思。一時想出了神。」趙明笑道︰「將來只怕你得上波斯走一遭,向他們的高手匠人請教請教。你瞧,這些聖火令上還刻有花紋文字。以屠龍刀、倚天劍之利,尚且不能損它分毫,這些花紋文字又用什麼傢伙刻它上去。」吳勁草道︰「要刻花紋文字,却倒不難。那是在聖火令上遍塗白臘,在臘上彫以花紋文字,然後注以烈性酸液,以數月功夫,慢慢腐蝕。待得刮去白臘,花紋文字便刻成了。小人所不懂的乃是鎔鑄之法。」夏炎叫道︰「喂,到底幹不幹啊?」吳勁草向無忌道︰「教主放心,夏兄弟的烈火雖然厲害,却損不了聖火令分毫。」

夏炎心中却有些惴惴,道︰「我盡力扇火,若是燒壞了本教聖物,我可吃罪不起。」吳勁草微笑道︰「量你也没有這等能耐,一切由我擔待。」於是將兩枚聖火令夾住屠龍刀的半截,再將兩枚聖火令夾住寶刀的另外半截,然後取過兩把新鋼鉗,分别夾住四枚聖火令,將寶刀放到爐火上再度燒了起來。爐火中的烈燄越衝越高,直燒了大半個時辰,眼看吳勁草、夏炎烈火旗副旗使在烈火烤炙之下,越來越是神情委頓,漸漸要支持不住。

范遙向周顚使個眼色,左手輕輕一揮,兩人一齊搶上前去,接替了夏炎與烈火旗副旗使的位子,用力扯動風箱。范周二人的内力比之那二人又自不同,爐子中筆直一條白色火燄,直衝而起。吳勁草突然喝道︰「顧兄弟,動手!」鋭金旗的掌旗副使顧孟魯手持利刃,奔到爐旁,白光一閃,一刀便向吳勁草胸口刺去。旁觀群雄無不失色,齊聲驚呼。吳勁草赤裸裸的胸膛上鮮血射出,一滴滴的落在屠龍刀上,血液遇熱,立化青煙嬝嬝冒起。吳勁草大叫︰「成了!」退了數步,一交坐在地下,只見那屠龍刀的兩段刀身已鑲在一起。衆人這纔明白,原來鑄造刀劍的大匠每逢鑄器不成,往往滴血刃内,古時干將莫邪夫婦甚至自身投入爐内,這纔鑄成無上的利器。吳勁草此舉,可説是古代大匠的遺風了。

無忌忙將吳勁草扶起,察看他的傷口,見這一刀入肉不深,並無大礙,當下用金創藥替他敷上,包紮了傷口,説道︰「吳兄何必如此?此刀能否續上,無足輕重,却讓吳兄吃了這許多苦。」吳勁草見他不看聖火令,不看屠龍刀,却先來看自己的創傷,心下好生感激,道︰「皮肉小傷,那算得什麼?倒讓教主費心了。」站起身來,提起屠龍刀一看,只見接續處天衣無縫,祇是隱隱有一條血痕,不禁十分得意。無忌看那四枚入爐燒過的聖火令果然絲毫無損,接過屠龍刀來,往兩根從蒙古兵手中搶來的長矛上砍去,嗤的一聲輕響,雙矛應手而斷,當眞是削鐵如泥,群雄大聲歡呼,均説︰「好刀!好刀!」

吳勁草捧過兩截倚天劍,想起鋭金旗掌旗使莊錚以及本旗的數十名兄弟,均是命喪此劍下,忍不住眼泪奪眶而出,説道︰「教主,此劍殺了我莊大哥,殺了我不少好兄弟,吳勁草恨此劍入骨,不能爲他接續。願領教主罪責。」説著泪如雨下。張無忌道︰「這是吳大哥的義氣,何罪之有?」拿起兩截斷劍,走到峨嵋派靜慧身前,説道︰「此劍原是貴派之物,仍請大師收管,日後交給周\dash{}交給宋夫人。」靜慧一言不發,將兩截斷劍接了過去。

無忌拿著那柄屠龍刀,微一沉吟,向空聞道︰「方丈,此刀是我義父得來,現下我義父皈依三寶,身屬少林,此刀該當由少林派執掌。」空聞雙手亂搖,説道︰「此刀數易其主,最後是張教主從千軍萬馬中搶來,人人親眼得見,又是貴教吳大哥接續復原。何況今日天下英雄共推張教主爲武林盟主,論才論德,論淵源,論名位,此刀由張教主掌管,那是天經地義的了。」群雄齊聲附和,均説︰「衆望所歸,張教主不必推辭。」無忌無奈,只得收下,心想︰「若憑此刀而號令天下武林豪傑,共驅胡虜,亦是一大快事。」只聽得人叢中許多人紛紛説道︰「武林至尊,寶刀屠龍,號令天下,莫敢不從!」下面本來還有「倚天不出,誰與爭鋒?」這兩句,但衆人看到倚天劍斷折後不能接續,這兩句話誰也無人再提了。明教鋭金旗下諸人與那倚天劍實有切齒的大恨,今日眼見屠龍刀復原如初,倚天劍却成了兩截斷劍,無不稱快。

這時洪水旗人衆從廟内抬了一口大鐵鍋出來,架在爐火之上,煮起一大鍋油滾油,只待蒙古兵衝上山來,便將滾油噴出傷人。少林寺是千年古刹,廟中所藏的香油堆滿數屋,可説用之不盡。

衆人忙了半天,肚中都餓了,除了明教五行旗及少林寺的半數僧侶留在各處要道守禦,餘人由僧衆接進寺裡吃齋。堪堪天色將晩,無忌躍到一株高樹之上,向山下敵軍瞭望,只見蒙古兵東一堆,西一堆的圍在山下,炊煙四起,正自埋鍋造飯。

\chapter{追奪祕笈}

張無忌躍下樹來,對韋一笑道︰「韋兄,天黑之後,你去探探敵軍的情勢,瞧他們是否會在夜中突襲。」韋一笑接令而去。楊逍道︰「教主,我看韃子在前山受挫之後,今日多半不會進攻,倒要防備他們自後山偸襲。」無忌道︰「不錯。咱們到那邉山峰上瞧瞧。」於是帶同楊逍、范遙、厚土旗掌旗使顏垣,走向曾經囚禁謝遜的那個山峰。趙明道︰「我也去!」跟著同行。

四人到了峰頂,眺望後山,不見動靜。無忌撫摸三株斷折了松樹,想起今日這番劇戰,實是兇險之極,突然間心中一動︰「義父叫我看看地牢中的石壁,險些忘了。」蓋在地牢口上的大石被他推開後,没再掩上,他輕輕往地牢中一跳,見是個丈許見方的石室。其時暮靄蒼茫,地牢中更是陰暗,無忌從懷中取出火摺,打著了火,見四面石壁上各刻著一幅圖畫。這四幅畫均是用尖石劃成,筆畫簡單,神韻却甚生動。東首第一幅畫上繪著兩個女子臥在地下,另一個女子伸左手點她穴道,右手到她懷中去取什麼物事,旁邉冩著「取藥」二字。南首第二幅畫有一艘海船,一個女子將另一個女子抛向船上,冩著「放逐」二字。無忌額頭冷汗涔涔而下,心道︰「原來果眞如此。芷若點了明妹的穴道,從她懷中取了十香軟筋散出來,下在我和義父的飲食之中,又將明妹擲上波斯人的海船,逼著她們遠駛。她幹麼不乾脆將明妺殺了?{\upstsl{嗯}},倘若留下明妹的屍身,不能滅跡,那就無法嫁禍於她。如此説來,表妹被害,自也是她下的毒手了。」在這幅圖的左角,又畫著兩個男子,一個睡得甚沉,另一個滿頭長髮,側耳傾聽。無忌暗暗心驚︰「原來芷若幹這場傷天害理之事,義父一一聽在耳中。他老人家好大的涵養功夫,在島上竟是不露半點聲色。是了,那時我和義父服了十香軟筋散後,全身功力盡失,性命在芷若掌握之中。無怪義父當時一口咬定是明妹所爲,顯得憤慨無比。他知我性子老實,若是跟我説了,我言語舉止之中,定會洩露機密了。」

再看西首第三幅圖,繪的是謝遜端坐,周芷若在他身後忽地襲擊,外面湧進一批丐幫幫衆。這情景正與趙明在大都遊皇城的戲文中命人所扮一模一樣,待再要去看第四幅圖,手中火摺燃盡,倏地熄滅。他叫道︰「明妹,你來,拿火摺給我一用。」趙明點著火摺,跳入地牢之中,一見那幾幅圖畫,已是了然於胸。無忌再看第四幅,乃是十多名漢子抬著謝遜行走,遠處有一個少女在樹後窺探。這四幅圖筆法極佳,但除了謝遜自己之外,旁人的面貌却極模糊,分辨不出這少女是誰。無忌微一沉吟,已明其理︰「義父失明之時,連我也還没出世,他只認得我和明妹,芷若等人的聲音,却不知咱們的相貌如何。圖畫中是畫不出來的。指著那少女道︰『這個是你呢,還是周姑娘?』趙明道︰「是我。成崑到丐幫去將謝大俠劫了出來,命人送來少林寺囚禁,他自己却一路上留下明教的記號,引得你大児圏子。我數度想劫奪謝大俠,都没成事,最後一次功敗垂成,只割了他一叢頭髮,作爲信物,讓你做不得新郎,眞是萬分的過意不去。」

無忌心中那纔是萬分的過意不去,怔怔的望著趙明,只見她容色憔悴,雙頰瘦削,這幾個月來對她的折磨眞是彀受的了,心下好生憐惜,突然伸出雙臂,將她抱住了,顫聲道︰「明妹,是\dash{}是我對你不起。」他這麼一抱,火摺登時熄了,地牢中黑漆一團,無忌又道︰「若不是你聰明機靈,胡塗的張無忌要是一劍將你殺了,那便是如何是好?」趙明笑道︰「你捨得殺我麼?那時眞相未明,你在大都見到我,怎麼又不殺我?」

無忌呆了一呆,嘆道︰「明妹,我對你是情之所鍾,不能自已。倘若我的表妹眞的是你所殺,我不知如何是好了。這些日子來抽絲剝繭,眞相逐步大白,我雖替芷若惋惜,可也忍不住心下竊喜。」趙明聽他説得誠懇,倚在他的懷裡,良久良久,兩人都不説話,仰起頭來,但見一彎新月,斜掛東首,四下裡寂靜無聲。楊逍、范遙、顏垣三人,却是早已遠遠的避在一旁。

趙明輕輕道︰「無忌哥哥,我和你初次相遇綠柳山莊,後來一起跌入地牢,這情景不跟今天差不多麼?」無忌嗤的一聲笑,伸手抓住她的左脚,脱下了她的鞋子。那日在綠柳山莊,趙明使詭計加害,無忌無可奈何之下,以九陽神功搔她足底「湧泉穴」,令也她麻癢難當,這纔逼得她開動地牢機括。殊不料由此一搔,趙明反而對他情深一往,化敵爲友,自後生出無數事端。趙明左足足踝被他抓住,笑道︰「一個大男人,却來欺侮弱女子。」張無忌道︰「你是弱女子麼?你詭計多端,比十個男子漢還要厲害。」趙明笑道︰「多承張大教主誇讚,小女子愧不敢當。」兩人説到這裡,一齊哈哈大笑。原來這幾句對答,正是當年兩人在綠柳山莊的地牢中所説。只是當年兩人説這幾句話時滿懷敵意,今夕却是柔情無限。

無忌笑道︰「你怕不怕我再搔你的脚底?」趙明笑道︰「不怕!」無忌伸手握住了她的脚,忽聽得西北角上隱隱有呼叱之聲,他側耳傾聽,聽到遠處有勁風互擊,顯是有人鬥毆,便道︰「咱們瞧瞧去!」擕了趙明之手,從地牢中一躍而上,循聲望去,只見三個人影正向西疾馳,身法迅速異常,均是武林中第一流的高手。這時楊逍也奔到無忌身邉説道︰「不是自己人。」無忌道︰「楊左使,你和范右使留在這裡,謹防是敵人的調虎離山之計,我過去察看一下。」楊逍道︰「教主想得不錯。」

無忌伸手摟住趙明腰間,左足一登,身子便縱了下去,這一展開輕功,將趙明帶得猶如騰雲駕霧一般。遠遠眺見前面一人奔逃,後面兩人發力追逐,前面那人只是在往山後林蔭深處疾奔,但後面兩人也寸步不放鬆。無忌一提氣,脚下越來越快,追出里許,月光下已見到後面二人乃是兩個老者,正是鹿杖客和鶴筆翁。無忌剛認出二人,鶴筆翁突然左手一揚,一枝鶴嘴判官筆向前面那人身後擲去。那人迴劍一擋,{\upstsl{噹}}的一聲響,將判官筆掠起,抛向空中。就這麼緩得一緩,鹿杖客已躍到那人身旁,手中的鹿杖刺了出去。

那人斜身閃避,拍出一掌,月光正好照射在她臉上,只見她臉色極是蒼白,長髮散亂,正是周芷若。無忌吃了一驚,忙帶同趙明隱身樹後,只見鶴筆翁接住空中掉下一枝鶴嘴筆,繞到周芷若左首,和鹿杖客已成左右合擊之勢。周芷若咬牙道︰「兩個老鬼,苦苦追我,到底幹什麼?」鹿杖客︰「今日明教張無忌奪得屠龍刀、倚天劍,咱們親眼得見,刀劍中的武功祕笈已然失去,那自是在宋夫人身上了。」無忌聽了,心中一驚︰「咱們奪刀救人之時,原來這兩個老傢伙早已躱在一旁,居然没能發覺。」只聽周芷若道︰「武功祕笈確是有的,我練成之後早已毀去。」鹿杖客冷笑道︰「『練成』二字,談何容易?這屠龍刀、倚天劍號稱是武林至尊,天下英雄人人欲得之而甘心,其中所藏祕笈豈同泛泛?宋夫人的武功雖已出類拔萃,却未必已到登峰造極的地步,否則的話,一舉手便將我師兄弟二人毀了,却又何必奔逃?」

周芷若道︰「我説毀了,便是毀了,誰有空跟你多説。少陪了!」鹿杖客和鶴筆翁齊聲喝道︰「且慢!」一枝鹿杖,兩根鶴筆同時揚起,攻向周芷若兩側。

周芷若長劍揮動,月光下如根蛇狂舞。玄冥二老一杖雙筆,聯手進攻。張無忌日間只見到周芷若使鞭的功夫,這時見她劍招神光離合,若往若還,在二大高手夾擊下竟是有守有攻,絲毫不露敗象,偶爾虛實奇變,鹿杖客和鶴筆翁若非内力渾厚,幾乎要爲她長劍所傷。無忌看到數十招後,心下暗叫︰「可惜,可惜!倘若她倚天劍在手,玄冥二老便奈何她不得。眼下她如不能脱身,纏鬥到二百餘招後,只怕内力不濟。」

再鬥數十合,却見周芷若劍招愈來愈奇,十招中倒有七招是極凌厲的攻勢。無忌知她是急謀脱身,但這種打法加速運用内力,得能險中取勝果然甚好,若是偶一疏神,那便立遭兇險。他悄悄從樹後出來,只盼周芷若能自行退敵,不用自己露面,否則在危急中只好出手相救。驀地裡周芷若一聲呼叱,長劍刷刷刷向鹿杖客連刺三劍。鹿杖客閃身稍緩,没攪清她第三劍的來勢,嗤的一響,長劍從他肩頭斜斜掠過,連衫帶肉,挑破了一條半尺長的口子,便在此時,鶴筆翁雙筆脱手,向她背心猛地擲過去。這一招兵刃脱手攻敵,正是鶴筆翁輕易不肯使用的絶招。他見師兄弟二人聯手,鬥到百招以外,兀自拾奪不下一個年輕女子,於玄冥二老的威名大大有損,何況此處少室山中,敵方高手雲集,只要有人來援,那麼劫奪祕笈之舉便是功敗垂成,是以突然使出這招「雙鶴唳空」。雙筆脱手,在空中{\upstsl{噹}}的一聲互撞,一筆在上,一筆在下,分襲周芷若後腦與後腰兩處要害。

周芷若聽得身後雙筆擲到,縮身閃避,却没料到兩枚鶴嘴筆在空中互相碰撞之後,竟會忽地改變方向。她這一閃躱,讓開了襲向腦門的一筆,却讓不開飛向她腰間的一筆。張無忌縱身急躍,伸手一抄,抓住了那枚鶴嘴筆,橫掌一立,擋開鶴筆翁拍來的一掌。周芷若那想到在這性命懸於一線之際,竟會有人出手相救,雙目一閃,只得就死,鹿杖客輕飄飄一掌拍出,正中她的小腹。那是威震武林的「玄冥神掌」。周芷若氣息立閉,登時便暈了過去。

這時變故奇速,張無忌一驚之下,擲去手中鶴嘴筆反手橫抱了周芷若的身子,斜躍丈餘,喝道︰「玄冥二老,竟是這等不要臉麼?」鹿杖客哈哈一笑,説道︰「我道是誰膽敢前來橫加插手,原來是張大教主。咱們郡主娘娘在那裡?你將她拐帶到那児去啦?」

趙明從樹後閃身出來,將無忌手中周芷若抱了過去,笑吟吟的道︰「鹿先生,你整日價神魂顚倒的牽記我,也不怕我爹爹著惱麼?」鹿杖客怒道︰「你這小妖女,挑撥離間我師兄弟的感情。我兄弟兩與你父早已恩斷義絶,汝陽王著惱不著惱,干我何事?」張無忌見鹿杖客對趙明無禮,又下毒手打傷周芷若,更想起幼時中了他二人的「玄冥神掌」,不知吃盡了多少苦頭,舊恨新仇,一齊都勾上心頭,説道︰「明妹,你且退後,這鹿鶴二老,我見了便心頭有氣,今日要好好的跟他們打上一架。」二老見他空手,便即放下兵刃,凝神以待。

張無忌踏上一步,喝道︰「看招!」一招「攬雀尾」,雙掌推了出去。這一招去勢甚緩,使的是太極拳法,掌力却是暗蓄九陽神功。他今日有心要以天下無雙的純陽之力,鬥一鬥這二老純陰的「玄冥神掌」。太極拳在今日雖是十分尋常,但於元末之際,却是張三丰初創未久,武林中極爲少見。鹿杖客從未見過這等輕柔無力的掌勢,不知中間有何等詭計,他對張無忌一直甚爲忌憚,當下不敢輕易便接,斜身一躍,閃了開去。張無忌轉身「白蛇吐信」一掌拍向鶴筆翁,手掌微顫,舌吐不定。

鶴筆翁左手食指往他掌心虛點一指,右掌斜下,拍向張無忌的小腿。無忌曾與玄冥二老數度交手,知道他二人本來已非自己對手,最近與渡厄等三高僧三度劇鬥,自己的武功又深了一層,要擊敗二人可説綽綽有餘。祇是二人究竟數十年的修爲,實是非同小可,倘若倏出怪招,自己一個疏神,莫要著了他們的道児,當下展開太極拳法,圏圏連環,成了有勝無敗的局面。那太極半原是運氣不運力的特異拳術,他的九陽神功從一個或正或斜的圓圏中透將出來,玄冥二老漸感陽氣熾烈,自己玄冥神掌中發出的陰寒之氣,往往被對方逼了回來。

張無忌越鬥越是順手,心想這兩個老児原是天下少有的高手,今日將二人傷了以後,日後只怕不易再逢到這般功夫的喂招對手,是以一拳一脚之中,只是將近日來所悟到的武學精義緩緩施展,倒不急於立時將二人擊倒。鬥到百餘回合時,偶一轉身,只見地下兩個黑影微微顫動,正是月光照射在趙明與周芷若身上的影子。無忌心中一凜,側目望去,見趙明搖晃,似有抱住周芷若之勢,心道︰「不好!芷若中了鹿老児的一掌玄冥神掌,只怕抵受不住。她練的本是陰寒功夫,再加上這玄冥神掌中天下陰毒之甚的寒氣,寒上加寒,看來明妹也是禁受不住了。」當下手上勁力一加,猛向鹿杖客壓了過去。鹿杖客極是機警,已猜知他的心意,側身閃過,叫道︰「師弟,跟他遊鬥。那姓周的女子身上寒毒發作,别讓他抽手解救。」鶴筆翁道︰「正是!」展開輕功,向外一躍,拾起地下的鶴嘴雙筆,「通天徹地」,上下交征的{\upstsl{砸}}了過來。無忌微微一哂︰「有無兵刃,還不是一樣!」呼的一掌拍去,勁風壓得鶴筆翁氣也喘不過來。鹿杖客不顧師弟死活,反手抄起鹿杖,挑向無忌腰脅。

這一杖雙筆,説要擊敗無忌,其時已絶無可能,但守禦自保,仗著内力渾厚,一時無忌倒也奈何不得。他連變數種拳法,使出學自少林神僧空性的龍爪擒拿手三十六式來,撫琴式鼓瑟式抱殘式守缺式,攻勢凌厲之極。鹿杖客叫道︰「這龍爪功練得很好啊,待會児用來在地下挖坑,倒也不錯。」鶴筆翁道︰「師哥,在地下挖個坑幹什麼?」鹿杖客笑道︰「那周姑娘死定了,挖坑埋人啊!」他一説話,心神微分,無忌飛起一脚,踢在他的左腿之上。鹿杖客一個踉蹌,隨即站定,將一根鹿杖舞得風雨不透。

張無忌回頭又望趙明與周芷若一眼,只見她二人顫抖得更是厲害了,問道︰「明妹,怎樣?」趙明道︰「糟糕!冷得緊!」無忌吃了一驚,微一思索,已明其理,本來周芷若身中玄冥神掌,陰寒縱然厲害,也只是她一人身受,這時連趙明也冷了起來,想必是趙明好心,伸掌助周芷若運功抗禦。她二人功力相差甚遠,周芷若的内功又是極怪異,趙明救人不得,反而受了她的牽累。無忌雙拳大開大闔,只盼儘速擊退二老,但二老離得他遠遠地,忽前忽後,只是拖延,不跟他正面爲敵。

無忌心下焦躁,叫道︰「明妹,你將周姑娘,放在地下,不能抱著她。」趙明道︰「我\dash{}我放不下。」無忌奇道︰「怎麼?」趙明道︰「她\dash{}我\dash{}她背心\dash{}黏住了我手掌。」説話時牙関打戰,身子搖搖欲墜。無忌一驚更甚,只聽得鹿杖客説道︰「張教主,這位周姑娘良心好狠,她正在將體内寒氣傳到郡主娘娘身上,郡主娘娘快要死了,咱們立個約定,好不好?」無忌道︰「什麼約定?」鹿杖客道︰「咱們兩下罷鬥,我得周姑娘身上的兩本書,你救郡主。」無忌哼了一聲,心想︰「這玄冥二老武功已如此了得,若再練成芷若的陰毒武功,此後作起惡來,再也無人制得了他們。」

張無忌百忙中回頭一看,只見趙明本來皓如美玉般的雙頰上已罩上了一片青色,滿臉是十分痛苦的神情。無忌退後兩步,左手抓住了她的右掌,體内的九陽眞氣便即從手掌上源源傳去。鹿杖客叫道︰「上前急攻!」玄冥二老的一杖雙筆,疾風暴雨般猛襲而來,他二人知道無忌此時不能離開趙明,只憑單掌之力,要招架已是不易,更無還手之能,是以全力進攻。無忌一大半眞力用以解救趙周二女,身子既不能移動,又只剩下一掌,霎時間兇險萬分。嗤的一聲響,左腿上褲子被鶴筆翁的鶴嘴筆劃破一條長縫,腿上鮮血淋漓。趙明本來被周芷若的陰寒之氣逼得幾欲凍僵,似乎全身血液都要凝結,得無忌的眞氣一衝,身上漸漸援和。但無忌一面要和玄冥二老這兩大高手相鬥,一面要抗拒玄冥神掌和周芷若的九陰内力,左支右絀,漸漸抵擋不住,提一口氣縱聲長嘯,要招呼楊逍、范遙等來援。忽聽得山右楊逍和范遙呼嘯相應,風聲中夾雜著{\upstsl{乒}}{\upstsl{乒}}{\upstsl{乓}}{\upstsl{乓}}的兵刃相擊之聲,原來楊、范、顏三人也是遇上了強敵。

鹿杖客呼呼呼三杖,杖上鹿角直戳向眼睛。無忌舉掌運力拍出,將鹿頭杖逼開。鶴筆翁著地滾進,左手筆一招「從心所欲」,點向無忌腰間。無忌無可趨避,只得施展挪移乾坤心法,要將他一筆之力卸開,但鶴筆翁這一筆點到,力道何等沉重,是否能彀卸開,心下殊無把握。忽聽得{\upstsl{噹}}的一聲響,腰間震了一震,却不感到疼痛,原來鶴筆翁這一筆正好點在他腰間懸著的屠龍寶刀之上。無忌平素臨敵,不用兵刃,偶爾也只以聖火令當鐵尺使,但從來不使刀劍之屬,是以屠龍刀雖是掛在腰間,却一直没想到拔出禦敵。

鶴筆翁這一筆點來,登時提醒了他,當下大喝一聲,一腿踢出,將鶴筆翁逼得退開三步,回手拔刀,正好鹿杖客再度刺到,張無忌屠龍刀一揮,嗤的一聲輕響,鹿杖上的鹿頭離杖落地,鹿杖客大吃一驚,叫道︰「不好!」鶴筆翁雙筆捲到,無忌寶刀揚處,嗤嗤兩聲,一對鶴嘴筆又是斷爲四截。屠龍刀盤旋飛舞化成一團白光,玄冥二老再也不敢搶近,他體内的九陽眞氣,盡數傳到了趙明身上。這一全神發揮,周芷若所中的玄冥神掌寒毒,立時被驅趕殆盡。但陰陽二氣在人體内交感,此強彼弱,彼強即此弱,玄冥神掌的寒毒一盡,那九陽眞氣便去抵銷她所練的九陰内力。

原來周芷若居得藏在倚天劍中的「九陰眞經」後,一來生怕謝遜和無忌知覺,只是晩間偸練,二來時日迫促,無法從紮根基的功夫中循序漸進,因此所習均眞經中落於下乘的陰毒武功,何如「九陰白骨爪」之類。當年這九陰眞經的下巻落在東邪黃藥師手中,被他弟子陳玄風、梅超風盜去,所練武功,與周芷若的便大同小異
\footnote{\footnotefon{}〔按〕請參閲拙作「射鵰英雄傳」}。如此速成的功夫,内力不深,遇上了眞正高明的對手,便在内力上落敗。周芷若中了玄冥神掌後,本想將陰寒之氣轉入趙明體内,待得張無忌出手相救,只覺全身暖洋洋地,十分舒適,不料正感氣力漸長,想要離開趙明的手掌,一掙之下,竟似被一股極強的黏力吸住了,掙之不脱。適纔趙明的手掌被她背心黏住,此刻她背心反被趙明手掌黏住,那均是内力強弱有别之故。

無忌力驅寒毒,但覺自己的九陽眞氣送將出去,趙明手上總是傳來一股寒氣與之相抗,他只道玄冥神掌的寒毒尚未驅盡,不住的加力施爲,那想到他每送一分九陽眞氣過去,便消去了周芷若苦苦練得的一分九陰眞氣。周芷若心中暗暗叫苦,却開不了口,自知只要一張口説話,立時狂噴鮮血,眞氣洩盡而亡。

趙明體内融和舒暢,笑道︰「無忌哥哥,我好啦,你專心去對付玄冥二老吧!」無忌道︰「好」内力一收。周芷若如遇大赦,脱了黏力,自知這麼一來,所中玄冥神掌的陰毒雖然已驅盡,但自身的九陰内力却也損耗極重,眼見無忌舞動屠龍刀,專心攻敵,當即伸出五根手指,一揮手,往趙明頂門上插了下來。趙明大叫一聲︰「啊喲!」天靈蓋上一陣劇痛,喀喇一聲響,周芷若五根手指齊斷,痛哼一聲,急奔而去。

張無忌吃了一驚,忙回過頭問道︰「怎麼啦?」趙明伸手一摸自己腦門,只嚇得魂飛天外,説不出話來。無忌只道她已爲「九陰白骨抓」所傷,右手舞刀擋住二老,一手去摸她頭頂,只覺著手處濕膩膩地,雖已出血,幸未破骨穿洞,心中一塊大石纔落地,安慰道︰「皮肉之傷,不礙事!」原來周芷若出手襲擊之時,一來無忌輸至趙明體内的九陽眞氣尚未退盡,二來周芷若的内力大損,以弱攻強,反而折斷了自己的五指。

無忌這一分心,玄冥二老又攻了過來。這時無忌手中有了天下第一鋒利的利刃,自覺仗此利器,勝人不武,高聲叫道︰「楊左使,范右使,你們那邉情形如何?」范遙叫道︰「已打倒三人,尚有七人,不勞教主掛心。」兩人遙遙應答,均是内力充沛,神完氣足,料知對方無恙。無忌心下一寬,反手將寶刀交於趙明,左手一牽一引,使出乾坤大挪移心法,將鶴筆翁拍來的一掌轉移了方向。這一牽一引之中,他貫注了九陽神功,使的乃是乾坤大挪移第七層的最高深的功夫。這第七層功夫最耗心血内力,絲毫疏忽不得,所謂錯以毫厘,失之千里,一個運用不善,便會自己走火入魔,因此適纔無忌分心助趙明周芷若驅除寒毒之時,雖然情勢危急,却不敢貿然動用。玄冥二老是武林中頂尖高手,如以輕淺的挪移乾坤神功對付,却又奈何二人不得。

他立定脚跟,體内氣血一順一逆,内息極迅速的流轉一周,凝神一撥之下,鶴筆翁一劍拍出,波的一響,正中鹿杖客肩頭。鹿杖客吃了一驚,怒道︰「師弟,你幹什麼?」鶴筆翁武功極精,性子比較遲鈍,一件事須得思索良久,方明其理,這一下事出倉卒,自己也莫明其妙,愕然難答,但知定是無忌搗鬼,心想只有加緊攻擊敵人,方能向師兄致歉,於是運勁右腿,一脚向無忌踢去,張無忌左手拂出,一黏之下,鶴筆翁這脚又踢向鹿杖客小腹丹田。這丹田是人身運氣的中樞要地,鹿杖客如何能讓他踢中?驚怒之下,側身避過,喝道︰「你瘋了麼?」

趙明叫道︰「不錯,鶴先生,快將你這大逆無道,好色貪淫的師兄擒住,我爹爹重重有賞。」無忌心下暗笑︰「這挑撥離間之計果然甚妙。」他本想以挪移乾坤之法,引得鶴筆翁去打鹿杖客,再引鹿杖客去打鶴筆翁,這時聽了趙明之言,當下單只牽引撥動鶴筆翁的拳脚,對付鹿杖客時却仍是太極拳的招數,口中叫道︰「鶴先生,不用擔心,你我二人合力,定能宰了這頭淫鹿。汝陽王已封你爲\dash{}封你爲\dash{}」他一時没想到合適的官職,趙明叫道︰「鶴先生,你封官的官誥,便在這児。」説著從懷中取出一束紙片一揚,讀道︰「{\upstsl{嗯}},是大元護國揚威大將軍,快加把勁啊。」

無忌一掌拍出,將鹿杖客逼向左側,正好鶴筆翁的一掌被他引得自左而右的擊到,成爲左右夾攻之局。鹿杖客和鶴筆翁數十年親厚勝於同胞,原不信他會出賣自己,但此刻眼見鶴筆翁接連五招,都是攻向自己要害,拳脚之中又是積蘊全力,直欲制自己死命,那裡還有半分情誼?他憤慨異常,喝道︰「你貪圖富貴,全不顧念義氣麼?」

鶴筆翁急道︰「我\dash{}我是\dash{}」趙明接口道︰「不錯,你這是迫不得已,爲了要做護國揚威大將軍,得罪師兄,那也是無話可説了。」張無忌右手加了十成力,凝神一帶,鶴筆翁一掌拍將過去,砰的一聲響,重重擊在鹿杖客肩頭。鹿杖客大怒,反手一掌,將鶴筆翁左邉牙齒打落數枚。鶴筆翁年紀已老,口中就只剩下左邉這幾枚牙齒,向來十分珍惜,這一來不禁也激發了怒氣,喝道︰「師哥,你也太不分好歹,又不是我故意打你。」鹿杖客怒道︰「是誰先動手了?」他武學雖高,却決不相信世間竟有乾坤大挪移第七層神功的可能,武功中雖有借力打力、四兩撥千斤的法門,但憑著玄冥二老的修爲,任何人不能撥動他們的掌力,是以絲毫没疑心到這是張無忌從中作怪。

鶴筆翁表明心跡,罵道︰「賊小子,你搗鬼!」趙明叫道︰「是啊,不用再叫他師哥,叫他『賊小子』便了。」無忌左掌壓住了鹿杖客掌力,右手一引,鶴筆翁一掌擊上了鹿杖客右頰,登時高高腫起。趙明叫道︰「無忌哥哥,咱們援助楊左使他們去。」無忌見鹿杖客紅了雙眼,掌力源源催動,知道離間之計已成,喝道︰「鶴先生,這淫賊交與你了。」左足一點,縱身躍開,擕了趙明的手便走。只見玄冥二老你一拳,我一脚,鬥得激烈異常。趙明叫道︰「鶴先生,你擒住師哥後,屠龍刀中的武功祕笈可以借你觀看一月。快立大功,良機莫失。」鹿杖客更是怒氣勃發,下手毫不容情。他二人藝出同門,武功半斤八兩,這一場惡戰,也不知鬥到何時方休。最後終於兩敗倶傷,鶴筆翁雖是説明原委,但鹿杖客誓死不信,從此師兄弟二人反目成仇,此是後話不提。

無忌循著兵刃相交之聲,趕到楊逍身側,只見地下躺了五具死屍,楊逍以一敵三,范遙與顏垣各與一名敵人相鬥。五名敵人中只范遙的對手最強,以范遙的武功,居然也攻他不下,只是招數上略佔上風。無忌也不上前相助,站著旁觀,片刻間楊逍又打倒了一人。楊逍剩下的二名敵手眼見必敗,齊聲呼嘯,向山下逃去,顏垣的對手跟著便走。顏垣揚手擲出毒沙,那人慘呼倒地而斃。與楊逍對敵的那人却是寧死不屈,仍是強悍異常的酣鬥。范遙砰的一拳,打得他身子一晃。楊逍道︰「這位兄台,我瞧你倒是一條好漢子,不如投降了吧!」那人怒道︰「投降之徒,還能是好漢子麼?」張無忌道︰「不錯!」舉起屠龍刀,左上右落,颼颼颼連砍八刀,刀鋒從他頭臉掠過,登時毛髮紛飛,將范遙一掠,笑道︰「范兄,饒他去吧!」那人只覺頭上臉間涼冰冰地,伸手一摸,張大了口合不攏來,原來頂門頭髮,頰上鬍子,已被無忌這八刀削去了大半,不由得死心蹋地,向無忌一揖,説道︰「拜服閣下神技,任憑處置。」無忌一笑,道︰「兄台請便。」那人長嘆一聲,掉頭而去。

張無忌道︰「這些都是汝陽王府中的武士麼?此人是誰?」趙明道︰「他是我哥哥的衛士長,叫做虯髯神拳魯仙客,現下要算是我爹爹府中的第一高手了。」楊逍笑道︰「虯髯變成了光下巴,他在王府中也耽不住了。」説話之間,少林寺僧衆與明教中人紛紛聞聲來援,一齊到玄冥二老打鬥處去看時,只見二老已奔出數里之外,從山上望將下去,月光下只見到細小的黑影,喝罵聲隱隱傳來,顯是仍在惡鬥不休。

衆人回到少林寺中,無忌察看趙明頭頂傷痕無礙,忽然想起一事,道︰「明妹,你身上湊巧帶著紙張,這一來不由得鹿杖客不信。」趙明笑吟吟的從懷中取出一束薄薄的紙片,在無忌面前一揚,笑道︰「你猜猜那是什麼?」

\chapter{是耶非耶}

張無忌笑道︰「你叫我猜的東西,我一輩子也猜不出。」趙明將兩束紙片放在他的手裡,無忌就燭光一看,只見這些紙片其實非紙,乃是薄如蟬翼的絹片,上面密密麻麻的冩滿了細如蠅頭的工整小楷。第一束上開頭冩著「武穆遺書」四字,内文均是行軍打仗,布陣用兵的精義要訣。無忌再看第二束時,見開頭冩的四字是︰「九陰眞經」,内文書滿各種神奇怪異的武功,翻到最後,「九陰白骨爪」和「摧心掌」等赫然在内。無忌一顆心怦然跳動,説道︰「你\dash{}你是從周姑娘身上取來的?」趙明道︰「當她不能動彈之時,我焉有不順手牽羊之理?這些陰毒功夫我便不想學,可是取來毀了,勝於留在她手中害人。」

無忌隨手翻閲九陰眞經,讀了幾頁,只覺文義深奥,一時不能盡解,然決非陰毒下流的武學,説道︰「這經上所載武功,其實極是精深,依法修練,一二十年之後,相信成就非同小可,若是只求速成,學得一些皮毛,那就害人害己了。」他頓了一頓,又道︰「那位身穿黃衫的姊姊,武功與周姑娘明明是一條路子,然而招數正大光明,醇正之極,似乎是從這九陰眞經中而來。」趙明道︰「她説什麼『終南山後,活死人墓。神鵰俠侶,絶跡江湖』,這四句話是什麼意思?」無忌搖頭道︰「日後咱們見到太師父,請教他老人家,或許能得其中緣由。」兩人閒談幾句,見山下軍情並無變化,當即分别安寢。

次晨無忌一早起身,躍上高樹瞭望,只見山下敵軍又多了萬餘人,顯是元軍夜中結集重兵,要再大舉進攻。群雄見到敵軍旌旗招展,兵甲鮮明的情勢,均是憂心忡忡。只聽得敵軍營中號角此起彼落,想是調兵遣將,十分忙碌。無忌道︰「明妹!」趙明應道︰「{\upstsl{嗯}},怎麼?」無忌微一遲疑,道︰「没有什麼,我隨口叫你一聲。」他本想與趙明商議打退元兵之法,以她之足智多謀,定有妙策,但轉念一想︰「她是朝廷郡主,背叛父兄而跟隨於我,再要她定計殘殺自己蒙古族人,未免強人所難。」是以話到口邉,又忍住了不説。趙明鑒貌辨色,已知其意,嘆了口氣,説道︰「無忌哥哥,你能體諒我的苦衷,我也不用多説了。」

無忌回到室中,一時徬徨無策,隨手取出趙明昨晩取來的兩束紙片,看了幾頁九陰眞經,翻到武穆遺書時,無意中看到「兵困牛頭山」五個小字,心中一動,仔細看下去,却是岳飛敘述當年如何爲金兵包圍、如何從間道脱困、如何突出奇兵、如何内外夾攻而大獲全勝,種種方略,冩得十分詳明。無忌拍案大叫︰「天助我也!」掩住兵書,靜靜思索,這少室山上的情勢,雖與岳武穆當年被困時的情景大不相同,然用其遺意,未始不能出奇制勝。他越想越是興奮,暗想岳武穆果是天縱奇才,如此險著,常人那裡想得到,又想用兵之道便如武功一般,若是未得高人指點,高下巧拙,相去實在不可以道里計。

他手指醮了茶水,在桌上繪畫圖形,雖覺行險,却未始不能僥倖求勝,須知以寡敵衆,終不能以堂堂正正之陣取勝。當下心意已決,來到大雄寶殿,請空聞方丈召集群雄。

片刻間各路英雄齊到殿中,無忌居中一站,説道︰「此刻韃子兵馬聚集到山下,料想不久便會大舉攻山。咱們雖然昨日小勝,挫了韃子的鋭氣,但韃子若是不顧性命的蜂擁而上,究屬難以抵擋。在下不才,蒙衆位英雄推舉,暫當盟主之位。今日同仇敵愾,請各位暫聽在下號令。」群雄齊道︰「盟主但有所命,自當凜遵,不敢有違。」無忌道︰「好!吳勁草聽令!」

鋭金旗掌旗使吳勁草踏上一步,躬身道︰「屬下聽令!」心下暗想︰「教主號令,第一個便差遣到我,實是我莫大榮幸。不論命我所作之事如何艱危,務須捨命以赴。」却聽無忌説道︰「命你率領本旗兄弟,執掌軍法,那一位英雄好漢不遵號令,鋭金旗長矛短斧齊往他身上招呼。縱然是本教耆宿、武林長輩,倶無例外。」吳勁草大聲道︰「得令!」弗的一聲,抽出了懷中一面小小白旗,捧在手中。吳勁草本人的武功聲望,在江湖上未臻一流之境,旁人對他原不如何重視。但自那日少林寺廣場上明教五行旗大顯神威,群雄知道他手中這面白旗所到之處,跟著而來的便是五百枝羽箭,五百枝長矛、五百柄短斧、任你本領通天,霎時之間也是爛成一團肉漿,是以見他白旗展動,心中都是一凜。原來無忌翻閲「武穆遺書」,見第一章便説︰「治軍之道,嚴令爲先」。他知道這些江湖豪士向來人人自負,各行其是,各别的武功雖強,聚在一起却是烏合之衆,若非申令部勒,不能與蒙古精兵相抗,因此上第一件事便是要鋭金旗監令執法。

無忌指著殿前的一堵照壁,説道︰「衆位英雄,凡是輕功高強,能一躍而上此堵照壁的,請一獻身手。」群雄之中,登時有不少人臉現不滿之色,心道︰「這是什麼當口,却叫咱們來幹這種無関緊要的縱高竄低?」有些前輩高手,更覺無忌出言小覷衆人,大是不愉。張松溪突然排衆而出,説道︰「我能躍上。」一躍上了照壁,輕輕從另一面翻下,武當派梯雲輕功名聞天下,以張松溪的能耐,要躍過這堵照壁可説是不費吹灰之力,但他既不賣弄,更無不悦,只是老老實實的遵從躍過。

張松溪這一首先遵令,兪蓮舟、殷利亨、楊逍、范遙、韋一笑、殷野王等高手一一遵行,祇見群雄如穿花蝴蝶,接二連三的躍過牆去,有的炫耀輕功,更在半空中演出種種花式,躍到四百餘人,餘下便無人再試。要知這堵照壁著實不低,若非輕功了得,原是不易一躍而上。參與此次英雄大會的群雄武功各有不同造詣,但輕功一道,却非盡人所長,往往擅於拳脚兵刃的,輕功便甚平常,蓋武功學之道,極難遍通,有些人單是研習一指一掌,便他了畢生的功夫。江湖上的成名人物,無不有自知之明,如不是自己所長,決不勉強而致出醜。

無忌見這四百餘人之中,少林派的僧衆倒佔了八九十人,心想︰「少林是武林中第一大派,果是名不虛傳。單以輕功一項而論,好手便遠遠較别派爲多。」於是傳令道︰「兪二伯、張四伯、殷六叔,請你們三位帶同擅長輕功的衆英雄,虛張聲勢,假裝寺中人衆全體逃走,引得敵軍來追,乃是第一大功,一到後山,即便如此如此。」武當派兪張殷三俠齊聲接令。無忌又道︰「舅舅,你與楊左使、范右使、韋蝠王四位,請助我居中管應。」當下一一分派,何者埋伏,何者斷後,何者攻堅,何者側擊,倶各詳細安排。楊逍等見他佈陣迎敵,竟是如此井井有條,若有預謀,無不驚訝,却不知他乃是襲用岳武穆遺法,只是因地形不同、部屬不同,而略加更改而已。

無忌分派已畢,最後説道︰「空聞方丈、空智神僧兩位,請率同峨嵋派諸位,救護死傷。」周芷若既不在山上,峨嵋派無人爲首,無忌自覺與峨嵋嫌隙甚深,不便指揮,因此推舉空聞、空智這兩位德高望重的神僧率領,料想峨嵋群弟子不致抗命。他號令一下,峨嵋派的男女弟子果然默然接令,並無異言。不料空聞、空智却對望一眼,相互點了點頭,空聞躬身説道︰「盟主機謀深遠,指揮若定,老僧極是欽佩。但老僧師兄弟有一不情之請,要請盟主俯允。」無忌道︰「方丈不必客氣,請示尊意。」

空聞説道︰「非是老僧不遵盟主號令,只是向盟主討令,由老僧師兄弟二人,留守本寺。」張無忌一聽,已知其意,蓋他現在的策劃,乃是棄去少林,假裝向後山遁逃,引得敵人追逐,然後設法解圍。但當年岳飛困牛頭山,那牛頭山乃是光禿禿的一座山頭,説棄便棄,毫不足惜。這少室山上却有歷時千年的少林古刹,佛門聖地,群雄棄守之後,萬一敵軍派兵上來査察,只見到一座空寺,那元軍何等殘暴,勢必舉火焚燒。空聞空智二人討令守寺,那是決心與寺院共存共亡之意了。無忌微一沉吟,説道︰「很好,二位大師壯志可佩,便請二位留守。」群雄臉上均有詫異之色,本想無忌必會勸阻,那知竟是一口答應,少林弟子有人欲待陳辭,空聞厲聲道︰「軍令森嚴,本派弟子中若有違犯,立即除名,逐出本派。」當下誰也不敢再説。

無忌朗聲説道︰「今日中原志士,齊心合力,共與韃子周旋。少林派執掌鐘鼓的諸位師父便請擂鼓鳴鐘。」群雄轟然歡呼,抽刀拔劍,無不意氣昂揚。

烈火旗掌旗使夏炎一聲號令,旗下教衆將寺中積儲的柴草都搬了出來,堆在寺前左,發火燃燒,片刻間煙燄衝天而起。山下元軍先聽得鐘鼓響動,已自戒備,但見山上火起,都道︰「不好,蠻子放火燒寺,定要逃走。」那烈火旗縱火的法門最是巧妙,在佛殿頂都澆油放火,却不延燒殿身,從山下遠遠望將上來,只見數百間寺屋到處都有態態大火冒上。

兪蓮舟左手一揮,率領一百五十餘名輕功卓越的好漢,從少室山的左側奔了下去。奔不到山腰,元軍已是大聲鼓噪,列隊追來。群雄故意四散亂走,好教元軍最擅長的弓箭之技無法聚集射發。第二批由張松溪率領,第三批由殷利亨率領,其巾有僧有俗,各背負一個厚厚包袱,包中藏的不是木板,便是衣被。在元軍看來,果是棄寺逃命的狼狽景象,但一箭箭射去,中在包袱之上,却都傷不到本人。元軍見衆人燒寺逃走,煙霧之中也看不清人數多寡,當下分兵一萬追趕,其餘一萬兵留在原地防變。

無忌向楊逍道︰「楊左使,韃子的首領倒是頗爲用兵,並不全軍追逐。」楊逍道︰「是,此事確實可愛。」只聽得山下號角聲響起,元軍的兩個千人隊分從左右攻上山來。無忌等望見元軍縱馬從山坡上奔來,山石雖是崎嶇,那些蒙古小馬却是馳騁如飛,長矛鐵甲,軍容甚盛。待那先鋒部隊攻到半山亭邉,無忌左手一揮,烈火旗人衆從兩側搶開,伏在草中,等得敵軍二千人馬又前進百餘丈,夏炎一聲忽哨,噴筒中石油射出,烈火忽發,都是往馬匹身上燒去。群馬悲嘶驚叫,一大半滾下山去,登時大亂。

但元軍軍紀嚴明,前隊雖敗,後隊毫不爲動,一聲令下,三個千人隊棄去馬匹,步攻而前。烈火旗再噴火燄時,雖燒死燒傷了數百人,餘人却是奮勇而上。洪水旗掌旗使唐洋一聲忽哨,毒水噴出,跟著厚土旗擲出毒砂,又將二千餘元打得七零八落。雖有數百人攻到山峰,均被鋭金、巨木二旗人衆一一殲滅。

猛聽得山下鼓聲擂得甚急,五個千人隊人衆突然豎起巨大盾牌,列成橫隊,如一道鐵牆般緩緩推前。這麼一來,烈火、毒水、毒砂等均是無所施其技,即令巨木旗用巨木上前撞擊,看來也只能撞開幾個缺口,無濟於事。空聞方丈一見事急,説道︰「張教主,請各位迅速退去,保存我中原武林的元氣。今日雖敗,日後更可捲土重來。」

無忌遙望敵軍中軍,只見一桿大纛高高舉起,旗下一位將軍跨了一匹青驄馬,手持長槍,鐵甲上金光閃閃,飾以黃金,形相甚是威武,只是頭盔戴得甚低,瞧不見他的容貌。無忌轉身向吳勁草道︰「吳旗使,你襲擊這將軍左右的衛護。」吳勁草應道︰「是!」白旗一揮,奮勇當先,向那將軍衝了過去。白旗指處,屬下教衆的一百根長矛紛紛向那將軍左右擲出。無忌道︰「韋蝠王,咱們去擒了那將軍來。楊范二位,你們給咱們掩護。」三人笑道︰「此計大妙。」楊逍范遙便向山下衝去。

韋一笑和張無忌縱身躍起,幾個起落,已搶在楊范二人之前,這輕功的身法一加展開,當眞是便如兩溜輕煙一般。二人奔到盾牌手前,撥落紛紛射來的羽箭,各出右足在盾牌上一點,已然翻身躍過盾牌隊組成的鐵牆。元軍官兵大聲{\upstsl{吆}}喝,槍矛如林般刺到。韋張二人並不抵敵,只是東閃、西避,從人叢中直穿了過去,搶到那將軍馬前。那將軍一槍刺來,無忌反手勾拿,抓住了槍桿,順勢一拉,拉得那位將軍向前一撞,韋一笑飛身而上,已抓住他的後頸。

那將軍倒也了得,左手抽出寶劍,攔腰揮去。無忌猿臂伸處,抓住了他的手腕,和韋一笑二人橫拖直曳的拉下馬來。四周護衛失色驚呼,捨命來救,却被楊逍和范遙擋住了,不得近前。無忌手一鬆,笑道︰「走吧!」韋一笑反身點了那將軍穴道,扛在肩頭,反身却向山下無人處奔去。元軍見主帥被擒,喊聲震動山谷,群向韋一笑追去。想那韋一笑輕功何等了得,當年肩頭扛了峨嵋派的弟子,以滅絶師太如此身手,尚且追他不上。那將軍的護衛中雖有若干武林好手,被他邁開步子衝出數丈,早已望塵莫及。韋一笑越奔越遠,忽而躍上樹巓,忽而縱上岩頂,元軍衆官兵只瞧得心膽倶裂,眼見他從半空中摔將下來,不知如何的一竄,又從斜刺裡橫了過去。張無忌、楊逍、范遙見他得手,縱聲長笑,先後躍回。

韋一笑賣弄本領,將那將軍遠遠向前擲出,元軍大聲呼喝,只道主帥要在岩石上撞得頭崩額裂,筋折骨斷。那知韋一笑身法好快,那將軍落下時離地五尺,他已趕到接住,原來他抛擲之時,手上的勁力算得不爽厘毫,空中飛人剛好跌落,他也正及時趕到。如此擲得數下,已到山峰,他大聲喝道︰「楊左使,大買賣來了!」用勁一擲,將那將軍的身子向楊逍急飛而至。楊逍輕輕接住,扯開他的頭盔一看,只見這將軍面目英俊,只是雙眉豎起,顯是心中憤怒無比。趙明叫道︰「哥哥!」撲了過來。原來這少年將軍正是趙明的兄長王保保。

這一下却也大出無忌意料之外,他眉頭微皺,説道︰「得罪!」抱起王保保的身子,放在空聞與空智之間,低聲道︰「兩位大師以他爲質當可保存少林寺院。此人與在下頗有淵源,還請勿傷他性命。」空聞、空智大喜,各從弟子手中掠過一柄戒刀,架在王保保的頸中。其實二人若要取他性命,原只一掌之勞,但兩柄明晃晃的戒刀架在他的頸,在元軍官兵看來,更增幾分兇險。

楊逍朗聲叫道︰「蒙古官兵聽著,你們小王爺已落入咱們手中,急速退至山下,免得害了他的性命。」指揮這元軍萬人隊的萬夫長又驚又急,心想若是當眞傷了小王爺的性命,汝陽王執掌兵馬大權,赫然震怒,説不定要全軍都要殺頭,只盼傳令退兵。山頂群雄正歡呼鼓噪間,忽聽得山下金鼓大振,一枚火箭衝天而起,殺聲四起。楊逍大喜,説道︰「教主,咱們的援兵來啦。」從山頂望將下去,瞧不見山下情景,但煙塵騰空,人喧馬嘶,中軍顯是來得甚衆。

張無忌高聲叫道︰「援軍已到,大夥児衝啊!」山上群雄各挺兵刃,衝殺下去。無忌又叫︰「各位英雄,先殺官,後殺兵。」群雄紛紛吶喊︰「先殺官,後殺兵!」這六個字頗奏奇效,要知蒙古兵紀律嚴明,每十名士兵爲一個十人隊,由什長率領,其上爲百人隊,千人隊,萬人隊,層層統屬,臨陣時如心使臂,如臂使手,如手使指。但明教援軍在山下一攻,群雄再自山上衝殺,元軍陣勢先自亂了。張無忌又傳令揀專揀敵軍官長殺戳,一支蒙古精兵登時亂成一團。

無忌等衝到山腰,只見山下旌旗招展,南首旗上一個「徐」字,北首旗上一個「常」字,知道是徐達與常遇春到了。徐常二人本在淮泗,此時恰在豫南,得到布袋和尚説不得傳訊,獲悉教主被圍少室山,盡起部屬,星夜來援。其時豫南鄂北一帶,明教義軍與元軍混戰經年,雙方所佔地域,犬牙交差,説來便來,甚是近便,是以不到兩日,便已趕到。徐達與常遇春所率教衆,都是久經戰陣之士,武功雖與山上群雄相差甚遠,但列隊群鬥,威力却是極強,兼之人數衆多,逼著元軍一路向西奔逃。

無忌本就定下計謀,引得一個元軍萬人隊追向西方山谷。那山谷三邉均是懸崖。兪蓮舟張松溪殷利亨率同數百名輕功卓越的好漢,邉鬥邉退,逃入谷中。元軍的萬夫長雖覺地勢兇險,但眼敵人爲數不多,谷中縱有埋伏,那也儘能對付得了,是以揮軍追入谷中。兪蓮舟等奔到懸崖之下,崖上早有數十條長索垂下,各人攀援而上。這些人均是一等一的輕功好手,高來高去的本領最是拿手不過,元軍頂盔貫甲,身手笨拙,如何攀援得上?那萬夫長眼見中計,急令退軍,不料谷口烈火、毒砂、羽箭、毒水不住射來,巨木旗將一段段巨木堆起,封住了谷口。

便在此時,元軍第二路敗兵又到,谷口已然封住,當即漫山遍野的四散奔逃,張無忌和徐達先後趕到,均叫︰「可惜,可惜!」若是事先聯絡妥善,將元軍第二支萬人隊一齊驅入谷中,便可一鼓而殲。須知無忌没料到援軍來得如此神速,並無全殲元軍的雄心,只求殺敗敵軍,保存少林,便已心滿意足了。當下徐達號令部衆搬土運石,再在谷口加封。一隊隊弓箭手攀到崖頂,居高臨下的向谷中射箭,元軍身處絶地,無力還手,唯有找尋山石隱身躱藏。不久常遇春率隊趕到,與無忌會見,久别重逢,均是不勝之喜。常遇春性烈如火,大叫︰「搬開土石,待我衝進去將韃子殺個乾乾淨淨。」徐達笑道︰「谷中無水無米,不出三四日,韃子渴的渴死,餓的餓死,何勞你我兄弟動手?」常遇春笑道︰「總是親手殺的乾脆。」他年紀雖較徐達爲長,但平時素服徐達智謀,又見無忌附和徐達之言,當下也不再説,自去指揮部屬,搜殺潰散的元軍。無忌應趙明之請,放了王保保,派吳勁草率領本旗兄弟,送出五十里外,由其自去。趙明親自送了十里,連聲致歉,王保保一眼也没瞧她,自始至終,不發一言。趙明只得怏怏而回。

這一晩少室山下歡聲雷動,明教義軍和各路英雄慶功祝捷。群雄連日在少林寺中吃的都是素齋,口中已淡得難過,這時大酒大肉,開懷飽啖。

徐達滿斟了一杯酒,奉給無忌,説道︰「恭賀教主,請盡此杯!」無忌接過飲了。徐達説道︰「屬下平日欽佩教主肝膽照人,武功絶倫,不料用兵竟亦如此神妙,實是本教之福,蒼生之幸。」無忌哈哈大笑,説道︰「徐大哥,你不用恭維我了。今日大勝,一來是你徐常二位大哥來得神速,二來是靠了岳武穆的遺愛。小弟實無半分功勞。」

徐達奇道︰「怎地是岳武穆的遺愛?還盼教主明示。」無忌從懷中取出一束薄薄的黃紙,正是原來藏於屠龍刀中的「武穆遺書」,翻到「兵困牛頭山」那一節,遞了過去。徐達雙手接過,細細讀了一遍,不禁又驚又佩,嘆道︰「武穆神機,實非後人所及。若已武穆今日尚在世間,率領中原豪傑,何愁不將韃子逐回漠北。」説著恭恭敬敬的將遺書交回無忌。無忌却不接過,説道︰「『武林至尊,寶刀屠龍,號令天下,莫敢不從』,這十六個字的眞義,我今日方知。所謂『武林至尊』,不在寶刀,而在刀中所藏的遺書。以此兵法臨敵,定能戰必勝,攻必克,最終自是『號令天下,莫敢不從』了。徐大哥,這部兵書轉贈於你,望你克承武穆遺志,還我河山,直搗黃龍。」

徐達大吃一驚,忙道︰「屬下何德何能,焉敢受教主如此重賜?」張無忌道︰「徐大哥不必推辭。我爲天下蒼生授此兵書於你。」徐達捧著兵書,雙手顫抖。張無忌道︰「武林傳言之中。劍中所藏,乃是一部極是厲害的武功祕笈。我體會其意,兵書是驅趕韃子之用,但若有人一旦手掌大權,竟然作威作福起來,以暴易暴,世間百姓受其苦,那却也未必便能當倚天劍之一擊。徐大哥,這番話請你記下了。」徐達汗流浹背,不敢再辭,説道︰「屬下謹遵教主令旨。」將「武穆遺書」供在桌上,對著恭恭敬敬的磕了四個頭,又拜謝張無忌贈書之德。此後徐達果然用兵如神,連敗元軍,最後統兵北伐,直將蒙古人趕至塞外,威震漠北,建立一代功業。

自此中原英雄傾心歸附明教,張無忌號令到處,無不凜遵。那明教數百年來一直爲人所不齒,被目爲妖魔淫邪,不料經此一番天翻地覆的大變,竟成爲中原群雄之首,克成大漢子孫中興的大業。其後朱元璋雖起異心,迭施計謀而登帝位,但助他打下江山的都是明教中人,是以國號却不得不稱一個「明」字。明朝自洪武元年戊甲至崇禎十七年甲申,二百七十七年的天下,均從明教而來。此是後語不提。

這一晩群雄歡飲達旦。盡酔方休。到得午後,群雄紛紛向空聞、空智兩位神僧告辭。張無忌見峨嵋派弟子七零八落,心下頗是惻然,又見宋青書躺在擔架之中,不知生死如何,便走近前去,向靜慧説道︰「我瞧瞧宋大哥的傷勢。」靜慧冷冷的道︰「貓哭耗子,也不用假慈悲了。」周顚隨侍在側,忍不住罵道︰「我教主顧念你掌門人的舊日情分,纔設法給他治傷。其實這等欺師叛父之徒,人人均得而殺之。你這惡尼姑囉唆什麼?」靜慧待要反唇相稽,但見周顚狠狠霸霸的,只怕他蠻不講理,當眞動起手來,却要吃個眼前的虧,於是強忍怒氣,冷笑道︰「我峨嵋派掌門人世代相傳,都是冰清玉潔的女子。周掌門若非守身如玉的黃花閨女,焉能做本派掌門?哼,宋青書這種奸人留在本派,莫要汚了周掌門的名頭。李師侄、龍師侄,將這傢伙送回武當派去吧!」抬著宋青書的兩名峨嵋弟子齊聲答應,將擔架抬到兪蓮舟身前,放下便走。

衆人都是吃了一驚,兪蓮舟道︰「什\dash{}什麼?他不是你掌門人的丈夫麼?」靜慧恨恨的道︰「哼,我掌門人怎能將這種人瞧在眼中?她氣不過張無忌這小子變心逃婚,在天下英雄之前羞辱本派,纔騙得這小子來冒充什麼丈夫。那知\dash{}哼哼,早知如此,我掌門人又何必負此醜名?眼下她\dash{}她\dash{}」

張無忌在一旁聽得呆了,忍不住上前問道︰「你説宋夫人\dash{}她\dash{}她其實不是宋夫人?」靜慧轉過了頭,恨恨的道︰「我不跟你説話。」便在此時,躺在擔架中的宋青書身子動了一動,呻吟道︰「殺了\dash{}殺了張無忌麼?」靜慧冷笑道︰「别做夢啦!死到臨頭,還想得挺美。」殷利亨見靜慧氣鼓鼓的,説話始終不得明白,低聲向峨嵋派另一名女弟子問道︰「李師妹,到底是怎麼回事?」

這李明霞是個中年女子,當年與紀曉芙甚是交好,聽殷利亨問起,沉吟半晌,道︰「靜慧師姊,殷六俠也不是外人,小妹跟他説了,好不好?」靜慧道︰「什麼外人不外人的?不是外人要説,是外人更加要説。咱們周掌門清清白白,跟這姓宋的奸徒没半絲瓜葛。你們親眼得見掌門人臂上的守宮砂。此事須得普天下武林同道衆所週知,免得壞了我峨嵋派百年來的規矩\dash{}」殷利亨心想︰「這位靜慧師太腦筋不大清楚,説話有點児顚三倒四。」於是向李明霞道︰「李師妹,既是如此,便盼詳示。我這宋師侄如何投身貴派,與貴派掌門人到底有何干係,小兄日後得須向家師稟告。此事関涉貴我兩派,總要不傷了兩派和氣纔好。」

李明霞嘆了口氣,道︰「這位宋少俠人品武功,均是武林少見的人物,只是一念情痴,墮入業障。我掌門人似乎答應過他,待得殺了張無忌,洗雪棄婚之辱,便即下嫁於他。因此他甘心投入本派,向我掌門人討教奇妙的武功。前日英雄大會之上,掌門人突然聲稱自己是『宋夫人』,説是這宋少俠的妻子,當時本派男女弟子,人人十分驚異。當日掌門人威震群雄,懾服各派\dash{}」周顚突然接口道︰「是張教主故意相讓的,有什麼大氣好吹!」李明霞不去理他,續道︰「木派衆弟子雖是十分高興,但到得晩間,衆人問她宋夫人這三字的由來。掌門人露出左臂,森然道︰『大夥児都來瞧瞧!』咱們人人親眼看到,她臂上一粒守宮砂殷紅如昔,果然是個知法守禮的處子。掌門道︰『我自稱宋夫人,乃是一時權宜之計。只是氣氣張無忌那小子,叫他心神不定,比武時便能乘機勝他。這小子武功絶倫我實是及不上他。爲了本派的聲名,我自己的聲名何足道哉。』」

她這番話朗然説來,有意要讓旁邉許多人都聽得明白,又道︰「本派男女弟子,若非出家修道,原是不禁嫁娶,只是自創派祖師郭祖師以來,凡是最高深的功夫,只傳授守身如玉的處女。每個女子拜師之時,師父均在咱們臂上點下守宮砂。每年逢到郭祖師誕辰,先師均要檢視,當年紀師姊\dash{}就是這樣\dash{}」她説到這裡,含糊其辭,不再説了。要知她雖已兩鬢蕭蕭,嫁了丈夫,生児育女,但説到男女間的風化之事,總是不便出口。殷利亨等却已了然,知道李明霞乃是想説當年紀曉芙爲楊逍所誘失身,守宮砂消失,這纔給滅絶師太發覺。殷利亨與楊不悔婚後夫妻情愛甚篤,可是此時想起紀曉芙來,心下不禁憮然,忍不住向楊逍瞥了一眼,只見他熱泪盈眶,轉過了頭去。

李明霞道︰「殷六俠,我掌門人存心要氣一氣明教張教主,偏巧這位宋少俠又對我掌門人痴纏不休,以致中間生出許多事來。只盼宋少俠身子復原,殷六俠再向張眞人和宋大俠美言幾句,以免貴我兩派之間生下嫌隙。」殷利亨點頭道︰「自當如此。我這師侄忤逆無道,死不足惜,實是本派門戸之羞,我倒盼他早些死了乾淨。」殷利亨心腸本軟,但想到宋青書害死莫聲谷的罪行,實是痛恨無比。

正説話間,忽聽得遠遠傳來一聲尖鋭的驚呼,似乎是周芷若之聲,聲音中充滿了極度的恐懼,當是遇上了什麼兇險之極的變故。

\chapter{黑衣少女}

衆人聽得這一下驚呼,不由得毛骨悚然,此時在光天化日之下,前後左右都站滿了人,然而這一聲驚呼,却如突然有惡鬼出現一般。衆人一凜之下,不約而同的將頭轉向聲音來處,張無忌、靜慧、李明霞等人都迎了上去。無忌身法最快,生怕周芷若遇上了什麼厲害的敵人,或是猛獸毒物之類,幾個起落,已穿過樹林,只見一個青影急馳而來,正是周芷若。無忌急忙迎將上去,問道︰「芷若,怎麼啦?」周芷若滿臉恐怖之極的神情,叫道︰「鬼、鬼,有鬼追我!」見到無忌,縱身撲入他的懷中,身子兀自瑟瑟發抖。

無忌見她怕得失魂落魄,當下也顧不得嫌疑,輕怕她的肩膀,安慰道︰「别怕,别怕!不會有鬼的。你瞧見了什麼?」只見周芷若上衣被荊棘樹枝扯得稀爛,臉上手上都有許多血痕,左臂上半隻衣袖都被扯落了,露出一條雪藕般潔白的臂膀,上臂正中一點,如瑚珊,如紅玉,正是處女的守宮砂。無忌精通醫藥,知道處子臂上點了這守宮砂後,若非嫁人或是失身,終身不退。他先前聽了靜慧和李明霞的言語,原是將信將疑,此刻親眼得見,已無半分懷疑,霎時之間,心中轉了無數念頭︰「她前先跟我説,被囚於丐幫之時,曾失身於宋青書,腹中懷了孩子。當時我搭她脈博,絶無懷孕之象,其時還道診斷有誤,如此説來,她是有意騙我的了。至於嫁宋青書爲室云云,更是全無其事。」轉念又想︰「張無忌啊張無忌,周姑娘是害死你表妹的大仇人,她是處女也好,是人家的妻室也好,跟你又有什麼相干?」但見周芷若實在怕得厲害,不忍便推開她。

周芷若伏在他的懷中,感到他胸膛上壯實肌肉,聞到他身上男姓氣息,漸漸鎭定,説道︰「無忌哥哥,是你麼?」無忌道︰「是我!你見到了什麼?何以這等害怕?」張無忌一問,周芷若突然又驚惶起來,哇的一聲,熱泪迸流,靠在無忌的肩上,抽抽噎噎的哭個不住。這時楊逍、韋一笑、靜慧殷利亨等衆人均已趕到,突然看到這等情景,相互使個眼色,都悄悄的退了回去。在明教武當派峨嵋派群俠心中,均盼周芷若與張無忌言歸於好,結爲夫婦。各人一來不免懷念與趙明爲敵時的怨仇,二來總覺得趙明是蒙古貴女,無忌若是娶她爲妻,只怕有礙興復大業。

周芷若哭了一陣,忽道︰「無忌哥哥,有人追來麼?」無忌道︰「没有!是誰追你?是玄冥二老麼?」周芷若道︰「不是!不是!你瞧清楚了,眞的没人\dash{}不,不是人\dash{}没有什麼東西追來麼?」無忌微笑道︰「青天白日之下,有什麼看不清楚的。」他聲轉溫柔,説道︰「芷若,你連日使力過,實在是累了,想必頭暈眼花,看錯了什麼東西。」周芷若道︰「不會,決計不會的。我見了它三次,接連三次。」在話聲顫抖,兀有餘悸,無忌道︰「見到三次什麼?」

周芷若扶著無忌的身子,顫巍巍的站了起來,回頭望了一眼。這一眼似乎是使了極大的勇氣,一望之下,立即又轉眼向著無忌,見到他溫柔関懷的神色,心中一酸,全身乏力,軟倒在地下,説道︰「無忌哥哥,我\dash{}我都是騙你的,倚天劍和屠龍刀是我盜的,殷\dash{}殷姑娘是我殺\dash{}殺的,謝大俠是我下手點的穴道。我\dash{}我没嫁宋青書。我心中實在自始至終,只有一個你。」無忌嘆道︰「這些事情,我都知道,可是\dash{}可是你又何苦如此?」周芷若哭道︰「你却不知道我師父在萬法寺的黃塔之上,跟我説過些什麼。她將倚天劍與屠龍刀中的祕密,説與我知曉,要我立誓盜到寶刀寶劍,光大峨嵋。要我立下毒誓,假意與你相好,却不許我對你眞的動情\dash{}」

張無忌輕輕撫著周芷若的手臂,想起當年自己親眼見到滅絶師太一掌擊斃紀曉芙,見她在大漠中立誓殲滅明教,又見她手持執倚天劍亂殺鋭金旗下教衆,直至後來萬法寺高塔下見她寧可身死,不願受自己救援,可以想見她對明教怨毒之深,痛恨之切。周芷若既承她衣砵,受她遺命,種種陰狠毒辣的行逕,想必均是出於師父所囑。無忌本性原是極易諒解旁人的過失,向來不善記仇,又想到那日光明頂上惡鬥何太沖夫婦及華山派的高矮二老,若不是周芷若從旁指點,説不定自己當時便已死於非命。這時周芷若苦楚楚嬌弱,伏在他的懷中,不禁頗生憐惜之心,柔聲道︰「芷若,你到底見到了什麼,竟是這等害怕?」

周芷若霍地躍起,説道︰「我不説。是那冤魂纏上了我,我自己作惡多端,原是當有此報。我今日一切跟你説明了,我\dash{}我已命不久長\dash{}」説著掩面疾走,奔向少至山下。無忌茫無頭緒,心想︰「什麼怨魂纏上了我?難道是丐幫幫衆復仇,裝神弄鬼嚇{\upstsl{唬}}於她麼?」慢慢在後跟去。只見周芷若走入峨嵋派群弟子之中,李明霞取過一件外衣給她披上。周芷若低聲吩咐什麼,群弟子一聲凜遵。無忌正自呆呆出神,楊逍將一柄斷成兩截的倚天劍捧在手中,説道︰「教主,咱們將這劍還了峨嵋派,好不好?」無忌點了點頭,楊逍捧著自去迸還。

這時山下群雄又走了一大批,空聞、空智二人忙著送别。無忌道︰「咱們也好走了。」只見周芷若走到空聞跟前,低聲跟他説了幾句話,空聞臉色大變,怔了一怔,隨即搖頭,意似不信。周芷若再説了幾句話,忽地跪了下來,雙手合什,喃喃禱祝什麼。空聞神色莊嚴,口誦佛偈。周顚道︰「這可奇了。教主,此事你非得阻止不可。」無忌道︰「阻止什麼?」周顚道︰「周姑娘要出家做和尚她\dash{}她身入空門,你可糟了。」楊逍冷笑道︰「周姑娘就算出家,也只是做尼姑,不做和尚,那有拜少林僧爲師之理?」周顚用力在自己額頭上擊了一記,説道︰「對,對!我周顚一時胡塗了。那周姑娘求空聞大師幹什麼?一個少林派掌門,一個是峨嵋派掌門,分庭抗禮,不用跪下啊。」

只見周芷若站起身來,臉上略有寬慰之色。無忌嘆道︰「别人的閒事,咱們不用多管了。」回頭説道︰「明妹,咱們該得走了。」那知這一回頭,却不見趙明。這些日來,趙明伴在他的身旁,形影不離,無忌微微一驚,問道︰「趙姑娘呢?」心中暗叫︰「不妙,莫要芷若伏在我的懷中,給明妹看了去,只道我舊情不斷,竟爾捨我而去?」忙打發人四下尋覓,烈火旗掌旗使夏炎説道︰「啓稟教主,屬下見趙姑娘下山去了!」無忌好生難過︰「明妹不顧一切的隨我,經歷了多少患難,我豈可負她?」當下即向楊逍道︰「楊兄,此間事務,請你代我料理,我先走一步。」於是向空聞、空智告别,又别過兪蓮舟、張松溪等人,向周芷若道︰「芷若,好生保重,後會有期。」周芷若低目垂眉,並不回答,只是微微點了點頭,數滴珠泪,落在塵之中。

張無忌展開輕功,向山下疾馳。山道上一列數里,都是從少林寺歸去的各路英雄,無忌不願多所驚動,從各人身旁一晃即過,始終不見趙明的蹤跡,他一口氣追出三十餘里,天色將晩,道上人跡漸稀,忽想︰「明妹計謀甚深,她既有心避開,多半不從大路行走。她若走這條路,以我脚程之快,早就趕上了。莫非她躱在少室山中,待我走後,她再背道而馳。」一時心急如焚,顧不得腹中飢渴,在群山叢中又児了轉來,時時躍上樹顚高坡,四下眺望。空山寂寂,唯見歸鴉。

無忌直繞到少室山後,仍不見趙明,心想︰「不論如何,我對你此心不渝,縱是天涯海角,終究也要找到你。」這麼一想,心下便坦然了,見東北角山坳裡兩枝大棗樹並肩聳立,當下躍下樹去,找到一根橫伸的枝幹,展身臥倒,他累了一日,多經變故,這一躺下,不久便已沉沉睡去。

睡到午夜,睡夢中忽聽數十丈外有輕輕的脚步之聲,以他這等武功高強之人,立時便即驚覺。他身子不動,睜眼一看。其時一輪新月已斜至西天,淡淡月光之下,只見山坡上有一人飄行極快,正向南行。無忌見那人的背影纖細,一搦瘦腰,是個身材苗條的女子,大喜之下,一聲「明妹」險些児便叫出口來,但立即覺察不對,那女子身形比趙明略高,輕功身法更是大不相同。她脚步輕靈勝於趙明,飄忽處却又不及周芷若。無忌好奇心起︰「一個少女深宵獨行,不知爲了何事?」本來此事與他毫不相干,更不願去窺探人家姑娘的私事,但不禁想到︰「説不定能從這少女身上找到明妹。倘若她與明妹截然無関,我悄悄走開便了,原他無礙。還是别輕易放過任何線索爲是。」於是扶著樹幹,輕輕溜了下來。

他生怕被那少女發覺,離得她極遠,須知深宵跟隨一個不相識的少女,難免有輕薄之嫌。只見那少女一身黑衣,一路走向少林寺去。無忌認明了她行走的方向,心道︰「她是向少林寺而去,即使與明妹無関,所圖謀的也必是武林中事。我蒙天下英雄推爲盟主,倘若她意欲不利於少林,這閒事我也得插手管他一管。」停步傾聽,四下無旁人,知那少那女並無後援,更是放心。

行了約莫一個更次,那少女始終没有回頭一次。無忌瞧著她的背影,隱隱覺得有些眼熟,似乎從前曾經見過,心想︰「是武青嬰姑娘麼?是峨嵋派的丁敏君麼?」細看却均不是,在樹木山石間躱躱閃閃,顯是知道少林寺中高手甚衆,生怕被人發見了蹤跡。忽聽得清磬數聲,從少林寺的大殿中傳了出來,跟著梵唱聲起,竟是數百名僧人的聲音。無忌大奇︰「少林僧人居然半夜三更還在誦經禮佛,而且是這許多僧人,難道在做什麼重要法事麼?」

那少女聽得梵唱之聲,行得更加慢了,又前往數十丈,已到了大殿之旁。忽聽得脚步聲響,那少女在草叢中一伏,跟著四名少林僧手提戒刀禪杖,巡視過來。原來英雄大會雖已過去,寺中仍是嚴密戒備,提防仍有敵人來犯。那少女待四名僧人走過,這纔長身,縱身一躍,已到了殿外長窗之旁。這一縱躍飄如飛絮,已是武林中一流的輕功。無忌見她雙手無帶兵刃,孤身一人,也不是到少林寺來生事的模樣,有心要瞧瞧她的容貌,到底自己是否認識,於是彎腰從她身後繞過,斜行到大殿西北角上。這時他自知處境十分{\upstsl{尷}}尬,若是被那少女發覺,倒也罷了,倘被少林寺中僧人知覺,以他身份,竟然深夜中來寺窺探,對方縱是佯作不知,却也是大損顏面,是以加倍小心,一步一動,比貓鼠還更輕捷。

這時殿中誦經之聲又起,無忌湊眼到窗縫中一張,只見大殿上數百名僧人,整整齊齊,一行行的坐在蒲團之上,各人身披黃袍,外罩大紅金線袈裟,有的手執法器,有的合什低誦,正在做超渡亡魂的法事,無忌登即領悟︰「是了,這次英雄大會傷了不少人,元軍攻山,雙方陣亡更衆。寺中僧侶慈悲在懷,連夜爲死者超度,祝他們往生極樂。」空聞大師站在供桌之前,親自主祭,他右首站的却是一個少女,無忌一見,不禁吃了一驚。

原來站在空聞大師身旁的少女,正是周芷若。無忌望將過去,見到她的側面,只見她神色怔忡不定,秀眉深蹙,若有深憂。無忌心道︰「是了。日間芷若在空聞大師面前,原來是求他做法事超度,想必是她深深懺悔自己所作所爲,她爪下劍底,傷的無辜太多。」凝目向供桌上瞧去,只見中間一塊靈牌之上,冩的赫然是「女俠殷離之靈位」七字。無忌一陣神傷,想起這位表妹身世之慘,對己之一往情深,不由得怔怔的掉下泪來。

鐘磬木魚聲中,周芷若盈盈的下拜,口唇微動,低聲禱祝,無忌依稀聽到︰「殷姑娘\dash{}你在天\dash{}安息\dash{}别來擾我\dash{}」無忌手扶牆,心中思潮起伏︰「表妹命喪於她劍底,固然命苦,但芷若内心深受折磨,所受痛苦,未必比表妹更少。」腦海中突然之間,隱隱湧起了光明頂上聽到明教教衆所誦的幾句歌來︰「生亦何歡,死亦何苦?憐我世人,憂患實多!憐我世人,憂患實多!」

周芷若緩緩站了起來,微一側身,臉孔向著東首,突然間臉色大變,叫道︰「你\dash{}你\dash{}你又來了!」聲音尖鋭,壓住了滿殿的鐘磬之聲。無忌順著她的目光瞧去,只見長窗上糊的窗紙不知何時破了,破孔中露出一張少女的臉來,滿臉都是一條條的血痕。無忌嚇得身子發顫,忍不住一聲驚呼,原來那臉上雖是血痕斑斑,却清清楚楚便是已死的殷離!

無忌待要上前招呼,只是一雙脚一時不聽自己使喚,竟是僵住了不能移動。祇見那張臉突然隱去,大殿中砰的一聲,周芷若往後摔倒。無忌這時再也顧不得少林派生嫌,大聲叫道︰「蛛児,蛛児!是你麼?」這聲音震得山谷鳴響,却是無人回答。無忌微一定神,飛身往來路上追去,只見冷月斜懸,滿地樹影,那黑衣少女已是不知去向。他雖是素來不信鬼神,但此情此景,禁不住出了一身冷汗,心中暗暗發毛。無忌自言自語︰「是她,是她!怪不得背影好熟,原來是蛛児。難道她鬼魂知道少林高僧爲她超度,特來領經麼?難道她死得冤屈,眞的是陰魂不散?」

少林群僧聽得殿外有聲,早有數人搶將出來察看,見是張無忌,都是呆了。一名年長僧人上前行禮,説道︰「不知張教主夤夜降臨,未曾迎迓,伏乞恕罪。」無忌拱手道︰「不敢!」閃身便進殿中,只見周芷若雙目緊閉,臉上無半點血色,兀自未醒。無忌搶上前去,在她人中上用力捏了幾下,再在她背上推拿數過。周芷若悠悠醒轉,一見無忌,縱體入懷,摟住了他,叫道︰「有鬼,有鬼!」無忌道︰「此事好生奇怪,你别害怕。眼前這許多高僧在此,定能解此冤孼。」周芷若向來端莊穩重,這時實是怕得狠了,纔在衆目睽睽之下抱住了無忌,聽他這麼一説,臉上一紅,忙放開了他,站了起來,但身子兀自不住發抖,抓著無忌的手掌,死也不敢放脱。

無忌和空聞見過了禮,説起適纔在窗孔中所見的那張血臉。空聞與群僧却是都没有看見。周芷若道︰「無忌\dash{}張教主,我見到的,確然是她。」無忌沉吟半晌,道︰「我\dash{}我也見到的。」周芷若身子劇烈一顫,道︰「你\dash{}你也見到了?」無忌點了點頭。周芷若顫聲道︰「你\dash{}你見到的是誰?」無忌道︰「是殷姑娘,我的表妹蛛児。」周芷若低低一聲驚呼,又暈了過去。這一次無忌拉著她的手,是以她並没摔倒,略一昏暈,便即醒轉。無忌道︰「我見到了蛛児,可是\dash{}可是她是人,不是鬼!」周芷若道︰「她不是鬼?」無忌道︰「我一路跟著她到少林寺來。她行走如常,決非鬼神之輩。」無忌這幾句話,原是爲安慰周芷若而説,在他内心,可實是不敢相信自己的話。

周芷若問道︰「你當眞見她行走如常,確非鬼物?」無忌經她這麼一問,回想一路跟著那黑衣少女來到少林,又見她躱在長窗之外向殿中窺探,全是一個身懷武功的姑娘,絲毫無特異之態,於是問空聞大師道︰「方丈,在下有一事不明,要向方丈請教。人死之後,是否眞有鬼魂?」空聞沉思半晌,道︰「幽冥之事,實所難言。佛説無我相,無人相,無衆生相,無壽者相。萬物皆空,何況鬼魂?」無忌道︰「然則大師何以虔誠誡行法,超度幽靈?」空聞道︰「善哉!幽靈不須超度,佛家行法,乃在求生人心之所安,超度的乃是活人。」無忌登時領悟,拱手道︰「多謝指點。在下深夜攪擾,至爲不安,萬望方丈恕罪。」空聞微笑道︰「教主乃敝派的大恩人,數度拯救,使少林得免於難,何必客氣。」

當下無忌與群僧作别,向周芷若道︰「咱們走吧!」周芷若臉有遲疑之色,不敢離開佛殿。無忌也不便強勸,拱手道︰「既是如此,咱們就此别過。」説著走出殿門。周芷若望著他的背影!知道從此一别,只怕再也不能相見,突然叫道︰「無忌哥哥!我\dash{}我和你一起去。」縱身奔到他的身旁,和他並肩出了寺門。

二人一離少林寺數十丈,周芷若便靠到無忌身邉,拉住了他的手。無忌知她害怕,但握著她軟滑柔膩的手掌,身畔幽香陣陣,不能無感。二人默不作聲的走了一陣,周芷若幽幽嘆了一口長氣,説道︰「無忌哥哥,那日我和你初次在漢水之中相逢,得蒙張眞人搭救,早知日後要受這麼多苦楚,當時死在漢水之中,倒是乾淨得多。」無忌不答,心中又想起了明教所唱的那首歌,忍不住輕輕哼道︰「生亦何歡?死亦何苦?憐我世人,憂患實多。」周芷若聽著歌詞,握著無忌的手微微顫動,低聲道︰「張眞人雖是爲我好,但若他老人家收留我在武當山上,讓我歸入武當門下,今日一切又是大不相同。唉,恩師對我何嘗不好?可是\dash{}可是她逼我罰這些毒誓,要我痛恨明教,要我深恨你害你,可是我心中\dash{}實在\dash{}」張無忌聽她説得眞誠,心下頗自感動,知道她確是有許多離難處,種種狠毒之事,大都是承奉滅絶師太的遺命,對她憐惜之情又深了一層。

山道上晩風習習,送來陣陣花香,其時正當初夏,良夜露清,耳邉聽著一個極美少女吐露深情,張無忌不能不怦然心動,何況當時在小島替她逼毒時曾有肌膚之親,既是總角之交,又有婚姻之知,不由心中迷迷惘惘。周芷若道︰「無忌哥哥,那日在濠州你正要和我拜堂,爲什麼趙姑娘一叫你,你便隨他而去?你心中眞的十分喜歡她麼?」無忌道︰「我正要將件事跟你説知。」眼見再行數里,便是明教衆人住宿之處,於是擕著她手,走到道旁的一塊山石邉,兩人並肩坐下,將趙明手中握著謝遜一束金髮,引得他非走不可的種種情事,一一説了。周芷若聽畢,半晌不語。無忌道︰「芷若,你怪我麼?」周芷若哽咽道︰「我做了這許多錯事,祇怪我自己,還能怪你麼?」無忌輕撫她的肩頭,柔聲道︰「世間事陰錯陽差,原難逆料,你也不用太過傷心。」

周芷若仰起頭來,説道︰「無忌哥哥,我有一句話問你,你須得眞心答我,不可隱瞞。」無忌道︰「好,我不會瞞你。」周芷若道︰「我知道這世上曾有四個女子眞心愛你。一個是去了波斯的小昭,一個是趙姑娘,另一個是\dash{}她\dash{}」她心中要説「殷姑娘」,但終不敢説出口來,頓一頓,道︰「除了小昭,我們三個都做過對不起你的事。倘若我們四個姑娘,這會児都是好好的活在世上。在你心中,你眞眞喜歡的是那一個?」無忌心中一陣迷亂,道︰「這個\dash{}{\upstsl{嗯}}\dash{}這個\dash{}」

當無忌與周芷若、趙明、殷離、小昭四人同時乘船出海之時,確是不止一次的想起︰「這四位姑娘個個對我情深愛重我,如何自處纔好?不論我和那一個成親,定會大傷其餘三人之心。到底在我内心深處,我最愛的是那一個呢?他始終彷徨難決,自己便只得逃避,一時想︰『韃子尚未逐出,河山未得光復。匈奴未滅,何以爲家?儘想這些児女私情幹什麼?』一時又想︰『我身爲明教教主,一言一動,與本教及武林興衰均有関連。我自信一生品行無虧,但若耽於女色,莫要惹得天下英雄恥笑,壞了本教的名聲。』過一時又想︰『我媽媽臨終之時,一再囑咐於我,美麗的女子最會騙人,要我這一生千萬小心提防。媽媽的遺言豈可不謹放心頭?』」

其實他多方辯解,不過是自欺而已,當眞心有所決,專心致志的愛了那一位姑娘,未必便與光復大業有妨,更未必會壞了明教的名聲,只是他覺得這個很好,那個也好,於是索性不敢去多想。有時他内心深處,不免也想︰「要是我能和這四位姑娘終身一起厮守,大家和和睦睦,豈不逍遙快活?」要知其時乃是元末,不論文士商賈,江湖豪客,三妻四妾實是尋常之極,一夫一妻的反倒罕有。只是明教源自波斯,向來諸教衆節儉刻苦,除妻子外少有妾侍,無忌生性謙和,也深覺不論和那一位姑娘匹配,在自己都是莫大的福澤,倘是再娶妾侍,未免太也對不起這般天仙一樣的人児,因此這種念頭在心中一閃即逝,從來不敢多想偶爾念及,往往便即自責︰「張無忌啊張無忌,爲人須當自足,你心中竟存這種念頭,那不是太過卑鄙可恥麼?」

後來小昭去了波斯,殷離逝世,自己又以爲殷離乃是趙明所害,那麼順理成章,自是要與周芷若成婚。不料變生不測,中間大起波折,然而眞相逐步揭露,周趙二女原來善惡顚倒,幸好自己並未與周芷若鑄成大錯,趙明更公然與父兄決裂,則此事已不爲難。萬不料趙明突然不告而别,而周芷若突然又有此一問。

周芷若見他沉吟不答,説道︰「我問你的乃是虛幻之事。小昭當了波斯明教的處女教主,我又\dash{}又殺害了殷姑娘,四個女子之中,只剩下了趙姑娘。我只問你,倘若咱們四人都是好端端的在你身邉,你便如何?」無忌道︰「芷若,這件事我在心中已想了很久。我似乎一直難決,但到今天,我才知道眞正愛的是誰。」周芷若問道︰「是誰?是\dash{}是趙姑娘麼?」無忌道︰「不錯。我今日尋她不見,恨不得自己死了纔好。要是從此不能見她,我的命也是活不久長。小昭離我而去,我自是傷心。殷表妹逝世,我也是難過。你\dash{}你後來這樣,我既是痛心又是惋惜。然而,芷若,我不能瞞你,如果我這一生再不能見到明妹,我是寧可死了的好。這種心意,我以前對旁人從未有過。」

無忌初時對殷離、周芷若、小昭、趙明四女,似是不分軒輊,但今日趙明這一走,他纔突然發覺,原來趙明在他心中所佔位置,竟是與其餘三女不同。周芷若聽他這般説,輕輕道︰「那日在大都,我見你到那小酒店去會她,便知你内心情愛之所繫。只是我還痴心妄想,若是與你\dash{}與你成親之後,那便\dash{}那便可以拉得你回心轉意,實在\dash{}實在那是萬萬不能的。」無忌歉然道︰「芷若,我對你一向敬重,對殷表妹是心生感激,對小昭是意存憐惜,但對趙姑娘却是\dash{}却是銘心刻骨的相愛。」周芷若喃喃道︰「銘心刻骨相愛,銘心刻骨相愛。」頓了一頓,低聲道︰「無忌哥哥,\dash{}我對你可也是銘心刻骨的相愛。你\dash{}你竟是不知道麼?」無忌大是感動,握著她的手,歉然道︰「芷若,我是知道的。你對我這番心意,今生今世,我不知要如何報答纔好。我\dash{}我是對你不起。」周芷若道︰「你没有對我不起,你一直待我很好,難道我不知道麼?我問你︰倘若趙姑娘此番不别而行,你永遠再找她不到了,倘若她被奸人害死了,倘若她對你變了心,你\dash{}你便如何?」無忌心中已難過了良久,聽她這麼説,再也忍耐不住,哽咽道︰「我\dash{}我不知道!總而言之,上天下地,我也非尋著她不可。」

周芷若嘆了口氣,道︰「她不會對你變心的,你要尋著她,那也很容易。」無忌又驚又喜,站了起來,道︰「她在那裡?芷若,你快説。」周芷若一對妙目凝視著,見到他臉上大喜若狂的神情,輕輕道︰「你對於我永遠不會這麼関心。你要知道趙姑娘的所在,須得答應我一件事,否則你永遠找她不到的了。」無忌嘆息道︰「你要我答應什麼事?」周芷若道︰「這件事我現下還没想起,日後想到了再跟你説。總之這事不違俠義之道,於光復大業有利,也於明教及你自己的名聲無礙,只是做起來未必容易。」無忌一呆,心想︰「當日明妹要我做三件事,也説什麼不違俠義之道,迄今爲止,她只要我做過兩件事。那兩件事可眞不易辦,怎麼芷若也學起她的樣來?」周芷若道︰「你不答應,那也由得你。不過大丈夫言而有信,要是答應了我,事到臨頭,可不能推諉抵賴。」

無忌沉吟道︰「你説此事不違俠義之道,既於光復大業有利,也於明教及我自己的名聲無礙?」周芷若道︰「不錯!」無忌道︰「好,當眞不違俠義之道,無損於光復大業,我便答應你了。」周芷若道︰「咱們擊掌爲誓。」伸出手掌,要與無忌互擊。無忌心下隱隱覺得,跟她擊掌立誓之後,等於在自己身上套了一層無形的枷鎖,這位周姑娘外表溫柔斯文,但心計之工,行事之辣,其實是遠勝趙明,一時提起手掌,拍不下去。周芷若微笑道︰「你祇須答應我這件事,我教你傾刻之間,便見到你的心上人。」無忌胸口一熱,再也不計其他,便和她擊掌三下。周芷若笑道︰「你瞧這裡是誰?」伸手撥開了身後的樹叢。只見在一叢花葉之後,坐著一個少女,臉上似笑非笑,却不是趙明是誰。無忌大叫一聲︰「明妹!」忽聽得身後數丈之外,一個女子聲音「咦」的一聲,似乎突然見到趙明現身,忍不住驚呼了出來。這一聲驚呼聲音甚輕,但無忌已是聽得清清楚楚。他一呆之下,心中轉了無數念頭,緩緩伸出手掌去拉趙明的手,雙掌相接,只覺她手掌頗爲僵直,登時省悟,原來她日間不别而行,自己到處找她不到,却被周芷若擒住了,點了她的穴道,藏在這裡。他順手一搭趙明的脈博,察覺氣血運行如常,並未受傷。月光之下,只見她眉間眼角,笑意盈盈,説不盡的嬌媚可愛,想是無忌適纔與周芷若這番對話,都教她一一聽在耳中。她雖是身不能動,口不能言,但聽到無忌背後吐露心曲,對她竟是如此銘心刻骨,情意懇切,自是禁不住心花怒放。

周芷若彎下腰來,在無忌耳邉低聲説了幾句話,無忌低聲回答一句。周芷若不禁大怒,喝道︰「張無忌,你竟是全然没將我放在眼裡,你仔細瞧瞧,這趙明中毒後,還活得成麼。」無忌大吃一驚,道︰「她\dash{}她中了毒!是你下的毒麼?」俯身察看,剛翻開趙明左眼的眼臉,只覺背心一麻,已被周芷若點了穴道。無忌「啊喲」一聲,身子搖晃。他内功深厚,雖被點中要穴,却不摔倒,忙運氣衝穴時,周芷若出手如風,纖指連動,又點了無忌左肩、腰脅、後心一共五處大穴。

\chapter{萬縷柔絲}

無忌武功再強,接連受了這六下襲擊,那也是抵受不住,仰天便倒。只見青光一閃,周芷若拔出長劍,抵住了他的口,喝道︰「一不做、二不休,今日便取了你的性命。反正殷離的冤魂纏上了我。我終究是活不成了,咱們一起同歸於盡。」説著提起了長劍,便往無忌胸口刺了下去。

忽聽得身後一個女子的聲音叫道︰「且慢!周芷若,殷離並没有死!」周芷若回過頭來,但見一個黑衣女子從草叢中疾奔而出,一指往周芷若後心戳了過去。周芷若斜身閃開,那女子回過頭來,月光側照,只見她臉容俏麗,淡淡的佈著幾條血痕。張無忌看得明白,這女子正是他表妹殷離,只是臉上浮腫盡褪,雖有縱橫血痕,却不掩其美,依稀便是當年蝴蝶谷中,金花婆婆身畔那個清秀絶俗的小姑娘。周芷若退後兩步,左掌護胸,右手中長劍的劍尖指住了無忌胸口,喝道︰「你再上前一步,我一劍先刺死了他。」殷離果然不敢再動,急道︰「你\dash{}你做的惡事還不彀多麼?」

周芷若道︰「你到底是人是鬼?」殷離道︰「我自然是人。」張無忌突然大叫一聲︰「蛛児!」一躍而起,抱住了殷離,叫道︰「蛛児,\dash{}你\dash{}你想得我好苦!」這一下出其不意,殷離嚇得尖叫一聲,被張無忌圍住了雙臂,動彈不得,周芷若嘻嘻一笑,説道︰「若非如此,你還是不肯出來。」回身去解開了趙明的穴道,替她推宮過血,按摩筋脈。趙明被她制住了半日,冷清清的抛在這裡,心下好不惱怒,幸好後來聽到張無忌吐露了心事,這纔轉怒爲喜。只是突然之間又多了一個殷離出來,却更平添了無數心事,正是舊恨甫除,新愁轉生。

殷離嗔道︰「你拉拉扯扯的幹什麼?趙明姑娘、周姑娘都在這児,成什麼樣子了?」無忌微笑道︰「我見你死後還魂,歡喜無盡,表妹,你到底\dash{}到底是怎樣的?」殷離拉著他手臂,將他臉孔轉到月光下,凝視半晌,突然抓住他的左耳,用力一扭。無忌痛叫︰「啊喲!你幹什麼?」殷離道︰「你這千刀萬剮的醜八怪,冒充什麼曾阿牛來騙我,叫我向你吐露心事,要我在人前出醜\dash{}你\dash{}你將我活埋在土中教我吃了多少苦頭。」説著在胸口連搥三拳,砰砰有聲。無忌不敢運九陽神功相抗,忍痛受了她三拳,笑道︰「表妹,我實在是以爲你已經\dash{}已經死了,累我傷心得痛哭了幾場。你怎麼又活了轉來,當眞是老天爺有眼。」殷離道︰「老天爺有眼,你這醜八怪便是没眼。你是蝶谷醫仙的弟子,連人家是死是活也不知道。我不相信。你是嫌我的臉腫得難看,没等我斷氣,便將我埋在土中,你這没良心的,狠心短命的死鬼!」她一連串的咒罵,神情語態,一如往昔。

無忌嘻嘻的聽著,搔頭道︰「你罵得是罵得是。當時我眞是胡塗,見到你滿臉鮮血,没了呼吸,心又不跳了,只道已是無救\dash{}」殷離跳將起來,伸手又去扭他右耳。無忌嘻嘻一笑,閃身避開,作揖道︰「好表妹,你饒了我吧!」殷離道︰「我纔不饒你呢,那日我不知怎樣醒了過來,上下四周冷冰冰的,都是石塊。你既要活埋我,幹麼又替我作個石坑?爲什麼不在我身上堆滿泥土,我透不過氣來,不就眞的死了?」無忌道︰「謝天謝地,幸好我替你築一個石坑。」忍不住向周芷若斜睨一眼。殷離怒道︰「這人壞透啦,我不許你看她。」無忌道︰「爲什麼?」殷離道︰「她是殺死我的兇手,你還理她作什麼?」趙明突然插口道︰「你既没死,她便不是兇手。」殷離道︰「我已死過了一次,她就是兇手!」

無忌勸道︰「好表妹,你脱險歸來,我們都是歡喜得緊。你安安靜靜的坐下來,跟我們説説這番死裡逃生的經過。」殷離道︰「什麼我們不我們的。我來問你,你説『我們』這兩個字,到底那幾個人才是『我們』?」無忌笑道︰「這裡只有四個人,那自是我和周姑娘、趙姑娘了。」殷離冷笑道︰「哼!你或許還有幾分眞心歡喜,可是周姑娘和趙姑娘呢?她們也都歡喜麼?」周芷若道︰「殷姑娘,那日我起下歹心,傷害於你,事後不但白天深自痛悔,連夢魂之中也是不安,否則今日突然在樹林中見到你,也不會嚇成這個樣子了。此刻見你平安無恙,免了我的罪孼,老天在上,我確是歡喜無量。」殷離側著頭想了片刻,點頭道︰「那也有幾分道理。我本想找你算帳,既是如此,那就罷了。」周芷若雙膝跪倒,嗚咽道︰「我\dash{}我當眞太對你不起。」

殷離向來性子執拗,但一見周芷若服輸,心下登時軟了,忙扶起了她,説道︰「周姊姊,過去的事,誰也别放在心上,反正我也没死。」兩人手拉著手,並肩坐下。殷離掠了掠頭髮,説道︰「你在我臉上劃了這幾劍,也不是全無好處。我本來臉上浮腫,中劍的毒血流盡,浮腫倒漸漸消了。」周芷若心下歉仄無已,不知説什麼好。無忌道︰「我和義父、芷若後來在島山住了很久。表妹,你從墓中出來後,怎會不見到我們?」殷離怒道︰「我是不高興見你。你和周姑娘這般卿卿我我,聽得我好不生氣。哼!『我此後只有倍加疼你愛你!我二人夫婦一體,我怎會給你氣受?』」他學著張無忌的口氣説了這幾句話後,又學著周芷若的口氣道︰「『要是我做錯什麼,你會打我,罵我、殺我麼?我從小没爹娘教導,難保不會一時胡塗。』」她咳嗽一聲,又學著男子的嗓子,説道︰「『芷若,你是我的愛妻。就算你做錯了什麼,我是重話也不捨得責備你一句。』」手指西天明月,説道︰「『天上的明月,是咱倆證人。』」

原來當晩張無忌與周芷若定情時所説的言語,都讓殷離聽在耳中。這時她一一述覆出來,只聽得周芷若滿臉通紅,張無忌忸怩不安。他向趙明偸瞧一眼,她一張俏臉氣得慘白,於是伸手過去,握住了她的手腕。趙明手腕一翻,兩根長長的指甲,刺入無忌的手背。無忌吃痛,既不敢叫出聲來,也不敢動。

殷離伸手入懷,取出一根木條來,放在無忌眼前,説道︰「你瞧清楚了,這是什麼?」無忌一看,見木條上刻著一行字道︰「愛妻蛛児殷離之墓。張無忌謹立。」正是無忌當日在殷離墓前所立。殷離恨恨的道︰「我從墓中爬了出來,見到這根木條,當時便胡塗了,怎麼?是那個狠心短命的小鬼無忌?我百思不得其解,直到後來偸聽到你二人的説話,『無忌哥哥』長,『無忌哥哥短』的,這纔恍然大悟。原來張無忌便是曾阿牛,曾阿牛便是張無忌。你這没良心的,騙得我好苦?」説著舉起木條,用力往張無忌頭上擊了下去,拍的一聲響,木條斷成數截,飛落四處。

趙明怒道︰「怎麼動不動便打人?」殷離哈哈一笑,説道︰「我打了他,怎麼樣?你心疼了是不是?」趙明臉上一紅,道︰「他是在讓你,你算不知好歹。」殷離笑道︰「我有什麼不知好歹?你放心,我纔不會跟你爭這個醜八怪,我一心一意只喜歡一個人,那是蝴蝶谷咬傷我手背的小張無忌。眼前這個醜八怪啊,他叫曾阿牛,叫張無忌也好,我一點也不喜歡。」她轉過頭來,柔聲説道︰「阿牛哥哥,你一直待我很好,我好生感激。可是我的心,早就許了給那個狠心的兇惡的、小張無忌了。你不是他,不,不是他\dash{}」

無忌好生奇怪,道︰「我明明是張無忌,怎地\dash{}怎地\dash{}」殷離溫柔地瞧著他,呆呆的看了半晌,目光神情變幻,終於搖搖頭,説道︰「阿牛哥哥,你不懂的,在西域大漠之中,你與我同生共死,在那海外小島之上,你對我仁至義盡。你是個好人,不過我對你説過,我的心早就給了那個張無忌啦。我要尋他去。」説著又瞧了他半晌,轉身緩緩走了開去。張無忌陡地領會,原來她眞正所愛的,乃是她心中所想像的張無忌,是她記憶中在蝴蝶谷所遇上的張無忌,却不是眼前這個眞正的張無忌,不是這個長大了的、待人仁恕寬厚的張無忌。

無忌心中三分傷感、三分留戀、又有三分寬慰,望著她的背影消失在黑暗之中。他知道殷離這一生,永遠會記著蝴蝶谷中那個狠勁的少年,她是要去找尋他。她自然找不到,但也可以説,她早已尋到了,因爲那個少年早就藏在她的心底。眞正的人、眞正的事,往往不及心中所想的那麼好。周芷若嘆了口氣,道︰「都是我不,害得她這麼瘋瘋癲癲地。」無忌心中却想︰「她確是有一點児瘋瘋癲癲,這是我害的。可是比之腦筋清楚的人,她未必不是更加快活些。」

趙明心中所思量的,却是另一回事。殷離來了又去了,使她心頭放下了一塊石頭,然而周芷若呢?殷離既没死,謝遜也是好端端的平安無恙,倚天劍中所藏的武功、屠龍刀中所藏的兵書,都已交給了無忌,周芷若所犯的過錯,這時看來都没什麼大不了的了。當然,宋青書爲了她而害死了莫聲谷,然而這是宋青書自己的罪孼,周芷若在事先確是全不知情,也絶無唆使之意。無忌曾與她有婚姻之約,他可不是棄信絶義之人。

周芷若站起身來,説道︰「咱們走吧!」趙明道︰「到那裡去?」周芷若道︰「我適纔在少林寺時,見彭瑩玉和尚匆匆前來尋他,似乎明教中出了什麼要緊事。」無忌一凜,心道︰「我莫要爲了児女之情,誤了教中大事。」忙道︰「既是如此,咱們快去瞧瞧。」當下三人加快脚步,不多時便到了明教教衆營之所。

楊逍、范遙、彭瑩玉等正命人到處找尋教主,見他回來,倶各欣慰,但見周趙二女和他同歸,心中又均詫異,無忌見衆人神色沮喪,隱隱知道不妙,問道︰「彭大師,你有事尋我麼?」彭瑩玉尚未回答,周芷若挽了趙明的手,説道︰「咱們到那邉坐坐,我可走得去了。」趙明會意,知她避嫌,不願與聞明教的祕密,於是並肩齊出。楊逍范遙等更是奇怪,均想︰「那日濠州教主成婚之日,這兩位姑娘鬥得何等厲害,此刻却是似親姊妹。不知教主是如何調處的。」

彭瑩玉待周趙二女走出,説道︰「啓稟教主,咱們在濠州打了一個大敗仗,韓山童韓兄殉難。」無忌叫了一聲「啊喲!」極是痛惜。彭瑩玉又道︰「眼下淮泗軍務,由朱元璋兄弟指揮。徐達、常遇春兩位兄弟得知訊息,已領兵馳去應援,韓林児兄弟也同去了。事在緊急,不及等候教主將令。」無忌道︰「該當如此。」正商議軍情間,殷野王匆匆進來,説道︰「啓稟教主,丐幫中有人前來報知,陳友諒那厮的下落已然査明。」無忌道︰「在那裡?」殷野王道︰「這厮竟是混到了本教徐壽輝兄弟的屬下,聽説徐兄弟對他很是寵信。」張無忌沉吟道︰「既是如此,咱們倒不便躁急行事。舅舅,煩你派人通知徐兄弟,陳友諒這厮陰狠狡猾,留在身畔大是禍胎,最好是疏遠之爲妙。」殷野王答應了,又道︰「不如一刀殺了,乾乾淨淨。就讓我去辦吧!」

無忌正沉吟間,忽有教衆送來徐壽輝的一封緊急文書,無忌接在手中,楊逍皺眉道︰「糟糕,糟糕,竟被他佔了先著。」無忌拆開文書一看,原來是徐壽輝的一封長稟,説道陳友諒曾得罪教主,自知罪重,悔悟殊深,現下誠心投入本教,決意痛改前非,但求教主給予自新之路。無忌對這長稟給楊逍等看了。殷野王道︰「徐兄弟受此人蠱惑,必有後患。」楊逍嘆道︰「陳友諒這厮極是陰險,但咱們這時若是將他殺了,不免示人以不廣,寒了天下英雄之心。」無忌道︰「楊左使之言不錯。彭大師,你與徐兄弟交好,請你便中乘機勸導,小心提防於他,切不可讓兵馬大權落入他的手中。」彭瑩玉答應了。不料徐壽輝並未受勸,對陳友諒極是信任,終於命喪陳友諒之手。陳友諒統率明教西路義軍,自稱漢王,與明教東路義軍爭奪天下,直至鄱陽湖大戰,方始兵敗身死,令明教英雄豪傑遭受重大傷亡,此是後話不提。

當晩張無忌與楊逍、彭瑩玉等詳細計議,分派人衆,前赴各路義軍策應。他自以張三丰分别日久,甚是想念,次晨即與群雄分手,帶同趙明、宋青書上武當山去。周芷若道︰「張眞人於我曾有大恩。宋少俠逆父{\upstsl{弒}}叔,未始非因我而起,我當去領受罪責。」當下帶同峨嵋衆弟子,齊上武當。

少室山與武當山相距不遠,南下一過豫鄂邉界,不數日便到山上。張無忌隨同兪蓮舟、張松溪、殷利亨三人,入内拜見張三丰,又見了宋遠橋及兪岱岩。宋遠橋聽説児子在外,鐵青著臉,手執長劍,搶將出來。張無忌等均覺勸也不是,不勸也不是,一齊跟著到了大殿。宋遠橋喝道︰「迕逆不孝的畜生在那裡?」一瞥眼間,見宋青書躺在軟床之中,頭上綁滿了白布,連眼睛也遮没了,一挺長劍,劍尖指向他的身上,但手一軟,竟是刺不下去。霎時之間,想起父子之情,同門之義,不由得百感交集,回過劍來,一劍往自己小腹上刺去。

張無忌一伸手,便以乾坤大挪移功夫,奪下了他手中長劍,勸道︰「大師伯,萬萬不可。此事如何處理,該當請太師父示下。」張三丰嘆道︰「我武當門下出此不肖子弟,遠橋,那也不是你一人的不幸。這等逆子有不如無!」右手一掌推出,拍的一聲響,擊在宋青書的胸口。他這一掌何等威力,宋青書臟腑震裂,立時氣絶。宋遠橋跪下哭道︰「師父,弟子疏於管教,累得七弟命喪畜生之手。弟子如何對得起你老人家和七弟?」張三丰伸手扶了他起來,説道︰「此事你確有罪愆,本派掌門弟子之位,今日起由蓮舟接任。你專心精研太極拳法,掌門的俗務,不必再管了。」宋遠橋拜謝奉命。兪蓮舟忙推辭不就,但張三丰堅不許辭,只得拜領。

衆人見張三丰斃宋青書、革宋遠橋,門規嚴峻,心下無不凜然。張三丰問起英雄大會及義軍抗元之事,對張無忌溫勉有加。周芷若站在一旁,張三丰始終正眼也不瞧她一眼。待得觀中道人收拾了宋青書的屍身下去,張三丰忽從宋遠橋身邉抽出長劍,指著周芷若道︰「周姑娘,你是峨嵋派掌門,學得了滅絶師太幾成劍法?」周芷若道︰「晩輩所學,最多只有恩師劍法的三成。」張三丰道︰「當年郭女俠手創峨嵋一派,只盼群弟子卓然成家,在江湖上獨樹一幟。你以滅絶師太的三成功夫,憑什麼來光大峨嵋?你學得一些陰毒狠辣的武功,在英雄大會中爭勝逞能,以後峨嵋弟子,便學你這些陰毒武功麼?郭女俠於我有恩,老道雖是風燭殘年,却也不能眼見峨嵋派沉淪衰亡,毀於一旦!」周芷若道︰「張眞人這番話問得是,晩輩早已安排。」張三丰道︰「如何安排?」

周芷若不答張三丰的話,却轉過頭來,向張無忌道︰「張教主,當年在光明頂上會鬥六大派之時,我曾聽你言道,你並非武當派門下受業弟子,是也不是?」無忌不知她何以忽然問起此事,便道︰「先父是武當門下,太師父曾授過我太極拳法,若説我是武當弟子,也可説得。」周芷若道︰「我曾聽你言道,你初學武功的受業恩師,乃是你義父謝大俠,他是混元霹靂掌成崑的門人。你的九陽神功學自達摩老祖的遺書,乾坤大挪移心法學自明教前代教主的遺篇。咱們武林中人,最講究的是師門派别,你到底是那一門派的門人?」無忌道︰「我武功所學甚雜,認眞起來,並不是那一派的門人。」周芷若問張三丰道︰「張眞人,他這番話没錯吧?」張三丰點頭道︰「實情確是如此。武林中,這種情形甚爲稀有,那是他迭遇奇逢所致。」

周芷若刷的一聲,從腰間抽出半截倚天劍,左手握住自己頭上一把青絲,迴劍一掠,萬縷柔絲竟是一劍割斷。衆人都吃了一驚,齊道︰「你\dash{}你\dash{}」周芷若道︰「我罪孼深重,早有落髮出家之意,張教主,我問你,你曾答應過我,我有一事求你,你務須做到,是也不是?」張無忌點頭道︰「不錯,不過\dash{}」周芷若搶著道︰「不過此事須得不違俠義之道,既於光復大業有利,也不得有損明教的聲名,是也不是?」無忌道︰「是。若是如此,但有所命,自當遵從。」周芷若道︰「大丈夫千金一諾,當著你太師父與衆位師叔伯之前,可不能言而無信。」無忌見她割斷了頭髮,神色堅毅,心下不勝傷感,尋思︰「她眞有什麼爲難之事,我自當盡力替她辦到。」便道︰「你\dash{}你吩咐下來便是了。」

周芷若道︰「張眞人,須借寶殿一用。」解開背上包袱,取出兩塊靈牌來,一塊冩著「峨嵋派創派祖師郭女俠襄之靈位」,另一塊冩著「峨嵋派第三代掌門恩師滅絶師太之靈位」,恭恭敬敬的供在殿中方桌之上。張三丰與宋遠橋張無忌等一見,一齊躬身下拜。周芷若與本門弟子也拜過了,除下手上的鐵指環,轉身説道︰「張無忌張教主,峨嵋第四代掌門人周芷若,謹將掌門之位,傳授於你。」衆人一聽,都是驚得呆了,只聽她繼續説道︰「你仍兼任明教教主,盼你光大本門,興旺明教,率領中原豪傑,驅逐韃子,自今而後,峨嵋派門下弟子,盡皆聽你號令。」

無忌雙手齊搖,道︰「這\dash{}這\dash{}這如何可以?」周芷若道︰「峨嵋派乃郭女俠手創,請你出任掌門,那也不辱没了你。」無忌眼望張三丰,眼光中露出乞援之色。張三丰一怔之下,突然哈哈大笑,聲震屋瓦,説道︰「周姑娘,眞有你的。單憑你這一手,便不枉了滅絶師太的託付之重。峨嵋派交在無忌手中,發揚光大,那是的了。」周芷若從懷中取出一本黃紙薄本,連著兩截倚天劍的斷劍,交給無忌,説道︰「這是郭女俠手書的本門武學,劍掌精義,盡在其中。」

此事雖是大出意料之外,但無忌並不屬於任何門派,接掌峨嵋,並非違了江湖規矩,而此事確與光復大業有利,也不損明教聲威,祇聽張三丰又道︰「無忌孩児,你不是答應過周姑娘,説過的話可不能不算數。」無忌無奈,祇得將峨嵋派武學祕本和兩截斷劍接了過來,戴上指環,重新向兩座靈位跪倒。周芷若率同衆門人,一一參見第五代掌門人。張三丰、宋遠橋等依次道賀。峨嵋群弟子均知張無忌武功卓絶,威望極隆,於本門將有莫大好處,雖有數人心懷不服,却也不敢公然反對。

張三丰瞧著郭襄的遺書,眼前似乎又看到了那個明慧瀟灑的少女,可是,那是一百年前的事了。

周芷若削髮爲尼,不問世事,自此一盞青燈,長伴古佛。

張無忌率領峨嵋弟子偕同趙明,拜别張三丰、宋遠橋等,回歸峨嵋山,他到得山上,冩了一封長信,將明教教主之位讓與楊逍。

趙明見無忌冩完給楊逍的書信,手中毛筆尚未放下,神色間頗是不樂,便道︰「無忌哥哥,你曾答應我做三件事,第一件是替我借屠龍刀,第二件不許與周姑娘成婚,這兩件事你都做了。還有第三件事,你可不能言而無信。」無忌微微一驚,道︰「你\dash{}你\dash{}你又有什麼古靈精怪的事要我做?」趙明嫣然一笑,道︰「我的眉毛太淡,你替我畫一畫。這可不違反武林中俠義賢達吧?」無忌提起筆來,笑道︰「從今而後,我天天給你畫眉。」

(全書完)


\end{multicols}


\cleardoublepage
\pagestyle{empty}
\null\clearpage
\mcfamily

\vspace*{20mm}

	\begin{center}
\begin{minipage}<y>[htpb]{120mm}
\vspace{180mm} %奥付のページ上部からの位置
		\begin{tabular}{l}
			\multicolumn{1}{c}{\LARGE{\mcfamily\bfseries\makebox[6zw][s]{倚 天 屠 龍 記}}}\\[0mm] %%タイトル
				\hline
				%\\[-3mm]
			\hspace{2mm}\normalsize\makebox[5zw][s]{著 者}\hspace{5mm}%
			金 庸\\[0mm]  %%著者
			\hspace{2mm}\normalsize\makebox[5zw][s]{總 漢 字 數}\hspace{5mm}%
			約 900,526 字\\[0mm]  %%字數
			\hspace{2mm}\normalsize\makebox[5zw][s]{発 行 日}\hspace{5mm}\today\\[0mm] %%発行日。「(西元\number\year~年\number\month~月\number\day~日)」のところに任意の日付を入れてもいい。
			\hspace{2mm}\normalsize\makebox[5zw][s]{発 行 者}\hspace{5mm}%
			{\normalsize 子康(SteveCheung)}\\[0mm]  %%発行者
			\hspace{2mm}\normalsize\makebox[5zw][s]{聯絡方式}\hspace{5.2mm}%
			{\normalsize dongfang0571@gmail.com}\hspace{25mm}%
									\normalsize{\gt{\CID{734}商用禁止 }}
			 %\\[-3mm]  %%発行者
				\\\hline
		\end{tabular}
\end{minipage}
	\end{center}

\endinput




\end{document}

